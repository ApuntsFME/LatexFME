\usepackage[catalan]{babel}

% Normal
\declaretheorem[style=normal,name=Lema,numberwithin=section]{lema}
\declaretheorem[style=normal,name=Lema,numbered=no]{lema*}
\declaretheorem[style=normal,name=Observació,sibling=lema]{obs}
\declaretheorem[style=normal,name=Observació,numbered=no]{obs*}
\declaretheorem[style=normal,name=Proposició,sibling=lema]{prop}
\declaretheorem[style=normal,name=Proposició,numbered=no]{prop*}
\declaretheorem[style=normal,name=Definició,sibling=lema]{defi*}
\declaretheorem[style=normal,name=Corol·lari,sibling=lema]{col}
\declaretheorem[style=normal,name=Corol·lari,numbered=no]{col*}
\declaretheorem[style=normal,name=Exercici,sibling=lema]{ej}
\declaretheorem[style=normal,name=Exercici,numbered=no]{ej*}
\declaretheorem[style=normal,name=Exemple,sibling=lema]{example}
\declaretheorem[style=normal,name=Exemple,numbered=no]{example*}
\declaretheorem[style=normal,name=Problema,sibling=lema]{problema}
\declaretheorem[style=normal,name=Problema,numbered=no]{problema*}

% Autodefi
\declaretheorem[style=autodefi,name=Definició,sibling=lema]{defi}

% Demo
\declaretheorem[style=demo,name=Demostració,qed=$\square$,numbered=no]{proof}
\declaretheorem[style=demo,name=Solució,numbered=no]{sol}

% Break
\declaretheorem[style=break,name=Teorema,sibling=lema]{teo*}

% Breakthm
\declaretheorem[style=breakthm,name=Teorema,sibling=lema]{teo}
