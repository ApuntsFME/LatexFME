\chapter{Permutaciones}

\section{Repaso de permutaciones}

El grupo simétrico $\lp S_n, \circ \rp$ es el grupo de las permutaciones de los elementos
$\setb{1, 2, \dots, n}$ y el cardinal de $S_n$ es $\# S_n = \abs{S_n} = n!$

Si $\sigma \in S_n$, podemos escribir $\sigma$ como
\[
    \begin{pmatrix}
        1 & 2 & 3 & \cdots & n \\
        \sigma(1) & \sigma(2) & \sigma(3) & \cdots & \sigma(n)
    \end{pmatrix}
\]

Cualquier permutación se descompone en ciclos, por ejemplo
$\sigma = (1, 4, 5, 2) \in S_5$ es lo mismo que
\[
    \sigma = 
    \begin{pmatrix}
        1 & 2 & 3 & 4 & 5 \\
        4 & 1 & 3 & 5 & 2
    \end{pmatrix}
\]

Entones $\sigma = \sigma_1 \cdots \sigma_r$ con $\sigma_i$ son ciclos disjuntos.

\begin{obs}
    La multiplicación no es conmutativa. Pero las permutaciones con elementos disjuntos sí que conmutan.
\end{obs}

Todo ciclo se puede descomponer como producto de transposiciones $z = (i, j)$

Por lo tanto, podemos descomponer toda permutación como producto de transposiciones, pero esta
descomposición no es única.

Lo que sí que se mantiene es la paridad del número de trasposiciones. Es decir,
\[
    \begin{aligned}
        \sigma &= z_1 \cdots z_r \\
        \sigma &= \bar{z}_1 \cdots \bar{z}_s
    \end{aligned} \implies r \text{ par} \iff s \text{ par}.
\]

Esto nos permite definir unequívocamente el signo de la permutación:
\[
    \sgn(\sigma) = (-1)^{r},
\]
donde $r$ es el número de trasposiciones.

\begin{defi}[orden de una permutación]
    Definimos el orden de una permutación como el mínimo $k$ tal que $\sigma^k = \Id$
\end{defi}

\begin{example}
    $\sigma = (1, 4, 5, 2)$. Calcular el orden de $\sigma$.
    \[
        \sigma^2 = \sigma \cdot \sigma =
        \begin{pmatrix}
            1 & 2 & 3 & 4 & 5 \\
            4 & 1 & 3 & 5 &2
        \end{pmatrix}
        \begin{pmatrix}
            1 & 2 & 3 & 4 & 5 \\
            4 & 1 & 3 & 5 & 2
        \end{pmatrix} = 
        \begin{pmatrix}
            1 & 2 & 3 & 4 & 5 \\
            5 & 4 & 3 & 2 & 1
        \end{pmatrix}
    \]
    Y así sucesivamente, llegaremos a que $\sigma^4 = \Id$.
\end{example}

Más en general, se tiene que, si $\sigma = \sigma_1 \sigma_2 \cdots \sigma_n$ entonces
\[
    \ord(\sigma) = \mcm \left( \ord (\sigma_1), \ord(\sigma_2), \dots, \ord(\sigma_r) \right)
\]

\section{Ejercicios}

\begin{ej} %TODO poner ejercicios 3 y 4
    En general toda permutación de $S_n$ descompone en producto de trasposiciones
    $(1, 2), (1, 3), \dots (1, n)$
\end{ej}

\begin{proof}
    En general tenemos que una trasposición cualquiera
    \[
        (i, j) = (1, i)(1, j)(1,i)
    \]
\end{proof}

\begin{ej}
    \[
        \sigma =
        \begin{pmatrix}
            1 & 2 & 3 & 4 & 5 & 6 & 7 & 8 & 9 \\
            3 & 7 & 8 & 9 & 4 & 5 & 2 & 1 & 6
        \end{pmatrix}
        = (1, 3, 8) (2, 7) (4, 9, 6, 5)
    \]
    Por lo tanto, el orden de $\sigma$ es:
    \[
        \ord(\sigma) = \mcm\left( 3, 2, 4 \right) = 12
    \]
    Ahora, descomponemos en trasposiciones:
    \[
        \begin{aligned}
            (1, 3, 8) &= (1, 8) (1, 3) \\
            (2, 7) &= (2, 7) \\
            (4, 9, 6, 5) &= (4, 5) (4, 6) (4, 9)
        \end{aligned}
    \]
    Con lo cual, $\sgn(\sigma) = (-1)^6 = 1$.
\end{ej}

\begin{ej} %TODO de verdad nos queremos aburrir con los casos? preguno eh.
    Encontrar todos los valores $x, y, z, t$ tales que 
    \[
        \sigma =
        \begin{pmatrix}
            1 & 2 & 3 & 4 & 5 & 6 & 7 & 8 \\
            3 & 5 & 6 & x & y & 1 & z & t
        \end{pmatrix}
    \]
    tenga orden tres.
    Primero descomponemos $\sigma$:
    \[
        \sigma = (1, 3, 6)
        \begin{pmatrix}
            2 & 4 & 5 & 7 & 8 \\
            5 & x & y & z & t
        \end{pmatrix}
    \]
    y queremos que el segundo miembro tenga orden 3.

    \begin{itemize}
        \item Si $y = 2$, tenemos el ciclo $(2, 5)$ que tiene orden 2 y por lo tanto $\ord(\sigma)$
            es multiplo de 2.
        \item Si $y = 4$ 
    \end{itemize}
\end{ej}
