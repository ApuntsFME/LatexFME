\chapter{Grupos}

\section{Grupos}

\begin{defi}[grupo]
    Un grupo es un par $\lp G, \cdot \rp$, donde $G$ es un conjunto no vacío y $\cdot$ es una operación interna, es decir, una aplicación
    \[
        \begin{aligned}
            \cdot \colon G \times G &\to G \\
            (a, b) &\mapsto a \cdot b
        \end{aligned}
    \]
    que satisface
    \begin{enumerate}[i)]
        \item $(a\cdot b)\cdot c = a\cdot (b\cdot c)$, i.e., la propiedad asociativa,
        \item $\exists e \tq \forall a \in G, \, a\cdot e = e\cdot a = a$, i.e., existe un elemento neutro,
        \item $\forall a \in G, \, \exists \tilde{a} \in G \tq a \cdot \tilde{a} = \tilde{a}\cdot a = e$, i.e., todo elemento tiene inverso.
    \end{enumerate}
    \emph{Nota.} Cuando la operación del grupo sea irrelevante, evidente o se deduzca del contexto, escribiremos $G$ en lugar de $\lp G, \cdot\rp$, o de $\lp G, +\rp$, etc., cometiendo un abuso de notación.
    
    \vspace{1.15ex}
    
    \noindent \emph{Nota segunda.} Además, a menudo también escribiremos $ab$ en lugar de $a\cdot b$ y $a\cdot b$ en lugar de $a\circ b$, como por ejemplo en la composición de permutaciones.
\end{defi}

\begin{defi}[grupo!abeliano o conmutativo]
    Decimos que $G$ es un grupo abeliano o conmutativo si es un grupo y además satisface la propiedad conmutativa:
    \[
        ab = ba, \quad \forall a, b \in G.
    \]
\end{defi}

\begin{obs}
    Existen varias notaciones para referirnos a esta operación:
    \begin{center}
        \begin{tabular}{|c|c|c|c|} \hline
            Operación & S\'imbolo & Elemento neutro & Elemento inverso \\ \hline \hline
            Aditiva & $+$ & 0 & $-a$ (e. opuesto) \\ \hline
            Multiplicativa & $\cdot$ & 1 & $a^{-1}$ \\ \hline
        \end{tabular}
    \end{center}
    \emph{Nota.} Siempre que utilicemos $+$ para la operación del grupo, la operación será conmutativa.
\end{obs}

\begin{defi}[subgrupo]
    Sea $\lp G, \cdot \rp$ un grupo. Decimos que $\lp H, \cdot_{\mid H} \rp$ es un subgrupo de $\lp G, \cdot \rp$ si $H \subseteq G$ y se satisface
    \begin{enumerate}[i)]
        \item $H \neq \emptyset$,
        \item $a, b \in H \implies a\cdot b \in H$ (la operación es cerrada),
        \item $\forall a \in H, a^{-1} \in H$.
    \end{enumerate}
    \emph{Nota. } A menudo cometeremos un abuso de notación, escribiendo $\lp H, \cdot \rp$ en lugar de $\lp H, \cdot_{\mid H} \rp$.
\end{defi}

\begin{prop}
    Los subgrupos son aquellos grupos $\lp H, \cdot_{\mid H} \rp$ con $H\subseteq G$.
\end{prop}
\begin{proof}
    Sea $\lp H, \cdot \rp$ un subgrupo de $\lp G, \cdot \rp$. Queremos ver que $\lp H, \cdot \rp$ es un grupo.
    Tenemos la operación
    \[
        \begin{aligned}
            \cdot \colon H \times H &\to H \\
            (a, b) &\mapsto a\cdot b \in H.
        \end{aligned}
    \]
    Tiene la propiedad asociativa porque es la restricción de una operación con la propiedad asociativa.
    Existe elemento neutro ya que $\exists a \in H$ y $\exists a^{-1} \in H$, de modo que $a\cdot a^{-1}=e \in H$.
    La última propiedad está impuesta.
    
    Recíprocamente, veamos que si $H\subseteq G$ y $\lp H, \cdot_{\mid H} \rp$ es un grupo, entonces $\lp H, \cdot_{\mid H} \rp$ es un subgrupo de $G$. Por ser $\lp H, \cdot_{\mid H} \rp$ un grupo, $1\in H \implies H\neq\varnothing$. Las otras dos propiedades están en la propia definición de grupo.
\end{proof}

\begin{example}
    \begin{enumerate}[1.]
        \item[]
        \item Sea $G$ un grupo. Los subgrupos impropios son
                $\begin{cases}
                    \setb{1} &\text{(el subgrupo trivial)}, \\
                    G. &
                \end{cases}$
        \item $\lp \z, + \rp, \lp \n, + \rp, \lp \real, + \rp, \lp \cx, + \rp$ son grupos y subgrupos.
        \item $\lp \z/n\z, + \rp$ es un grupo.
        \item Si $G$ y $H$ son dos grupos, entonces
            \[
                 G \times H = \setb{(x,y) \vert x \in G, \, y \in H}   
            \]
            es un grupo, con $(a, b) \cdot (c, d) = (ac, bd)$.
        \item $\lp S_n, \circ \rp$ es el grupo simétrico de $n$ elementos (permutaciones de $n$ elementos).
        \item Grupo diedial. $\lp D_{2n}, \circ \rp$, donde $D_{2n}$ son los conjuntos de las isometrías del plano que dejan invariante $P_n$.
            $P_n$ es un polígono regular de $n$ lados (raices $n$-esimas de 1). Por ejemplo,
            \[
                D_{2 \cdot 4} = \setb{id, r, r^2, r^3, s, rs, r^2s, r^3s}
            \]
            Con $r$ la rotación horaria de $\pi / 2$ y $s$ la simetría respecto del eje $x$.
    \end{enumerate}
\end{example}

\section{Intersección y producto de subgrupos}

\begin{defi}[intersección de subgrupos]
    Sea $G$ un grupo y sean $H, K \subset G$ subgrupos de $G$. Definimos la intersección de
    $H$ y $K$ como
    \[
        H \cap K = \setb{x \in G \; \vert \; x \in H \text{ y } x \in K}.
    \]
\end{defi}

\begin{obs}
    Si $H$ y $K$ son subgrupos de $G$, $H \cap K$ es un subgrupo de $G$.
    También es cierto con la instersección arbitraria.
\end{obs}

\begin{defi}[unión de subgrupos]
    Sea $G$ un grupo y sean $H, K \subseteq G$ subgrupos de $G$. Llamamos unión de $H$ y $K$ a
    \[
        H \cup K = \setb{x \in G \, \vert\, x \in H \text{ o } x \in K}.
    \]
\end{defi}

\begin{obs}
    En general, la unión de subgrupos no es un grupo.
\end{obs}

\begin{example}
    Tomamos el grupo simétrico como ejemplo:
    \[
        \Sim_3 = \setb{ \Id, (1 \, 2), (1 \, 3), (2 \, 3), (1 \, 2 \, 3), (1 \, 3 \, 2)}
    \]
    y tomamos
    \[
        H = \setb{\Id, (1 \, 2)}, \, K = \setb{\Id, (1 \, 3)}
    \]
    ahora
    \[
        H \cup K = \setb{\Id, (1 \, 2), (1\, 3)}
    \]
    pero
    \[
        (1 \, 2)(1 \, 3) = (1 \, 3 \, 2) \notin H \cup K.
    \]
\end{example}

\begin{defi}[producto de subgrupos]
    Sea $G$ un grupo y sean $H, K \subset G$ subgrupos. Definimos el producto $H \cdot K$ como
    \[
        H \cdot K = \setb{xy \,\vert\, x \in H \text{ y } x \in K}.
    \]
\end{defi}

\begin{obs}
    En general, el producto de subgrupos, no es grupo.
\end{obs}

\begin{example}
    Tomando las definiciones de $G$, $H$ y $K$ del ejemplo anterior, tenemos que
    \[
        H \cdot K = \setb{\Id, (1 \, 3), (1 \, 2), (1 \, 2)(1 \, 3) = (1 \, 3 \, 2)},
    \]
    que no es un grupo.
\end{example}

\begin{obs}
    Si $G$ es conmutativo, el producto de subgrupos es un grupo.
\end{obs}

\begin{proof}
    Comprobemos que $H \cdot K$ satisface las propiedades de los grupos.
    \begin{enumerate}[i)]
        \item 
            $
                \begin{rcases}
                    H \text{ sg.} \implies 1 \in H \\
                    K \text{ sg.} \implies 1 \in K
                \end{rcases}
                \implies 1 = 1\cdot 1 \in H \cdot K.
            $
        \item
            $
                \begin{rcases}
                    xy \in HK \\
                    zt \in HK
                \end{rcases}
                \implies (xy)(zt) = (xyzt) = (xz)(yt) \in HK.
            $
        \item
            $
                (xy)^{-1} = y^{-1} x^{.1} = x^{-1}y^{-1} \in HK.
            $
    \end{enumerate}
\end{proof}

\begin{obs}
    Se tiene que
    \[
        H \cap K \subseteq H, K \subseteq H \cup K \subseteq H \cdot K.
    \]
\end{obs}

\begin{obs}
    Si $HK$ es un subgrupo, entonces es el menor subgrupo que contiene a $H \cup K$. 
\end{obs}

\begin{proof}
    Es claro que $H \cup K \subseteq HK$, ya que
    \[
        \forall x \in H, \, x\cdot1 = x \in HK
    \]
    y análogamente para $K$.

    Suponemos ahora que $L$ es un subgrupo de $G$ que contiene a $H \cup K$. Queremos ver que
    $ H \cdot K \subseteq L$. Tenemos que, dado cualquier $z = ab \in HK$ ($a \in H$, $b \in K$)
    \[
        \begin{rcases}
            a \in H \subset L \\
            b \in K \subset L \\
            L \text{ es subgrupo}
        \end{rcases}
        \implies ab = z \in L
    \]
    por lo tanto, $HK \in L$.
\end{proof}

\begin{defi}[subgrupo!generado]
        Sea $\lp G, \cdot\rp$ un grupo y sea $S\subseteq G$. Definimos el subgrupo generado por $S$ a
        \[
            \left<S\right> = \lp \lc 1 \rc \cup \setb{a_1 \cdots a_r \,\vert\, a_i \in S \text{ ó } a^{-1}_i \in S}, \cdot \rp.
        \]
\end{defi}

\begin{obs}
    Si $S = \emptyset$, entonces $\left<S\right> = \lp \setb{1}, \cdot \rp$.
\end{obs}

\begin{obs}
    $\left<S\right>$ es el menor subgrupo de de $G$ que contiene a $S$.
\end{obs}

\begin{proof}
    Si $S = \emptyset$ entonces es trivial.
    Si $S \neq \emptyset$, es trivial que $S \subset \left<S\right>$, veamos ahora que es un subgrupo de $G$.
    \begin{enumerate}[i)]
        \item $\exists a \in S \implies a \in \left<S\right> \implies \left<S\right> \neq \emptyset$,
        \item Si $a_1\cdots a_r, b_1 \cdots b_s \in \left<S\right>$, entonces $a_1 \cdots a_r b_1 \cdots b_s \in \left<S\right>$,
        \item Si $a_1 \cdots a_r \in \left<S\right>$, entonces $\left( a_1 \cdots a_r \right)^{-1} =
            a^{-1}_r \cdots a^{-1}_1 \in \left<S\right>$.
    \end{enumerate}
    Tomamos ahora $L$ un subgrupo de $G$ que contiene a $S$. Queremos ver que $\left<S\right> \subseteq L$.
    Para cualquier $a_1 \cdots a_r \in \left<S\right>$ tenemos que
    \[
        \begin{rcases}
            a_1 \in S \subseteq L \text{ o } a^{-1}_1 \in S \implies \left( a^{-1}_1 \right)^{-1} \in L\\
            \vdots \\
            a_r \in S \subseteq L\text{ o } a^{-1}_r \in S \implies \left( a^{-1}_r \right)^{-1} \in L
        \end{rcases}
        \implies a_1 \cdots a_r \in L.
    \]
    y por lo tanto, $\left<S\right> \subset L$.
\end{proof}

\begin{ej}
    Demostrar que
    \[
        \left<S\right> = \setb{a^{n_1}_1 \cdots a^{n_r}_r \,\vert\, a_i \in S, \, n_i \in \z}.
    \]
\end{ej}

\begin{ej}
    Demostrar que
    \[
        \left<S\right> = \bigcap_{\substack{H \text{ sg. de } G \\ S \subseteq H}} H.
    \]
\end{ej}

\section{Orden de un elemento}

\begin{defi}[orden!de los elementos de un grupo]
    Sea $G$ un grupo y sea $x \in G$. Llamamos orden de $x$, si existe, al menor entero $n \geq 1$ tal que
    \[
        x^n = 1.
    \]
    Si no existe, decimos que $x$ tiene orden infinito.
\end{defi}

\begin{obs}
    Escribimos el orden de $x$ como $\orden(x)$ o $\ord(x)$.
\end{obs}

\begin{defi}[orden!de un grupo]
    Sea $G$ un grupo. Llamamos orden de $G$ a su cardinal y lo denotamos $\orden(G)$, $\ord(G)$, $\abs{G}$ o $\card(G)$.
\end{defi}

\begin{example}
    \begin{enumerate}[1.]
        \item[]
        \item $\ord(e) = 1$ y es el único elemento (el neutro) que tiene orden 1.
        \item En el grupo simétrico $G = \Sim_n$, $\ord\left( a_1, \dots, a_n \right) = n$.
        \item En los grupos $\lp \z, +\rp , \lp \q, +\rp, \lp \real, +\rp$ y $\lp \cx, +\rp$, $\forall x \neq 0 \, \ord(x) = \infty$.
        \item En el grupo $\z/p\z$ con $p$ primo, $\forall \bar{x} \neq \bar{0} \, \ord\left( \bar{x} \right) = p$.
        \item En los grupos $\q^{\ast}=\lp \q\setminus \lc 0\rc, \cdot \rp, \real^{\ast}=\lp \real\setminus \lc 0\rc, \cdot\rp$, $\ord(-1) = 2$, $\ord(1) = 1$ y $\forall x \notin \setb{-1, 1} \, \ord(x) = \infty$.
        \item En el grupo $\cx^{\ast}=\lp \cx\setminus \lc 0\rc, \cdot \rp$, $\forall n \geq 1$ $\ord\left( e^{\frac{2 \pi i}{n}} \right) = n$ y $\forall z \in \cx \tq \abs{z} \neq 1, \, \ord(z) = \infty$.
    \end{enumerate}
\end{example}

\begin{lema}
    Sea $G$ un grupo y $x \in G$ con $\ord(x) = n\in\n$. Entonces,
    \begin{enumerate}[i)]
        \item $x^m = 1 \iff n \vert m$.
        \item $x^m = x^{m^\prime} \iff m \cong m^\prime \, \pmod{n}$.
        \item $\ord\left( x^m \right) = \frac{n}{\mcd(n, m)} = \frac{\ord(x)}{\mcd\left( \ord(x), m \right)}$.
    \end{enumerate}
\end{lema}

\begin{proof}
    \begin{enumerate}[i)]
        \item[]
        \item Si $n \vert m$, entonces $m = n \cdot d$, con lo cual, $x^m = \left( x^n \right)^d = 1^d = 1$. Recíprocamente, pongamos $m = nq + r$, con $0 \leq r < n$. Entonces,
            \[
                \begin{rcases}
                    1 = x^m = x^{nq + r} = \left( x^n \right)^q x^r = x^r \\
                    0 \leq r < n
                \end{rcases}
                \implies r = 0 \implies n \vert m.
            \]
        \item $x^m = x^{m^\prime} \iff x^{m - m^\prime} = 1 \iff n \vert m - m^\prime \iff
            m \cong m^\prime \pmod{n}$.
        \item Sean $k = \ord\left( x^m \right)$ y $g = \mcd(n, m)$. Queremos ver que $k = n/g$.
            \[
                \left( x^m \right)^{\frac{n}{g}} = x^{\frac{mn}{g}} = \left( x^n \right)^{\frac{m}{g}} = 1
                \implies k \left\vert \frac{n}{g} \right. .
            \]
            Por otro lado,
            \[
                1 = \left( x^m \right)^k = x^{mk} \implies n \vert mk \implies \left.\frac{n}{g} \right\vert \frac{m}{g} k
                \substack{n/g \text{ y } m/g \\ \implies \\ \text{ primos entre si}} \left.\frac{n}{g} \right\vert k.
            \]
            Y sumando los dos resultados, tenemos que $\frac{n}{g} = k$.
    \end{enumerate}
\end{proof}

\begin{defi}[grupo!cíclico]
    Diremos que un grupo $G$ es cíclico si está generado por un solo elemento $x \in G$. Escribimos $G = \left< x \right>$, $G$ generado por $x$ o $x$ generdor de $G$.
\end{defi}

\begin{obs}
    Si $\ord(G) = n$ (con $n$ finito), entonces
    \[
        G = \setb{ 1 \left( = x^0 \right), x, x^2, \dots, x^{n-1}}.
    \]
    Y lo denotaremos cono $G = C_n$ (grupo cíclico de orden $n$).
    Si $\ord(G) = \infty$, entonces
    \[
        G = \setb{x^k \vert k \in \z}.
    \]
\end{obs}

\begin{example}
    \begin{enumerate}[1.]
        \item[]
        \item $\z = \left< 1 \right> = \left< -1 \right>$.
        \item $\z/n\z = \left< \bar{k} \right>$ con $\mcd(n, k) = 1$.
    \end{enumerate}
\end{example}

\begin{defi}[función de Euler]
    Sea $d \in \z, d \geq 1$, definimos la función $\varphi$ de Euler como
    \[
        \varphi(d) = \card \setb{ 1 \leq k \leq d \vert \mcd(k, d) = 1}.
    \]
\end{defi}

\begin{example}
    \begin{align*}
        \varphi(1) &= 1, & \varphi(5) &= 4, & \varphi(3) &= 2, & \varphi(7) &= 6, \\
        \varphi(2) &= 1, & \varphi(6) &= 2, & \varphi(4) &= 2, & \varphi(p) &= p-1 \text{(con } p \text{ primo).} 
    \end{align*}
\end{example}

\begin{prop}
    Sea $G = \left< x \right>$ un grupo cíclico, con $\ord(x) = n$, entonces
    \begin{enumerate}[i)]
        \item $\forall y \in G, \ord(y) \vert n$,
        \item $\forall d \vert n,$ existen $\varphi(d)$ elementos de $G = C_n$ de orden $d$. De hecho, son
            \[
                \setb{\left. x^{\frac{n}{d} k} \right\vert 1 \leq k \leq d \tq \mcd(k, d) = 1}
            \]
    \end{enumerate}
\end{prop}

\begin{proof}
    \begin{enumerate}[i)]
        \item[]
        \item Por ser $y$ un elemento de $G$, es de la forma $y=x^m,\,0\leq m\leq n$. Entonces,
                \[
                    \ord(y) = \ord\left( x^m \right) = \frac{\ord(x)}{\mcd\left(\ord(x), m \right)} = \frac{n}{\mcd(n, m)}.
                \]
            Y conclumos $\ord\lp y\rp \mid n$.
        \item Sea $y \in G \tq \ord(y) = d$, tenemos que
                \[
                    y = x^m \implies \ord(y) = d = \frac{n}{\mcd(n, m)} \iff \mcd(n, m) = \frac{n}{d}.
                \]
                Buscamos los $m$ tales que $\mcd(m, n) = \frac{n}{d}$.
                \[
                    \mcd(n, m) = \frac{n}{d} \iff \mcd\left( \frac{n}{d}d, \frac{n}{d}k \right) = \frac{n}{d}
                    \iff \mcd(k, d) = 1.
                \]
                Con esta última condición, tenemos que
                \[
                    \varphi(d) = \card \setb{ k \in \z \vert \mcd(k, d) = 1 \text{ y } 1 \leq k \leq d}
                    = \card \setb{ x^m \vert \ord\left( x^m \right) = d}.
                \]
    \end{enumerate}
\end{proof}

\begin{col}
    Se tiene que
    \[
        n = \sum_{d \vert n} \varphi(d).
    \]
\end{col}

\begin{proof}
    Tomamos $G = C_n = \left< x \right>$. Entonces,
    \[
        n = \abs{G} = \sum_{d \vert n} \card \setb{ x \in G \vert \ord(x) =d } = \sum_{d \vert n} \varphi(d),
    \]
    ya que
    \[
        G = \bigcup_{d \vert n} \setb{x \in G \vert \ord(x) = d}.
    \]
\end{proof}

\begin{prop}
    Sea $G = C_n = \left< x \right>$ (con $n$ finito). Entonces
    \begin{enumerate}[i)]
        \item Si $d \vert n$, entonces $x^{\frac{n}{d}}$ es un elemento de orden $d$
            y el subgrupo $H_d := \left< x^{\frac{n}{d}} \right>$ es subgrupo cíclico de orden $d$.
        \item Si $H$ es un subgrupo de $G$, entonces $\exists! d \vert n$ tal que $H = H_d$.
    \end{enumerate}
\end{prop}

\begin{proof}
    \begin{enumerate}[i)]
        \item[]
        \item Se tiene que
            \[
                \orden\left( x^{\frac{n}{d}} \right) = \frac{n}{\mcd(n, \frac{n}{d})} = d.
            \]
            Como $x{\frac{n}{d}}$ tiene orden $d$, $H_d = \left< x^{\frac{n}{d}} \right>$ es un grupo
            cíclico de orden $d$.
        \item Sea $H$ un subgrupo de $G = C_n$ y sea $1 \leq t \leq n$ el menor exponente tal que 
            $x^t \in H$. Veremos que $t \vert n$. Expresamos $n = td + r$ (con $0 \leq r < t$).
            \[
                1 = x^n = x^{td + r} = \left( x^t \right)^d x^r.
            \]
            Como $x^t\in H$, se tiene que $\lp x^t\rp^d, \lp \lp x^t\rp^d \rp ^{-1} \in H$. Así,
            \[
                x^r=\lp \lp x^t\rp^d \rp ^{-1} \lp x^t\rp^d x^r = \lp \lp x^t\rp^d \rp ^{-1} \in H.
            \]
            Pero $t$ es el exponente más pequeño (a excepción del 0) tal que $x^r\in H$ y, en consecuencia, $r = 0$ y $n = td$.
            Veremos ahora que $H = H_d$.
            
            Claramente, $\left< x^{\frac{n}{d}} \right> = H_d \subseteq H$, ya que $x^{\frac{n}{d}} = x^t \in H$
            y, por lo tanto, todos sus múltiplos están en $H$.

            Sea $y = x^m \in H$. Necesariamente, $m\geq t$. Escribimos $m = tq + s$
            (con $0 \leq s < t$). Entonces,
            \[
                x^m = x^{tq + s} = \left( x^t \right)^q x^s \implies
                x^s = \underbrace{\left( \left( x^t \right)^q \right)^{-1}}_{\in H} \cdot
                \underbrace{x^m}_{\in H} \in H.
            \]
            de nuevo, por la definición de $t$, $s = 0$ y concluimos que $y = \left( x^t \right)^q \in 
            \left< x^{\frac{n}{d}} \right> = H_d$, es decir, $H \subseteq H_d$.

            Solo resta ver que $d$ es único, pero es obvio ya que, si $H = H_d = H_e$, entonces,
            \[
                \orden(H) = \orden\left( H_d \right) = \orden\left( H_e \right) \implies
                d = e.
            \]
    \end{enumerate}
\end{proof}

\begin{col}[Retículo de subgrupo de un grupo cíclico]
    Sea $G = C_n = \left< x^n \right>$ un grupo cíclico de orden $n \geq 1$. Existe una biyección
    \[
        \begin{aligned}
            \setb{d \in \n \vert 1 \leq d \leq n, d \vert n } &\longleftrightarrow
            \setb{\text{subgrupos de }G} \\
            d &\longleftrightarrow H_d.
        \end{aligned}
    \]
\end{col}

\section{Morfismos de grupos}

\begin{defi}[homeomorfismo]
    Sean $G_1$, $G_2$ dos grupos y sea $f \colon G_1 \to G_2$ una aplicación. Decimos que $f$ es un
    (homeo)morfismo de grupos si
    \[
        f(xy) = f(x)f(y).
    \]
\end{defi}

\begin{prop}
    Si $f$ es un morfismo de grupos, entonces
    \begin{enumerate}[i)]
        \item $f(1) = 1$,
        \item $f\left( x^{-1} \right) = \left( f(x) \right)^{-1}$.
    \end{enumerate}
\end{prop}

\begin{proof}
    Ejercicio.
\end{proof}

\begin{obs}
    Notación:
    \begin{center}
        \begin{tabular}{|c|c|}
            \hline
            Nombre & Propiedades \\
            \hline\hline
            Monomorfismo & Inyectiva \\\hline
            Epimorfismo  & Exhaustiva \\\hline
            Isomorfismo  & Biyectiva \\\hline
            Endomorfismo & $G_1 = G_2$ \\\hline
            Automorfismo & Biyectiva y $G_1 = G_2$ \\
            \hline
        \end{tabular}
    \end{center}
\end{obs}

\begin{prop}
    Sea $f \colon G_1 \to G_2$ un morfismo biyectivo (isomorfismo). Entonces
    $f^{-1} \colon G_2 \to G_1$ es un morfismo de grupos.
\end{prop}

\begin{proof}
    Ejercicio.
\end{proof}

\begin{defi}[grupo!isomorfo]
    Sean $G_1$ y $G_2$ grupos. Decimos que $G_1$ y $G_2$ son isomorfos si $\exists f \colon G_1 \to G_2$
    isomorfismo. Y lo notaremos como $G_1\cong G_2$.
\end{defi}

\begin{prop}
    Sea $f \colon G_1 \to G_2$ un morfismo de grupos.
    \begin{enumerate}
        \item Si $H$ es un subgrupo de $G_1$, entonces $f(H)$ es subgrupo de $G_2$.
        \item Si $K$ es un subgrupo de $G_2$, entonces, $f^{-1}(K)$ es subgrupo de $G_1$.
    \end{enumerate}
\end{prop}

\begin{proof}
    Ejercicio.
\end{proof}

\begin{obs}
    \begin{itemize}
        \item[]
        \item $f\left( G_1 \right) = \im(f)$.
        \item $f^{-1}(1) = \ker(f)$.
    \end{itemize}
\end{obs}

\begin{prop} Sean $G_1, G_2$ grupos y sea $f\colon G_1\to G_2$ un morfismo de grupos. Entonces,
    \begin{enumerate}[i)]
        \item $f$ inyectiva $\iff \ker(f) = \setb{1}$.
        \item $f$ exhaustiva $\iff \im(f) = G_2$.
    \end{enumerate}
\end{prop}

\begin{proof}
    Ejercicio.
\end{proof}

\section{Clases laterales}

\begin{defi}[elemento relacionado!por la izquierda]
    Sea $G$ un grupo y $H \subseteq G$ un subgrupo. Dados $a, b \in G$, decimos que
    $a$ está relacionado con $b$ por la izquierda si
    \[
        a^{-1}b \in H.
    \]
\end{defi}

\begin{ej}
    Demostrar que que la relación definida es una relación de equivalencia. Para ello, hace falta ver que es
    reflexivo, simétrico y transitivo.
\end{ej}

\begin{defi}[clase lateral!por la izquierda]
    Con la relación que hemos visto ahora, denotamos la clase de equivalencia de $a \in G$ como
    \[
        \begin{aligned}
            \bar{a} &= \setb{b \in G \vert a^{-1}b = x, \, x \in H} =\\ &=
            \setb{b \in G \vert b = ax, \, x \in H} =\\&= aH.
        \end{aligned}
    \]
    y llamaremos a $aH$ clase lateral por la izquierda del elemento $a$ módulo el subgrupo $H$.
\end{defi}

\begin{example}
    Si $a = 1$, tenemos que
    \[
        1\cdot H = \setb{1x \vert x \in H} = H.
    \]
\end{example}

\begin{obs}
    Tomamos
    \[
        \begin{aligned}
            f_a \colon G &\to G \\
            x &\mapsto f_a(x) = ax
        \end{aligned}
    \]
    una aplicación biyectiva ($f^{-1}_a = f_{a^{-1}}$). Notemos, que $f$ no es un morfismo
    de grupos (en general), ya que $f(1) = a$ (en general $a \neq 1$). Se tiene también que
    \[
        f_a(H) = \setb{f_a(x) \vert x \in H} = \setb{ax \vert x \in H} = aH.
    \]
    Diremos pues que hay una biyección $H \leftrightarrow aH$, en particular, si $G$ es finito, se tiene que
    $\abs{H} = \abs{aH}$.
\end{obs}

\begin{defi}[conjunto cociente de un grupo]
    Sea $G$ un grupo y sea $H$ un subgrupo de $G$. Llamamos conjunto cociente de $G$ módulo $H$ a 
    \[
        \faktor{G}{H} = \setb{aH \vert a \in G} = \setb{ \bar{a} \vert a \in G},
    \]
    es decir, el conjunto de las clases laterales por la izquierda de $G$ módulo $H$.
    
\end{defi}

\begin{teo}[de Lagrange]
    Sea $G$ un grupo finito y sea $H$ un subgrupo de $G$. Entonces,
    \[
        \abs{G} = \abs{H} \abs{\faktor{G}{H}}.
    \]
\end{teo}

\begin{proof}
    Se tiene que
    \[
        G = \bigsqcup \bar{a}H.
    \]
    Por lo tanto,
    \[
        \abs{G} = \abs{\bigsqcup \bar{a}H} = \sum \abs{\bar{a}H} = \sum^{\abs{\faktor{G}{H}}}_{k = 1} \abs{H} = \abs{H} \abs{\faktor{G}{H}}.
    \]
\end{proof}

\begin{col}
    Si $G$ es finito y $H$ es subgrupo, entonces $\abs{H}$ divide a $\abs{G}$. Si $x \in G$,
    \[
        \orden(x) \vert \orden(G) = \abs{G}.
    \]
\end{col}

\begin{proof}
    Ya que $\orden(x) = \orden\left( \left< x \right> \right)$ y $\orden(H) \vert \orden(G)$.
\end{proof}

\begin{defi}[elemento relacionado!por la derecha]
    Sea $G$ un grupo y sea $H \subseteq G$ un subgrupo, decimos que $a, b \in G$ están relacionados
    por la derecha si
    \[
        ab^{-1} \in H.
    \]
\end{defi}

\begin{ej}
    Demostrar que se trata de una relación de equivalencia (propiedades simétrica, transitiva y reflexiva).
\end{ej}

\begin{defi}[clase lateral!por la derecha]
    Tenemos que
    \[
        \bar{a} = \setb{b \in G \vert ab^{-1} = x, \, y \in H} = \setb{b \in G \vert b = ya, \, y \in H} = Ha.
    \]
    Llamamos clase lateral por la derecha de $a$ módulo $H$ a $Ha$.
\end{defi}

\begin{obs}
    Sea
    \[
        \begin{aligned}
            g_a \colon G &\to G \\
            x &\mapsto g_a(x) = xa.
        \end{aligned}
    \]
    Notamos que es una aplicación biyectiva ($g^{-1}_a = g_{a^{-1}}$), pero que no es un morfismo de grupos.
    Además, $g_a(H) = Ha$ y, por lo tanto, se tiene una biyección $H \leftrightarrow Ha$. En particular, si
    $G$ es finito, $\abs{H} = \abs{Ha}$.
\end{obs}

\begin{prop}
    Existe una biyección
    \[
        \begin{aligned}
            \setb{aH \vert a \in G} &\to \setb{Hb \vert b \in G} \\
            xH &\mapsto Hx^{-1}.
        \end{aligned}
    \]
\end{prop}

\begin{proof}
    Ejercicio: demostrar que está bien definida y que es biyectiva.
\end{proof}

\begin{defi}[índice de un grupo en un subgrupo]
    Sea $G$ un grupo y $H \subseteq G$ un subgrupo. Llamamos índice de $G$ en $H$ al cardinal de $\faktor{G}{H}$.
    Y lo denotamos como
    \[
        \left[G : H \right] = \abs{\faktor{G}{H}} \stackrel{\text{TL}}{=} \frac{\abs{G}}{\abs{H}}.
    \]
\end{defi}

\section{Subgrupos normales. Grupo cociente}

\begin{defi}
    Sea $G$ un grupo y $H \subseteq G$ un subgrupo. Decimos que $H$ es un subgrupo normal de $G$ si
    $\forall a \in G$
    \[
        aH = Ha,
    \]
    y lo denotaremos como $H \triangleleft G$.
\end{defi}

\begin{obs}
    $H \triangleleft G$ no quiere decir que $ax = xa$ ($x \in H, \, a \in G$). Quiere decir que
    $\forall x \in H, \, a \in G, \, \exists y \in H$ tal que $ax = ya$.
\end{obs}

\begin{prop}
    Sea $G$ un grupo y $H \subseteq G$ un subgrupo, entonces son equivalentes
    \begin{enumerate}[(i)]
        \item\label{item:eq_norm_1} $H \triangleleft G$,
        \item\label{item:eq_norm_2} $aH = Ha, \, \forall a \in G$,
        \item\label{item:eq_norm_3} $aH \subseteq Ha, \, \forall a \in G$,
        \item\label{item:eq_norm_4} $aHa^{-1} = H, \, \forall a \in G$,
        \item\label{item:eq_norm_5} $aHa^{-1} \subseteq H, \, \forall a \in G$.
    \end{enumerate}
\end{prop}

\begin{proof}
    En primer lugar, \ref{item:eq_norm_1}$\iff$\ref{item:eq_norm_2} por definición y \ref{item:eq_norm_2}$\implies$\ref{item:eq_norm_3} es inmediato.
    
    \noindent Veamos que \ref{item:eq_norm_3}$\implies$\ref{item:eq_norm_5}. Sea $x=aba^{-1}\in aHa^{-1}, b\in H$. Entonces, 
\end{proof}

\begin{example}
    \begin{enumerate}[1.]
        \item Tomamos $G = \Sim_3 = \setb{\Id, (12), (13), (23), (123), (132)}$ y
            $H = A_3 = \setb{\Id, (123), (132)}$. Sabemos ahora que
            \[
                \abs{G} = \abs{H}\abs{\faktor{G}{H}} \implies
                \frac{\abs{G}}{\abs{H}} = \frac{6}{3} = 2 = \abs{\faktor{G}{H}}
            \]
            Como $\forall x \in H, \, xH = H = Hx$, $H$ es un grupo normal, ya que solo existen 2 clases.
        \item Tomamos $G = D_{2 \cdot 4} = \setb{\Id, r, r^2, r^3, s, rs, r^2s, r^3s}$ y
            $H = \setb{ \Id, r, r^2, r^3}$.
    \end{enumerate}
\end{example}

\begin{prop}
    Sea $G$ un grupo finito y $H \subseteq G$ un subgrupo,
    \[
        [G:H] = 2 \implies H \triangleleft G
    \]
\end{prop}

\begin{example}
    Tomamos $G = \Sim_3$ y $H = \setb{ \Id, (12)}$. Tenemos que
    \[
        \begin{aligned}
            (13)H &= (13) \setb{\Id, (12)} = \setb{(13), (123)} \\
            H(13) &= \setb{\Id, (12)} (13) = \setb{(13), (132)}
        \end{aligned}
    \]
\end{example}

\begin{lema}
    Sean $G_1, G_2$ grupos y sea $f \colon G_1 \to G_2$ un morfismo de grupos, entonces
    \begin{enumerate}[i)]
        \item $H \triangleleft G_1 \implies f(H) \triangleleft f\left(G_1 \right)$
        \item $K \triangleleft G_2 \implies f^{-1}(K) \triangleleft G_1$
    \end{enumerate}
\end{lema}

\begin{proof}
    \begin{enumerate}[i)]
        \item Veremos que $f(H) \triangleleft f\left( G_1 \right)$. Sea $f(x) \in f(H)$ y
            $\forall f(a) \in f\left( G_1 \right)$.
            \[
                f(a) f(x) f(a)^{-1} = f(a) f(x) f\left( a^{-1} \right) =
                f\left( axa^{-1} \right) \in f(H)
            \]
            Ya que $axa^{-1} \in H$.
        \item Ejercicio.
    \end{enumerate}
\end{proof}

\begin{obs}
    Si $G$ es un grupo conmutativo, todo subgrupo es normal.
\end{obs}

\begin{obs}
    Sea $G$ un grupo, entonces $G$ y $\setb{\Id}$ son subgrupos normales.
\end{obs}

\begin{prop}
    Sea $G$ un grupo y sean $H \subseteq K \subseteq G$ subgrupos.
    \[
        H \triangleleft G \implies H \triangleleft K
    \]
\end{prop}

\begin{proof}
    Para todo $a \in K$ y $x \in H$ se tiene que
    \[
        axa^{-1} \stackrel{H \triangleleft G}{\in} H
    \]
    Y por lo tanto, $H \triangleleft K$.
\end{proof}

\begin{obs}
    \[
        H \triangleleft K \triangleleft G \not\implies H \triangleleft G
    \]
\end{obs}

\begin{example}
    Sea $G = \Sim_4$, $H = \setb{ \Id, (12)(34)}$ y $K = \setb{ \Id, (12)(34), (13)(24), (23)(14)}$.
    Tenemos que $[K : H]] = 2$, es decir, $H \triangleleft K$. Ejercicio: $K \triangleleft G$. Pero
    $H \not\triangleleft G$:
    \[
        \begin{aligned}
            (123)H &= \setb{(123), (134)} \\
            H(123) &= \setb{(123), (243)}
        \end{aligned}
    \]
    %TODO ordenar esto
    Queremos ver que $\sigma (12)(34) \sigma^{-1} \in K$, se tien que
    \[
        \sigma(12)(34)\sigma^{-1} = \sigma(12)\sigma^{-1}\sigma(34)\sigma^{-1} = (\sigma_1 \sigma_2)
        (\sigma_3 \sigma_4) \in K
    \]
\end{example}
