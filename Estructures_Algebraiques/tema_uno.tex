\chapter{Grupos}

\begin{defi}[grupo]
    Un grupo es un par $\lp G, \cdot \rp$, donde $G$ es un conjunto no vacío y $\cdot$ es una operación interna, es decir, una aplicación
    \[
        \begin{aligned}
            \cdot \colon G \times G &\to G \\
            (a, b) &\mapsto a \cdot b
        \end{aligned}
    \]
    que satisface
    \begin{enumerate}
        \item $(a\cdot b)\cdot c = a\cdot (b\cdot c)$, i.e., la propiedad asociativa,
        \item $\exists e \tq \forall a \in G, \, a\cdot e = e\cdot a = a$, i.e., existe un elemento neutro,
        \item $\forall a \in G, \, \exists \tilde{a} \in G \tq a \cdot \tilde{a} = \tilde{a}\cdot a = e$, i.e., todo elemento tiene inverso.
    \end{enumerate}
\end{defi}

\begin{defi}[grupo!abeliano o conmutativo]
    Diremos que $\lp G, \cdot \rp$ es un grupo abeliano o conmutativo si es un grupo y además satisface la propiedad conmutativa:
    \[
        a\cdot b = b\cdot a, \quad \forall a, b \in G.
    \]
\end{defi}

\begin{obs}
    Existen varias notaciones para referirnos a esta operación:
    \begin{center}
        \begin{tabular}{|c|c|c|c|} \hline
            Operación & S\'imbolo & Elemento neutro & Elemento inverso \\ \hline \hline
            Aditiva & $+$ & 0 & $-a$ (e. opuesto) \\ \hline
            Multiplicativa & $\cdot$ & 1 & $a^{-1}$ \\ \hline
        \end{tabular}
    \end{center}
\end{obs}

\begin{defi}[subgrupo]
    Sea $\lp G, \cdot \rp$ un grupo. Decimos que $\lp H, \cdot_{\mid H} \rp$ (o, cometiendo un abuso de notación, $\lp H, \cdot \rp$) es un subgrupo de $\lp G, \cdot \rp$ si $H \subseteq G$ y se satisface
    \begin{itemize}
        \item $H \neq \emptyset$
        \item $a, b \in H \implies a\cdot b \in H$ (la operación es cerrada)
        \item $\forall a \in H, a^{-1} \in H$
    \end{itemize}
\end{defi}

\begin{obs}
    Los subgrupos son aquellos grupos $\lp H, \cdot_{\mid H} \rp$ con $H\subseteq G$.
\end{obs}
\begin{proof}
    Sea $\lp H, \cdot \rp$ un subgrupo de $\lp G, \cdot \rp$. Queremos ver que $\lp H, \cdot \rp$ es un grupo.
    Tenemos la operación
    \[
        \begin{aligned}
            \cdot \colon H \times H &\to H \\
            (a, b) &\mapsto a\cdot b \in H.
        \end{aligned}
    \]
    Tiene la propiedad asociativa porque es la restricción de una operación con la propiedad asociativa.
    Existe elemento neutro ya que $\exists a \in H$ y $\exists a^{-1} \in H$, de modo que $a\cdot a^{-1}=e \in H$.
    La última propiedad está impuesta.
    
    (creo que hace falta ver el recíproco)
\end{proof}

\begin{example}
    \begin{itemize}
        \item[]
        \item Sea $\lp G, \cdot \rp$ un grupo. Los subgrupos impropios son
                $\begin{cases}
                    \lp \setb{1}, \cdot \rp &\text{(el subgrupo trivial)} \\
                    \lp G, \cdot \rp. &
                \end{cases}$
        \item $\lp \z, + \rp, \lp \n, + \rp, \lp \real, + \rp, \lp \cx, + \rp$ son grupos y subgrupos.
        \item $\lp \z/n\z, + \rp$.
        \item Si $G$ y $H$ son dos grupos, entonces
            \[
                 G \times H = \setb{(x,y) \vert x \in G, \, y \in H}   
            \]
            es un grupo, con $(a, b) \cdot (c, d) = (ac, bd)$.
        \item $\lp S_n, \circ \rp$ es el grupo simétrico de $n$ elementos (permutaciones de $n$ elementos).
        \item Grupo diedial. $\lp D_{2n}, \circ \rp$, donde $D_{2n}$ son los conjuntos de las isometrías del plano que dejan invariante $P_n$.
            $P_n$ es un polígono regular de $n$ lados (raices $n$-esimas de 1). Por ejemplo,

            \[
                D_{2 \cdot 4} = \setb{id, r, r^2, r^3, s, rs, r^2s, r^3s}
            \]
            Con $r$ la rotación horaria de $\pi / 2$ y $s$ la simetría respecto del eje $x$.
    \end{itemize}
\end{example}


