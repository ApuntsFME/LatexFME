\chapter{Grupos}

\section{Grupos}

\begin{defi}[grupo]
    Un grupo es un par $\lp G, \cdot \rp$, donde $G$ es un conjunto no vacío y $\cdot$ es una operación interna, es decir, una aplicación
    \[
        \begin{aligned}
            \cdot \colon G \times G &\to G \\
            (a, b) &\mapsto a \cdot b
        \end{aligned}
    \]
    que satisface
    \begin{enumerate}[i)]
        \item $(a\cdot b)\cdot c = a\cdot (b\cdot c)$, i.e., la propiedad asociativa,
        \item $\exists e \tq \forall a \in G, \, a\cdot e = e\cdot a = a$, i.e., existe un elemento neutro,
        \item $\forall a \in G, \, \exists \tilde{a} \in G \tq a \cdot \tilde{a} = \tilde{a}\cdot a = e$, i.e., todo elemento tiene inverso.
    \end{enumerate}
    \emph{Nota.} Cuando la operación del grupo sea irrelevante, evidente o se deduzca del contexto, escribiremos $G$ en lugar de $\lp G, \cdot\rp$, o de $\lp G, +\rp$, etc., cometiendo un abuso de notación.
    
    \vspace{1.15ex} %TODO asi no eh
    
    \noindent \emph{Nota segunda.} Además, a menudo también escribiremos $ab$ en lugar de $a\cdot b$ y $a\cdot b$ en lugar de $a\circ b$, como por ejemplo en la composición de permutaciones.
\end{defi}

\begin{defi}[grupo!abeliano o conmutativo]
    Decimos que $G$ es un grupo abeliano o conmutativo si es un grupo y además satisface la propiedad conmutativa:
    \[
        ab = ba, \quad \forall a, b \in G.
    \]
\end{defi}

\begin{obs}
    Existen varias notaciones para referirnos a esta operación:
    \begin{center}
        \begin{tabular}{|c|c|c|c|} \hline
            Operación & S\'imbolo & Elemento neutro & Elemento inverso \\ \hline \hline
            Aditiva & $+$ & 0 & $-a$ (e. opuesto) \\ \hline
            Multiplicativa & $\cdot$ & 1 & $a^{-1}$ \\ \hline
        \end{tabular}
    \end{center}
    \emph{Nota.} Siempre que utilicemos $+$ para la operación del grupo, la operación será conmutativa.
\end{obs}

\begin{defi}[subgrupo] \label{defi:subgrupo}
    Sea $\lp G, \cdot \rp$ un grupo. Decimos que $\lp H, \cdot_{\mid H} \rp$ es un subgrupo de $\lp G, \cdot \rp$ si $H \subseteq G$ y se satisface
    \begin{enumerate}[i)]
        \item $H \neq \emptyset$,
        \item $a, b \in H \implies a\cdot b \in H$ (la operación es cerrada),
        \item $\forall a \in H, a^{-1} \in H$.
    \end{enumerate}
    \emph{Nota. } A menudo cometeremos un abuso de notación, escribiendo $\lp H, \cdot \rp$ en lugar de $\lp H, \cdot_{\mid H} \rp$.
\end{defi}

\begin{prop}
    Los subgrupos son aquellos grupos $\lp H, \cdot_{\mid H} \rp$ con $H\subseteq G$.
\end{prop}
\begin{proof}
    Sea $\lp H, \cdot \rp$ un subgrupo de $\lp G, \cdot \rp$. Queremos ver que $\lp H, \cdot \rp$ es un grupo.
    Tenemos la operación
    \[
        \begin{aligned}
            \cdot \colon H \times H &\to H \\
            (a, b) &\mapsto a\cdot b \in H.
        \end{aligned}
    \]
    Tiene la propiedad asociativa porque es la restricción de una operación con la propiedad asociativa.
    Existe elemento neutro ya que $\exists a \in H$ y $\exists a^{-1} \in H$, de modo que $a\cdot a^{-1}=e \in H$.
    La última propiedad está impuesta.
    
    Recíprocamente, veamos que si $H\subseteq G$ y $\lp H, \cdot_{\mid H} \rp$ es un grupo, entonces $\lp H, \cdot_{\mid H} \rp$ es un subgrupo de $G$. Por ser $\lp H, \cdot_{\mid H} \rp$ un grupo, $1\in H \implies H\neq\varnothing$. Las otras dos propiedades están en la propia definición de grupo.
\end{proof}

\begin{example}
    \begin{enumerate}[1.]
        \item[]
        \item Sea $G$ un grupo. Los subgrupos impropios son $\setb{1}$ (el grupo trivial) y $G$.
        \item $\lp \z, + \rp, \lp \n, + \rp, \lp \real, + \rp, \lp \cx, + \rp$ son grupos y subgrupos.
        \item $\lp \faktor{\z}{n\z}, + \rp$ es un grupo.
        \item Si $G$ y $H$ son dos grupos, entonces
            \[
                 G \times H = \setb{(x,y) \mid x \in G, \, y \in H}   
            \]
            es un grupo, con $(a, b) \cdot (c, d) = (ac, bd)$.
        \item \emph{Grupo simétrico}. $\lp \Sim_n, \circ \rp$ es el grupo simétrico de $n$ elementos (permutaciones de $n$ elementos).
        \item \emph{Grupo diedral}. $\lp D_{2n}, \circ \rp$, donde $D_{2n}$ son los conjuntos de las isometrías del plano que dejan invariante $P_n$.
            $P_n$ es un polígono regular de $n$ lados (raíces $n$-esimas de 1). Por ejemplo,
            \[
                D_{2 \cdot 4} = \setb{id, r, r^2, r^3, s, rs, r^2s, r^3s}
            \]
            Con $r$ la rotación horaria de $\pi / 2$ y $s$ la simetría respecto del eje $x$.
    \end{enumerate}
\end{example}

\section{Intersección y producto de subgrupos}

\begin{defi}[intersección de subgrupos]
    Sea $G$ un grupo y sean $H, K \subset G$ subgrupos de $G$. Definimos la intersección de
    $H$ y $K$ como
    \[
        H \cap K = \setb{x \in G \mid x \in H \text{ y } x \in K}.
    \]
\end{defi}

\begin{obs}
    Si $H$ y $K$ son subgrupos de $G$, $H \cap K$ es un subgrupo de $G$.
    También es cierto con la instersección arbitraria.
\end{obs}

\begin{defi}[unión de subgrupos]
    Sea $G$ un grupo y sean $H, K \subseteq G$ subgrupos de $G$. Llamamos unión de $H$ y $K$ a
    \[
        H \cup K = \setb{x \in G \mid x \in H \text{ o } x \in K}.
    \]
\end{defi}

\begin{obs}
    En general, la unión de subgrupos no es un grupo.
\end{obs}

\begin{example}
    Tomamos el grupo simétrico como ejemplo:
    \[
        \Sim_3 = \setb{ \Id, (1 \, 2), (1 \, 3), (2 \, 3), (1 \, 2 \, 3), (1 \, 3 \, 2)}
    \]
    y tomamos
    \[
        H = \setb{\Id, (1 \, 2)}, \, K = \setb{\Id, (1 \, 3)}
    \]
    ahora
    \[
        H \cup K = \setb{\Id, (1 \, 2), (1\, 3)}
    \]
    pero
    \[
        (1 \, 2)(1 \, 3) = (1 \, 3 \, 2) \notin H \cup K.
    \]
\end{example}

\begin{defi}[producto de subgrupos]
    Sea $G$ un grupo y sean $H, K \subset G$ subgrupos. Definimos el producto $H \cdot K$ como
    \[
        H \cdot K = \setb{xy \mid x \in H \text{ y } x \in K}.
    \]
\end{defi}

\begin{obs}
    En general, el producto de subgrupos, no es grupo.
\end{obs}

\begin{example}
    Tomando las definiciones de $G$, $H$ y $K$ del ejemplo anterior, tenemos que
    \[
        H \cdot K = \setb{\Id, (1 \, 3), (1 \, 2), (1 \, 2)(1 \, 3) = (1 \, 3 \, 2)},
    \]
    que no es un grupo.
\end{example}

\begin{obs}
    Si $G$ es conmutativo, el producto de subgrupos es un grupo.
\end{obs}

\begin{proof}
    Comprobemos que $H \cdot K$ satisface las propiedades de los grupos.
    \begin{enumerate}[i)]
        \item 
            $
                \begin{rcases}
                    H \text{ sg.} \implies 1 \in H \\
                    K \text{ sg.} \implies 1 \in K
                \end{rcases}
                \implies 1 = 1\cdot 1 \in H \cdot K.
            $
        \item
            $
                \begin{rcases}
                    xy \in HK \\
                    zt \in HK
                \end{rcases}
                \implies (xy)(zt) = (xyzt) = (xz)(yt) \in HK.
            $
        \item
            $
                (xy)^{-1} = y^{-1} x^{-1} = x^{-1}y^{-1} \in HK.
            $
    \end{enumerate}
\end{proof}

\begin{obs} \label{obs:inclusiones_ops}
    Se tiene que
    \[
        H \cap K \subseteq H, K \subseteq H \cup K \subseteq H \cdot K.
    \]
\end{obs}

\begin{proof}
    Tenemos que $H \cup K \subseteq HK$ ya que $\forall x \in H, \, x\cdot1 = x \in HK$
    y análogamente para $K$. Las otras inclusiones son triviales a partir de las definiciones.
\end{proof}

\begin{obs}
    Si $HK$ es un subgrupo, entonces es el menor subgrupo que contiene a $H \cup K$. 
\end{obs}

\begin{proof}
    Sabemos que $H \cup K \subseteq HK$ por \ref{obs:inclusiones_ops}.
    Suponemos ahora que $L$ es un subgrupo de $G$ que contiene a $H \cup K$. Queremos ver que
    $ H \cdot K \subseteq L$. Sea $z = ab \in HK$ ($a \in H$, $b \in K$),
    \[
        \begin{rcases}
            a \in H \subset L \\
            b \in K \subset L \\
            L \text{ es subgrupo}
        \end{rcases}
        \implies ab = z \in L \implies HK \in L.
    \]
\end{proof}

\begin{defi}[subgrupo!generado]
        Sea $\lp G, \cdot\rp$ un grupo y sea $S\subseteq G$. Definimos el subgrupo generado por $S$ a
        \[
            \left<S\right> = \lp \lc 1 \rc \cup \setb{a_1 \cdots a_r \mid a_i \in S \text{ ó } a^{-1}_i \in S}, \cdot \rp.
        \]
\end{defi}

\begin{obs}
    Si $S = \emptyset$, entonces $\left<S\right> = \lp \setb{1}, \cdot \rp$.
\end{obs}

\begin{obs}
    $\left<S\right>$ es el menor subgrupo de de $G$ que contiene a $S$.
\end{obs}

\begin{proof}
    Si $S = \emptyset$ entonces es trivial.
    Si $S \neq \emptyset$, es trivial que $S \subset \left<S\right>$, veamos ahora que es un subgrupo de $G$.
    \begin{enumerate}[i)]
        \item $\exists a \in S \implies a \in \left<S\right> \implies \left<S\right> \neq \emptyset$,
        \item Si $a_1\cdots a_r, b_1 \cdots b_s \in \left<S\right>$, entonces $a_1 \cdots a_r b_1 \cdots b_s \in \left<S\right>$,
        \item Si $a_1 \cdots a_r \in \left<S\right>$, entonces $\left( a_1 \cdots a_r \right)^{-1} =
            a^{-1}_r \cdots a^{-1}_1 \in \left<S\right>$.
    \end{enumerate}
    Tomamos ahora $L$ un subgrupo de $G$ que contiene a $S$. Queremos ver que $\left<S\right> \subseteq L$.
    Para cualquier $a_1 \cdots a_r \in \left<S\right>$ tenemos que
    \[
        \begin{rcases}
            a_1 \in S \subseteq L \text{ o } a^{-1}_1 \in S \implies \left( a^{-1}_1 \right)^{-1} \in L\\
            \vdots \\
            a_r \in S \subseteq L\text{ o } a^{-1}_r \in S \implies \left( a^{-1}_r \right)^{-1} \in L
        \end{rcases}
        \implies a_1 \cdots a_r \in L.
    \]
    y por lo tanto, $\left<S\right> \subset L$.
\end{proof}

\begin{ej}
    Demostrar que
    \[
        \left<S\right> = \setb{a^{n_1}_1 \cdots a^{n_r}_r \mid a_i \in S, \, n_i \in \z}.
    \]
\end{ej}

\begin{ej}
    Demostrar que
    \[
        \left<S\right> = \bigcap_{\substack{H \text{ sg. de } G \\ S \subseteq H}} H.
    \]
\end{ej}

\section{Orden de un elemento}

\begin{defi}[orden!de los elementos de un grupo]
    Sea $G$ un grupo y sea $x \in G$. Llamamos orden de $x$, si existe, al menor entero $n \geq 1$ tal que
    \[
        x^n = 1.
    \]
    Si no existe, decimos que $x$ tiene orden infinito.
\end{defi}

\begin{obs}
    Escribimos el orden de $x$ como $\orden(x)$ o $\ord(x)$.
\end{obs}

\begin{defi}[orden!de un grupo]
    Sea $G$ un grupo. Llamamos orden de $G$ a su cardinal y lo denotamos $\orden(G)$, $\ord(G)$, $\abs{G}$ o $\card(G)$.
\end{defi}

\begin{example}
    \begin{enumerate}[1.]
        \item[]
        \item $\ord(e) = 1$ y es el único elemento (el neutro) que tiene orden 1.
        \item En el grupo simétrico $G = \Sim_n$, sean $a_1, \dots, a_n \in G$, $\ord\left(\left( a_1, \dots, a_n \right)\right) = n$.
        \item En los grupos $\lp \z, +\rp , \lp \q, +\rp, \lp \real, +\rp$ y $\lp \cx, +\rp$, $\forall x \neq 0 \, \ord(x) = \infty$.
        \item En el grupo $\faktor{\z}{p\z}$ con $p$ primo, $\forall \bar{x} \neq \bar{0} \, \ord\left( \bar{x} \right) = p$.
        \item En los grupos $\q^{\ast}=\lp \q\setminus \lc 0\rc, \cdot \rp, \real^{\ast}=\lp \real\setminus \lc 0\rc, \cdot\rp$, $\ord(-1) = 2$, $\ord(1) = 1$ y $\forall x \notin \setb{-1, 1} \, \ord(x) = \infty$.
        \item En el grupo $\cx^{\ast}=\lp \cx\setminus \lc 0\rc, \cdot \rp$, $\forall n \geq 1$ $\ord\left( e^{\frac{2 \pi i}{n}} \right) = n$ y $\forall z \in \cx \tq \abs{z} \neq 1, \, \ord(z) = \infty$.
    \end{enumerate}
\end{example}

\begin{lema}
    Sea $G$ un grupo y $x \in G$ con $\ord(x) = n\in\n$. Entonces,
    \begin{enumerate}[i)]
        \item $x^m = 1 \iff n \vert m$.
        \item $x^m = x^{m^\prime} \iff m \equiv m^\prime \, \pmod{n}$.
        \item $\ord\left( x^m \right) = \frac{n}{\mcd(n, m)} = \frac{\ord(x)}{\mcd\left( \ord(x), m \right)}$.
    \end{enumerate}
\end{lema}

\begin{proof}
    \begin{enumerate}[i)]
        \item[]
        \item Si $n \vert m$, entonces $m = n \cdot d$, con lo cual, $x^m = \left( x^n \right)^d = 1^d = 1$. Recíprocamente, pongamos $m = nq + r$, con $0 \leq r < n$. Entonces,
            \[
                \begin{rcases}
                    1 = x^m = x^{nq + r} = \left( x^n \right)^q x^r = x^r \\
                    0 \leq r < n
                \end{rcases}
                \implies r = 0 \implies n \vert m.
            \]
        \item $x^m = x^{m^\prime} \iff x^{m - m^\prime} = 1 \iff n \vert m - m^\prime \iff
            m \equiv m^\prime \pmod{n}$.
        \item Sean $k = \ord\left( x^m \right)$ y $g = \mcd(n, m)$. Queremos ver que $k = n/g$.
            \[
                \left( x^m \right)^{\frac{n}{g}} = x^{\frac{mn}{g}} = \left( x^n \right)^{\frac{m}{g}} = 1
                \implies k \left\vert \frac{n}{g} \right. .
            \]
            Por otro lado,
            \[
                1 = \left( x^m \right)^k = x^{mk} \implies n \vert mk \implies \left.\frac{n}{g} \right\vert \frac{m}{g} k
                \substack{n/g \text{ y } m/g \\ \implies \\ \text{ primos entre si}} \left.\frac{n}{g} \right\vert k.
            \]
            Y sumando los dos resultados, tenemos que $\frac{n}{g} = k$.
    \end{enumerate}
\end{proof}

\begin{defi}[grupo!cíclico]
    Diremos que un grupo $G$ es cíclico si está generado por un solo elemento $x \in G$. Escribimos $G = \left< x \right>$, $G$
    generado por $x$ o $x$ generador de $G$.
\end{defi}

\begin{obs}
    Sea $G$ un grupo cíclico. Si $\ord(G) = n$ (con $n$ finito), entonces
    \[
        G = \setb{ 1 \left( = x^0 \right), x, x^2, \dots, x^{n-1}}.
    \]
    Lo denotaremos como $G = C_n$ (grupo cíclico de orden $n$).
    Si $\ord(G) = \infty$, entonces
    \[
        G = \setb{x^k \mid k \in \z}.
    \]
\end{obs}

\begin{example}
    \begin{enumerate}[1.]
        \item[]
        \item $\z = \left< 1 \right> = \left< -1 \right>$.
        \item $\faktor{\z}{n\z} = \left< \bar{k} \right>$ con $\mcd(n, k) = 1$.
    \end{enumerate}
\end{example}

\begin{defi}[función $\varphi$ de Euler]
    Sea $d \in \z, d \geq 1$, definimos la función $\varphi$ de Euler como
    \[
        \varphi(d) = \card \setb{ 1 \leq k \leq d \mid \mcd(k, d) = 1}.
    \]
\end{defi}

\begin{example}
    \begin{align*}
        \varphi(1) &= 1, & \varphi(5) &= 4, & \varphi(3) &= 2, & \varphi(7) &= 6, \\
        \varphi(2) &= 1, & \varphi(6) &= 2, & \varphi(4) &= 2, & \varphi(p) &= p-1 \text{ (con } p \text{ primo).} 
    \end{align*}
\end{example}

\begin{prop}\label{prop:euler}
    Sea $G = \left< x \right>$ un grupo cíclico, con $\ord(x) = n$, entonces
    \begin{enumerate}[i)]
        \item $\forall y \in G, \ord(y) \vert n$,
        \item $\forall d \vert n,$ existen $\varphi(d)$ elementos de $G = C_n$ de orden $d$. De hecho, son
            \[
                \setb{\left. x^{\frac{n}{d} k} \,\right\vert 1 \leq k \leq d \tq \mcd(k, d) = 1}.
            \]
    \end{enumerate}
\end{prop}

\begin{proof}
    \begin{enumerate}[i)]
        \item[]
        \item Por ser $y$ un elemento de $G$, es de la forma $y=x^m,\,0\leq m\leq n$. Entonces,
                \[
                    \ord(y) = \ord\left( x^m \right) = \frac{\ord(x)}{\mcd\left(\ord(x), m \right)} = \frac{n}{\mcd(n, m)}.
                \]
            Y concluimos $\ord(y) \vert n$.
        \item Sea $y \in G \tq \ord(y) = d$, tenemos que
                \[
                    y = x^m \implies \ord(y) = d = \frac{n}{\mcd(n, m)} \iff \mcd(n, m) = \frac{n}{d}.
                \]
                Buscamos los $m$ tales que $\mcd(n, m) = \frac{n}{d}$.
                \[
                    \mcd(n, m) = \frac{n}{d} \iff \mcd\left( \frac{n}{d}d, \frac{n}{d}k \right) = \frac{n}{d}
                    \iff \mcd(d, k) = 1.
                \]
                Con esta última condición, tenemos que
                \[
                    \varphi(d) = \card \setb{ k \in \z \mid \mcd(k, d) = 1 \text{ y } 1 \leq k \leq d}
                    = \card \setb{ x^m \mid \ord\left( x^m \right) = d}.
                \]
    \end{enumerate}
\end{proof}

\begin{col}
    Se tiene que
    \[
        n = \sum_{d \vert n} \varphi(d).
    \]
\end{col}

\begin{proof}
    Tomamos $G = C_n = \left< x \right>$. Entonces,
    \[
        n = \abs{G} = \sum_{d \vert n} \card \setb{ x \in G \mid \ord(x) =d } = \sum_{d \vert n} \varphi(d),
    \]
    ya que
    \[
        G = \bigcup_{d \vert n} \setb{x \in G \mid \ord(x) = d}.
    \]
\end{proof}

\begin{prop}
    Sea $G = C_n = \left< x \right>$ (con $n$ finito). Entonces
    \begin{enumerate}[i)]
        \item Si $d \mid n$, entonces $x^{\frac{n}{d}}$ es un elemento de orden $d$
            y el subgrupo $H_d := \left< x^{\frac{n}{d}} \right>$ es subgrupo cíclico de orden $d$.
        \item Si $H$ es un subgrupo de $G$, entonces $\exists! d \vert n$ tal que $H = H_d$.
    \end{enumerate}
\end{prop}

\begin{proof}
    \begin{enumerate}[i)]
        \item[]
        \item Se tiene que
            \[
                \orden\left( x^{\frac{n}{d}} \right) = \frac{n}{\mcd(n, \frac{n}{d})} = d.
            \]
            Como $x^{\frac{n}{d}}$ tiene orden $d$, $H_d = \left< x^{\frac{n}{d}} \right>$ es un grupo
            cíclico de orden $d$.
        \item Sea $H$ un subgrupo de $G = C_n$ y sea $1 \leq t \leq n$ el menor exponente tal que 
            $x^t \in H$. Veremos que $t \vert n$. Expresamos $n = tk + r$ (con $0 \leq r < t$).
            \[
                1 = x^n = x^{tk + r} = \left( x^t \right)^k x^r.
            \]
            Como $x^t\in H$, se tiene que $\lp x^t\rp^k, \lp \lp x^t\rp^k \rp ^{-1} \in H$. Así,
            \[
                x^r=\lp \lp x^t\rp^k \rp ^{-1} \lp x^t\rp^k x^r = \lp \lp x^t\rp^k \rp ^{-1} \in H.
            \]
            Pero $t$ es el exponente más pequeño (a excepción del 0) tal que $x^t\in H$ y, en consecuencia, como $r<t$, $r = 0$ y $n = tk$.
            Veremos ahora que $H = H_k$.
            
            Claramente, $\left< x^{\frac{n}{k}} \right> = H_k \subseteq H$, ya que $x^{\frac{n}{k}} = x^t \in H$
            y, por lo tanto, todos sus múltiplos están en $H$.

            Sea $y = x^m \in H$. Necesariamente, $m\geq t$. Escribimos $m = tq + s$
            (con $0 \leq s < t$). Entonces,
            \[
                x^m = x^{tq + s} = \left( x^t \right)^q x^s \implies
                x^s = \underbrace{\left( \left( x^t \right)^q \right)^{-1}}_{\in H} \cdot
                \underbrace{x^m}_{\in H} \in H.
            \]
            de nuevo, por la definición de $t$, $s = 0$ y concluimos que $y = \left( x^t \right)^q \in 
            \left< x^{\frac{n}{k}} \right> = H_k$, es decir, $H \subseteq H_k$.

            Solo resta ver que $k$ es único, pero es obvio ya que, si $H = H_k = H_e$, entonces,
            \[
                \frac{n}{k} = \orden\left( H_k \right) = \orden\left( H \right)  = \orden\left( H_e \right)  = \frac{n}{e}\implies
                k = e.
            \]
    \end{enumerate}
\end{proof}

\begin{col}[Retículo de subgrupo de un grupo cíclico]
    Sea $G = C_n = \left< x^n \right>$ un grupo cíclico de orden $n \geq 1$. Existe una biyección
    \[
        \begin{aligned}
            \setb{d \in \n \mid 1 \leq d \leq n, d \vert n } &\longleftrightarrow
            \setb{\text{subgrupos de }G} \\
            d &\longleftrightarrow H_d.
        \end{aligned}
    \]
\end{col}

\section{Morfismos de grupos}

\begin{defi}[homeomorfismo de grupos]\index{morfismo de grupos((tlab))}
    Sean $G_1$, $G_2$ dos grupos y sea $f \colon G_1 \to G_2$ una aplicación. Decimos que $f$ es un
    (homeo)morfismo de grupos si
    \[
        f(xy) = f(x)f(y).
    \]
\end{defi}

\begin{prop}
    Si $f$ es un morfismo de grupos, entonces
    \begin{enumerate}[i)]
        \item \label{item:morf1} $f(1) = 1$,
        \item $f\left( x^{-1} \right) = \left( f(x) \right)^{-1}$.
    \end{enumerate}
\end{prop}

\begin{proof}
    \begin{enumerate}[i)]
        \item[]
        \item Sea $x \in G_1$, $f \lp x \rp = f\lp x\cdot 1\rp = f\left(x \right)f\lp 1 \rp \implies f\lp 1 \rp = 1$.
        \item Sea $x \in G_1$, $f \lp xx^{-1} \rp = f\lp 1\rp \stackrel{\ref{item:morf1}}{=} 1 = f\left(x \right)f\lp x^{-1} \rp \implies f\lp x^{-1} \rp = f\lp x \rp^{-1}$.
    \end{enumerate}
\end{proof}

\begin{obs}
    Notación:
    \begin{center}
        \begin{tabular}{|c|c|}
            \hline
            Nombre & Propiedades \\
            \hline\hline
            Monomorfismo & Inyectiva \\\hline
            Epimorfismo  & Exhaustiva \\\hline
            Isomorfismo  & Biyectiva \\\hline
            Endomorfismo & $G_1 = G_2$ \\\hline
            Automorfismo & Biyectiva y $G_1 = G_2$ \\
            \hline
        \end{tabular}
    \end{center}
\end{obs}

\begin{prop}
    Sea $f \colon G_1 \to G_2$ un morfismo biyectivo (isomorfismo). Entonces
    $f^{-1} \colon G_2 \to G_1$ es un morfismo de grupos.
\end{prop}

\begin{proof}
    Como $f$ es biyectiva, en particular es exhaustiva y inyectiva y tenemos que $\forall x' \in G_2,\, \exists! x \in G_1 \tq f\lp x \rp = x'$. Sean $f\lp x\rp, f\lp y\rp \in G_2$ dos elementos cualesquiera de $G_2$,
    \[
        f^{-1}\lp f\lp x\rp f\lp y\rp \rp = f^{-1} \lp f\lp xy\rp \rp = xy = f^{-1} \lp f\lp x\rp \rp f^{-1}\lp f\lp y\rp \rp.
    \]
\end{proof}

\begin{defi}[grupo!isomorfo]
    Sean $G_1$ y $G_2$ grupos. Decimos que $G_1$ y $G_2$ son isomorfos si $\exists f \colon G_1 \to G_2$
    isomorfismo. Lo notaremos como $G_1\cong G_2$.
\end{defi}

\begin{prop}
    Sea $f \colon G_1 \to G_2$ un morfismo de grupos.
    \begin{enumerate}[i)]
        \item Si $H$ es un subgrupo de $G_1$, entonces $f(H)$ es subgrupo de $G_2$.
        \item Si $K$ es un subgrupo de $G_2$, entonces, $f^{-1}(K)$ es subgrupo de $G_1$.
    \end{enumerate}
\end{prop}

\begin{proof}
    \begin{enumerate}[i)]
        \item[]
        \item Veamos que se cumplen las tres propiedades de la definición \ref{defi:subgrupo}.
        \begin{enumerate}[i)]
            \item $1 \in H \implies f(1) = 1 \in f(H) \implies f(H) \neq \emptyset$.
            \item Sean $a, b \in H$, entonces $ab \in H$ y $f(a) f(b) = f(ab) \in f(H)$.
            \item Sea $a \in H$, entonces $a^{-1} \in H$ y $f(a)^{-1} = f\lp a\rp ^{-1} \in H$.
        \end{enumerate}
        \item Veamos que se cumplen las tres propiedades de la definición \ref{defi:subgrupo}.
        \begin{enumerate}[i)]
            \item $1 \in K, f(1) = 1 \implies 1 \in f^{-1}(K) \neq \emptyset$.
            \item Sean $a, b \in f^{-1}\lp K\rp$, entonces $\exists a', b' \in K \tq f\lp a\rp = a', \, f\lp b\rp = b'$ y tenemos que $ab = f^{-1}\lp f\lp ab\rp \rp = f^{-1}\lp f\lp a \rp f\lp b\rp\rp = f^{-1}\lp a'b'\rp \in f^{-1}\lp K\rp$, ya que $a'b' \in K$.
            \item Sea $a \in f^{-1}\lp K\rp$, entonces $\exists a' \in K \tq f\lp a \rp = a'$, y tenemos que $a^{-1} = f^{-1}\lp f\lp a^{-1}\rp \rp = f^{-1}\lp f\lp a \rp ^{-1} \rp = f^{-1}\lp \lp a'\rp^{-1}\rp \in f^{-1}\lp K\rp$, ya que $\lp a'\rp^{-1} \in K$.
        \end{enumerate}
    \end{enumerate}
\end{proof}

\begin{obs}
    \begin{itemize}
        \item[]
        \item $f\left( G_1 \right) = \im(f)$.
        \item $f^{-1}(1) = \ker(f)$.
    \end{itemize}
\end{obs}

\begin{prop} Sean $G_1, G_2$ grupos y sea $f\colon G_1\to G_2$ un morfismo de grupos. Entonces,
    \begin{enumerate}[i)]
        \item $f$ inyectiva $\iff \ker(f) = \setb{1}$.
        \item $f$ exhaustiva $\iff \im(f) = G_2$.
    \end{enumerate}
\end{prop}

\begin{proof}
    Ejercicio.
\end{proof}

\section{Clases laterales}

\begin{defi}[elemento relacionado!por la izquierda]
    Sea $G$ un grupo y $H \subseteq G$ un subgrupo. Dados $a, b \in G$, decimos que
    $a$ está relacionado con $b$ por la izquierda si
    \[
        a^{-1}b \in H.
    \]
\end{defi}

\begin{ej}
    Demostrar que que la relación definida es una relación de equivalencia. Para ello, hace falta ver que es
    reflexivo, simétrico y transitivo.
\end{ej}

\begin{defi}[clase lateral!por la izquierda]
    Con la relación que hemos visto ahora, denotamos la clase de equivalencia de $a \in G$ como
    \[
        \begin{aligned}
            \bar{a} &= \setb{b \in G \mid a^{-1}b = x, \, x \in H} \\ &=
            \setb{b \in G \mid b = ax, \, x \in H} \\&= aH.
        \end{aligned}
    \]
    y llamaremos a $aH$ clase lateral por la izquierda del elemento $a$ módulo el subgrupo $H$.
\end{defi}

\begin{example}
    Si $a = 1$, tenemos que
    \[
        1\cdot H = \setb{1x \mid x \in H} = H.
    \]
\end{example}

\begin{obs}
    Tomamos
    \[
        \begin{aligned}
            f_a \colon G &\to G \\
            x &\mapsto f_a(x) = ax
        \end{aligned}
    \]
    una aplicación biyectiva ($f^{-1}_a = f_{a^{-1}}$). Notemos, que $f$ no es un morfismo
    de grupos (en general), ya que $f(1) = a$ (en general $a \neq 1$). Se tiene también que
    \[
        f_a(H) = \setb{f_a(x) \mid x \in H} = \setb{ax \mid x \in H} = aH.
    \]
    Diremos pues que hay una biyección $H \leftrightarrow aH$, en particular, si $G$ es finito, se tiene que
    $\abs{H} = \abs{aH}$.
\end{obs}

\begin{defi}[conjunto cociente de un grupo]
    Sea $G$ un grupo y sea $H$ un subgrupo de $G$. Llamamos conjunto cociente de $G$ módulo $H$ a 
    \[
        \faktor{G}{H} = \setb{aH \mid a \in G} = \setb{ \bar{a} \mid a \in G},
    \]
    es decir, el conjunto de las clases laterales por la izquierda de $G$ módulo $H$.
    
\end{defi}

\begin{teo}[de Lagrange]
    Sea $G$ un grupo finito y sea $H$ un subgrupo de $G$. Entonces,
    \[
        \abs{G} = \abs{H} \abs{\faktor{G}{H}}.
    \]
\end{teo}

\begin{proof}
    Se tiene que
    \[
        G = \bigsqcup \bar{a}H.
    \]
    Por lo tanto,
    \[
        \abs{G} = \abs{\bigsqcup \bar{a}H} = \sum \abs{\bar{a}H} = \sum^{\abs{\faktor{G}{H}}}_{k = 1} \abs{H} = \abs{H} \abs{\faktor{G}{H}}.
    \]
\end{proof}

\begin{col}
    Si $G$ es finito y $H$ es subgrupo, entonces $\abs{H}$ divide a $\abs{G}$. Si $x \in G$,
    \[
        \orden(x) \vert \orden(G) = \abs{G}.
    \]
\end{col}

\begin{proof}
    Ya que $\orden(x) = \orden\left( \left< x \right> \right)$ y $\orden(H) \vert \orden(G)$.
\end{proof}

\begin{defi}[elemento relacionado!por la derecha]
    Sea $G$ un grupo y sea $H \subseteq G$ un subgrupo, decimos que $a, b \in G$ están relacionados
    por la derecha si
    \[
        ab^{-1} \in H.
    \]
\end{defi}

\begin{ej}
    Demostrar que se trata de una relación de equivalencia (propiedades simétrica, transitiva y reflexiva).
\end{ej}

\begin{defi}[clase lateral!por la derecha]
    Tenemos que
    \[
        \bar{a} = \setb{b \in G \mid ab^{-1} = x, \, y \in H} = \setb{b \in G \mid b = ya, \, y \in H} = Ha.
    \]
    Llamamos clase lateral por la derecha de $a$ módulo $H$ a $Ha$.
\end{defi}

\begin{obs}
    Sea
    \[
        \begin{aligned}
            g_a \colon G &\to G \\
            x &\mapsto g_a(x) = xa.
        \end{aligned}
    \]
    Notamos que es una aplicación biyectiva ($g^{-1}_a = g_{a^{-1}}$), pero que no es un morfismo de grupos.
    Además, $g_a(H) = Ha$ y, por lo tanto, se tiene una biyección $H \leftrightarrow Ha$. En particular, si
    $G$ es finito, $\abs{H} = \abs{Ha}$.
\end{obs}

\begin{prop}
    Existe una biyección
    \[
        \begin{aligned}
            \setb{aH \mid a \in G} &\to \setb{Hb \mid b \in G} \\
            xH &\mapsto Hx^{-1}.
        \end{aligned}
    \]
\end{prop}

\begin{proof}
    Ejercicio: demostrar que está bien definida y que es biyectiva.
\end{proof}

\begin{defi}[índice de un grupo en un subgrupo]
    Sea $G$ un grupo y $H \subseteq G$ un subgrupo. Llamamos índice de $G$ en $H$ al cardinal de $\faktor{G}{H}$.
    Y lo denotamos como
    \[
        \left[G : H \right] = \abs{\faktor{G}{H}} \stackrel{\text{TL}}{=} \frac{\abs{G}}{\abs{H}}.
    \]
\end{defi}

\section{Subgrupos normales. Grupo cociente}

\begin{defi}[subgrupo!normal a un grupo]
    Sea $G$ un grupo y $H \subseteq G$ un subgrupo. Decimos que $H$ es un subgrupo normal de $G$ si
    $\forall a \in G$
    \[
        aH = Ha,
    \]
    y lo denotaremos como $H \triangleleft G$.
\end{defi}

\begin{obs}
    $H \triangleleft G$ no quiere decir que $ax = xa$ ($x \in H, \, a \in G$). Quiere decir que
    $\forall x \in H, \, a \in G, \, \exists y \in H$ tal que $ax = ya$.
\end{obs}

\begin{prop}
    Sea $G$ un grupo y $H \subseteq G$ un subgrupo, entonces son equivalentes
    \begin{enumerate}[(i)]
        \item\label{item:eq_norm_1} $H \triangleleft G$,
        \item\label{item:eq_norm_2} $aH = Ha, \, \forall a \in G$,
        \item\label{item:eq_norm_3} $aH \subseteq Ha, \, \forall a \in G$,
        \item\label{item:eq_norm_4} $aHa^{-1} = H, \, \forall a \in G$,
        \item\label{item:eq_norm_5} $aHa^{-1} \subseteq H, \, \forall a \in G$.
    \end{enumerate}
\end{prop}

\begin{proof}
    En primer lugar, \ref{item:eq_norm_1}$\iff$\ref{item:eq_norm_2} por definición y \ref{item:eq_norm_2}$\implies$\ref{item:eq_norm_3} es inmediato.
    
    \noindent Veamos que \ref{item:eq_norm_3}$\implies$\ref{item:eq_norm_5}. Sea $x=aba^{-1}\in aHa^{-1}$, de modo que $b\in H$. Por \ref{item:eq_norm_3}, sabemos que $ab=ca$, con $c\in H$; entonces, $x=aba^{-1}=caa^{-1}=c\in H$.
    
    \noindent Veamos que \ref{item:eq_norm_5}$\implies$\ref{item:eq_norm_4}. Basta provar que $\abs{aHa^{-1}}=\abs{H}$. Tomemos $b, c\in H$ tales que $aba^{-1}=aca^{-1}$. Se tiene que $b=c$ y sigue que $\abs{aHa^{-1}}=\abs{H}$.
    
    \noindent Veamos, por último, que \ref{item:eq_norm_4}$\implies$\ref{item:eq_norm_2}. Sea $x=ab=aba^{-1}a\in aH$. Por \ref{item:eq_norm_4}, $aba^{-1}=c\in H$, de modo que $x=ca\in Ha$ y concluimos que $aH\subseteq Ha$. Análogamente, tenemos que $Ha\subseteq aH$.
\end{proof}

\begin{example}
    \begin{enumerate}[1.]
        \item[]
        \item Tomamos $G = \Sim_3 = \setb{\Id, (12), (13), (23), (123), (132)}$ y
            $H = A_3 = \setb{\Id, (123), (132)}$. Sabemos ahora que
            \[
                \abs{G} = \abs{H}\abs{\faktor{G}{H}} \implies
                \frac{\abs{G}}{\abs{H}} = \frac{6}{3} = 2 = \abs{\faktor{G}{H}}.
            \]
            Como $\forall x \in H, \, xH = H = Hx$, $H$ es un grupo normal, ya que solo existen 2 clases.
        \item Tomamos $G = D_{2 \cdot 4} = \setb{\Id, r, r^2, r^3, s, rs, r^2s, r^3s}$ y
            $H = \setb{ \Id, r, r^2, r^3}$.
    \end{enumerate}
\end{example}

\begin{prop}
    Sea $G$ un grupo finito y $H \subseteq G$ un subgrupo,
    \[
        [G:H] = 2 \implies H \triangleleft G
    \]
\end{prop}
\begin{proof}
    Por ser $H$ un grupo, se tiene que $aH=H=Ha, \,\forall a\in H$. Además, como solamente hay dos clases laterales, se tiene que $aH=G\setminus H=Ha,\,\forall a\in G\setminus H$.
\end{proof}

\begin{example}
    Tomamos $G = \Sim_3$ y $H = \setb{ \Id, (1,2)}$. Tenemos que
    \[
        \begin{aligned}
            (1,3)H &= (1,3) \setb{\Id, (1,2)} = \setb{(1,3), (1,2,3)}, \\
            H(1,3) &= \setb{\Id, (1,2)} (1,3) = \setb{(1,3), (1,3,2)}.
        \end{aligned}
    \]
\end{example}

\begin{lema}
    Sean $G_1, G_2$ grupos, sean $H, K$ subgrupos de $G_1$ y $G_2$ respectivamente y sea $f \colon G_1 \to G_2$ un morfismo de grupos, entonces
    \begin{enumerate}[i)]
        \item $H \triangleleft G_1 \implies f(H) \triangleleft f\left(G_1 \right).$
        \item $K \triangleleft G_2 \implies f^{-1}(K) \triangleleft G_1.$
    \end{enumerate}
\end{lema}

\begin{proof}
    \begin{enumerate}[i)]
        \item[]
        \item Sea $f(x) \in f(H)$ y sea $f(a) \in f\left( G_1 \right)$. Entonces,
            \[
                f(a) f(x) f(a)^{-1} = f(a) f(x) f\left( a^{-1} \right) =
                f\left( axa^{-1} \right) \in f(H),
            \]
            puesto que $axa^{-1} \in H$.
        \item Sea $x\in f^{-1}\lp K\rp$ y sea $a\in G_1$. Entonces,
            \[
                axa^{-1} \in f^{-1}\lp f\lp axa^{-1} \rp \rp = f^{-1}\lp f\lp a\rp f\lp x\rp f\lp a\rp ^{-1} \rp \subseteq f^{-1}\lp K\rp,
            \]
            puesto que $f\lp x\rp\in K$, $f\lp a\rp, f^{-1}\lp a\rp \in G_2$ y $K \triangleleft G_2$.
    \end{enumerate}
\end{proof}

\begin{obs}
    Si $G$ es un grupo conmutativo, entonces todo subgrupo es normal.
\end{obs}

\begin{obs}
    Sea $G$ un grupo. $G$ y $\setb{\Id}$ son subgrupos normales.
\end{obs}

\begin{prop}
    Sea $G$ un grupo y sean $H \subseteq K \subseteq G$ subgrupos.
    \[
        H \triangleleft G \implies H \triangleleft K.
    \]
\end{prop}

\begin{proof}
    Para todo $a \in K$ y $x \in H$ se tiene que $axa^{-1} \in H$ y, por lo tanto, $H \triangleleft K$.
\end{proof}

\begin{obs}
    \[
        H \triangleleft K \triangleleft G \notimplies H \triangleleft G.
    \]
\end{obs}

\begin{example}
    Sean
    \begin{align*}
        G &= \Sim_4,\\
        H &= \setb{ \Id, (1,2)(3,4)}, \\
        K &= \setb{ \Id, (1,2)(3,4), (1,3)(2,4), (2,3)(1,4)}.
    \end{align*}
    Tenemos que $[K : H] = 2$, lo cual implica que $H \triangleleft K$. También es cierto que $K \triangleleft G$:
    \[
        \sigma(1,2)(3,4)\sigma^{-1} = \sigma(1,2)\sigma^{-1}\sigma(3,4)\sigma^{-1} = (\sigma\lp 1\rp \sigma\lp 2\rp)
        (\sigma\lp 3\rp \sigma\lp 4\rp) \in K
    \]
    y análogamente para el resto de permutaciones de $K$. Sin embargo,
    $H \centernot\triangleleft G$:
    \[
        \begin{aligned}
            (1,2,3)H &= \setb{(1,2,3), (1,3,4)}, \\
            H(1,2,3) &= \setb{(1,2,3), (2,4,3)}.
        \end{aligned}
    \]
\end{example}

\begin{prop}\label{prop:normalbiyec}
    Sea $G$ un grupo, y sea $H \triangleleft G$ un subgrupo normal, entonces
    \begin{enumerate}[i)]
        \item En $\faktor{G}{H}$ existe una estructura de grupo definida por 
            \[
                (xH)(yH) = (xy)H.
            \]
        \item La función
            \[
                \begin{aligned}
                    \Pi \colon G &\to  \faktor{G}{H} \\
                    x &\mapsto xH = \bar{x}
                \end{aligned}
            \]
            es un morfismo de grupos exhaustivo y de núcleo $H$.
        \item Existe una biyección entre
            \[
                \begin{aligned}
                    \setb{ \text{sg. (normales) de $G$ que contienen a $H$}} &\leftrightarrow
                    \setb{ \text{sg. (normales) de $\faktor{G}{H}$}} \\
                    K \supseteq H &\mapsto \Pi(K) \\
                    \Pi^{-1}(L) &\mapsfrom L
                \end{aligned}
            \]
    \end{enumerate}
\end{prop}

\begin{proof}% TODO revisar demo
    \begin{enumerate}[i)]
        \item[]
        \item Sean $x\sim x'$ y $y\sim y'$, queremos ver que $xy\sim x'y'$.
            \[
                \begin{rcases}
                    x\sim x' \implies x^{-1}x' \in H\\
                    y\sim y' \implies y^{-1}y' \in H
                \end{rcases}
                \implies \lp xy\rp ^{-1}\lp x'y'\rp = y^{-1} \underbrace{\lp x^{-1}x'\rp}_{\in H} y'\stackrel{H\triangleleft G}{=} \underbrace{y^{-1}y'}_{\in H}t \in H.
            \]
            Comprovamos las propiedades de la operación de un grupo:
            \begin{itemize}
                \item \emph{Associativa}: $\lp xH\rp\lp yHzH\rp = \lp xH\rp \lp\lp yz\rp H\rp = \lp x\lp yz\rp \rp H = \lp \lp xy\rp z\rp H = \lp xyH\rp \lp zH\rp = \lp \lp xH\rp \lp yH\rp \rp \lp zH\rp$.
                \item \emph{Elemento neutro}:
                    \begin{align*}
                        H\lp xH\rp &= \lp 1H\rp \lp xH\rp = \lp 1x\rp H = xH, \\
                        \lp xH\rp H &= \lp xH\rp \lp 1H\rp = \lp x1\rp H = xH.
                    \end{align*}
                \item \emph{Elemento inverso}: $\lp xH\rp\lp x^{-1}H\rp = \lp xx^{-1}\rp H = 1H = H$.
            \end{itemize}
        \item Sean $x, y\in G$. Entonces, $\Pi\lp xy\rp = \lp xy\rp H=\lp xH\rp\lp yH\rp = \Pi\lp x\rp\Pi\lp y\rp$, con lo que $\Pi$ es un morfismo de grupos. Ver que es un morfismo exhaustivo es inmediato, puesto que, dada una clase, cualquiera de sus respresentates la tiene por imagen. Su núcleo es $\Pi^{-1}\lp H\rp$, es decir los elementos que tienen $H$ por imagen. Sabemos que $xH=H,\,\forall x\in H$ y que $y\notin H\implies y1\notin H\implies \Pi\lp y\rp=yH\neq H$, de modo que $\nuc\lp\Pi\rp=H$.
        \item Demostraremos primero que $\Pi$ define un morfismo biyectivo entre los subgrupos de $G$ que contienen $H$ y los subgrupos de $\faktor{G}{H}$. Sea $K\supseteq H$ un subgrupo de $G$. Veamos que $\Pi\lp K\rp$ es un subgrupo de $\faktor{G}{H}$.
            \begin{itemize}
                \item $H=\Pi\lp H\rp \subseteq\Pi\lp K\rp$, así que $\Pi\lp K\rp$ tiene elemento neutro.
                \item Sea $a\in\Pi\lp K\rp$. Entonces, $a=xH$, para algún $x\in K$, de modo que $x^{-1}\in K$ y concluimos que $x^{-1}H\in\Pi\lp K\rp$, es decir, $a$ tiene elemento inverso en $\Pi\lp K\rp$.
                \item Sean $a,b\in\Pi\lp K\rp$. Entonces, $a=xH$ y $b=yH$, para algún par $x,y\in K$, de modo que $xy\in K$ y concluimos que $\lp xy\rp H\in\Pi\lp K\rp$, es decir, $\Pi\lp K\rp$ es cerrado por la operación.
            \end{itemize}
            Sea $L$ un subgrupo de $\faktor{G}{H}$. Veamos que $\Pi^{-1}\lp L\rp$ es un subgrupo de $G$ que contiene $H$.
            \begin{itemize}
                \item $1\in H = \Pi^{-1}\lp H\rp \subseteq \Pi^{-1}\lp L\rp$, es decir, $\Pi^{-1}\lp L\rp$ contiene a $H$ y, por ende, tiene elemento neutro. 
                \item $x\in\Pi^{-1}\lp L\rp \implies \Pi\lp x\rp=xH\in L\implies x^{-1}H\in L\implies x^{-1}\in\Pi^{-1}\lp L\rp$, es decir, todo elemento de $\Pi^{-1}\lp L\rp$ tiene inverso en $\Pi^{-1}\lp L\rp$.
                \item $x,y\in\Pi^{-1}\lp L\rp \implies \Pi\lp x\rp = xH,\, \Pi\lp y\rp=yH\in L\implies \lp xH\rp \lp yH\rp = \lp xy\rp H\in L \implies xy\in \Pi^{-1}\lp L\rp$, es decir, $\Pi^{-1}\lp L\rp$ es cerrado por la operación.
            \end{itemize}
            Para acabar, basta ver que $\Pi$ restringida a los subgrupos de $G$ que contienen $H$ es inyectiva. Sean $K_1, K_2$ subgrupos de $G$ que contienen $H$ tales que $\Pi\lp K_1\rp = \Pi\lp K_2\rp$. Supongamos que $K_1\neq K_2$. Sin pérdida de generalidad, $\exists x\in K_1$ tal que $x\notin K_2$. Puesto que $xH\in\Pi\lp K_1\rp=\Pi\lp K_2\rp$, necesariamente $\exists y\in K_2$ tal que $yH=xH$. Entonces, $x\in xH=yH\implies y^{-1}x\in H\subseteq K_2\implies yy^{-1}x=x\in K_2$ e incurrimos en una contradicción. Por consiguiente, $K_1 = K_2$ y sigue que $\Pi$ restringida a los subgrupos de $G$ que contienen $H$ es inyectiva.
            
            \noindent Para demostrar que $\Pi$ define un morfismo biyectivo entre los subgrupos normales de $G$ que contienen $H$ y los subgrupos normales de $\faktor{G}{H}$, es suficiente comprobar que la imagen y la antiimagen de subgrupos normales (que, como ya hemos visto, son subgrupos) son normales. Sea $K\supseteq H$ un subgrupo normal de $G$. Veamos que $\Pi\lp K\rp\triangleleft \faktor{G}{H}$. Sea $xH\in\faktor{G}{H}$. Entonces, se tiene que 
            \begin{align*}
                \lp xH\rp \Pi\lp K\rp &= \bigcup_{yH\in\Pi\lp K\rp}\lc\lp xH\rp\lp yH\rp\rc = \bigcup_{y\in K}\lc\lp xH\rp\lp yH\rp\rc =\\
                &= \bigcup_{y\in K}\lc\lp xy\rp H\rc = \bigcup_{y\in K}\bigcup_{z\in H}\lc xyz\rc = \bigcup_{z\in H}\bigcup_{y\in K}\lc xyz\rc =\\ 
                &= \bigcup_{z\in H}\lc \lp xK\rp z\rc = \bigcup_{z\in H}\lc \lp Kx\rp z\rc = \bigcup_{z\in H}\bigcup_{y\in K}\lc yxz\rc =\\ 
                &= \bigcup_{y\in K}\bigcup_{z\in H}\lc yxz\rc = \bigcup_{y\in K}\lc\lp yx\rp H\rc = \bigcup_{y\in K}\lc\lp yH\rp\lp xH\rp\rc =\\
                &= \bigcup_{yH\in \Pi\lp K\rp}\lc\lp yH\rp\lp xH\rp\rc = \Pi\lp K\rp\lp xH\rp,
            \end{align*}    
	    de modo que $\Pi\lp K\rp$ es un subgrupo normal de $\faktor{G}{H}$. Por otro lado, sea $L$ un subgrupo normal de $\faktor{G}{H}$. Veamos que $\Pi^{-1}\lp L\rp\triangleleft G$. Sea $x\in G$. Se tiene que $x\Pi^{-1}\lp L\rp x^{-1}$ es un grupo, de lo que sigue
            \begin{align*}
                x\Pi^{-1}\lp L\rp x^{-1} &= \Pi^{-1}\lp \Pi\lp x\Pi^{-1}\lp L\rp x^{-1}\rp\rp =\\
                &=\Pi^{-1}\lp \lp xH\rp L \lp x^{-1}H\rp\rp =\\
                &=\Pi^{-1}\lp \lp xH\rp \lp x^{-1}H\rp L\rp =\\
                &=\Pi^{-1}\lp HL\rp =\\
                &=\Pi^{-1}\lp L\rp.
            \end{align*}
            Así pues, $\Pi^{-1}\lp L\rp$ es un subgrupo normal de $G$ y hemos terminado.
    \end{enumerate}
\end{proof}

\begin{teo}[de isomorfía (primero)] \label{teo:1_iso}
    Sean $G_1, G_2$ grupos, sea $f \colon G_1 \to G_2$ un morfismo de grupos. Sea $H \triangleleft G_1$
    un subgrupo normal. Definimos
    \begin{center}
        \begin{tabular}{cc}
            $
            \begin{aligned}
                \tilde{f} \colon \faktor{G_1}{H} &\to G_2 \\
                xH &\mapsto \tilde{f}(xH) = f(x).
            \end{aligned}
            $
            \qquad \qquad & \qquad \qquad
            \tikzexternaldisable
            $
                \begin{tikzcd}
                    G_1\ar{r}{f}\ar{d} & G_2\\[2ex]
                    \faktor{G_1}{H} \arrow[ur, swap, dashed, "\tilde{f}\lp xH\rp = f\lp x\rp"] & 
                \end{tikzcd}
            $
            \tikzexternalenable
        \end{tabular}
    \end{center}
    Entonces,
    \begin{enumerate}[i)]
        \item $\tilde{f}$ está bien definida $\iff H \subseteq \ker(f)$.
    \end{enumerate}
     \noindent Si $\tilde{f}$ está bien definida, se cumple que 
    \begin{enumerate}[i)]
        \setcounter{enumi}{1}
        \item $\tilde{f}$ es un morfismo de grupos,
        \item $xH \in \ker\left( \tilde{f} \right) \iff x \in \ker(f)$,
        \item $\im\left( \tilde{f} \right) = \im(f)$.
    \end{enumerate}
\end{teo}
\begin{proof}
    \begin{enumerate}[i)]
        \item[]
        \item 
            \begin{align*}
                \tilde{f}\text{ está bien definida } &\iff \lp xH=yH\implies f\lp x\rp = f\lp y\rp \rp \iff \\
                &\iff \lp x^{-1}y\in H \implies f\lp x\rp = f\lp y\rp \rp \iff \\
                &\iff \lp x^{-1}y\in H \implies f\lp x\rp f^{-1}\lp y\rp = 1 \rp \iff \\
                &\iff \lp x^{-1}y\in H \implies f\lp x\rp f\lp y^{-1}\rp = 1 \rp \iff \\
                &\iff \lp x^{-1}y\in H \implies f\lp x y^{-1}\rp = 1 \rp \iff \\
                &\iff H\subseteq \ker\lp f\rp.
            \end{align*}
        \item $\tilde{f}\lp xH\rp\tilde{f}\lp yH\rp=f\lp x\rp f\lp y\rp = f\lp xy\rp = \tilde{f}\lp \lp xy\rp H\rp$.
        \item $xH\in\ker\lp\tilde{f}\rp\iff\tilde{f}\lp xH\rp = f\lp x\rp = 1\iff x\in\ker\lp f\rp$.
        \item $\im\lp\tilde{f}\rp=\lc \tilde{f}\lp xH\rp \mid x\in G_1\rc =\lc f\lp x\rp \mid x\in G_1\rc = \im\lp f\rp$.
    \end{enumerate}

\end{proof}

\begin{col}\label{col:1_teo_iso}
    En particular $\tilde{f} \colon \faktor{G_1}{\ker(f)} \to f\left( G_1 \right)$ es un morfismo de
    grupos biyectivo (isomorfismo).
\end{col}

\begin{col}
    Hay un único grupo cíclico de orden $n$ (salvo isomorfismos).
\end{col}

\begin{proof}
    Sea $G = C_n(x) = \setb{ 1, x, \dots, x^{n-1}} = \langle x \rangle$, tomamos
    \[
        \begin{aligned}
            f \colon \z &\to C_n(x) \\
            k &\mapsto x^k
        \end{aligned}
    \]
    que es un morfismo de grupos exhaustivo, $\ker(f) = n\z$. Por el primer teorema de isomorfía (\ref{teo:1_iso}),
    \[
        \faktor{\z}{\z_{n}} \cong C_n.
    \]
\end{proof}

\section{El grupo multiplicativo de un cuerpo finito}

\begin{obs}[Notación]
    Sea $G$ un grupo finito con $\orden(G) = n$ y $d \vert n$, notaremos
    \[
        \O_d = \setb{y \in G \mid \orden(y) = d}.
    \]
    Sea $x \in G$ con $\orden(x) = m$, notaremos
    \[
        C_m(x) = \langle x \rangle = \setb{1, x, \dots, x^{m-1}}
    \]
\end{obs}

\begin{obs}[Notación]
    Dado un cuerpo $\k$, notaremos $\k^\ast = \k \setminus \setb{0}$.
\end{obs}

\begin{lema}
    Sea $\k$ un cuerpo y sea $p(T) \in \k[T]$ un polinomio de grado $n$. Entonces,
    \[
        \abs{\setb{\text{raíces de } p(T)}} \leq n.
    \]
\end{lema}
\begin{proof}
    Pongamos $p\lp T\rp = a_nT^n+a_{n-1}T^{n-1}+\cdots+a_1T+a_0$. Entonces, si $t$ es una raíz de $p$ o, dicho de otro modo, $p\lp t\rp = 0$, se tiene que
    \[
        p\lp t\rp = a_nt^n+a_{n-1}t^{n-1}+\cdots+a_1t+a_0=0,
    \]
    de modo que
    \begin{align*}
        p\lp T\rp &= p\lp T\rp - p\lp t\rp =\\
        &= a_nT^n+a_{n-1}T^{n-1}+\cdots+a_1T+a_0-a_nt^n-a_{n-1}t^{n-1}-\cdots-a_1t-a_0=\\
        &= \lp T-t\rp\lb a_n\lp T^{n-1} + T^{n-2}t+ \cdots +Tt^{n-2} +t^{n-1}\rp + \cdots+ a_1\rb=\\
        &= \lp T-t\rp q\lp T\rp,
    \end{align*}
    donde $q\lp T\rp$ es un polinomio de grado $n-1$. Así pues, las demás raíces de $p\lp T\rp$ dividen $q\lp T\rp$ y así sucesivamente. Entonces, si $p\lp T\rp$ tubiera más de $n$ raíces, sería de la forma $\lp T-t_1\rp \cdots \lp T-t_{n+1}\rp q\lp T\rp$ y tendría, almenos, grado $n+1$, lo cual supone una contradicción.
\end{proof}

\begin{lema}
    Sea $\k$ un cuerpo y sea $x \in \k^\ast$, $\orden(x) = n$. Entonces,
    \[
        \O_n\left( \k^\ast \right) \subseteq \setb{ \text{raíces de } T^n -1}
        = C_n(x) \subseteq \k^\ast.
    \]
    Además, $\abs{\O_n\left( \k^\ast \right)} = \varphi(n)$.
\end{lema}

\begin{proof}
    Sea $y \in \O_n(\k^\ast)$. Se tiene que  $y^n = 1 \implies y^n -1 = 0 \implies y$ es raíz de $T^n -1$, y tenemos la primera inclusión $\O_n\left( \k^\ast \right) \subseteq\lc\text{raíces de } T^n -1\rc$.
    
    \noindent Ahora, veamos que $\setb{ \text{raíces de } T^n -1} = C_n(x)$. Por un lado,
    \[
        y \in C_n(x) \implies y = x^k \implies y^n = \left( x^k \right)^n = \left( x^n \right)^k = 1,
    \]
    con lo que $C_n(X) \subseteq \setb{\text{raíces de } T^n -1}$. Por otro lado,
    \[
        \abs{\lc\text{raíces de } T^n -1\rc}\leq n =\abs{C_n\lp x\rp},
    \]
    y concluimos que $\setb{ \text{raíces de } T^n -1} = C_n(x)$.
    
    \noindent El último resultado sigue inmediatamente de la inclusión $\O_n\left( \k^\ast \right)\subseteq C_n\lp x\rp$ y de la proposición \ref{prop:euler}.
\end{proof}

\begin{teo*}
    Sea $\k$ un cuerpo y sea $G$ un subgrupo finito de $\k^\ast$. Entonces, $G$ es un grupo
    cíclico. En particular, si $\k$ es finito, $\k^\ast$ es un grupo cíclico.
\end{teo*}

\begin{proof}
    Sea $\abs{G} = n$ y sea $d \vert n$. Tenemos que
    \[
        \O_d(G) = \setb{y \in G \mid \orden(y) = d} \subseteq
        \setb{y \in \k^\ast \mid \orden(y) = d} = \O_d\left( \k^\ast \right).
    \]
    Definimos $m_d = \abs{\O_d(G)} \leq \abs{\O_d\left( \k^\ast \right)} \leq \varphi(d)$.
    Entonces,
    \[
        n = \sum_{d \vert n} m_d \leq \sum_{d \vert n} \varphi(d) = n \implies m_d = \varphi(d).
    \]
    Tomamos ahora $d = n$, se tiene que
    \[
        \begin{rcases}
            m_n = \varphi(n) \geq 1 \implies \exists y \in G \tq \orden(y) = n \\
            \abs{G} = n
        \end{rcases}
        \implies G = C_n(y).
    \]
\end{proof}

\section{Grupos simples}

\begin{defi}[grupo!simple]
    Sea $G$ un grupo no trivial ($G \neq \setb{1}$). Decimos que $G$ es simple si los únicos subgrupos
    normales de $G$ son $\setb{1}$ y $G$.
\end{defi}

\begin{teo*}\label{teo:ordprimo}
    Sea $G$ un grupo. Son equivalentes
    \begin{enumerate}[(i)]
        \item\label{item:teo-random1} $G$ es simple y abeliano,
        \item\label{item:teo-random2} $\abs{G} = p$, con $p$ primo,
        \item\label{item:teo-random3} $G \cong \faktor{\z}{p\z}$ con $p$ primo.
    \end{enumerate}
\end{teo*}

\begin{proof}
    Empecemos probando que \ref{item:teo-random1}$\implies$\ref{item:teo-random2}.
    Como $G \neq \setb{1}, \, \exists x \in G$ con $x \neq 1$. Veamos que $\orden(x) = \infty \implies
    \langle x^2 \rangle \subsetneq G$. Supongamos que $\orden(x) = \infty$. Entonces, $\langle x^2 \rangle = G \implies x \in \langle x^2\rangle
    \implies x = \left( x^2 \right)^n = x^{2n} \implies x^{-1}x = x^{-1}x^{2n} \implies
    1 = x^{2n-1} \implies \orden(x) \leq 2n-1$ lo cual contradice la hipótesis $\orden(x) = \infty$.
    Tenemos ahora que $\langle x^2 \rangle$ es subgrupo propio de $G$. Pero $G$ es abeliano y, por lo tanto, 
    $\langle x^2 \rangle \triangleleft G$, cosa que supone una contradicción con la hipótesis de que $G$ es simple.
    Así pues, $\orden(x) = n > 1$. 
    Tomamos $p \vert n$, con $p$ primo. Sabemos que
    \[
        \orden\left( x^{\frac{n}{p}} \right) = \frac{n}{\mcd\left(n, \frac{n}{p}\right)} =
        \frac{n}{\frac{n}{p}} = p.
    \]
    Es decir, $x^{\frac{n}{p}}$ tiene orden $p$. Tomamos $H = \langle x^{\frac{n}{p}}\rangle \neq \setb{1}$.
    $H$ es un subgrupo de $G$ y $G$ es abeliano, de modo que $H \triangleleft G$. Sin embargo, $G$ es simple,
    con lo cual $H = G$ y concluimos que $\abs{G} = \abs{H} = p$.
    \\
    
    \noindent Veamos ahora que \ref{item:teo-random2}$\implies$\ref{item:teo-random3}. Suponemos que $\abs{G} = p$
    con $p$ primo, y sigue que $\exists x \in G, \, x \neq 1$. En particular $\orden(x) \vert \orden(G) = p$ 
    y, puesto que $p$ es primo, $\orden(x) = p$. Ahora se tiene que $\langle x\rangle \subseteq G$, pero los órdenes
    de los grupos son iguales y, por consiguiente, $G = \langle x\rangle \cong \faktor{\z}{p\z}$.
    \\
    
    \noindent Por último, veamos que \ref{item:teo-random3}$\implies$\ref{item:teo-random1}. $G \cong \faktor{\z}{p\z}$ es el grupo cíclico de $p$ elementos y es abeliano.
    Además, vimos que los siguientes conjuntos están en biyección:
    \[
        \begin{aligned}
            \setb{d \in \z \bigm\vert d \vert n, \; 1 \leq d \leq n } &\leftrightarrow
            \setb{H \mid H \text{ sg. de } G, \; \abs{H} = d} \\
            d &\mapsto \langle y^{\frac{n}{d}} \rangle.
        \end{aligned}
    \]
    Como $n$ es primo, los únicos subgrupos de $G$ son $\setb{1}$ y $G$ y, en particular, son los únicos normales. Concluimos que
    $G$ es simple.
\end{proof}

\begin{teo}[de Feit-Thompson]\label{teo:ordnonprimo}
    Sea $G$ un grupo simple tal que $\abs{G} = n\in\n$, con $n$ non. Entonces, $n$ es un número primo.
\end{teo}
\begin{proof}
    La demostración es demasiado larga y complicada como para abarcarla en esta asignatura.
\end{proof}
\begin{col}
    Sea $G$ un grupo simple tal que $\abs{G} = n\in\n$, con $n$ non. Se tiene que
    \[
        G \equiv \faktor{\z}{p\z},
    \]
    con $p=n$ primo.
\end{col}
\begin{proof}
    El resultado es una consecuencia inmediata de los teoremas \ref{teo:ordnonprimo} y \ref{teo:ordprimo}.
\end{proof}

\begin{teo*}
    Sea $n \geq 5$, entonces $A_n$ es simple.
\end{teo*}

\begin{proof}
    Sabemos (por problemas) que $A_n = \langle \text{3-ciclos}\rangle$. Sean ahora
    $\left( a_1, a_2, a_3 \right)$ y $\left( b_1, b_2, b_3 \right)$ dos 3-ciclos, veremos que
    $\exists \sigma \in A_n$ tal que
    $\sigma \left( a_1, a_2, a_3 \right) \sigma^{-1} = \left( b_1, b_2, b_3 \right)$. Tomamos
    $\sigma$ la permutación que envía $a_1$ a $b_1$, $a_2$ a $b_2$, $a_3$ a $b_3$ y el resto a donde sea.
    Si $\sigma \in A_n$ ya hemos acabado. Si $\sigma \notin A_n$, tomamos
    \[
        \tilde{\sigma} = \sigma \left( a_4, a_5 \right)
    \]
    con todos los $a_{i}$ diferentes entre ellos ($a_4$ y $a_5$ existen porque $n \geq 5$). Entonces se tiene que
    \[
        \begin{aligned}
            \tilde{\sigma} \left( a_1, a_2, a_3 \right) \tilde{\sigma}^{-1} &=
            \sigma \left( a_4, a_5 \right) \left( a_1, a_2, a_3 \right) \left( a_4, a_5 \right) \sigma^{-1} \\
            &= \sigma \left( a_1, a_2, a_3 \right) \cancel{\left( a_4, a_5 \right)}
            \cancel{\left( a_4, a_5 \right)} \sigma^{-1} \\
            &= \sigma\left( a_1, a_2, a_3 \right)\sigma^{-1} = \left( b_1, b_2, b_3 \right)
        \end{aligned}
    \]
    y $\tilde{\sigma} \in A_n$. Tomamos $H \triangleleft A_n$,  con $H$ no trivial. Veamos que $H$ contiene un 3-ciclo. Tomamos 
    $\sigma \in H$, con $\sigma \neq \Id$. Por el ejercicio 20 se tiene que o bien
    \[
            \exists \tau \in A_n \tq \tau \sigma \tau^{-1} \sigma^{-1} \text{ es un 3-ciclo},
    \]
    o bien
    \[
        \exists \tau_1 \tau_2 \in A_n \tq \tau_2 \tau_1\sigma\tau_1^{-1}\sigma^{-1}\tau_2^{-1}
            \sigma\tau_1\sigma^{-1}\tau_1^{-1} \text{ es un 3-ciclo}.
    \]

    \noindent Ahora queremos ver que $H = A_n$. Acabamos de ver que $\exists \sigma \in H$ 3-ciclo y que todo $\tau$ 3-ciclo es el conjugado de $\sigma$
    por un elemento $\rho\in A_n$. Por lo tanto, $\tau\in H$, lo que nos dice que $A_n=\langle\text{3-ciclos}\rangle \subseteq H$ y hemos acabado.
\end{proof}

\begin{lema}\label{lema:maximal_normal}
    Sea $G$ un grupo no trivial y sea $H$ un subgrupo normal a $G$. Entonces, $\faktor{G}{H}$ es simple si y solo si $H$ es un elemeno maximal del conjunto $\lc K\mid K\triangleleft G, K\neq G, H\subseteq K\rc$.
\end{lema}

\begin{proof}
    Como establece la proposición \ref{prop:normalbiyec}, existe una biyección entre los subgrupos normales de $\faktor{G}{H}$ y los subgrupos normales de $G$ que contien $H$. Por definición, $\faktor{G}{H}$ es simple si y solo su tiene exactamente dos subgrupos normales. Así, si $\faktor{G}{H}$ es simple, $H\neq G$ y hay exactamente dos subgrupos normales de $G$: $H$ y $G$. Entonces, $H$ es el elemento maximal de los subgrupos propios de $G$ que contienen $H$. Recíprocamente, si $H$ es un elemeno maximal del conjunto $\lc K\mid K\triangleleft G, K\neq G, H\subseteq K\rc$, entonces $\lc K\mid K\triangleleft G, H\subseteq K\rc = \lc K\mid K\triangleleft G, K\neq G, H\subseteq K\rc \cup \lc G\rc=\lc H, G\rc$. Por consiguiente, $\faktor{G}{H}$ tiene exactamente dos subgrupos normales, es decir, es simple.
\end{proof}
\begin{defi}[torre!normal]
    Sea $G$ un grupo. Llamamos torre normal de $G$ a una cadena de subgrupos de $G$ tal que
    \[
        G=G_0\triangleright G_1\triangleright\cdots\triangleright G_n=\lc 1\rc,
    \]
    es decir, que $G_i$ es un subgrupo normal de $G_{i+1}$. Decimos que los grupos
    \[
        \faktor{G_0}{G_1}, \faktor{G_1}{G_2}, \dots, \faktor{G_{n-1}}{G_n}
    \]
    son los cocientes de la torre y que la longitud de la torre es $n$.
\end{defi}
\begin{defi}[torre!normal abeliana]
    Sea $G$ un grupo. Decimos que una torre normal de $G$ es una torre normal abeliana de $G$ si todos los grupos
    \[
        \faktor{G_0}{G_1}, \faktor{G_1}{G_2}, \dots, \faktor{G_{n-1}}{G_n}
    \]
    son abelianos.
\end{defi}
\begin{defi}[grupo!resoluble]
    Sea $G$ un grupo. Decimos que es resoluble si tiene una torre normal abeliana.
\end{defi}
\begin{defi}[serie!de composición]
    Sea $G$ un grupo. Decimos que una torre normal de $G$ es una serie de composición de $G$ si todos los grupos
    \[
        \faktor{G_0}{G_1}, \faktor{G_1}{G_2}, \dots, \faktor{G_{n-1}}{G_n}
    \]
    son simples.
\end{defi}
\begin{example}
    \begin{enumerate}[1.]
        \item[]
        \item Todo grupo tiene, almenos, una torre normal. En particular, $G=G_0\triangleright G_1=\lc 1\rc$ es una torre normal. Además, es una torre abeliana si y solo si $G$ es abeliano, y es simple si y solo si $G$ es simple. Así pues, si $G$ es abeliano, es resoluble porque $G\triangleright \lc 1\rc$ es una torre normal abeliana.
        \item Consideremos $G=D_{2n}$. Se tiene que $D_{2n}\triangleright \left<r\right> \triangleright\lc 1\rc$ es una torre normal. Veamos que es abeliana.
            \[
                \faktor{\left< r\right>}{\lc 1\rc}=\left< r\right>\cong \faktor{\z}{n\z},
            \]
            que es abeliano, y
            \[
                \faktor{D_{2n}}{\left< r\right>}\cong C_2,
            \]
            que es abeliano. Por lo tanto, $D_{2n}$ es resoluble. Además, puesto que $C_2$ es simple, $D_{2n}\triangleright \left<r\right> \triangleright\lc 1\rc$ es una serie de composición si y solo si $n$ es primo.
        \item Consideremos $G=\Sim_4$. Se tiene que $\Sim_4\trir V_4=\lc \Id, \lp 1, 2\rp\lp 3,4\rp,\lp 1,3\rp\lp 2,4\rp,\lp 1,4\rp\lp 2,3\rp\rc\trir\lc \Id\rc$ es una torre normal. Para ver que $\Sim_4\trir V_4$, observamos que $\sigma\lp a, b\rp\lp c, d\rp\sigma^{-1}=\sigma\lp a, b\rp\sigma^{-1}\sigma\lp c, d\rp\sigma^{-1}=\lp\sigma\lp a\rp\sigma\lp b\rp\rp\lp\sigma\lp c\rp\sigma\lp d\rp\rp\in V_4$. Veamos que $\faktor{G_4}{V_4}$ no conmuta.
        \begin{align*}
            \lp 1,2,3\rp V_4\lp 1,4\rp V_4 &= \lp 1,2,3\rp\lp 1,4\rp V_4 = \\
            &= \lp 1,4,2,3\rp V_4 \neq\\
            &\neq \lp 1,2,3,4\rp V_4 =\\
            &= \lp 1,4\rp\lp 1,2,3\rp V_4 = \\
            &= \lp 1,4\rp V_4\lp 1,2,3\rp V_4.
        \end{align*}
        Así pues, la torre no es abeliana y no es una serie de composición porque $V_4$ no es simple.
        \item Podemos refinar la torre anterior para que sea abeliana y serie de composición. Consideremos la torre
            \[
                \Sim_4\trir A_4 \trir V_4 \trir \lc \Id, \lp 1,2\rp \rc\trir\lc \Id\rc.
            \]
            Veamos que efectivamente es una torre. $\Sim_4\trir A_4$ porque $A_4$ es un subgrupo y $\lb \Sim_4 : A_4\rb=2$. $A_4\trir V_4$ porque $\Sim_4\trir V_4$. $V_4\trir \lc \Id, \lp 1,2\rp \rc$ porque $\lc \Id, \lp 1,2\rp \rc$ es un subgrupo y $\lb V_4 : \lc \Id, \lp 1,2\rp \rc\rb=2$.
            
            \noindent Veamos ahora que cada uno de los cocientes de la torre son simples y conmutativos. 
            \begin{align*}
                \faktor{\Sim_4}{A_4} &\cong \faktor{\z}{2\z},\\
                \faktor{A_4}{V_4} &\cong \faktor{\z}{3\z},\\
                \faktor{V_4}{\lc \Id,\lp 1,2\rp\rc} &\cong \faktor{\z}{2\z},\\
                \faktor{\lc \Id,\lp 1,2\rp\rc}{\lc\Id\rc} &\cong \faktor{\z}{2\z},
            \end{align*}
            que son todos simples y conmutativos.
        \item $\z$ no tiene serie de composición. En primer lugar, observamos que $\z$ no es simple porque $2\z\tril\z$. Sea $G$ un subgrupo de $\z$ (en particular, será un subgrupo normal porque $\z$ es conmutativo). El máximo común divisor $m$ de todos los elementos de $G$ es un elemento de $G$ porque de obtiene sumando (finitos) elementos de $G$. Entonces, $G=\left< m\right>=m\z\cong\z$, y deducimos que todos los elementos de cualquier torre normal de $\z$ serán de la forma $n\z$. Pero $\faktor{n\z}{0}=n\z\cong\z$ no es simple, de manera que siempre habrá un cociente de la torre que no es simple. Por lo tanto, $\z$ no tiene serie de composición.
    \end{enumerate}
\end{example}

\begin{prop}
    Todo grupo finito tiene una serie de composición
\end{prop}

\begin{proof}
    Sea $G$ un grupo finito y supongamos que no tiene serie de composición. De entre sus torres normales (el primer ejemplo nos muestra que hay, almenos, una) tomemos una que no pueda ser refinada, es decir, que para todo par de grupos consecutivos $G_i$ y $G_{i+1}$ de la torre, no exista un grupo $H$ (distinto de $G_i$ y $G_{i+1}$) con $G_i\trir H\trir G_{i+1}$. Sabemos que una tal torre normal existe porque, de no ser así, la torre tendría infinitos grupos de distinto cardinal, lo cual supondría una contradicción con la finitud de $G$. Entonces, la torre en cuestión tiene un par de grupos consecutivos $G_k$ y $G_{k+1}$ tales que $\faktor{G_k}{G_{k+1}}$ no es simple ya que, de lo contrario, seria una serie de composición. Sin embargo, el lema \ref{lema:maximal_normal} nos asegura que $G_{k+1}$ no es un elemento maximal del conjunto de grupos $\lc H\mid H\tril G_k, H\neq G_k, G_{k+1}\subseteq H\rc$. Por lo tanto, existe un grupo $H$ (distinto de $G_k$ y $G_{k+1}$) con $G_k\trir H\supseteq G_{k+1}$. Pero $G_k\trir G_{k+1}$, y necesariamente $H\trir G_{k+1}$. Esto contradice la suposición que no existía un tal $H$ y concluimos que $G$ tiene serie de composición.
\end{proof}

\begin{teo}[de isomorfía (segundo)]\label{teo_dos_iso}
    Sea $G$ un grupo y sean $H,K$ subgrupos de $G$ con $H\tril G$. Entonces,
    \begin{enumerate}[i)]
        \item $\lp H\cap K\rp\tril K$,
        \item $HK$ es un subgrupo de $G$,
        \item $H\tril HK$.
    \end{enumerate}
    Además, se tiene que
    \[
        \faktor{K}{H\cap K}\cong\faktor{HK}{H}.
    \]
\end{teo}
\begin{proof}
    \begin{enumerate}[i)]
        \item[]
        \item Sean $x\in H\cap K,\, a\in K$. Entonces, $axa^{-1}\in K$ trivialmente. Además, puesto que $H\tril G$ y $a\in K\subseteq G$, $axa^{-1}\in H$. Así pues, $axa^{-1}\in H\cap K$ y $\lp H\cap K\rp\tril K$.
        \item Naturalmente, $HK\subseteq G$. Veamos que $HK$ es un grupo. $1\in H, K\implies 1\in HK$. Sean $h_1,h_2\in H$ y sean $k_1,k_2\in K$. Entonces, como que $H\tril G\implies aH=Ha$, $h_1k_1h_2k_2=h_1h_3k_1k_2$, para algún $h_3\in H$. Por ser $H$ y $K$ grupos, $h_1k_1h_2k_2=h_1h_3k_1k_2\in HK$. Finalmente, sea $hk\in HK$. Entonces, $\lp hk\rp ^{-1} = k^{-1}h^{-1}\in k^{-1}H=Hk^{-1}$. Así pues, existe $\comp{h}\in H$ tal que $\lp hk\rp^{-1} = \comp{h}k^{-1}\in HK$.
        \item Sea $a\in H$ y sea $hk\in HK$. Entonces, $hka\lp hk\rp ^{-1} = hkak^{-1}h^{-1}$. Puesto que $H\tril G$, $kak^{-1}=\comp{a}\in H$ y concluimos que $hka\lp hk\rp ^{-1} = hkak^{-1}h^{-1} = h\comp{a}h^{-1}\in H$, de modo que $H\tril HK$.
    \end{enumerate}
    \noindent Sea $\varphi\colon K\xhookrightarrow{i} HK \xrightarrow{\pi} \faktor{HK}{H}$. El cociente $\faktor{HK}{H}$ es un grupo porque $HK$ es un grupo y $H\tril HK$. La composición de morfismos de grupos es un morfismo de grupos, de manera que $\varphi$ es un morfismo de grupos.
    
    \noindent Veamos que $\varphi$ es exhaustivo. Sea $\comp{hk}=hkH\in\faktor{HK}{H}$ la clase de $hk\in HK$. Por ser $H$ un subgrupo normal de $G$, existe algún $h^{\prime}\in H$ tal que $hk=kh^{\prime}$. Entonces, $\comp{hk}=hkH=kh^{\prime}H=kH=\comp{k}$. Así pues, $\comp{hk}=\comp{k}=\varphi\lp k\rp,\,\forall\comp{hk}\in\faktor{HK}{H}$, y concluimos que $\varphi$ es exhaustivo.
    
    \noindent Veamos que $\ker\lp\varphi\rp=H\cap K$. Sea $k\in\ker\lp\varphi\rp$, es decir, $k\in K$ tal que $\varphi\lp k\rp=\comp{1}=H$. Entonces, $\varphi\lp k\rp=\comp{k}=kH=H$, lo cual implica que $k\in H$, de modo que $k\in H\cap K$ y sigue que $\ker\lp\varphi\rp\subseteq H\cap K$. Sea ahora $k\in H\cap K$. Puesto que $k\in K$, $k$ es del dominio de $\varphi$ y tenemos que $\varphi\lp k\rp = \comp{k}=kH=H=\comp{1}$, por ser $k$ de $H$. Finalmente, $\ker\lp\varphi\rp\supseteq H\cap K$ y se da la igualdad.
    
    \noindent Para terminar, $\faktor{K}{H\cap K}$ es un grupo porque $\lp H\cap K\rp\tril K$ y podemos aplicar el primer teorema de isomorfía para deducir que
    \[
        \faktor{K}{H\cap K}\cong\faktor{K}{\ker\lp\varphi\rp}\cong\faktor{HK}{H}.
    \]
\end{proof}

\begin{teo}[de Jordan-Hölder]
    Sea $G$ un grupo y sean
    \[
        \begin{aligned}
            G &= G_0 \triangleright G_1 \triangleright \cdots \triangleright G_n = \setb{1}, \\
            G &= H_0 \triangleright H_1 \triangleright \cdots \triangleright H_m = \setb{1}
        \end{aligned}
    \]
    dos series de composición de $G$. Entonces $n = m$ y $\exists \sigma \in \Sim_n$ tal que
    \begin{equation}\label{equation:series_equivalentes}
        \faktor{H_{i-1}}{H_i} \cong \faktor{G_{\sigma(i)-1}}{G_{\sigma(i)}},\,\forall 1\leq i\leq n.
    \end{equation}
\end{teo}

\begin{proof}
    Aunque el teorema es cierto para cualquier grupo, realizaremos la demostración para grupos
    finitos solamente, por inducción sobre $\abs{G}$. Para simplificar la notación, diremos que dos series de composición
    son equivalentes si satisfacen \ref{equation:series_equivalentes}.

    Sea $G$ un grupo finito y sean $\setb{G_i}_{i=0,\dots,n}$ y $\setb{H_i}_{i=0, \dots, m}$ series de composición de $G$.

    La hipóteis de inducción dice que si $H$ es un grupo con $\abs{H} < \abs{G}$, entonces dos series de 
    composición cualesquiera de $H$ son equivalentes.

    Si $H_1 = G_1$ entonces $\abs{G_1} = \abs{H_1} < \abs{G}$, ya que $\faktor{G_0}{G_1}$ es simple.
    Tenemos entonces que $n-1 = m-1$ y las series de composición $\left\{ G_i \right\}_{i=1,\dots,n}$
    y $\left\{ H_i \right\}_{i=1,\dots,n}$ son equivalentes. Por lo tanto, $n = m$ y las series de composición completas
    son equivalentes.

    Consideremos ahora el caso $H_1 \neq G_1$. Sabemos que $G_1, H_1 \triangleleft G$, lo que implica que $G_1 H_1 \triangleleft G$.
    Como $\faktor{G}{G_1}$ es simple y $H_1 G_1 \supset G_1$, se tiene que $G_1H_1  = G$. Definamos el grupo
    $K_2 = H_1 \cap G_1$ y veamos que $K_2 \triangleleft H_1$. Sea $x \in K_2$ y sea $a \in H_1$, entonces
    \[
        x \in G_1, a \in H_1 \subset G \stackrel{G_1 \triangleleft G}{\implies} axa^{-1} \in G_1.
    \]
    Además, claramente $axa^{-1} \in H_1$, de modo que $K_2 \triangleleft H_1$. Análogamente,
    se demuestra que $K_2 \triangleleft G_1$. Por el segundo teorema del isomorfismo (\ref{teo_dos_iso}), tenemos que
    \[
        \faktor{G_1}{K_2} = \faktor{G_1}{H_1 \cap G_1} \cong \faktor{G_1H_1}{H_1} = \faktor{G}{H_1}.
    \]
    Por el mismo razonamiento, $\faktor{H_1}{K_2} \cong \faktor{G}{G_1}$. Por ser serie de composición,
    $K_2$ tiene serie de composición $K_2 \triangleright K_3 \triangleright \cdots \triangleright K_p =\setb{1}$.
    Hasta ahora tenemos las series
    \begin{gather*}
        G_1 \triangleright K_2 \triangleright \cdots \triangleright K_p = \setb{1}, \quad \quad
        G_1 \triangleright G_2 \triangleright \cdots \triangleright G_n = \left\{ 1 \right\}, \\
        H_1 \triangleright K_2 \triangleright \cdots \triangleright K_p = \setb{1}, \quad \quad
        H_1 \triangleright H_2 \triangleright \cdots \triangleright H_m = \left\{ 1 \right\}.
    \end{gather*}
    Puesto que $\abs{G_1} < \abs{G} \implies$, podemos afirmar que, en virtud de la hipótesis de inducción, $n-1 = p-1$ y que
    la series de composición
    \begin{gather*}
        G_1 \triangleright K_2 \triangleright \cdots \triangleright K_p=\lc 1\rc,\\
        G_1 \triangleright G_2 \triangleright \cdots \triangleright G_n = \left\{ 1 \right\}
    \end{gather*}
    son equivalentes. Análogamente,
    \begin{gather*}
        H_1 \triangleright K_2 \triangleright \cdots \triangleright K_p=\lc 1\rc,\\
        H_1 \triangleright H_2 \triangleright \cdots \triangleright H_m = \left\{ 1 \right\}
    \end{gather*}
    también son series de composición equivalentes. En particular, $n=m$. Recordemos las series de composición originales
    \[
        \begin{aligned}
            G &= G_0 \triangleright G_1 \triangleright \cdots \triangleright G_n = \setb{1}, \\
            G &= H_0 \triangleright H_1 \triangleright \cdots \triangleright H_n = \setb{1}.
        \end{aligned}
    \]
    Debido a que $\faktor{G_1}{K_2} \cong \faktor{G}{H_1}$ y que $\faktor{H_1}{K_2} \cong \faktor{G}{G_1}$, concluimos que
    las dos series de composición de $G$ son equivalentes.
\end{proof}

\begin{prop}
    Sea $G$ un gupo y $H$ un subgrupo. Entonces,
    \begin{enumerate}[i)]
        \item Si $G$ es resoluble, $H$ es resoluble.
        \item Si $H \triangleleft G$ y $G$ es resoluble, $\faktor{G}{H}$ es resoluble.
        \item Si $H \triangleleft G$ y $H, \, \faktor{G}{H}$ son resolubles, $G$ es resoluble.
    \end{enumerate}
\end{prop}

\begin{proof}
    Ejercicio.
\end{proof}
