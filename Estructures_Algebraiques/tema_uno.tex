\chapter{Grupos}

\begin{defi}[grupo]
    Un grupo es un par $\lp G, \cdot \rp$, donde $G$ es un conjunto no vacío y $\cdot$ es una operación interna, es decir, una aplicación
    \[
        \begin{aligned}
            \cdot \colon G \times G &\to G \\
            (a, b) &\mapsto a \cdot b
        \end{aligned}
    \]
    que satisface
    \begin{enumerate}[i)]
        \item $(a\cdot b)\cdot c = a\cdot (b\cdot c)$, i.e., la propiedad asociativa,
        \item $\exists e \tq \forall a \in G, \, a\cdot e = e\cdot a = a$, i.e., existe un elemento neutro,
        \item $\forall a \in G, \, \exists \tilde{a} \in G \tq a \cdot \tilde{a} = \tilde{a}\cdot a = e$, i.e., todo elemento tiene inverso.
    \end{enumerate}
    \emph{Nota.} Cuando la operación del grupo sea irrelevante, evidente o se deduzca del contexto, escribiremos $G$ en lugar de $\lp G, \cdot\rp$, o de $\lp G, +\rp$, etc., cometiendo un abuso de notación.
    
    \vspace{1.15ex}
    
    \noindent \emph{Nota segunda.} Además, a menudo también escribiremos $ab$ en lugar de $a\cdot b$ y $a\cdot b$ en lugar de $a\circ b$, como por ejemplo en la composición de permutaciones.
\end{defi}

\begin{defi}[grupo!abeliano o conmutativo]
    Diremos que $G$ es un grupo abeliano o conmutativo si es un grupo y además satisface la propiedad conmutativa:
    \[
        ab = ba, \quad \forall a, b \in G.
    \]
\end{defi}

\begin{obs}
    Existen varias notaciones para referirnos a esta operación:
    \begin{center}
        \begin{tabular}{|c|c|c|c|} \hline
            Operación & S\'imbolo & Elemento neutro & Elemento inverso \\ \hline \hline
            Aditiva & $+$ & 0 & $-a$ (e. opuesto) \\ \hline
            Multiplicativa & $\cdot$ & 1 & $a^{-1}$ \\ \hline
        \end{tabular}
    \end{center}
    \emph{Nota.} Siempre que utilicemos $+$ para la operación del grupo, la operación será conmutativa.
\end{obs}

\begin{defi}[subgrupo]
    Sea $\lp G, \cdot \rp$ un grupo. Decimos que $\lp H, \cdot_{\mid H} \rp$ es un subgrupo de $\lp G, \cdot \rp$ si $H \subseteq G$ y se satisface
    \begin{enumerate}[i)]
        \item $H \neq \emptyset$,
        \item $a, b \in H \implies a\cdot b \in H$ (la operación es cerrada),
        \item $\forall a \in H, a^{-1} \in H$.
    \end{enumerate}
    \emph{Nota. } A menudo cometeremos un abuso de notación, escribiendo $\lp H, \cdot \rp$ en lugar de $\lp H, \cdot_{\mid H} \rp$.
\end{defi}

\begin{prop}
    Los subgrupos son aquellos grupos $\lp H, \cdot_{\mid H} \rp$ con $H\subseteq G$.
\end{prop}
\begin{proof}
    Sea $\lp H, \cdot \rp$ un subgrupo de $\lp G, \cdot \rp$. Queremos ver que $\lp H, \cdot \rp$ es un grupo.
    Tenemos la operación
    \[
        \begin{aligned}
            \cdot \colon H \times H &\to H \\
            (a, b) &\mapsto a\cdot b \in H.
        \end{aligned}
    \]
    Tiene la propiedad asociativa porque es la restricción de una operación con la propiedad asociativa.
    Existe elemento neutro ya que $\exists a \in H$ y $\exists a^{-1} \in H$, de modo que $a\cdot a^{-1}=e \in H$.
    La última propiedad está impuesta.
    
    Recíprocamente, veamos que si $H\subseteq G$ y $\lp H, \cdot_{\mid H} \rp$ es un grupo, entonces $\lp H, \cdot_{\mid H} \rp$ es un subgrupo de $G$. Por ser $\lp H, \cdot_{\mid H} \rp$ un grupo, $1\in H \implies H\neq\varnothing$. Las otras dos propiedades están en la propia definición de grupo.
\end{proof}

\begin{example}
    \begin{enumerate}[1.]
        \item[]
        \item Sea $G$ un grupo. Los subgrupos impropios son
                $\begin{cases}
                    \setb{1} &\text{(el subgrupo trivial)}, \\
                    G. &
                \end{cases}$
        \item $\lp \z, + \rp, \lp \n, + \rp, \lp \real, + \rp, \lp \cx, + \rp$ son grupos y subgrupos.
        \item $\lp \z/n\z, + \rp$ es un grupo.
        \item Si $G$ y $H$ son dos grupos, entonces
            \[
                 G \times H = \setb{(x,y) \vert x \in G, \, y \in H}   
            \]
            es un grupo, con $(a, b) \cdot (c, d) = (ac, bd)$.
        \item $\lp S_n, \circ \rp$ es el grupo simétrico de $n$ elementos (permutaciones de $n$ elementos).
        \item Grupo diedial. $\lp D_{2n}, \circ \rp$, donde $D_{2n}$ son los conjuntos de las isometrías del plano que dejan invariante $P_n$.
            $P_n$ es un polígono regular de $n$ lados (raices $n$-esimas de 1). Por ejemplo,
            \[
                D_{2 \cdot 4} = \setb{id, r, r^2, r^3, s, rs, r^2s, r^3s}
            \]
            Con $r$ la rotación horaria de $\pi / 2$ y $s$ la simetría respecto del eje $x$.
    \end{enumerate}
\end{example}

\section{Intersección y producto de subgrupos}

\begin{defi}[intersección de subgrupos]
    Sea $G$ un grupo y sean $H, K \subset G$ subgrupos de $G$. Definimos la intersección de
    $H$ y $K$ como
    \[
        H \cap K = \setb{x \in G \; \vert \; x \in H \text{ y } x \in K}.
    \]
\end{defi}

\begin{obs}
    Si $H$ y $K$ son subgrupos de $G$, $H \cap K$ es un subgrupo de $G$.
    También es cierto con la instersección arbitraria.
\end{obs}

\begin{defi}[unión de subgrupos]
    Sea $G$ un grupo y sean $H, K \subseteq G$ subgrupos de $G$. Llamaremos unión de $H$ y $K$ a
    \[
        H \cup K = \setb{x \in G \, \vert\, x \in H \text{ o } x \in K}.
    \]
\end{defi}

\begin{obs}
    En general, la unión de subgrupos no es un grupo.
\end{obs}

\begin{example}
    Tomamos el grupo simétrico como ejemplo:
    \[
        \Sim_3 = \setb{ \Id, (1 \, 2), (1 \, 3), (2 \, 3), (1 \, 2 \, 3), (1 \, 3 \, 2)}
    \]
    y tomamos
    \[
        H = \setb{\Id, (1 \, 2)}, \, K = \setb{\Id, (1 \, 3)}
    \]
    ahora
    \[
        H \cup K = \setb{\Id, (1 \, 2), (1\, 3)}
    \]
    pero
    \[
        (1 \, 2)(1 \, 3) = (1 \, 3 \, 2) \notin H \cup K.
    \]
\end{example}

\begin{defi}[producto de subgrupos]
    Sea $G$ un grupo y sean $H, K \subset G$ subgrupos. Definimos el producto $H \cdot K$ como
    \[
        H \cdot K = \setb{xy \,\vert\, x \in H \text{ y } x \in K}.
    \]
\end{defi}

\begin{obs}
    En general, el producto de subgrupos, no es grupo.
\end{obs}

\begin{example}
    Tomando las definiciones de $G$, $H$ y $K$ del ejemplo anterior, tenemos que
    \[
        H \cdot K = \setb{\Id, (1 \, 3), (1 \, 2), (1 \, 2)(1 \, 3) = (1 \, 3 \, 2)},
    \]
    que no es un grupo.
\end{example}

\begin{obs}
    Si $G$ es conmutativo, el producto de subgrupos es un grupo.
\end{obs}

\begin{proof}
    Comprobemos que $H \cdot K$ satisface las propiedades de los grupos.
    \begin{enumerate}[i)]
        \item 
            $
                \begin{rcases}
                    H \text{ sg.} \implies 1 \in H \\
                    K \text{ sg.} \implies 1 \in K
                \end{rcases}
                \implies 1 = 1\cdot 1 \in H \cdot K.
            $
        \item
            $
                \begin{rcases}
                    xy \in HK \\
                    zt \in HK
                \end{rcases}
                \implies (xy)(zt) = (xyzt) = (xz)(yt) \in HK.
            $
        \item
            $
                (xy)^{-1} = y^{-1} x^{.1} = x^{-1}y^{-1} \in HK.
            $
    \end{enumerate}
\end{proof}

\begin{obs}
    Se tiene que
    \[
        H \cap K \subseteq H, K \subseteq H \cup K \subseteq H \cdot K.
    \]
\end{obs}

\begin{obs}
    Si $HK$ es un subgrupo, entonces es el menor subgrupo que contiene a $H \cup K$. 
\end{obs}

\begin{proof}
    Es claro que $H \cup K \subseteq HK$, ya que
    \[
        \forall x \in H, \, x\cdot1 = x \in HK
    \]
    y análogamente para $K$.

    Suponemos ahora que $L$ es un subgrupo de $G$ que contiene a $H \cup K$. Queremos ver que
    $ H \cdot K \subseteq L$. Tenemos que, dado cualquier $z = ab \in HK$ ($a \in H$, $b \in K$)
    \[
        \begin{rcases}
            a \in H \subset L \\
            b \in K \subset L \\
            L \text{ es subgrupo}
        \end{rcases}
        \implies ab = z \in L
    \]
    por lo tanto, $HK \in L$.
\end{proof}

\begin{defi}[subgrupo!generado]
        Sea $\lp G, \cdot\rp$ un grupo y sea $S\subseteq G$. Definimos 
        \[
            \left<S\right> = \lp \lc 1 \rc \cup \setb{a_1 \cdots a_r \,\vert\, a_i \in S \text{ ó } a^{-1}_i \in S}, \cdot \rp.
        \]
\end{defi}

\begin{obs}
    Si $S = \emptyset$, entonces $\left<S\right> = \lp \setb{1}, \cdot \rp$.
\end{obs}

\begin{obs}
    $\left<S\right>$ es el menor subgrupo de de $G$ que contiene a $S$.
\end{obs}

\begin{proof}
    Si $S = \emptyset$ entonces es trivial.
    Si $S \neq \emptyset$, es trivial que $S \subset \left<S\right>$, veamos ahora que es un subgrupo de $G$.
    \begin{enumerate}[i)]
        \item $\exists a \in S \implies a \in \left<S\right> \implies \left<S\right> \neq \emptyset$,
        \item Si $a_1\cdots a_r, b_1 \cdots b_s \in \left<S\right>$, entonces $a_1 \cdots a_r b_1 \cdots b_s \in \left<S\right>$,
        \item Si $a_1 \cdots a_r \in \left<S\right>$, entonces $\left( a_1 \cdots a_r \right)^{-1} =
            a^{-1}_r \cdots a^{-1}_1 \in \left<S\right>$.
    \end{enumerate}
    Tomamos ahora $L$ un subgrupo de $G$ que contiene a $S$. Queremos ver que $\left<S\right> \subseteq L$.
    Para cualquier $a_1 \cdots a_r \in \left<S\right>$ tenemos que
    \[
        \begin{rcases}
            a_1 \in S \subseteq L \text{ o } a^{-1}_1 \in S \implies \left( a^{-1}_1 \right)^{-1} \in L\\
            \vdots \\
            a_r \in S \subseteq L\text{ o } a^{-1}_r \in S \implies \left( a^{-1}_r \right)^{-1} \in L
        \end{rcases}
        \implies a_1 \cdots a_r \in L.
    \]
    y por lo tanto, $\left<S\right> \subset L$.
\end{proof}

\begin{ej}
    Demostrar que
    \[
        \left<S\right> = \setb{a^{n_1}_1 \cdots a^{n_r}_r \,\vert\, a_i \in S, \, n_i \in \z}.
    \]
\end{ej}

\begin{ej}
    Demostrar que
    \[
        \left<S\right> = \bigcap_{\substack{H \text{ sg. de } G \\ S \subseteq H}} H.
    \]
\end{ej}

\section{Orden de un elemento}

\begin{defi}[orden!de los elementos de un grupo]
    Sea $G$ un grupo y sea $x \in G$. Llamamos orden de $x$, si existe, al menor entero $n \geq 1$ tal que
    \[
        x^n = 1.
    \]
    Si no existe, decimos que $x$ tiene orden infinito.
\end{defi}

\begin{obs}
    Escribimos el orden de $x$ como $\orden(x)$ o $\ord(x)$.
\end{obs}

\begin{defi}[orden!de un grupo]
    Sea $G$ un grupo. Llamamos orden de $G$ a su cardinal y lo denotamos $\orden(G)$, $\ord(G)$, $\abs{G}$ o $\card(G)$.
\end{defi}

\begin{example}
    \begin{enumerate}[1.]
        \item[]
        \item $\ord(e) = 1$ y es el único elemento (el neutro) que tiene orden 1.
        \item En el grupo simétrico $G = \Sim_n$, $\ord\left( a_1, \dots, a_n \right) = n$.
        \item En los grupos $\lp \z, +\rp , \lp \q, +\rp, \lp \real, +\rp$ y $\lp \cx, +\rp$, $\forall x \neq 0 \, \ord(x) = \infty$.
        \item En el grupo $\z/p\z$ con $p$ primo, $\forall \bar{x} \neq \bar{0} \, \ord\left( \bar{x} \right) = p$.
        \item En los grupos $\q^{\ast}=\lp \q\setminus \lc 0\rc, \cdot \rp, \real^{\ast}=\lp \real\setminus \lc 0\rc, \cdot\rp$, $\ord(-1) = 2$, $\ord(1) = 1$ y $\forall x \notin \setb{-1, 1} \, \ord(x) = \infty$.
        \item En el grupo $\cx^{\ast}=\lp \cx\setminus \lc 0\rc, \cdot \rp$, $\forall n \geq 1$ $\ord\left( e^{\frac{2 \pi i}{n}} \right) = n$ y $\forall z \in \cx \tq \abs{z} \neq 1, \, \ord(z) = \infty$.
    \end{enumerate}
\end{example}

\begin{lema}
    Sea $G$ un grupo y $x \in G$ con $\ord(x) = n\in\n$. Entonces,
    \begin{enumerate}[i)]
        \item $x^m = 1 \iff n \vert m$.
        \item $x^m = x^{m^\prime} \iff m \cong m^\prime \, \pmod{n}$.
        \item $\ord\left( x^m \right) = \frac{n}{\mcd(n, m)} = \frac{\ord(x)}{\mcd\left( \ord(x), m \right)}$.
    \end{enumerate}
\end{lema}

\begin{proof}
    \begin{enumerate}[i)]
        \item[]
        \item Si $n \vert m$, entonces $m = n \cdot d$, con lo cual, $x^m = \left( x^n \right)^d = 1^d = 1$. Recíprocamente, pongamos $m = nq + r$, con $0 \leq r < n$. Entonces,
            \[
                \begin{rcases}
                    1 = x^m = x^{nq + r} = \left( x^n \right)^q x^r = x^r \\
                    0 \leq r < n
                \end{rcases}
                \implies r = 0 \implies n \vert m.
            \]
        \item $x^m = x^{m^\prime} \iff x^{m - m^\prime} = 1 \iff n \vert m - m^\prime \iff
            m \cong m^\prime \pmod{n}$.
        \item Sean $k = \ord\left( x^m \right)$ y $g = \mcd(n, m)$. Queremos ver que $k = n/g$.
            \[
                \left( x^m \right)^{\frac{n}{g}} = x^{\frac{mn}{g}} = \left( x^n \right)^{\frac{m}{g}} = 1
                \implies k \left\vert \frac{n}{g} \right. .
            \]
            Por otro lado,
            \[
                1 = \left( x^m \right)^k = x^{mk} \implies n \vert mk \implies \left.\frac{n}{g} \right\vert \frac{m}{g} k
                \substack{n/g \text{ y } m/g \\ \implies \\ \text{ primos entre si}} \left.\frac{n}{g} \right\vert k.
            \]
            Y sumando los dos resultados, tenemos que $\frac{n}{g} = k$.
    \end{enumerate}
\end{proof}

\begin{defi}[grupo!cíclico]
    Diremos que un grupo $G$ es cíclico si está generado por un solo elemento $x \in G$.
    Escribiremos $G = \left< x \right>$, $G$ generado por $x$ o $x$ generdor de $G$.
\end{defi}

\begin{obs}
    Si $\ord(G) = n$ (con $n$ finito), entonces
    \[
        G = \setb{ 1 \left( = x^0 \right), x, x^2, \dots, x^{n-1}}.
    \]
    Y lo denotaremos cono $G = C_n$ (grupo cíclico de orden $n$).
    Si $\ord(G) = \infty$, entonces
    \[
        G = \setb{x^k \vert k \in \z}.
    \]
\end{obs}

\begin{example}
    \begin{enumerate}[1.]
        \item[]
        \item $\z = \left< 1 \right> = \left< -1 \right>$
        \item $\z/n\z = \left< \bar{k} \right>$ con $\mcd(n, k) = 1$
    \end{enumerate}
\end{example}

\begin{defi}[función de Euler]
    Sea $d \in \z, d \geq 1$, definimos la función de Euler como
    \[
        \phi(d) = \card \setb{ 1 \leq k \leq d \vert \mcd(k, d) = 1}.
    \]
\end{defi}

\begin{example}
    \begin{align*}
        \phi(1) &= 1, & \phi(5) &= 4, & \phi(3) &= 2, & \phi(7) &= 6, \\
        \phi(2) &= 1, & \phi(6) &= 2, & \phi(4) &= 2, & \phi(p) &= p-1 \text{(con } p \text{ primo).} 
    \end{align*}
\end{example}

\begin{prop}
    Sea $G = \left< x \right>$ un grupo cíclico, con $\ord(x) = n$, entonces
    \begin{enumerate}[i)]
        \item $\forall y \in G, \ord(y) \vert n$,
        \item $\forall d \vert n \, \exists \phi(d)$ elementos de $G = C_n$ de orden $d$.
    \end{enumerate}
\end{prop}

\begin{proof}
    \begin{enumerate}[i)]
        \item[]
        \item $y \in G = \left< x \right> = \setb{1, x, \dots, x^{n-1}} \implies$
                \[
                    \ord(y) = \ord\left( x^m \right) = \frac{\ord(x)}{\mcd\left(\ord(x), m \right)} =
                    \left.\frac{n}{\mcd(n, m)} \right\vert n
                \]
        \item Sea $y \in G \tq \ord(y) = d$, tenemos que
                \[
                    y = x^m \implies \ord(y) = \frac{n}{\mcd(n, m)} \implies d = \frac{n}{\mcd(n, m)} 
                    \implies \mcd(n, m) = \frac{n}{d}.
                \]
                Buscamos los $m$ tales que $\mcd(m, n) = \frac{n}{d}$ y por lo tanto
                \[
                    \begin{rcases}
                        \frac{n}{d} \vert m \implies m = \left( \frac{n}{d} \right)k \\
                        1 \leq m \leq n
                    \end{rcases}
                    \implies 1 \leq \left( \frac{n}{d} \right)k \leq n = \left( \frac{n}{d} \right)d
                    \implies 1 \leq k \leq d.
                \]
                Buscamos por lo tanto, $k\, 1 \leq k \leq d$ tal que $\mcd(n, m) = \frac{n}{d}$ y tenemos
                \[
                    \mcd(n, m) = \frac{n}{d} \iff \mcd\left( \frac{n}{d}d, \frac{n}{d}k \right) = \frac{n}{d}
                    \implies \mcd(k, d) = 1.
                \]
                Con esta última condición, tenemos que
                \[
                    \phi(d) = \card \setb{ k \in \z \vert \mcd(k, d) = 1 \text{ y } 1 \leq k \leq d}
                    = \card \setb{ x^m \vert \ord\left( x^m \right) = d}.
                \]
    \end{enumerate}
\end{proof}

\begin{col}
    Se tiene que
    \[
        n = \sum_{d \vert n} \phi(d).
    \]
\end{col}

\begin{proof}
    Tomamos $G = C_n = \left< x \right>$, entonces
    \[
        n = \abs{G} = \sum_{d \vert n} \card \setb{ x \in G \vert \ord(x) =d } = \sum_{d \vert n} \phi(d).
    \]
    Ya que
    \[
        G = \bigcup_{d \vert n} \setb{x \in G \vert \ord(x) = d}.
    \]
\end{proof}
