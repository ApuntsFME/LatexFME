\chapter{Grupos}

\begin{defi}[grupo]
    Un grupo es un par $\lp G, \cdot \rp$, donde $G$ es un conjunto no vacío y $\cdot$ es una operación interna, es decir, una aplicación
    \[
        \begin{aligned}
            \cdot \colon G \times G &\to G \\
            (a, b) &\mapsto a \cdot b
        \end{aligned}
    \]
    que satisface
    \begin{enumerate}
        \item $(a\cdot b)\cdot c = a\cdot (b\cdot c)$, i.e., la propiedad asociativa,
        \item $\exists e \tq \forall a \in G, \, a\cdot e = e\cdot a = a$, i.e., existe un elemento neutro,
        \item $\forall a \in G, \, \exists \tilde{a} \in G \tq a \cdot \tilde{a} = \tilde{a}\cdot a = e$, i.e., todo elemento tiene inverso.
    \end{enumerate}
    \emph{Nota.} Cuando la operación del grupo sea irrelevante, evidente o se deduzca del contexto, escribiremos $G$ en lugar de $\lp G, \cdot\rp$, o de $\lp G, +\rp$, etc., cometiendo un abuso de notación.
    
    \vspace{1.15ex}
    
    \noindent \emph{Nota segunda.} Además, a menudo también escribiremos $ab$ en lugar de $a\cdot b$ o en lugar de $a\circ b$, como por ejemplo en la composición de permutaciones.
\end{defi}

\begin{defi}[grupo!abeliano o conmutativo]
    Diremos que $G$ es un grupo abeliano o conmutativo si es un grupo y además satisface la propiedad conmutativa:
    \[
        ab = ba, \quad \forall a, b \in G.
    \]
\end{defi}

\begin{obs}
    Existen varias notaciones para referirnos a esta operación:
    \begin{center}
        \begin{tabular}{|c|c|c|c|} \hline
            Operación & S\'imbolo & Elemento neutro & Elemento inverso \\ \hline \hline
            Aditiva & $+$ & 0 & $-a$ (e. opuesto) \\ \hline
            Multiplicativa & $\cdot$ & 1 & $a^{-1}$ \\ \hline
        \end{tabular}
    \end{center}
    \emph{Nota.} Siempre que utilicemos $+$ para la operación del grupo, la operación será conmutativa.
\end{obs}

\begin{defi}[subgrupo]
    Sea $\lp G, \cdot \rp$ un grupo. Decimos que $\lp H, \cdot_{\mid H} \rp$ es un subgrupo de $\lp G, \cdot \rp$ si $H \subseteq G$ y se satisface
    \begin{itemize}
        \item $H \neq \emptyset$
        \item $a, b \in H \implies a\cdot b \in H$ (la operación es cerrada)
        \item $\forall a \in H, a^{-1} \in H$
    \end{itemize}
    \emph{Nota. } A menudo cometeremos un abuso de notación, escribiendo $\lp H, \cdot \rp$ en lugar de $\lp H, \cdot_{\mid H} \rp$.
\end{defi}

\begin{obs}
    Los subgrupos son aquellos grupos $\lp H, \cdot_{\mid H} \rp$ con $H\subseteq G$.
\end{obs}
\begin{proof}
    Sea $\lp H, \cdot \rp$ un subgrupo de $\lp G, \cdot \rp$. Queremos ver que $\lp H, \cdot \rp$ es un grupo.
    Tenemos la operación
    \[
        \begin{aligned}
            \cdot \colon H \times H &\to H \\
            (a, b) &\mapsto a\cdot b \in H.
        \end{aligned}
    \]
    Tiene la propiedad asociativa porque es la restricción de una operación con la propiedad asociativa.
    Existe elemento neutro ya que $\exists a \in H$ y $\exists a^{-1} \in H$, de modo que $a\cdot a^{-1}=e \in H$.
    La última propiedad está impuesta.
    
    Recíprocamente, veamos que si $H\subseteq G$ y $\lp H, \cdot_{\mid H} \rp$ es un grupo, entonces $\lp H, \cdot_{\mid H} \rp$ es un subgrupo de $G$. Por ser $\lp H, \cdot_{\mid H} \rp$ un grupo, $1\in H \implies H\neq\varnothing$. Las otras dos propiedades están en la propia definición de grupo.
\end{proof}

\begin{example}
    \begin{itemize}
        \item[]
        \item Sea $G$ un grupo. Los subgrupos impropios son
                $\begin{cases}
                    \setb{1} &\text{(el subgrupo trivial)} \\
                    G. &
                \end{cases}$
        \item $\lp \z, + \rp, \lp \n, + \rp, \lp \real, + \rp, \lp \cx, + \rp$ son grupos y subgrupos.
        \item $\lp \z/n\z, + \rp$ es un grupo.
        \item Si $G$ y $H$ son dos grupos, entonces
            \[
                 G \times H = \setb{(x,y) \vert x \in G, \, y \in H}   
            \]
            es un grupo, con $(a, b) \cdot (c, d) = (ac, bd)$.
        \item $\lp S_n, \circ \rp$ es el grupo simétrico de $n$ elementos (permutaciones de $n$ elementos).
        \item Grupo diedial. $\lp D_{2n}, \circ \rp$, donde $D_{2n}$ son los conjuntos de las isometrías del plano que dejan invariante $P_n$.
            $P_n$ es un polígono regular de $n$ lados (raices $n$-esimas de 1). Por ejemplo,

            \[
                D_{2 \cdot 4} = \setb{id, r, r^2, r^3, s, rs, r^2s, r^3s}
            \]
            Con $r$ la rotación horaria de $\pi / 2$ y $s$ la simetría respecto del eje $x$.
    \end{itemize}
\end{example}

\section{Intersección y producto de subgrupos}

\begin{defi}[intesección de subgrupos]
    Sea $G$ un grupo y sean $H, K \subset G$ subgrupos de $G$. Definimos la intersección de
    $H$ y $K$ como
    \[
        H \cap K = \setb{x \in G \; \vert \; x \in H \text{ y } x \in K}.
    \]
\end{defi}

\begin{obs}
    Si $H$ y $K$ son subgrupos de $G$, $H \cap K$ es un subgrupo de $G$.
    También es cierto con la instersección arbitraria.
\end{obs}

\begin{defi}[unión de subgrupos]
    Sea $G$ un grupo y sean $H, K \subseteq G$ subgrupos de $G$. Llamaremos unión de $H$ y $K$ a
    \[
        H \cup K = \setb{x \in G \, \vert\, x \in H \text{ o } x \in K}.
    \]
\end{defi}

\begin{obs}
    En general, la unión de subgrupos no es un grupo.
\end{obs}

\begin{example}
    Tomamos el grupo simétrico como ejemplo:
    \[
        \Sim_3 = \setb{ \Id, (1 \, 2), (1 \, 3), (2 \, 3), (1 \, 2 \, 3), (1 \, 3 \, 2)}
    \]
    y tomamos
    \[
        H = \setb{\Id, (1 \, 2)}, \, K = \setb{\Id, (1 \, 3)}
    \]
    ahora
    \[
        H \cup K = \setb{\Id, (1 \, 2), (1\, 3)}
    \]
    pero
    \[
        (1 \, 2)(1 \, 3) = (1 \, 3 \, 2) \notin H \cup K.
    \]
\end{example}

\begin{defi}[producto de subgrupos]
    Sea $G$ un grupo y sean $H, K \subset G$ subgrupos. Definimos el producto $H \cdot K$ como
    \[
        H \cdot K = \setb{xy \,\vert\, x \in H \text{ y } x \in K}.
    \]
\end{defi}

\begin{obs}
    En general, el producto de subgrupos, no es grupo.
\end{obs}

\begin{example}
    Tomando las definiciones de $G$, $H$ y $K$ del ejemplo anterior, tenemos que
    \[
        H \cdot K = \setb{\Id, (1 \, 3), (1 \, 2), (1 \, 2)(1 \, 3) = (1 \, 3 \, 2)},
    \]
    que no es un subgrupo.
\end{example}

\begin{obs}
    Si $G$ es conmutativo, el producto de subgrupos es un grupo.
\end{obs}

\begin{proof}
    Vamos a comprobar que $H \cdot K$ es subgrupo comprobando las propiedades.
    \begin{itemize}
        \item 
            $
                \begin{rcases}
                    H \text{ sg.} \implies 1 \in H \\
                    K \text{ sg.} \implies 1 \in K
                \end{rcases}
                \implies 1 = 1\cdot 1 \in H \cdot K
            $
        \item
            $
                \begin{rcases}
                    xy \in HK \\
                    zt \in HK
                \end{rcases}
                \implies (xy)(zt) = (xyzt) = (xz)(yt) \in HK
            $
        \item
            $
                (xy)^{-1} = y^{-1} x^{.1} = x^{-1}y^{-1} \in HK
            $
    \end{itemize}
\end{proof}

\begin{obs}
    Se tiene que
    \[
        H \cap K \subset H, K \subset H \cup K \subset H \cdot K.
    \]
\end{obs}

\begin{obs}
    Si $HK$ es un subgrupo, entonces es el menor subgrupo que contiene a $H \cup K$. 
\end{obs}

\begin{proof}
    Es claro que $H \cup K \subset HK$, ya que
    \[
        \forall x \in H, \, x\cdot1 = x \in HK
    \]
    y análogamente para $K$.

    Suponemos ahora que $L$ es un subgrupo de $G$ que contiene a $H \cup K$. Queremos ver que
    $ H \cdot K \subseteq L$. Tenemos que, dado cualquier $z = ab \in HK$ ($a \in H$, $b \in K$)
    \[
        \begin{rcases}
            a \in H \subset L \\
            b \in K \subset L \\
            L \text{ es sg.}
        \end{rcases}
        \implies ab = z \in L
    \]
    por lo tanto, $HK \in L$.
\end{proof}

\begin{defi}[subgrupo!generado]
        Sea $G$ un grupo y sea $S$ un subconjunto de $G$. Definimos 
        \[
            \left<S\right> = \setb{a_1 \cdots a_r \,\vert\, a_i \in S \text{ ó } a^{-1}_i \in S}.
        \]
\end{defi}

\begin{obs}
    Si $S = \emptyset$, entonces $\left<S\right> = \setb{1}$.
\end{obs}

\begin{obs}
    $\left<S\right>$ es el menor subgrupo de de $G$ que contiene a $S$.
\end{obs}

\begin{proof}
    Si $S = \emptyset$ entonces es trivial.
    Si $S \neq \emptyset$, es trivial que $S \subset \left<S\right>$, veremos ahora que es un subgrupo:
    \begin{itemize}
        \item $\exists a \in S \implies a \in \left<S\right> \implies \left<S\right> \neq \emptyset$
        \item Si $a_1\cdots a_r, b_1 \cdots b_s \in \left<S\right> \implies a_1 \cdots a_r b_1 \cdots b_s \in \left<S\right>$
        \item Si $a_1 \cdots a_r \in \left<S\right> \implies \left( a_1 \cdots a_r \right)^{-1} =
            a^{-1}_r \cdots a^{-1}_1 \in \left<S\right>$
    \end{itemize}
    Tomamos ahora $L$ un subgrupo, queremos ver que $\left<S\right> \subset L$.
    Para cualquier $a_1 \cdots a_r \in \left<S\right>$ tenemos que
    \[
        \begin{rcases}
            a_1 \in S \subset L \text{ ó } a^{-1}_1 \in S \implies \left( a^{-1}_1 \right)^{-1} \in L\\
            \vdots \\
            a_r \in S \subset L
        \end{rcases}
        \implies a_1 \cdots a_r \in L
    \]
    y por lo tanto, $\left<S\right> \subset L$
\end{proof}

\begin{ej}
    Demostrar que
    \[
        \left<S\right> = \setb{a^{n_1}_1 \cdots a^{n_r}_r \,\vert\, a_i \in S, \, n_i \in \z}.
    \]
\end{ej}

\begin{ej}
    Demostrar que
    \[
        \left<S\right> = \bigcap_{\substack{H \text{ sg. de } G \\ S \subset H}} H.
    \]
\end{ej}
