\chapter{Espai de probabilitat}

\section{Definició axiomàtica de probabilitat}

\begin{defi}[espai!de probabilitat]
    Un espai de probabilitat és un espai de mesura $(\Omega,\Asuc, p)$ tal que $p(\Omega)=1$, sent .
\end{defi}

\begin{defi}[espai!mostral]
    $\Omega$ és l'espai mostral.
\end{defi}

\begin{defi}[conjunt!d'esdeveniments]
    $\Asuc$ és el conjunt d'esdeveniments o de successos.
\end{defi}

\begin{defi}[funció!de probabilitat]
    $p$ és la funció de probabilitat.
\end{defi}

\begin{obs}
    $\lp\Omega,\Asuc\rp$ és un espai mesurable si compleix les tres condicions de $\sigma$-àlgebra:
    \begin{enumerate}
        \item $\emptyset \in \Asuc$
        \item $A \in \Asuc \iff A^C \in \Asuc$
        \item Si $\lc A_i\rc _{i\ge1}$ és una família en $\Asuc$, aleshores $\bigcup_{i\ge1}{A_i} \in \Asuc$.
    \end{enumerate}
    Si $\mu$ és una mesura sobre l'espai de mesura ($\lp\Omega,\Asuc,\mu\rp$ és un espai mesurable):
    \begin{enumerate}
        \item $\mu(\emptyset) = 0$
        \item $\forall A \in \Asuc,\quad \mu(A) \ge0$
        \item ($\sigma$-additivitat) Si $\lc A_i\rc_{i\ge 1}$ és una família en $\Asuc$ tal que $\forall i \neq j,\quad A_i \cap A_j = \emptyset$,
        aleshores \[\mu\lp\bigcup_{i\ge 1}{A_i}\rp = \sum_{i\ge1}{\mu(A_i)}\]
    \end{enumerate}
\end{obs}

\begin{prop}
    Sigui $(\Omega,\Asuc, p)$ espai de probabilitat. Aleshores:
    \begin{enumerate}
	\item Si $A_1, \dots, A_r \in \Asuc\; \tq \forall i\neq j,\, A_i \cap A_j = \emptyset \implies 
        p\lp \bigcap\limits_{i=1}^{r} A_i \rp= \sum\limits_{i=1}^{r} p\lp A_i \rp $
        \item \label{item:esp_prob_2}$A \in \Asuc \implies p(\overline{A})=1-p(A)$
        \item \label{item:esp_prob_3}$A,B\in \Asuc \tq A\subseteq B,\, p\lp B\setminus A \rp = p\lp B\rp - p\lp A\rp$
        \item $A,B \in \Asuc, A \subseteq B \implies p(A) \le p(B)$
        \item Succecions monòtones:
        \begin{enumerate}[a)]
         \item Si $\left\{A_i\right\}_{i\geq 1} \in \Asuc,\, A_1\subseteq A_2 \subseteq \dots \implies 
         p\lp \bigcap\limits_{i\geq 1} A_i\rp = \lim\limits_{i\to\infty} p\lp A_i\rp$
         \item Si $\left\{A_i\right\}_{i\geq 1} \in \Asuc,\, A_1\supseteq A_2 \supseteq \dots \implies 
         p\lp \bigcup\limits_{i\geq 1} A_i\rp = \lim\limits_{i\to\infty} p\lp A_i\rp$.
        \end{enumerate}
    \end{enumerate}
\end{prop}
\begin{proof}
    \begin{enumerate}
     \item[]
     \item Conseqüència directa de la $\sigma$-additivitat.
     \item Conseqüència diecta de \ref{item:esp_prob_2} usant que $\Asuc = A\cap A^C$.
     \item Com $A\subseteq B,\, B=\lp B\setminus A\rp \cap A$ i per tant $p\lp B\setminus A \rp = p\lp B\rp - p\lp A\rp$.
     \item Conseqüència directa de \ref{item:esp_prob_3} ja que $p\lp B\setminus A\rp\geq 0$.
     \item 
     \begin{enumerate}[a)]
       \item[]
       \item Sigui $B_1=A_1$ i per $i>0$ sigui $B_i = A_i\setminus A_{i-1}$. Aleshores es compleix que
       \begin{gather*}
        \begin{rcases}
	  \forall i\neq j \tq B_i \cup \B_j =\emptyset \\
	  \bigcap\limits_{i\geq 1} B_i = \bigcap\limits_{i\geq 1} A_i
        \end{rcases}
        \implies p\lp\bigcap\limits_{i\geq 1} A_i\rp = p\lp\bigcap\limits_{i\geq 1} B_i\rp \\
        = \sum\limits_{i\geq 1} p\lp B_i\rp = \lim\limits_{N\to\infty} p\lp B_i\rp = \lim\limits_{N\to\infty} p\lp\bigcap\limits_{i=1}^N B_i\rp = 
        \lim\limits_{N\to\infty} p\lp\bigcap\limits_{i=1}^N A_i\rp.
       \end{gather*}
      \item Anàleg al cas anterior.
     \end{enumerate}
    \end{enumerate}
\end{proof}
