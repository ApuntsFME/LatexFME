\chapter{Espai de probabilitat}

\section{Definició axiomàtica de probabilitat}

\begin{defi}[espai!de probabilitat]\index{espai!mostral((tlab))}\index{conjunt!d'esdeveniments((tlab))}\index{conjunt!de successos((tlab))}\index{funció!de probabilitat((tlab))}
    Un espai de probabilitat és un espai de mesura $(\Omega,\Asuc, p)$ tal que $p(\Omega)=1$. Diem que 
    \begin{itemize}
        \item $\Omega$ és l'espai mostral,
        \item $\Asuc$ és el conjunt d'esdeveniments o de successos,
        \item $p$ és la funció de probabilitat.
    \end{itemize}
\end{defi}

\begin{obs}
    Recordem que $\lp\Omega,\Asuc\rp$ és un espai mesurable si $\Asuc\subseteq \Pa\lp\Omega\rp$ és una $\sigma$-àlgebra d'$\Omega$, és a dir,
    \begin{enumerate}[i)]
        \item $\emptyset \in \Asuc$,
        \item $A \in \Asuc \iff \comp{A} \in \Asuc$,
        \item Si $\lc A_i\rc _{i\in\n}\subseteq \Asuc$, aleshores $\bigcup_{i\in\n}{A_i} \in \Asuc$.
    \end{enumerate}
    I que $\lp\Omega,\Asuc,\mu\rp$ és un espai de mesura si $\mu$ és una mesura sobre l'espai mesurable $\lp\Omega,\Asuc\rp$, és a dir,
    \begin{enumerate}[i)]
        \item $\mu(\emptyset) = 0$,
        \item $\forall A \in \Asuc,\quad \mu(A) \ge0$,
        \item ($\sigma$-additivitat) Si $\lc A_i\rc_{i\in\n}\subseteq\Asuc$ és tal que $\forall i \neq j, \, A_i \cap A_j = \emptyset$,
        aleshores 
        \[
            \mu\lp\bigcup_{i\in\n}{A_i}\rp = \sum_{i\in\n}{\mu(A_i)}.
        \]
    \end{enumerate}
\end{obs}

\begin{prop}
    Sigui $(\Omega,\Asuc, p)$ un espai de probabilitat. Aleshores,
    \begin{enumerate}[i)]
        \item Si $A_1, \dots, A_r \in \Asuc$ són tals que $\forall i\neq j,\, A_i \cup A_j = \emptyset,$ aleshores 
	  $p\lp \bigcap\limits_{i=1}^{r} A_i \rp= \sum\limits_{i=1}^{r} p\lp A_i \rp$.
        \item \label{item:esp_prob_2}$A \in \Asuc \implies p(\comp{A})=1-p(A)$.
        \item \label{item:esp_prob_3}$A,B \in \Asuc, A \subseteq B \implies p\lp B\setminus A \rp = p\lp B\rp - p\lp A\rp$.
        \item $A,B \in \Asuc, A \subseteq B \implies p(A) \le p(B)$.
        \item \label{item:esp_prob_5}Successions monòtones:
        \begin{enumerate}[a)]
         \item Si $\left\{A_i\right\}_{i\in\n} \subseteq \Asuc$ són tals que $A_i\subseteq A_{i+1}$, aleshores $p\lp \bigcup\limits_{i\in\n} A_i\rp = \lim\limits_{i\to\infty} p\lp A_i\rp$.
         \item Si $\left\{A_i\right\}_{i\in\n} \subseteq \Asuc$ són tals que $A_i\supseteq A_{i+1}$, aleshores $p\lp \bigcap\limits_{i\in\n} A_i\rp = \lim\limits_{i\to\infty} p\lp A_i\rp$.
        \end{enumerate}
    \end{enumerate}
\end{prop}
\begin{proof}
    \begin{enumerate}
        \item[]
        \item Conseqüència directa de la $\sigma$-additivitat.
        \item Conseqüència diecta de \ref{item:esp_prob_2} usant que $\Asuc = A\cup \comp{A}$.
        \item Com que $A\subseteq B,\, B=\lp B\setminus A\rp \cup A$ i, per tant, $p\lp B\setminus A \rp = p\lp B\rp - p\lp A\rp$.
        \item Conseqüència directa de \ref{item:esp_prob_3} ja que $p\lp B\setminus A\rp\geq 0$.
        \item 
        \begin{enumerate}[a)]
            \item[]
            \item Sigui $B_0=A_0$ i per $i>0$ sigui $B_i = A_i\setminus A_{i-1}$. Aleshores, es compleix que $\forall i\neq j, \, B_i \cap B_j =\emptyset$ i que $\bigcup\limits_{i\in \n} B_i = \bigcup\limits_{i\in\n} A_i$, de manera que
            \begin{gather*}
                p\lp\bigcup\limits_{i\in\n} A_i\rp = p\lp\bigcup\limits_{i\in\n} B_i\rp = \sum\limits_{i\in\n} p\lp B_i\rp =\\
                = \lim\limits_{N\to\infty} \sum_{i=0}^N p\lp B_i\rp = \lim\limits_{N\to\infty} p\lp \bigcup_{i=0}^N B_i\rp = \lim\limits_{N\to\infty} p\lp A_N\rp.
            \end{gather*}
            \item Anàleg al cas anterior.
        \end{enumerate}
    \end{enumerate}
\end{proof}

Observem que l'apartat \ref{item:esp_prob_5} només es pot aplicar en casos molt particulars. En general, si tenim $A_i,\dots,A_r$ succcessos,
hi ha estimacions per a $p(\bigcup_{i=1}^{r}{A_i})$:

\begin{specialteo}[Desigualtats de Bonferroni]
    Siguin $A_1,\dots,A_r\in\Asuc$, i per $I\subseteq\{1,\dots,r\}$ sigui $A_I = \bigcap\limits_{i \in I}{A_i}$. Definim
    \[
        S_k = \sum_{I \in \{1,\dots,n\},\#I=k}{p(A_I)},
    \]
    això és, $S_1 = \sum{p(A_i)}$, $S_2 = \sum_{i \neq j}p\lp A_i \cap A_j\rp,\dots$. Aleshores:
    \begin{enumerate}[i)]
         \item Si $t$ és parell,
            \[p\lp\bigcup_{i=1}^{r}{A_i}\rp \geq \sum_{i=1}^{r}{(-1)^{i+1}S_i}\]
         \item Si $t$ és senar,
            \[p\lp\bigcup_{i=1}^{r}{A_i}\rp \leq \sum_{i=1}^{r}{(-1)^{i+1}S_i}\]
    \end{enumerate}
\end{specialteo}

\begin{obs}
    Amb els casos $t=1$ (desigualtat de Boole) i $t=2$ es poden donar fites inferiors i superiors.
\end{obs}

\begin{example}
    \begin{enumerate}
     \item []
     \item Espais de probabilitat numerables.
     
     Prenem $\Omega$ un conjunt numerable $\Omega =\left\{a_i\right\}_{i\geq 1}$. Prenem $\Asuc = \Pa\lp \Omega\rp$
     (que és una $\sigma$-àlgebra). Per a definir la probabilitat sobre $\lp\Omega,\,\Asuc\rp$ prenem una successió 
     $\left\{p_i\right\}_{i\geq 1} \tq 0\geq p_i\geq 1$ que cumpleix que $\forall i,\, p\lp a_i\rp = p_i$ i 
     $\sum\limits_{i\geq1}p_i=1$. Per tant, per a qualsevol element $A\in\Asuc$, tenim que
     \[
	p\lp\bigcup\limits_{a\in A} \left\{a\right\}\rp = p\lp A\rp = \sum\limits_{i\geq 1} p\lp \left\{a\right\}\rp.
     \]
     Si, a més, $|\Omega|<+\infty$, $\Asuc = \Pa\lp\Omega\rp$ té $2^{|\Omega|}$ elements i si premnem 
     $\Omega =\left\{a_i\right\}_{i=1}^{N}$ i $p_1=p_2=\dots=p_N = \frac{1}{N}$ obtenim un espai clàssic de probabilitat.
     
     \item Espai de probabilitat en $\left[a,b\right]\subseteq \real$.
     
      Sigui $\Omega=\left[a,b\right]$ i prenem $\Asuc=\B\cap\left[a,b\right]$ amb $\B$ un borelià i com a funció de probabilitat $p=\frac{\lambda}{b-a}$, on $\lambda$
      és la mesura de Lebesgue. Observem que no podem prendre tot $\real$ perquè no podem normalitzar $\lambda\lp\real\rp$. Malgrat això, usant $\lambda$ construirem
      més endavant funcions de probabilitat sobre $\lp\real,\,\B\rp$.
      
     \item Tirada indefinida d'una moneda.
     
     En aquest cas tenim que $\Omega = \left\{a_i\right\}_{i\geq1}$, $a_i \in \left\{0,1\right\}$ de la forma
     \begin{gather*}
	00010001110110\dots \\
	01001110101101\dots \\
	10010111110010\dots 
     \end{gather*}
      sent $0$ creu i $1$ cara. Aquest conjunt és no numerable fàcilment demostrable amb l'argument de la diagonal de Cantor.
      Per a construir una $\sigma$-àlgebra sobre $\Omega$ trobem una ``bijecció'' amb $\left[0,1\right]$ de la forma
      \[
        \begin{aligned}
            \varphi\colon\Omega &\to [0,1]\subseteq\real \\
            a=a_1a_2\dots a_n &\mapsto 0.a_1a_2\dots a_n
        \end{aligned}.
      \]
      No és una bijecció completa ja que hi ha elements diferrents que van a la mateixa imatge degut als nombres que acaben en $1$ periòdic, però al ser tots racionals,
      el conjunt d'aquests nombres és numerable i per tant té mesura nul·la. És per això que podem definir una $\sigma$-àlgebra sobre $\Omega$ prenent 
      $\left\{\varphi^{-1}\lp A\rp \right\}_{A\subseteq \B\cup[0,1]}$. Similarment ho fem amb la mesura.
    \end{enumerate}
\end{example}


\section{Probabilitat condicionada}

\begin{defi}[probabilitat condicionada]
    Sigui $\lp \Omega, \Asuc, p\rp$ un espai de probabilitat i siguin $A, B \in \Asuc$. Definim la probabilitat d'$A$ condicionada a $B$ com
    \[
        p\lp A\mid B\rp = \frac{p\lp A\cap B\rp}{p\lp B\rp}.
    \]
\end{defi}

\begin{obs}
    Sigui $\lp \Omega, \Asuc, p\rp$ un espai de probabilitat i sigui $B \in \Asuc$ tal que $p\lp B\rp > 0$. Aleshores, l'aplicació
    \begin{align*}
        p_B\colon \Asuc &\to \real \\
        A &\mapsto p_B\lp A\rp := p\lp A\mid B\rp
    \end{align*}
    defineix un espai de probabilitat $\lp \Omega, \Asuc, p_B\rp$.
\end{obs}

\begin{prop}
    Sigui $I$ un conjunt numerable o finit i siguin $\lc A_i \rc_{i\in I} \subseteq \Asuc$ tals que 
    \begin{enumerate}[i)]
        \item $p\lp A_i\rp>0$,
        \item $i\neq j \implies A_i \cap A_j = \varnothing$,
        \item $\bigcup\limits_{i\in I} A_i = \Omega$.
    \end{enumerate}
    Aleshores,
    \begin{enumerate}[1)]
        \item Probabilitat total:
            \[
                p\lp B\rp=\sum_{i\in I} p\lp B\mid A_i\rp p\lp A_i\rp, \quad \forall B\in \Asuc.
            \]
        \item Fórmula de Bayes:
            \[
                p\lp A_i\mid B\rp=\frac{P\lp B\mid A_i\rp p\lp A_i\rp}{\sum_{j\in I} p\lp B\mid A_j\rp p\lp A_j\rp}, \quad \forall B\in \Asuc \text{ amb } p\lp B\rp>0.
            \]
    \end{enumerate}
\end{prop}
\begin{proof}
    \begin{enumerate}[1)]
        \item[]
        \item Com que els $A_i$ són disjunts i $\bigcup_{i \in I}{A_i} = \Omega$, $\forall B \in \Asuc$,
        $B = \bigcup_{i\in I}{B \cap A_i}$, i la unió és disjunta. Es té
        \[
            p(B) = p\lp\bigcup_{i\in I}{B \cap A_i}\rp \stackrel{\sigma-add.}{=} \sum_{i\in I}{p(B \cap A_i)} =
            \sum_{i\in I}{p(B|A_i)p(A_i)}.
        \]
        \item
        \begin{gather*}
            p(A_i|B) \sum_{j \in I}{p(B|A_j)p(A_j)} \stackrel{i)}{=} p(A_i|B)p(B) =\\
            \frac{p(B\cap A_i)}{p(B)}p(B) = p(B \cap A_i) = P\lp B\mid A_i\rp p\lp A_i\rp.
        \end{gather*}
    \end{enumerate}
\end{proof}

\begin{problema}[Ruïna del jugador]
    Partim d'un capital de $k$ unitats i, en cada jugada (sense memòria) augmenta o disminueix el capital en una unitat,
    amb probabilitats 1/2 i 1/2. El joc acaba si ens quedem sense capital o si assolim un objectiu $N$ ($N>k$).
    Quina és la probabilitat de perdre tot el capital?
\end{problema}
\begin{sol}
    Sigui $A_k$ el succés ``el jugador, començant amb capital $k$, perd''.Condicionem $A_k$ a la primera tirada de la moneda, 
    definim $B$: ``la primera tirada surt cara''.
    \begin{gather*}
      p(A_k) = p(A_k|B)p(B) + p(A_k|\comp{B})p(\comp{B}) = p(A_k|B)\frac{1}{2} + p(A_k|\comp{B})\frac{1}{2}\\
      \implies 2p(A_k)=p(A_{k-1}) + p(A_{k+1}) \implies p(A_k) - p(A_{k-1}) = p(A_{k+1}) - p(A_k) = C,
    \end{gather*}

    el que ens diu que la diferència entre nivells és constant. Per tant $p(A_k) = p(A_0)+kC$. Sabent que $p(A_0)=1$ i $p(A_N)=0$ ens queda que
    \[0 = 1 + CN \implies C = -\frac{1}{n} \implies p(A_k) = 1 - \frac{k}{N}.\]
\end{sol}


\section{Independència}

\begin{defi}[esdeveniments independents]
    Sigui $\lp \Omega, \Asuc, p\rp$ un espai de probabilitat, sigui $I$ un conjunt finit o numerable i sigui $\lc A_i\rc_{i\in I} \subseteq \Asuc$. Diem que els esdeveniments $A_i$ són independents si per tot $J\subseteq I$ amb $\abs{J}\in\n$ es té que
    \[
        p\lp\bigcap_{j\in J} A_j\rp = \prod_{j\in J} p\lp A_j\rp.
    \]
\end{defi}

\begin{example}
    \begin{enumerate}[1.]
        \item[]
        \item $\varnothing, \Omega$ són independents entre si.
        \item $A$ és independent amb si mateix si i només si $p\lp A\rp=1$ o $p\lp A\rp =0$.
        %TODO
    \end{enumerate}
\end{example}

\section{Espai producte}

Donats dos espais de probabilitat $\lp\Omega_1,\Asuc_1,p_1\rp$ i $\lp\Omega_2,\Asuc_2,p_2\rp$, volem construir un nou espai de probabilitat $\lp\Omega_3,\Asuc_3,p_3\rp$
que codifiqui els dos espais de probabilitat inicials. A aquest espai de probabilitat l'anomenarem espai de probabilitat producte.

\begin{defi}[espai!de probabilitat!producte]
  Siguin $\lp\Omega_1,\Asuc_1,p_1\rp$ i $\lp\Omega_2,\Asuc_2,p_2\rp$ dos espais de probabilitat. Anomenem espai de probabilitat producte a la terna $\lp\Omega_3,\Asuc_3,p_3\rp$
  tal que
  \begin{enumerate}[i)]
   \item $\Omega_3=\Omega_1\times\Omega_2$
   \item $\Asuc_3 = \sigma\lp\Asuc_1\times\Asuc_2\rp$ ($\sigma$-àlgebra generada per $\Asuc_1\times\Asuc_2$)
   \item \label{item:espai_prod_3}$p_3$ és una funció de probabilitat que cumpleix que $\forall A_1, A_2 \tq A_1\times A_2\in\Asuc_1\times\Asuc_2$ aleshores 
   $p_3\lp A_1\times A_2\rp = p_1\lp A_1\rp p_2\lp A_2\rp$.
  \end{enumerate}
\end{defi}

\begin{obs}
  $p_3$ està ben definida ja que pel Teorema d'extensió de Carathéodory podem construir una $\sigma$-àlgebra sobre $\Omega_1\times\Omega_2$ a partir d'una 
  extensió de $\sigma\lp\Asuc_1\times\Asuc_2\rp$ i restringir $p_3$ segons \ref{item:espai_prod_3}.
\end{obs}

\begin{obs}
  Podem extendre $\lambda$ (la mesura de Lebesgue) a $\real^2$ de la següent forma. Sabem que $\lp[0,1], \B\cap[0,1],\lambda_{[0,1]}\rp$ és un espai de probabilitat.
  Aleshores \[\lp [0,1]\times[0,1], \sigma\lp\B\cap[0,1]\times\B\cap[0,1]\rp, \lambda_{[0,1]\times[0,1]}\rp\] defineix un espai de probabilitat a $\real^2$.
\end{obs}

\begin{problema}[Agulla de Buffon]
  Considerem el pla $\real^2$ tesel·lat amb linies paral·leles indefinides separades per una distància $L$. Llancem una agulla de longitud $l\leq L$ sobre el pla.
  Trobar quina és la probabilitat que l'agulla toqui una de les linies.
\end{problema}
\begin{sol}
  Considerarem dues variables: $x$ com la distància del centre de l'agulla a la linia més propera i $\theta$ com l'angle de l'agulla amb la direcció de les lines.
  Tenim que $x\in\left[0,\frac{L}{2}\right]$ i $\theta\in[0,\pi)$ i per tant, $\Omega =\left[0,\frac{L}{2}\right]\times[0,\pi)$, $\Asuc$ són els borelians 
  del conjunt i $p$ la mesura de Lebesgue normalitzada en $\Asuc$. Sigui $A\in\Asuc$ l'esdeveniment ``l'agulla talla una recta'' i $\omega\in\Omega$ una tirada.
  Aleshores $w\in A \iff x\leq \frac{l}{2}\sin\theta$. Per tant,
  \[
   p\lp A\rp = \frac{\int_0^{\pi}\frac{l}{2}\sin\theta \dif\theta}{\frac{L\pi}{2}}=\frac{2l}{L\pi}.
  \]
\end{sol}

\section{Lema de Borel-Cantelli}
Siguin $\lp\Omega,\Asuc ,p\rp$ un espai de probabilitat i $\left\{ A_n\right\}_{n\geq1} \subseteq \Asuc$. Volem donar-li 
un sentit a ``límit de $\left\{ A_n\right\}_{n\geq1}$''. Farem com a $\real$ i definirem els límits superior i inferior 
(que sempre existiran) i, si coincideixen, aquest serà el límit.

\begin{defi}[límit!superior d'esdeveniments]\index{límit!inferior d'esdeveniments((tlab))}
  Sigui $\lp\Omega,\Asuc,p\rp$ un espai de probabilitat. Donats $\left\{ A_n\right\}_{n\geq1} \subseteq \Asuc$, definim els límits superior i inferior de la successió de successos $\left\{ A_n\right\}_{n\geq1}$ com
  \begin{gather*}
    \limsup_{n\to\infty} A_n = \bigcap_{n=1}^{\infty}\bigcup_{k=n}^{\infty} A_k ,\\
    \liminf_{n\to\infty} A_n = \bigcup_{n=1}^{\infty}\bigcap_{k=n}^{\infty} A_k.
  \end{gather*}
\end{defi}

\begin{obs}
  Els dos límits pertanyen a $\Asuc$ ja que son unió i intersecció numerable de sucessos.
\end{obs}

\begin{prop}
    Sigui $\lp\Omega,\Asuc,p\rp$ un espai de probabilitat i siguin $\left\{ A_n\right\}_{n\geq1} \subseteq \Asuc$. Aleshores,
    \begin{enumerate}[i)]
        \item $\liminf\limits_{n\to\infty} A_n= \left\{\omega\in\Omega\colon \exists m\equiv m(\omega) \text{ amb } \omega\in A_r \;\forall r\geq m(\omega)\right\}$,
        \item $\limsup\limits_{n\to\infty} A_n= \left\{\omega\in\Omega \colon \omega \text{ pertany a un nombre infinit dels } A_n\right\}$,
        \item $\liminf\limits_{n\to\infty} A_n \subseteq \limsup\limits_{n\to\infty} A_n$.
    \end{enumerate}
\end{prop}
\begin{proof}
    \begin{enumerate}[i)]
        \item[]
        \item $\omega\in\liminf{A_n} \iff \omega\in\bigcup_{n=1}^{\infty}\bigcap_{k=n}^{\infty} A_k \iff
        \exists m \equiv m(\omega) \tq \omega\in\bigcap_{k=m(\omega)}^{\infty} A_k \iff
        \omega\in A_r \quad\forall r \ge m(\omega).$
        \item $\omega\in\limsup{A_n} \iff \omega\in\bigcap_{n=1}^{\infty}\bigcup_{k=n}^{\infty} A_k \iff
        \omega\in\bigcup_{k=n}^{\infty}A_k \quad\forall n \iff \forall n,\, \exists n_0\geq n \tq \omega\in A_{n_0}\iff \omega$ pertany a un nombre infinit dels $A_n$. 
        \item Si $\omega\in\liminf A_n$, aleshores $\omega\in A_r, \, \forall r\geq m\lp\omega\rp$, de manera que pertany a un nombre infinit dels $A_n$ i, en conseqüència, pertany a $\limsup A_n$.
    \end{enumerate}
\end{proof}

\begin{prop}
    Sigui $\lp\Omega,\Asuc,p\rp$ un espai de probabilitat i siguin $\left\{ A_n\right\}_{n\geq1} \subseteq \Asuc$, amb $\lim A_n=A$. Aleshores, $p\lp A\rp = p\lp \lim A_n\rp = \lim p \lp A_n\rp$ i aquest límit existeix.
\end{prop}
\begin{proof}
    Definim $B_n=\cup_{k\geq n}A_k$ i $C_n=\cap_{k\geq n}A_k$. Observem que $\lc B_n\rc_{n\geq 1}$ és decreixent i que $\lc B_n\rc_{n\geq 1}$ és creixent. Naturalment, $C_n\subseteq A_n\subseteq B_n$, $\limsup A_n = \cap_{n\geq 1} B_n$ i $\liminf A_n = \cup_{n\geq 1} C_n$.
    
    \noindent Vegem que $p\lp\liminf A_n\rp\leq\liminf p\lp A_n\rp$.
    \[
        p\lp\liminf A_n\rp=p\lp\bigcup_{n\geq 1} C_n\rp = \lim p\lp C_n\rp=\lim p\lp \bigcap_{k\geq n} A_k\rp\leq\liminf p\lp A_n\rp.
    \]
    Al darrer pas hem utilitzat el fet que $p\lp \cap_{k\geq n} A_k\rp\leq p\lp A_n\rp$. Anàlogament, $\limsup p\lp A_n\rp\leq p\lp\limsup A_n\rp$. Així doncs, tenim que
    \[
        p\lp\liminf A_n\rp\leq\liminf p\lp A_n\rp\leq\limsup p\lp A_n\rp\leq p\lp\limsup A_n\rp.
    \]
    Atàs que $p\lp\liminf A_n\rp=p\lp\limsup A_n\rp=p\lp A\rp$, concloem que 
    \[
        \liminf p\lp A_n\rp=\limsup p\lp A_n\rp=\lim p\lp A_n\rp=p\lp A\rp.
    \]
\end{proof}
\begin{teolema}[de Borel-Cantelli]
    Sigui $\lp\Omega,\Asuc,p\rp$ un espai de probabilitat i siguin $\left\{ A_n\right\}_{n\geq1} \subseteq \Asuc$. Aleshores,
    \begin{enumerate}[i)]
        \item $\sum\limits_{n\geq 1} p\lp A_n\rp<\infty\implies p\lp\limsup A_n\rp=0$.
        \item Si $\left\{ A_n\right\}_{n\geq1}$ és independent, $\sum\limits_{n\geq 1} p\lp A_n\rp=\infty\implies p\lp\limsup A_n\rp=1$.
    \end{enumerate}
\end{teolema}
\begin{proof}
    Posem $A=\limsup A_n=\bigcap\limits_{n\geq 1}\bigcup\limits_{k\geq n} A_k$.
    \begin{enumerate}[i)]
        \item Sabem que
            \[
                0\leq p\lp A\rp\leq p\lp\bigcup_{k\geq n}A_k\rp\leq\sum_{k\geq n} p\lp A_k\rp ,\,\forall n\in\n
            \]
            i que $\sum_{n\geq 1} p\lp A_n\rp<\infty$, de manera que $\lim \sum_{k\geq n} p\lp A_k\rp=0$ i immediatament deduïm que $p\lp A\rp=0$.
        \item Observem primer que $\comp{A}=\comp{\bigcap\limits_{n\geq 1}\bigcup\limits_{k\geq n} A_k}=\bigcup\limits_{n\geq 1}\bigcap\limits_{k\geq n} \comp{A_k}=\liminf\comp{A_n}$. Veurem que $p\lp \comp{A}\rp=0$. Calculem $p\lp \bigcap\limits_{m\geq n} \comp{A_m}\rp$.
        \begin{align*}
            0&\leq p\lp \bigcap\limits_{m\geq n} \comp{A_m}\rp = \lim_{r\to\infty} p\lp \bigcap\limits_{m= n}^r \comp{A_m}\rp = \lim_{r\to\infty} \prod_{m= n}^r \lp p\lp\comp{A_m}\rp\rp =\\
            &= \lim_{r\to\infty} \prod_{m= n}^r \lp 1-p\lp A_m\rp\rp \leq \lim_{r\to\infty} \prod_{m= n}^r \lp e^{-p\lp A_m\rp}\rp =\\
            &=\lim_{r\to\infty} e^{-\sum_{m= n}^r p\lp A_m\rp} = 0,
        \end{align*}
        de manera que $p\lp \bigcap\limits_{m\geq n} \comp{A_m}\rp=0, \,\forall n\in\n$.
        Finalment,
        \[
            0\leq p\lp\comp{A}\rp=p\lp\bigcup\limits_{n\geq 1}\bigcap\limits_{m\geq n} \comp{A_m}\rp\leq\sum_{n\geq 1}p\lp\bigcap\limits_{m\geq n} \comp{A_m}\rp=0+0+\cdots=0,
        \]
        i concloem que $p\lp A\rp=1$.
    \end{enumerate}
\end{proof}
