\chapter{Variables aleatòries}

\section[Definició i propietats bàsiques de les variables aleatòries]
    {Definició i propietats bàsiques de les variables aleatòries
    \sectionmark{Definició i propietats de les variables aleatòries}}
    \sectionmark{Definició i propietats de les variables aleatòries}

\begin{defi}[variable aleatòria]
    Siguin $\lp \Omega_1, \Asuc_1\rp$ i $\lp \Omega_2, \Asuc_2\rp$ espais mesurables. Diem que $X\colon \Omega_1\to\Omega_2$ és una variable aleatòria si
    \[
        X^{-1}\lp A_2\rp\in\Asuc_1,\,\forall A_2\in\Asuc_2.
    \]
\end{defi}
\noindent En aquest curs, sempre pendrem $\lp \Omega_2, \Asuc_2\rp=\lp \real,\B\rp$. Per tant, quan parlem de variable aleatòria ens estarem referint a una aplicació $X\colon\Omega\to\real$ amb $B\in\B\implies X^{-1}\lp B\rp\in\Asuc$, on $\lp \Omega, \Asuc\rp$ és un espai de mesura.
\begin{example}
    \begin{enumerate}[1.]
        \item[]
        \item Sigui $\lp \Omega, \Asuc\rp$ un espai de mesura. Aleshores, $\forall c\in\real$, l'aplicació
            \begin{align*}
                X\colon \Omega &\to\real\\
                \omega &\mapsto c
            \end{align*}
            és una variable aleatòria, atès que $\forall B\in\B$, es té que
            \[
                X^{-1}\lp B\rp=\begin{cases}
                    \Omega, \text{ si } c\in B,\\
                    \varnothing, \text{ si } c\notin B.
                \end{cases}
            \]
        \item Siguin $X$ i $Y$ variables aleatòries. Aleshores, també son variables aleatòries les següents funcions
            \begin{itemize}
                \item $X+Y$
                \item $X-Y$
                \item $aX,\,\forall a\in\real$
                \item $XY$
                \item $\abs{X}$
                \item $\max\lc X, Y\rc$
                \item $\min\lc X, Y\rc$
                \item $X^+$
                \item $X^-$
                \item $g\lp X, Y\rp$, on $g\colon \real^2\to\real$ és una funció mesurable.
            \end{itemize}
        \item Sigui $\lp \Omega, \Asuc\rp$ un espai de mesura i sigui $A\in\Asuc$. Definim la variable aleatòria indicadora d'$A$ com
            \begin{align*}
                \mathbb{I}_A\equiv\mathds{1}_A\colon \Omega &\to\real\\
                \omega &\mapsto\mathbb{I}_A\lp\omega\rp=\begin{cases}
                    0, \text{ si } \omega\notin A,\\
                    1, \text{ si } \omega\in A.
                \end{cases}
            \end{align*}
            Vegem que, efectivament, es tracta d'una variable aleatòria. Sigui $B\in\B$. Aleshores,
            \[
		    \mathbb{I}_A^{-1}\lp B\rp=\begin{cases}
                    \Omega, \text{ si } 0\in B,\, 1\in B,\\
                    \comp{A}, \text{ si } 0\in B,\, 1\notin B,\\
                    A, \text{ si } 0\notin B,\, 1\in B,\\
                    \varnothing, \text{ si } 0\notin B,\, 1\notin B.
                \end{cases}
            \]
    \end{enumerate}
\end{example}

\begin{obs}
    A partir d'ara, emprarem la notació següent. Sigui $(\Omega,\Asuc, p)$ un espai de probabilitat i sigui $B\in\B$, escrivim
    \[
        p\lp X\in B\rp:= p\lp\lc\omega\in\Omega\mid\omega\in X^{-1}\lp B\rp\rc\rp.
    \]
\end{obs}

\begin{example}
    Per a considerar la probabilitat de que $X$ sigui més petit o igual que $2$ escriurem
    \[
      p\lp X\leq 2\rp = p\lp\lc\omega\in\Omega\mid X\lp\omega\rp\leq 2\rc\rp.
    \]
\end{example}

\begin{obs}
    Sigui $\lp\Omega,\Asuc,p\rp$ un espai de probabilitat i sigui $X$ una variable aleatòria. $X$ indueix una funció de probabilitat $P_X$ 
    sobre l'espai de mesura $\lp\real,\B\rp$ de la forma
    \[
        P_X\lp B\rp:=p\lp X\in B\rp.
    \]
    És a dir, $\lp\real,\B,P_x\rp$ és un espai de probabilitat. Comprovem, primer, que és un espai de mesura.
    \begin{enumerate}[i)]
        \item $P_X\lp\varnothing\rp=p\lp\lc\omega\in\Omega\mid\omega\in X^{-1}\lp \varnothing\rp\rc\rp=p\lp\varnothing\rp=0$, atès que $p$ és una funció de probabilitat.
        \item $0\leq p\lp\lc\omega\in\Omega\mid\omega\in X^{-1}\lp B\rp\rc\rp = P_X\lp B\rp$, atès que $p$ és una funció de probabilitat. 
        \item Si $\lc B_i\rc_{i\in\n}\subseteq\B$ són disjunts dos a dos, aleshores $\lc X^{-1}\lp B_i\rp\rc_{i\in\n}\subseteq\Asuc$ també són disjunts dos a dos. I, per ser $p$ una funció de probabilitat, es té que
        \begin{align*}
            P_X\lp \bigcup_{i\in\n} B_i\rp &= p\lp\lc\omega\in\Omega\mid\omega\in X^{-1}\lp \bigcup_{i\in\n} B_i\rp\rc\rp =\\
            &= \sum_{i\in\n} p\lp\lc\omega\in\Omega\mid\omega\in X^{-1}\lp B_i\rp\rc\rp=\\
            &=\sum_{i\in\n}P_X\lp B_i\rp.
        \end{align*}
    \end{enumerate}
    A més a més, per ser $p$ una funció de probabilitat,
    \[
        P_X\lp\real\rp = p\lp\lc\omega\in\Omega\mid\omega\in X^{-1}\lp\real\rp\rc\rp = p\lp\Omega\rp=1
    \]
    i $\lp\real,\B,P_x\rp$ és un espai de probabilitat.
\end{obs}

\begin{obs}
    Sigui $\lp\Omega,\Asuc\rp$ un espai mesurable. Recordem que $X\colon\Omega\to\real$ és una funció mesurable si i només si $X^{-1}\lp\lp-\infty, a\rb\rp\in\Asuc,\,\forall a\in\real$.
\end{obs}

\begin{defi}[funció!de distribució!de probabilitat]
    Sigui $\lp\Omega,\Asuc, p\rp$ un espai de probabilitat i sigui $X$ una variable aleatòria. Anomenem funció de distribució de probabilitat d'$X$ a l'aplicació
    \begin{align*}
        F_X\colon\real &\to\lb 0,1\rb\\
        x &\mapsto F_X\lp x\rp = p\lp X\leq x\rp = P_X\lp\lp-\infty,x\rb\rp.
    \end{align*}
\end{defi}

\begin{prop}
    Sigui $\lp\Omega,\Asuc, p\rp$ un espai de probabilitat i sigui $F_X$ la funció de distribució de probabilitat d'una variable aleatòria $X$ sobre $\lp\Omega,\Asuc, p\rp$. Aleshores,
    \begin{enumerate}[i)]
        \item $x_1\leq x_2\implies F_X\lp x_1\rp \leq F_X\lp x_2\rp$.
        \item $\lim\limits_{x\to-\infty}F_X\lp x\rp = 0$ i $\lim\limits_{x\to\infty}F_X\lp x\rp = 1$.
        \item $F_X$ és contínua per la dreta, és a dir, $\lim\limits_{h\to 0^+}F_X\lp x+h\rp = F_X\lp x\rp$.
    \end{enumerate}
\end{prop}
\begin{proof}
    \begin{enumerate}[i)]
        \item[]
        \item $F_X\lp x_1\rp = p\lp\lc \omega\in\Omega\mid X\lp\omega\rp\leq x_1\rc\rp \leq p\lp\lc \omega\in\Omega\mid X\lp\omega\rp\leq x_2\rc\rp = F_X\lp x_2\rp$, atès que $\lc \omega\in\Omega\mid X\lp\omega\rp\leq x_1\rc \subseteq \lc \omega\in\Omega\mid X\lp\omega\rp\leq x_2\rc$ i que $p$ és una funció mesurable.
        \item Vegem que $\forall\lc x_n\rc_{n\in\n}$ tal que $\lim\limits_{n\to\infty} x_n = -\infty$, es té que $\lim\limits_{n\to\infty}F_X\lp x_n\rp = 0$. Definim $A_n=\lc\omega\in\Omega\mid X\lp\omega\rp\leq x_n\rc$. Tenim que $\varnothing\subseteq\liminf A_n\subseteq\limsup A_n$. A més, $\limsup A_n=\varnothing$ perquè, altrament, hi hauria un nombre infinit de conjunts $A_n$ contenint un $\omega\in\Omega$ determinat. Per tant, 
            \[
                \lim\limits_{n\to\infty} F_X\lp x_n\rp=\lim\limits_{n\to\infty} p\lp A_n\rp = p\lp\lim\limits_{n\to\infty} A_n \rp = p\lp\varnothing\rp=0.
            \]
            Anàlogament, es demostra que $\lim\limits_{x\to\infty}F_X\lp x\rp = 1$.
        \item Fixat $x$, volem veure que $\lim\limits_{h \to 0^+} F_X \lp x+h\rp =F_X \lp x\rp$. 
        
        Prenem $C_n=\left\{\omega\in\Omega\mid X(\omega) \leq x + h_n\right\}$,
	 on $\left\{h_n\right\}$ és una successió de reals no negatius amb límit zero. Aleshores, $\liminf C_n =\limsup C_n = \lc \omega\in\Omega\mid X(\omega)\leq x\rc$.
	 Això ens diu que
	 \[
	  \lim\limits_{n\to \infty} F_X\lp x+h_n\rp = \lim\limits_{n\to\infty} p\lp C_n\rp = p\lp\lim\limits_{n\to\infty} C_n \rp = p\lp C\rp =F_X (x).
	 \]
	 Com això és cert $\forall h \tq \lc h_n\rc \to 0$, tenim que $\lim\limits_{h\to 0^+} F_X\lp x+h\rp = F_X (x)$.
    \end{enumerate}
\end{proof}

\begin{obs}
  En general no podem assegurar que sigui contínua per l'esquerra. Fent la mateixa prova prenent $x-h_n$ amb $h_n\to 0^+$ en comptes de $x+h_n$, obtenim que
  $C=\lc \omega\in\Omega\mid X(\omega)<x\rc$ i, per tant
  \[
   \lim\limits_{h\to 0^-} F_X\lp x+h\rp = p\lp X<x\rp = F_X(x)-p\lp X=x\rp.
  \]
\end{obs}

\begin{lema}
 Sigui $f\colon \real \to \real$ una funció creixent i fitada. Aleshores $f$ és mesurable Lebesgue.
\end{lema}
\begin{proof}
    Suposem que $f$ té un nombre no numerable de discontinuïats. Observem que totes les discontinuïtats són de salt. Sigui $D\subseteq \real$ el conjunt de punts on $f$ és discontínua. Aleshores, tenim que, per tots els punts $x_d\in D$, existeixen els límits $\lim_{x\to x_d^+}f\lp x\rp$ i $\lim_{x\to x_d^-}f\lp x\rp$. Definim, per tot $n\in\n$, els conjunts
    \[
        A_n=\lc x_d\in D\mid \frac{1}{n+1}\leq\lim_{x\to x_d^+}f\lp x\rp-\lim_{x\to x_d^-}f\lp x\rp<\frac{1}{n}\rc,
    \]
    on cometem l'abús de notació $\frac{1}{0}=\infty$. Com que $D$ és no numerable, hi ha un nombre numerable de conjunts $A_n$ i $\cup_{n\in\n}A_n=D$, necessàriament $\exists n\in\n$ tal que $\abs{A_n}\notin\n$. Per tant, hi ha un nombre infinit de salts de, com a mínim $\frac{1}{n+1}$, la qual cosa contradiu la hipòtesi que $f$ és fitada. Per tant, $f$ té un nombre numerable de discontinuïtats i és, doncs, mesurable.
\end{proof}

\begin{teo}[Teorema de l'existència d'una funció de distribució]
  Sigui $F\colon \real \to [0,1]$ una funció de probabilitat tal que
  \begin{enumerate}[i)]
       \item $x_1\leq x_2\implies F\lp x_1\rp \leq F\lp x_2\rp$.
       \item $\lim\limits_{x\to-\infty}F\lp x\rp = 0$ i $\lim\limits_{x\to\infty}F\lp x\rp = 1$.
       \item $F$ és contínua per la dreta, és a dir, $\lim\limits_{h\to 0^+}F\lp x+h\rp = F\lp x\rp$.
   \end{enumerate}
   Aleshores, existeixen un espai de probabilitat $\lp \Omega,\Asuc,p\rp$ i una variàble aleatòria $X\colon\Omega\to\real$ tals que $F_X(x)=F(x)$.
\end{teo}
\begin{proof}
    Prenem $\lp\Omega,\Asuc,p\rp=\lp\lb 0,1\rb,\B\cap\lb 0,1\rb,\lambda_{\lb 0,1\rb}\rp$ i definim
    \begin{align*}
        X\colon\lb 0,1\rb &\to\lb 0,1\rb\\
        \omega &\mapsto X\lp\omega\rp=\sup\lc y\in\real\mid F\lp y\rp\leq\omega\rc.
    \end{align*}
    Observem que a tots els punts on $F$ és contínua $X$ també ho és, de manera que $X$ és una funció mesurable. Vegem que $F_X\lp x\rp=F\lp x\rp,\,\forall x\in\real$. Donat $x\in\real$, definim els conjunts
    \begin{align*}
        A&=\lc\omega\in\lb 0,1\rb\mid X\lp\omega\rp\leq x\rc,\\
        B&=\lc\omega\in\lb 0,1\rb\mid \omega\leq F\lp x\rp\rc
    \end{align*}
    i observem que
    \begin{align*}
        P\lp A\rp&=P\lp X\leq x\rp = F_X\lp x\rp,\\
        P\lp B\rp&=\lambda\lp\lb 0,F\lp x\rp\rb\rp=F\lp x\rp.\\
    \end{align*}
    Si demostrem que $A=B$, haurem acabat. Però tenim que
    \begin{itemize}
        \item $\omega\in B\implies\omega\leq F\lp x\rp\implies x\notin\lc y\in\real\mid F\lp y\rp<\omega\rc\implies x\geq X\lp\omega\rp$$\implies\omega\in A$.
        \item $\omega\notin B$$\implies\omega>F\lp x\rp\implies\exists\varepsilon>0\tq\omega>F\lp x+\varepsilon\rp$$\implies x\lp\omega\rp\geq x+\varepsilon>x\implies X\lp\omega\rp>x\implies\omega\notin A.$
    \end{itemize}
    
\end{proof}

%TODO Aqui comença el 3/10. No se si hi falta res d'abans
\section[Esperança d'una variable aleatòria. Desigualtats de Markov i Chebyshev]
    {Esperança d'una variable aleatòria. Desigualtats de Markov i Chebyshev
    \sectionmark{Esperança d'una variable aleatòria}}
    \sectionmark{Esperança d'una variable aleatòria}

\begin{defi}[esperança d'una variable aleatòria]
    Sigui $(\Omega,\Asuc,p)$ un espai de probabilitat i sigui $X\colon\Omega\to\real$ una variable aleatòria. Com ja sabem, $X$ indueix una probabilitat $P_X$ sobre $\lp\real,\B\rp$. Definim l'esperança de la variable aleatòria d'$X$, $\esp[X]$ com
    \[
        \esp[X] = \int_{\Omega}{X \dif p} = \int_{\real}{x \dif P_X},
    \]
    si existeix aquesta integral.
\end{defi}

\begin{obs}
    La demostració que aquestes dues integrals són iguals resulta de l'aplicació de la definició de la integral de Lebesgue, però escapa dels objectius d'aquest curs i no l'escriurem.
\end{obs}

\begin{obs}
    Igual que es va veure al curs de teoria de la mesura, pot ser que $\esp[X]$ no existeixi o que sigui infinita.
    No obstant això, atès que $|\int_{\Omega}{f\dif p}| \le \int_{\Omega}{|f|\dif p}$, sovint demanarem que
    $\esp[|X|] \le +\infty$ per poder afirmar que $\esp[X] \le +\infty$.
\end{obs}

%TODO aquest exemple el farà bé el proper dia
\begin{example}
    Sigui $\lp \Omega, \Asuc, p\rp$ un espai de probabilitat i sigui $A\in\Asuc$. Considerem la variable aleatòria
    \[
        X(\omega) = \mathbb{I}_A(\omega) = \begin{cases}
        1&\text{ si } \omega\in A, \\
        0&\text{ si } \omega \notin A.
        \end{cases}
    \]
    Observem que, donat $B\in\B$,
    \[
        P_{\mathbb{I}_A}\lp B\rp = \begin{cases}
            1 &\text{ si } 0,1\in B,\\
            p\lp A\rp &\text{ si } 0\notin B,\, 1\in B,\\
            p\,( \comp{A}) &\text{ si } 0\in B,\, 1\notin B,\\
            0 & \text{ si } 0,1\notin B.
        \end{cases}
    \]
    Així, podem calcular l'esperança amb les dues integrals i comprovar que el resultat és el matex.
    \[
        \esp[\mathbb{I}_A] = \int_\Omega{\mathbb{I}_A \dif p} = 1\cdot p\,(A) + 0\cdot p\,(\comp{A}) = p\,(A),
    \]
    \[
        \esp[\mathbb{I}_A] = \int_\real{x\dif P_{\mathbb{I}_A}} = 1\cdot P_{\mathbb{I}_A}\lp\lc 1\rc\rp + 0\cdot P_{\mathbb{I}_A}\lp\lc 0\rc\rp+ \int_{\real\setminus\lc 0,1\rc} x\dif P_{\mathbb{I}_A} = P_{\mathbb{I}_A}\lp\lc 1\rc\rp = p\lp A\rp.
    \]
\end{example}


\begin{prop}
    Sigui $f\colon\real\to\real$ una funció mesurable i sigui $X$ una variable aleatòria. Aleshores, $f(X)$ és una variable aleatòria i
    \[
        \esp[f(X)] = \int_{\Omega}{f(X) \dif p} = \int_\real{x \dif P_{f\lp X\rp}}.
    \]
\end{prop}
\begin{proof}
    Si $f$ és mesurable, aleshores $f\lp X\rp$ també, de manera que $f\lp X\rp$ és una varible aleatòria i la resta segueix de la definició.
\end{proof}

\begin{defi}[moment!d'ordre $r$]\index{moment!factorial d'ordre $r$((tlab))}\index{variància((tlab))}\index{desviació típica((tlab))}
    Sigui $\lp\Omega,\Asuc,p\rp$ un espai de probabilitat i sigui $X\colon\Omega\to\real$ una variable aleatòria. Aleshores, definim
    \begin{itemize}
        \item Moment d'ordre $r$ d'$X$: 
            \[
                \esp\lb X^r\rb,
            \]
            on $r\in\real$ i hem suposat que $\esp\lb\abs{X}^r\rb<+\infty$.
        \item Moment factorial d'ordre $r$ d'$X$: 
            \[
                \esp\lb \lp X\rp_r\rb=X\lp X-1\rp\cdots\lp X-r+1\rp,
            \]
            on $r\in\n$.
        \item Variància d'$X$:
            \[
                \var\lb X\rb=\esp\lb\lp X-\esp\lb X\rb\rp^2\rb.
            \]
        \item Desviació típica d'$X$:
            \[
                \sigma=\sqrt{\var\lb X\rb}.
            \]
    \end{itemize}
\end{defi}

\begin{prop}
    Siguin $X, Y$ variables aleatòries, siguin $a, b\in\real$ i sigui $A\in\Asuc$. Es tenen les següents propietats de l'esperança.
    \begin{enumerate}[i)]
        \item $\esp\lb a\rb=a$,
        \item $\esp\lb aX+bY\rb=a\esp\lb X\rb+b\esp\lb Y\rb$,
        \item $\esp\lb \mathbb{I}_A\rb=p\lp A\rp$,
        \item $\abs{\esp\lb X\rb}\leq\esp\lb\abs{X}\rb$.
    \end{enumerate}
\end{prop}
\begin{proof}
 La demostració d'aquestes propietats es deixa com a exercici ja que es dedueixen directament de la definició o d'altres propietats bàsiques.
\end{proof}

\begin{prop}
    Siguin $X, Y$ variables aleatòries, siguin $a, b\in\real$ i sigui $A\in\Asuc$. Es tenen les següents propietats de la variància.
    \begin{enumerate}[i)]
        \item $\var\lb X\rb=\esp\lb\lp X-\esp\lb X\rb\rp^2\rb=\esp\lb X^2+\esp\lb X\rb^2-2X\esp\lb X\rb\rb=\esp\lb X^2\rb+\esp\lb X\rb^2-2\esp\lb X\rb^2=\esp\lb X^2\rb-\esp\lb X\rb^2$,
        \item $\var\lb a\rb=0$,
        \item $\var\lb a+X\rb=\var\lb X\rb$,
        \item $\var\lb aX\rb=a^2\var\lb X\rb$.
    \end{enumerate}
\end{prop}
\begin{proof}
 La demostració d'aquestes propietats es deixa com a exercici ja que es dedueixen directament de la definició o d'altres propietats bàsiques.
\end{proof}

\begin{prop}
    \textit{Desigualtat de Holder}. Siguin $X,Y$ variables aleatòries i siguin $p,q\in\real$ tals que $\frac{1}{p}+\frac{1}{q}=1$. Si $\esp\lb\abs{X}^p\rb,\esp\lb\abs{Y}^q\rb<+\infty$, aleshores
    \[
        \esp\lb\abs{XY}\rb\leq\esp\lb\abs{X}^p\rb^{\frac{1}{p}}\esp\lb\abs{Y}^q\rb^{\frac{1}{q}}<+\infty.
    \]
    \textit{Desigualtat de Cauchy-Schwarz}. Siguin $X,Y$ variables aleatòries. Si $\esp\lb\abs{X}^2\rb,\esp\lb\abs{Y}^2\rb<+\infty$, aleshores
    \[
        \esp\lb\abs{XY}\rb\leq\esp\lb\abs{X}^2\rb^{\frac{1}{2}}\esp\lb\abs{Y}^2\rb^{\frac{1}{2}}<+\infty.
    \]
    \textit{Desigualtat de Minkowsky}. Siguin $X,Y$ variables aleatòries i sigui $p\in\real$.
    Si $\esp\lb\abs{X}^p\rb,\allowbreak \esp\lb\abs{Y}^p\rb<+\infty$, aleshores
    \[
        \esp\lb\abs{X+Y}^p\rb^{\frac{1}{p}}\leq\esp\lb\abs{X}^p\rb^{\frac{1}{p}}+\esp\lb\abs{Y}^p\rb^{\frac{1}{p}}<+\infty.
    \]
\end{prop}
\begin{proof}
    Tots aquests resultats són l'aplicació de les desigualtats corresponents demostrades al curs de teoria de la mesura.
\end{proof}
\begin{obs}
    La desigualtat de Cauchy-Schwarz és el cas particular $p=q=2$ de la desigualtat de Holder.
\end{obs}
\begin{teo}[Desigualtat de Markov]
    Sigui $\lp\Omega,\Asuc,p\rp$ un espai de probabilitat, sigui $X\colon\Omega\to\real$ una variable aleatòria amb $X>0$ i sigui $a\in\real^+$. Aleshores,
    \[
        P\lp X\geq a\rp \leq \frac{\esp\lb X\rb}{a}.
    \]
\end{teo}
\begin{proof}
    Sigui $A=\lc \omega\in\Omega\mid X\lp\omega\rp\geq a\rc$. Com que $X$ és mesurable, $A$ és un succés. Observem que
    \[
        a\mathbb{I}_A\lp\omega\rp\leq X\lp\omega\rp,\,\forall\omega\in\Omega.
    \]
    Aleshores,
    \[
        ap\lp X\geq a\rp = \esp\lb a\mathbb{I}_A\lp\omega\rp\rb \leq \esp\lb X\lp\omega\rp\rb = \esp\lb X\rb.
    \]
\end{proof}
\begin{teo}[Desigualtat de Txebixov]
    Sigui $X$ una variable aleatòria amb $\esp\lb X\rb<+\infty$, $\var\lb X\rb$ i $\var\lb X\rb\neq 0$. Aleshores, per tot $k>0$, es té que
    \[
        P\lp \abs{X-\esp\lb X\rb}\geq k\var\lb X\rb^{\frac{1}{2}}\rp\leq\frac{1}{k^2}.
    \]
\end{teo}
\begin{proof}
    Posem $Y:=\abs{X-\esp\lb X\rb}$. Observem que $Y$ és una variable aleatòria. Tenim que, per tot $a>0$,
    \begin{align*}
        P\lp Y\geq a\rp &= P\lp\abs{X-\esp\lb X\rb}\geq a\rp = \\
        &= P\lp\lp X-\esp\lb X\rb\rp^2\geq a^2\rp \leq\\
        &\leq \frac{\esp\lb \lp X-\esp\lb X\rb\rp^2\rb}{a^2} =\\
        &=\frac{\var\lb X\rb}{a^2},
    \end{align*}
    on hem aplicat la desigualtat de Markov. Prenent ara $a=k\var\lb X\rb^{\frac{1}{2}}$, deduïm el resultat volgut.
\end{proof}
\begin{obs}
    Quan $\var\lb X\rb=0$, seguim tenint que, per tot $a>0$,
    \[
        0\leq P\lp\abs{X-\esp\lb X\rb} \geq a\rp\leq \frac{\var\lb X\rb}{a^2} =0,
    \]
    i tenim que $P\lp X=\esp\lb X\rb\rp=1$.
\end{obs}

\section[Vectors de variables aleatòries. Independència de variables aleatòries]
    {Vectors de variables aleatòries. Independència de variables aleatòries
    \sectionmark{Vectors i independència de variables aleatòries}}
    \sectionmark{Vectors i independència de variables aleatòries}

\begin{defi}[vector de variables aleatòries]\index{variable aleatòria!multidimensional((tlab))}
    Sigui $\lp\Omega,\Asuc,p\rp$ un espai de probabilitat. Diem que un vector de variables aleatòries o una variable aleatòria multidimensional de dimensió $n$ és una funció mesurable $\vec{X}\colon\Omega\to\real^n$.
\end{defi}
\begin{obs}
    La funció 
    \begin{align*}
        \pi_i\colon\real^n &\to\real\\
        \lp x_1,\cdots,x_n\rp &\mapsto x_i
    \end{align*}
    és una funció mesurable. Per tant, les components $\pi_i\circ\vec{X}=\pi_i\lp\vec{X}\rp=X_i$ d'$\vec{X}$ són variables aleatòries.
\end{obs}
\begin{defi}[funció!de distribució!de probabilitat (2)]\index{distribució!marginal((tlab))}
    Sigui $\vec{X}=\lp X_1,\cdots, X_n\rp$ un vector de variables aleatòries de dimensió $n$. Anomenem funció de distribució de probabilitat d'$\vec{X}$ a 
    \[
        F_{\vec{X}}\lp x_1,\dots,x_n\rp = P\lp X_1\leq x_1,\dots, X_n\leq x_n\rp.
    \]
\end{defi}
\begin{prop}
    Sigui $\lp\Omega,\Asuc,p\rp$ un espai de probabilitat i sigui $\vec{X}=\lp X,Y\rp\colon\Omega\to\real^2$ un vector de variables aleatòries amb funció de distribució $F_{\vec{X}}\lp x,y\rp$. Aleshores,
    \begin{enumerate}[i)]
        \item $\lim\limits_{\lp x,y \rp\to\lp-\infty,-\infty\rp}F_{\vec{x}}\lp x, y\rp=0$ i $\lim\limits_{\lp x,y \rp\to\lp+\infty,+\infty\rp}F_{\vec{x}}\lp x, y\rp=1$.
        \item $x_1\leq x_2\implies F_{\vec{X}}\lp x_1,y\rp\leq F_{\vec{X}}\lp x_2,y\rp,\,\forall y\in\real$.
        \item $\lim\limits_{\lp x,y \rp\to\lp x_0^+,y_0^+\rp}F_{\vec{x}}\lp x, y\rp=F_{\vec{X}}\lp x_0,y_0\rp$.
        \item $\lim\limits_{y\to+\infty}F_{\vec{x}}\lp x, y\rp=F_X\lp x\rp$ i les anomenem distribucions marginals.
        \item $P\lp a<x\leq b, c<y\leq d\rp=F_{\vec{X}}\lp b,d\rp-F_{\vec{X}}\lp b,c\rp-F_{\vec{X}}\lp a,d\rp+F_{\vec{X}}\lp a,c\rp=\Delta_RF$,
    \end{enumerate}
    on $R$ és el rectangle que determinen $a, b, c, d$.
\end{prop}
\begin{proof}
    Les demostracions de les tres primeres propietats son anàlogues a les de una sola dimensió. Les altres no les farem.
\end{proof}

\begin{prop}
    Sigui una $F\lp x,y\rp$ una funció que satisfà les cinc propietats de la proposició anterior i tal que $\Delta_RF>0$ per tot rectangle $R$. Aleshores,
    \[
        \exists \vec{X}=\lp X,Y\rp \text{ variable aleatòria tal que } F\lp x,y\rp = F_{\vec{X}}.
    \]
\end{prop}
\begin{proof}
    La demostració és llarga i pesada, i no la farem.
\end{proof}
\begin{defi}[independència de variables aleatòries]
    Sigui $\lc X_i\rc_{i\in I}$ una família de variables aletòries sobre un espai de probabilitat $\lp\Omega,\Asuc,p\rp$. Diem que $\lc X_i\rc_{i\in I}$ és independent si per tot $J\subseteq I$ amb $\abs{J}<+\infty$ i per qualssevol $B_1,\dots,B_{\abs{J}}\in\B$ es té que
    \[
        P\lp\bigcap_{j\in J}X_j\in B_j\rp = \prod_{j\in J} P\lp X_j\in B_j\rp.
    \]
\end{defi}
En general, no podem conéixer la distribució de $X_1,\dots,X_n$ si coneixem només les seves distribucions marginals. Tanmateix, sí que podem conéixer-la si les variables aleatòries són independents.
\begin{prop}
    Si prenem $B_i=\lp-\infty,x_i\rp$ i una família finita de variables aleatòries independents $\lc X_i\rc_{i=1}^n$ (i posant $\vec{X}=\lp X_1,\dots,X_n\rp$), tenim que
    \[
        F_{\vec{X}}\lp x_1,\dots,x_n\rp = \prod_{i=1}^n F_{X_i}\lp x_i\rp.
    \]
\end{prop}
\begin{proof}
    Directament a partir de la definició tenim que
    \begin{align*}
        F_{\vec{X}}\lp x_1,\dots,x_n\rp &= p\lp \bigcap_{i=1}^n X_i\in B_i\rp =p\lp \bigcap_{i=1}^n X_i\leq x_i\rp =\\
        &= \prod_{i=1}^n P\lp X_i\leq x_i\rp = \prod_{i=1}^n F_{X_i}\lp x_i\rp.
    \end{align*}
\end{proof}
\begin{obs}
    Es pot demostrar que la impliació recíproca és certa, és a dir, que si una determinada família de variables aleatòries satisfà la igualtat anterior, aleshores és independent.
\end{obs}
\begin{obs}
    Siguin $\lc X_i\rc_{i=1}^n$ una família de variables aleatòries independent i siguin $f_1,\dots,f_n\colon \real\to\real$ funcions mesurables. Aleshores, $\lc f_1\lp X_i\rp\rc_{i=1}^n$ també són independents.
\end{obs}
\begin{prop}
    Siguin $X_1, \dots, X_n$ variables aleatòries independents. Aleshores 
    \[
        \esp[X_1\dots X_n] = \prod \limits_{i=1}^n \esp[X_i].
    \]
\end{prop}

%TODO no se si falta quelcom d'abans
\begin{defi}[covariància]
    Siguin $X, Y$ variables aleatòries. La covariància de $X$ i $Y$ \'es:
    \[
        \cov(X, Y) = \esp \left[\lp X - \esp(X) \rp \lp Y - \esp(Y) \rp \right]
    \]
\end{defi}

\begin{obs}
    Està ben definida sempre que $\esp[|X|], \esp[|Y|] < +\infty$ i
    \[
        \cov(X, Y) = \esp[XY] - \esp[X]\esp[Y]
    \]
\end{obs}

\begin{proof}
    Directament de la definició de covariància i usant les propietats de l'esperança
    \begin{gather*} 
        \cov(X,Y) = \esp\left[XY - \esp[X]Y - \esp[Y]X + \esp[X]\esp[Y]\right] = \\
        = \esp[XY] - 2\esp[X]\esp[Y] + \esp[X]\esp[Y] 
        = \esp[XY] - \esp[X]\esp[Y].
    \end{gather*} 
\end{proof}

\begin{obs}
    Si $X$ i $Y$ són independents, aleshores $\cov(X,Y)$ = 0.
\end{obs}

\begin{obs}
    Per $X, Y$ variables aleatòries, es satisfà $\var[X + Y] = \var[X] + \var[Y] 
    + 2 \cov(X, Y)$. Si a m\'es són independents, $\var[X+Y] = \var[X] + \var[Y]$.
\end{obs}

\begin{proof}
    Usant la definició de variància tenim que
    \begin{gather*} 
        \esp[(X+Y)^2] - \esp[X+Y]^2 =\\
        = \esp[X^2] + \esp[Y^2] + 2\esp[XY] - \esp[X]^2
        - \esp[Y]^2 - 2\esp[X]\esp[Y] =\\
        = \var[X] + \var[Y] + 2\cov(X, Y).
    \end{gather*} 
\end{proof}

\begin{prop}
    Siguin $X, Y, Z$ variables aleatòries, i $a, b, C$  constants. Aleshores
    \begin{enumerate}[i)]
    \item $\cov(C, X) = 0.$
    \item $\cov(X, X) = \var[X].$
    \item $\cov(X, Y) = \cov(Y, X).$
    \item $\cov(aX + bY, Z) = a \cov(X,Z) + b\cov(Y, Z).$
    \item $\cov(X, Y)^2 \leq \var[X]\var[Y].$
    \end{enumerate}
\end{prop}

\begin{proof}
    Demostrarem nom\'es l'últim apartat. Els altres es deixen com exercici pel lector.
    Tenim doncs que
    \[
        \left|\cov(X, Y)\right| = 
        \left|\esp\lb\lp X - \esp[X] \rp \lp Y- \esp[Y]\rp\rb\right| 
        \leq \esp\lb\left|\lp X - \esp[X] \rp \lp Y- \esp[Y]\rp\right|\rb
    \]
    que, aplicant la desigualtat de Cauchy-Schwarz, \'es menor que
    \[
        \sqrt{\esp \lb \lp X - \esp[X] \rp^2\rb} 
        \sqrt{\esp \lb \lp Y - \esp[Y] \rp^2\rb}
        = \sqrt{\lp\var[X]\rp} \sqrt{\lp \var[Y] \rp}.
    \]
    \'Es a dir, $\cov(X, Y)^2 \leq \var[X] \var[Y]$.
\end{proof}

\begin{obs}
    Com es conclou de la desigualtat de Cauchy-Schwarz, la desigualtat anterior \'es una
    igualtat si i nom\'es si $X - \esp[X] = \lambda (Y - \esp[Y])$.
\end{obs}

\begin{obs}
    $\cov(X, Y) = 0 \centernot \implies X,Y$ independents.
\end{obs}

\begin{proof}
    En donem un contraexemple. Definim $X, Y$ variables aleatòries tals que
    \[
        X = 
        \begin{cases}
            1 &\text{amb probabilitat } \frac{1}{2}\\
            -1 &\text{amb probabilitat } \frac{1}{2} 
        \end{cases}, \,
        Y =
        \begin{cases}
            0 &\text{si } X = -1 \\
            \pm 1 &\text{amb probabilitat } \frac{1}{2} \text{ si } X = 1
        \end{cases}.
    \]
    En aquest cas,
    \begin{align*}
        \esp[XY] &= \int\limits_{\Omega} XY \dif p = \\
        &= 0 \cdot P(X = -1) + 1 \cdot 
        P(X = 1, Y = 1) - 1 \cdot P(X= 1, Y= -1) = \\
        &= \frac{1}{4} - \frac{1}{4} = 0,
    \end{align*}
    i tamb\'e tenim que
    \[
        \esp[X] = 1\cdot P(x = 1) - 1 \cdot P(X = -1) = 0. 
    \]
    Per les últimes dues igualtats, $\cov(X, Y) = \esp[XY] - \esp[X]\esp[Y] = 0$. Vegem
    però que no són indepedents:
    \[
        P(X = 1)P(Y = 0) = \frac{1}{2} \cdot \frac{1}{2} = \frac{1}{4}.
    \]
    Però $P(X = 1, Y = 0) = 0$, \'es a dir, com volíem veure $X$ i $Y$ no són independents.
\end{proof}


