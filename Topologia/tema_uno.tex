\chapter{Espacios m\'etricos y aplicaciones continuas}
\section{Distancias}

\begin{ej}
    Hay que comprobar que
    \begin{enumerate}[i)]
        \item $ d(x,y) \geq 0 \; \forall x,y \in X $ y que $ d(x,y) = 0 \iff x = y $, lo cual es trivial por la definición de $ d $.
        \item $ d(x,y) = d(y,x) \; \forall x, y \in X $, que de nuevo, esinmediato por la definición.
        \item $ d(x,y) \leq d(x,z) + d(z,y) \; \forall x, y, z \in X $, suponemos que $ x, y, z $ son distintos dos a dos. Entonces,
            $ d(x,y) = 1 \leq d(x, z) + d(z, y) = 1 + 1 = 2 $.
    \end{enumerate}
\end{ej}

\begin{ej}
    De nuevo, hemos de comprobar que
    \begin{enumerate}[i)]
        \item $d_{\text{cent}}(x, y) \geq 0 \; \forall x, y \in X$. Esto es fácil de ver, ya que $d(c, x), d(c,y) \geq 0$, por lo tanto
            $d_{\text{cent}}(x,y) = d(c,x) + d(c,y) \geq 0$.
        \item $d_{\text{cent}}(x, y) = 0 \iff x = y$. Si $x = y \implies d_{\text{cent}}(x,y) = 0$ por definición. Y si $d_{\text{cent}}(x,y) = 0$,
            hay dos posiblidades, o $x = y$ (y ya hemos acabado) o $d(c, x) = d(c,y) = 0$. Pero si ocurre lo segundo, entonces $x = c = y$.
        \item $d_{\text{cent}}(x, y) = d_{\text{cent}}(y, x) \; \forall x, y \in X$. Esto es obvio ya que la suma es conmutativa y $d$ tambi\'en.
        \item $d_{\text{cent}}(x,y) \leq d_{\text{cent}}(x, z) + d_{\text{cent}}(z, y)$. Vamos a ver que se cumple:
            \[
                d_{\text{cent}}(x, y) = d(c,x) + d(c,y) \leq d(c,x) + d(c,z) + d(c,z) + d(c,y) =
                d_{\text{cent}}(x, z) + d_{\text{cent}}(z,y)
            \]
    \end{enumerate}
    Ahora, dibujaremos las bolas. En azul, está $B\left( (0,0), 1 \right)$ y en rojo $B\left( (1,1), 2 \right)$:
    \begin{center}
        %% Creator: Matplotlib, PGF backend
%%
%% To include the figure in your LaTeX document, write
%%   \input{<filename>.pgf}
%%
%% Make sure the required packages are loaded in your preamble
%%   \usepackage{pgf}
%%
%% Figures using additional raster images can only be included by \input if
%% they are in the same directory as the main LaTeX file. For loading figures
%% from other directories you can use the `import` package
%%   \usepackage{import}
%% and then include the figures with
%%   \import{<path to file>}{<filename>.pgf}
%%
%% Matplotlib used the following preamble
%%   \usepackage{fontspec}
%%   \setmainfont{DejaVu Serif}
%%   \setsansfont{DejaVu Sans}
%%   \setmonofont{DejaVu Sans Mono}
%%
\begingroup%
\makeatletter%
\begin{pgfpicture}%
\pgfpathrectangle{\pgfpointorigin}{\pgfqpoint{4.400000in}{4.400000in}}%
\pgfusepath{use as bounding box, clip}%
\begin{pgfscope}%
\pgfsetbuttcap%
\pgfsetmiterjoin%
\definecolor{currentfill}{rgb}{1.000000,1.000000,1.000000}%
\pgfsetfillcolor{currentfill}%
\pgfsetlinewidth{0.000000pt}%
\definecolor{currentstroke}{rgb}{1.000000,1.000000,1.000000}%
\pgfsetstrokecolor{currentstroke}%
\pgfsetdash{}{0pt}%
\pgfpathmoveto{\pgfqpoint{0.000000in}{0.000000in}}%
\pgfpathlineto{\pgfqpoint{4.400000in}{0.000000in}}%
\pgfpathlineto{\pgfqpoint{4.400000in}{4.400000in}}%
\pgfpathlineto{\pgfqpoint{0.000000in}{4.400000in}}%
\pgfpathclose%
\pgfusepath{fill}%
\end{pgfscope}%
\begin{pgfscope}%
\pgfsetbuttcap%
\pgfsetmiterjoin%
\definecolor{currentfill}{rgb}{1.000000,1.000000,1.000000}%
\pgfsetfillcolor{currentfill}%
\pgfsetlinewidth{0.000000pt}%
\definecolor{currentstroke}{rgb}{0.000000,0.000000,0.000000}%
\pgfsetstrokecolor{currentstroke}%
\pgfsetstrokeopacity{0.000000}%
\pgfsetdash{}{0pt}%
\pgfpathmoveto{\pgfqpoint{0.135000in}{0.135000in}}%
\pgfpathlineto{\pgfqpoint{4.265000in}{0.135000in}}%
\pgfpathlineto{\pgfqpoint{4.265000in}{4.265000in}}%
\pgfpathlineto{\pgfqpoint{0.135000in}{4.265000in}}%
\pgfpathclose%
\pgfusepath{fill}%
\end{pgfscope}%
\begin{pgfscope}%
\pgfpathrectangle{\pgfqpoint{0.135000in}{0.135000in}}{\pgfqpoint{4.130000in}{4.130000in}} %
\pgfusepath{clip}%
\pgfsetbuttcap%
\pgfsetroundjoin%
\definecolor{currentfill}{rgb}{0.000000,0.000000,1.000000}%
\pgfsetfillcolor{currentfill}%
\pgfsetlinewidth{0.000000pt}%
\definecolor{currentstroke}{rgb}{0.000000,0.000000,0.000000}%
\pgfsetstrokecolor{currentstroke}%
\pgfsetdash{}{0pt}%
\pgfpathmoveto{\pgfqpoint{2.200000in}{2.161964in}}%
\pgfpathcurveto{\pgfqpoint{2.210087in}{2.161964in}}{\pgfqpoint{2.219763in}{2.165971in}}{\pgfqpoint{2.226896in}{2.173104in}}%
\pgfpathcurveto{\pgfqpoint{2.234029in}{2.180237in}}{\pgfqpoint{2.238036in}{2.189913in}}{\pgfqpoint{2.238036in}{2.200000in}}%
\pgfpathcurveto{\pgfqpoint{2.238036in}{2.210087in}}{\pgfqpoint{2.234029in}{2.219763in}}{\pgfqpoint{2.226896in}{2.226896in}}%
\pgfpathcurveto{\pgfqpoint{2.219763in}{2.234029in}}{\pgfqpoint{2.210087in}{2.238036in}}{\pgfqpoint{2.200000in}{2.238036in}}%
\pgfpathcurveto{\pgfqpoint{2.189913in}{2.238036in}}{\pgfqpoint{2.180237in}{2.234029in}}{\pgfqpoint{2.173104in}{2.226896in}}%
\pgfpathcurveto{\pgfqpoint{2.165971in}{2.219763in}}{\pgfqpoint{2.161964in}{2.210087in}}{\pgfqpoint{2.161964in}{2.200000in}}%
\pgfpathcurveto{\pgfqpoint{2.161964in}{2.189913in}}{\pgfqpoint{2.165971in}{2.180237in}}{\pgfqpoint{2.173104in}{2.173104in}}%
\pgfpathcurveto{\pgfqpoint{2.180237in}{2.165971in}}{\pgfqpoint{2.189913in}{2.161964in}}{\pgfqpoint{2.200000in}{2.161964in}}%
\pgfpathclose%
\pgfusepath{fill}%
\end{pgfscope}%
\begin{pgfscope}%
\pgfpathrectangle{\pgfqpoint{0.135000in}{0.135000in}}{\pgfqpoint{4.130000in}{4.130000in}} %
\pgfusepath{clip}%
\pgfsetbuttcap%
\pgfsetroundjoin%
\definecolor{currentfill}{rgb}{1.000000,0.000000,0.000000}%
\pgfsetfillcolor{currentfill}%
\pgfsetlinewidth{0.000000pt}%
\definecolor{currentstroke}{rgb}{0.000000,0.000000,0.000000}%
\pgfsetstrokecolor{currentstroke}%
\pgfsetdash{}{0pt}%
\pgfpathmoveto{\pgfqpoint{4.185577in}{4.154520in}}%
\pgfpathcurveto{\pgfqpoint{4.193813in}{4.154520in}}{\pgfqpoint{4.201713in}{4.157793in}}{\pgfqpoint{4.207537in}{4.163617in}}%
\pgfpathcurveto{\pgfqpoint{4.213361in}{4.169441in}}{\pgfqpoint{4.216633in}{4.177341in}}{\pgfqpoint{4.216633in}{4.185577in}}%
\pgfpathcurveto{\pgfqpoint{4.216633in}{4.193813in}}{\pgfqpoint{4.213361in}{4.201713in}}{\pgfqpoint{4.207537in}{4.207537in}}%
\pgfpathcurveto{\pgfqpoint{4.201713in}{4.213361in}}{\pgfqpoint{4.193813in}{4.216633in}}{\pgfqpoint{4.185577in}{4.216633in}}%
\pgfpathcurveto{\pgfqpoint{4.177341in}{4.216633in}}{\pgfqpoint{4.169441in}{4.213361in}}{\pgfqpoint{4.163617in}{4.207537in}}%
\pgfpathcurveto{\pgfqpoint{4.157793in}{4.201713in}}{\pgfqpoint{4.154520in}{4.193813in}}{\pgfqpoint{4.154520in}{4.185577in}}%
\pgfpathcurveto{\pgfqpoint{4.154520in}{4.177341in}}{\pgfqpoint{4.157793in}{4.169441in}}{\pgfqpoint{4.163617in}{4.163617in}}%
\pgfpathcurveto{\pgfqpoint{4.169441in}{4.157793in}}{\pgfqpoint{4.177341in}{4.154520in}}{\pgfqpoint{4.185577in}{4.154520in}}%
\pgfpathclose%
\pgfusepath{fill}%
\end{pgfscope}%
\begin{pgfscope}%
\pgfsetbuttcap%
\pgfsetroundjoin%
\definecolor{currentfill}{rgb}{0.000000,0.000000,0.000000}%
\pgfsetfillcolor{currentfill}%
\pgfsetlinewidth{0.803000pt}%
\definecolor{currentstroke}{rgb}{0.000000,0.000000,0.000000}%
\pgfsetstrokecolor{currentstroke}%
\pgfsetdash{}{0pt}%
\pgfsys@defobject{currentmarker}{\pgfqpoint{0.000000in}{-0.048611in}}{\pgfqpoint{0.000000in}{0.000000in}}{%
\pgfpathmoveto{\pgfqpoint{0.000000in}{0.000000in}}%
\pgfpathlineto{\pgfqpoint{0.000000in}{-0.048611in}}%
\pgfusepath{stroke,fill}%
}%
\begin{pgfscope}%
\pgfsys@transformshift{0.214423in}{2.200000in}%
\pgfsys@useobject{currentmarker}{}%
\end{pgfscope}%
\end{pgfscope}%
\begin{pgfscope}%
\pgftext[x=0.214423in,y=2.102778in,,top]{\sffamily\fontsize{10.000000}{12.000000}\selectfont -1}%
\end{pgfscope}%
\begin{pgfscope}%
\pgfsetbuttcap%
\pgfsetroundjoin%
\definecolor{currentfill}{rgb}{0.000000,0.000000,0.000000}%
\pgfsetfillcolor{currentfill}%
\pgfsetlinewidth{0.803000pt}%
\definecolor{currentstroke}{rgb}{0.000000,0.000000,0.000000}%
\pgfsetstrokecolor{currentstroke}%
\pgfsetdash{}{0pt}%
\pgfsys@defobject{currentmarker}{\pgfqpoint{0.000000in}{-0.048611in}}{\pgfqpoint{0.000000in}{0.000000in}}{%
\pgfpathmoveto{\pgfqpoint{0.000000in}{0.000000in}}%
\pgfpathlineto{\pgfqpoint{0.000000in}{-0.048611in}}%
\pgfusepath{stroke,fill}%
}%
\begin{pgfscope}%
\pgfsys@transformshift{1.207212in}{2.200000in}%
\pgfsys@useobject{currentmarker}{}%
\end{pgfscope}%
\end{pgfscope}%
\begin{pgfscope}%
\pgftext[x=1.207212in,y=2.102778in,,top]{\sffamily\fontsize{10.000000}{12.000000}\selectfont -0.5}%
\end{pgfscope}%
\begin{pgfscope}%
\pgfsetbuttcap%
\pgfsetroundjoin%
\definecolor{currentfill}{rgb}{0.000000,0.000000,0.000000}%
\pgfsetfillcolor{currentfill}%
\pgfsetlinewidth{0.803000pt}%
\definecolor{currentstroke}{rgb}{0.000000,0.000000,0.000000}%
\pgfsetstrokecolor{currentstroke}%
\pgfsetdash{}{0pt}%
\pgfsys@defobject{currentmarker}{\pgfqpoint{0.000000in}{-0.048611in}}{\pgfqpoint{0.000000in}{0.000000in}}{%
\pgfpathmoveto{\pgfqpoint{0.000000in}{0.000000in}}%
\pgfpathlineto{\pgfqpoint{0.000000in}{-0.048611in}}%
\pgfusepath{stroke,fill}%
}%
\begin{pgfscope}%
\pgfsys@transformshift{2.200000in}{2.200000in}%
\pgfsys@useobject{currentmarker}{}%
\end{pgfscope}%
\end{pgfscope}%
\begin{pgfscope}%
\pgfsetbuttcap%
\pgfsetroundjoin%
\definecolor{currentfill}{rgb}{0.000000,0.000000,0.000000}%
\pgfsetfillcolor{currentfill}%
\pgfsetlinewidth{0.803000pt}%
\definecolor{currentstroke}{rgb}{0.000000,0.000000,0.000000}%
\pgfsetstrokecolor{currentstroke}%
\pgfsetdash{}{0pt}%
\pgfsys@defobject{currentmarker}{\pgfqpoint{0.000000in}{-0.048611in}}{\pgfqpoint{0.000000in}{0.000000in}}{%
\pgfpathmoveto{\pgfqpoint{0.000000in}{0.000000in}}%
\pgfpathlineto{\pgfqpoint{0.000000in}{-0.048611in}}%
\pgfusepath{stroke,fill}%
}%
\begin{pgfscope}%
\pgfsys@transformshift{3.192788in}{2.200000in}%
\pgfsys@useobject{currentmarker}{}%
\end{pgfscope}%
\end{pgfscope}%
\begin{pgfscope}%
\pgftext[x=3.192788in,y=2.102778in,,top]{\sffamily\fontsize{10.000000}{12.000000}\selectfont 0.5}%
\end{pgfscope}%
\begin{pgfscope}%
\pgfsetbuttcap%
\pgfsetroundjoin%
\definecolor{currentfill}{rgb}{0.000000,0.000000,0.000000}%
\pgfsetfillcolor{currentfill}%
\pgfsetlinewidth{0.803000pt}%
\definecolor{currentstroke}{rgb}{0.000000,0.000000,0.000000}%
\pgfsetstrokecolor{currentstroke}%
\pgfsetdash{}{0pt}%
\pgfsys@defobject{currentmarker}{\pgfqpoint{0.000000in}{-0.048611in}}{\pgfqpoint{0.000000in}{0.000000in}}{%
\pgfpathmoveto{\pgfqpoint{0.000000in}{0.000000in}}%
\pgfpathlineto{\pgfqpoint{0.000000in}{-0.048611in}}%
\pgfusepath{stroke,fill}%
}%
\begin{pgfscope}%
\pgfsys@transformshift{4.185577in}{2.200000in}%
\pgfsys@useobject{currentmarker}{}%
\end{pgfscope}%
\end{pgfscope}%
\begin{pgfscope}%
\pgftext[x=4.185577in,y=2.102778in,,top]{\sffamily\fontsize{10.000000}{12.000000}\selectfont 1}%
\end{pgfscope}%
\begin{pgfscope}%
\pgfsetbuttcap%
\pgfsetroundjoin%
\definecolor{currentfill}{rgb}{0.000000,0.000000,0.000000}%
\pgfsetfillcolor{currentfill}%
\pgfsetlinewidth{0.602250pt}%
\definecolor{currentstroke}{rgb}{0.000000,0.000000,0.000000}%
\pgfsetstrokecolor{currentstroke}%
\pgfsetdash{}{0pt}%
\pgfsys@defobject{currentmarker}{\pgfqpoint{0.000000in}{-0.027778in}}{\pgfqpoint{0.000000in}{0.000000in}}{%
\pgfpathmoveto{\pgfqpoint{0.000000in}{0.000000in}}%
\pgfpathlineto{\pgfqpoint{0.000000in}{-0.027778in}}%
\pgfusepath{stroke,fill}%
}%
\begin{pgfscope}%
\pgfsys@transformshift{0.412981in}{2.200000in}%
\pgfsys@useobject{currentmarker}{}%
\end{pgfscope}%
\end{pgfscope}%
\begin{pgfscope}%
\pgfsetbuttcap%
\pgfsetroundjoin%
\definecolor{currentfill}{rgb}{0.000000,0.000000,0.000000}%
\pgfsetfillcolor{currentfill}%
\pgfsetlinewidth{0.602250pt}%
\definecolor{currentstroke}{rgb}{0.000000,0.000000,0.000000}%
\pgfsetstrokecolor{currentstroke}%
\pgfsetdash{}{0pt}%
\pgfsys@defobject{currentmarker}{\pgfqpoint{0.000000in}{-0.027778in}}{\pgfqpoint{0.000000in}{0.000000in}}{%
\pgfpathmoveto{\pgfqpoint{0.000000in}{0.000000in}}%
\pgfpathlineto{\pgfqpoint{0.000000in}{-0.027778in}}%
\pgfusepath{stroke,fill}%
}%
\begin{pgfscope}%
\pgfsys@transformshift{0.611538in}{2.200000in}%
\pgfsys@useobject{currentmarker}{}%
\end{pgfscope}%
\end{pgfscope}%
\begin{pgfscope}%
\pgfsetbuttcap%
\pgfsetroundjoin%
\definecolor{currentfill}{rgb}{0.000000,0.000000,0.000000}%
\pgfsetfillcolor{currentfill}%
\pgfsetlinewidth{0.602250pt}%
\definecolor{currentstroke}{rgb}{0.000000,0.000000,0.000000}%
\pgfsetstrokecolor{currentstroke}%
\pgfsetdash{}{0pt}%
\pgfsys@defobject{currentmarker}{\pgfqpoint{0.000000in}{-0.027778in}}{\pgfqpoint{0.000000in}{0.000000in}}{%
\pgfpathmoveto{\pgfqpoint{0.000000in}{0.000000in}}%
\pgfpathlineto{\pgfqpoint{0.000000in}{-0.027778in}}%
\pgfusepath{stroke,fill}%
}%
\begin{pgfscope}%
\pgfsys@transformshift{0.810096in}{2.200000in}%
\pgfsys@useobject{currentmarker}{}%
\end{pgfscope}%
\end{pgfscope}%
\begin{pgfscope}%
\pgfsetbuttcap%
\pgfsetroundjoin%
\definecolor{currentfill}{rgb}{0.000000,0.000000,0.000000}%
\pgfsetfillcolor{currentfill}%
\pgfsetlinewidth{0.602250pt}%
\definecolor{currentstroke}{rgb}{0.000000,0.000000,0.000000}%
\pgfsetstrokecolor{currentstroke}%
\pgfsetdash{}{0pt}%
\pgfsys@defobject{currentmarker}{\pgfqpoint{0.000000in}{-0.027778in}}{\pgfqpoint{0.000000in}{0.000000in}}{%
\pgfpathmoveto{\pgfqpoint{0.000000in}{0.000000in}}%
\pgfpathlineto{\pgfqpoint{0.000000in}{-0.027778in}}%
\pgfusepath{stroke,fill}%
}%
\begin{pgfscope}%
\pgfsys@transformshift{1.008654in}{2.200000in}%
\pgfsys@useobject{currentmarker}{}%
\end{pgfscope}%
\end{pgfscope}%
\begin{pgfscope}%
\pgfsetbuttcap%
\pgfsetroundjoin%
\definecolor{currentfill}{rgb}{0.000000,0.000000,0.000000}%
\pgfsetfillcolor{currentfill}%
\pgfsetlinewidth{0.602250pt}%
\definecolor{currentstroke}{rgb}{0.000000,0.000000,0.000000}%
\pgfsetstrokecolor{currentstroke}%
\pgfsetdash{}{0pt}%
\pgfsys@defobject{currentmarker}{\pgfqpoint{0.000000in}{-0.027778in}}{\pgfqpoint{0.000000in}{0.000000in}}{%
\pgfpathmoveto{\pgfqpoint{0.000000in}{0.000000in}}%
\pgfpathlineto{\pgfqpoint{0.000000in}{-0.027778in}}%
\pgfusepath{stroke,fill}%
}%
\begin{pgfscope}%
\pgfsys@transformshift{1.207212in}{2.200000in}%
\pgfsys@useobject{currentmarker}{}%
\end{pgfscope}%
\end{pgfscope}%
\begin{pgfscope}%
\pgfsetbuttcap%
\pgfsetroundjoin%
\definecolor{currentfill}{rgb}{0.000000,0.000000,0.000000}%
\pgfsetfillcolor{currentfill}%
\pgfsetlinewidth{0.602250pt}%
\definecolor{currentstroke}{rgb}{0.000000,0.000000,0.000000}%
\pgfsetstrokecolor{currentstroke}%
\pgfsetdash{}{0pt}%
\pgfsys@defobject{currentmarker}{\pgfqpoint{0.000000in}{-0.027778in}}{\pgfqpoint{0.000000in}{0.000000in}}{%
\pgfpathmoveto{\pgfqpoint{0.000000in}{0.000000in}}%
\pgfpathlineto{\pgfqpoint{0.000000in}{-0.027778in}}%
\pgfusepath{stroke,fill}%
}%
\begin{pgfscope}%
\pgfsys@transformshift{1.405769in}{2.200000in}%
\pgfsys@useobject{currentmarker}{}%
\end{pgfscope}%
\end{pgfscope}%
\begin{pgfscope}%
\pgfsetbuttcap%
\pgfsetroundjoin%
\definecolor{currentfill}{rgb}{0.000000,0.000000,0.000000}%
\pgfsetfillcolor{currentfill}%
\pgfsetlinewidth{0.602250pt}%
\definecolor{currentstroke}{rgb}{0.000000,0.000000,0.000000}%
\pgfsetstrokecolor{currentstroke}%
\pgfsetdash{}{0pt}%
\pgfsys@defobject{currentmarker}{\pgfqpoint{0.000000in}{-0.027778in}}{\pgfqpoint{0.000000in}{0.000000in}}{%
\pgfpathmoveto{\pgfqpoint{0.000000in}{0.000000in}}%
\pgfpathlineto{\pgfqpoint{0.000000in}{-0.027778in}}%
\pgfusepath{stroke,fill}%
}%
\begin{pgfscope}%
\pgfsys@transformshift{1.604327in}{2.200000in}%
\pgfsys@useobject{currentmarker}{}%
\end{pgfscope}%
\end{pgfscope}%
\begin{pgfscope}%
\pgfsetbuttcap%
\pgfsetroundjoin%
\definecolor{currentfill}{rgb}{0.000000,0.000000,0.000000}%
\pgfsetfillcolor{currentfill}%
\pgfsetlinewidth{0.602250pt}%
\definecolor{currentstroke}{rgb}{0.000000,0.000000,0.000000}%
\pgfsetstrokecolor{currentstroke}%
\pgfsetdash{}{0pt}%
\pgfsys@defobject{currentmarker}{\pgfqpoint{0.000000in}{-0.027778in}}{\pgfqpoint{0.000000in}{0.000000in}}{%
\pgfpathmoveto{\pgfqpoint{0.000000in}{0.000000in}}%
\pgfpathlineto{\pgfqpoint{0.000000in}{-0.027778in}}%
\pgfusepath{stroke,fill}%
}%
\begin{pgfscope}%
\pgfsys@transformshift{1.802885in}{2.200000in}%
\pgfsys@useobject{currentmarker}{}%
\end{pgfscope}%
\end{pgfscope}%
\begin{pgfscope}%
\pgfsetbuttcap%
\pgfsetroundjoin%
\definecolor{currentfill}{rgb}{0.000000,0.000000,0.000000}%
\pgfsetfillcolor{currentfill}%
\pgfsetlinewidth{0.602250pt}%
\definecolor{currentstroke}{rgb}{0.000000,0.000000,0.000000}%
\pgfsetstrokecolor{currentstroke}%
\pgfsetdash{}{0pt}%
\pgfsys@defobject{currentmarker}{\pgfqpoint{0.000000in}{-0.027778in}}{\pgfqpoint{0.000000in}{0.000000in}}{%
\pgfpathmoveto{\pgfqpoint{0.000000in}{0.000000in}}%
\pgfpathlineto{\pgfqpoint{0.000000in}{-0.027778in}}%
\pgfusepath{stroke,fill}%
}%
\begin{pgfscope}%
\pgfsys@transformshift{2.001442in}{2.200000in}%
\pgfsys@useobject{currentmarker}{}%
\end{pgfscope}%
\end{pgfscope}%
\begin{pgfscope}%
\pgfsetbuttcap%
\pgfsetroundjoin%
\definecolor{currentfill}{rgb}{0.000000,0.000000,0.000000}%
\pgfsetfillcolor{currentfill}%
\pgfsetlinewidth{0.602250pt}%
\definecolor{currentstroke}{rgb}{0.000000,0.000000,0.000000}%
\pgfsetstrokecolor{currentstroke}%
\pgfsetdash{}{0pt}%
\pgfsys@defobject{currentmarker}{\pgfqpoint{0.000000in}{-0.027778in}}{\pgfqpoint{0.000000in}{0.000000in}}{%
\pgfpathmoveto{\pgfqpoint{0.000000in}{0.000000in}}%
\pgfpathlineto{\pgfqpoint{0.000000in}{-0.027778in}}%
\pgfusepath{stroke,fill}%
}%
\begin{pgfscope}%
\pgfsys@transformshift{2.200000in}{2.200000in}%
\pgfsys@useobject{currentmarker}{}%
\end{pgfscope}%
\end{pgfscope}%
\begin{pgfscope}%
\pgfsetbuttcap%
\pgfsetroundjoin%
\definecolor{currentfill}{rgb}{0.000000,0.000000,0.000000}%
\pgfsetfillcolor{currentfill}%
\pgfsetlinewidth{0.602250pt}%
\definecolor{currentstroke}{rgb}{0.000000,0.000000,0.000000}%
\pgfsetstrokecolor{currentstroke}%
\pgfsetdash{}{0pt}%
\pgfsys@defobject{currentmarker}{\pgfqpoint{0.000000in}{-0.027778in}}{\pgfqpoint{0.000000in}{0.000000in}}{%
\pgfpathmoveto{\pgfqpoint{0.000000in}{0.000000in}}%
\pgfpathlineto{\pgfqpoint{0.000000in}{-0.027778in}}%
\pgfusepath{stroke,fill}%
}%
\begin{pgfscope}%
\pgfsys@transformshift{2.398558in}{2.200000in}%
\pgfsys@useobject{currentmarker}{}%
\end{pgfscope}%
\end{pgfscope}%
\begin{pgfscope}%
\pgfsetbuttcap%
\pgfsetroundjoin%
\definecolor{currentfill}{rgb}{0.000000,0.000000,0.000000}%
\pgfsetfillcolor{currentfill}%
\pgfsetlinewidth{0.602250pt}%
\definecolor{currentstroke}{rgb}{0.000000,0.000000,0.000000}%
\pgfsetstrokecolor{currentstroke}%
\pgfsetdash{}{0pt}%
\pgfsys@defobject{currentmarker}{\pgfqpoint{0.000000in}{-0.027778in}}{\pgfqpoint{0.000000in}{0.000000in}}{%
\pgfpathmoveto{\pgfqpoint{0.000000in}{0.000000in}}%
\pgfpathlineto{\pgfqpoint{0.000000in}{-0.027778in}}%
\pgfusepath{stroke,fill}%
}%
\begin{pgfscope}%
\pgfsys@transformshift{2.597115in}{2.200000in}%
\pgfsys@useobject{currentmarker}{}%
\end{pgfscope}%
\end{pgfscope}%
\begin{pgfscope}%
\pgfsetbuttcap%
\pgfsetroundjoin%
\definecolor{currentfill}{rgb}{0.000000,0.000000,0.000000}%
\pgfsetfillcolor{currentfill}%
\pgfsetlinewidth{0.602250pt}%
\definecolor{currentstroke}{rgb}{0.000000,0.000000,0.000000}%
\pgfsetstrokecolor{currentstroke}%
\pgfsetdash{}{0pt}%
\pgfsys@defobject{currentmarker}{\pgfqpoint{0.000000in}{-0.027778in}}{\pgfqpoint{0.000000in}{0.000000in}}{%
\pgfpathmoveto{\pgfqpoint{0.000000in}{0.000000in}}%
\pgfpathlineto{\pgfqpoint{0.000000in}{-0.027778in}}%
\pgfusepath{stroke,fill}%
}%
\begin{pgfscope}%
\pgfsys@transformshift{2.795673in}{2.200000in}%
\pgfsys@useobject{currentmarker}{}%
\end{pgfscope}%
\end{pgfscope}%
\begin{pgfscope}%
\pgfsetbuttcap%
\pgfsetroundjoin%
\definecolor{currentfill}{rgb}{0.000000,0.000000,0.000000}%
\pgfsetfillcolor{currentfill}%
\pgfsetlinewidth{0.602250pt}%
\definecolor{currentstroke}{rgb}{0.000000,0.000000,0.000000}%
\pgfsetstrokecolor{currentstroke}%
\pgfsetdash{}{0pt}%
\pgfsys@defobject{currentmarker}{\pgfqpoint{0.000000in}{-0.027778in}}{\pgfqpoint{0.000000in}{0.000000in}}{%
\pgfpathmoveto{\pgfqpoint{0.000000in}{0.000000in}}%
\pgfpathlineto{\pgfqpoint{0.000000in}{-0.027778in}}%
\pgfusepath{stroke,fill}%
}%
\begin{pgfscope}%
\pgfsys@transformshift{2.994231in}{2.200000in}%
\pgfsys@useobject{currentmarker}{}%
\end{pgfscope}%
\end{pgfscope}%
\begin{pgfscope}%
\pgfsetbuttcap%
\pgfsetroundjoin%
\definecolor{currentfill}{rgb}{0.000000,0.000000,0.000000}%
\pgfsetfillcolor{currentfill}%
\pgfsetlinewidth{0.602250pt}%
\definecolor{currentstroke}{rgb}{0.000000,0.000000,0.000000}%
\pgfsetstrokecolor{currentstroke}%
\pgfsetdash{}{0pt}%
\pgfsys@defobject{currentmarker}{\pgfqpoint{0.000000in}{-0.027778in}}{\pgfqpoint{0.000000in}{0.000000in}}{%
\pgfpathmoveto{\pgfqpoint{0.000000in}{0.000000in}}%
\pgfpathlineto{\pgfqpoint{0.000000in}{-0.027778in}}%
\pgfusepath{stroke,fill}%
}%
\begin{pgfscope}%
\pgfsys@transformshift{3.192788in}{2.200000in}%
\pgfsys@useobject{currentmarker}{}%
\end{pgfscope}%
\end{pgfscope}%
\begin{pgfscope}%
\pgfsetbuttcap%
\pgfsetroundjoin%
\definecolor{currentfill}{rgb}{0.000000,0.000000,0.000000}%
\pgfsetfillcolor{currentfill}%
\pgfsetlinewidth{0.602250pt}%
\definecolor{currentstroke}{rgb}{0.000000,0.000000,0.000000}%
\pgfsetstrokecolor{currentstroke}%
\pgfsetdash{}{0pt}%
\pgfsys@defobject{currentmarker}{\pgfqpoint{0.000000in}{-0.027778in}}{\pgfqpoint{0.000000in}{0.000000in}}{%
\pgfpathmoveto{\pgfqpoint{0.000000in}{0.000000in}}%
\pgfpathlineto{\pgfqpoint{0.000000in}{-0.027778in}}%
\pgfusepath{stroke,fill}%
}%
\begin{pgfscope}%
\pgfsys@transformshift{3.391346in}{2.200000in}%
\pgfsys@useobject{currentmarker}{}%
\end{pgfscope}%
\end{pgfscope}%
\begin{pgfscope}%
\pgfsetbuttcap%
\pgfsetroundjoin%
\definecolor{currentfill}{rgb}{0.000000,0.000000,0.000000}%
\pgfsetfillcolor{currentfill}%
\pgfsetlinewidth{0.602250pt}%
\definecolor{currentstroke}{rgb}{0.000000,0.000000,0.000000}%
\pgfsetstrokecolor{currentstroke}%
\pgfsetdash{}{0pt}%
\pgfsys@defobject{currentmarker}{\pgfqpoint{0.000000in}{-0.027778in}}{\pgfqpoint{0.000000in}{0.000000in}}{%
\pgfpathmoveto{\pgfqpoint{0.000000in}{0.000000in}}%
\pgfpathlineto{\pgfqpoint{0.000000in}{-0.027778in}}%
\pgfusepath{stroke,fill}%
}%
\begin{pgfscope}%
\pgfsys@transformshift{3.589904in}{2.200000in}%
\pgfsys@useobject{currentmarker}{}%
\end{pgfscope}%
\end{pgfscope}%
\begin{pgfscope}%
\pgfsetbuttcap%
\pgfsetroundjoin%
\definecolor{currentfill}{rgb}{0.000000,0.000000,0.000000}%
\pgfsetfillcolor{currentfill}%
\pgfsetlinewidth{0.602250pt}%
\definecolor{currentstroke}{rgb}{0.000000,0.000000,0.000000}%
\pgfsetstrokecolor{currentstroke}%
\pgfsetdash{}{0pt}%
\pgfsys@defobject{currentmarker}{\pgfqpoint{0.000000in}{-0.027778in}}{\pgfqpoint{0.000000in}{0.000000in}}{%
\pgfpathmoveto{\pgfqpoint{0.000000in}{0.000000in}}%
\pgfpathlineto{\pgfqpoint{0.000000in}{-0.027778in}}%
\pgfusepath{stroke,fill}%
}%
\begin{pgfscope}%
\pgfsys@transformshift{3.788462in}{2.200000in}%
\pgfsys@useobject{currentmarker}{}%
\end{pgfscope}%
\end{pgfscope}%
\begin{pgfscope}%
\pgfsetbuttcap%
\pgfsetroundjoin%
\definecolor{currentfill}{rgb}{0.000000,0.000000,0.000000}%
\pgfsetfillcolor{currentfill}%
\pgfsetlinewidth{0.602250pt}%
\definecolor{currentstroke}{rgb}{0.000000,0.000000,0.000000}%
\pgfsetstrokecolor{currentstroke}%
\pgfsetdash{}{0pt}%
\pgfsys@defobject{currentmarker}{\pgfqpoint{0.000000in}{-0.027778in}}{\pgfqpoint{0.000000in}{0.000000in}}{%
\pgfpathmoveto{\pgfqpoint{0.000000in}{0.000000in}}%
\pgfpathlineto{\pgfqpoint{0.000000in}{-0.027778in}}%
\pgfusepath{stroke,fill}%
}%
\begin{pgfscope}%
\pgfsys@transformshift{3.987019in}{2.200000in}%
\pgfsys@useobject{currentmarker}{}%
\end{pgfscope}%
\end{pgfscope}%
\begin{pgfscope}%
\pgfsetbuttcap%
\pgfsetroundjoin%
\definecolor{currentfill}{rgb}{0.000000,0.000000,0.000000}%
\pgfsetfillcolor{currentfill}%
\pgfsetlinewidth{0.602250pt}%
\definecolor{currentstroke}{rgb}{0.000000,0.000000,0.000000}%
\pgfsetstrokecolor{currentstroke}%
\pgfsetdash{}{0pt}%
\pgfsys@defobject{currentmarker}{\pgfqpoint{0.000000in}{-0.027778in}}{\pgfqpoint{0.000000in}{0.000000in}}{%
\pgfpathmoveto{\pgfqpoint{0.000000in}{0.000000in}}%
\pgfpathlineto{\pgfqpoint{0.000000in}{-0.027778in}}%
\pgfusepath{stroke,fill}%
}%
\begin{pgfscope}%
\pgfsys@transformshift{4.185577in}{2.200000in}%
\pgfsys@useobject{currentmarker}{}%
\end{pgfscope}%
\end{pgfscope}%
\begin{pgfscope}%
\pgfsetbuttcap%
\pgfsetroundjoin%
\definecolor{currentfill}{rgb}{0.000000,0.000000,0.000000}%
\pgfsetfillcolor{currentfill}%
\pgfsetlinewidth{0.803000pt}%
\definecolor{currentstroke}{rgb}{0.000000,0.000000,0.000000}%
\pgfsetstrokecolor{currentstroke}%
\pgfsetdash{}{0pt}%
\pgfsys@defobject{currentmarker}{\pgfqpoint{-0.048611in}{0.000000in}}{\pgfqpoint{0.000000in}{0.000000in}}{%
\pgfpathmoveto{\pgfqpoint{0.000000in}{0.000000in}}%
\pgfpathlineto{\pgfqpoint{-0.048611in}{0.000000in}}%
\pgfusepath{stroke,fill}%
}%
\begin{pgfscope}%
\pgfsys@transformshift{2.200000in}{0.214423in}%
\pgfsys@useobject{currentmarker}{}%
\end{pgfscope}%
\end{pgfscope}%
\begin{pgfscope}%
\pgftext[x=1.964296in,y=0.161662in,left,base]{\sffamily\fontsize{10.000000}{12.000000}\selectfont -1}%
\end{pgfscope}%
\begin{pgfscope}%
\pgfsetbuttcap%
\pgfsetroundjoin%
\definecolor{currentfill}{rgb}{0.000000,0.000000,0.000000}%
\pgfsetfillcolor{currentfill}%
\pgfsetlinewidth{0.803000pt}%
\definecolor{currentstroke}{rgb}{0.000000,0.000000,0.000000}%
\pgfsetstrokecolor{currentstroke}%
\pgfsetdash{}{0pt}%
\pgfsys@defobject{currentmarker}{\pgfqpoint{-0.048611in}{0.000000in}}{\pgfqpoint{0.000000in}{0.000000in}}{%
\pgfpathmoveto{\pgfqpoint{0.000000in}{0.000000in}}%
\pgfpathlineto{\pgfqpoint{-0.048611in}{0.000000in}}%
\pgfusepath{stroke,fill}%
}%
\begin{pgfscope}%
\pgfsys@transformshift{2.200000in}{1.207212in}%
\pgfsys@useobject{currentmarker}{}%
\end{pgfscope}%
\end{pgfscope}%
\begin{pgfscope}%
\pgftext[x=1.831782in,y=1.154450in,left,base]{\sffamily\fontsize{10.000000}{12.000000}\selectfont -0.5}%
\end{pgfscope}%
\begin{pgfscope}%
\pgfsetbuttcap%
\pgfsetroundjoin%
\definecolor{currentfill}{rgb}{0.000000,0.000000,0.000000}%
\pgfsetfillcolor{currentfill}%
\pgfsetlinewidth{0.803000pt}%
\definecolor{currentstroke}{rgb}{0.000000,0.000000,0.000000}%
\pgfsetstrokecolor{currentstroke}%
\pgfsetdash{}{0pt}%
\pgfsys@defobject{currentmarker}{\pgfqpoint{-0.048611in}{0.000000in}}{\pgfqpoint{0.000000in}{0.000000in}}{%
\pgfpathmoveto{\pgfqpoint{0.000000in}{0.000000in}}%
\pgfpathlineto{\pgfqpoint{-0.048611in}{0.000000in}}%
\pgfusepath{stroke,fill}%
}%
\begin{pgfscope}%
\pgfsys@transformshift{2.200000in}{2.200000in}%
\pgfsys@useobject{currentmarker}{}%
\end{pgfscope}%
\end{pgfscope}%
\begin{pgfscope}%
\pgfsetbuttcap%
\pgfsetroundjoin%
\definecolor{currentfill}{rgb}{0.000000,0.000000,0.000000}%
\pgfsetfillcolor{currentfill}%
\pgfsetlinewidth{0.803000pt}%
\definecolor{currentstroke}{rgb}{0.000000,0.000000,0.000000}%
\pgfsetstrokecolor{currentstroke}%
\pgfsetdash{}{0pt}%
\pgfsys@defobject{currentmarker}{\pgfqpoint{-0.048611in}{0.000000in}}{\pgfqpoint{0.000000in}{0.000000in}}{%
\pgfpathmoveto{\pgfqpoint{0.000000in}{0.000000in}}%
\pgfpathlineto{\pgfqpoint{-0.048611in}{0.000000in}}%
\pgfusepath{stroke,fill}%
}%
\begin{pgfscope}%
\pgfsys@transformshift{2.200000in}{3.192788in}%
\pgfsys@useobject{currentmarker}{}%
\end{pgfscope}%
\end{pgfscope}%
\begin{pgfscope}%
\pgftext[x=1.881898in,y=3.140027in,left,base]{\sffamily\fontsize{10.000000}{12.000000}\selectfont 0.5}%
\end{pgfscope}%
\begin{pgfscope}%
\pgfsetbuttcap%
\pgfsetroundjoin%
\definecolor{currentfill}{rgb}{0.000000,0.000000,0.000000}%
\pgfsetfillcolor{currentfill}%
\pgfsetlinewidth{0.803000pt}%
\definecolor{currentstroke}{rgb}{0.000000,0.000000,0.000000}%
\pgfsetstrokecolor{currentstroke}%
\pgfsetdash{}{0pt}%
\pgfsys@defobject{currentmarker}{\pgfqpoint{-0.048611in}{0.000000in}}{\pgfqpoint{0.000000in}{0.000000in}}{%
\pgfpathmoveto{\pgfqpoint{0.000000in}{0.000000in}}%
\pgfpathlineto{\pgfqpoint{-0.048611in}{0.000000in}}%
\pgfusepath{stroke,fill}%
}%
\begin{pgfscope}%
\pgfsys@transformshift{2.200000in}{4.185577in}%
\pgfsys@useobject{currentmarker}{}%
\end{pgfscope}%
\end{pgfscope}%
\begin{pgfscope}%
\pgftext[x=2.014413in,y=4.132815in,left,base]{\sffamily\fontsize{10.000000}{12.000000}\selectfont 1}%
\end{pgfscope}%
\begin{pgfscope}%
\pgfsetbuttcap%
\pgfsetroundjoin%
\definecolor{currentfill}{rgb}{0.000000,0.000000,0.000000}%
\pgfsetfillcolor{currentfill}%
\pgfsetlinewidth{0.602250pt}%
\definecolor{currentstroke}{rgb}{0.000000,0.000000,0.000000}%
\pgfsetstrokecolor{currentstroke}%
\pgfsetdash{}{0pt}%
\pgfsys@defobject{currentmarker}{\pgfqpoint{-0.027778in}{0.000000in}}{\pgfqpoint{0.000000in}{0.000000in}}{%
\pgfpathmoveto{\pgfqpoint{0.000000in}{0.000000in}}%
\pgfpathlineto{\pgfqpoint{-0.027778in}{0.000000in}}%
\pgfusepath{stroke,fill}%
}%
\begin{pgfscope}%
\pgfsys@transformshift{2.200000in}{0.412981in}%
\pgfsys@useobject{currentmarker}{}%
\end{pgfscope}%
\end{pgfscope}%
\begin{pgfscope}%
\pgfsetbuttcap%
\pgfsetroundjoin%
\definecolor{currentfill}{rgb}{0.000000,0.000000,0.000000}%
\pgfsetfillcolor{currentfill}%
\pgfsetlinewidth{0.602250pt}%
\definecolor{currentstroke}{rgb}{0.000000,0.000000,0.000000}%
\pgfsetstrokecolor{currentstroke}%
\pgfsetdash{}{0pt}%
\pgfsys@defobject{currentmarker}{\pgfqpoint{-0.027778in}{0.000000in}}{\pgfqpoint{0.000000in}{0.000000in}}{%
\pgfpathmoveto{\pgfqpoint{0.000000in}{0.000000in}}%
\pgfpathlineto{\pgfqpoint{-0.027778in}{0.000000in}}%
\pgfusepath{stroke,fill}%
}%
\begin{pgfscope}%
\pgfsys@transformshift{2.200000in}{0.611538in}%
\pgfsys@useobject{currentmarker}{}%
\end{pgfscope}%
\end{pgfscope}%
\begin{pgfscope}%
\pgfsetbuttcap%
\pgfsetroundjoin%
\definecolor{currentfill}{rgb}{0.000000,0.000000,0.000000}%
\pgfsetfillcolor{currentfill}%
\pgfsetlinewidth{0.602250pt}%
\definecolor{currentstroke}{rgb}{0.000000,0.000000,0.000000}%
\pgfsetstrokecolor{currentstroke}%
\pgfsetdash{}{0pt}%
\pgfsys@defobject{currentmarker}{\pgfqpoint{-0.027778in}{0.000000in}}{\pgfqpoint{0.000000in}{0.000000in}}{%
\pgfpathmoveto{\pgfqpoint{0.000000in}{0.000000in}}%
\pgfpathlineto{\pgfqpoint{-0.027778in}{0.000000in}}%
\pgfusepath{stroke,fill}%
}%
\begin{pgfscope}%
\pgfsys@transformshift{2.200000in}{0.810096in}%
\pgfsys@useobject{currentmarker}{}%
\end{pgfscope}%
\end{pgfscope}%
\begin{pgfscope}%
\pgfsetbuttcap%
\pgfsetroundjoin%
\definecolor{currentfill}{rgb}{0.000000,0.000000,0.000000}%
\pgfsetfillcolor{currentfill}%
\pgfsetlinewidth{0.602250pt}%
\definecolor{currentstroke}{rgb}{0.000000,0.000000,0.000000}%
\pgfsetstrokecolor{currentstroke}%
\pgfsetdash{}{0pt}%
\pgfsys@defobject{currentmarker}{\pgfqpoint{-0.027778in}{0.000000in}}{\pgfqpoint{0.000000in}{0.000000in}}{%
\pgfpathmoveto{\pgfqpoint{0.000000in}{0.000000in}}%
\pgfpathlineto{\pgfqpoint{-0.027778in}{0.000000in}}%
\pgfusepath{stroke,fill}%
}%
\begin{pgfscope}%
\pgfsys@transformshift{2.200000in}{1.008654in}%
\pgfsys@useobject{currentmarker}{}%
\end{pgfscope}%
\end{pgfscope}%
\begin{pgfscope}%
\pgfsetbuttcap%
\pgfsetroundjoin%
\definecolor{currentfill}{rgb}{0.000000,0.000000,0.000000}%
\pgfsetfillcolor{currentfill}%
\pgfsetlinewidth{0.602250pt}%
\definecolor{currentstroke}{rgb}{0.000000,0.000000,0.000000}%
\pgfsetstrokecolor{currentstroke}%
\pgfsetdash{}{0pt}%
\pgfsys@defobject{currentmarker}{\pgfqpoint{-0.027778in}{0.000000in}}{\pgfqpoint{0.000000in}{0.000000in}}{%
\pgfpathmoveto{\pgfqpoint{0.000000in}{0.000000in}}%
\pgfpathlineto{\pgfqpoint{-0.027778in}{0.000000in}}%
\pgfusepath{stroke,fill}%
}%
\begin{pgfscope}%
\pgfsys@transformshift{2.200000in}{1.207212in}%
\pgfsys@useobject{currentmarker}{}%
\end{pgfscope}%
\end{pgfscope}%
\begin{pgfscope}%
\pgfsetbuttcap%
\pgfsetroundjoin%
\definecolor{currentfill}{rgb}{0.000000,0.000000,0.000000}%
\pgfsetfillcolor{currentfill}%
\pgfsetlinewidth{0.602250pt}%
\definecolor{currentstroke}{rgb}{0.000000,0.000000,0.000000}%
\pgfsetstrokecolor{currentstroke}%
\pgfsetdash{}{0pt}%
\pgfsys@defobject{currentmarker}{\pgfqpoint{-0.027778in}{0.000000in}}{\pgfqpoint{0.000000in}{0.000000in}}{%
\pgfpathmoveto{\pgfqpoint{0.000000in}{0.000000in}}%
\pgfpathlineto{\pgfqpoint{-0.027778in}{0.000000in}}%
\pgfusepath{stroke,fill}%
}%
\begin{pgfscope}%
\pgfsys@transformshift{2.200000in}{1.405769in}%
\pgfsys@useobject{currentmarker}{}%
\end{pgfscope}%
\end{pgfscope}%
\begin{pgfscope}%
\pgfsetbuttcap%
\pgfsetroundjoin%
\definecolor{currentfill}{rgb}{0.000000,0.000000,0.000000}%
\pgfsetfillcolor{currentfill}%
\pgfsetlinewidth{0.602250pt}%
\definecolor{currentstroke}{rgb}{0.000000,0.000000,0.000000}%
\pgfsetstrokecolor{currentstroke}%
\pgfsetdash{}{0pt}%
\pgfsys@defobject{currentmarker}{\pgfqpoint{-0.027778in}{0.000000in}}{\pgfqpoint{0.000000in}{0.000000in}}{%
\pgfpathmoveto{\pgfqpoint{0.000000in}{0.000000in}}%
\pgfpathlineto{\pgfqpoint{-0.027778in}{0.000000in}}%
\pgfusepath{stroke,fill}%
}%
\begin{pgfscope}%
\pgfsys@transformshift{2.200000in}{1.604327in}%
\pgfsys@useobject{currentmarker}{}%
\end{pgfscope}%
\end{pgfscope}%
\begin{pgfscope}%
\pgfsetbuttcap%
\pgfsetroundjoin%
\definecolor{currentfill}{rgb}{0.000000,0.000000,0.000000}%
\pgfsetfillcolor{currentfill}%
\pgfsetlinewidth{0.602250pt}%
\definecolor{currentstroke}{rgb}{0.000000,0.000000,0.000000}%
\pgfsetstrokecolor{currentstroke}%
\pgfsetdash{}{0pt}%
\pgfsys@defobject{currentmarker}{\pgfqpoint{-0.027778in}{0.000000in}}{\pgfqpoint{0.000000in}{0.000000in}}{%
\pgfpathmoveto{\pgfqpoint{0.000000in}{0.000000in}}%
\pgfpathlineto{\pgfqpoint{-0.027778in}{0.000000in}}%
\pgfusepath{stroke,fill}%
}%
\begin{pgfscope}%
\pgfsys@transformshift{2.200000in}{1.802885in}%
\pgfsys@useobject{currentmarker}{}%
\end{pgfscope}%
\end{pgfscope}%
\begin{pgfscope}%
\pgfsetbuttcap%
\pgfsetroundjoin%
\definecolor{currentfill}{rgb}{0.000000,0.000000,0.000000}%
\pgfsetfillcolor{currentfill}%
\pgfsetlinewidth{0.602250pt}%
\definecolor{currentstroke}{rgb}{0.000000,0.000000,0.000000}%
\pgfsetstrokecolor{currentstroke}%
\pgfsetdash{}{0pt}%
\pgfsys@defobject{currentmarker}{\pgfqpoint{-0.027778in}{0.000000in}}{\pgfqpoint{0.000000in}{0.000000in}}{%
\pgfpathmoveto{\pgfqpoint{0.000000in}{0.000000in}}%
\pgfpathlineto{\pgfqpoint{-0.027778in}{0.000000in}}%
\pgfusepath{stroke,fill}%
}%
\begin{pgfscope}%
\pgfsys@transformshift{2.200000in}{2.001442in}%
\pgfsys@useobject{currentmarker}{}%
\end{pgfscope}%
\end{pgfscope}%
\begin{pgfscope}%
\pgfsetbuttcap%
\pgfsetroundjoin%
\definecolor{currentfill}{rgb}{0.000000,0.000000,0.000000}%
\pgfsetfillcolor{currentfill}%
\pgfsetlinewidth{0.602250pt}%
\definecolor{currentstroke}{rgb}{0.000000,0.000000,0.000000}%
\pgfsetstrokecolor{currentstroke}%
\pgfsetdash{}{0pt}%
\pgfsys@defobject{currentmarker}{\pgfqpoint{-0.027778in}{0.000000in}}{\pgfqpoint{0.000000in}{0.000000in}}{%
\pgfpathmoveto{\pgfqpoint{0.000000in}{0.000000in}}%
\pgfpathlineto{\pgfqpoint{-0.027778in}{0.000000in}}%
\pgfusepath{stroke,fill}%
}%
\begin{pgfscope}%
\pgfsys@transformshift{2.200000in}{2.200000in}%
\pgfsys@useobject{currentmarker}{}%
\end{pgfscope}%
\end{pgfscope}%
\begin{pgfscope}%
\pgfsetbuttcap%
\pgfsetroundjoin%
\definecolor{currentfill}{rgb}{0.000000,0.000000,0.000000}%
\pgfsetfillcolor{currentfill}%
\pgfsetlinewidth{0.602250pt}%
\definecolor{currentstroke}{rgb}{0.000000,0.000000,0.000000}%
\pgfsetstrokecolor{currentstroke}%
\pgfsetdash{}{0pt}%
\pgfsys@defobject{currentmarker}{\pgfqpoint{-0.027778in}{0.000000in}}{\pgfqpoint{0.000000in}{0.000000in}}{%
\pgfpathmoveto{\pgfqpoint{0.000000in}{0.000000in}}%
\pgfpathlineto{\pgfqpoint{-0.027778in}{0.000000in}}%
\pgfusepath{stroke,fill}%
}%
\begin{pgfscope}%
\pgfsys@transformshift{2.200000in}{2.398558in}%
\pgfsys@useobject{currentmarker}{}%
\end{pgfscope}%
\end{pgfscope}%
\begin{pgfscope}%
\pgfsetbuttcap%
\pgfsetroundjoin%
\definecolor{currentfill}{rgb}{0.000000,0.000000,0.000000}%
\pgfsetfillcolor{currentfill}%
\pgfsetlinewidth{0.602250pt}%
\definecolor{currentstroke}{rgb}{0.000000,0.000000,0.000000}%
\pgfsetstrokecolor{currentstroke}%
\pgfsetdash{}{0pt}%
\pgfsys@defobject{currentmarker}{\pgfqpoint{-0.027778in}{0.000000in}}{\pgfqpoint{0.000000in}{0.000000in}}{%
\pgfpathmoveto{\pgfqpoint{0.000000in}{0.000000in}}%
\pgfpathlineto{\pgfqpoint{-0.027778in}{0.000000in}}%
\pgfusepath{stroke,fill}%
}%
\begin{pgfscope}%
\pgfsys@transformshift{2.200000in}{2.597115in}%
\pgfsys@useobject{currentmarker}{}%
\end{pgfscope}%
\end{pgfscope}%
\begin{pgfscope}%
\pgfsetbuttcap%
\pgfsetroundjoin%
\definecolor{currentfill}{rgb}{0.000000,0.000000,0.000000}%
\pgfsetfillcolor{currentfill}%
\pgfsetlinewidth{0.602250pt}%
\definecolor{currentstroke}{rgb}{0.000000,0.000000,0.000000}%
\pgfsetstrokecolor{currentstroke}%
\pgfsetdash{}{0pt}%
\pgfsys@defobject{currentmarker}{\pgfqpoint{-0.027778in}{0.000000in}}{\pgfqpoint{0.000000in}{0.000000in}}{%
\pgfpathmoveto{\pgfqpoint{0.000000in}{0.000000in}}%
\pgfpathlineto{\pgfqpoint{-0.027778in}{0.000000in}}%
\pgfusepath{stroke,fill}%
}%
\begin{pgfscope}%
\pgfsys@transformshift{2.200000in}{2.795673in}%
\pgfsys@useobject{currentmarker}{}%
\end{pgfscope}%
\end{pgfscope}%
\begin{pgfscope}%
\pgfsetbuttcap%
\pgfsetroundjoin%
\definecolor{currentfill}{rgb}{0.000000,0.000000,0.000000}%
\pgfsetfillcolor{currentfill}%
\pgfsetlinewidth{0.602250pt}%
\definecolor{currentstroke}{rgb}{0.000000,0.000000,0.000000}%
\pgfsetstrokecolor{currentstroke}%
\pgfsetdash{}{0pt}%
\pgfsys@defobject{currentmarker}{\pgfqpoint{-0.027778in}{0.000000in}}{\pgfqpoint{0.000000in}{0.000000in}}{%
\pgfpathmoveto{\pgfqpoint{0.000000in}{0.000000in}}%
\pgfpathlineto{\pgfqpoint{-0.027778in}{0.000000in}}%
\pgfusepath{stroke,fill}%
}%
\begin{pgfscope}%
\pgfsys@transformshift{2.200000in}{2.994231in}%
\pgfsys@useobject{currentmarker}{}%
\end{pgfscope}%
\end{pgfscope}%
\begin{pgfscope}%
\pgfsetbuttcap%
\pgfsetroundjoin%
\definecolor{currentfill}{rgb}{0.000000,0.000000,0.000000}%
\pgfsetfillcolor{currentfill}%
\pgfsetlinewidth{0.602250pt}%
\definecolor{currentstroke}{rgb}{0.000000,0.000000,0.000000}%
\pgfsetstrokecolor{currentstroke}%
\pgfsetdash{}{0pt}%
\pgfsys@defobject{currentmarker}{\pgfqpoint{-0.027778in}{0.000000in}}{\pgfqpoint{0.000000in}{0.000000in}}{%
\pgfpathmoveto{\pgfqpoint{0.000000in}{0.000000in}}%
\pgfpathlineto{\pgfqpoint{-0.027778in}{0.000000in}}%
\pgfusepath{stroke,fill}%
}%
\begin{pgfscope}%
\pgfsys@transformshift{2.200000in}{3.192788in}%
\pgfsys@useobject{currentmarker}{}%
\end{pgfscope}%
\end{pgfscope}%
\begin{pgfscope}%
\pgfsetbuttcap%
\pgfsetroundjoin%
\definecolor{currentfill}{rgb}{0.000000,0.000000,0.000000}%
\pgfsetfillcolor{currentfill}%
\pgfsetlinewidth{0.602250pt}%
\definecolor{currentstroke}{rgb}{0.000000,0.000000,0.000000}%
\pgfsetstrokecolor{currentstroke}%
\pgfsetdash{}{0pt}%
\pgfsys@defobject{currentmarker}{\pgfqpoint{-0.027778in}{0.000000in}}{\pgfqpoint{0.000000in}{0.000000in}}{%
\pgfpathmoveto{\pgfqpoint{0.000000in}{0.000000in}}%
\pgfpathlineto{\pgfqpoint{-0.027778in}{0.000000in}}%
\pgfusepath{stroke,fill}%
}%
\begin{pgfscope}%
\pgfsys@transformshift{2.200000in}{3.391346in}%
\pgfsys@useobject{currentmarker}{}%
\end{pgfscope}%
\end{pgfscope}%
\begin{pgfscope}%
\pgfsetbuttcap%
\pgfsetroundjoin%
\definecolor{currentfill}{rgb}{0.000000,0.000000,0.000000}%
\pgfsetfillcolor{currentfill}%
\pgfsetlinewidth{0.602250pt}%
\definecolor{currentstroke}{rgb}{0.000000,0.000000,0.000000}%
\pgfsetstrokecolor{currentstroke}%
\pgfsetdash{}{0pt}%
\pgfsys@defobject{currentmarker}{\pgfqpoint{-0.027778in}{0.000000in}}{\pgfqpoint{0.000000in}{0.000000in}}{%
\pgfpathmoveto{\pgfqpoint{0.000000in}{0.000000in}}%
\pgfpathlineto{\pgfqpoint{-0.027778in}{0.000000in}}%
\pgfusepath{stroke,fill}%
}%
\begin{pgfscope}%
\pgfsys@transformshift{2.200000in}{3.589904in}%
\pgfsys@useobject{currentmarker}{}%
\end{pgfscope}%
\end{pgfscope}%
\begin{pgfscope}%
\pgfsetbuttcap%
\pgfsetroundjoin%
\definecolor{currentfill}{rgb}{0.000000,0.000000,0.000000}%
\pgfsetfillcolor{currentfill}%
\pgfsetlinewidth{0.602250pt}%
\definecolor{currentstroke}{rgb}{0.000000,0.000000,0.000000}%
\pgfsetstrokecolor{currentstroke}%
\pgfsetdash{}{0pt}%
\pgfsys@defobject{currentmarker}{\pgfqpoint{-0.027778in}{0.000000in}}{\pgfqpoint{0.000000in}{0.000000in}}{%
\pgfpathmoveto{\pgfqpoint{0.000000in}{0.000000in}}%
\pgfpathlineto{\pgfqpoint{-0.027778in}{0.000000in}}%
\pgfusepath{stroke,fill}%
}%
\begin{pgfscope}%
\pgfsys@transformshift{2.200000in}{3.788462in}%
\pgfsys@useobject{currentmarker}{}%
\end{pgfscope}%
\end{pgfscope}%
\begin{pgfscope}%
\pgfsetbuttcap%
\pgfsetroundjoin%
\definecolor{currentfill}{rgb}{0.000000,0.000000,0.000000}%
\pgfsetfillcolor{currentfill}%
\pgfsetlinewidth{0.602250pt}%
\definecolor{currentstroke}{rgb}{0.000000,0.000000,0.000000}%
\pgfsetstrokecolor{currentstroke}%
\pgfsetdash{}{0pt}%
\pgfsys@defobject{currentmarker}{\pgfqpoint{-0.027778in}{0.000000in}}{\pgfqpoint{0.000000in}{0.000000in}}{%
\pgfpathmoveto{\pgfqpoint{0.000000in}{0.000000in}}%
\pgfpathlineto{\pgfqpoint{-0.027778in}{0.000000in}}%
\pgfusepath{stroke,fill}%
}%
\begin{pgfscope}%
\pgfsys@transformshift{2.200000in}{3.987019in}%
\pgfsys@useobject{currentmarker}{}%
\end{pgfscope}%
\end{pgfscope}%
\begin{pgfscope}%
\pgfsetbuttcap%
\pgfsetroundjoin%
\definecolor{currentfill}{rgb}{0.000000,0.000000,0.000000}%
\pgfsetfillcolor{currentfill}%
\pgfsetlinewidth{0.602250pt}%
\definecolor{currentstroke}{rgb}{0.000000,0.000000,0.000000}%
\pgfsetstrokecolor{currentstroke}%
\pgfsetdash{}{0pt}%
\pgfsys@defobject{currentmarker}{\pgfqpoint{-0.027778in}{0.000000in}}{\pgfqpoint{0.000000in}{0.000000in}}{%
\pgfpathmoveto{\pgfqpoint{0.000000in}{0.000000in}}%
\pgfpathlineto{\pgfqpoint{-0.027778in}{0.000000in}}%
\pgfusepath{stroke,fill}%
}%
\begin{pgfscope}%
\pgfsys@transformshift{2.200000in}{4.185577in}%
\pgfsys@useobject{currentmarker}{}%
\end{pgfscope}%
\end{pgfscope}%
\begin{pgfscope}%
\pgfsetrectcap%
\pgfsetmiterjoin%
\pgfsetlinewidth{0.803000pt}%
\definecolor{currentstroke}{rgb}{0.000000,0.000000,0.000000}%
\pgfsetstrokecolor{currentstroke}%
\pgfsetdash{}{0pt}%
\pgfpathmoveto{\pgfqpoint{2.200000in}{0.135000in}}%
\pgfpathlineto{\pgfqpoint{2.200000in}{4.265000in}}%
\pgfusepath{stroke}%
\end{pgfscope}%
\begin{pgfscope}%
\pgfsetrectcap%
\pgfsetmiterjoin%
\pgfsetlinewidth{0.803000pt}%
\definecolor{currentstroke}{rgb}{0.000000,0.000000,0.000000}%
\pgfsetstrokecolor{currentstroke}%
\pgfsetdash{}{0pt}%
\pgfpathmoveto{\pgfqpoint{0.135000in}{2.200000in}}%
\pgfpathlineto{\pgfqpoint{4.265000in}{2.200000in}}%
\pgfusepath{stroke}%
\end{pgfscope}%
\begin{pgfscope}%
\pgftext[x=2.299279in,y=2.299279in,,]{\sffamily\fontsize{10.000000}{12.000000}\selectfont O}%
\end{pgfscope}%
\begin{pgfscope}%
\pgftext[x=4.096226in,y=4.096226in,,]{\sffamily\fontsize{10.000000}{12.000000}\selectfont P}%
\end{pgfscope}%
\begin{pgfscope}%
\pgfpathrectangle{\pgfqpoint{0.135000in}{0.135000in}}{\pgfqpoint{4.130000in}{4.130000in}} %
\pgfusepath{clip}%
\pgfsetbuttcap%
\pgfsetmiterjoin%
\definecolor{currentfill}{rgb}{0.000000,0.000000,1.000000}%
\pgfsetfillcolor{currentfill}%
\pgfsetfillopacity{0.300000}%
\pgfsetlinewidth{1.003750pt}%
\definecolor{currentstroke}{rgb}{0.000000,0.000000,1.000000}%
\pgfsetstrokecolor{currentstroke}%
\pgfsetstrokeopacity{0.300000}%
\pgfsetdash{}{0pt}%
\pgfpathmoveto{\pgfqpoint{2.200000in}{0.214423in}}%
\pgfpathcurveto{\pgfqpoint{2.726581in}{0.214423in}}{\pgfqpoint{3.231666in}{0.423636in}}{\pgfqpoint{3.604015in}{0.795985in}}%
\pgfpathcurveto{\pgfqpoint{3.976364in}{1.168334in}}{\pgfqpoint{4.185577in}{1.673419in}}{\pgfqpoint{4.185577in}{2.200000in}}%
\pgfpathcurveto{\pgfqpoint{4.185577in}{2.726581in}}{\pgfqpoint{3.976364in}{3.231666in}}{\pgfqpoint{3.604015in}{3.604015in}}%
\pgfpathcurveto{\pgfqpoint{3.231666in}{3.976364in}}{\pgfqpoint{2.726581in}{4.185577in}}{\pgfqpoint{2.200000in}{4.185577in}}%
\pgfpathcurveto{\pgfqpoint{1.673419in}{4.185577in}}{\pgfqpoint{1.168334in}{3.976364in}}{\pgfqpoint{0.795985in}{3.604015in}}%
\pgfpathcurveto{\pgfqpoint{0.423636in}{3.231666in}}{\pgfqpoint{0.214423in}{2.726581in}}{\pgfqpoint{0.214423in}{2.200000in}}%
\pgfpathcurveto{\pgfqpoint{0.214423in}{1.673419in}}{\pgfqpoint{0.423636in}{1.168334in}}{\pgfqpoint{0.795985in}{0.795985in}}%
\pgfpathcurveto{\pgfqpoint{1.168334in}{0.423636in}}{\pgfqpoint{1.673419in}{0.214423in}}{\pgfqpoint{2.200000in}{0.214423in}}%
\pgfpathclose%
\pgfusepath{stroke,fill}%
\end{pgfscope}%
\begin{pgfscope}%
\pgfpathrectangle{\pgfqpoint{0.135000in}{0.135000in}}{\pgfqpoint{4.130000in}{4.130000in}} %
\pgfusepath{clip}%
\pgfsetbuttcap%
\pgfsetmiterjoin%
\definecolor{currentfill}{rgb}{1.000000,0.000000,0.000000}%
\pgfsetfillcolor{currentfill}%
\pgfsetfillopacity{0.700000}%
\pgfsetlinewidth{1.003750pt}%
\definecolor{currentstroke}{rgb}{1.000000,0.000000,0.000000}%
\pgfsetstrokecolor{currentstroke}%
\pgfsetstrokeopacity{0.700000}%
\pgfsetdash{}{0pt}%
\pgfpathmoveto{\pgfqpoint{2.200000in}{1.036876in}}%
\pgfpathcurveto{\pgfqpoint{2.508464in}{1.036876in}}{\pgfqpoint{2.804336in}{1.159430in}}{\pgfqpoint{3.022453in}{1.377547in}}%
\pgfpathcurveto{\pgfqpoint{3.240570in}{1.595664in}}{\pgfqpoint{3.363124in}{1.891536in}}{\pgfqpoint{3.363124in}{2.200000in}}%
\pgfpathcurveto{\pgfqpoint{3.363124in}{2.508464in}}{\pgfqpoint{3.240570in}{2.804336in}}{\pgfqpoint{3.022453in}{3.022453in}}%
\pgfpathcurveto{\pgfqpoint{2.804336in}{3.240570in}}{\pgfqpoint{2.508464in}{3.363124in}}{\pgfqpoint{2.200000in}{3.363124in}}%
\pgfpathcurveto{\pgfqpoint{1.891536in}{3.363124in}}{\pgfqpoint{1.595664in}{3.240570in}}{\pgfqpoint{1.377547in}{3.022453in}}%
\pgfpathcurveto{\pgfqpoint{1.159430in}{2.804336in}}{\pgfqpoint{1.036876in}{2.508464in}}{\pgfqpoint{1.036876in}{2.200000in}}%
\pgfpathcurveto{\pgfqpoint{1.036876in}{1.891536in}}{\pgfqpoint{1.159430in}{1.595664in}}{\pgfqpoint{1.377547in}{1.377547in}}%
\pgfpathcurveto{\pgfqpoint{1.595664in}{1.159430in}}{\pgfqpoint{1.891536in}{1.036876in}}{\pgfqpoint{2.200000in}{1.036876in}}%
\pgfpathclose%
\pgfusepath{stroke,fill}%
\end{pgfscope}%
\begin{pgfscope}%
\pgfsetbuttcap%
\pgfsetmiterjoin%
\definecolor{currentfill}{rgb}{0.300000,0.300000,0.300000}%
\pgfsetfillcolor{currentfill}%
\pgfsetfillopacity{0.500000}%
\pgfsetlinewidth{1.003750pt}%
\definecolor{currentstroke}{rgb}{0.300000,0.300000,0.300000}%
\pgfsetstrokecolor{currentstroke}%
\pgfsetstrokeopacity{0.500000}%
\pgfsetdash{}{0pt}%
\pgfpathmoveto{\pgfqpoint{0.232222in}{3.729508in}}%
\pgfpathlineto{\pgfqpoint{0.964915in}{3.729508in}}%
\pgfpathlineto{\pgfqpoint{0.964915in}{4.167778in}}%
\pgfpathlineto{\pgfqpoint{0.232222in}{4.167778in}}%
\pgfpathclose%
\pgfusepath{stroke,fill}%
\end{pgfscope}%
\begin{pgfscope}%
\pgfsetbuttcap%
\pgfsetmiterjoin%
\definecolor{currentfill}{rgb}{1.000000,1.000000,1.000000}%
\pgfsetfillcolor{currentfill}%
\pgfsetfillopacity{0.800000}%
\pgfsetlinewidth{1.003750pt}%
\definecolor{currentstroke}{rgb}{0.800000,0.800000,0.800000}%
\pgfsetstrokecolor{currentstroke}%
\pgfsetstrokeopacity{0.800000}%
\pgfsetdash{}{0pt}%
\pgfpathmoveto{\pgfqpoint{0.204444in}{3.757286in}}%
\pgfpathlineto{\pgfqpoint{0.937137in}{3.757286in}}%
\pgfpathlineto{\pgfqpoint{0.937137in}{4.195556in}}%
\pgfpathlineto{\pgfqpoint{0.204444in}{4.195556in}}%
\pgfpathclose%
\pgfusepath{stroke,fill}%
\end{pgfscope}%
\begin{pgfscope}%
\pgfsetbuttcap%
\pgfsetmiterjoin%
\definecolor{currentfill}{rgb}{0.000000,0.000000,1.000000}%
\pgfsetfillcolor{currentfill}%
\pgfsetfillopacity{0.300000}%
\pgfsetlinewidth{1.003750pt}%
\definecolor{currentstroke}{rgb}{0.000000,0.000000,1.000000}%
\pgfsetstrokecolor{currentstroke}%
\pgfsetstrokeopacity{0.300000}%
\pgfsetdash{}{0pt}%
\pgfpathmoveto{\pgfqpoint{0.287778in}{4.006699in}}%
\pgfpathlineto{\pgfqpoint{0.294722in}{4.006699in}}%
\pgfpathlineto{\pgfqpoint{0.294722in}{4.103921in}}%
\pgfpathlineto{\pgfqpoint{0.287778in}{4.103921in}}%
\pgfpathclose%
\pgfusepath{stroke,fill}%
\end{pgfscope}%
\begin{pgfscope}%
\pgftext[x=0.364167in,y=4.006699in,left,base]{\sffamily\fontsize{10.000000}{12.000000}\selectfont B(O, 1)}%
\end{pgfscope}%
\begin{pgfscope}%
\pgfsetbuttcap%
\pgfsetmiterjoin%
\definecolor{currentfill}{rgb}{1.000000,0.000000,0.000000}%
\pgfsetfillcolor{currentfill}%
\pgfsetfillopacity{0.700000}%
\pgfsetlinewidth{1.003750pt}%
\definecolor{currentstroke}{rgb}{1.000000,0.000000,0.000000}%
\pgfsetstrokecolor{currentstroke}%
\pgfsetstrokeopacity{0.700000}%
\pgfsetdash{}{0pt}%
\pgfpathmoveto{\pgfqpoint{0.287778in}{3.869509in}}%
\pgfpathlineto{\pgfqpoint{0.294722in}{3.869509in}}%
\pgfpathlineto{\pgfqpoint{0.294722in}{3.966731in}}%
\pgfpathlineto{\pgfqpoint{0.287778in}{3.966731in}}%
\pgfpathclose%
\pgfusepath{stroke,fill}%
\end{pgfscope}%
\begin{pgfscope}%
\pgftext[x=0.364167in,y=3.869509in,left,base]{\sffamily\fontsize{10.000000}{12.000000}\selectfont B(P,2)}%
\end{pgfscope}%
\end{pgfpicture}%
\makeatother%
\endgroup%

    \end{center}
\end{ej}

\begin{ej}
    Tenemos que comprobar que
    \begin{enumerate}[i)]
        \item $d(a,b) \geq 0 \; \forall a,b \in A^n$ y que $d(a,b) = 0 \iff a = b$, lo cual es trivial por la definición de $d$.
        \item $d(a,b) = d(b,a) \; \forall a,b \in A^n$. Trivial por la definición.
        \item $d(a, b) \leq d(a,c) + d(c, b)$. Sea $i \in R = \setb{j \vert a_j \neq b_j}$, entonces, se cumple al menos una de las dos siguientes afirmaciones
            \[
                \begin{cases}
                    i \in P = \setb{j \vert a_j \neq c_j} \\
                    i \in Q = \setb{j \vert c_j \neq b_j}
                \end{cases}
            \]
            Por lo tanto, $R \subseteq P \cup Q \implies d(a,b) = \abs{R} \leq \abs{P \cup Q} \leq \abs{P} + \abs{Q} = d(a,c) + d(a,b)$.
    \end{enumerate}
\end{ej}

\begin{ej}
    Tenemos que comprobar que
    \begin{enumerate}[i)]
        \item $d(a,b) \geq 0 \;  \forall a,b \in \mathcal{S}(A)$, y que $d(a,b) = 0 \iff a = b$, lo cual es trivial por la definición.
        \item $d(a,b) = d(b,a) \; \forall a,b \in \mathcal{S}(A)$, tambi\'en trivial por la definición.
        \item $d(a,b) \leq d(a,c) + d(c,b)$. Podemos suponer sin p\'erdida de generalidad que $d(a,c) \geq d(b,c)$. Sea $s$ tal que $e^{-s} = d(a,c)$ y sea
            $S$ tal que $e^{-S} = d(b,c)$, entonces, si $S = s$, entonces, $d(b, c) = d(a,c) \implies d(a, b) \leq d(a,c)$. Si $S > s$, entonces, 
            $d(a, b) \leq d(a,c)$.
    \end{enumerate}
\end{ej}

\begin{ej}
    \begin{enumerate}[(a)]
        \item Sí que define una m\'etrica, ya que
            \begin{enumerate}[i)]
                \item $d(x,y) \geq 0$ y $d(x,y) = 0 \iff x = y$ son triviales.
                \item $d(x,y) = d(y,x)$ es obvio.
                \item $d(x,y) = \abs{e^x - e^y} = \abs{e^x - e^z + e^z - e^y} \leq \abs{e^x - e^z} + \abs{e^z - e^y} = d(x,z) + d(y,z).$
            \end{enumerate}
        \item No es una m\'etrica. Los puntos $x = \frac{\pi}{2}$ y $y = \frac{3\pi}{2}$ son distintos, pero $d(x, y) = 0$.
        \item Sí que es una m\'etrica:
            \begin{enumerate}[i)]
                \item $d(x, y) \geq 0$ es trivial y $d(x,y) = 0 \iff \cos(x) = \cos(y) \iff x = y$.
                \item $d(x, y) = d(y,x)$ es obvio.
                \item Finalmente,
                    \begin{align*}
                        d(x, y) &= \abs{\cos(x) - \cos(y)} \\
                        &= \abs{\cos(x) - \cos(z) + \cos(z) - \cos(y)} \\
                        &\leq \abs{\cos(x) - \cos(z)} + \abs{\cos(z) - \cos(y)} = d(x,z) + d(z,y).
                    \end{align*}
            \end{enumerate}
        \item Sí define una m\'etrica:
            \begin{enumerate}[i)]
                \item $d(x, y) \geq 0$ es trivial y $d(x, y) = 0 \iff \arctan(x) = \arctan(y) \iff x = y$.
                \item $d(x, y) = d(y, x)$ es obvio.
                \item Finalmente,
                    \begin{align*}
                        d(x, y) &= \abs{\arctan(x) - \arctan(y)} \\
                        &= \abs{\arctan(x) - \arctan(z) + \arctan(z) - \arctan(y)} \\
                        &\leq \abs{\arctan(x) - \arctan(z)} + \abs{\arctan(z) - \arctan(y)} \\
                        &= d(x, z) + d(y,z).
                    \end{align*}
            \end{enumerate}
    \end{enumerate}

    En general, basta con que $f$ sea inyectiva para que $d_f(x,y) = \abs{f(x) - f(y)}$ sea una distancia. Vemos que si $f$ es inyectiva
    \begin{enumerate}[i)]
        \item $d(x,y) \geq 0$ por el valor absoluto. $d(x, y) = 0 \iff f(x) - f(y) = 0 \iff f(x) = f(y) \stackrel{f \text{ inyectiva}}{\iff} x = y$.
        \item $d(x,y) = \abs{f(x) - f(y)} = \abs{f(y) - f(x)} = d(y,x)$.
        \item $d(x,y) = \abs{f(x) - f(y)} = \abs{f(x) - f(z) + f(z) - f(y)} \leq \abs{f(x) - f(z)} + \abs{f(z) - f(y)}$ $ = d(x, z) + d(z, y)$.
    \end{enumerate}
\end{ej}

\begin{ej}
    \begin{enumerate}[(a)]
        \item Sí que es una m\'etrica, de hecho es la m\'etrica habitual en $\real^n$.
            \begin{center}
                %% Creator: Matplotlib, PGF backend
%%
%% To include the figure in your LaTeX document, write
%%   \input{<filename>.pgf}
%%
%% Make sure the required packages are loaded in your preamble
%%   \usepackage{pgf}
%%
%% Figures using additional raster images can only be included by \input if
%% they are in the same directory as the main LaTeX file. For loading figures
%% from other directories you can use the `import` package
%%   \usepackage{import}
%% and then include the figures with
%%   \import{<path to file>}{<filename>.pgf}
%%
%% Matplotlib used the following preamble
%%   \usepackage{fontspec}
%%   \setmainfont{DejaVu Serif}
%%   \setsansfont{DejaVu Sans}
%%   \setmonofont{DejaVu Sans Mono}
%%
\begingroup%
\makeatletter%
\begin{pgfpicture}%
\pgfpathrectangle{\pgfpointorigin}{\pgfqpoint{2.136666in}{2.147420in}}%
\pgfusepath{use as bounding box, clip}%
\begin{pgfscope}%
\pgfsetbuttcap%
\pgfsetmiterjoin%
\definecolor{currentfill}{rgb}{1.000000,1.000000,1.000000}%
\pgfsetfillcolor{currentfill}%
\pgfsetlinewidth{0.000000pt}%
\definecolor{currentstroke}{rgb}{1.000000,1.000000,1.000000}%
\pgfsetstrokecolor{currentstroke}%
\pgfsetdash{}{0pt}%
\pgfpathmoveto{\pgfqpoint{0.000000in}{0.000000in}}%
\pgfpathlineto{\pgfqpoint{2.136666in}{0.000000in}}%
\pgfpathlineto{\pgfqpoint{2.136666in}{2.147420in}}%
\pgfpathlineto{\pgfqpoint{0.000000in}{2.147420in}}%
\pgfpathclose%
\pgfusepath{fill}%
\end{pgfscope}%
\begin{pgfscope}%
\pgfsetbuttcap%
\pgfsetmiterjoin%
\definecolor{currentfill}{rgb}{1.000000,1.000000,1.000000}%
\pgfsetfillcolor{currentfill}%
\pgfsetlinewidth{0.000000pt}%
\definecolor{currentstroke}{rgb}{0.000000,0.000000,0.000000}%
\pgfsetstrokecolor{currentstroke}%
\pgfsetstrokeopacity{0.000000}%
\pgfsetdash{}{0pt}%
\pgfpathmoveto{\pgfqpoint{0.135000in}{0.145754in}}%
\pgfpathlineto{\pgfqpoint{2.001666in}{0.145754in}}%
\pgfpathlineto{\pgfqpoint{2.001666in}{2.012420in}}%
\pgfpathlineto{\pgfqpoint{0.135000in}{2.012420in}}%
\pgfpathclose%
\pgfusepath{fill}%
\end{pgfscope}%
\begin{pgfscope}%
\pgfpathrectangle{\pgfqpoint{0.135000in}{0.145754in}}{\pgfqpoint{1.866666in}{1.866666in}} %
\pgfusepath{clip}%
\pgfsetbuttcap%
\pgfsetroundjoin%
\definecolor{currentfill}{rgb}{0.000000,0.000000,1.000000}%
\pgfsetfillcolor{currentfill}%
\pgfsetfillopacity{0.300000}%
\pgfsetlinewidth{0.000000pt}%
\definecolor{currentstroke}{rgb}{0.000000,0.000000,0.000000}%
\pgfsetstrokecolor{currentstroke}%
\pgfsetdash{}{0pt}%
\pgfpathmoveto{\pgfqpoint{0.929803in}{0.192408in}}%
\pgfpathlineto{\pgfqpoint{0.941849in}{0.190610in}}%
\pgfpathlineto{\pgfqpoint{0.953895in}{0.188978in}}%
\pgfpathlineto{\pgfqpoint{0.965941in}{0.187512in}}%
\pgfpathlineto{\pgfqpoint{0.977987in}{0.186211in}}%
\pgfpathlineto{\pgfqpoint{0.990033in}{0.185074in}}%
\pgfpathlineto{\pgfqpoint{1.002079in}{0.184100in}}%
\pgfpathlineto{\pgfqpoint{1.014126in}{0.183290in}}%
\pgfpathlineto{\pgfqpoint{1.026172in}{0.182642in}}%
\pgfpathlineto{\pgfqpoint{1.038218in}{0.182157in}}%
\pgfpathlineto{\pgfqpoint{1.050264in}{0.181833in}}%
\pgfpathlineto{\pgfqpoint{1.062310in}{0.181671in}}%
\pgfpathlineto{\pgfqpoint{1.074356in}{0.181671in}}%
\pgfpathlineto{\pgfqpoint{1.086402in}{0.181833in}}%
\pgfpathlineto{\pgfqpoint{1.098448in}{0.182157in}}%
\pgfpathlineto{\pgfqpoint{1.110494in}{0.182642in}}%
\pgfpathlineto{\pgfqpoint{1.122541in}{0.183290in}}%
\pgfpathlineto{\pgfqpoint{1.134587in}{0.184100in}}%
\pgfpathlineto{\pgfqpoint{1.146633in}{0.185074in}}%
\pgfpathlineto{\pgfqpoint{1.158679in}{0.186211in}}%
\pgfpathlineto{\pgfqpoint{1.170725in}{0.187512in}}%
\pgfpathlineto{\pgfqpoint{1.182771in}{0.188978in}}%
\pgfpathlineto{\pgfqpoint{1.194817in}{0.190610in}}%
\pgfpathlineto{\pgfqpoint{1.206863in}{0.192408in}}%
\pgfpathlineto{\pgfqpoint{1.214772in}{0.193697in}}%
\pgfpathlineto{\pgfqpoint{1.218909in}{0.194374in}}%
\pgfpathlineto{\pgfqpoint{1.230956in}{0.196509in}}%
\pgfpathlineto{\pgfqpoint{1.243002in}{0.198814in}}%
\pgfpathlineto{\pgfqpoint{1.255048in}{0.201290in}}%
\pgfpathlineto{\pgfqpoint{1.267094in}{0.203939in}}%
\pgfpathlineto{\pgfqpoint{1.274804in}{0.205743in}}%
\pgfpathlineto{\pgfqpoint{1.279140in}{0.206762in}}%
\pgfpathlineto{\pgfqpoint{1.291186in}{0.209762in}}%
\pgfpathlineto{\pgfqpoint{1.303232in}{0.212940in}}%
\pgfpathlineto{\pgfqpoint{1.315278in}{0.216296in}}%
\pgfpathlineto{\pgfqpoint{1.320375in}{0.217790in}}%
\pgfpathlineto{\pgfqpoint{1.327325in}{0.219836in}}%
\pgfpathlineto{\pgfqpoint{1.339371in}{0.223560in}}%
\pgfpathlineto{\pgfqpoint{1.351417in}{0.227470in}}%
\pgfpathlineto{\pgfqpoint{1.358387in}{0.229836in}}%
\pgfpathlineto{\pgfqpoint{1.363463in}{0.231569in}}%
\pgfpathlineto{\pgfqpoint{1.375509in}{0.235861in}}%
\pgfpathlineto{\pgfqpoint{1.387555in}{0.240346in}}%
\pgfpathlineto{\pgfqpoint{1.391521in}{0.241882in}}%
\pgfpathlineto{\pgfqpoint{1.399601in}{0.245032in}}%
\pgfpathlineto{\pgfqpoint{1.411647in}{0.249918in}}%
\pgfpathlineto{\pgfqpoint{1.421154in}{0.253928in}}%
\pgfpathlineto{\pgfqpoint{1.423693in}{0.255007in}}%
\pgfpathlineto{\pgfqpoint{1.435740in}{0.260309in}}%
\pgfpathlineto{\pgfqpoint{1.447786in}{0.265818in}}%
\pgfpathlineto{\pgfqpoint{1.448116in}{0.265974in}}%
\pgfpathlineto{\pgfqpoint{1.459832in}{0.271552in}}%
\pgfpathlineto{\pgfqpoint{1.471878in}{0.277500in}}%
\pgfpathlineto{\pgfqpoint{1.472899in}{0.278020in}}%
\pgfpathlineto{\pgfqpoint{1.483924in}{0.283683in}}%
\pgfpathlineto{\pgfqpoint{1.495927in}{0.290066in}}%
\pgfpathlineto{\pgfqpoint{1.495970in}{0.290089in}}%
\pgfpathlineto{\pgfqpoint{1.508016in}{0.296744in}}%
\pgfpathlineto{\pgfqpoint{1.517424in}{0.302112in}}%
\pgfpathlineto{\pgfqpoint{1.520062in}{0.303634in}}%
\pgfpathlineto{\pgfqpoint{1.532108in}{0.310780in}}%
\pgfpathlineto{\pgfqpoint{1.537640in}{0.314159in}}%
\pgfpathlineto{\pgfqpoint{1.544155in}{0.318183in}}%
\pgfpathlineto{\pgfqpoint{1.556201in}{0.325844in}}%
\pgfpathlineto{\pgfqpoint{1.556754in}{0.326205in}}%
\pgfpathlineto{\pgfqpoint{1.568247in}{0.333791in}}%
\pgfpathlineto{\pgfqpoint{1.574819in}{0.338251in}}%
\pgfpathlineto{\pgfqpoint{1.580293in}{0.342013in}}%
\pgfpathlineto{\pgfqpoint{1.592025in}{0.350297in}}%
\pgfpathlineto{\pgfqpoint{1.592339in}{0.350522in}}%
\pgfpathlineto{\pgfqpoint{1.604385in}{0.359344in}}%
\pgfpathlineto{\pgfqpoint{1.608378in}{0.362343in}}%
\pgfpathlineto{\pgfqpoint{1.616431in}{0.368478in}}%
\pgfpathlineto{\pgfqpoint{1.624005in}{0.374389in}}%
\pgfpathlineto{\pgfqpoint{1.628477in}{0.377933in}}%
\pgfpathlineto{\pgfqpoint{1.638959in}{0.386435in}}%
\pgfpathlineto{\pgfqpoint{1.640524in}{0.387724in}}%
\pgfpathlineto{\pgfqpoint{1.652570in}{0.397871in}}%
\pgfpathlineto{\pgfqpoint{1.653279in}{0.398481in}}%
\pgfpathlineto{\pgfqpoint{1.664616in}{0.408394in}}%
\pgfpathlineto{\pgfqpoint{1.667003in}{0.410527in}}%
\pgfpathlineto{\pgfqpoint{1.676662in}{0.419304in}}%
\pgfpathlineto{\pgfqpoint{1.680187in}{0.422574in}}%
\pgfpathlineto{\pgfqpoint{1.688708in}{0.430619in}}%
\pgfpathlineto{\pgfqpoint{1.692862in}{0.434620in}}%
\pgfpathlineto{\pgfqpoint{1.700754in}{0.442363in}}%
\pgfpathlineto{\pgfqpoint{1.705057in}{0.446666in}}%
\pgfpathlineto{\pgfqpoint{1.712800in}{0.454558in}}%
\pgfpathlineto{\pgfqpoint{1.716801in}{0.458712in}}%
\pgfpathlineto{\pgfqpoint{1.724846in}{0.467233in}}%
\pgfpathlineto{\pgfqpoint{1.728116in}{0.470758in}}%
\pgfpathlineto{\pgfqpoint{1.736892in}{0.480416in}}%
\pgfpathlineto{\pgfqpoint{1.739025in}{0.482804in}}%
\pgfpathlineto{\pgfqpoint{1.748939in}{0.494141in}}%
\pgfpathlineto{\pgfqpoint{1.749549in}{0.494850in}}%
\pgfpathlineto{\pgfqpoint{1.759696in}{0.506896in}}%
\pgfpathlineto{\pgfqpoint{1.760985in}{0.508460in}}%
\pgfpathlineto{\pgfqpoint{1.769487in}{0.518942in}}%
\pgfpathlineto{\pgfqpoint{1.773031in}{0.523415in}}%
\pgfpathlineto{\pgfqpoint{1.778942in}{0.530989in}}%
\pgfpathlineto{\pgfqpoint{1.785077in}{0.539042in}}%
\pgfpathlineto{\pgfqpoint{1.788075in}{0.543035in}}%
\pgfpathlineto{\pgfqpoint{1.796898in}{0.555081in}}%
\pgfpathlineto{\pgfqpoint{1.797123in}{0.555395in}}%
\pgfpathlineto{\pgfqpoint{1.805407in}{0.567127in}}%
\pgfpathlineto{\pgfqpoint{1.809169in}{0.572601in}}%
\pgfpathlineto{\pgfqpoint{1.813629in}{0.579173in}}%
\pgfpathlineto{\pgfqpoint{1.821215in}{0.590666in}}%
\pgfpathlineto{\pgfqpoint{1.821576in}{0.591219in}}%
\pgfpathlineto{\pgfqpoint{1.829237in}{0.603265in}}%
\pgfpathlineto{\pgfqpoint{1.833261in}{0.609780in}}%
\pgfpathlineto{\pgfqpoint{1.836640in}{0.615311in}}%
\pgfpathlineto{\pgfqpoint{1.843786in}{0.627358in}}%
\pgfpathlineto{\pgfqpoint{1.845307in}{0.629996in}}%
\pgfpathlineto{\pgfqpoint{1.850676in}{0.639404in}}%
\pgfpathlineto{\pgfqpoint{1.857331in}{0.651450in}}%
\pgfpathlineto{\pgfqpoint{1.857354in}{0.651493in}}%
\pgfpathlineto{\pgfqpoint{1.863737in}{0.663496in}}%
\pgfpathlineto{\pgfqpoint{1.869400in}{0.674521in}}%
\pgfpathlineto{\pgfqpoint{1.869920in}{0.675542in}}%
\pgfpathlineto{\pgfqpoint{1.875868in}{0.687588in}}%
\pgfpathlineto{\pgfqpoint{1.881446in}{0.699304in}}%
\pgfpathlineto{\pgfqpoint{1.881602in}{0.699634in}}%
\pgfpathlineto{\pgfqpoint{1.887111in}{0.711680in}}%
\pgfpathlineto{\pgfqpoint{1.892413in}{0.723726in}}%
\pgfpathlineto{\pgfqpoint{1.893492in}{0.726265in}}%
\pgfpathlineto{\pgfqpoint{1.897502in}{0.735773in}}%
\pgfpathlineto{\pgfqpoint{1.902388in}{0.747819in}}%
\pgfpathlineto{\pgfqpoint{1.905538in}{0.755899in}}%
\pgfpathlineto{\pgfqpoint{1.907074in}{0.759865in}}%
\pgfpathlineto{\pgfqpoint{1.911559in}{0.771911in}}%
\pgfpathlineto{\pgfqpoint{1.915851in}{0.783957in}}%
\pgfpathlineto{\pgfqpoint{1.917584in}{0.789033in}}%
\pgfpathlineto{\pgfqpoint{1.919950in}{0.796003in}}%
\pgfpathlineto{\pgfqpoint{1.923860in}{0.808049in}}%
\pgfpathlineto{\pgfqpoint{1.927584in}{0.820095in}}%
\pgfpathlineto{\pgfqpoint{1.929630in}{0.827045in}}%
\pgfpathlineto{\pgfqpoint{1.931124in}{0.832141in}}%
\pgfpathlineto{\pgfqpoint{1.934480in}{0.844188in}}%
\pgfpathlineto{\pgfqpoint{1.937658in}{0.856234in}}%
\pgfpathlineto{\pgfqpoint{1.940658in}{0.868280in}}%
\pgfpathlineto{\pgfqpoint{1.941676in}{0.872616in}}%
\pgfpathlineto{\pgfqpoint{1.943481in}{0.880326in}}%
\pgfpathlineto{\pgfqpoint{1.946130in}{0.892372in}}%
\pgfpathlineto{\pgfqpoint{1.948606in}{0.904418in}}%
\pgfpathlineto{\pgfqpoint{1.950911in}{0.916464in}}%
\pgfpathlineto{\pgfqpoint{1.953046in}{0.928510in}}%
\pgfpathlineto{\pgfqpoint{1.953723in}{0.932648in}}%
\pgfpathlineto{\pgfqpoint{1.955012in}{0.940557in}}%
\pgfpathlineto{\pgfqpoint{1.956810in}{0.952603in}}%
\pgfpathlineto{\pgfqpoint{1.958442in}{0.964649in}}%
\pgfpathlineto{\pgfqpoint{1.959908in}{0.976695in}}%
\pgfpathlineto{\pgfqpoint{1.961209in}{0.988741in}}%
\pgfpathlineto{\pgfqpoint{1.962346in}{1.000787in}}%
\pgfpathlineto{\pgfqpoint{1.963320in}{1.012833in}}%
\pgfpathlineto{\pgfqpoint{1.964130in}{1.024879in}}%
\pgfpathlineto{\pgfqpoint{1.964778in}{1.036925in}}%
\pgfpathlineto{\pgfqpoint{1.965263in}{1.048972in}}%
\pgfpathlineto{\pgfqpoint{1.965587in}{1.061018in}}%
\pgfpathlineto{\pgfqpoint{1.965748in}{1.073064in}}%
\pgfpathlineto{\pgfqpoint{1.965748in}{1.085110in}}%
\pgfpathlineto{\pgfqpoint{1.965587in}{1.097156in}}%
\pgfpathlineto{\pgfqpoint{1.965263in}{1.109202in}}%
\pgfpathlineto{\pgfqpoint{1.964778in}{1.121248in}}%
\pgfpathlineto{\pgfqpoint{1.964130in}{1.133294in}}%
\pgfpathlineto{\pgfqpoint{1.963320in}{1.145340in}}%
\pgfpathlineto{\pgfqpoint{1.962346in}{1.157387in}}%
\pgfpathlineto{\pgfqpoint{1.961209in}{1.169433in}}%
\pgfpathlineto{\pgfqpoint{1.959908in}{1.181479in}}%
\pgfpathlineto{\pgfqpoint{1.958442in}{1.193525in}}%
\pgfpathlineto{\pgfqpoint{1.956810in}{1.205571in}}%
\pgfpathlineto{\pgfqpoint{1.955012in}{1.217617in}}%
\pgfpathlineto{\pgfqpoint{1.953723in}{1.225525in}}%
\pgfpathlineto{\pgfqpoint{1.953046in}{1.229663in}}%
\pgfpathlineto{\pgfqpoint{1.950911in}{1.241709in}}%
\pgfpathlineto{\pgfqpoint{1.948606in}{1.253756in}}%
\pgfpathlineto{\pgfqpoint{1.946130in}{1.265802in}}%
\pgfpathlineto{\pgfqpoint{1.943481in}{1.277848in}}%
\pgfpathlineto{\pgfqpoint{1.941676in}{1.285558in}}%
\pgfpathlineto{\pgfqpoint{1.940658in}{1.289894in}}%
\pgfpathlineto{\pgfqpoint{1.937658in}{1.301940in}}%
\pgfpathlineto{\pgfqpoint{1.934480in}{1.313986in}}%
\pgfpathlineto{\pgfqpoint{1.931124in}{1.326032in}}%
\pgfpathlineto{\pgfqpoint{1.929630in}{1.331129in}}%
\pgfpathlineto{\pgfqpoint{1.927584in}{1.338078in}}%
\pgfpathlineto{\pgfqpoint{1.923860in}{1.350124in}}%
\pgfpathlineto{\pgfqpoint{1.919950in}{1.362171in}}%
\pgfpathlineto{\pgfqpoint{1.917584in}{1.369141in}}%
\pgfpathlineto{\pgfqpoint{1.915851in}{1.374217in}}%
\pgfpathlineto{\pgfqpoint{1.911559in}{1.386263in}}%
\pgfpathlineto{\pgfqpoint{1.907074in}{1.398309in}}%
\pgfpathlineto{\pgfqpoint{1.905538in}{1.402275in}}%
\pgfpathlineto{\pgfqpoint{1.902388in}{1.410355in}}%
\pgfpathlineto{\pgfqpoint{1.897502in}{1.422401in}}%
\pgfpathlineto{\pgfqpoint{1.893492in}{1.431908in}}%
\pgfpathlineto{\pgfqpoint{1.892413in}{1.434447in}}%
\pgfpathlineto{\pgfqpoint{1.887111in}{1.446493in}}%
\pgfpathlineto{\pgfqpoint{1.881602in}{1.458539in}}%
\pgfpathlineto{\pgfqpoint{1.881446in}{1.458870in}}%
\pgfpathlineto{\pgfqpoint{1.875868in}{1.470586in}}%
\pgfpathlineto{\pgfqpoint{1.869920in}{1.482632in}}%
\pgfpathlineto{\pgfqpoint{1.869400in}{1.483653in}}%
\pgfpathlineto{\pgfqpoint{1.863737in}{1.494678in}}%
\pgfpathlineto{\pgfqpoint{1.857354in}{1.506681in}}%
\pgfpathlineto{\pgfqpoint{1.857331in}{1.506724in}}%
\pgfpathlineto{\pgfqpoint{1.850676in}{1.518770in}}%
\pgfpathlineto{\pgfqpoint{1.845307in}{1.528178in}}%
\pgfpathlineto{\pgfqpoint{1.843786in}{1.530816in}}%
\pgfpathlineto{\pgfqpoint{1.836640in}{1.542862in}}%
\pgfpathlineto{\pgfqpoint{1.833261in}{1.548394in}}%
\pgfpathlineto{\pgfqpoint{1.829237in}{1.554908in}}%
\pgfpathlineto{\pgfqpoint{1.821576in}{1.566955in}}%
\pgfpathlineto{\pgfqpoint{1.821215in}{1.567507in}}%
\pgfpathlineto{\pgfqpoint{1.813629in}{1.579001in}}%
\pgfpathlineto{\pgfqpoint{1.809169in}{1.585573in}}%
\pgfpathlineto{\pgfqpoint{1.805407in}{1.591047in}}%
\pgfpathlineto{\pgfqpoint{1.797123in}{1.602779in}}%
\pgfpathlineto{\pgfqpoint{1.796898in}{1.603093in}}%
\pgfpathlineto{\pgfqpoint{1.788075in}{1.615139in}}%
\pgfpathlineto{\pgfqpoint{1.785077in}{1.619132in}}%
\pgfpathlineto{\pgfqpoint{1.778942in}{1.627185in}}%
\pgfpathlineto{\pgfqpoint{1.773031in}{1.634758in}}%
\pgfpathlineto{\pgfqpoint{1.769487in}{1.639231in}}%
\pgfpathlineto{\pgfqpoint{1.760985in}{1.649713in}}%
\pgfpathlineto{\pgfqpoint{1.759696in}{1.651277in}}%
\pgfpathlineto{\pgfqpoint{1.749549in}{1.663323in}}%
\pgfpathlineto{\pgfqpoint{1.748939in}{1.664033in}}%
\pgfpathlineto{\pgfqpoint{1.739025in}{1.675370in}}%
\pgfpathlineto{\pgfqpoint{1.736892in}{1.677757in}}%
\pgfpathlineto{\pgfqpoint{1.728116in}{1.687416in}}%
\pgfpathlineto{\pgfqpoint{1.724846in}{1.690941in}}%
\pgfpathlineto{\pgfqpoint{1.716801in}{1.699462in}}%
\pgfpathlineto{\pgfqpoint{1.712800in}{1.703615in}}%
\pgfpathlineto{\pgfqpoint{1.705057in}{1.711508in}}%
\pgfpathlineto{\pgfqpoint{1.700754in}{1.715811in}}%
\pgfpathlineto{\pgfqpoint{1.692862in}{1.723554in}}%
\pgfpathlineto{\pgfqpoint{1.688708in}{1.727554in}}%
\pgfpathlineto{\pgfqpoint{1.680187in}{1.735600in}}%
\pgfpathlineto{\pgfqpoint{1.676662in}{1.738870in}}%
\pgfpathlineto{\pgfqpoint{1.667003in}{1.747646in}}%
\pgfpathlineto{\pgfqpoint{1.664616in}{1.749779in}}%
\pgfpathlineto{\pgfqpoint{1.653279in}{1.759692in}}%
\pgfpathlineto{\pgfqpoint{1.652570in}{1.760303in}}%
\pgfpathlineto{\pgfqpoint{1.640524in}{1.770450in}}%
\pgfpathlineto{\pgfqpoint{1.638959in}{1.771738in}}%
\pgfpathlineto{\pgfqpoint{1.628477in}{1.780241in}}%
\pgfpathlineto{\pgfqpoint{1.624005in}{1.783785in}}%
\pgfpathlineto{\pgfqpoint{1.616431in}{1.789696in}}%
\pgfpathlineto{\pgfqpoint{1.608378in}{1.795831in}}%
\pgfpathlineto{\pgfqpoint{1.604385in}{1.798829in}}%
\pgfpathlineto{\pgfqpoint{1.592339in}{1.807652in}}%
\pgfpathlineto{\pgfqpoint{1.592025in}{1.807877in}}%
\pgfpathlineto{\pgfqpoint{1.580293in}{1.816160in}}%
\pgfpathlineto{\pgfqpoint{1.574819in}{1.819923in}}%
\pgfpathlineto{\pgfqpoint{1.568247in}{1.824383in}}%
\pgfpathlineto{\pgfqpoint{1.556754in}{1.831969in}}%
\pgfpathlineto{\pgfqpoint{1.556201in}{1.832330in}}%
\pgfpathlineto{\pgfqpoint{1.544155in}{1.839991in}}%
\pgfpathlineto{\pgfqpoint{1.537640in}{1.844015in}}%
\pgfpathlineto{\pgfqpoint{1.532108in}{1.847394in}}%
\pgfpathlineto{\pgfqpoint{1.520062in}{1.854540in}}%
\pgfpathlineto{\pgfqpoint{1.517424in}{1.856061in}}%
\pgfpathlineto{\pgfqpoint{1.508016in}{1.861430in}}%
\pgfpathlineto{\pgfqpoint{1.495970in}{1.868084in}}%
\pgfpathlineto{\pgfqpoint{1.495927in}{1.868107in}}%
\pgfpathlineto{\pgfqpoint{1.483924in}{1.874491in}}%
\pgfpathlineto{\pgfqpoint{1.472899in}{1.880154in}}%
\pgfpathlineto{\pgfqpoint{1.471878in}{1.880673in}}%
\pgfpathlineto{\pgfqpoint{1.459832in}{1.886622in}}%
\pgfpathlineto{\pgfqpoint{1.448116in}{1.892200in}}%
\pgfpathlineto{\pgfqpoint{1.447786in}{1.892356in}}%
\pgfpathlineto{\pgfqpoint{1.435740in}{1.897865in}}%
\pgfpathlineto{\pgfqpoint{1.423693in}{1.903167in}}%
\pgfpathlineto{\pgfqpoint{1.421154in}{1.904246in}}%
\pgfpathlineto{\pgfqpoint{1.411647in}{1.908256in}}%
\pgfpathlineto{\pgfqpoint{1.399601in}{1.913142in}}%
\pgfpathlineto{\pgfqpoint{1.391521in}{1.916292in}}%
\pgfpathlineto{\pgfqpoint{1.387555in}{1.917828in}}%
\pgfpathlineto{\pgfqpoint{1.375509in}{1.922312in}}%
\pgfpathlineto{\pgfqpoint{1.363463in}{1.926605in}}%
\pgfpathlineto{\pgfqpoint{1.358387in}{1.928338in}}%
\pgfpathlineto{\pgfqpoint{1.351417in}{1.930704in}}%
\pgfpathlineto{\pgfqpoint{1.339371in}{1.934614in}}%
\pgfpathlineto{\pgfqpoint{1.327325in}{1.938338in}}%
\pgfpathlineto{\pgfqpoint{1.320375in}{1.940384in}}%
\pgfpathlineto{\pgfqpoint{1.315278in}{1.941877in}}%
\pgfpathlineto{\pgfqpoint{1.303232in}{1.945234in}}%
\pgfpathlineto{\pgfqpoint{1.291186in}{1.948411in}}%
\pgfpathlineto{\pgfqpoint{1.279140in}{1.951412in}}%
\pgfpathlineto{\pgfqpoint{1.274804in}{1.952430in}}%
\pgfpathlineto{\pgfqpoint{1.267094in}{1.954235in}}%
\pgfpathlineto{\pgfqpoint{1.255048in}{1.956883in}}%
\pgfpathlineto{\pgfqpoint{1.243002in}{1.959360in}}%
\pgfpathlineto{\pgfqpoint{1.230956in}{1.961665in}}%
\pgfpathlineto{\pgfqpoint{1.218909in}{1.963800in}}%
\pgfpathlineto{\pgfqpoint{1.214772in}{1.964476in}}%
\pgfpathlineto{\pgfqpoint{1.206863in}{1.965766in}}%
\pgfpathlineto{\pgfqpoint{1.194817in}{1.967564in}}%
\pgfpathlineto{\pgfqpoint{1.182771in}{1.969196in}}%
\pgfpathlineto{\pgfqpoint{1.170725in}{1.970662in}}%
\pgfpathlineto{\pgfqpoint{1.158679in}{1.971963in}}%
\pgfpathlineto{\pgfqpoint{1.146633in}{1.973100in}}%
\pgfpathlineto{\pgfqpoint{1.134587in}{1.974073in}}%
\pgfpathlineto{\pgfqpoint{1.122541in}{1.974884in}}%
\pgfpathlineto{\pgfqpoint{1.110494in}{1.975532in}}%
\pgfpathlineto{\pgfqpoint{1.098448in}{1.976017in}}%
\pgfpathlineto{\pgfqpoint{1.086402in}{1.976341in}}%
\pgfpathlineto{\pgfqpoint{1.074356in}{1.976502in}}%
\pgfpathlineto{\pgfqpoint{1.062310in}{1.976502in}}%
\pgfpathlineto{\pgfqpoint{1.050264in}{1.976341in}}%
\pgfpathlineto{\pgfqpoint{1.038218in}{1.976017in}}%
\pgfpathlineto{\pgfqpoint{1.026172in}{1.975532in}}%
\pgfpathlineto{\pgfqpoint{1.014126in}{1.974884in}}%
\pgfpathlineto{\pgfqpoint{1.002079in}{1.974073in}}%
\pgfpathlineto{\pgfqpoint{0.990033in}{1.973100in}}%
\pgfpathlineto{\pgfqpoint{0.977987in}{1.971963in}}%
\pgfpathlineto{\pgfqpoint{0.965941in}{1.970662in}}%
\pgfpathlineto{\pgfqpoint{0.953895in}{1.969196in}}%
\pgfpathlineto{\pgfqpoint{0.941849in}{1.967564in}}%
\pgfpathlineto{\pgfqpoint{0.929803in}{1.965766in}}%
\pgfpathlineto{\pgfqpoint{0.921894in}{1.964476in}}%
\pgfpathlineto{\pgfqpoint{0.917757in}{1.963800in}}%
\pgfpathlineto{\pgfqpoint{0.905710in}{1.961665in}}%
\pgfpathlineto{\pgfqpoint{0.893664in}{1.959360in}}%
\pgfpathlineto{\pgfqpoint{0.881618in}{1.956883in}}%
\pgfpathlineto{\pgfqpoint{0.869572in}{1.954235in}}%
\pgfpathlineto{\pgfqpoint{0.861862in}{1.952430in}}%
\pgfpathlineto{\pgfqpoint{0.857526in}{1.951412in}}%
\pgfpathlineto{\pgfqpoint{0.845480in}{1.948411in}}%
\pgfpathlineto{\pgfqpoint{0.833434in}{1.945234in}}%
\pgfpathlineto{\pgfqpoint{0.821388in}{1.941877in}}%
\pgfpathlineto{\pgfqpoint{0.816291in}{1.940384in}}%
\pgfpathlineto{\pgfqpoint{0.809342in}{1.938338in}}%
\pgfpathlineto{\pgfqpoint{0.797295in}{1.934614in}}%
\pgfpathlineto{\pgfqpoint{0.785249in}{1.930704in}}%
\pgfpathlineto{\pgfqpoint{0.778279in}{1.928338in}}%
\pgfpathlineto{\pgfqpoint{0.773203in}{1.926605in}}%
\pgfpathlineto{\pgfqpoint{0.761157in}{1.922312in}}%
\pgfpathlineto{\pgfqpoint{0.749111in}{1.917828in}}%
\pgfpathlineto{\pgfqpoint{0.745145in}{1.916292in}}%
\pgfpathlineto{\pgfqpoint{0.737065in}{1.913142in}}%
\pgfpathlineto{\pgfqpoint{0.725019in}{1.908256in}}%
\pgfpathlineto{\pgfqpoint{0.715512in}{1.904246in}}%
\pgfpathlineto{\pgfqpoint{0.712973in}{1.903167in}}%
\pgfpathlineto{\pgfqpoint{0.700927in}{1.897865in}}%
\pgfpathlineto{\pgfqpoint{0.688880in}{1.892356in}}%
\pgfpathlineto{\pgfqpoint{0.688550in}{1.892200in}}%
\pgfpathlineto{\pgfqpoint{0.676834in}{1.886622in}}%
\pgfpathlineto{\pgfqpoint{0.664788in}{1.880673in}}%
\pgfpathlineto{\pgfqpoint{0.663767in}{1.880154in}}%
\pgfpathlineto{\pgfqpoint{0.652742in}{1.874491in}}%
\pgfpathlineto{\pgfqpoint{0.640739in}{1.868107in}}%
\pgfpathlineto{\pgfqpoint{0.640696in}{1.868084in}}%
\pgfpathlineto{\pgfqpoint{0.628650in}{1.861430in}}%
\pgfpathlineto{\pgfqpoint{0.619242in}{1.856061in}}%
\pgfpathlineto{\pgfqpoint{0.616604in}{1.854540in}}%
\pgfpathlineto{\pgfqpoint{0.604558in}{1.847394in}}%
\pgfpathlineto{\pgfqpoint{0.599026in}{1.844015in}}%
\pgfpathlineto{\pgfqpoint{0.592511in}{1.839991in}}%
\pgfpathlineto{\pgfqpoint{0.580465in}{1.832330in}}%
\pgfpathlineto{\pgfqpoint{0.579913in}{1.831969in}}%
\pgfpathlineto{\pgfqpoint{0.568419in}{1.824383in}}%
\pgfpathlineto{\pgfqpoint{0.561847in}{1.819923in}}%
\pgfpathlineto{\pgfqpoint{0.556373in}{1.816160in}}%
\pgfpathlineto{\pgfqpoint{0.544641in}{1.807877in}}%
\pgfpathlineto{\pgfqpoint{0.544327in}{1.807652in}}%
\pgfpathlineto{\pgfqpoint{0.532281in}{1.798829in}}%
\pgfpathlineto{\pgfqpoint{0.528288in}{1.795831in}}%
\pgfpathlineto{\pgfqpoint{0.520235in}{1.789696in}}%
\pgfpathlineto{\pgfqpoint{0.512662in}{1.783785in}}%
\pgfpathlineto{\pgfqpoint{0.508189in}{1.780241in}}%
\pgfpathlineto{\pgfqpoint{0.497707in}{1.771738in}}%
\pgfpathlineto{\pgfqpoint{0.496143in}{1.770450in}}%
\pgfpathlineto{\pgfqpoint{0.484096in}{1.760303in}}%
\pgfpathlineto{\pgfqpoint{0.483387in}{1.759692in}}%
\pgfpathlineto{\pgfqpoint{0.472050in}{1.749779in}}%
\pgfpathlineto{\pgfqpoint{0.469663in}{1.747646in}}%
\pgfpathlineto{\pgfqpoint{0.460004in}{1.738870in}}%
\pgfpathlineto{\pgfqpoint{0.456479in}{1.735600in}}%
\pgfpathlineto{\pgfqpoint{0.447958in}{1.727554in}}%
\pgfpathlineto{\pgfqpoint{0.443805in}{1.723554in}}%
\pgfpathlineto{\pgfqpoint{0.435912in}{1.715811in}}%
\pgfpathlineto{\pgfqpoint{0.431609in}{1.711508in}}%
\pgfpathlineto{\pgfqpoint{0.423866in}{1.703615in}}%
\pgfpathlineto{\pgfqpoint{0.419865in}{1.699462in}}%
\pgfpathlineto{\pgfqpoint{0.411820in}{1.690941in}}%
\pgfpathlineto{\pgfqpoint{0.408550in}{1.687416in}}%
\pgfpathlineto{\pgfqpoint{0.399774in}{1.677757in}}%
\pgfpathlineto{\pgfqpoint{0.397641in}{1.675370in}}%
\pgfpathlineto{\pgfqpoint{0.387728in}{1.664033in}}%
\pgfpathlineto{\pgfqpoint{0.387117in}{1.663323in}}%
\pgfpathlineto{\pgfqpoint{0.376970in}{1.651277in}}%
\pgfpathlineto{\pgfqpoint{0.375681in}{1.649713in}}%
\pgfpathlineto{\pgfqpoint{0.367179in}{1.639231in}}%
\pgfpathlineto{\pgfqpoint{0.363635in}{1.634758in}}%
\pgfpathlineto{\pgfqpoint{0.357724in}{1.627185in}}%
\pgfpathlineto{\pgfqpoint{0.351589in}{1.619132in}}%
\pgfpathlineto{\pgfqpoint{0.348591in}{1.615139in}}%
\pgfpathlineto{\pgfqpoint{0.339768in}{1.603093in}}%
\pgfpathlineto{\pgfqpoint{0.339543in}{1.602779in}}%
\pgfpathlineto{\pgfqpoint{0.331259in}{1.591047in}}%
\pgfpathlineto{\pgfqpoint{0.327497in}{1.585573in}}%
\pgfpathlineto{\pgfqpoint{0.323037in}{1.579001in}}%
\pgfpathlineto{\pgfqpoint{0.315451in}{1.567507in}}%
\pgfpathlineto{\pgfqpoint{0.315090in}{1.566955in}}%
\pgfpathlineto{\pgfqpoint{0.307429in}{1.554908in}}%
\pgfpathlineto{\pgfqpoint{0.303405in}{1.548394in}}%
\pgfpathlineto{\pgfqpoint{0.300026in}{1.542862in}}%
\pgfpathlineto{\pgfqpoint{0.292880in}{1.530816in}}%
\pgfpathlineto{\pgfqpoint{0.291359in}{1.528178in}}%
\pgfpathlineto{\pgfqpoint{0.285990in}{1.518770in}}%
\pgfpathlineto{\pgfqpoint{0.279335in}{1.506724in}}%
\pgfpathlineto{\pgfqpoint{0.279312in}{1.506681in}}%
\pgfpathlineto{\pgfqpoint{0.272929in}{1.494678in}}%
\pgfpathlineto{\pgfqpoint{0.267266in}{1.483653in}}%
\pgfpathlineto{\pgfqpoint{0.266746in}{1.482632in}}%
\pgfpathlineto{\pgfqpoint{0.260798in}{1.470586in}}%
\pgfpathlineto{\pgfqpoint{0.255220in}{1.458870in}}%
\pgfpathlineto{\pgfqpoint{0.255064in}{1.458539in}}%
\pgfpathlineto{\pgfqpoint{0.249555in}{1.446493in}}%
\pgfpathlineto{\pgfqpoint{0.244253in}{1.434447in}}%
\pgfpathlineto{\pgfqpoint{0.243174in}{1.431908in}}%
\pgfpathlineto{\pgfqpoint{0.239164in}{1.422401in}}%
\pgfpathlineto{\pgfqpoint{0.234278in}{1.410355in}}%
\pgfpathlineto{\pgfqpoint{0.231128in}{1.402275in}}%
\pgfpathlineto{\pgfqpoint{0.229592in}{1.398309in}}%
\pgfpathlineto{\pgfqpoint{0.225107in}{1.386263in}}%
\pgfpathlineto{\pgfqpoint{0.220815in}{1.374217in}}%
\pgfpathlineto{\pgfqpoint{0.219082in}{1.369141in}}%
\pgfpathlineto{\pgfqpoint{0.216716in}{1.362171in}}%
\pgfpathlineto{\pgfqpoint{0.212806in}{1.350124in}}%
\pgfpathlineto{\pgfqpoint{0.209082in}{1.338078in}}%
\pgfpathlineto{\pgfqpoint{0.207036in}{1.331129in}}%
\pgfpathlineto{\pgfqpoint{0.205543in}{1.326032in}}%
\pgfpathlineto{\pgfqpoint{0.202186in}{1.313986in}}%
\pgfpathlineto{\pgfqpoint{0.199008in}{1.301940in}}%
\pgfpathlineto{\pgfqpoint{0.196008in}{1.289894in}}%
\pgfpathlineto{\pgfqpoint{0.194990in}{1.285558in}}%
\pgfpathlineto{\pgfqpoint{0.193185in}{1.277848in}}%
\pgfpathlineto{\pgfqpoint{0.190536in}{1.265802in}}%
\pgfpathlineto{\pgfqpoint{0.188060in}{1.253756in}}%
\pgfpathlineto{\pgfqpoint{0.185755in}{1.241709in}}%
\pgfpathlineto{\pgfqpoint{0.183620in}{1.229663in}}%
\pgfpathlineto{\pgfqpoint{0.182944in}{1.225525in}}%
\pgfpathlineto{\pgfqpoint{0.181654in}{1.217617in}}%
\pgfpathlineto{\pgfqpoint{0.179856in}{1.205571in}}%
\pgfpathlineto{\pgfqpoint{0.178224in}{1.193525in}}%
\pgfpathlineto{\pgfqpoint{0.176758in}{1.181479in}}%
\pgfpathlineto{\pgfqpoint{0.175457in}{1.169433in}}%
\pgfpathlineto{\pgfqpoint{0.174320in}{1.157387in}}%
\pgfpathlineto{\pgfqpoint{0.173346in}{1.145340in}}%
\pgfpathlineto{\pgfqpoint{0.172536in}{1.133294in}}%
\pgfpathlineto{\pgfqpoint{0.171888in}{1.121248in}}%
\pgfpathlineto{\pgfqpoint{0.171403in}{1.109202in}}%
\pgfpathlineto{\pgfqpoint{0.171079in}{1.097156in}}%
\pgfpathlineto{\pgfqpoint{0.170918in}{1.085110in}}%
\pgfpathlineto{\pgfqpoint{0.170918in}{1.073064in}}%
\pgfpathlineto{\pgfqpoint{0.171079in}{1.061018in}}%
\pgfpathlineto{\pgfqpoint{0.171403in}{1.048972in}}%
\pgfpathlineto{\pgfqpoint{0.171888in}{1.036925in}}%
\pgfpathlineto{\pgfqpoint{0.172536in}{1.024879in}}%
\pgfpathlineto{\pgfqpoint{0.173346in}{1.012833in}}%
\pgfpathlineto{\pgfqpoint{0.174320in}{1.000787in}}%
\pgfpathlineto{\pgfqpoint{0.175457in}{0.988741in}}%
\pgfpathlineto{\pgfqpoint{0.176758in}{0.976695in}}%
\pgfpathlineto{\pgfqpoint{0.178224in}{0.964649in}}%
\pgfpathlineto{\pgfqpoint{0.179856in}{0.952603in}}%
\pgfpathlineto{\pgfqpoint{0.181654in}{0.940557in}}%
\pgfpathlineto{\pgfqpoint{0.182944in}{0.932648in}}%
\pgfpathlineto{\pgfqpoint{0.183620in}{0.928510in}}%
\pgfpathlineto{\pgfqpoint{0.185755in}{0.916464in}}%
\pgfpathlineto{\pgfqpoint{0.188060in}{0.904418in}}%
\pgfpathlineto{\pgfqpoint{0.190536in}{0.892372in}}%
\pgfpathlineto{\pgfqpoint{0.193185in}{0.880326in}}%
\pgfpathlineto{\pgfqpoint{0.194990in}{0.872616in}}%
\pgfpathlineto{\pgfqpoint{0.196008in}{0.868280in}}%
\pgfpathlineto{\pgfqpoint{0.199008in}{0.856234in}}%
\pgfpathlineto{\pgfqpoint{0.202186in}{0.844188in}}%
\pgfpathlineto{\pgfqpoint{0.205543in}{0.832141in}}%
\pgfpathlineto{\pgfqpoint{0.207036in}{0.827045in}}%
\pgfpathlineto{\pgfqpoint{0.209082in}{0.820095in}}%
\pgfpathlineto{\pgfqpoint{0.212806in}{0.808049in}}%
\pgfpathlineto{\pgfqpoint{0.216716in}{0.796003in}}%
\pgfpathlineto{\pgfqpoint{0.219082in}{0.789033in}}%
\pgfpathlineto{\pgfqpoint{0.220815in}{0.783957in}}%
\pgfpathlineto{\pgfqpoint{0.225107in}{0.771911in}}%
\pgfpathlineto{\pgfqpoint{0.229592in}{0.759865in}}%
\pgfpathlineto{\pgfqpoint{0.231128in}{0.755899in}}%
\pgfpathlineto{\pgfqpoint{0.234278in}{0.747819in}}%
\pgfpathlineto{\pgfqpoint{0.239164in}{0.735773in}}%
\pgfpathlineto{\pgfqpoint{0.243174in}{0.726265in}}%
\pgfpathlineto{\pgfqpoint{0.244253in}{0.723726in}}%
\pgfpathlineto{\pgfqpoint{0.249555in}{0.711680in}}%
\pgfpathlineto{\pgfqpoint{0.255064in}{0.699634in}}%
\pgfpathlineto{\pgfqpoint{0.255220in}{0.699304in}}%
\pgfpathlineto{\pgfqpoint{0.260798in}{0.687588in}}%
\pgfpathlineto{\pgfqpoint{0.266746in}{0.675542in}}%
\pgfpathlineto{\pgfqpoint{0.267266in}{0.674521in}}%
\pgfpathlineto{\pgfqpoint{0.272929in}{0.663496in}}%
\pgfpathlineto{\pgfqpoint{0.279312in}{0.651493in}}%
\pgfpathlineto{\pgfqpoint{0.279335in}{0.651450in}}%
\pgfpathlineto{\pgfqpoint{0.285990in}{0.639404in}}%
\pgfpathlineto{\pgfqpoint{0.291359in}{0.629996in}}%
\pgfpathlineto{\pgfqpoint{0.292880in}{0.627358in}}%
\pgfpathlineto{\pgfqpoint{0.300026in}{0.615311in}}%
\pgfpathlineto{\pgfqpoint{0.303405in}{0.609780in}}%
\pgfpathlineto{\pgfqpoint{0.307429in}{0.603265in}}%
\pgfpathlineto{\pgfqpoint{0.315090in}{0.591219in}}%
\pgfpathlineto{\pgfqpoint{0.315451in}{0.590666in}}%
\pgfpathlineto{\pgfqpoint{0.323037in}{0.579173in}}%
\pgfpathlineto{\pgfqpoint{0.327497in}{0.572601in}}%
\pgfpathlineto{\pgfqpoint{0.331259in}{0.567127in}}%
\pgfpathlineto{\pgfqpoint{0.339543in}{0.555395in}}%
\pgfpathlineto{\pgfqpoint{0.339768in}{0.555081in}}%
\pgfpathlineto{\pgfqpoint{0.348591in}{0.543035in}}%
\pgfpathlineto{\pgfqpoint{0.351589in}{0.539042in}}%
\pgfpathlineto{\pgfqpoint{0.357724in}{0.530989in}}%
\pgfpathlineto{\pgfqpoint{0.363635in}{0.523415in}}%
\pgfpathlineto{\pgfqpoint{0.367179in}{0.518942in}}%
\pgfpathlineto{\pgfqpoint{0.375681in}{0.508460in}}%
\pgfpathlineto{\pgfqpoint{0.376970in}{0.506896in}}%
\pgfpathlineto{\pgfqpoint{0.387117in}{0.494850in}}%
\pgfpathlineto{\pgfqpoint{0.387728in}{0.494141in}}%
\pgfpathlineto{\pgfqpoint{0.397641in}{0.482804in}}%
\pgfpathlineto{\pgfqpoint{0.399774in}{0.480416in}}%
\pgfpathlineto{\pgfqpoint{0.408550in}{0.470758in}}%
\pgfpathlineto{\pgfqpoint{0.411820in}{0.467233in}}%
\pgfpathlineto{\pgfqpoint{0.419865in}{0.458712in}}%
\pgfpathlineto{\pgfqpoint{0.423866in}{0.454558in}}%
\pgfpathlineto{\pgfqpoint{0.431609in}{0.446666in}}%
\pgfpathlineto{\pgfqpoint{0.435912in}{0.442363in}}%
\pgfpathlineto{\pgfqpoint{0.443805in}{0.434620in}}%
\pgfpathlineto{\pgfqpoint{0.447958in}{0.430619in}}%
\pgfpathlineto{\pgfqpoint{0.456479in}{0.422574in}}%
\pgfpathlineto{\pgfqpoint{0.460004in}{0.419304in}}%
\pgfpathlineto{\pgfqpoint{0.469663in}{0.410527in}}%
\pgfpathlineto{\pgfqpoint{0.472050in}{0.408394in}}%
\pgfpathlineto{\pgfqpoint{0.483387in}{0.398481in}}%
\pgfpathlineto{\pgfqpoint{0.484096in}{0.397871in}}%
\pgfpathlineto{\pgfqpoint{0.496143in}{0.387724in}}%
\pgfpathlineto{\pgfqpoint{0.497707in}{0.386435in}}%
\pgfpathlineto{\pgfqpoint{0.508189in}{0.377933in}}%
\pgfpathlineto{\pgfqpoint{0.512662in}{0.374389in}}%
\pgfpathlineto{\pgfqpoint{0.520235in}{0.368478in}}%
\pgfpathlineto{\pgfqpoint{0.528288in}{0.362343in}}%
\pgfpathlineto{\pgfqpoint{0.532281in}{0.359344in}}%
\pgfpathlineto{\pgfqpoint{0.544327in}{0.350522in}}%
\pgfpathlineto{\pgfqpoint{0.544641in}{0.350297in}}%
\pgfpathlineto{\pgfqpoint{0.556373in}{0.342013in}}%
\pgfpathlineto{\pgfqpoint{0.561847in}{0.338251in}}%
\pgfpathlineto{\pgfqpoint{0.568419in}{0.333791in}}%
\pgfpathlineto{\pgfqpoint{0.579913in}{0.326205in}}%
\pgfpathlineto{\pgfqpoint{0.580465in}{0.325844in}}%
\pgfpathlineto{\pgfqpoint{0.592511in}{0.318183in}}%
\pgfpathlineto{\pgfqpoint{0.599026in}{0.314159in}}%
\pgfpathlineto{\pgfqpoint{0.604558in}{0.310780in}}%
\pgfpathlineto{\pgfqpoint{0.616604in}{0.303634in}}%
\pgfpathlineto{\pgfqpoint{0.619242in}{0.302112in}}%
\pgfpathlineto{\pgfqpoint{0.628650in}{0.296744in}}%
\pgfpathlineto{\pgfqpoint{0.640696in}{0.290089in}}%
\pgfpathlineto{\pgfqpoint{0.640739in}{0.290066in}}%
\pgfpathlineto{\pgfqpoint{0.652742in}{0.283683in}}%
\pgfpathlineto{\pgfqpoint{0.663767in}{0.278020in}}%
\pgfpathlineto{\pgfqpoint{0.664788in}{0.277500in}}%
\pgfpathlineto{\pgfqpoint{0.676834in}{0.271552in}}%
\pgfpathlineto{\pgfqpoint{0.688550in}{0.265974in}}%
\pgfpathlineto{\pgfqpoint{0.688880in}{0.265818in}}%
\pgfpathlineto{\pgfqpoint{0.700927in}{0.260309in}}%
\pgfpathlineto{\pgfqpoint{0.712973in}{0.255007in}}%
\pgfpathlineto{\pgfqpoint{0.715512in}{0.253928in}}%
\pgfpathlineto{\pgfqpoint{0.725019in}{0.249918in}}%
\pgfpathlineto{\pgfqpoint{0.737065in}{0.245032in}}%
\pgfpathlineto{\pgfqpoint{0.745145in}{0.241882in}}%
\pgfpathlineto{\pgfqpoint{0.749111in}{0.240346in}}%
\pgfpathlineto{\pgfqpoint{0.761157in}{0.235861in}}%
\pgfpathlineto{\pgfqpoint{0.773203in}{0.231569in}}%
\pgfpathlineto{\pgfqpoint{0.778279in}{0.229836in}}%
\pgfpathlineto{\pgfqpoint{0.785249in}{0.227470in}}%
\pgfpathlineto{\pgfqpoint{0.797295in}{0.223560in}}%
\pgfpathlineto{\pgfqpoint{0.809342in}{0.219836in}}%
\pgfpathlineto{\pgfqpoint{0.816291in}{0.217790in}}%
\pgfpathlineto{\pgfqpoint{0.821388in}{0.216296in}}%
\pgfpathlineto{\pgfqpoint{0.833434in}{0.212940in}}%
\pgfpathlineto{\pgfqpoint{0.845480in}{0.209762in}}%
\pgfpathlineto{\pgfqpoint{0.857526in}{0.206762in}}%
\pgfpathlineto{\pgfqpoint{0.861862in}{0.205743in}}%
\pgfpathlineto{\pgfqpoint{0.869572in}{0.203939in}}%
\pgfpathlineto{\pgfqpoint{0.881618in}{0.201290in}}%
\pgfpathlineto{\pgfqpoint{0.893664in}{0.198814in}}%
\pgfpathlineto{\pgfqpoint{0.905710in}{0.196509in}}%
\pgfpathlineto{\pgfqpoint{0.917757in}{0.194374in}}%
\pgfpathlineto{\pgfqpoint{0.921894in}{0.193697in}}%
\pgfpathclose%
\pgfusepath{fill}%
\end{pgfscope}%
\begin{pgfscope}%
\pgfpathrectangle{\pgfqpoint{0.135000in}{0.145754in}}{\pgfqpoint{1.866666in}{1.866666in}} %
\pgfusepath{clip}%
\pgfsetbuttcap%
\pgfsetroundjoin%
\definecolor{currentfill}{rgb}{0.000000,0.000000,1.000000}%
\pgfsetfillcolor{currentfill}%
\pgfsetlinewidth{0.000000pt}%
\definecolor{currentstroke}{rgb}{0.000000,0.000000,0.000000}%
\pgfsetstrokecolor{currentstroke}%
\pgfsetdash{}{0pt}%
\pgfpathmoveto{\pgfqpoint{0.929803in}{0.192408in}}%
\pgfpathlineto{\pgfqpoint{0.941849in}{0.190610in}}%
\pgfpathlineto{\pgfqpoint{0.953895in}{0.188978in}}%
\pgfpathlineto{\pgfqpoint{0.965941in}{0.187512in}}%
\pgfpathlineto{\pgfqpoint{0.977987in}{0.186211in}}%
\pgfpathlineto{\pgfqpoint{0.990033in}{0.185074in}}%
\pgfpathlineto{\pgfqpoint{1.002079in}{0.184100in}}%
\pgfpathlineto{\pgfqpoint{1.014126in}{0.183290in}}%
\pgfpathlineto{\pgfqpoint{1.026172in}{0.182642in}}%
\pgfpathlineto{\pgfqpoint{1.038218in}{0.182157in}}%
\pgfpathlineto{\pgfqpoint{1.050264in}{0.181833in}}%
\pgfpathlineto{\pgfqpoint{1.062310in}{0.181671in}}%
\pgfpathlineto{\pgfqpoint{1.074356in}{0.181671in}}%
\pgfpathlineto{\pgfqpoint{1.086402in}{0.181833in}}%
\pgfpathlineto{\pgfqpoint{1.098448in}{0.182157in}}%
\pgfpathlineto{\pgfqpoint{1.110494in}{0.182642in}}%
\pgfpathlineto{\pgfqpoint{1.122541in}{0.183290in}}%
\pgfpathlineto{\pgfqpoint{1.134587in}{0.184100in}}%
\pgfpathlineto{\pgfqpoint{1.146633in}{0.185074in}}%
\pgfpathlineto{\pgfqpoint{1.158679in}{0.186211in}}%
\pgfpathlineto{\pgfqpoint{1.170725in}{0.187512in}}%
\pgfpathlineto{\pgfqpoint{1.182771in}{0.188978in}}%
\pgfpathlineto{\pgfqpoint{1.194817in}{0.190610in}}%
\pgfpathlineto{\pgfqpoint{1.206863in}{0.192408in}}%
\pgfpathlineto{\pgfqpoint{1.214772in}{0.193697in}}%
\pgfpathlineto{\pgfqpoint{1.218909in}{0.194374in}}%
\pgfpathlineto{\pgfqpoint{1.230956in}{0.196509in}}%
\pgfpathlineto{\pgfqpoint{1.243002in}{0.198814in}}%
\pgfpathlineto{\pgfqpoint{1.255048in}{0.201290in}}%
\pgfpathlineto{\pgfqpoint{1.267094in}{0.203939in}}%
\pgfpathlineto{\pgfqpoint{1.274804in}{0.205743in}}%
\pgfpathlineto{\pgfqpoint{1.279140in}{0.206762in}}%
\pgfpathlineto{\pgfqpoint{1.291186in}{0.209762in}}%
\pgfpathlineto{\pgfqpoint{1.303232in}{0.212940in}}%
\pgfpathlineto{\pgfqpoint{1.315278in}{0.216296in}}%
\pgfpathlineto{\pgfqpoint{1.320375in}{0.217790in}}%
\pgfpathlineto{\pgfqpoint{1.327325in}{0.219836in}}%
\pgfpathlineto{\pgfqpoint{1.339371in}{0.223560in}}%
\pgfpathlineto{\pgfqpoint{1.351417in}{0.227470in}}%
\pgfpathlineto{\pgfqpoint{1.358387in}{0.229836in}}%
\pgfpathlineto{\pgfqpoint{1.363463in}{0.231569in}}%
\pgfpathlineto{\pgfqpoint{1.375509in}{0.235861in}}%
\pgfpathlineto{\pgfqpoint{1.387555in}{0.240346in}}%
\pgfpathlineto{\pgfqpoint{1.391521in}{0.241882in}}%
\pgfpathlineto{\pgfqpoint{1.399601in}{0.245032in}}%
\pgfpathlineto{\pgfqpoint{1.411647in}{0.249918in}}%
\pgfpathlineto{\pgfqpoint{1.421154in}{0.253928in}}%
\pgfpathlineto{\pgfqpoint{1.423693in}{0.255007in}}%
\pgfpathlineto{\pgfqpoint{1.435740in}{0.260309in}}%
\pgfpathlineto{\pgfqpoint{1.447786in}{0.265818in}}%
\pgfpathlineto{\pgfqpoint{1.448116in}{0.265974in}}%
\pgfpathlineto{\pgfqpoint{1.459832in}{0.271552in}}%
\pgfpathlineto{\pgfqpoint{1.471878in}{0.277500in}}%
\pgfpathlineto{\pgfqpoint{1.472899in}{0.278020in}}%
\pgfpathlineto{\pgfqpoint{1.483924in}{0.283683in}}%
\pgfpathlineto{\pgfqpoint{1.495927in}{0.290066in}}%
\pgfpathlineto{\pgfqpoint{1.495970in}{0.290089in}}%
\pgfpathlineto{\pgfqpoint{1.508016in}{0.296744in}}%
\pgfpathlineto{\pgfqpoint{1.517424in}{0.302112in}}%
\pgfpathlineto{\pgfqpoint{1.520062in}{0.303634in}}%
\pgfpathlineto{\pgfqpoint{1.532108in}{0.310780in}}%
\pgfpathlineto{\pgfqpoint{1.537640in}{0.314159in}}%
\pgfpathlineto{\pgfqpoint{1.544155in}{0.318183in}}%
\pgfpathlineto{\pgfqpoint{1.556201in}{0.325844in}}%
\pgfpathlineto{\pgfqpoint{1.556754in}{0.326205in}}%
\pgfpathlineto{\pgfqpoint{1.568247in}{0.333791in}}%
\pgfpathlineto{\pgfqpoint{1.574819in}{0.338251in}}%
\pgfpathlineto{\pgfqpoint{1.580293in}{0.342013in}}%
\pgfpathlineto{\pgfqpoint{1.592025in}{0.350297in}}%
\pgfpathlineto{\pgfqpoint{1.592339in}{0.350522in}}%
\pgfpathlineto{\pgfqpoint{1.604385in}{0.359344in}}%
\pgfpathlineto{\pgfqpoint{1.608378in}{0.362343in}}%
\pgfpathlineto{\pgfqpoint{1.616431in}{0.368478in}}%
\pgfpathlineto{\pgfqpoint{1.624005in}{0.374389in}}%
\pgfpathlineto{\pgfqpoint{1.628477in}{0.377933in}}%
\pgfpathlineto{\pgfqpoint{1.638959in}{0.386435in}}%
\pgfpathlineto{\pgfqpoint{1.640524in}{0.387724in}}%
\pgfpathlineto{\pgfqpoint{1.652570in}{0.397871in}}%
\pgfpathlineto{\pgfqpoint{1.653279in}{0.398481in}}%
\pgfpathlineto{\pgfqpoint{1.664616in}{0.408394in}}%
\pgfpathlineto{\pgfqpoint{1.667003in}{0.410527in}}%
\pgfpathlineto{\pgfqpoint{1.676662in}{0.419304in}}%
\pgfpathlineto{\pgfqpoint{1.680187in}{0.422574in}}%
\pgfpathlineto{\pgfqpoint{1.688708in}{0.430619in}}%
\pgfpathlineto{\pgfqpoint{1.692862in}{0.434620in}}%
\pgfpathlineto{\pgfqpoint{1.700754in}{0.442363in}}%
\pgfpathlineto{\pgfqpoint{1.705057in}{0.446666in}}%
\pgfpathlineto{\pgfqpoint{1.712800in}{0.454558in}}%
\pgfpathlineto{\pgfqpoint{1.716801in}{0.458712in}}%
\pgfpathlineto{\pgfqpoint{1.724846in}{0.467233in}}%
\pgfpathlineto{\pgfqpoint{1.728116in}{0.470758in}}%
\pgfpathlineto{\pgfqpoint{1.736892in}{0.480416in}}%
\pgfpathlineto{\pgfqpoint{1.739025in}{0.482804in}}%
\pgfpathlineto{\pgfqpoint{1.748939in}{0.494141in}}%
\pgfpathlineto{\pgfqpoint{1.749549in}{0.494850in}}%
\pgfpathlineto{\pgfqpoint{1.759696in}{0.506896in}}%
\pgfpathlineto{\pgfqpoint{1.760985in}{0.508460in}}%
\pgfpathlineto{\pgfqpoint{1.769487in}{0.518942in}}%
\pgfpathlineto{\pgfqpoint{1.773031in}{0.523415in}}%
\pgfpathlineto{\pgfqpoint{1.778942in}{0.530989in}}%
\pgfpathlineto{\pgfqpoint{1.785077in}{0.539042in}}%
\pgfpathlineto{\pgfqpoint{1.788075in}{0.543035in}}%
\pgfpathlineto{\pgfqpoint{1.796898in}{0.555081in}}%
\pgfpathlineto{\pgfqpoint{1.797123in}{0.555395in}}%
\pgfpathlineto{\pgfqpoint{1.805407in}{0.567127in}}%
\pgfpathlineto{\pgfqpoint{1.809169in}{0.572601in}}%
\pgfpathlineto{\pgfqpoint{1.813629in}{0.579173in}}%
\pgfpathlineto{\pgfqpoint{1.821215in}{0.590666in}}%
\pgfpathlineto{\pgfqpoint{1.821576in}{0.591219in}}%
\pgfpathlineto{\pgfqpoint{1.829237in}{0.603265in}}%
\pgfpathlineto{\pgfqpoint{1.833261in}{0.609780in}}%
\pgfpathlineto{\pgfqpoint{1.836640in}{0.615311in}}%
\pgfpathlineto{\pgfqpoint{1.843786in}{0.627358in}}%
\pgfpathlineto{\pgfqpoint{1.845307in}{0.629996in}}%
\pgfpathlineto{\pgfqpoint{1.850676in}{0.639404in}}%
\pgfpathlineto{\pgfqpoint{1.857331in}{0.651450in}}%
\pgfpathlineto{\pgfqpoint{1.857354in}{0.651493in}}%
\pgfpathlineto{\pgfqpoint{1.863737in}{0.663496in}}%
\pgfpathlineto{\pgfqpoint{1.869400in}{0.674521in}}%
\pgfpathlineto{\pgfqpoint{1.869920in}{0.675542in}}%
\pgfpathlineto{\pgfqpoint{1.875868in}{0.687588in}}%
\pgfpathlineto{\pgfqpoint{1.881446in}{0.699304in}}%
\pgfpathlineto{\pgfqpoint{1.881602in}{0.699634in}}%
\pgfpathlineto{\pgfqpoint{1.887111in}{0.711680in}}%
\pgfpathlineto{\pgfqpoint{1.892413in}{0.723726in}}%
\pgfpathlineto{\pgfqpoint{1.893492in}{0.726265in}}%
\pgfpathlineto{\pgfqpoint{1.897502in}{0.735773in}}%
\pgfpathlineto{\pgfqpoint{1.902388in}{0.747819in}}%
\pgfpathlineto{\pgfqpoint{1.905538in}{0.755899in}}%
\pgfpathlineto{\pgfqpoint{1.907074in}{0.759865in}}%
\pgfpathlineto{\pgfqpoint{1.911559in}{0.771911in}}%
\pgfpathlineto{\pgfqpoint{1.915851in}{0.783957in}}%
\pgfpathlineto{\pgfqpoint{1.917584in}{0.789033in}}%
\pgfpathlineto{\pgfqpoint{1.919950in}{0.796003in}}%
\pgfpathlineto{\pgfqpoint{1.923860in}{0.808049in}}%
\pgfpathlineto{\pgfqpoint{1.927584in}{0.820095in}}%
\pgfpathlineto{\pgfqpoint{1.929630in}{0.827045in}}%
\pgfpathlineto{\pgfqpoint{1.931124in}{0.832141in}}%
\pgfpathlineto{\pgfqpoint{1.934480in}{0.844188in}}%
\pgfpathlineto{\pgfqpoint{1.937658in}{0.856234in}}%
\pgfpathlineto{\pgfqpoint{1.940658in}{0.868280in}}%
\pgfpathlineto{\pgfqpoint{1.941676in}{0.872616in}}%
\pgfpathlineto{\pgfqpoint{1.943481in}{0.880326in}}%
\pgfpathlineto{\pgfqpoint{1.946130in}{0.892372in}}%
\pgfpathlineto{\pgfqpoint{1.948606in}{0.904418in}}%
\pgfpathlineto{\pgfqpoint{1.950911in}{0.916464in}}%
\pgfpathlineto{\pgfqpoint{1.953046in}{0.928510in}}%
\pgfpathlineto{\pgfqpoint{1.953723in}{0.932648in}}%
\pgfpathlineto{\pgfqpoint{1.955012in}{0.940557in}}%
\pgfpathlineto{\pgfqpoint{1.956810in}{0.952603in}}%
\pgfpathlineto{\pgfqpoint{1.958442in}{0.964649in}}%
\pgfpathlineto{\pgfqpoint{1.959908in}{0.976695in}}%
\pgfpathlineto{\pgfqpoint{1.961209in}{0.988741in}}%
\pgfpathlineto{\pgfqpoint{1.962346in}{1.000787in}}%
\pgfpathlineto{\pgfqpoint{1.963320in}{1.012833in}}%
\pgfpathlineto{\pgfqpoint{1.964130in}{1.024879in}}%
\pgfpathlineto{\pgfqpoint{1.964778in}{1.036925in}}%
\pgfpathlineto{\pgfqpoint{1.965263in}{1.048972in}}%
\pgfpathlineto{\pgfqpoint{1.965587in}{1.061018in}}%
\pgfpathlineto{\pgfqpoint{1.965748in}{1.073064in}}%
\pgfpathlineto{\pgfqpoint{1.965748in}{1.085110in}}%
\pgfpathlineto{\pgfqpoint{1.965587in}{1.097156in}}%
\pgfpathlineto{\pgfqpoint{1.965263in}{1.109202in}}%
\pgfpathlineto{\pgfqpoint{1.964778in}{1.121248in}}%
\pgfpathlineto{\pgfqpoint{1.964130in}{1.133294in}}%
\pgfpathlineto{\pgfqpoint{1.963320in}{1.145340in}}%
\pgfpathlineto{\pgfqpoint{1.962346in}{1.157387in}}%
\pgfpathlineto{\pgfqpoint{1.961209in}{1.169433in}}%
\pgfpathlineto{\pgfqpoint{1.959908in}{1.181479in}}%
\pgfpathlineto{\pgfqpoint{1.958442in}{1.193525in}}%
\pgfpathlineto{\pgfqpoint{1.956810in}{1.205571in}}%
\pgfpathlineto{\pgfqpoint{1.955012in}{1.217617in}}%
\pgfpathlineto{\pgfqpoint{1.953723in}{1.225525in}}%
\pgfpathlineto{\pgfqpoint{1.953046in}{1.229663in}}%
\pgfpathlineto{\pgfqpoint{1.950911in}{1.241709in}}%
\pgfpathlineto{\pgfqpoint{1.948606in}{1.253756in}}%
\pgfpathlineto{\pgfqpoint{1.946130in}{1.265802in}}%
\pgfpathlineto{\pgfqpoint{1.943481in}{1.277848in}}%
\pgfpathlineto{\pgfqpoint{1.941676in}{1.285558in}}%
\pgfpathlineto{\pgfqpoint{1.940658in}{1.289894in}}%
\pgfpathlineto{\pgfqpoint{1.937658in}{1.301940in}}%
\pgfpathlineto{\pgfqpoint{1.934480in}{1.313986in}}%
\pgfpathlineto{\pgfqpoint{1.931124in}{1.326032in}}%
\pgfpathlineto{\pgfqpoint{1.929630in}{1.331129in}}%
\pgfpathlineto{\pgfqpoint{1.927584in}{1.338078in}}%
\pgfpathlineto{\pgfqpoint{1.923860in}{1.350124in}}%
\pgfpathlineto{\pgfqpoint{1.919950in}{1.362171in}}%
\pgfpathlineto{\pgfqpoint{1.917584in}{1.369141in}}%
\pgfpathlineto{\pgfqpoint{1.915851in}{1.374217in}}%
\pgfpathlineto{\pgfqpoint{1.911559in}{1.386263in}}%
\pgfpathlineto{\pgfqpoint{1.907074in}{1.398309in}}%
\pgfpathlineto{\pgfqpoint{1.905538in}{1.402275in}}%
\pgfpathlineto{\pgfqpoint{1.902388in}{1.410355in}}%
\pgfpathlineto{\pgfqpoint{1.897502in}{1.422401in}}%
\pgfpathlineto{\pgfqpoint{1.893492in}{1.431908in}}%
\pgfpathlineto{\pgfqpoint{1.892413in}{1.434447in}}%
\pgfpathlineto{\pgfqpoint{1.887111in}{1.446493in}}%
\pgfpathlineto{\pgfqpoint{1.881602in}{1.458539in}}%
\pgfpathlineto{\pgfqpoint{1.881446in}{1.458870in}}%
\pgfpathlineto{\pgfqpoint{1.875868in}{1.470586in}}%
\pgfpathlineto{\pgfqpoint{1.869920in}{1.482632in}}%
\pgfpathlineto{\pgfqpoint{1.869400in}{1.483653in}}%
\pgfpathlineto{\pgfqpoint{1.863737in}{1.494678in}}%
\pgfpathlineto{\pgfqpoint{1.857354in}{1.506681in}}%
\pgfpathlineto{\pgfqpoint{1.857331in}{1.506724in}}%
\pgfpathlineto{\pgfqpoint{1.850676in}{1.518770in}}%
\pgfpathlineto{\pgfqpoint{1.845307in}{1.528178in}}%
\pgfpathlineto{\pgfqpoint{1.843786in}{1.530816in}}%
\pgfpathlineto{\pgfqpoint{1.836640in}{1.542862in}}%
\pgfpathlineto{\pgfqpoint{1.833261in}{1.548394in}}%
\pgfpathlineto{\pgfqpoint{1.829237in}{1.554908in}}%
\pgfpathlineto{\pgfqpoint{1.821576in}{1.566955in}}%
\pgfpathlineto{\pgfqpoint{1.821215in}{1.567507in}}%
\pgfpathlineto{\pgfqpoint{1.813629in}{1.579001in}}%
\pgfpathlineto{\pgfqpoint{1.809169in}{1.585573in}}%
\pgfpathlineto{\pgfqpoint{1.805407in}{1.591047in}}%
\pgfpathlineto{\pgfqpoint{1.797123in}{1.602779in}}%
\pgfpathlineto{\pgfqpoint{1.796898in}{1.603093in}}%
\pgfpathlineto{\pgfqpoint{1.788075in}{1.615139in}}%
\pgfpathlineto{\pgfqpoint{1.785077in}{1.619132in}}%
\pgfpathlineto{\pgfqpoint{1.778942in}{1.627185in}}%
\pgfpathlineto{\pgfqpoint{1.773031in}{1.634758in}}%
\pgfpathlineto{\pgfqpoint{1.769487in}{1.639231in}}%
\pgfpathlineto{\pgfqpoint{1.760985in}{1.649713in}}%
\pgfpathlineto{\pgfqpoint{1.759696in}{1.651277in}}%
\pgfpathlineto{\pgfqpoint{1.749549in}{1.663323in}}%
\pgfpathlineto{\pgfqpoint{1.748939in}{1.664033in}}%
\pgfpathlineto{\pgfqpoint{1.739025in}{1.675370in}}%
\pgfpathlineto{\pgfqpoint{1.736892in}{1.677757in}}%
\pgfpathlineto{\pgfqpoint{1.728116in}{1.687416in}}%
\pgfpathlineto{\pgfqpoint{1.724846in}{1.690941in}}%
\pgfpathlineto{\pgfqpoint{1.716801in}{1.699462in}}%
\pgfpathlineto{\pgfqpoint{1.712800in}{1.703615in}}%
\pgfpathlineto{\pgfqpoint{1.705057in}{1.711508in}}%
\pgfpathlineto{\pgfqpoint{1.700754in}{1.715811in}}%
\pgfpathlineto{\pgfqpoint{1.692862in}{1.723554in}}%
\pgfpathlineto{\pgfqpoint{1.688708in}{1.727554in}}%
\pgfpathlineto{\pgfqpoint{1.680187in}{1.735600in}}%
\pgfpathlineto{\pgfqpoint{1.676662in}{1.738870in}}%
\pgfpathlineto{\pgfqpoint{1.667003in}{1.747646in}}%
\pgfpathlineto{\pgfqpoint{1.664616in}{1.749779in}}%
\pgfpathlineto{\pgfqpoint{1.653279in}{1.759692in}}%
\pgfpathlineto{\pgfqpoint{1.652570in}{1.760303in}}%
\pgfpathlineto{\pgfqpoint{1.640524in}{1.770450in}}%
\pgfpathlineto{\pgfqpoint{1.638959in}{1.771738in}}%
\pgfpathlineto{\pgfqpoint{1.628477in}{1.780241in}}%
\pgfpathlineto{\pgfqpoint{1.624005in}{1.783785in}}%
\pgfpathlineto{\pgfqpoint{1.616431in}{1.789696in}}%
\pgfpathlineto{\pgfqpoint{1.608378in}{1.795831in}}%
\pgfpathlineto{\pgfqpoint{1.604385in}{1.798829in}}%
\pgfpathlineto{\pgfqpoint{1.592339in}{1.807652in}}%
\pgfpathlineto{\pgfqpoint{1.592025in}{1.807877in}}%
\pgfpathlineto{\pgfqpoint{1.580293in}{1.816160in}}%
\pgfpathlineto{\pgfqpoint{1.574819in}{1.819923in}}%
\pgfpathlineto{\pgfqpoint{1.568247in}{1.824383in}}%
\pgfpathlineto{\pgfqpoint{1.556754in}{1.831969in}}%
\pgfpathlineto{\pgfqpoint{1.556201in}{1.832330in}}%
\pgfpathlineto{\pgfqpoint{1.544155in}{1.839991in}}%
\pgfpathlineto{\pgfqpoint{1.537640in}{1.844015in}}%
\pgfpathlineto{\pgfqpoint{1.532108in}{1.847394in}}%
\pgfpathlineto{\pgfqpoint{1.520062in}{1.854540in}}%
\pgfpathlineto{\pgfqpoint{1.517424in}{1.856061in}}%
\pgfpathlineto{\pgfqpoint{1.508016in}{1.861430in}}%
\pgfpathlineto{\pgfqpoint{1.495970in}{1.868084in}}%
\pgfpathlineto{\pgfqpoint{1.495927in}{1.868107in}}%
\pgfpathlineto{\pgfqpoint{1.483924in}{1.874491in}}%
\pgfpathlineto{\pgfqpoint{1.472899in}{1.880154in}}%
\pgfpathlineto{\pgfqpoint{1.471878in}{1.880673in}}%
\pgfpathlineto{\pgfqpoint{1.459832in}{1.886622in}}%
\pgfpathlineto{\pgfqpoint{1.448116in}{1.892200in}}%
\pgfpathlineto{\pgfqpoint{1.447786in}{1.892356in}}%
\pgfpathlineto{\pgfqpoint{1.435740in}{1.897865in}}%
\pgfpathlineto{\pgfqpoint{1.423693in}{1.903167in}}%
\pgfpathlineto{\pgfqpoint{1.421154in}{1.904246in}}%
\pgfpathlineto{\pgfqpoint{1.411647in}{1.908256in}}%
\pgfpathlineto{\pgfqpoint{1.399601in}{1.913142in}}%
\pgfpathlineto{\pgfqpoint{1.391521in}{1.916292in}}%
\pgfpathlineto{\pgfqpoint{1.387555in}{1.917828in}}%
\pgfpathlineto{\pgfqpoint{1.375509in}{1.922312in}}%
\pgfpathlineto{\pgfqpoint{1.363463in}{1.926605in}}%
\pgfpathlineto{\pgfqpoint{1.358387in}{1.928338in}}%
\pgfpathlineto{\pgfqpoint{1.351417in}{1.930704in}}%
\pgfpathlineto{\pgfqpoint{1.339371in}{1.934614in}}%
\pgfpathlineto{\pgfqpoint{1.327325in}{1.938338in}}%
\pgfpathlineto{\pgfqpoint{1.320375in}{1.940384in}}%
\pgfpathlineto{\pgfqpoint{1.315278in}{1.941877in}}%
\pgfpathlineto{\pgfqpoint{1.303232in}{1.945234in}}%
\pgfpathlineto{\pgfqpoint{1.291186in}{1.948411in}}%
\pgfpathlineto{\pgfqpoint{1.279140in}{1.951412in}}%
\pgfpathlineto{\pgfqpoint{1.274804in}{1.952430in}}%
\pgfpathlineto{\pgfqpoint{1.267094in}{1.954235in}}%
\pgfpathlineto{\pgfqpoint{1.255048in}{1.956883in}}%
\pgfpathlineto{\pgfqpoint{1.243002in}{1.959360in}}%
\pgfpathlineto{\pgfqpoint{1.230956in}{1.961665in}}%
\pgfpathlineto{\pgfqpoint{1.218909in}{1.963800in}}%
\pgfpathlineto{\pgfqpoint{1.214772in}{1.964476in}}%
\pgfpathlineto{\pgfqpoint{1.206863in}{1.965766in}}%
\pgfpathlineto{\pgfqpoint{1.194817in}{1.967564in}}%
\pgfpathlineto{\pgfqpoint{1.182771in}{1.969196in}}%
\pgfpathlineto{\pgfqpoint{1.170725in}{1.970662in}}%
\pgfpathlineto{\pgfqpoint{1.158679in}{1.971963in}}%
\pgfpathlineto{\pgfqpoint{1.146633in}{1.973100in}}%
\pgfpathlineto{\pgfqpoint{1.134587in}{1.974073in}}%
\pgfpathlineto{\pgfqpoint{1.122541in}{1.974884in}}%
\pgfpathlineto{\pgfqpoint{1.110494in}{1.975532in}}%
\pgfpathlineto{\pgfqpoint{1.098448in}{1.976017in}}%
\pgfpathlineto{\pgfqpoint{1.086402in}{1.976341in}}%
\pgfpathlineto{\pgfqpoint{1.074356in}{1.976502in}}%
\pgfpathlineto{\pgfqpoint{1.062310in}{1.976502in}}%
\pgfpathlineto{\pgfqpoint{1.050264in}{1.976341in}}%
\pgfpathlineto{\pgfqpoint{1.038218in}{1.976017in}}%
\pgfpathlineto{\pgfqpoint{1.026172in}{1.975532in}}%
\pgfpathlineto{\pgfqpoint{1.014126in}{1.974884in}}%
\pgfpathlineto{\pgfqpoint{1.002079in}{1.974073in}}%
\pgfpathlineto{\pgfqpoint{0.990033in}{1.973100in}}%
\pgfpathlineto{\pgfqpoint{0.977987in}{1.971963in}}%
\pgfpathlineto{\pgfqpoint{0.965941in}{1.970662in}}%
\pgfpathlineto{\pgfqpoint{0.953895in}{1.969196in}}%
\pgfpathlineto{\pgfqpoint{0.941849in}{1.967564in}}%
\pgfpathlineto{\pgfqpoint{0.929803in}{1.965766in}}%
\pgfpathlineto{\pgfqpoint{0.921894in}{1.964476in}}%
\pgfpathlineto{\pgfqpoint{0.917757in}{1.963800in}}%
\pgfpathlineto{\pgfqpoint{0.905710in}{1.961665in}}%
\pgfpathlineto{\pgfqpoint{0.893664in}{1.959360in}}%
\pgfpathlineto{\pgfqpoint{0.881618in}{1.956883in}}%
\pgfpathlineto{\pgfqpoint{0.869572in}{1.954235in}}%
\pgfpathlineto{\pgfqpoint{0.861862in}{1.952430in}}%
\pgfpathlineto{\pgfqpoint{0.857526in}{1.951412in}}%
\pgfpathlineto{\pgfqpoint{0.845480in}{1.948411in}}%
\pgfpathlineto{\pgfqpoint{0.833434in}{1.945234in}}%
\pgfpathlineto{\pgfqpoint{0.821388in}{1.941877in}}%
\pgfpathlineto{\pgfqpoint{0.816291in}{1.940384in}}%
\pgfpathlineto{\pgfqpoint{0.809342in}{1.938338in}}%
\pgfpathlineto{\pgfqpoint{0.797295in}{1.934614in}}%
\pgfpathlineto{\pgfqpoint{0.785249in}{1.930704in}}%
\pgfpathlineto{\pgfqpoint{0.778279in}{1.928338in}}%
\pgfpathlineto{\pgfqpoint{0.773203in}{1.926605in}}%
\pgfpathlineto{\pgfqpoint{0.761157in}{1.922312in}}%
\pgfpathlineto{\pgfqpoint{0.749111in}{1.917828in}}%
\pgfpathlineto{\pgfqpoint{0.745145in}{1.916292in}}%
\pgfpathlineto{\pgfqpoint{0.737065in}{1.913142in}}%
\pgfpathlineto{\pgfqpoint{0.725019in}{1.908256in}}%
\pgfpathlineto{\pgfqpoint{0.715512in}{1.904246in}}%
\pgfpathlineto{\pgfqpoint{0.712973in}{1.903167in}}%
\pgfpathlineto{\pgfqpoint{0.700927in}{1.897865in}}%
\pgfpathlineto{\pgfqpoint{0.688880in}{1.892356in}}%
\pgfpathlineto{\pgfqpoint{0.688550in}{1.892200in}}%
\pgfpathlineto{\pgfqpoint{0.676834in}{1.886622in}}%
\pgfpathlineto{\pgfqpoint{0.664788in}{1.880673in}}%
\pgfpathlineto{\pgfqpoint{0.663767in}{1.880154in}}%
\pgfpathlineto{\pgfqpoint{0.652742in}{1.874491in}}%
\pgfpathlineto{\pgfqpoint{0.640739in}{1.868107in}}%
\pgfpathlineto{\pgfqpoint{0.640696in}{1.868084in}}%
\pgfpathlineto{\pgfqpoint{0.628650in}{1.861430in}}%
\pgfpathlineto{\pgfqpoint{0.619242in}{1.856061in}}%
\pgfpathlineto{\pgfqpoint{0.616604in}{1.854540in}}%
\pgfpathlineto{\pgfqpoint{0.604558in}{1.847394in}}%
\pgfpathlineto{\pgfqpoint{0.599026in}{1.844015in}}%
\pgfpathlineto{\pgfqpoint{0.592511in}{1.839991in}}%
\pgfpathlineto{\pgfqpoint{0.580465in}{1.832330in}}%
\pgfpathlineto{\pgfqpoint{0.579913in}{1.831969in}}%
\pgfpathlineto{\pgfqpoint{0.568419in}{1.824383in}}%
\pgfpathlineto{\pgfqpoint{0.561847in}{1.819923in}}%
\pgfpathlineto{\pgfqpoint{0.556373in}{1.816160in}}%
\pgfpathlineto{\pgfqpoint{0.544641in}{1.807877in}}%
\pgfpathlineto{\pgfqpoint{0.544327in}{1.807652in}}%
\pgfpathlineto{\pgfqpoint{0.532281in}{1.798829in}}%
\pgfpathlineto{\pgfqpoint{0.528288in}{1.795831in}}%
\pgfpathlineto{\pgfqpoint{0.520235in}{1.789696in}}%
\pgfpathlineto{\pgfqpoint{0.512662in}{1.783785in}}%
\pgfpathlineto{\pgfqpoint{0.508189in}{1.780241in}}%
\pgfpathlineto{\pgfqpoint{0.497707in}{1.771738in}}%
\pgfpathlineto{\pgfqpoint{0.496143in}{1.770450in}}%
\pgfpathlineto{\pgfqpoint{0.484096in}{1.760303in}}%
\pgfpathlineto{\pgfqpoint{0.483387in}{1.759692in}}%
\pgfpathlineto{\pgfqpoint{0.472050in}{1.749779in}}%
\pgfpathlineto{\pgfqpoint{0.469663in}{1.747646in}}%
\pgfpathlineto{\pgfqpoint{0.460004in}{1.738870in}}%
\pgfpathlineto{\pgfqpoint{0.456479in}{1.735600in}}%
\pgfpathlineto{\pgfqpoint{0.447958in}{1.727554in}}%
\pgfpathlineto{\pgfqpoint{0.443805in}{1.723554in}}%
\pgfpathlineto{\pgfqpoint{0.435912in}{1.715811in}}%
\pgfpathlineto{\pgfqpoint{0.431609in}{1.711508in}}%
\pgfpathlineto{\pgfqpoint{0.423866in}{1.703615in}}%
\pgfpathlineto{\pgfqpoint{0.419865in}{1.699462in}}%
\pgfpathlineto{\pgfqpoint{0.411820in}{1.690941in}}%
\pgfpathlineto{\pgfqpoint{0.408550in}{1.687416in}}%
\pgfpathlineto{\pgfqpoint{0.399774in}{1.677757in}}%
\pgfpathlineto{\pgfqpoint{0.397641in}{1.675370in}}%
\pgfpathlineto{\pgfqpoint{0.387728in}{1.664033in}}%
\pgfpathlineto{\pgfqpoint{0.387117in}{1.663323in}}%
\pgfpathlineto{\pgfqpoint{0.376970in}{1.651277in}}%
\pgfpathlineto{\pgfqpoint{0.375681in}{1.649713in}}%
\pgfpathlineto{\pgfqpoint{0.367179in}{1.639231in}}%
\pgfpathlineto{\pgfqpoint{0.363635in}{1.634758in}}%
\pgfpathlineto{\pgfqpoint{0.357724in}{1.627185in}}%
\pgfpathlineto{\pgfqpoint{0.351589in}{1.619132in}}%
\pgfpathlineto{\pgfqpoint{0.348591in}{1.615139in}}%
\pgfpathlineto{\pgfqpoint{0.339768in}{1.603093in}}%
\pgfpathlineto{\pgfqpoint{0.339543in}{1.602779in}}%
\pgfpathlineto{\pgfqpoint{0.331259in}{1.591047in}}%
\pgfpathlineto{\pgfqpoint{0.327497in}{1.585573in}}%
\pgfpathlineto{\pgfqpoint{0.323037in}{1.579001in}}%
\pgfpathlineto{\pgfqpoint{0.315451in}{1.567507in}}%
\pgfpathlineto{\pgfqpoint{0.315090in}{1.566955in}}%
\pgfpathlineto{\pgfqpoint{0.307429in}{1.554908in}}%
\pgfpathlineto{\pgfqpoint{0.303405in}{1.548394in}}%
\pgfpathlineto{\pgfqpoint{0.300026in}{1.542862in}}%
\pgfpathlineto{\pgfqpoint{0.292880in}{1.530816in}}%
\pgfpathlineto{\pgfqpoint{0.291359in}{1.528178in}}%
\pgfpathlineto{\pgfqpoint{0.285990in}{1.518770in}}%
\pgfpathlineto{\pgfqpoint{0.279335in}{1.506724in}}%
\pgfpathlineto{\pgfqpoint{0.279312in}{1.506681in}}%
\pgfpathlineto{\pgfqpoint{0.272929in}{1.494678in}}%
\pgfpathlineto{\pgfqpoint{0.267266in}{1.483653in}}%
\pgfpathlineto{\pgfqpoint{0.266746in}{1.482632in}}%
\pgfpathlineto{\pgfqpoint{0.260798in}{1.470586in}}%
\pgfpathlineto{\pgfqpoint{0.255220in}{1.458870in}}%
\pgfpathlineto{\pgfqpoint{0.255064in}{1.458539in}}%
\pgfpathlineto{\pgfqpoint{0.249555in}{1.446493in}}%
\pgfpathlineto{\pgfqpoint{0.244253in}{1.434447in}}%
\pgfpathlineto{\pgfqpoint{0.243174in}{1.431908in}}%
\pgfpathlineto{\pgfqpoint{0.239164in}{1.422401in}}%
\pgfpathlineto{\pgfqpoint{0.234278in}{1.410355in}}%
\pgfpathlineto{\pgfqpoint{0.231128in}{1.402275in}}%
\pgfpathlineto{\pgfqpoint{0.229592in}{1.398309in}}%
\pgfpathlineto{\pgfqpoint{0.225107in}{1.386263in}}%
\pgfpathlineto{\pgfqpoint{0.220815in}{1.374217in}}%
\pgfpathlineto{\pgfqpoint{0.219082in}{1.369141in}}%
\pgfpathlineto{\pgfqpoint{0.216716in}{1.362171in}}%
\pgfpathlineto{\pgfqpoint{0.212806in}{1.350124in}}%
\pgfpathlineto{\pgfqpoint{0.209082in}{1.338078in}}%
\pgfpathlineto{\pgfqpoint{0.207036in}{1.331129in}}%
\pgfpathlineto{\pgfqpoint{0.205543in}{1.326032in}}%
\pgfpathlineto{\pgfqpoint{0.202186in}{1.313986in}}%
\pgfpathlineto{\pgfqpoint{0.199008in}{1.301940in}}%
\pgfpathlineto{\pgfqpoint{0.196008in}{1.289894in}}%
\pgfpathlineto{\pgfqpoint{0.194990in}{1.285558in}}%
\pgfpathlineto{\pgfqpoint{0.193185in}{1.277848in}}%
\pgfpathlineto{\pgfqpoint{0.190536in}{1.265802in}}%
\pgfpathlineto{\pgfqpoint{0.188060in}{1.253756in}}%
\pgfpathlineto{\pgfqpoint{0.185755in}{1.241709in}}%
\pgfpathlineto{\pgfqpoint{0.183620in}{1.229663in}}%
\pgfpathlineto{\pgfqpoint{0.182944in}{1.225525in}}%
\pgfpathlineto{\pgfqpoint{0.181654in}{1.217617in}}%
\pgfpathlineto{\pgfqpoint{0.179856in}{1.205571in}}%
\pgfpathlineto{\pgfqpoint{0.178224in}{1.193525in}}%
\pgfpathlineto{\pgfqpoint{0.176758in}{1.181479in}}%
\pgfpathlineto{\pgfqpoint{0.175457in}{1.169433in}}%
\pgfpathlineto{\pgfqpoint{0.174320in}{1.157387in}}%
\pgfpathlineto{\pgfqpoint{0.173346in}{1.145340in}}%
\pgfpathlineto{\pgfqpoint{0.172536in}{1.133294in}}%
\pgfpathlineto{\pgfqpoint{0.171888in}{1.121248in}}%
\pgfpathlineto{\pgfqpoint{0.171403in}{1.109202in}}%
\pgfpathlineto{\pgfqpoint{0.171079in}{1.097156in}}%
\pgfpathlineto{\pgfqpoint{0.170918in}{1.085110in}}%
\pgfpathlineto{\pgfqpoint{0.170918in}{1.073064in}}%
\pgfpathlineto{\pgfqpoint{0.171079in}{1.061018in}}%
\pgfpathlineto{\pgfqpoint{0.171403in}{1.048972in}}%
\pgfpathlineto{\pgfqpoint{0.171888in}{1.036925in}}%
\pgfpathlineto{\pgfqpoint{0.172536in}{1.024879in}}%
\pgfpathlineto{\pgfqpoint{0.173346in}{1.012833in}}%
\pgfpathlineto{\pgfqpoint{0.174320in}{1.000787in}}%
\pgfpathlineto{\pgfqpoint{0.175457in}{0.988741in}}%
\pgfpathlineto{\pgfqpoint{0.176758in}{0.976695in}}%
\pgfpathlineto{\pgfqpoint{0.178224in}{0.964649in}}%
\pgfpathlineto{\pgfqpoint{0.179856in}{0.952603in}}%
\pgfpathlineto{\pgfqpoint{0.181654in}{0.940557in}}%
\pgfpathlineto{\pgfqpoint{0.182944in}{0.932648in}}%
\pgfpathlineto{\pgfqpoint{0.183620in}{0.928510in}}%
\pgfpathlineto{\pgfqpoint{0.185755in}{0.916464in}}%
\pgfpathlineto{\pgfqpoint{0.188060in}{0.904418in}}%
\pgfpathlineto{\pgfqpoint{0.190536in}{0.892372in}}%
\pgfpathlineto{\pgfqpoint{0.193185in}{0.880326in}}%
\pgfpathlineto{\pgfqpoint{0.194990in}{0.872616in}}%
\pgfpathlineto{\pgfqpoint{0.196008in}{0.868280in}}%
\pgfpathlineto{\pgfqpoint{0.199008in}{0.856234in}}%
\pgfpathlineto{\pgfqpoint{0.202186in}{0.844188in}}%
\pgfpathlineto{\pgfqpoint{0.205543in}{0.832141in}}%
\pgfpathlineto{\pgfqpoint{0.207036in}{0.827045in}}%
\pgfpathlineto{\pgfqpoint{0.209082in}{0.820095in}}%
\pgfpathlineto{\pgfqpoint{0.212806in}{0.808049in}}%
\pgfpathlineto{\pgfqpoint{0.216716in}{0.796003in}}%
\pgfpathlineto{\pgfqpoint{0.219082in}{0.789033in}}%
\pgfpathlineto{\pgfqpoint{0.220815in}{0.783957in}}%
\pgfpathlineto{\pgfqpoint{0.225107in}{0.771911in}}%
\pgfpathlineto{\pgfqpoint{0.229592in}{0.759865in}}%
\pgfpathlineto{\pgfqpoint{0.231128in}{0.755899in}}%
\pgfpathlineto{\pgfqpoint{0.234278in}{0.747819in}}%
\pgfpathlineto{\pgfqpoint{0.239164in}{0.735773in}}%
\pgfpathlineto{\pgfqpoint{0.243174in}{0.726265in}}%
\pgfpathlineto{\pgfqpoint{0.244253in}{0.723726in}}%
\pgfpathlineto{\pgfqpoint{0.249555in}{0.711680in}}%
\pgfpathlineto{\pgfqpoint{0.255064in}{0.699634in}}%
\pgfpathlineto{\pgfqpoint{0.255220in}{0.699304in}}%
\pgfpathlineto{\pgfqpoint{0.260798in}{0.687588in}}%
\pgfpathlineto{\pgfqpoint{0.266746in}{0.675542in}}%
\pgfpathlineto{\pgfqpoint{0.267266in}{0.674521in}}%
\pgfpathlineto{\pgfqpoint{0.272929in}{0.663496in}}%
\pgfpathlineto{\pgfqpoint{0.279312in}{0.651493in}}%
\pgfpathlineto{\pgfqpoint{0.279335in}{0.651450in}}%
\pgfpathlineto{\pgfqpoint{0.285990in}{0.639404in}}%
\pgfpathlineto{\pgfqpoint{0.291359in}{0.629996in}}%
\pgfpathlineto{\pgfqpoint{0.292880in}{0.627358in}}%
\pgfpathlineto{\pgfqpoint{0.300026in}{0.615311in}}%
\pgfpathlineto{\pgfqpoint{0.303405in}{0.609780in}}%
\pgfpathlineto{\pgfqpoint{0.307429in}{0.603265in}}%
\pgfpathlineto{\pgfqpoint{0.315090in}{0.591219in}}%
\pgfpathlineto{\pgfqpoint{0.315451in}{0.590666in}}%
\pgfpathlineto{\pgfqpoint{0.323037in}{0.579173in}}%
\pgfpathlineto{\pgfqpoint{0.327497in}{0.572601in}}%
\pgfpathlineto{\pgfqpoint{0.331259in}{0.567127in}}%
\pgfpathlineto{\pgfqpoint{0.339543in}{0.555395in}}%
\pgfpathlineto{\pgfqpoint{0.339768in}{0.555081in}}%
\pgfpathlineto{\pgfqpoint{0.348591in}{0.543035in}}%
\pgfpathlineto{\pgfqpoint{0.351589in}{0.539042in}}%
\pgfpathlineto{\pgfqpoint{0.357724in}{0.530989in}}%
\pgfpathlineto{\pgfqpoint{0.363635in}{0.523415in}}%
\pgfpathlineto{\pgfqpoint{0.367179in}{0.518942in}}%
\pgfpathlineto{\pgfqpoint{0.375681in}{0.508460in}}%
\pgfpathlineto{\pgfqpoint{0.376970in}{0.506896in}}%
\pgfpathlineto{\pgfqpoint{0.387117in}{0.494850in}}%
\pgfpathlineto{\pgfqpoint{0.387728in}{0.494141in}}%
\pgfpathlineto{\pgfqpoint{0.397641in}{0.482804in}}%
\pgfpathlineto{\pgfqpoint{0.399774in}{0.480416in}}%
\pgfpathlineto{\pgfqpoint{0.408550in}{0.470758in}}%
\pgfpathlineto{\pgfqpoint{0.411820in}{0.467233in}}%
\pgfpathlineto{\pgfqpoint{0.419865in}{0.458712in}}%
\pgfpathlineto{\pgfqpoint{0.423866in}{0.454558in}}%
\pgfpathlineto{\pgfqpoint{0.431609in}{0.446666in}}%
\pgfpathlineto{\pgfqpoint{0.435912in}{0.442363in}}%
\pgfpathlineto{\pgfqpoint{0.443805in}{0.434620in}}%
\pgfpathlineto{\pgfqpoint{0.447958in}{0.430619in}}%
\pgfpathlineto{\pgfqpoint{0.456479in}{0.422574in}}%
\pgfpathlineto{\pgfqpoint{0.460004in}{0.419304in}}%
\pgfpathlineto{\pgfqpoint{0.469663in}{0.410527in}}%
\pgfpathlineto{\pgfqpoint{0.472050in}{0.408394in}}%
\pgfpathlineto{\pgfqpoint{0.483387in}{0.398481in}}%
\pgfpathlineto{\pgfqpoint{0.484096in}{0.397871in}}%
\pgfpathlineto{\pgfqpoint{0.496143in}{0.387724in}}%
\pgfpathlineto{\pgfqpoint{0.497707in}{0.386435in}}%
\pgfpathlineto{\pgfqpoint{0.508189in}{0.377933in}}%
\pgfpathlineto{\pgfqpoint{0.512662in}{0.374389in}}%
\pgfpathlineto{\pgfqpoint{0.520235in}{0.368478in}}%
\pgfpathlineto{\pgfqpoint{0.528288in}{0.362343in}}%
\pgfpathlineto{\pgfqpoint{0.532281in}{0.359344in}}%
\pgfpathlineto{\pgfqpoint{0.544327in}{0.350522in}}%
\pgfpathlineto{\pgfqpoint{0.544641in}{0.350297in}}%
\pgfpathlineto{\pgfqpoint{0.556373in}{0.342013in}}%
\pgfpathlineto{\pgfqpoint{0.561847in}{0.338251in}}%
\pgfpathlineto{\pgfqpoint{0.568419in}{0.333791in}}%
\pgfpathlineto{\pgfqpoint{0.579913in}{0.326205in}}%
\pgfpathlineto{\pgfqpoint{0.580465in}{0.325844in}}%
\pgfpathlineto{\pgfqpoint{0.592511in}{0.318183in}}%
\pgfpathlineto{\pgfqpoint{0.599026in}{0.314159in}}%
\pgfpathlineto{\pgfqpoint{0.604558in}{0.310780in}}%
\pgfpathlineto{\pgfqpoint{0.616604in}{0.303634in}}%
\pgfpathlineto{\pgfqpoint{0.619242in}{0.302112in}}%
\pgfpathlineto{\pgfqpoint{0.628650in}{0.296744in}}%
\pgfpathlineto{\pgfqpoint{0.640696in}{0.290089in}}%
\pgfpathlineto{\pgfqpoint{0.640739in}{0.290066in}}%
\pgfpathlineto{\pgfqpoint{0.652742in}{0.283683in}}%
\pgfpathlineto{\pgfqpoint{0.663767in}{0.278020in}}%
\pgfpathlineto{\pgfqpoint{0.664788in}{0.277500in}}%
\pgfpathlineto{\pgfqpoint{0.676834in}{0.271552in}}%
\pgfpathlineto{\pgfqpoint{0.688550in}{0.265974in}}%
\pgfpathlineto{\pgfqpoint{0.688880in}{0.265818in}}%
\pgfpathlineto{\pgfqpoint{0.700927in}{0.260309in}}%
\pgfpathlineto{\pgfqpoint{0.712973in}{0.255007in}}%
\pgfpathlineto{\pgfqpoint{0.715512in}{0.253928in}}%
\pgfpathlineto{\pgfqpoint{0.725019in}{0.249918in}}%
\pgfpathlineto{\pgfqpoint{0.737065in}{0.245032in}}%
\pgfpathlineto{\pgfqpoint{0.745145in}{0.241882in}}%
\pgfpathlineto{\pgfqpoint{0.749111in}{0.240346in}}%
\pgfpathlineto{\pgfqpoint{0.761157in}{0.235861in}}%
\pgfpathlineto{\pgfqpoint{0.773203in}{0.231569in}}%
\pgfpathlineto{\pgfqpoint{0.778279in}{0.229836in}}%
\pgfpathlineto{\pgfqpoint{0.785249in}{0.227470in}}%
\pgfpathlineto{\pgfqpoint{0.797295in}{0.223560in}}%
\pgfpathlineto{\pgfqpoint{0.809342in}{0.219836in}}%
\pgfpathlineto{\pgfqpoint{0.816291in}{0.217790in}}%
\pgfpathlineto{\pgfqpoint{0.821388in}{0.216296in}}%
\pgfpathlineto{\pgfqpoint{0.833434in}{0.212940in}}%
\pgfpathlineto{\pgfqpoint{0.845480in}{0.209762in}}%
\pgfpathlineto{\pgfqpoint{0.857526in}{0.206762in}}%
\pgfpathlineto{\pgfqpoint{0.861862in}{0.205743in}}%
\pgfpathlineto{\pgfqpoint{0.869572in}{0.203939in}}%
\pgfpathlineto{\pgfqpoint{0.881618in}{0.201290in}}%
\pgfpathlineto{\pgfqpoint{0.893664in}{0.198814in}}%
\pgfpathlineto{\pgfqpoint{0.905710in}{0.196509in}}%
\pgfpathlineto{\pgfqpoint{0.917757in}{0.194374in}}%
\pgfpathlineto{\pgfqpoint{0.921894in}{0.193697in}}%
\pgfpathclose%
\pgfpathmoveto{\pgfqpoint{0.921894in}{0.193697in}}%
\pgfpathlineto{\pgfqpoint{0.917757in}{0.194374in}}%
\pgfpathlineto{\pgfqpoint{0.905710in}{0.196509in}}%
\pgfpathlineto{\pgfqpoint{0.893664in}{0.198814in}}%
\pgfpathlineto{\pgfqpoint{0.881618in}{0.201290in}}%
\pgfpathlineto{\pgfqpoint{0.869572in}{0.203939in}}%
\pgfpathlineto{\pgfqpoint{0.861862in}{0.205743in}}%
\pgfpathlineto{\pgfqpoint{0.857526in}{0.206762in}}%
\pgfpathlineto{\pgfqpoint{0.845480in}{0.209762in}}%
\pgfpathlineto{\pgfqpoint{0.833434in}{0.212940in}}%
\pgfpathlineto{\pgfqpoint{0.821388in}{0.216296in}}%
\pgfpathlineto{\pgfqpoint{0.816291in}{0.217790in}}%
\pgfpathlineto{\pgfqpoint{0.809342in}{0.219836in}}%
\pgfpathlineto{\pgfqpoint{0.797295in}{0.223560in}}%
\pgfpathlineto{\pgfqpoint{0.785249in}{0.227470in}}%
\pgfpathlineto{\pgfqpoint{0.778279in}{0.229836in}}%
\pgfpathlineto{\pgfqpoint{0.773203in}{0.231569in}}%
\pgfpathlineto{\pgfqpoint{0.761157in}{0.235861in}}%
\pgfpathlineto{\pgfqpoint{0.749111in}{0.240346in}}%
\pgfpathlineto{\pgfqpoint{0.745145in}{0.241882in}}%
\pgfpathlineto{\pgfqpoint{0.737065in}{0.245032in}}%
\pgfpathlineto{\pgfqpoint{0.725019in}{0.249918in}}%
\pgfpathlineto{\pgfqpoint{0.715512in}{0.253928in}}%
\pgfpathlineto{\pgfqpoint{0.712973in}{0.255007in}}%
\pgfpathlineto{\pgfqpoint{0.700927in}{0.260309in}}%
\pgfpathlineto{\pgfqpoint{0.688880in}{0.265818in}}%
\pgfpathlineto{\pgfqpoint{0.688550in}{0.265974in}}%
\pgfpathlineto{\pgfqpoint{0.676834in}{0.271552in}}%
\pgfpathlineto{\pgfqpoint{0.664788in}{0.277500in}}%
\pgfpathlineto{\pgfqpoint{0.663767in}{0.278020in}}%
\pgfpathlineto{\pgfqpoint{0.652742in}{0.283683in}}%
\pgfpathlineto{\pgfqpoint{0.640739in}{0.290066in}}%
\pgfpathlineto{\pgfqpoint{0.640696in}{0.290089in}}%
\pgfpathlineto{\pgfqpoint{0.628650in}{0.296744in}}%
\pgfpathlineto{\pgfqpoint{0.619242in}{0.302112in}}%
\pgfpathlineto{\pgfqpoint{0.616604in}{0.303634in}}%
\pgfpathlineto{\pgfqpoint{0.604558in}{0.310780in}}%
\pgfpathlineto{\pgfqpoint{0.599026in}{0.314159in}}%
\pgfpathlineto{\pgfqpoint{0.592511in}{0.318183in}}%
\pgfpathlineto{\pgfqpoint{0.580465in}{0.325844in}}%
\pgfpathlineto{\pgfqpoint{0.579913in}{0.326205in}}%
\pgfpathlineto{\pgfqpoint{0.568419in}{0.333791in}}%
\pgfpathlineto{\pgfqpoint{0.561847in}{0.338251in}}%
\pgfpathlineto{\pgfqpoint{0.556373in}{0.342013in}}%
\pgfpathlineto{\pgfqpoint{0.544641in}{0.350297in}}%
\pgfpathlineto{\pgfqpoint{0.544327in}{0.350522in}}%
\pgfpathlineto{\pgfqpoint{0.532281in}{0.359344in}}%
\pgfpathlineto{\pgfqpoint{0.528288in}{0.362343in}}%
\pgfpathlineto{\pgfqpoint{0.520235in}{0.368478in}}%
\pgfpathlineto{\pgfqpoint{0.512662in}{0.374389in}}%
\pgfpathlineto{\pgfqpoint{0.508189in}{0.377933in}}%
\pgfpathlineto{\pgfqpoint{0.497707in}{0.386435in}}%
\pgfpathlineto{\pgfqpoint{0.496143in}{0.387724in}}%
\pgfpathlineto{\pgfqpoint{0.484096in}{0.397871in}}%
\pgfpathlineto{\pgfqpoint{0.483387in}{0.398481in}}%
\pgfpathlineto{\pgfqpoint{0.472050in}{0.408394in}}%
\pgfpathlineto{\pgfqpoint{0.469663in}{0.410527in}}%
\pgfpathlineto{\pgfqpoint{0.460004in}{0.419304in}}%
\pgfpathlineto{\pgfqpoint{0.456479in}{0.422574in}}%
\pgfpathlineto{\pgfqpoint{0.447958in}{0.430619in}}%
\pgfpathlineto{\pgfqpoint{0.443805in}{0.434620in}}%
\pgfpathlineto{\pgfqpoint{0.435912in}{0.442363in}}%
\pgfpathlineto{\pgfqpoint{0.431609in}{0.446666in}}%
\pgfpathlineto{\pgfqpoint{0.423866in}{0.454558in}}%
\pgfpathlineto{\pgfqpoint{0.419865in}{0.458712in}}%
\pgfpathlineto{\pgfqpoint{0.411820in}{0.467233in}}%
\pgfpathlineto{\pgfqpoint{0.408550in}{0.470758in}}%
\pgfpathlineto{\pgfqpoint{0.399774in}{0.480416in}}%
\pgfpathlineto{\pgfqpoint{0.397641in}{0.482804in}}%
\pgfpathlineto{\pgfqpoint{0.387728in}{0.494141in}}%
\pgfpathlineto{\pgfqpoint{0.387117in}{0.494850in}}%
\pgfpathlineto{\pgfqpoint{0.376970in}{0.506896in}}%
\pgfpathlineto{\pgfqpoint{0.375681in}{0.508460in}}%
\pgfpathlineto{\pgfqpoint{0.367179in}{0.518942in}}%
\pgfpathlineto{\pgfqpoint{0.363635in}{0.523415in}}%
\pgfpathlineto{\pgfqpoint{0.357724in}{0.530989in}}%
\pgfpathlineto{\pgfqpoint{0.351589in}{0.539042in}}%
\pgfpathlineto{\pgfqpoint{0.348591in}{0.543035in}}%
\pgfpathlineto{\pgfqpoint{0.339768in}{0.555081in}}%
\pgfpathlineto{\pgfqpoint{0.339543in}{0.555395in}}%
\pgfpathlineto{\pgfqpoint{0.331259in}{0.567127in}}%
\pgfpathlineto{\pgfqpoint{0.327497in}{0.572601in}}%
\pgfpathlineto{\pgfqpoint{0.323037in}{0.579173in}}%
\pgfpathlineto{\pgfqpoint{0.315451in}{0.590666in}}%
\pgfpathlineto{\pgfqpoint{0.315090in}{0.591219in}}%
\pgfpathlineto{\pgfqpoint{0.307429in}{0.603265in}}%
\pgfpathlineto{\pgfqpoint{0.303405in}{0.609780in}}%
\pgfpathlineto{\pgfqpoint{0.300026in}{0.615311in}}%
\pgfpathlineto{\pgfqpoint{0.292880in}{0.627358in}}%
\pgfpathlineto{\pgfqpoint{0.291359in}{0.629996in}}%
\pgfpathlineto{\pgfqpoint{0.285990in}{0.639404in}}%
\pgfpathlineto{\pgfqpoint{0.279335in}{0.651450in}}%
\pgfpathlineto{\pgfqpoint{0.279312in}{0.651493in}}%
\pgfpathlineto{\pgfqpoint{0.272929in}{0.663496in}}%
\pgfpathlineto{\pgfqpoint{0.267266in}{0.674521in}}%
\pgfpathlineto{\pgfqpoint{0.266746in}{0.675542in}}%
\pgfpathlineto{\pgfqpoint{0.260798in}{0.687588in}}%
\pgfpathlineto{\pgfqpoint{0.255220in}{0.699304in}}%
\pgfpathlineto{\pgfqpoint{0.255064in}{0.699634in}}%
\pgfpathlineto{\pgfqpoint{0.249555in}{0.711680in}}%
\pgfpathlineto{\pgfqpoint{0.244253in}{0.723726in}}%
\pgfpathlineto{\pgfqpoint{0.243174in}{0.726265in}}%
\pgfpathlineto{\pgfqpoint{0.239164in}{0.735773in}}%
\pgfpathlineto{\pgfqpoint{0.234278in}{0.747819in}}%
\pgfpathlineto{\pgfqpoint{0.231128in}{0.755899in}}%
\pgfpathlineto{\pgfqpoint{0.229592in}{0.759865in}}%
\pgfpathlineto{\pgfqpoint{0.225107in}{0.771911in}}%
\pgfpathlineto{\pgfqpoint{0.220815in}{0.783957in}}%
\pgfpathlineto{\pgfqpoint{0.219082in}{0.789033in}}%
\pgfpathlineto{\pgfqpoint{0.216716in}{0.796003in}}%
\pgfpathlineto{\pgfqpoint{0.212806in}{0.808049in}}%
\pgfpathlineto{\pgfqpoint{0.209082in}{0.820095in}}%
\pgfpathlineto{\pgfqpoint{0.207036in}{0.827045in}}%
\pgfpathlineto{\pgfqpoint{0.205543in}{0.832141in}}%
\pgfpathlineto{\pgfqpoint{0.202186in}{0.844188in}}%
\pgfpathlineto{\pgfqpoint{0.199008in}{0.856234in}}%
\pgfpathlineto{\pgfqpoint{0.196008in}{0.868280in}}%
\pgfpathlineto{\pgfqpoint{0.194990in}{0.872616in}}%
\pgfpathlineto{\pgfqpoint{0.193185in}{0.880326in}}%
\pgfpathlineto{\pgfqpoint{0.190536in}{0.892372in}}%
\pgfpathlineto{\pgfqpoint{0.188060in}{0.904418in}}%
\pgfpathlineto{\pgfqpoint{0.185755in}{0.916464in}}%
\pgfpathlineto{\pgfqpoint{0.183620in}{0.928510in}}%
\pgfpathlineto{\pgfqpoint{0.182944in}{0.932648in}}%
\pgfpathlineto{\pgfqpoint{0.181654in}{0.940557in}}%
\pgfpathlineto{\pgfqpoint{0.179856in}{0.952603in}}%
\pgfpathlineto{\pgfqpoint{0.178224in}{0.964649in}}%
\pgfpathlineto{\pgfqpoint{0.176758in}{0.976695in}}%
\pgfpathlineto{\pgfqpoint{0.175457in}{0.988741in}}%
\pgfpathlineto{\pgfqpoint{0.174320in}{1.000787in}}%
\pgfpathlineto{\pgfqpoint{0.173346in}{1.012833in}}%
\pgfpathlineto{\pgfqpoint{0.172536in}{1.024879in}}%
\pgfpathlineto{\pgfqpoint{0.171888in}{1.036925in}}%
\pgfpathlineto{\pgfqpoint{0.171403in}{1.048972in}}%
\pgfpathlineto{\pgfqpoint{0.171079in}{1.061018in}}%
\pgfpathlineto{\pgfqpoint{0.170918in}{1.073064in}}%
\pgfpathlineto{\pgfqpoint{0.170918in}{1.085110in}}%
\pgfpathlineto{\pgfqpoint{0.171079in}{1.097156in}}%
\pgfpathlineto{\pgfqpoint{0.171403in}{1.109202in}}%
\pgfpathlineto{\pgfqpoint{0.171888in}{1.121248in}}%
\pgfpathlineto{\pgfqpoint{0.172536in}{1.133294in}}%
\pgfpathlineto{\pgfqpoint{0.173346in}{1.145340in}}%
\pgfpathlineto{\pgfqpoint{0.174320in}{1.157387in}}%
\pgfpathlineto{\pgfqpoint{0.175457in}{1.169433in}}%
\pgfpathlineto{\pgfqpoint{0.176758in}{1.181479in}}%
\pgfpathlineto{\pgfqpoint{0.178224in}{1.193525in}}%
\pgfpathlineto{\pgfqpoint{0.179856in}{1.205571in}}%
\pgfpathlineto{\pgfqpoint{0.181654in}{1.217617in}}%
\pgfpathlineto{\pgfqpoint{0.182944in}{1.225525in}}%
\pgfpathlineto{\pgfqpoint{0.183620in}{1.229663in}}%
\pgfpathlineto{\pgfqpoint{0.185755in}{1.241709in}}%
\pgfpathlineto{\pgfqpoint{0.188060in}{1.253756in}}%
\pgfpathlineto{\pgfqpoint{0.190536in}{1.265802in}}%
\pgfpathlineto{\pgfqpoint{0.193185in}{1.277848in}}%
\pgfpathlineto{\pgfqpoint{0.194990in}{1.285558in}}%
\pgfpathlineto{\pgfqpoint{0.196008in}{1.289894in}}%
\pgfpathlineto{\pgfqpoint{0.199008in}{1.301940in}}%
\pgfpathlineto{\pgfqpoint{0.202186in}{1.313986in}}%
\pgfpathlineto{\pgfqpoint{0.205543in}{1.326032in}}%
\pgfpathlineto{\pgfqpoint{0.207036in}{1.331129in}}%
\pgfpathlineto{\pgfqpoint{0.209082in}{1.338078in}}%
\pgfpathlineto{\pgfqpoint{0.212806in}{1.350124in}}%
\pgfpathlineto{\pgfqpoint{0.216716in}{1.362171in}}%
\pgfpathlineto{\pgfqpoint{0.219082in}{1.369141in}}%
\pgfpathlineto{\pgfqpoint{0.220815in}{1.374217in}}%
\pgfpathlineto{\pgfqpoint{0.225107in}{1.386263in}}%
\pgfpathlineto{\pgfqpoint{0.229592in}{1.398309in}}%
\pgfpathlineto{\pgfqpoint{0.231128in}{1.402275in}}%
\pgfpathlineto{\pgfqpoint{0.234278in}{1.410355in}}%
\pgfpathlineto{\pgfqpoint{0.239164in}{1.422401in}}%
\pgfpathlineto{\pgfqpoint{0.243174in}{1.431908in}}%
\pgfpathlineto{\pgfqpoint{0.244253in}{1.434447in}}%
\pgfpathlineto{\pgfqpoint{0.249555in}{1.446493in}}%
\pgfpathlineto{\pgfqpoint{0.255064in}{1.458539in}}%
\pgfpathlineto{\pgfqpoint{0.255220in}{1.458870in}}%
\pgfpathlineto{\pgfqpoint{0.260798in}{1.470586in}}%
\pgfpathlineto{\pgfqpoint{0.266746in}{1.482632in}}%
\pgfpathlineto{\pgfqpoint{0.267266in}{1.483653in}}%
\pgfpathlineto{\pgfqpoint{0.272929in}{1.494678in}}%
\pgfpathlineto{\pgfqpoint{0.279312in}{1.506681in}}%
\pgfpathlineto{\pgfqpoint{0.279335in}{1.506724in}}%
\pgfpathlineto{\pgfqpoint{0.285990in}{1.518770in}}%
\pgfpathlineto{\pgfqpoint{0.291359in}{1.528178in}}%
\pgfpathlineto{\pgfqpoint{0.292880in}{1.530816in}}%
\pgfpathlineto{\pgfqpoint{0.300026in}{1.542862in}}%
\pgfpathlineto{\pgfqpoint{0.303405in}{1.548394in}}%
\pgfpathlineto{\pgfqpoint{0.307429in}{1.554908in}}%
\pgfpathlineto{\pgfqpoint{0.315090in}{1.566955in}}%
\pgfpathlineto{\pgfqpoint{0.315451in}{1.567507in}}%
\pgfpathlineto{\pgfqpoint{0.323037in}{1.579001in}}%
\pgfpathlineto{\pgfqpoint{0.327497in}{1.585573in}}%
\pgfpathlineto{\pgfqpoint{0.331259in}{1.591047in}}%
\pgfpathlineto{\pgfqpoint{0.339543in}{1.602779in}}%
\pgfpathlineto{\pgfqpoint{0.339768in}{1.603093in}}%
\pgfpathlineto{\pgfqpoint{0.348591in}{1.615139in}}%
\pgfpathlineto{\pgfqpoint{0.351589in}{1.619132in}}%
\pgfpathlineto{\pgfqpoint{0.357724in}{1.627185in}}%
\pgfpathlineto{\pgfqpoint{0.363635in}{1.634758in}}%
\pgfpathlineto{\pgfqpoint{0.367179in}{1.639231in}}%
\pgfpathlineto{\pgfqpoint{0.375681in}{1.649713in}}%
\pgfpathlineto{\pgfqpoint{0.376970in}{1.651277in}}%
\pgfpathlineto{\pgfqpoint{0.387117in}{1.663323in}}%
\pgfpathlineto{\pgfqpoint{0.387728in}{1.664033in}}%
\pgfpathlineto{\pgfqpoint{0.397641in}{1.675370in}}%
\pgfpathlineto{\pgfqpoint{0.399774in}{1.677757in}}%
\pgfpathlineto{\pgfqpoint{0.408550in}{1.687416in}}%
\pgfpathlineto{\pgfqpoint{0.411820in}{1.690941in}}%
\pgfpathlineto{\pgfqpoint{0.419865in}{1.699462in}}%
\pgfpathlineto{\pgfqpoint{0.423866in}{1.703615in}}%
\pgfpathlineto{\pgfqpoint{0.431609in}{1.711508in}}%
\pgfpathlineto{\pgfqpoint{0.435912in}{1.715811in}}%
\pgfpathlineto{\pgfqpoint{0.443805in}{1.723554in}}%
\pgfpathlineto{\pgfqpoint{0.447958in}{1.727554in}}%
\pgfpathlineto{\pgfqpoint{0.456479in}{1.735600in}}%
\pgfpathlineto{\pgfqpoint{0.460004in}{1.738870in}}%
\pgfpathlineto{\pgfqpoint{0.469663in}{1.747646in}}%
\pgfpathlineto{\pgfqpoint{0.472050in}{1.749779in}}%
\pgfpathlineto{\pgfqpoint{0.483387in}{1.759692in}}%
\pgfpathlineto{\pgfqpoint{0.484096in}{1.760303in}}%
\pgfpathlineto{\pgfqpoint{0.496143in}{1.770450in}}%
\pgfpathlineto{\pgfqpoint{0.497707in}{1.771738in}}%
\pgfpathlineto{\pgfqpoint{0.508189in}{1.780241in}}%
\pgfpathlineto{\pgfqpoint{0.512662in}{1.783785in}}%
\pgfpathlineto{\pgfqpoint{0.520235in}{1.789696in}}%
\pgfpathlineto{\pgfqpoint{0.528288in}{1.795831in}}%
\pgfpathlineto{\pgfqpoint{0.532281in}{1.798829in}}%
\pgfpathlineto{\pgfqpoint{0.544327in}{1.807652in}}%
\pgfpathlineto{\pgfqpoint{0.544641in}{1.807877in}}%
\pgfpathlineto{\pgfqpoint{0.556373in}{1.816160in}}%
\pgfpathlineto{\pgfqpoint{0.561847in}{1.819923in}}%
\pgfpathlineto{\pgfqpoint{0.568419in}{1.824383in}}%
\pgfpathlineto{\pgfqpoint{0.579913in}{1.831969in}}%
\pgfpathlineto{\pgfqpoint{0.580465in}{1.832330in}}%
\pgfpathlineto{\pgfqpoint{0.592511in}{1.839991in}}%
\pgfpathlineto{\pgfqpoint{0.599026in}{1.844015in}}%
\pgfpathlineto{\pgfqpoint{0.604558in}{1.847394in}}%
\pgfpathlineto{\pgfqpoint{0.616604in}{1.854540in}}%
\pgfpathlineto{\pgfqpoint{0.619242in}{1.856061in}}%
\pgfpathlineto{\pgfqpoint{0.628650in}{1.861430in}}%
\pgfpathlineto{\pgfqpoint{0.640696in}{1.868084in}}%
\pgfpathlineto{\pgfqpoint{0.640739in}{1.868107in}}%
\pgfpathlineto{\pgfqpoint{0.652742in}{1.874491in}}%
\pgfpathlineto{\pgfqpoint{0.663767in}{1.880154in}}%
\pgfpathlineto{\pgfqpoint{0.664788in}{1.880673in}}%
\pgfpathlineto{\pgfqpoint{0.676834in}{1.886622in}}%
\pgfpathlineto{\pgfqpoint{0.688550in}{1.892200in}}%
\pgfpathlineto{\pgfqpoint{0.688880in}{1.892356in}}%
\pgfpathlineto{\pgfqpoint{0.700927in}{1.897865in}}%
\pgfpathlineto{\pgfqpoint{0.712973in}{1.903167in}}%
\pgfpathlineto{\pgfqpoint{0.715512in}{1.904246in}}%
\pgfpathlineto{\pgfqpoint{0.725019in}{1.908256in}}%
\pgfpathlineto{\pgfqpoint{0.737065in}{1.913142in}}%
\pgfpathlineto{\pgfqpoint{0.745145in}{1.916292in}}%
\pgfpathlineto{\pgfqpoint{0.749111in}{1.917828in}}%
\pgfpathlineto{\pgfqpoint{0.761157in}{1.922312in}}%
\pgfpathlineto{\pgfqpoint{0.773203in}{1.926605in}}%
\pgfpathlineto{\pgfqpoint{0.778279in}{1.928338in}}%
\pgfpathlineto{\pgfqpoint{0.785249in}{1.930704in}}%
\pgfpathlineto{\pgfqpoint{0.797295in}{1.934614in}}%
\pgfpathlineto{\pgfqpoint{0.809342in}{1.938338in}}%
\pgfpathlineto{\pgfqpoint{0.816291in}{1.940384in}}%
\pgfpathlineto{\pgfqpoint{0.821388in}{1.941877in}}%
\pgfpathlineto{\pgfqpoint{0.833434in}{1.945234in}}%
\pgfpathlineto{\pgfqpoint{0.845480in}{1.948411in}}%
\pgfpathlineto{\pgfqpoint{0.857526in}{1.951412in}}%
\pgfpathlineto{\pgfqpoint{0.861862in}{1.952430in}}%
\pgfpathlineto{\pgfqpoint{0.869572in}{1.954235in}}%
\pgfpathlineto{\pgfqpoint{0.881618in}{1.956883in}}%
\pgfpathlineto{\pgfqpoint{0.893664in}{1.959360in}}%
\pgfpathlineto{\pgfqpoint{0.905710in}{1.961665in}}%
\pgfpathlineto{\pgfqpoint{0.917757in}{1.963800in}}%
\pgfpathlineto{\pgfqpoint{0.921894in}{1.964476in}}%
\pgfpathlineto{\pgfqpoint{0.929803in}{1.965766in}}%
\pgfpathlineto{\pgfqpoint{0.941849in}{1.967564in}}%
\pgfpathlineto{\pgfqpoint{0.953895in}{1.969196in}}%
\pgfpathlineto{\pgfqpoint{0.965941in}{1.970662in}}%
\pgfpathlineto{\pgfqpoint{0.977987in}{1.971963in}}%
\pgfpathlineto{\pgfqpoint{0.990033in}{1.973100in}}%
\pgfpathlineto{\pgfqpoint{1.002079in}{1.974073in}}%
\pgfpathlineto{\pgfqpoint{1.014126in}{1.974884in}}%
\pgfpathlineto{\pgfqpoint{1.026172in}{1.975532in}}%
\pgfpathlineto{\pgfqpoint{1.038218in}{1.976017in}}%
\pgfpathlineto{\pgfqpoint{1.050264in}{1.976341in}}%
\pgfpathlineto{\pgfqpoint{1.062310in}{1.976502in}}%
\pgfpathlineto{\pgfqpoint{1.074356in}{1.976502in}}%
\pgfpathlineto{\pgfqpoint{1.086402in}{1.976341in}}%
\pgfpathlineto{\pgfqpoint{1.098448in}{1.976017in}}%
\pgfpathlineto{\pgfqpoint{1.110494in}{1.975532in}}%
\pgfpathlineto{\pgfqpoint{1.122541in}{1.974884in}}%
\pgfpathlineto{\pgfqpoint{1.134587in}{1.974073in}}%
\pgfpathlineto{\pgfqpoint{1.146633in}{1.973100in}}%
\pgfpathlineto{\pgfqpoint{1.158679in}{1.971963in}}%
\pgfpathlineto{\pgfqpoint{1.170725in}{1.970662in}}%
\pgfpathlineto{\pgfqpoint{1.182771in}{1.969196in}}%
\pgfpathlineto{\pgfqpoint{1.194817in}{1.967564in}}%
\pgfpathlineto{\pgfqpoint{1.206863in}{1.965766in}}%
\pgfpathlineto{\pgfqpoint{1.214772in}{1.964476in}}%
\pgfpathlineto{\pgfqpoint{1.218909in}{1.963800in}}%
\pgfpathlineto{\pgfqpoint{1.230956in}{1.961665in}}%
\pgfpathlineto{\pgfqpoint{1.243002in}{1.959360in}}%
\pgfpathlineto{\pgfqpoint{1.255048in}{1.956883in}}%
\pgfpathlineto{\pgfqpoint{1.267094in}{1.954235in}}%
\pgfpathlineto{\pgfqpoint{1.274804in}{1.952430in}}%
\pgfpathlineto{\pgfqpoint{1.279140in}{1.951412in}}%
\pgfpathlineto{\pgfqpoint{1.291186in}{1.948411in}}%
\pgfpathlineto{\pgfqpoint{1.303232in}{1.945234in}}%
\pgfpathlineto{\pgfqpoint{1.315278in}{1.941877in}}%
\pgfpathlineto{\pgfqpoint{1.320375in}{1.940384in}}%
\pgfpathlineto{\pgfqpoint{1.327325in}{1.938338in}}%
\pgfpathlineto{\pgfqpoint{1.339371in}{1.934614in}}%
\pgfpathlineto{\pgfqpoint{1.351417in}{1.930704in}}%
\pgfpathlineto{\pgfqpoint{1.358387in}{1.928338in}}%
\pgfpathlineto{\pgfqpoint{1.363463in}{1.926605in}}%
\pgfpathlineto{\pgfqpoint{1.375509in}{1.922312in}}%
\pgfpathlineto{\pgfqpoint{1.387555in}{1.917828in}}%
\pgfpathlineto{\pgfqpoint{1.391521in}{1.916292in}}%
\pgfpathlineto{\pgfqpoint{1.399601in}{1.913142in}}%
\pgfpathlineto{\pgfqpoint{1.411647in}{1.908256in}}%
\pgfpathlineto{\pgfqpoint{1.421154in}{1.904246in}}%
\pgfpathlineto{\pgfqpoint{1.423693in}{1.903167in}}%
\pgfpathlineto{\pgfqpoint{1.435740in}{1.897865in}}%
\pgfpathlineto{\pgfqpoint{1.447786in}{1.892356in}}%
\pgfpathlineto{\pgfqpoint{1.448116in}{1.892200in}}%
\pgfpathlineto{\pgfqpoint{1.459832in}{1.886622in}}%
\pgfpathlineto{\pgfqpoint{1.471878in}{1.880673in}}%
\pgfpathlineto{\pgfqpoint{1.472899in}{1.880154in}}%
\pgfpathlineto{\pgfqpoint{1.483924in}{1.874491in}}%
\pgfpathlineto{\pgfqpoint{1.495927in}{1.868107in}}%
\pgfpathlineto{\pgfqpoint{1.495970in}{1.868084in}}%
\pgfpathlineto{\pgfqpoint{1.508016in}{1.861430in}}%
\pgfpathlineto{\pgfqpoint{1.517424in}{1.856061in}}%
\pgfpathlineto{\pgfqpoint{1.520062in}{1.854540in}}%
\pgfpathlineto{\pgfqpoint{1.532108in}{1.847394in}}%
\pgfpathlineto{\pgfqpoint{1.537640in}{1.844015in}}%
\pgfpathlineto{\pgfqpoint{1.544155in}{1.839991in}}%
\pgfpathlineto{\pgfqpoint{1.556201in}{1.832330in}}%
\pgfpathlineto{\pgfqpoint{1.556754in}{1.831969in}}%
\pgfpathlineto{\pgfqpoint{1.568247in}{1.824383in}}%
\pgfpathlineto{\pgfqpoint{1.574819in}{1.819923in}}%
\pgfpathlineto{\pgfqpoint{1.580293in}{1.816160in}}%
\pgfpathlineto{\pgfqpoint{1.592025in}{1.807877in}}%
\pgfpathlineto{\pgfqpoint{1.592339in}{1.807652in}}%
\pgfpathlineto{\pgfqpoint{1.604385in}{1.798829in}}%
\pgfpathlineto{\pgfqpoint{1.608378in}{1.795831in}}%
\pgfpathlineto{\pgfqpoint{1.616431in}{1.789696in}}%
\pgfpathlineto{\pgfqpoint{1.624005in}{1.783785in}}%
\pgfpathlineto{\pgfqpoint{1.628477in}{1.780241in}}%
\pgfpathlineto{\pgfqpoint{1.638959in}{1.771738in}}%
\pgfpathlineto{\pgfqpoint{1.640524in}{1.770450in}}%
\pgfpathlineto{\pgfqpoint{1.652570in}{1.760303in}}%
\pgfpathlineto{\pgfqpoint{1.653279in}{1.759692in}}%
\pgfpathlineto{\pgfqpoint{1.664616in}{1.749779in}}%
\pgfpathlineto{\pgfqpoint{1.667003in}{1.747646in}}%
\pgfpathlineto{\pgfqpoint{1.676662in}{1.738870in}}%
\pgfpathlineto{\pgfqpoint{1.680187in}{1.735600in}}%
\pgfpathlineto{\pgfqpoint{1.688708in}{1.727554in}}%
\pgfpathlineto{\pgfqpoint{1.692862in}{1.723554in}}%
\pgfpathlineto{\pgfqpoint{1.700754in}{1.715811in}}%
\pgfpathlineto{\pgfqpoint{1.705057in}{1.711508in}}%
\pgfpathlineto{\pgfqpoint{1.712800in}{1.703615in}}%
\pgfpathlineto{\pgfqpoint{1.716801in}{1.699462in}}%
\pgfpathlineto{\pgfqpoint{1.724846in}{1.690941in}}%
\pgfpathlineto{\pgfqpoint{1.728116in}{1.687416in}}%
\pgfpathlineto{\pgfqpoint{1.736892in}{1.677757in}}%
\pgfpathlineto{\pgfqpoint{1.739025in}{1.675370in}}%
\pgfpathlineto{\pgfqpoint{1.748939in}{1.664033in}}%
\pgfpathlineto{\pgfqpoint{1.749549in}{1.663323in}}%
\pgfpathlineto{\pgfqpoint{1.759696in}{1.651277in}}%
\pgfpathlineto{\pgfqpoint{1.760985in}{1.649713in}}%
\pgfpathlineto{\pgfqpoint{1.769487in}{1.639231in}}%
\pgfpathlineto{\pgfqpoint{1.773031in}{1.634758in}}%
\pgfpathlineto{\pgfqpoint{1.778942in}{1.627185in}}%
\pgfpathlineto{\pgfqpoint{1.785077in}{1.619132in}}%
\pgfpathlineto{\pgfqpoint{1.788075in}{1.615139in}}%
\pgfpathlineto{\pgfqpoint{1.796898in}{1.603093in}}%
\pgfpathlineto{\pgfqpoint{1.797123in}{1.602779in}}%
\pgfpathlineto{\pgfqpoint{1.805407in}{1.591047in}}%
\pgfpathlineto{\pgfqpoint{1.809169in}{1.585573in}}%
\pgfpathlineto{\pgfqpoint{1.813629in}{1.579001in}}%
\pgfpathlineto{\pgfqpoint{1.821215in}{1.567507in}}%
\pgfpathlineto{\pgfqpoint{1.821576in}{1.566955in}}%
\pgfpathlineto{\pgfqpoint{1.829237in}{1.554908in}}%
\pgfpathlineto{\pgfqpoint{1.833261in}{1.548394in}}%
\pgfpathlineto{\pgfqpoint{1.836640in}{1.542862in}}%
\pgfpathlineto{\pgfqpoint{1.843786in}{1.530816in}}%
\pgfpathlineto{\pgfqpoint{1.845307in}{1.528178in}}%
\pgfpathlineto{\pgfqpoint{1.850676in}{1.518770in}}%
\pgfpathlineto{\pgfqpoint{1.857331in}{1.506724in}}%
\pgfpathlineto{\pgfqpoint{1.857354in}{1.506681in}}%
\pgfpathlineto{\pgfqpoint{1.863737in}{1.494678in}}%
\pgfpathlineto{\pgfqpoint{1.869400in}{1.483653in}}%
\pgfpathlineto{\pgfqpoint{1.869920in}{1.482632in}}%
\pgfpathlineto{\pgfqpoint{1.875868in}{1.470586in}}%
\pgfpathlineto{\pgfqpoint{1.881446in}{1.458870in}}%
\pgfpathlineto{\pgfqpoint{1.881602in}{1.458539in}}%
\pgfpathlineto{\pgfqpoint{1.887111in}{1.446493in}}%
\pgfpathlineto{\pgfqpoint{1.892413in}{1.434447in}}%
\pgfpathlineto{\pgfqpoint{1.893492in}{1.431908in}}%
\pgfpathlineto{\pgfqpoint{1.897502in}{1.422401in}}%
\pgfpathlineto{\pgfqpoint{1.902388in}{1.410355in}}%
\pgfpathlineto{\pgfqpoint{1.905538in}{1.402275in}}%
\pgfpathlineto{\pgfqpoint{1.907074in}{1.398309in}}%
\pgfpathlineto{\pgfqpoint{1.911559in}{1.386263in}}%
\pgfpathlineto{\pgfqpoint{1.915851in}{1.374217in}}%
\pgfpathlineto{\pgfqpoint{1.917584in}{1.369141in}}%
\pgfpathlineto{\pgfqpoint{1.919950in}{1.362171in}}%
\pgfpathlineto{\pgfqpoint{1.923860in}{1.350124in}}%
\pgfpathlineto{\pgfqpoint{1.927584in}{1.338078in}}%
\pgfpathlineto{\pgfqpoint{1.929630in}{1.331129in}}%
\pgfpathlineto{\pgfqpoint{1.931124in}{1.326032in}}%
\pgfpathlineto{\pgfqpoint{1.934480in}{1.313986in}}%
\pgfpathlineto{\pgfqpoint{1.937658in}{1.301940in}}%
\pgfpathlineto{\pgfqpoint{1.940658in}{1.289894in}}%
\pgfpathlineto{\pgfqpoint{1.941676in}{1.285558in}}%
\pgfpathlineto{\pgfqpoint{1.943481in}{1.277848in}}%
\pgfpathlineto{\pgfqpoint{1.946130in}{1.265802in}}%
\pgfpathlineto{\pgfqpoint{1.948606in}{1.253756in}}%
\pgfpathlineto{\pgfqpoint{1.950911in}{1.241709in}}%
\pgfpathlineto{\pgfqpoint{1.953046in}{1.229663in}}%
\pgfpathlineto{\pgfqpoint{1.953723in}{1.225525in}}%
\pgfpathlineto{\pgfqpoint{1.955012in}{1.217617in}}%
\pgfpathlineto{\pgfqpoint{1.956810in}{1.205571in}}%
\pgfpathlineto{\pgfqpoint{1.958442in}{1.193525in}}%
\pgfpathlineto{\pgfqpoint{1.959908in}{1.181479in}}%
\pgfpathlineto{\pgfqpoint{1.961209in}{1.169433in}}%
\pgfpathlineto{\pgfqpoint{1.962346in}{1.157387in}}%
\pgfpathlineto{\pgfqpoint{1.963320in}{1.145340in}}%
\pgfpathlineto{\pgfqpoint{1.964130in}{1.133294in}}%
\pgfpathlineto{\pgfqpoint{1.964778in}{1.121248in}}%
\pgfpathlineto{\pgfqpoint{1.965263in}{1.109202in}}%
\pgfpathlineto{\pgfqpoint{1.965587in}{1.097156in}}%
\pgfpathlineto{\pgfqpoint{1.965748in}{1.085110in}}%
\pgfpathlineto{\pgfqpoint{1.965748in}{1.073064in}}%
\pgfpathlineto{\pgfqpoint{1.965587in}{1.061018in}}%
\pgfpathlineto{\pgfqpoint{1.965263in}{1.048972in}}%
\pgfpathlineto{\pgfqpoint{1.964778in}{1.036925in}}%
\pgfpathlineto{\pgfqpoint{1.964130in}{1.024879in}}%
\pgfpathlineto{\pgfqpoint{1.963320in}{1.012833in}}%
\pgfpathlineto{\pgfqpoint{1.962346in}{1.000787in}}%
\pgfpathlineto{\pgfqpoint{1.961209in}{0.988741in}}%
\pgfpathlineto{\pgfqpoint{1.959908in}{0.976695in}}%
\pgfpathlineto{\pgfqpoint{1.958442in}{0.964649in}}%
\pgfpathlineto{\pgfqpoint{1.956810in}{0.952603in}}%
\pgfpathlineto{\pgfqpoint{1.955012in}{0.940557in}}%
\pgfpathlineto{\pgfqpoint{1.953723in}{0.932648in}}%
\pgfpathlineto{\pgfqpoint{1.953046in}{0.928510in}}%
\pgfpathlineto{\pgfqpoint{1.950911in}{0.916464in}}%
\pgfpathlineto{\pgfqpoint{1.948606in}{0.904418in}}%
\pgfpathlineto{\pgfqpoint{1.946130in}{0.892372in}}%
\pgfpathlineto{\pgfqpoint{1.943481in}{0.880326in}}%
\pgfpathlineto{\pgfqpoint{1.941676in}{0.872616in}}%
\pgfpathlineto{\pgfqpoint{1.940658in}{0.868280in}}%
\pgfpathlineto{\pgfqpoint{1.937658in}{0.856234in}}%
\pgfpathlineto{\pgfqpoint{1.934480in}{0.844188in}}%
\pgfpathlineto{\pgfqpoint{1.931124in}{0.832141in}}%
\pgfpathlineto{\pgfqpoint{1.929630in}{0.827045in}}%
\pgfpathlineto{\pgfqpoint{1.927584in}{0.820095in}}%
\pgfpathlineto{\pgfqpoint{1.923860in}{0.808049in}}%
\pgfpathlineto{\pgfqpoint{1.919950in}{0.796003in}}%
\pgfpathlineto{\pgfqpoint{1.917584in}{0.789033in}}%
\pgfpathlineto{\pgfqpoint{1.915851in}{0.783957in}}%
\pgfpathlineto{\pgfqpoint{1.911559in}{0.771911in}}%
\pgfpathlineto{\pgfqpoint{1.907074in}{0.759865in}}%
\pgfpathlineto{\pgfqpoint{1.905538in}{0.755899in}}%
\pgfpathlineto{\pgfqpoint{1.902388in}{0.747819in}}%
\pgfpathlineto{\pgfqpoint{1.897502in}{0.735773in}}%
\pgfpathlineto{\pgfqpoint{1.893492in}{0.726265in}}%
\pgfpathlineto{\pgfqpoint{1.892413in}{0.723726in}}%
\pgfpathlineto{\pgfqpoint{1.887111in}{0.711680in}}%
\pgfpathlineto{\pgfqpoint{1.881602in}{0.699634in}}%
\pgfpathlineto{\pgfqpoint{1.881446in}{0.699304in}}%
\pgfpathlineto{\pgfqpoint{1.875868in}{0.687588in}}%
\pgfpathlineto{\pgfqpoint{1.869920in}{0.675542in}}%
\pgfpathlineto{\pgfqpoint{1.869400in}{0.674521in}}%
\pgfpathlineto{\pgfqpoint{1.863737in}{0.663496in}}%
\pgfpathlineto{\pgfqpoint{1.857354in}{0.651493in}}%
\pgfpathlineto{\pgfqpoint{1.857331in}{0.651450in}}%
\pgfpathlineto{\pgfqpoint{1.850676in}{0.639404in}}%
\pgfpathlineto{\pgfqpoint{1.845307in}{0.629996in}}%
\pgfpathlineto{\pgfqpoint{1.843786in}{0.627358in}}%
\pgfpathlineto{\pgfqpoint{1.836640in}{0.615311in}}%
\pgfpathlineto{\pgfqpoint{1.833261in}{0.609780in}}%
\pgfpathlineto{\pgfqpoint{1.829237in}{0.603265in}}%
\pgfpathlineto{\pgfqpoint{1.821576in}{0.591219in}}%
\pgfpathlineto{\pgfqpoint{1.821215in}{0.590666in}}%
\pgfpathlineto{\pgfqpoint{1.813629in}{0.579173in}}%
\pgfpathlineto{\pgfqpoint{1.809169in}{0.572601in}}%
\pgfpathlineto{\pgfqpoint{1.805407in}{0.567127in}}%
\pgfpathlineto{\pgfqpoint{1.797123in}{0.555395in}}%
\pgfpathlineto{\pgfqpoint{1.796898in}{0.555081in}}%
\pgfpathlineto{\pgfqpoint{1.788075in}{0.543035in}}%
\pgfpathlineto{\pgfqpoint{1.785077in}{0.539042in}}%
\pgfpathlineto{\pgfqpoint{1.778942in}{0.530989in}}%
\pgfpathlineto{\pgfqpoint{1.773031in}{0.523415in}}%
\pgfpathlineto{\pgfqpoint{1.769487in}{0.518942in}}%
\pgfpathlineto{\pgfqpoint{1.760985in}{0.508460in}}%
\pgfpathlineto{\pgfqpoint{1.759696in}{0.506896in}}%
\pgfpathlineto{\pgfqpoint{1.749549in}{0.494850in}}%
\pgfpathlineto{\pgfqpoint{1.748939in}{0.494141in}}%
\pgfpathlineto{\pgfqpoint{1.739025in}{0.482804in}}%
\pgfpathlineto{\pgfqpoint{1.736892in}{0.480416in}}%
\pgfpathlineto{\pgfqpoint{1.728116in}{0.470758in}}%
\pgfpathlineto{\pgfqpoint{1.724846in}{0.467233in}}%
\pgfpathlineto{\pgfqpoint{1.716801in}{0.458712in}}%
\pgfpathlineto{\pgfqpoint{1.712800in}{0.454558in}}%
\pgfpathlineto{\pgfqpoint{1.705057in}{0.446666in}}%
\pgfpathlineto{\pgfqpoint{1.700754in}{0.442363in}}%
\pgfpathlineto{\pgfqpoint{1.692862in}{0.434620in}}%
\pgfpathlineto{\pgfqpoint{1.688708in}{0.430619in}}%
\pgfpathlineto{\pgfqpoint{1.680187in}{0.422574in}}%
\pgfpathlineto{\pgfqpoint{1.676662in}{0.419304in}}%
\pgfpathlineto{\pgfqpoint{1.667003in}{0.410527in}}%
\pgfpathlineto{\pgfqpoint{1.664616in}{0.408394in}}%
\pgfpathlineto{\pgfqpoint{1.653279in}{0.398481in}}%
\pgfpathlineto{\pgfqpoint{1.652570in}{0.397871in}}%
\pgfpathlineto{\pgfqpoint{1.640524in}{0.387724in}}%
\pgfpathlineto{\pgfqpoint{1.638959in}{0.386435in}}%
\pgfpathlineto{\pgfqpoint{1.628477in}{0.377933in}}%
\pgfpathlineto{\pgfqpoint{1.624005in}{0.374389in}}%
\pgfpathlineto{\pgfqpoint{1.616431in}{0.368478in}}%
\pgfpathlineto{\pgfqpoint{1.608378in}{0.362343in}}%
\pgfpathlineto{\pgfqpoint{1.604385in}{0.359344in}}%
\pgfpathlineto{\pgfqpoint{1.592339in}{0.350522in}}%
\pgfpathlineto{\pgfqpoint{1.592025in}{0.350297in}}%
\pgfpathlineto{\pgfqpoint{1.580293in}{0.342013in}}%
\pgfpathlineto{\pgfqpoint{1.574819in}{0.338251in}}%
\pgfpathlineto{\pgfqpoint{1.568247in}{0.333791in}}%
\pgfpathlineto{\pgfqpoint{1.556754in}{0.326205in}}%
\pgfpathlineto{\pgfqpoint{1.556201in}{0.325844in}}%
\pgfpathlineto{\pgfqpoint{1.544155in}{0.318183in}}%
\pgfpathlineto{\pgfqpoint{1.537640in}{0.314159in}}%
\pgfpathlineto{\pgfqpoint{1.532108in}{0.310780in}}%
\pgfpathlineto{\pgfqpoint{1.520062in}{0.303634in}}%
\pgfpathlineto{\pgfqpoint{1.517424in}{0.302112in}}%
\pgfpathlineto{\pgfqpoint{1.508016in}{0.296744in}}%
\pgfpathlineto{\pgfqpoint{1.495970in}{0.290089in}}%
\pgfpathlineto{\pgfqpoint{1.495927in}{0.290066in}}%
\pgfpathlineto{\pgfqpoint{1.483924in}{0.283683in}}%
\pgfpathlineto{\pgfqpoint{1.472899in}{0.278020in}}%
\pgfpathlineto{\pgfqpoint{1.471878in}{0.277500in}}%
\pgfpathlineto{\pgfqpoint{1.459832in}{0.271552in}}%
\pgfpathlineto{\pgfqpoint{1.448116in}{0.265974in}}%
\pgfpathlineto{\pgfqpoint{1.447786in}{0.265818in}}%
\pgfpathlineto{\pgfqpoint{1.435740in}{0.260309in}}%
\pgfpathlineto{\pgfqpoint{1.423693in}{0.255007in}}%
\pgfpathlineto{\pgfqpoint{1.421154in}{0.253928in}}%
\pgfpathlineto{\pgfqpoint{1.411647in}{0.249918in}}%
\pgfpathlineto{\pgfqpoint{1.399601in}{0.245032in}}%
\pgfpathlineto{\pgfqpoint{1.391521in}{0.241882in}}%
\pgfpathlineto{\pgfqpoint{1.387555in}{0.240346in}}%
\pgfpathlineto{\pgfqpoint{1.375509in}{0.235861in}}%
\pgfpathlineto{\pgfqpoint{1.363463in}{0.231569in}}%
\pgfpathlineto{\pgfqpoint{1.358387in}{0.229836in}}%
\pgfpathlineto{\pgfqpoint{1.351417in}{0.227470in}}%
\pgfpathlineto{\pgfqpoint{1.339371in}{0.223560in}}%
\pgfpathlineto{\pgfqpoint{1.327325in}{0.219836in}}%
\pgfpathlineto{\pgfqpoint{1.320375in}{0.217790in}}%
\pgfpathlineto{\pgfqpoint{1.315278in}{0.216296in}}%
\pgfpathlineto{\pgfqpoint{1.303232in}{0.212940in}}%
\pgfpathlineto{\pgfqpoint{1.291186in}{0.209762in}}%
\pgfpathlineto{\pgfqpoint{1.279140in}{0.206762in}}%
\pgfpathlineto{\pgfqpoint{1.274804in}{0.205743in}}%
\pgfpathlineto{\pgfqpoint{1.267094in}{0.203939in}}%
\pgfpathlineto{\pgfqpoint{1.255048in}{0.201290in}}%
\pgfpathlineto{\pgfqpoint{1.243002in}{0.198814in}}%
\pgfpathlineto{\pgfqpoint{1.230956in}{0.196509in}}%
\pgfpathlineto{\pgfqpoint{1.218909in}{0.194374in}}%
\pgfpathlineto{\pgfqpoint{1.214772in}{0.193697in}}%
\pgfpathlineto{\pgfqpoint{1.206863in}{0.192408in}}%
\pgfpathlineto{\pgfqpoint{1.194817in}{0.190610in}}%
\pgfpathlineto{\pgfqpoint{1.182771in}{0.188978in}}%
\pgfpathlineto{\pgfqpoint{1.170725in}{0.187512in}}%
\pgfpathlineto{\pgfqpoint{1.158679in}{0.186211in}}%
\pgfpathlineto{\pgfqpoint{1.146633in}{0.185074in}}%
\pgfpathlineto{\pgfqpoint{1.134587in}{0.184100in}}%
\pgfpathlineto{\pgfqpoint{1.122541in}{0.183290in}}%
\pgfpathlineto{\pgfqpoint{1.110494in}{0.182642in}}%
\pgfpathlineto{\pgfqpoint{1.098448in}{0.182157in}}%
\pgfpathlineto{\pgfqpoint{1.086402in}{0.181833in}}%
\pgfpathlineto{\pgfqpoint{1.074356in}{0.181671in}}%
\pgfpathlineto{\pgfqpoint{1.062310in}{0.181671in}}%
\pgfpathlineto{\pgfqpoint{1.050264in}{0.181833in}}%
\pgfpathlineto{\pgfqpoint{1.038218in}{0.182157in}}%
\pgfpathlineto{\pgfqpoint{1.026172in}{0.182642in}}%
\pgfpathlineto{\pgfqpoint{1.014126in}{0.183290in}}%
\pgfpathlineto{\pgfqpoint{1.002079in}{0.184100in}}%
\pgfpathlineto{\pgfqpoint{0.990033in}{0.185074in}}%
\pgfpathlineto{\pgfqpoint{0.977987in}{0.186211in}}%
\pgfpathlineto{\pgfqpoint{0.965941in}{0.187512in}}%
\pgfpathlineto{\pgfqpoint{0.953895in}{0.188978in}}%
\pgfpathlineto{\pgfqpoint{0.941849in}{0.190610in}}%
\pgfpathlineto{\pgfqpoint{0.929803in}{0.192408in}}%
\pgfpathclose%
\pgfusepath{fill}%
\end{pgfscope}%
\begin{pgfscope}%
\pgfpathrectangle{\pgfqpoint{0.135000in}{0.145754in}}{\pgfqpoint{1.866666in}{1.866666in}} %
\pgfusepath{clip}%
\pgfsetbuttcap%
\pgfsetroundjoin%
\definecolor{currentfill}{rgb}{1.000000,1.000000,1.000000}%
\pgfsetfillcolor{currentfill}%
\pgfsetlinewidth{0.000000pt}%
\definecolor{currentstroke}{rgb}{0.000000,0.000000,0.000000}%
\pgfsetstrokecolor{currentstroke}%
\pgfsetdash{}{0pt}%
\pgfpathmoveto{\pgfqpoint{0.929803in}{0.192408in}}%
\pgfpathlineto{\pgfqpoint{0.941849in}{0.190610in}}%
\pgfpathlineto{\pgfqpoint{0.953895in}{0.188978in}}%
\pgfpathlineto{\pgfqpoint{0.965941in}{0.187512in}}%
\pgfpathlineto{\pgfqpoint{0.977987in}{0.186211in}}%
\pgfpathlineto{\pgfqpoint{0.990033in}{0.185074in}}%
\pgfpathlineto{\pgfqpoint{1.002079in}{0.184100in}}%
\pgfpathlineto{\pgfqpoint{1.014126in}{0.183290in}}%
\pgfpathlineto{\pgfqpoint{1.026172in}{0.182642in}}%
\pgfpathlineto{\pgfqpoint{1.038218in}{0.182157in}}%
\pgfpathlineto{\pgfqpoint{1.050264in}{0.181833in}}%
\pgfpathlineto{\pgfqpoint{1.062310in}{0.181671in}}%
\pgfpathlineto{\pgfqpoint{1.074356in}{0.181671in}}%
\pgfpathlineto{\pgfqpoint{1.086402in}{0.181833in}}%
\pgfpathlineto{\pgfqpoint{1.098448in}{0.182157in}}%
\pgfpathlineto{\pgfqpoint{1.110494in}{0.182642in}}%
\pgfpathlineto{\pgfqpoint{1.122541in}{0.183290in}}%
\pgfpathlineto{\pgfqpoint{1.134587in}{0.184100in}}%
\pgfpathlineto{\pgfqpoint{1.146633in}{0.185074in}}%
\pgfpathlineto{\pgfqpoint{1.158679in}{0.186211in}}%
\pgfpathlineto{\pgfqpoint{1.170725in}{0.187512in}}%
\pgfpathlineto{\pgfqpoint{1.182771in}{0.188978in}}%
\pgfpathlineto{\pgfqpoint{1.194817in}{0.190610in}}%
\pgfpathlineto{\pgfqpoint{1.206863in}{0.192408in}}%
\pgfpathlineto{\pgfqpoint{1.214772in}{0.193697in}}%
\pgfpathlineto{\pgfqpoint{1.218909in}{0.194374in}}%
\pgfpathlineto{\pgfqpoint{1.230956in}{0.196509in}}%
\pgfpathlineto{\pgfqpoint{1.243002in}{0.198814in}}%
\pgfpathlineto{\pgfqpoint{1.255048in}{0.201290in}}%
\pgfpathlineto{\pgfqpoint{1.267094in}{0.203939in}}%
\pgfpathlineto{\pgfqpoint{1.274804in}{0.205743in}}%
\pgfpathlineto{\pgfqpoint{1.279140in}{0.206762in}}%
\pgfpathlineto{\pgfqpoint{1.291186in}{0.209762in}}%
\pgfpathlineto{\pgfqpoint{1.303232in}{0.212940in}}%
\pgfpathlineto{\pgfqpoint{1.315278in}{0.216296in}}%
\pgfpathlineto{\pgfqpoint{1.320375in}{0.217790in}}%
\pgfpathlineto{\pgfqpoint{1.327325in}{0.219836in}}%
\pgfpathlineto{\pgfqpoint{1.339371in}{0.223560in}}%
\pgfpathlineto{\pgfqpoint{1.351417in}{0.227470in}}%
\pgfpathlineto{\pgfqpoint{1.358387in}{0.229836in}}%
\pgfpathlineto{\pgfqpoint{1.363463in}{0.231569in}}%
\pgfpathlineto{\pgfqpoint{1.375509in}{0.235861in}}%
\pgfpathlineto{\pgfqpoint{1.387555in}{0.240346in}}%
\pgfpathlineto{\pgfqpoint{1.391521in}{0.241882in}}%
\pgfpathlineto{\pgfqpoint{1.399601in}{0.245032in}}%
\pgfpathlineto{\pgfqpoint{1.411647in}{0.249918in}}%
\pgfpathlineto{\pgfqpoint{1.421154in}{0.253928in}}%
\pgfpathlineto{\pgfqpoint{1.423693in}{0.255007in}}%
\pgfpathlineto{\pgfqpoint{1.435740in}{0.260309in}}%
\pgfpathlineto{\pgfqpoint{1.447786in}{0.265818in}}%
\pgfpathlineto{\pgfqpoint{1.448116in}{0.265974in}}%
\pgfpathlineto{\pgfqpoint{1.459832in}{0.271552in}}%
\pgfpathlineto{\pgfqpoint{1.471878in}{0.277500in}}%
\pgfpathlineto{\pgfqpoint{1.472899in}{0.278020in}}%
\pgfpathlineto{\pgfqpoint{1.483924in}{0.283683in}}%
\pgfpathlineto{\pgfqpoint{1.495927in}{0.290066in}}%
\pgfpathlineto{\pgfqpoint{1.495970in}{0.290089in}}%
\pgfpathlineto{\pgfqpoint{1.508016in}{0.296744in}}%
\pgfpathlineto{\pgfqpoint{1.517424in}{0.302112in}}%
\pgfpathlineto{\pgfqpoint{1.520062in}{0.303634in}}%
\pgfpathlineto{\pgfqpoint{1.532108in}{0.310780in}}%
\pgfpathlineto{\pgfqpoint{1.537640in}{0.314159in}}%
\pgfpathlineto{\pgfqpoint{1.544155in}{0.318183in}}%
\pgfpathlineto{\pgfqpoint{1.556201in}{0.325844in}}%
\pgfpathlineto{\pgfqpoint{1.556754in}{0.326205in}}%
\pgfpathlineto{\pgfqpoint{1.568247in}{0.333791in}}%
\pgfpathlineto{\pgfqpoint{1.574819in}{0.338251in}}%
\pgfpathlineto{\pgfqpoint{1.580293in}{0.342013in}}%
\pgfpathlineto{\pgfqpoint{1.592025in}{0.350297in}}%
\pgfpathlineto{\pgfqpoint{1.592339in}{0.350522in}}%
\pgfpathlineto{\pgfqpoint{1.604385in}{0.359344in}}%
\pgfpathlineto{\pgfqpoint{1.608378in}{0.362343in}}%
\pgfpathlineto{\pgfqpoint{1.616431in}{0.368478in}}%
\pgfpathlineto{\pgfqpoint{1.624005in}{0.374389in}}%
\pgfpathlineto{\pgfqpoint{1.628477in}{0.377933in}}%
\pgfpathlineto{\pgfqpoint{1.638959in}{0.386435in}}%
\pgfpathlineto{\pgfqpoint{1.640524in}{0.387724in}}%
\pgfpathlineto{\pgfqpoint{1.652570in}{0.397871in}}%
\pgfpathlineto{\pgfqpoint{1.653279in}{0.398481in}}%
\pgfpathlineto{\pgfqpoint{1.664616in}{0.408394in}}%
\pgfpathlineto{\pgfqpoint{1.667003in}{0.410527in}}%
\pgfpathlineto{\pgfqpoint{1.676662in}{0.419304in}}%
\pgfpathlineto{\pgfqpoint{1.680187in}{0.422574in}}%
\pgfpathlineto{\pgfqpoint{1.688708in}{0.430619in}}%
\pgfpathlineto{\pgfqpoint{1.692862in}{0.434620in}}%
\pgfpathlineto{\pgfqpoint{1.700754in}{0.442363in}}%
\pgfpathlineto{\pgfqpoint{1.705057in}{0.446666in}}%
\pgfpathlineto{\pgfqpoint{1.712800in}{0.454558in}}%
\pgfpathlineto{\pgfqpoint{1.716801in}{0.458712in}}%
\pgfpathlineto{\pgfqpoint{1.724846in}{0.467233in}}%
\pgfpathlineto{\pgfqpoint{1.728116in}{0.470758in}}%
\pgfpathlineto{\pgfqpoint{1.736892in}{0.480416in}}%
\pgfpathlineto{\pgfqpoint{1.739025in}{0.482804in}}%
\pgfpathlineto{\pgfqpoint{1.748939in}{0.494141in}}%
\pgfpathlineto{\pgfqpoint{1.749549in}{0.494850in}}%
\pgfpathlineto{\pgfqpoint{1.759696in}{0.506896in}}%
\pgfpathlineto{\pgfqpoint{1.760985in}{0.508460in}}%
\pgfpathlineto{\pgfqpoint{1.769487in}{0.518942in}}%
\pgfpathlineto{\pgfqpoint{1.773031in}{0.523415in}}%
\pgfpathlineto{\pgfqpoint{1.778942in}{0.530989in}}%
\pgfpathlineto{\pgfqpoint{1.785077in}{0.539042in}}%
\pgfpathlineto{\pgfqpoint{1.788075in}{0.543035in}}%
\pgfpathlineto{\pgfqpoint{1.796898in}{0.555081in}}%
\pgfpathlineto{\pgfqpoint{1.797123in}{0.555395in}}%
\pgfpathlineto{\pgfqpoint{1.805407in}{0.567127in}}%
\pgfpathlineto{\pgfqpoint{1.809169in}{0.572601in}}%
\pgfpathlineto{\pgfqpoint{1.813629in}{0.579173in}}%
\pgfpathlineto{\pgfqpoint{1.821215in}{0.590666in}}%
\pgfpathlineto{\pgfqpoint{1.821576in}{0.591219in}}%
\pgfpathlineto{\pgfqpoint{1.829237in}{0.603265in}}%
\pgfpathlineto{\pgfqpoint{1.833261in}{0.609780in}}%
\pgfpathlineto{\pgfqpoint{1.836640in}{0.615311in}}%
\pgfpathlineto{\pgfqpoint{1.843786in}{0.627358in}}%
\pgfpathlineto{\pgfqpoint{1.845307in}{0.629996in}}%
\pgfpathlineto{\pgfqpoint{1.850676in}{0.639404in}}%
\pgfpathlineto{\pgfqpoint{1.857331in}{0.651450in}}%
\pgfpathlineto{\pgfqpoint{1.857354in}{0.651493in}}%
\pgfpathlineto{\pgfqpoint{1.863737in}{0.663496in}}%
\pgfpathlineto{\pgfqpoint{1.869400in}{0.674521in}}%
\pgfpathlineto{\pgfqpoint{1.869920in}{0.675542in}}%
\pgfpathlineto{\pgfqpoint{1.875868in}{0.687588in}}%
\pgfpathlineto{\pgfqpoint{1.881446in}{0.699304in}}%
\pgfpathlineto{\pgfqpoint{1.881602in}{0.699634in}}%
\pgfpathlineto{\pgfqpoint{1.887111in}{0.711680in}}%
\pgfpathlineto{\pgfqpoint{1.892413in}{0.723726in}}%
\pgfpathlineto{\pgfqpoint{1.893492in}{0.726265in}}%
\pgfpathlineto{\pgfqpoint{1.897502in}{0.735773in}}%
\pgfpathlineto{\pgfqpoint{1.902388in}{0.747819in}}%
\pgfpathlineto{\pgfqpoint{1.905538in}{0.755899in}}%
\pgfpathlineto{\pgfqpoint{1.907074in}{0.759865in}}%
\pgfpathlineto{\pgfqpoint{1.911559in}{0.771911in}}%
\pgfpathlineto{\pgfqpoint{1.915851in}{0.783957in}}%
\pgfpathlineto{\pgfqpoint{1.917584in}{0.789033in}}%
\pgfpathlineto{\pgfqpoint{1.919950in}{0.796003in}}%
\pgfpathlineto{\pgfqpoint{1.923860in}{0.808049in}}%
\pgfpathlineto{\pgfqpoint{1.927584in}{0.820095in}}%
\pgfpathlineto{\pgfqpoint{1.929630in}{0.827045in}}%
\pgfpathlineto{\pgfqpoint{1.931124in}{0.832141in}}%
\pgfpathlineto{\pgfqpoint{1.934480in}{0.844188in}}%
\pgfpathlineto{\pgfqpoint{1.937658in}{0.856234in}}%
\pgfpathlineto{\pgfqpoint{1.940658in}{0.868280in}}%
\pgfpathlineto{\pgfqpoint{1.941676in}{0.872616in}}%
\pgfpathlineto{\pgfqpoint{1.943481in}{0.880326in}}%
\pgfpathlineto{\pgfqpoint{1.946130in}{0.892372in}}%
\pgfpathlineto{\pgfqpoint{1.948606in}{0.904418in}}%
\pgfpathlineto{\pgfqpoint{1.950911in}{0.916464in}}%
\pgfpathlineto{\pgfqpoint{1.953046in}{0.928510in}}%
\pgfpathlineto{\pgfqpoint{1.953723in}{0.932648in}}%
\pgfpathlineto{\pgfqpoint{1.955012in}{0.940557in}}%
\pgfpathlineto{\pgfqpoint{1.956810in}{0.952603in}}%
\pgfpathlineto{\pgfqpoint{1.958442in}{0.964649in}}%
\pgfpathlineto{\pgfqpoint{1.959908in}{0.976695in}}%
\pgfpathlineto{\pgfqpoint{1.961209in}{0.988741in}}%
\pgfpathlineto{\pgfqpoint{1.962346in}{1.000787in}}%
\pgfpathlineto{\pgfqpoint{1.963320in}{1.012833in}}%
\pgfpathlineto{\pgfqpoint{1.964130in}{1.024879in}}%
\pgfpathlineto{\pgfqpoint{1.964778in}{1.036925in}}%
\pgfpathlineto{\pgfqpoint{1.965263in}{1.048972in}}%
\pgfpathlineto{\pgfqpoint{1.965587in}{1.061018in}}%
\pgfpathlineto{\pgfqpoint{1.965748in}{1.073064in}}%
\pgfpathlineto{\pgfqpoint{1.965748in}{1.085110in}}%
\pgfpathlineto{\pgfqpoint{1.965587in}{1.097156in}}%
\pgfpathlineto{\pgfqpoint{1.965263in}{1.109202in}}%
\pgfpathlineto{\pgfqpoint{1.964778in}{1.121248in}}%
\pgfpathlineto{\pgfqpoint{1.964130in}{1.133294in}}%
\pgfpathlineto{\pgfqpoint{1.963320in}{1.145340in}}%
\pgfpathlineto{\pgfqpoint{1.962346in}{1.157387in}}%
\pgfpathlineto{\pgfqpoint{1.961209in}{1.169433in}}%
\pgfpathlineto{\pgfqpoint{1.959908in}{1.181479in}}%
\pgfpathlineto{\pgfqpoint{1.958442in}{1.193525in}}%
\pgfpathlineto{\pgfqpoint{1.956810in}{1.205571in}}%
\pgfpathlineto{\pgfqpoint{1.955012in}{1.217617in}}%
\pgfpathlineto{\pgfqpoint{1.953723in}{1.225525in}}%
\pgfpathlineto{\pgfqpoint{1.953046in}{1.229663in}}%
\pgfpathlineto{\pgfqpoint{1.950911in}{1.241709in}}%
\pgfpathlineto{\pgfqpoint{1.948606in}{1.253756in}}%
\pgfpathlineto{\pgfqpoint{1.946130in}{1.265802in}}%
\pgfpathlineto{\pgfqpoint{1.943481in}{1.277848in}}%
\pgfpathlineto{\pgfqpoint{1.941676in}{1.285558in}}%
\pgfpathlineto{\pgfqpoint{1.940658in}{1.289894in}}%
\pgfpathlineto{\pgfqpoint{1.937658in}{1.301940in}}%
\pgfpathlineto{\pgfqpoint{1.934480in}{1.313986in}}%
\pgfpathlineto{\pgfqpoint{1.931124in}{1.326032in}}%
\pgfpathlineto{\pgfqpoint{1.929630in}{1.331129in}}%
\pgfpathlineto{\pgfqpoint{1.927584in}{1.338078in}}%
\pgfpathlineto{\pgfqpoint{1.923860in}{1.350124in}}%
\pgfpathlineto{\pgfqpoint{1.919950in}{1.362171in}}%
\pgfpathlineto{\pgfqpoint{1.917584in}{1.369141in}}%
\pgfpathlineto{\pgfqpoint{1.915851in}{1.374217in}}%
\pgfpathlineto{\pgfqpoint{1.911559in}{1.386263in}}%
\pgfpathlineto{\pgfqpoint{1.907074in}{1.398309in}}%
\pgfpathlineto{\pgfqpoint{1.905538in}{1.402275in}}%
\pgfpathlineto{\pgfqpoint{1.902388in}{1.410355in}}%
\pgfpathlineto{\pgfqpoint{1.897502in}{1.422401in}}%
\pgfpathlineto{\pgfqpoint{1.893492in}{1.431908in}}%
\pgfpathlineto{\pgfqpoint{1.892413in}{1.434447in}}%
\pgfpathlineto{\pgfqpoint{1.887111in}{1.446493in}}%
\pgfpathlineto{\pgfqpoint{1.881602in}{1.458539in}}%
\pgfpathlineto{\pgfqpoint{1.881446in}{1.458870in}}%
\pgfpathlineto{\pgfqpoint{1.875868in}{1.470586in}}%
\pgfpathlineto{\pgfqpoint{1.869920in}{1.482632in}}%
\pgfpathlineto{\pgfqpoint{1.869400in}{1.483653in}}%
\pgfpathlineto{\pgfqpoint{1.863737in}{1.494678in}}%
\pgfpathlineto{\pgfqpoint{1.857354in}{1.506681in}}%
\pgfpathlineto{\pgfqpoint{1.857331in}{1.506724in}}%
\pgfpathlineto{\pgfqpoint{1.850676in}{1.518770in}}%
\pgfpathlineto{\pgfqpoint{1.845307in}{1.528178in}}%
\pgfpathlineto{\pgfqpoint{1.843786in}{1.530816in}}%
\pgfpathlineto{\pgfqpoint{1.836640in}{1.542862in}}%
\pgfpathlineto{\pgfqpoint{1.833261in}{1.548394in}}%
\pgfpathlineto{\pgfqpoint{1.829237in}{1.554908in}}%
\pgfpathlineto{\pgfqpoint{1.821576in}{1.566955in}}%
\pgfpathlineto{\pgfqpoint{1.821215in}{1.567507in}}%
\pgfpathlineto{\pgfqpoint{1.813629in}{1.579001in}}%
\pgfpathlineto{\pgfqpoint{1.809169in}{1.585573in}}%
\pgfpathlineto{\pgfqpoint{1.805407in}{1.591047in}}%
\pgfpathlineto{\pgfqpoint{1.797123in}{1.602779in}}%
\pgfpathlineto{\pgfqpoint{1.796898in}{1.603093in}}%
\pgfpathlineto{\pgfqpoint{1.788075in}{1.615139in}}%
\pgfpathlineto{\pgfqpoint{1.785077in}{1.619132in}}%
\pgfpathlineto{\pgfqpoint{1.778942in}{1.627185in}}%
\pgfpathlineto{\pgfqpoint{1.773031in}{1.634758in}}%
\pgfpathlineto{\pgfqpoint{1.769487in}{1.639231in}}%
\pgfpathlineto{\pgfqpoint{1.760985in}{1.649713in}}%
\pgfpathlineto{\pgfqpoint{1.759696in}{1.651277in}}%
\pgfpathlineto{\pgfqpoint{1.749549in}{1.663323in}}%
\pgfpathlineto{\pgfqpoint{1.748939in}{1.664033in}}%
\pgfpathlineto{\pgfqpoint{1.739025in}{1.675370in}}%
\pgfpathlineto{\pgfqpoint{1.736892in}{1.677757in}}%
\pgfpathlineto{\pgfqpoint{1.728116in}{1.687416in}}%
\pgfpathlineto{\pgfqpoint{1.724846in}{1.690941in}}%
\pgfpathlineto{\pgfqpoint{1.716801in}{1.699462in}}%
\pgfpathlineto{\pgfqpoint{1.712800in}{1.703615in}}%
\pgfpathlineto{\pgfqpoint{1.705057in}{1.711508in}}%
\pgfpathlineto{\pgfqpoint{1.700754in}{1.715811in}}%
\pgfpathlineto{\pgfqpoint{1.692862in}{1.723554in}}%
\pgfpathlineto{\pgfqpoint{1.688708in}{1.727554in}}%
\pgfpathlineto{\pgfqpoint{1.680187in}{1.735600in}}%
\pgfpathlineto{\pgfqpoint{1.676662in}{1.738870in}}%
\pgfpathlineto{\pgfqpoint{1.667003in}{1.747646in}}%
\pgfpathlineto{\pgfqpoint{1.664616in}{1.749779in}}%
\pgfpathlineto{\pgfqpoint{1.653279in}{1.759692in}}%
\pgfpathlineto{\pgfqpoint{1.652570in}{1.760303in}}%
\pgfpathlineto{\pgfqpoint{1.640524in}{1.770450in}}%
\pgfpathlineto{\pgfqpoint{1.638959in}{1.771738in}}%
\pgfpathlineto{\pgfqpoint{1.628477in}{1.780241in}}%
\pgfpathlineto{\pgfqpoint{1.624005in}{1.783785in}}%
\pgfpathlineto{\pgfqpoint{1.616431in}{1.789696in}}%
\pgfpathlineto{\pgfqpoint{1.608378in}{1.795831in}}%
\pgfpathlineto{\pgfqpoint{1.604385in}{1.798829in}}%
\pgfpathlineto{\pgfqpoint{1.592339in}{1.807652in}}%
\pgfpathlineto{\pgfqpoint{1.592025in}{1.807877in}}%
\pgfpathlineto{\pgfqpoint{1.580293in}{1.816160in}}%
\pgfpathlineto{\pgfqpoint{1.574819in}{1.819923in}}%
\pgfpathlineto{\pgfqpoint{1.568247in}{1.824383in}}%
\pgfpathlineto{\pgfqpoint{1.556754in}{1.831969in}}%
\pgfpathlineto{\pgfqpoint{1.556201in}{1.832330in}}%
\pgfpathlineto{\pgfqpoint{1.544155in}{1.839991in}}%
\pgfpathlineto{\pgfqpoint{1.537640in}{1.844015in}}%
\pgfpathlineto{\pgfqpoint{1.532108in}{1.847394in}}%
\pgfpathlineto{\pgfqpoint{1.520062in}{1.854540in}}%
\pgfpathlineto{\pgfqpoint{1.517424in}{1.856061in}}%
\pgfpathlineto{\pgfqpoint{1.508016in}{1.861430in}}%
\pgfpathlineto{\pgfqpoint{1.495970in}{1.868084in}}%
\pgfpathlineto{\pgfqpoint{1.495927in}{1.868107in}}%
\pgfpathlineto{\pgfqpoint{1.483924in}{1.874491in}}%
\pgfpathlineto{\pgfqpoint{1.472899in}{1.880154in}}%
\pgfpathlineto{\pgfqpoint{1.471878in}{1.880673in}}%
\pgfpathlineto{\pgfqpoint{1.459832in}{1.886622in}}%
\pgfpathlineto{\pgfqpoint{1.448116in}{1.892200in}}%
\pgfpathlineto{\pgfqpoint{1.447786in}{1.892356in}}%
\pgfpathlineto{\pgfqpoint{1.435740in}{1.897865in}}%
\pgfpathlineto{\pgfqpoint{1.423693in}{1.903167in}}%
\pgfpathlineto{\pgfqpoint{1.421154in}{1.904246in}}%
\pgfpathlineto{\pgfqpoint{1.411647in}{1.908256in}}%
\pgfpathlineto{\pgfqpoint{1.399601in}{1.913142in}}%
\pgfpathlineto{\pgfqpoint{1.391521in}{1.916292in}}%
\pgfpathlineto{\pgfqpoint{1.387555in}{1.917828in}}%
\pgfpathlineto{\pgfqpoint{1.375509in}{1.922312in}}%
\pgfpathlineto{\pgfqpoint{1.363463in}{1.926605in}}%
\pgfpathlineto{\pgfqpoint{1.358387in}{1.928338in}}%
\pgfpathlineto{\pgfqpoint{1.351417in}{1.930704in}}%
\pgfpathlineto{\pgfqpoint{1.339371in}{1.934614in}}%
\pgfpathlineto{\pgfqpoint{1.327325in}{1.938338in}}%
\pgfpathlineto{\pgfqpoint{1.320375in}{1.940384in}}%
\pgfpathlineto{\pgfqpoint{1.315278in}{1.941877in}}%
\pgfpathlineto{\pgfqpoint{1.303232in}{1.945234in}}%
\pgfpathlineto{\pgfqpoint{1.291186in}{1.948411in}}%
\pgfpathlineto{\pgfqpoint{1.279140in}{1.951412in}}%
\pgfpathlineto{\pgfqpoint{1.274804in}{1.952430in}}%
\pgfpathlineto{\pgfqpoint{1.267094in}{1.954235in}}%
\pgfpathlineto{\pgfqpoint{1.255048in}{1.956883in}}%
\pgfpathlineto{\pgfqpoint{1.243002in}{1.959360in}}%
\pgfpathlineto{\pgfqpoint{1.230956in}{1.961665in}}%
\pgfpathlineto{\pgfqpoint{1.218909in}{1.963800in}}%
\pgfpathlineto{\pgfqpoint{1.214772in}{1.964476in}}%
\pgfpathlineto{\pgfqpoint{1.206863in}{1.965766in}}%
\pgfpathlineto{\pgfqpoint{1.194817in}{1.967564in}}%
\pgfpathlineto{\pgfqpoint{1.182771in}{1.969196in}}%
\pgfpathlineto{\pgfqpoint{1.170725in}{1.970662in}}%
\pgfpathlineto{\pgfqpoint{1.158679in}{1.971963in}}%
\pgfpathlineto{\pgfqpoint{1.146633in}{1.973100in}}%
\pgfpathlineto{\pgfqpoint{1.134587in}{1.974073in}}%
\pgfpathlineto{\pgfqpoint{1.122541in}{1.974884in}}%
\pgfpathlineto{\pgfqpoint{1.110494in}{1.975532in}}%
\pgfpathlineto{\pgfqpoint{1.098448in}{1.976017in}}%
\pgfpathlineto{\pgfqpoint{1.086402in}{1.976341in}}%
\pgfpathlineto{\pgfqpoint{1.074356in}{1.976502in}}%
\pgfpathlineto{\pgfqpoint{1.062310in}{1.976502in}}%
\pgfpathlineto{\pgfqpoint{1.050264in}{1.976341in}}%
\pgfpathlineto{\pgfqpoint{1.038218in}{1.976017in}}%
\pgfpathlineto{\pgfqpoint{1.026172in}{1.975532in}}%
\pgfpathlineto{\pgfqpoint{1.014126in}{1.974884in}}%
\pgfpathlineto{\pgfqpoint{1.002079in}{1.974073in}}%
\pgfpathlineto{\pgfqpoint{0.990033in}{1.973100in}}%
\pgfpathlineto{\pgfqpoint{0.977987in}{1.971963in}}%
\pgfpathlineto{\pgfqpoint{0.965941in}{1.970662in}}%
\pgfpathlineto{\pgfqpoint{0.953895in}{1.969196in}}%
\pgfpathlineto{\pgfqpoint{0.941849in}{1.967564in}}%
\pgfpathlineto{\pgfqpoint{0.929803in}{1.965766in}}%
\pgfpathlineto{\pgfqpoint{0.921894in}{1.964476in}}%
\pgfpathlineto{\pgfqpoint{0.917757in}{1.963800in}}%
\pgfpathlineto{\pgfqpoint{0.905710in}{1.961665in}}%
\pgfpathlineto{\pgfqpoint{0.893664in}{1.959360in}}%
\pgfpathlineto{\pgfqpoint{0.881618in}{1.956883in}}%
\pgfpathlineto{\pgfqpoint{0.869572in}{1.954235in}}%
\pgfpathlineto{\pgfqpoint{0.861862in}{1.952430in}}%
\pgfpathlineto{\pgfqpoint{0.857526in}{1.951412in}}%
\pgfpathlineto{\pgfqpoint{0.845480in}{1.948411in}}%
\pgfpathlineto{\pgfqpoint{0.833434in}{1.945234in}}%
\pgfpathlineto{\pgfqpoint{0.821388in}{1.941877in}}%
\pgfpathlineto{\pgfqpoint{0.816291in}{1.940384in}}%
\pgfpathlineto{\pgfqpoint{0.809342in}{1.938338in}}%
\pgfpathlineto{\pgfqpoint{0.797295in}{1.934614in}}%
\pgfpathlineto{\pgfqpoint{0.785249in}{1.930704in}}%
\pgfpathlineto{\pgfqpoint{0.778279in}{1.928338in}}%
\pgfpathlineto{\pgfqpoint{0.773203in}{1.926605in}}%
\pgfpathlineto{\pgfqpoint{0.761157in}{1.922312in}}%
\pgfpathlineto{\pgfqpoint{0.749111in}{1.917828in}}%
\pgfpathlineto{\pgfqpoint{0.745145in}{1.916292in}}%
\pgfpathlineto{\pgfqpoint{0.737065in}{1.913142in}}%
\pgfpathlineto{\pgfqpoint{0.725019in}{1.908256in}}%
\pgfpathlineto{\pgfqpoint{0.715512in}{1.904246in}}%
\pgfpathlineto{\pgfqpoint{0.712973in}{1.903167in}}%
\pgfpathlineto{\pgfqpoint{0.700927in}{1.897865in}}%
\pgfpathlineto{\pgfqpoint{0.688880in}{1.892356in}}%
\pgfpathlineto{\pgfqpoint{0.688550in}{1.892200in}}%
\pgfpathlineto{\pgfqpoint{0.676834in}{1.886622in}}%
\pgfpathlineto{\pgfqpoint{0.664788in}{1.880673in}}%
\pgfpathlineto{\pgfqpoint{0.663767in}{1.880154in}}%
\pgfpathlineto{\pgfqpoint{0.652742in}{1.874491in}}%
\pgfpathlineto{\pgfqpoint{0.640739in}{1.868107in}}%
\pgfpathlineto{\pgfqpoint{0.640696in}{1.868084in}}%
\pgfpathlineto{\pgfqpoint{0.628650in}{1.861430in}}%
\pgfpathlineto{\pgfqpoint{0.619242in}{1.856061in}}%
\pgfpathlineto{\pgfqpoint{0.616604in}{1.854540in}}%
\pgfpathlineto{\pgfqpoint{0.604558in}{1.847394in}}%
\pgfpathlineto{\pgfqpoint{0.599026in}{1.844015in}}%
\pgfpathlineto{\pgfqpoint{0.592511in}{1.839991in}}%
\pgfpathlineto{\pgfqpoint{0.580465in}{1.832330in}}%
\pgfpathlineto{\pgfqpoint{0.579913in}{1.831969in}}%
\pgfpathlineto{\pgfqpoint{0.568419in}{1.824383in}}%
\pgfpathlineto{\pgfqpoint{0.561847in}{1.819923in}}%
\pgfpathlineto{\pgfqpoint{0.556373in}{1.816160in}}%
\pgfpathlineto{\pgfqpoint{0.544641in}{1.807877in}}%
\pgfpathlineto{\pgfqpoint{0.544327in}{1.807652in}}%
\pgfpathlineto{\pgfqpoint{0.532281in}{1.798829in}}%
\pgfpathlineto{\pgfqpoint{0.528288in}{1.795831in}}%
\pgfpathlineto{\pgfqpoint{0.520235in}{1.789696in}}%
\pgfpathlineto{\pgfqpoint{0.512662in}{1.783785in}}%
\pgfpathlineto{\pgfqpoint{0.508189in}{1.780241in}}%
\pgfpathlineto{\pgfqpoint{0.497707in}{1.771738in}}%
\pgfpathlineto{\pgfqpoint{0.496143in}{1.770450in}}%
\pgfpathlineto{\pgfqpoint{0.484096in}{1.760303in}}%
\pgfpathlineto{\pgfqpoint{0.483387in}{1.759692in}}%
\pgfpathlineto{\pgfqpoint{0.472050in}{1.749779in}}%
\pgfpathlineto{\pgfqpoint{0.469663in}{1.747646in}}%
\pgfpathlineto{\pgfqpoint{0.460004in}{1.738870in}}%
\pgfpathlineto{\pgfqpoint{0.456479in}{1.735600in}}%
\pgfpathlineto{\pgfqpoint{0.447958in}{1.727554in}}%
\pgfpathlineto{\pgfqpoint{0.443805in}{1.723554in}}%
\pgfpathlineto{\pgfqpoint{0.435912in}{1.715811in}}%
\pgfpathlineto{\pgfqpoint{0.431609in}{1.711508in}}%
\pgfpathlineto{\pgfqpoint{0.423866in}{1.703615in}}%
\pgfpathlineto{\pgfqpoint{0.419865in}{1.699462in}}%
\pgfpathlineto{\pgfqpoint{0.411820in}{1.690941in}}%
\pgfpathlineto{\pgfqpoint{0.408550in}{1.687416in}}%
\pgfpathlineto{\pgfqpoint{0.399774in}{1.677757in}}%
\pgfpathlineto{\pgfqpoint{0.397641in}{1.675370in}}%
\pgfpathlineto{\pgfqpoint{0.387728in}{1.664033in}}%
\pgfpathlineto{\pgfqpoint{0.387117in}{1.663323in}}%
\pgfpathlineto{\pgfqpoint{0.376970in}{1.651277in}}%
\pgfpathlineto{\pgfqpoint{0.375681in}{1.649713in}}%
\pgfpathlineto{\pgfqpoint{0.367179in}{1.639231in}}%
\pgfpathlineto{\pgfqpoint{0.363635in}{1.634758in}}%
\pgfpathlineto{\pgfqpoint{0.357724in}{1.627185in}}%
\pgfpathlineto{\pgfqpoint{0.351589in}{1.619132in}}%
\pgfpathlineto{\pgfqpoint{0.348591in}{1.615139in}}%
\pgfpathlineto{\pgfqpoint{0.339768in}{1.603093in}}%
\pgfpathlineto{\pgfqpoint{0.339543in}{1.602779in}}%
\pgfpathlineto{\pgfqpoint{0.331259in}{1.591047in}}%
\pgfpathlineto{\pgfqpoint{0.327497in}{1.585573in}}%
\pgfpathlineto{\pgfqpoint{0.323037in}{1.579001in}}%
\pgfpathlineto{\pgfqpoint{0.315451in}{1.567507in}}%
\pgfpathlineto{\pgfqpoint{0.315090in}{1.566955in}}%
\pgfpathlineto{\pgfqpoint{0.307429in}{1.554908in}}%
\pgfpathlineto{\pgfqpoint{0.303405in}{1.548394in}}%
\pgfpathlineto{\pgfqpoint{0.300026in}{1.542862in}}%
\pgfpathlineto{\pgfqpoint{0.292880in}{1.530816in}}%
\pgfpathlineto{\pgfqpoint{0.291359in}{1.528178in}}%
\pgfpathlineto{\pgfqpoint{0.285990in}{1.518770in}}%
\pgfpathlineto{\pgfqpoint{0.279335in}{1.506724in}}%
\pgfpathlineto{\pgfqpoint{0.279312in}{1.506681in}}%
\pgfpathlineto{\pgfqpoint{0.272929in}{1.494678in}}%
\pgfpathlineto{\pgfqpoint{0.267266in}{1.483653in}}%
\pgfpathlineto{\pgfqpoint{0.266746in}{1.482632in}}%
\pgfpathlineto{\pgfqpoint{0.260798in}{1.470586in}}%
\pgfpathlineto{\pgfqpoint{0.255220in}{1.458870in}}%
\pgfpathlineto{\pgfqpoint{0.255064in}{1.458539in}}%
\pgfpathlineto{\pgfqpoint{0.249555in}{1.446493in}}%
\pgfpathlineto{\pgfqpoint{0.244253in}{1.434447in}}%
\pgfpathlineto{\pgfqpoint{0.243174in}{1.431908in}}%
\pgfpathlineto{\pgfqpoint{0.239164in}{1.422401in}}%
\pgfpathlineto{\pgfqpoint{0.234278in}{1.410355in}}%
\pgfpathlineto{\pgfqpoint{0.231128in}{1.402275in}}%
\pgfpathlineto{\pgfqpoint{0.229592in}{1.398309in}}%
\pgfpathlineto{\pgfqpoint{0.225107in}{1.386263in}}%
\pgfpathlineto{\pgfqpoint{0.220815in}{1.374217in}}%
\pgfpathlineto{\pgfqpoint{0.219082in}{1.369141in}}%
\pgfpathlineto{\pgfqpoint{0.216716in}{1.362171in}}%
\pgfpathlineto{\pgfqpoint{0.212806in}{1.350124in}}%
\pgfpathlineto{\pgfqpoint{0.209082in}{1.338078in}}%
\pgfpathlineto{\pgfqpoint{0.207036in}{1.331129in}}%
\pgfpathlineto{\pgfqpoint{0.205543in}{1.326032in}}%
\pgfpathlineto{\pgfqpoint{0.202186in}{1.313986in}}%
\pgfpathlineto{\pgfqpoint{0.199008in}{1.301940in}}%
\pgfpathlineto{\pgfqpoint{0.196008in}{1.289894in}}%
\pgfpathlineto{\pgfqpoint{0.194990in}{1.285558in}}%
\pgfpathlineto{\pgfqpoint{0.193185in}{1.277848in}}%
\pgfpathlineto{\pgfqpoint{0.190536in}{1.265802in}}%
\pgfpathlineto{\pgfqpoint{0.188060in}{1.253756in}}%
\pgfpathlineto{\pgfqpoint{0.185755in}{1.241709in}}%
\pgfpathlineto{\pgfqpoint{0.183620in}{1.229663in}}%
\pgfpathlineto{\pgfqpoint{0.182944in}{1.225525in}}%
\pgfpathlineto{\pgfqpoint{0.181654in}{1.217617in}}%
\pgfpathlineto{\pgfqpoint{0.179856in}{1.205571in}}%
\pgfpathlineto{\pgfqpoint{0.178224in}{1.193525in}}%
\pgfpathlineto{\pgfqpoint{0.176758in}{1.181479in}}%
\pgfpathlineto{\pgfqpoint{0.175457in}{1.169433in}}%
\pgfpathlineto{\pgfqpoint{0.174320in}{1.157387in}}%
\pgfpathlineto{\pgfqpoint{0.173346in}{1.145340in}}%
\pgfpathlineto{\pgfqpoint{0.172536in}{1.133294in}}%
\pgfpathlineto{\pgfqpoint{0.171888in}{1.121248in}}%
\pgfpathlineto{\pgfqpoint{0.171403in}{1.109202in}}%
\pgfpathlineto{\pgfqpoint{0.171079in}{1.097156in}}%
\pgfpathlineto{\pgfqpoint{0.170918in}{1.085110in}}%
\pgfpathlineto{\pgfqpoint{0.170918in}{1.073064in}}%
\pgfpathlineto{\pgfqpoint{0.171079in}{1.061018in}}%
\pgfpathlineto{\pgfqpoint{0.171403in}{1.048972in}}%
\pgfpathlineto{\pgfqpoint{0.171888in}{1.036925in}}%
\pgfpathlineto{\pgfqpoint{0.172536in}{1.024879in}}%
\pgfpathlineto{\pgfqpoint{0.173346in}{1.012833in}}%
\pgfpathlineto{\pgfqpoint{0.174320in}{1.000787in}}%
\pgfpathlineto{\pgfqpoint{0.175457in}{0.988741in}}%
\pgfpathlineto{\pgfqpoint{0.176758in}{0.976695in}}%
\pgfpathlineto{\pgfqpoint{0.178224in}{0.964649in}}%
\pgfpathlineto{\pgfqpoint{0.179856in}{0.952603in}}%
\pgfpathlineto{\pgfqpoint{0.181654in}{0.940557in}}%
\pgfpathlineto{\pgfqpoint{0.182944in}{0.932648in}}%
\pgfpathlineto{\pgfqpoint{0.183620in}{0.928510in}}%
\pgfpathlineto{\pgfqpoint{0.185755in}{0.916464in}}%
\pgfpathlineto{\pgfqpoint{0.188060in}{0.904418in}}%
\pgfpathlineto{\pgfqpoint{0.190536in}{0.892372in}}%
\pgfpathlineto{\pgfqpoint{0.193185in}{0.880326in}}%
\pgfpathlineto{\pgfqpoint{0.194990in}{0.872616in}}%
\pgfpathlineto{\pgfqpoint{0.196008in}{0.868280in}}%
\pgfpathlineto{\pgfqpoint{0.199008in}{0.856234in}}%
\pgfpathlineto{\pgfqpoint{0.202186in}{0.844188in}}%
\pgfpathlineto{\pgfqpoint{0.205543in}{0.832141in}}%
\pgfpathlineto{\pgfqpoint{0.207036in}{0.827045in}}%
\pgfpathlineto{\pgfqpoint{0.209082in}{0.820095in}}%
\pgfpathlineto{\pgfqpoint{0.212806in}{0.808049in}}%
\pgfpathlineto{\pgfqpoint{0.216716in}{0.796003in}}%
\pgfpathlineto{\pgfqpoint{0.219082in}{0.789033in}}%
\pgfpathlineto{\pgfqpoint{0.220815in}{0.783957in}}%
\pgfpathlineto{\pgfqpoint{0.225107in}{0.771911in}}%
\pgfpathlineto{\pgfqpoint{0.229592in}{0.759865in}}%
\pgfpathlineto{\pgfqpoint{0.231128in}{0.755899in}}%
\pgfpathlineto{\pgfqpoint{0.234278in}{0.747819in}}%
\pgfpathlineto{\pgfqpoint{0.239164in}{0.735773in}}%
\pgfpathlineto{\pgfqpoint{0.243174in}{0.726265in}}%
\pgfpathlineto{\pgfqpoint{0.244253in}{0.723726in}}%
\pgfpathlineto{\pgfqpoint{0.249555in}{0.711680in}}%
\pgfpathlineto{\pgfqpoint{0.255064in}{0.699634in}}%
\pgfpathlineto{\pgfqpoint{0.255220in}{0.699304in}}%
\pgfpathlineto{\pgfqpoint{0.260798in}{0.687588in}}%
\pgfpathlineto{\pgfqpoint{0.266746in}{0.675542in}}%
\pgfpathlineto{\pgfqpoint{0.267266in}{0.674521in}}%
\pgfpathlineto{\pgfqpoint{0.272929in}{0.663496in}}%
\pgfpathlineto{\pgfqpoint{0.279312in}{0.651493in}}%
\pgfpathlineto{\pgfqpoint{0.279335in}{0.651450in}}%
\pgfpathlineto{\pgfqpoint{0.285990in}{0.639404in}}%
\pgfpathlineto{\pgfqpoint{0.291359in}{0.629996in}}%
\pgfpathlineto{\pgfqpoint{0.292880in}{0.627358in}}%
\pgfpathlineto{\pgfqpoint{0.300026in}{0.615311in}}%
\pgfpathlineto{\pgfqpoint{0.303405in}{0.609780in}}%
\pgfpathlineto{\pgfqpoint{0.307429in}{0.603265in}}%
\pgfpathlineto{\pgfqpoint{0.315090in}{0.591219in}}%
\pgfpathlineto{\pgfqpoint{0.315451in}{0.590666in}}%
\pgfpathlineto{\pgfqpoint{0.323037in}{0.579173in}}%
\pgfpathlineto{\pgfqpoint{0.327497in}{0.572601in}}%
\pgfpathlineto{\pgfqpoint{0.331259in}{0.567127in}}%
\pgfpathlineto{\pgfqpoint{0.339543in}{0.555395in}}%
\pgfpathlineto{\pgfqpoint{0.339768in}{0.555081in}}%
\pgfpathlineto{\pgfqpoint{0.348591in}{0.543035in}}%
\pgfpathlineto{\pgfqpoint{0.351589in}{0.539042in}}%
\pgfpathlineto{\pgfqpoint{0.357724in}{0.530989in}}%
\pgfpathlineto{\pgfqpoint{0.363635in}{0.523415in}}%
\pgfpathlineto{\pgfqpoint{0.367179in}{0.518942in}}%
\pgfpathlineto{\pgfqpoint{0.375681in}{0.508460in}}%
\pgfpathlineto{\pgfqpoint{0.376970in}{0.506896in}}%
\pgfpathlineto{\pgfqpoint{0.387117in}{0.494850in}}%
\pgfpathlineto{\pgfqpoint{0.387728in}{0.494141in}}%
\pgfpathlineto{\pgfqpoint{0.397641in}{0.482804in}}%
\pgfpathlineto{\pgfqpoint{0.399774in}{0.480416in}}%
\pgfpathlineto{\pgfqpoint{0.408550in}{0.470758in}}%
\pgfpathlineto{\pgfqpoint{0.411820in}{0.467233in}}%
\pgfpathlineto{\pgfqpoint{0.419865in}{0.458712in}}%
\pgfpathlineto{\pgfqpoint{0.423866in}{0.454558in}}%
\pgfpathlineto{\pgfqpoint{0.431609in}{0.446666in}}%
\pgfpathlineto{\pgfqpoint{0.435912in}{0.442363in}}%
\pgfpathlineto{\pgfqpoint{0.443805in}{0.434620in}}%
\pgfpathlineto{\pgfqpoint{0.447958in}{0.430619in}}%
\pgfpathlineto{\pgfqpoint{0.456479in}{0.422574in}}%
\pgfpathlineto{\pgfqpoint{0.460004in}{0.419304in}}%
\pgfpathlineto{\pgfqpoint{0.469663in}{0.410527in}}%
\pgfpathlineto{\pgfqpoint{0.472050in}{0.408394in}}%
\pgfpathlineto{\pgfqpoint{0.483387in}{0.398481in}}%
\pgfpathlineto{\pgfqpoint{0.484096in}{0.397871in}}%
\pgfpathlineto{\pgfqpoint{0.496143in}{0.387724in}}%
\pgfpathlineto{\pgfqpoint{0.497707in}{0.386435in}}%
\pgfpathlineto{\pgfqpoint{0.508189in}{0.377933in}}%
\pgfpathlineto{\pgfqpoint{0.512662in}{0.374389in}}%
\pgfpathlineto{\pgfqpoint{0.520235in}{0.368478in}}%
\pgfpathlineto{\pgfqpoint{0.528288in}{0.362343in}}%
\pgfpathlineto{\pgfqpoint{0.532281in}{0.359344in}}%
\pgfpathlineto{\pgfqpoint{0.544327in}{0.350522in}}%
\pgfpathlineto{\pgfqpoint{0.544641in}{0.350297in}}%
\pgfpathlineto{\pgfqpoint{0.556373in}{0.342013in}}%
\pgfpathlineto{\pgfqpoint{0.561847in}{0.338251in}}%
\pgfpathlineto{\pgfqpoint{0.568419in}{0.333791in}}%
\pgfpathlineto{\pgfqpoint{0.579913in}{0.326205in}}%
\pgfpathlineto{\pgfqpoint{0.580465in}{0.325844in}}%
\pgfpathlineto{\pgfqpoint{0.592511in}{0.318183in}}%
\pgfpathlineto{\pgfqpoint{0.599026in}{0.314159in}}%
\pgfpathlineto{\pgfqpoint{0.604558in}{0.310780in}}%
\pgfpathlineto{\pgfqpoint{0.616604in}{0.303634in}}%
\pgfpathlineto{\pgfqpoint{0.619242in}{0.302112in}}%
\pgfpathlineto{\pgfqpoint{0.628650in}{0.296744in}}%
\pgfpathlineto{\pgfqpoint{0.640696in}{0.290089in}}%
\pgfpathlineto{\pgfqpoint{0.640739in}{0.290066in}}%
\pgfpathlineto{\pgfqpoint{0.652742in}{0.283683in}}%
\pgfpathlineto{\pgfqpoint{0.663767in}{0.278020in}}%
\pgfpathlineto{\pgfqpoint{0.664788in}{0.277500in}}%
\pgfpathlineto{\pgfqpoint{0.676834in}{0.271552in}}%
\pgfpathlineto{\pgfqpoint{0.688550in}{0.265974in}}%
\pgfpathlineto{\pgfqpoint{0.688880in}{0.265818in}}%
\pgfpathlineto{\pgfqpoint{0.700927in}{0.260309in}}%
\pgfpathlineto{\pgfqpoint{0.712973in}{0.255007in}}%
\pgfpathlineto{\pgfqpoint{0.715512in}{0.253928in}}%
\pgfpathlineto{\pgfqpoint{0.725019in}{0.249918in}}%
\pgfpathlineto{\pgfqpoint{0.737065in}{0.245032in}}%
\pgfpathlineto{\pgfqpoint{0.745145in}{0.241882in}}%
\pgfpathlineto{\pgfqpoint{0.749111in}{0.240346in}}%
\pgfpathlineto{\pgfqpoint{0.761157in}{0.235861in}}%
\pgfpathlineto{\pgfqpoint{0.773203in}{0.231569in}}%
\pgfpathlineto{\pgfqpoint{0.778279in}{0.229836in}}%
\pgfpathlineto{\pgfqpoint{0.785249in}{0.227470in}}%
\pgfpathlineto{\pgfqpoint{0.797295in}{0.223560in}}%
\pgfpathlineto{\pgfqpoint{0.809342in}{0.219836in}}%
\pgfpathlineto{\pgfqpoint{0.816291in}{0.217790in}}%
\pgfpathlineto{\pgfqpoint{0.821388in}{0.216296in}}%
\pgfpathlineto{\pgfqpoint{0.833434in}{0.212940in}}%
\pgfpathlineto{\pgfqpoint{0.845480in}{0.209762in}}%
\pgfpathlineto{\pgfqpoint{0.857526in}{0.206762in}}%
\pgfpathlineto{\pgfqpoint{0.861862in}{0.205743in}}%
\pgfpathlineto{\pgfqpoint{0.869572in}{0.203939in}}%
\pgfpathlineto{\pgfqpoint{0.881618in}{0.201290in}}%
\pgfpathlineto{\pgfqpoint{0.893664in}{0.198814in}}%
\pgfpathlineto{\pgfqpoint{0.905710in}{0.196509in}}%
\pgfpathlineto{\pgfqpoint{0.917757in}{0.194374in}}%
\pgfpathlineto{\pgfqpoint{0.921894in}{0.193697in}}%
\pgfpathclose%
\pgfpathmoveto{\pgfqpoint{0.921894in}{0.193697in}}%
\pgfpathlineto{\pgfqpoint{0.917757in}{0.194374in}}%
\pgfpathlineto{\pgfqpoint{0.905710in}{0.196509in}}%
\pgfpathlineto{\pgfqpoint{0.893664in}{0.198814in}}%
\pgfpathlineto{\pgfqpoint{0.881618in}{0.201290in}}%
\pgfpathlineto{\pgfqpoint{0.869572in}{0.203939in}}%
\pgfpathlineto{\pgfqpoint{0.861862in}{0.205743in}}%
\pgfpathlineto{\pgfqpoint{0.857526in}{0.206762in}}%
\pgfpathlineto{\pgfqpoint{0.845480in}{0.209762in}}%
\pgfpathlineto{\pgfqpoint{0.833434in}{0.212940in}}%
\pgfpathlineto{\pgfqpoint{0.821388in}{0.216296in}}%
\pgfpathlineto{\pgfqpoint{0.816291in}{0.217790in}}%
\pgfpathlineto{\pgfqpoint{0.809342in}{0.219836in}}%
\pgfpathlineto{\pgfqpoint{0.797295in}{0.223560in}}%
\pgfpathlineto{\pgfqpoint{0.785249in}{0.227470in}}%
\pgfpathlineto{\pgfqpoint{0.778279in}{0.229836in}}%
\pgfpathlineto{\pgfqpoint{0.773203in}{0.231569in}}%
\pgfpathlineto{\pgfqpoint{0.761157in}{0.235861in}}%
\pgfpathlineto{\pgfqpoint{0.749111in}{0.240346in}}%
\pgfpathlineto{\pgfqpoint{0.745145in}{0.241882in}}%
\pgfpathlineto{\pgfqpoint{0.737065in}{0.245032in}}%
\pgfpathlineto{\pgfqpoint{0.725019in}{0.249918in}}%
\pgfpathlineto{\pgfqpoint{0.715512in}{0.253928in}}%
\pgfpathlineto{\pgfqpoint{0.712973in}{0.255007in}}%
\pgfpathlineto{\pgfqpoint{0.700927in}{0.260309in}}%
\pgfpathlineto{\pgfqpoint{0.688880in}{0.265818in}}%
\pgfpathlineto{\pgfqpoint{0.688550in}{0.265974in}}%
\pgfpathlineto{\pgfqpoint{0.676834in}{0.271552in}}%
\pgfpathlineto{\pgfqpoint{0.664788in}{0.277500in}}%
\pgfpathlineto{\pgfqpoint{0.663767in}{0.278020in}}%
\pgfpathlineto{\pgfqpoint{0.652742in}{0.283683in}}%
\pgfpathlineto{\pgfqpoint{0.640739in}{0.290066in}}%
\pgfpathlineto{\pgfqpoint{0.640696in}{0.290089in}}%
\pgfpathlineto{\pgfqpoint{0.628650in}{0.296744in}}%
\pgfpathlineto{\pgfqpoint{0.619242in}{0.302112in}}%
\pgfpathlineto{\pgfqpoint{0.616604in}{0.303634in}}%
\pgfpathlineto{\pgfqpoint{0.604558in}{0.310780in}}%
\pgfpathlineto{\pgfqpoint{0.599026in}{0.314159in}}%
\pgfpathlineto{\pgfqpoint{0.592511in}{0.318183in}}%
\pgfpathlineto{\pgfqpoint{0.580465in}{0.325844in}}%
\pgfpathlineto{\pgfqpoint{0.579913in}{0.326205in}}%
\pgfpathlineto{\pgfqpoint{0.568419in}{0.333791in}}%
\pgfpathlineto{\pgfqpoint{0.561847in}{0.338251in}}%
\pgfpathlineto{\pgfqpoint{0.556373in}{0.342013in}}%
\pgfpathlineto{\pgfqpoint{0.544641in}{0.350297in}}%
\pgfpathlineto{\pgfqpoint{0.544327in}{0.350522in}}%
\pgfpathlineto{\pgfqpoint{0.532281in}{0.359344in}}%
\pgfpathlineto{\pgfqpoint{0.528288in}{0.362343in}}%
\pgfpathlineto{\pgfqpoint{0.520235in}{0.368478in}}%
\pgfpathlineto{\pgfqpoint{0.512662in}{0.374389in}}%
\pgfpathlineto{\pgfqpoint{0.508189in}{0.377933in}}%
\pgfpathlineto{\pgfqpoint{0.497707in}{0.386435in}}%
\pgfpathlineto{\pgfqpoint{0.496143in}{0.387724in}}%
\pgfpathlineto{\pgfqpoint{0.484096in}{0.397871in}}%
\pgfpathlineto{\pgfqpoint{0.483387in}{0.398481in}}%
\pgfpathlineto{\pgfqpoint{0.472050in}{0.408394in}}%
\pgfpathlineto{\pgfqpoint{0.469663in}{0.410527in}}%
\pgfpathlineto{\pgfqpoint{0.460004in}{0.419304in}}%
\pgfpathlineto{\pgfqpoint{0.456479in}{0.422574in}}%
\pgfpathlineto{\pgfqpoint{0.447958in}{0.430619in}}%
\pgfpathlineto{\pgfqpoint{0.443805in}{0.434620in}}%
\pgfpathlineto{\pgfqpoint{0.435912in}{0.442363in}}%
\pgfpathlineto{\pgfqpoint{0.431609in}{0.446666in}}%
\pgfpathlineto{\pgfqpoint{0.423866in}{0.454558in}}%
\pgfpathlineto{\pgfqpoint{0.419865in}{0.458712in}}%
\pgfpathlineto{\pgfqpoint{0.411820in}{0.467233in}}%
\pgfpathlineto{\pgfqpoint{0.408550in}{0.470758in}}%
\pgfpathlineto{\pgfqpoint{0.399774in}{0.480416in}}%
\pgfpathlineto{\pgfqpoint{0.397641in}{0.482804in}}%
\pgfpathlineto{\pgfqpoint{0.387728in}{0.494141in}}%
\pgfpathlineto{\pgfqpoint{0.387117in}{0.494850in}}%
\pgfpathlineto{\pgfqpoint{0.376970in}{0.506896in}}%
\pgfpathlineto{\pgfqpoint{0.375681in}{0.508460in}}%
\pgfpathlineto{\pgfqpoint{0.367179in}{0.518942in}}%
\pgfpathlineto{\pgfqpoint{0.363635in}{0.523415in}}%
\pgfpathlineto{\pgfqpoint{0.357724in}{0.530989in}}%
\pgfpathlineto{\pgfqpoint{0.351589in}{0.539042in}}%
\pgfpathlineto{\pgfqpoint{0.348591in}{0.543035in}}%
\pgfpathlineto{\pgfqpoint{0.339768in}{0.555081in}}%
\pgfpathlineto{\pgfqpoint{0.339543in}{0.555395in}}%
\pgfpathlineto{\pgfqpoint{0.331259in}{0.567127in}}%
\pgfpathlineto{\pgfqpoint{0.327497in}{0.572601in}}%
\pgfpathlineto{\pgfqpoint{0.323037in}{0.579173in}}%
\pgfpathlineto{\pgfqpoint{0.315451in}{0.590666in}}%
\pgfpathlineto{\pgfqpoint{0.315090in}{0.591219in}}%
\pgfpathlineto{\pgfqpoint{0.307429in}{0.603265in}}%
\pgfpathlineto{\pgfqpoint{0.303405in}{0.609780in}}%
\pgfpathlineto{\pgfqpoint{0.300026in}{0.615311in}}%
\pgfpathlineto{\pgfqpoint{0.292880in}{0.627358in}}%
\pgfpathlineto{\pgfqpoint{0.291359in}{0.629996in}}%
\pgfpathlineto{\pgfqpoint{0.285990in}{0.639404in}}%
\pgfpathlineto{\pgfqpoint{0.279335in}{0.651450in}}%
\pgfpathlineto{\pgfqpoint{0.279312in}{0.651493in}}%
\pgfpathlineto{\pgfqpoint{0.272929in}{0.663496in}}%
\pgfpathlineto{\pgfqpoint{0.267266in}{0.674521in}}%
\pgfpathlineto{\pgfqpoint{0.266746in}{0.675542in}}%
\pgfpathlineto{\pgfqpoint{0.260798in}{0.687588in}}%
\pgfpathlineto{\pgfqpoint{0.255220in}{0.699304in}}%
\pgfpathlineto{\pgfqpoint{0.255064in}{0.699634in}}%
\pgfpathlineto{\pgfqpoint{0.249555in}{0.711680in}}%
\pgfpathlineto{\pgfqpoint{0.244253in}{0.723726in}}%
\pgfpathlineto{\pgfqpoint{0.243174in}{0.726265in}}%
\pgfpathlineto{\pgfqpoint{0.239164in}{0.735773in}}%
\pgfpathlineto{\pgfqpoint{0.234278in}{0.747819in}}%
\pgfpathlineto{\pgfqpoint{0.231128in}{0.755899in}}%
\pgfpathlineto{\pgfqpoint{0.229592in}{0.759865in}}%
\pgfpathlineto{\pgfqpoint{0.225107in}{0.771911in}}%
\pgfpathlineto{\pgfqpoint{0.220815in}{0.783957in}}%
\pgfpathlineto{\pgfqpoint{0.219082in}{0.789033in}}%
\pgfpathlineto{\pgfqpoint{0.216716in}{0.796003in}}%
\pgfpathlineto{\pgfqpoint{0.212806in}{0.808049in}}%
\pgfpathlineto{\pgfqpoint{0.209082in}{0.820095in}}%
\pgfpathlineto{\pgfqpoint{0.207036in}{0.827045in}}%
\pgfpathlineto{\pgfqpoint{0.205543in}{0.832141in}}%
\pgfpathlineto{\pgfqpoint{0.202186in}{0.844188in}}%
\pgfpathlineto{\pgfqpoint{0.199008in}{0.856234in}}%
\pgfpathlineto{\pgfqpoint{0.196008in}{0.868280in}}%
\pgfpathlineto{\pgfqpoint{0.194990in}{0.872616in}}%
\pgfpathlineto{\pgfqpoint{0.193185in}{0.880326in}}%
\pgfpathlineto{\pgfqpoint{0.190536in}{0.892372in}}%
\pgfpathlineto{\pgfqpoint{0.188060in}{0.904418in}}%
\pgfpathlineto{\pgfqpoint{0.185755in}{0.916464in}}%
\pgfpathlineto{\pgfqpoint{0.183620in}{0.928510in}}%
\pgfpathlineto{\pgfqpoint{0.182944in}{0.932648in}}%
\pgfpathlineto{\pgfqpoint{0.181654in}{0.940557in}}%
\pgfpathlineto{\pgfqpoint{0.179856in}{0.952603in}}%
\pgfpathlineto{\pgfqpoint{0.178224in}{0.964649in}}%
\pgfpathlineto{\pgfqpoint{0.176758in}{0.976695in}}%
\pgfpathlineto{\pgfqpoint{0.175457in}{0.988741in}}%
\pgfpathlineto{\pgfqpoint{0.174320in}{1.000787in}}%
\pgfpathlineto{\pgfqpoint{0.173346in}{1.012833in}}%
\pgfpathlineto{\pgfqpoint{0.172536in}{1.024879in}}%
\pgfpathlineto{\pgfqpoint{0.171888in}{1.036925in}}%
\pgfpathlineto{\pgfqpoint{0.171403in}{1.048972in}}%
\pgfpathlineto{\pgfqpoint{0.171079in}{1.061018in}}%
\pgfpathlineto{\pgfqpoint{0.170918in}{1.073064in}}%
\pgfpathlineto{\pgfqpoint{0.170918in}{1.085110in}}%
\pgfpathlineto{\pgfqpoint{0.171079in}{1.097156in}}%
\pgfpathlineto{\pgfqpoint{0.171403in}{1.109202in}}%
\pgfpathlineto{\pgfqpoint{0.171888in}{1.121248in}}%
\pgfpathlineto{\pgfqpoint{0.172536in}{1.133294in}}%
\pgfpathlineto{\pgfqpoint{0.173346in}{1.145340in}}%
\pgfpathlineto{\pgfqpoint{0.174320in}{1.157387in}}%
\pgfpathlineto{\pgfqpoint{0.175457in}{1.169433in}}%
\pgfpathlineto{\pgfqpoint{0.176758in}{1.181479in}}%
\pgfpathlineto{\pgfqpoint{0.178224in}{1.193525in}}%
\pgfpathlineto{\pgfqpoint{0.179856in}{1.205571in}}%
\pgfpathlineto{\pgfqpoint{0.181654in}{1.217617in}}%
\pgfpathlineto{\pgfqpoint{0.182944in}{1.225525in}}%
\pgfpathlineto{\pgfqpoint{0.183620in}{1.229663in}}%
\pgfpathlineto{\pgfqpoint{0.185755in}{1.241709in}}%
\pgfpathlineto{\pgfqpoint{0.188060in}{1.253756in}}%
\pgfpathlineto{\pgfqpoint{0.190536in}{1.265802in}}%
\pgfpathlineto{\pgfqpoint{0.193185in}{1.277848in}}%
\pgfpathlineto{\pgfqpoint{0.194990in}{1.285558in}}%
\pgfpathlineto{\pgfqpoint{0.196008in}{1.289894in}}%
\pgfpathlineto{\pgfqpoint{0.199008in}{1.301940in}}%
\pgfpathlineto{\pgfqpoint{0.202186in}{1.313986in}}%
\pgfpathlineto{\pgfqpoint{0.205543in}{1.326032in}}%
\pgfpathlineto{\pgfqpoint{0.207036in}{1.331129in}}%
\pgfpathlineto{\pgfqpoint{0.209082in}{1.338078in}}%
\pgfpathlineto{\pgfqpoint{0.212806in}{1.350124in}}%
\pgfpathlineto{\pgfqpoint{0.216716in}{1.362171in}}%
\pgfpathlineto{\pgfqpoint{0.219082in}{1.369141in}}%
\pgfpathlineto{\pgfqpoint{0.220815in}{1.374217in}}%
\pgfpathlineto{\pgfqpoint{0.225107in}{1.386263in}}%
\pgfpathlineto{\pgfqpoint{0.229592in}{1.398309in}}%
\pgfpathlineto{\pgfqpoint{0.231128in}{1.402275in}}%
\pgfpathlineto{\pgfqpoint{0.234278in}{1.410355in}}%
\pgfpathlineto{\pgfqpoint{0.239164in}{1.422401in}}%
\pgfpathlineto{\pgfqpoint{0.243174in}{1.431908in}}%
\pgfpathlineto{\pgfqpoint{0.244253in}{1.434447in}}%
\pgfpathlineto{\pgfqpoint{0.249555in}{1.446493in}}%
\pgfpathlineto{\pgfqpoint{0.255064in}{1.458539in}}%
\pgfpathlineto{\pgfqpoint{0.255220in}{1.458870in}}%
\pgfpathlineto{\pgfqpoint{0.260798in}{1.470586in}}%
\pgfpathlineto{\pgfqpoint{0.266746in}{1.482632in}}%
\pgfpathlineto{\pgfqpoint{0.267266in}{1.483653in}}%
\pgfpathlineto{\pgfqpoint{0.272929in}{1.494678in}}%
\pgfpathlineto{\pgfqpoint{0.279312in}{1.506681in}}%
\pgfpathlineto{\pgfqpoint{0.279335in}{1.506724in}}%
\pgfpathlineto{\pgfqpoint{0.285990in}{1.518770in}}%
\pgfpathlineto{\pgfqpoint{0.291359in}{1.528178in}}%
\pgfpathlineto{\pgfqpoint{0.292880in}{1.530816in}}%
\pgfpathlineto{\pgfqpoint{0.300026in}{1.542862in}}%
\pgfpathlineto{\pgfqpoint{0.303405in}{1.548394in}}%
\pgfpathlineto{\pgfqpoint{0.307429in}{1.554908in}}%
\pgfpathlineto{\pgfqpoint{0.315090in}{1.566955in}}%
\pgfpathlineto{\pgfqpoint{0.315451in}{1.567507in}}%
\pgfpathlineto{\pgfqpoint{0.323037in}{1.579001in}}%
\pgfpathlineto{\pgfqpoint{0.327497in}{1.585573in}}%
\pgfpathlineto{\pgfqpoint{0.331259in}{1.591047in}}%
\pgfpathlineto{\pgfqpoint{0.339543in}{1.602779in}}%
\pgfpathlineto{\pgfqpoint{0.339768in}{1.603093in}}%
\pgfpathlineto{\pgfqpoint{0.348591in}{1.615139in}}%
\pgfpathlineto{\pgfqpoint{0.351589in}{1.619132in}}%
\pgfpathlineto{\pgfqpoint{0.357724in}{1.627185in}}%
\pgfpathlineto{\pgfqpoint{0.363635in}{1.634758in}}%
\pgfpathlineto{\pgfqpoint{0.367179in}{1.639231in}}%
\pgfpathlineto{\pgfqpoint{0.375681in}{1.649713in}}%
\pgfpathlineto{\pgfqpoint{0.376970in}{1.651277in}}%
\pgfpathlineto{\pgfqpoint{0.387117in}{1.663323in}}%
\pgfpathlineto{\pgfqpoint{0.387728in}{1.664033in}}%
\pgfpathlineto{\pgfqpoint{0.397641in}{1.675370in}}%
\pgfpathlineto{\pgfqpoint{0.399774in}{1.677757in}}%
\pgfpathlineto{\pgfqpoint{0.408550in}{1.687416in}}%
\pgfpathlineto{\pgfqpoint{0.411820in}{1.690941in}}%
\pgfpathlineto{\pgfqpoint{0.419865in}{1.699462in}}%
\pgfpathlineto{\pgfqpoint{0.423866in}{1.703615in}}%
\pgfpathlineto{\pgfqpoint{0.431609in}{1.711508in}}%
\pgfpathlineto{\pgfqpoint{0.435912in}{1.715811in}}%
\pgfpathlineto{\pgfqpoint{0.443805in}{1.723554in}}%
\pgfpathlineto{\pgfqpoint{0.447958in}{1.727554in}}%
\pgfpathlineto{\pgfqpoint{0.456479in}{1.735600in}}%
\pgfpathlineto{\pgfqpoint{0.460004in}{1.738870in}}%
\pgfpathlineto{\pgfqpoint{0.469663in}{1.747646in}}%
\pgfpathlineto{\pgfqpoint{0.472050in}{1.749779in}}%
\pgfpathlineto{\pgfqpoint{0.483387in}{1.759692in}}%
\pgfpathlineto{\pgfqpoint{0.484096in}{1.760303in}}%
\pgfpathlineto{\pgfqpoint{0.496143in}{1.770450in}}%
\pgfpathlineto{\pgfqpoint{0.497707in}{1.771738in}}%
\pgfpathlineto{\pgfqpoint{0.508189in}{1.780241in}}%
\pgfpathlineto{\pgfqpoint{0.512662in}{1.783785in}}%
\pgfpathlineto{\pgfqpoint{0.520235in}{1.789696in}}%
\pgfpathlineto{\pgfqpoint{0.528288in}{1.795831in}}%
\pgfpathlineto{\pgfqpoint{0.532281in}{1.798829in}}%
\pgfpathlineto{\pgfqpoint{0.544327in}{1.807652in}}%
\pgfpathlineto{\pgfqpoint{0.544641in}{1.807877in}}%
\pgfpathlineto{\pgfqpoint{0.556373in}{1.816160in}}%
\pgfpathlineto{\pgfqpoint{0.561847in}{1.819923in}}%
\pgfpathlineto{\pgfqpoint{0.568419in}{1.824383in}}%
\pgfpathlineto{\pgfqpoint{0.579913in}{1.831969in}}%
\pgfpathlineto{\pgfqpoint{0.580465in}{1.832330in}}%
\pgfpathlineto{\pgfqpoint{0.592511in}{1.839991in}}%
\pgfpathlineto{\pgfqpoint{0.599026in}{1.844015in}}%
\pgfpathlineto{\pgfqpoint{0.604558in}{1.847394in}}%
\pgfpathlineto{\pgfqpoint{0.616604in}{1.854540in}}%
\pgfpathlineto{\pgfqpoint{0.619242in}{1.856061in}}%
\pgfpathlineto{\pgfqpoint{0.628650in}{1.861430in}}%
\pgfpathlineto{\pgfqpoint{0.640696in}{1.868084in}}%
\pgfpathlineto{\pgfqpoint{0.640739in}{1.868107in}}%
\pgfpathlineto{\pgfqpoint{0.652742in}{1.874491in}}%
\pgfpathlineto{\pgfqpoint{0.663767in}{1.880154in}}%
\pgfpathlineto{\pgfqpoint{0.664788in}{1.880673in}}%
\pgfpathlineto{\pgfqpoint{0.676834in}{1.886622in}}%
\pgfpathlineto{\pgfqpoint{0.688550in}{1.892200in}}%
\pgfpathlineto{\pgfqpoint{0.688880in}{1.892356in}}%
\pgfpathlineto{\pgfqpoint{0.700927in}{1.897865in}}%
\pgfpathlineto{\pgfqpoint{0.712973in}{1.903167in}}%
\pgfpathlineto{\pgfqpoint{0.715512in}{1.904246in}}%
\pgfpathlineto{\pgfqpoint{0.725019in}{1.908256in}}%
\pgfpathlineto{\pgfqpoint{0.737065in}{1.913142in}}%
\pgfpathlineto{\pgfqpoint{0.745145in}{1.916292in}}%
\pgfpathlineto{\pgfqpoint{0.749111in}{1.917828in}}%
\pgfpathlineto{\pgfqpoint{0.761157in}{1.922312in}}%
\pgfpathlineto{\pgfqpoint{0.773203in}{1.926605in}}%
\pgfpathlineto{\pgfqpoint{0.778279in}{1.928338in}}%
\pgfpathlineto{\pgfqpoint{0.785249in}{1.930704in}}%
\pgfpathlineto{\pgfqpoint{0.797295in}{1.934614in}}%
\pgfpathlineto{\pgfqpoint{0.809342in}{1.938338in}}%
\pgfpathlineto{\pgfqpoint{0.816291in}{1.940384in}}%
\pgfpathlineto{\pgfqpoint{0.821388in}{1.941877in}}%
\pgfpathlineto{\pgfqpoint{0.833434in}{1.945234in}}%
\pgfpathlineto{\pgfqpoint{0.845480in}{1.948411in}}%
\pgfpathlineto{\pgfqpoint{0.857526in}{1.951412in}}%
\pgfpathlineto{\pgfqpoint{0.861862in}{1.952430in}}%
\pgfpathlineto{\pgfqpoint{0.869572in}{1.954235in}}%
\pgfpathlineto{\pgfqpoint{0.881618in}{1.956883in}}%
\pgfpathlineto{\pgfqpoint{0.893664in}{1.959360in}}%
\pgfpathlineto{\pgfqpoint{0.905710in}{1.961665in}}%
\pgfpathlineto{\pgfqpoint{0.917757in}{1.963800in}}%
\pgfpathlineto{\pgfqpoint{0.921894in}{1.964476in}}%
\pgfpathlineto{\pgfqpoint{0.929803in}{1.965766in}}%
\pgfpathlineto{\pgfqpoint{0.941849in}{1.967564in}}%
\pgfpathlineto{\pgfqpoint{0.953895in}{1.969196in}}%
\pgfpathlineto{\pgfqpoint{0.965941in}{1.970662in}}%
\pgfpathlineto{\pgfqpoint{0.977987in}{1.971963in}}%
\pgfpathlineto{\pgfqpoint{0.990033in}{1.973100in}}%
\pgfpathlineto{\pgfqpoint{1.002079in}{1.974073in}}%
\pgfpathlineto{\pgfqpoint{1.014126in}{1.974884in}}%
\pgfpathlineto{\pgfqpoint{1.026172in}{1.975532in}}%
\pgfpathlineto{\pgfqpoint{1.038218in}{1.976017in}}%
\pgfpathlineto{\pgfqpoint{1.050264in}{1.976341in}}%
\pgfpathlineto{\pgfqpoint{1.062310in}{1.976502in}}%
\pgfpathlineto{\pgfqpoint{1.074356in}{1.976502in}}%
\pgfpathlineto{\pgfqpoint{1.086402in}{1.976341in}}%
\pgfpathlineto{\pgfqpoint{1.098448in}{1.976017in}}%
\pgfpathlineto{\pgfqpoint{1.110494in}{1.975532in}}%
\pgfpathlineto{\pgfqpoint{1.122541in}{1.974884in}}%
\pgfpathlineto{\pgfqpoint{1.134587in}{1.974073in}}%
\pgfpathlineto{\pgfqpoint{1.146633in}{1.973100in}}%
\pgfpathlineto{\pgfqpoint{1.158679in}{1.971963in}}%
\pgfpathlineto{\pgfqpoint{1.170725in}{1.970662in}}%
\pgfpathlineto{\pgfqpoint{1.182771in}{1.969196in}}%
\pgfpathlineto{\pgfqpoint{1.194817in}{1.967564in}}%
\pgfpathlineto{\pgfqpoint{1.206863in}{1.965766in}}%
\pgfpathlineto{\pgfqpoint{1.214772in}{1.964476in}}%
\pgfpathlineto{\pgfqpoint{1.218909in}{1.963800in}}%
\pgfpathlineto{\pgfqpoint{1.230956in}{1.961665in}}%
\pgfpathlineto{\pgfqpoint{1.243002in}{1.959360in}}%
\pgfpathlineto{\pgfqpoint{1.255048in}{1.956883in}}%
\pgfpathlineto{\pgfqpoint{1.267094in}{1.954235in}}%
\pgfpathlineto{\pgfqpoint{1.274804in}{1.952430in}}%
\pgfpathlineto{\pgfqpoint{1.279140in}{1.951412in}}%
\pgfpathlineto{\pgfqpoint{1.291186in}{1.948411in}}%
\pgfpathlineto{\pgfqpoint{1.303232in}{1.945234in}}%
\pgfpathlineto{\pgfqpoint{1.315278in}{1.941877in}}%
\pgfpathlineto{\pgfqpoint{1.320375in}{1.940384in}}%
\pgfpathlineto{\pgfqpoint{1.327325in}{1.938338in}}%
\pgfpathlineto{\pgfqpoint{1.339371in}{1.934614in}}%
\pgfpathlineto{\pgfqpoint{1.351417in}{1.930704in}}%
\pgfpathlineto{\pgfqpoint{1.358387in}{1.928338in}}%
\pgfpathlineto{\pgfqpoint{1.363463in}{1.926605in}}%
\pgfpathlineto{\pgfqpoint{1.375509in}{1.922312in}}%
\pgfpathlineto{\pgfqpoint{1.387555in}{1.917828in}}%
\pgfpathlineto{\pgfqpoint{1.391521in}{1.916292in}}%
\pgfpathlineto{\pgfqpoint{1.399601in}{1.913142in}}%
\pgfpathlineto{\pgfqpoint{1.411647in}{1.908256in}}%
\pgfpathlineto{\pgfqpoint{1.421154in}{1.904246in}}%
\pgfpathlineto{\pgfqpoint{1.423693in}{1.903167in}}%
\pgfpathlineto{\pgfqpoint{1.435740in}{1.897865in}}%
\pgfpathlineto{\pgfqpoint{1.447786in}{1.892356in}}%
\pgfpathlineto{\pgfqpoint{1.448116in}{1.892200in}}%
\pgfpathlineto{\pgfqpoint{1.459832in}{1.886622in}}%
\pgfpathlineto{\pgfqpoint{1.471878in}{1.880673in}}%
\pgfpathlineto{\pgfqpoint{1.472899in}{1.880154in}}%
\pgfpathlineto{\pgfqpoint{1.483924in}{1.874491in}}%
\pgfpathlineto{\pgfqpoint{1.495927in}{1.868107in}}%
\pgfpathlineto{\pgfqpoint{1.495970in}{1.868084in}}%
\pgfpathlineto{\pgfqpoint{1.508016in}{1.861430in}}%
\pgfpathlineto{\pgfqpoint{1.517424in}{1.856061in}}%
\pgfpathlineto{\pgfqpoint{1.520062in}{1.854540in}}%
\pgfpathlineto{\pgfqpoint{1.532108in}{1.847394in}}%
\pgfpathlineto{\pgfqpoint{1.537640in}{1.844015in}}%
\pgfpathlineto{\pgfqpoint{1.544155in}{1.839991in}}%
\pgfpathlineto{\pgfqpoint{1.556201in}{1.832330in}}%
\pgfpathlineto{\pgfqpoint{1.556754in}{1.831969in}}%
\pgfpathlineto{\pgfqpoint{1.568247in}{1.824383in}}%
\pgfpathlineto{\pgfqpoint{1.574819in}{1.819923in}}%
\pgfpathlineto{\pgfqpoint{1.580293in}{1.816160in}}%
\pgfpathlineto{\pgfqpoint{1.592025in}{1.807877in}}%
\pgfpathlineto{\pgfqpoint{1.592339in}{1.807652in}}%
\pgfpathlineto{\pgfqpoint{1.604385in}{1.798829in}}%
\pgfpathlineto{\pgfqpoint{1.608378in}{1.795831in}}%
\pgfpathlineto{\pgfqpoint{1.616431in}{1.789696in}}%
\pgfpathlineto{\pgfqpoint{1.624005in}{1.783785in}}%
\pgfpathlineto{\pgfqpoint{1.628477in}{1.780241in}}%
\pgfpathlineto{\pgfqpoint{1.638959in}{1.771738in}}%
\pgfpathlineto{\pgfqpoint{1.640524in}{1.770450in}}%
\pgfpathlineto{\pgfqpoint{1.652570in}{1.760303in}}%
\pgfpathlineto{\pgfqpoint{1.653279in}{1.759692in}}%
\pgfpathlineto{\pgfqpoint{1.664616in}{1.749779in}}%
\pgfpathlineto{\pgfqpoint{1.667003in}{1.747646in}}%
\pgfpathlineto{\pgfqpoint{1.676662in}{1.738870in}}%
\pgfpathlineto{\pgfqpoint{1.680187in}{1.735600in}}%
\pgfpathlineto{\pgfqpoint{1.688708in}{1.727554in}}%
\pgfpathlineto{\pgfqpoint{1.692862in}{1.723554in}}%
\pgfpathlineto{\pgfqpoint{1.700754in}{1.715811in}}%
\pgfpathlineto{\pgfqpoint{1.705057in}{1.711508in}}%
\pgfpathlineto{\pgfqpoint{1.712800in}{1.703615in}}%
\pgfpathlineto{\pgfqpoint{1.716801in}{1.699462in}}%
\pgfpathlineto{\pgfqpoint{1.724846in}{1.690941in}}%
\pgfpathlineto{\pgfqpoint{1.728116in}{1.687416in}}%
\pgfpathlineto{\pgfqpoint{1.736892in}{1.677757in}}%
\pgfpathlineto{\pgfqpoint{1.739025in}{1.675370in}}%
\pgfpathlineto{\pgfqpoint{1.748939in}{1.664033in}}%
\pgfpathlineto{\pgfqpoint{1.749549in}{1.663323in}}%
\pgfpathlineto{\pgfqpoint{1.759696in}{1.651277in}}%
\pgfpathlineto{\pgfqpoint{1.760985in}{1.649713in}}%
\pgfpathlineto{\pgfqpoint{1.769487in}{1.639231in}}%
\pgfpathlineto{\pgfqpoint{1.773031in}{1.634758in}}%
\pgfpathlineto{\pgfqpoint{1.778942in}{1.627185in}}%
\pgfpathlineto{\pgfqpoint{1.785077in}{1.619132in}}%
\pgfpathlineto{\pgfqpoint{1.788075in}{1.615139in}}%
\pgfpathlineto{\pgfqpoint{1.796898in}{1.603093in}}%
\pgfpathlineto{\pgfqpoint{1.797123in}{1.602779in}}%
\pgfpathlineto{\pgfqpoint{1.805407in}{1.591047in}}%
\pgfpathlineto{\pgfqpoint{1.809169in}{1.585573in}}%
\pgfpathlineto{\pgfqpoint{1.813629in}{1.579001in}}%
\pgfpathlineto{\pgfqpoint{1.821215in}{1.567507in}}%
\pgfpathlineto{\pgfqpoint{1.821576in}{1.566955in}}%
\pgfpathlineto{\pgfqpoint{1.829237in}{1.554908in}}%
\pgfpathlineto{\pgfqpoint{1.833261in}{1.548394in}}%
\pgfpathlineto{\pgfqpoint{1.836640in}{1.542862in}}%
\pgfpathlineto{\pgfqpoint{1.843786in}{1.530816in}}%
\pgfpathlineto{\pgfqpoint{1.845307in}{1.528178in}}%
\pgfpathlineto{\pgfqpoint{1.850676in}{1.518770in}}%
\pgfpathlineto{\pgfqpoint{1.857331in}{1.506724in}}%
\pgfpathlineto{\pgfqpoint{1.857354in}{1.506681in}}%
\pgfpathlineto{\pgfqpoint{1.863737in}{1.494678in}}%
\pgfpathlineto{\pgfqpoint{1.869400in}{1.483653in}}%
\pgfpathlineto{\pgfqpoint{1.869920in}{1.482632in}}%
\pgfpathlineto{\pgfqpoint{1.875868in}{1.470586in}}%
\pgfpathlineto{\pgfqpoint{1.881446in}{1.458870in}}%
\pgfpathlineto{\pgfqpoint{1.881602in}{1.458539in}}%
\pgfpathlineto{\pgfqpoint{1.887111in}{1.446493in}}%
\pgfpathlineto{\pgfqpoint{1.892413in}{1.434447in}}%
\pgfpathlineto{\pgfqpoint{1.893492in}{1.431908in}}%
\pgfpathlineto{\pgfqpoint{1.897502in}{1.422401in}}%
\pgfpathlineto{\pgfqpoint{1.902388in}{1.410355in}}%
\pgfpathlineto{\pgfqpoint{1.905538in}{1.402275in}}%
\pgfpathlineto{\pgfqpoint{1.907074in}{1.398309in}}%
\pgfpathlineto{\pgfqpoint{1.911559in}{1.386263in}}%
\pgfpathlineto{\pgfqpoint{1.915851in}{1.374217in}}%
\pgfpathlineto{\pgfqpoint{1.917584in}{1.369141in}}%
\pgfpathlineto{\pgfqpoint{1.919950in}{1.362171in}}%
\pgfpathlineto{\pgfqpoint{1.923860in}{1.350124in}}%
\pgfpathlineto{\pgfqpoint{1.927584in}{1.338078in}}%
\pgfpathlineto{\pgfqpoint{1.929630in}{1.331129in}}%
\pgfpathlineto{\pgfqpoint{1.931124in}{1.326032in}}%
\pgfpathlineto{\pgfqpoint{1.934480in}{1.313986in}}%
\pgfpathlineto{\pgfqpoint{1.937658in}{1.301940in}}%
\pgfpathlineto{\pgfqpoint{1.940658in}{1.289894in}}%
\pgfpathlineto{\pgfqpoint{1.941676in}{1.285558in}}%
\pgfpathlineto{\pgfqpoint{1.943481in}{1.277848in}}%
\pgfpathlineto{\pgfqpoint{1.946130in}{1.265802in}}%
\pgfpathlineto{\pgfqpoint{1.948606in}{1.253756in}}%
\pgfpathlineto{\pgfqpoint{1.950911in}{1.241709in}}%
\pgfpathlineto{\pgfqpoint{1.953046in}{1.229663in}}%
\pgfpathlineto{\pgfqpoint{1.953723in}{1.225525in}}%
\pgfpathlineto{\pgfqpoint{1.955012in}{1.217617in}}%
\pgfpathlineto{\pgfqpoint{1.956810in}{1.205571in}}%
\pgfpathlineto{\pgfqpoint{1.958442in}{1.193525in}}%
\pgfpathlineto{\pgfqpoint{1.959908in}{1.181479in}}%
\pgfpathlineto{\pgfqpoint{1.961209in}{1.169433in}}%
\pgfpathlineto{\pgfqpoint{1.962346in}{1.157387in}}%
\pgfpathlineto{\pgfqpoint{1.963320in}{1.145340in}}%
\pgfpathlineto{\pgfqpoint{1.964130in}{1.133294in}}%
\pgfpathlineto{\pgfqpoint{1.964778in}{1.121248in}}%
\pgfpathlineto{\pgfqpoint{1.965263in}{1.109202in}}%
\pgfpathlineto{\pgfqpoint{1.965587in}{1.097156in}}%
\pgfpathlineto{\pgfqpoint{1.965748in}{1.085110in}}%
\pgfpathlineto{\pgfqpoint{1.965748in}{1.073064in}}%
\pgfpathlineto{\pgfqpoint{1.965587in}{1.061018in}}%
\pgfpathlineto{\pgfqpoint{1.965263in}{1.048972in}}%
\pgfpathlineto{\pgfqpoint{1.964778in}{1.036925in}}%
\pgfpathlineto{\pgfqpoint{1.964130in}{1.024879in}}%
\pgfpathlineto{\pgfqpoint{1.963320in}{1.012833in}}%
\pgfpathlineto{\pgfqpoint{1.962346in}{1.000787in}}%
\pgfpathlineto{\pgfqpoint{1.961209in}{0.988741in}}%
\pgfpathlineto{\pgfqpoint{1.959908in}{0.976695in}}%
\pgfpathlineto{\pgfqpoint{1.958442in}{0.964649in}}%
\pgfpathlineto{\pgfqpoint{1.956810in}{0.952603in}}%
\pgfpathlineto{\pgfqpoint{1.955012in}{0.940557in}}%
\pgfpathlineto{\pgfqpoint{1.953723in}{0.932648in}}%
\pgfpathlineto{\pgfqpoint{1.953046in}{0.928510in}}%
\pgfpathlineto{\pgfqpoint{1.950911in}{0.916464in}}%
\pgfpathlineto{\pgfqpoint{1.948606in}{0.904418in}}%
\pgfpathlineto{\pgfqpoint{1.946130in}{0.892372in}}%
\pgfpathlineto{\pgfqpoint{1.943481in}{0.880326in}}%
\pgfpathlineto{\pgfqpoint{1.941676in}{0.872616in}}%
\pgfpathlineto{\pgfqpoint{1.940658in}{0.868280in}}%
\pgfpathlineto{\pgfqpoint{1.937658in}{0.856234in}}%
\pgfpathlineto{\pgfqpoint{1.934480in}{0.844188in}}%
\pgfpathlineto{\pgfqpoint{1.931124in}{0.832141in}}%
\pgfpathlineto{\pgfqpoint{1.929630in}{0.827045in}}%
\pgfpathlineto{\pgfqpoint{1.927584in}{0.820095in}}%
\pgfpathlineto{\pgfqpoint{1.923860in}{0.808049in}}%
\pgfpathlineto{\pgfqpoint{1.919950in}{0.796003in}}%
\pgfpathlineto{\pgfqpoint{1.917584in}{0.789033in}}%
\pgfpathlineto{\pgfqpoint{1.915851in}{0.783957in}}%
\pgfpathlineto{\pgfqpoint{1.911559in}{0.771911in}}%
\pgfpathlineto{\pgfqpoint{1.907074in}{0.759865in}}%
\pgfpathlineto{\pgfqpoint{1.905538in}{0.755899in}}%
\pgfpathlineto{\pgfqpoint{1.902388in}{0.747819in}}%
\pgfpathlineto{\pgfqpoint{1.897502in}{0.735773in}}%
\pgfpathlineto{\pgfqpoint{1.893492in}{0.726265in}}%
\pgfpathlineto{\pgfqpoint{1.892413in}{0.723726in}}%
\pgfpathlineto{\pgfqpoint{1.887111in}{0.711680in}}%
\pgfpathlineto{\pgfqpoint{1.881602in}{0.699634in}}%
\pgfpathlineto{\pgfqpoint{1.881446in}{0.699304in}}%
\pgfpathlineto{\pgfqpoint{1.875868in}{0.687588in}}%
\pgfpathlineto{\pgfqpoint{1.869920in}{0.675542in}}%
\pgfpathlineto{\pgfqpoint{1.869400in}{0.674521in}}%
\pgfpathlineto{\pgfqpoint{1.863737in}{0.663496in}}%
\pgfpathlineto{\pgfqpoint{1.857354in}{0.651493in}}%
\pgfpathlineto{\pgfqpoint{1.857331in}{0.651450in}}%
\pgfpathlineto{\pgfqpoint{1.850676in}{0.639404in}}%
\pgfpathlineto{\pgfqpoint{1.845307in}{0.629996in}}%
\pgfpathlineto{\pgfqpoint{1.843786in}{0.627358in}}%
\pgfpathlineto{\pgfqpoint{1.836640in}{0.615311in}}%
\pgfpathlineto{\pgfqpoint{1.833261in}{0.609780in}}%
\pgfpathlineto{\pgfqpoint{1.829237in}{0.603265in}}%
\pgfpathlineto{\pgfqpoint{1.821576in}{0.591219in}}%
\pgfpathlineto{\pgfqpoint{1.821215in}{0.590666in}}%
\pgfpathlineto{\pgfqpoint{1.813629in}{0.579173in}}%
\pgfpathlineto{\pgfqpoint{1.809169in}{0.572601in}}%
\pgfpathlineto{\pgfqpoint{1.805407in}{0.567127in}}%
\pgfpathlineto{\pgfqpoint{1.797123in}{0.555395in}}%
\pgfpathlineto{\pgfqpoint{1.796898in}{0.555081in}}%
\pgfpathlineto{\pgfqpoint{1.788075in}{0.543035in}}%
\pgfpathlineto{\pgfqpoint{1.785077in}{0.539042in}}%
\pgfpathlineto{\pgfqpoint{1.778942in}{0.530989in}}%
\pgfpathlineto{\pgfqpoint{1.773031in}{0.523415in}}%
\pgfpathlineto{\pgfqpoint{1.769487in}{0.518942in}}%
\pgfpathlineto{\pgfqpoint{1.760985in}{0.508460in}}%
\pgfpathlineto{\pgfqpoint{1.759696in}{0.506896in}}%
\pgfpathlineto{\pgfqpoint{1.749549in}{0.494850in}}%
\pgfpathlineto{\pgfqpoint{1.748939in}{0.494141in}}%
\pgfpathlineto{\pgfqpoint{1.739025in}{0.482804in}}%
\pgfpathlineto{\pgfqpoint{1.736892in}{0.480416in}}%
\pgfpathlineto{\pgfqpoint{1.728116in}{0.470758in}}%
\pgfpathlineto{\pgfqpoint{1.724846in}{0.467233in}}%
\pgfpathlineto{\pgfqpoint{1.716801in}{0.458712in}}%
\pgfpathlineto{\pgfqpoint{1.712800in}{0.454558in}}%
\pgfpathlineto{\pgfqpoint{1.705057in}{0.446666in}}%
\pgfpathlineto{\pgfqpoint{1.700754in}{0.442363in}}%
\pgfpathlineto{\pgfqpoint{1.692862in}{0.434620in}}%
\pgfpathlineto{\pgfqpoint{1.688708in}{0.430619in}}%
\pgfpathlineto{\pgfqpoint{1.680187in}{0.422574in}}%
\pgfpathlineto{\pgfqpoint{1.676662in}{0.419304in}}%
\pgfpathlineto{\pgfqpoint{1.667003in}{0.410527in}}%
\pgfpathlineto{\pgfqpoint{1.664616in}{0.408394in}}%
\pgfpathlineto{\pgfqpoint{1.653279in}{0.398481in}}%
\pgfpathlineto{\pgfqpoint{1.652570in}{0.397871in}}%
\pgfpathlineto{\pgfqpoint{1.640524in}{0.387724in}}%
\pgfpathlineto{\pgfqpoint{1.638959in}{0.386435in}}%
\pgfpathlineto{\pgfqpoint{1.628477in}{0.377933in}}%
\pgfpathlineto{\pgfqpoint{1.624005in}{0.374389in}}%
\pgfpathlineto{\pgfqpoint{1.616431in}{0.368478in}}%
\pgfpathlineto{\pgfqpoint{1.608378in}{0.362343in}}%
\pgfpathlineto{\pgfqpoint{1.604385in}{0.359344in}}%
\pgfpathlineto{\pgfqpoint{1.592339in}{0.350522in}}%
\pgfpathlineto{\pgfqpoint{1.592025in}{0.350297in}}%
\pgfpathlineto{\pgfqpoint{1.580293in}{0.342013in}}%
\pgfpathlineto{\pgfqpoint{1.574819in}{0.338251in}}%
\pgfpathlineto{\pgfqpoint{1.568247in}{0.333791in}}%
\pgfpathlineto{\pgfqpoint{1.556754in}{0.326205in}}%
\pgfpathlineto{\pgfqpoint{1.556201in}{0.325844in}}%
\pgfpathlineto{\pgfqpoint{1.544155in}{0.318183in}}%
\pgfpathlineto{\pgfqpoint{1.537640in}{0.314159in}}%
\pgfpathlineto{\pgfqpoint{1.532108in}{0.310780in}}%
\pgfpathlineto{\pgfqpoint{1.520062in}{0.303634in}}%
\pgfpathlineto{\pgfqpoint{1.517424in}{0.302112in}}%
\pgfpathlineto{\pgfqpoint{1.508016in}{0.296744in}}%
\pgfpathlineto{\pgfqpoint{1.495970in}{0.290089in}}%
\pgfpathlineto{\pgfqpoint{1.495927in}{0.290066in}}%
\pgfpathlineto{\pgfqpoint{1.483924in}{0.283683in}}%
\pgfpathlineto{\pgfqpoint{1.472899in}{0.278020in}}%
\pgfpathlineto{\pgfqpoint{1.471878in}{0.277500in}}%
\pgfpathlineto{\pgfqpoint{1.459832in}{0.271552in}}%
\pgfpathlineto{\pgfqpoint{1.448116in}{0.265974in}}%
\pgfpathlineto{\pgfqpoint{1.447786in}{0.265818in}}%
\pgfpathlineto{\pgfqpoint{1.435740in}{0.260309in}}%
\pgfpathlineto{\pgfqpoint{1.423693in}{0.255007in}}%
\pgfpathlineto{\pgfqpoint{1.421154in}{0.253928in}}%
\pgfpathlineto{\pgfqpoint{1.411647in}{0.249918in}}%
\pgfpathlineto{\pgfqpoint{1.399601in}{0.245032in}}%
\pgfpathlineto{\pgfqpoint{1.391521in}{0.241882in}}%
\pgfpathlineto{\pgfqpoint{1.387555in}{0.240346in}}%
\pgfpathlineto{\pgfqpoint{1.375509in}{0.235861in}}%
\pgfpathlineto{\pgfqpoint{1.363463in}{0.231569in}}%
\pgfpathlineto{\pgfqpoint{1.358387in}{0.229836in}}%
\pgfpathlineto{\pgfqpoint{1.351417in}{0.227470in}}%
\pgfpathlineto{\pgfqpoint{1.339371in}{0.223560in}}%
\pgfpathlineto{\pgfqpoint{1.327325in}{0.219836in}}%
\pgfpathlineto{\pgfqpoint{1.320375in}{0.217790in}}%
\pgfpathlineto{\pgfqpoint{1.315278in}{0.216296in}}%
\pgfpathlineto{\pgfqpoint{1.303232in}{0.212940in}}%
\pgfpathlineto{\pgfqpoint{1.291186in}{0.209762in}}%
\pgfpathlineto{\pgfqpoint{1.279140in}{0.206762in}}%
\pgfpathlineto{\pgfqpoint{1.274804in}{0.205743in}}%
\pgfpathlineto{\pgfqpoint{1.267094in}{0.203939in}}%
\pgfpathlineto{\pgfqpoint{1.255048in}{0.201290in}}%
\pgfpathlineto{\pgfqpoint{1.243002in}{0.198814in}}%
\pgfpathlineto{\pgfqpoint{1.230956in}{0.196509in}}%
\pgfpathlineto{\pgfqpoint{1.218909in}{0.194374in}}%
\pgfpathlineto{\pgfqpoint{1.214772in}{0.193697in}}%
\pgfpathlineto{\pgfqpoint{1.206863in}{0.192408in}}%
\pgfpathlineto{\pgfqpoint{1.194817in}{0.190610in}}%
\pgfpathlineto{\pgfqpoint{1.182771in}{0.188978in}}%
\pgfpathlineto{\pgfqpoint{1.170725in}{0.187512in}}%
\pgfpathlineto{\pgfqpoint{1.158679in}{0.186211in}}%
\pgfpathlineto{\pgfqpoint{1.146633in}{0.185074in}}%
\pgfpathlineto{\pgfqpoint{1.134587in}{0.184100in}}%
\pgfpathlineto{\pgfqpoint{1.122541in}{0.183290in}}%
\pgfpathlineto{\pgfqpoint{1.110494in}{0.182642in}}%
\pgfpathlineto{\pgfqpoint{1.098448in}{0.182157in}}%
\pgfpathlineto{\pgfqpoint{1.086402in}{0.181833in}}%
\pgfpathlineto{\pgfqpoint{1.074356in}{0.181671in}}%
\pgfpathlineto{\pgfqpoint{1.062310in}{0.181671in}}%
\pgfpathlineto{\pgfqpoint{1.050264in}{0.181833in}}%
\pgfpathlineto{\pgfqpoint{1.038218in}{0.182157in}}%
\pgfpathlineto{\pgfqpoint{1.026172in}{0.182642in}}%
\pgfpathlineto{\pgfqpoint{1.014126in}{0.183290in}}%
\pgfpathlineto{\pgfqpoint{1.002079in}{0.184100in}}%
\pgfpathlineto{\pgfqpoint{0.990033in}{0.185074in}}%
\pgfpathlineto{\pgfqpoint{0.977987in}{0.186211in}}%
\pgfpathlineto{\pgfqpoint{0.965941in}{0.187512in}}%
\pgfpathlineto{\pgfqpoint{0.953895in}{0.188978in}}%
\pgfpathlineto{\pgfqpoint{0.941849in}{0.190610in}}%
\pgfpathlineto{\pgfqpoint{0.929803in}{0.192408in}}%
\pgfpathclose%
\pgfusepath{fill}%
\end{pgfscope}%
\begin{pgfscope}%
\pgfpathrectangle{\pgfqpoint{0.135000in}{0.145754in}}{\pgfqpoint{1.866666in}{1.866666in}} %
\pgfusepath{clip}%
\pgfsetbuttcap%
\pgfsetroundjoin%
\pgfsetlinewidth{0.000000pt}%
\definecolor{currentstroke}{rgb}{0.000000,0.000000,0.000000}%
\pgfsetstrokecolor{currentstroke}%
\pgfsetdash{}{0pt}%
\pgfpathmoveto{\pgfqpoint{0.182944in}{0.181651in}}%
\pgfpathlineto{\pgfqpoint{0.194990in}{0.181651in}}%
\pgfpathlineto{\pgfqpoint{0.207036in}{0.181651in}}%
\pgfpathlineto{\pgfqpoint{0.219082in}{0.181651in}}%
\pgfpathlineto{\pgfqpoint{0.231128in}{0.181651in}}%
\pgfpathlineto{\pgfqpoint{0.243174in}{0.181651in}}%
\pgfpathlineto{\pgfqpoint{0.255220in}{0.181651in}}%
\pgfpathlineto{\pgfqpoint{0.267266in}{0.181651in}}%
\pgfpathlineto{\pgfqpoint{0.279312in}{0.181651in}}%
\pgfpathlineto{\pgfqpoint{0.291359in}{0.181651in}}%
\pgfpathlineto{\pgfqpoint{0.303405in}{0.181651in}}%
\pgfpathlineto{\pgfqpoint{0.315451in}{0.181651in}}%
\pgfpathlineto{\pgfqpoint{0.327497in}{0.181651in}}%
\pgfpathlineto{\pgfqpoint{0.339543in}{0.181651in}}%
\pgfpathlineto{\pgfqpoint{0.351589in}{0.181651in}}%
\pgfpathlineto{\pgfqpoint{0.363635in}{0.181651in}}%
\pgfpathlineto{\pgfqpoint{0.375681in}{0.181651in}}%
\pgfpathlineto{\pgfqpoint{0.387728in}{0.181651in}}%
\pgfpathlineto{\pgfqpoint{0.399774in}{0.181651in}}%
\pgfpathlineto{\pgfqpoint{0.411820in}{0.181651in}}%
\pgfpathlineto{\pgfqpoint{0.423866in}{0.181651in}}%
\pgfpathlineto{\pgfqpoint{0.435912in}{0.181651in}}%
\pgfpathlineto{\pgfqpoint{0.447958in}{0.181651in}}%
\pgfpathlineto{\pgfqpoint{0.460004in}{0.181651in}}%
\pgfpathlineto{\pgfqpoint{0.472050in}{0.181651in}}%
\pgfpathlineto{\pgfqpoint{0.484096in}{0.181651in}}%
\pgfpathlineto{\pgfqpoint{0.496143in}{0.181651in}}%
\pgfpathlineto{\pgfqpoint{0.508189in}{0.181651in}}%
\pgfpathlineto{\pgfqpoint{0.520235in}{0.181651in}}%
\pgfpathlineto{\pgfqpoint{0.532281in}{0.181651in}}%
\pgfpathlineto{\pgfqpoint{0.544327in}{0.181651in}}%
\pgfpathlineto{\pgfqpoint{0.556373in}{0.181651in}}%
\pgfpathlineto{\pgfqpoint{0.568419in}{0.181651in}}%
\pgfpathlineto{\pgfqpoint{0.580465in}{0.181651in}}%
\pgfpathlineto{\pgfqpoint{0.592511in}{0.181651in}}%
\pgfpathlineto{\pgfqpoint{0.604558in}{0.181651in}}%
\pgfpathlineto{\pgfqpoint{0.616604in}{0.181651in}}%
\pgfpathlineto{\pgfqpoint{0.628650in}{0.181651in}}%
\pgfpathlineto{\pgfqpoint{0.640696in}{0.181651in}}%
\pgfpathlineto{\pgfqpoint{0.652742in}{0.181651in}}%
\pgfpathlineto{\pgfqpoint{0.664788in}{0.181651in}}%
\pgfpathlineto{\pgfqpoint{0.676834in}{0.181651in}}%
\pgfpathlineto{\pgfqpoint{0.688880in}{0.181651in}}%
\pgfpathlineto{\pgfqpoint{0.700927in}{0.181651in}}%
\pgfpathlineto{\pgfqpoint{0.712973in}{0.181651in}}%
\pgfpathlineto{\pgfqpoint{0.725019in}{0.181651in}}%
\pgfpathlineto{\pgfqpoint{0.737065in}{0.181651in}}%
\pgfpathlineto{\pgfqpoint{0.749111in}{0.181651in}}%
\pgfpathlineto{\pgfqpoint{0.761157in}{0.181651in}}%
\pgfpathlineto{\pgfqpoint{0.773203in}{0.181651in}}%
\pgfpathlineto{\pgfqpoint{0.785249in}{0.181651in}}%
\pgfpathlineto{\pgfqpoint{0.797295in}{0.181651in}}%
\pgfpathlineto{\pgfqpoint{0.809342in}{0.181651in}}%
\pgfpathlineto{\pgfqpoint{0.821388in}{0.181651in}}%
\pgfpathlineto{\pgfqpoint{0.833434in}{0.181651in}}%
\pgfpathlineto{\pgfqpoint{0.845480in}{0.181651in}}%
\pgfpathlineto{\pgfqpoint{0.857526in}{0.181651in}}%
\pgfpathlineto{\pgfqpoint{0.869572in}{0.181651in}}%
\pgfpathlineto{\pgfqpoint{0.881618in}{0.181651in}}%
\pgfpathlineto{\pgfqpoint{0.893664in}{0.181651in}}%
\pgfpathlineto{\pgfqpoint{0.905710in}{0.181651in}}%
\pgfpathlineto{\pgfqpoint{0.917757in}{0.181651in}}%
\pgfpathlineto{\pgfqpoint{0.929803in}{0.181651in}}%
\pgfpathlineto{\pgfqpoint{0.941849in}{0.181651in}}%
\pgfpathlineto{\pgfqpoint{0.953895in}{0.181651in}}%
\pgfpathlineto{\pgfqpoint{0.965941in}{0.181651in}}%
\pgfpathlineto{\pgfqpoint{0.977987in}{0.181651in}}%
\pgfpathlineto{\pgfqpoint{0.990033in}{0.181651in}}%
\pgfpathlineto{\pgfqpoint{1.002079in}{0.181651in}}%
\pgfpathlineto{\pgfqpoint{1.014126in}{0.181651in}}%
\pgfpathlineto{\pgfqpoint{1.026172in}{0.181651in}}%
\pgfpathlineto{\pgfqpoint{1.038218in}{0.181651in}}%
\pgfpathlineto{\pgfqpoint{1.050264in}{0.181651in}}%
\pgfpathlineto{\pgfqpoint{1.062310in}{0.181651in}}%
\pgfpathlineto{\pgfqpoint{1.074356in}{0.181651in}}%
\pgfpathlineto{\pgfqpoint{1.086402in}{0.181651in}}%
\pgfpathlineto{\pgfqpoint{1.098448in}{0.181651in}}%
\pgfpathlineto{\pgfqpoint{1.110494in}{0.181651in}}%
\pgfpathlineto{\pgfqpoint{1.122541in}{0.181651in}}%
\pgfpathlineto{\pgfqpoint{1.134587in}{0.181651in}}%
\pgfpathlineto{\pgfqpoint{1.146633in}{0.181651in}}%
\pgfpathlineto{\pgfqpoint{1.158679in}{0.181651in}}%
\pgfpathlineto{\pgfqpoint{1.170725in}{0.181651in}}%
\pgfpathlineto{\pgfqpoint{1.182771in}{0.181651in}}%
\pgfpathlineto{\pgfqpoint{1.194817in}{0.181651in}}%
\pgfpathlineto{\pgfqpoint{1.206863in}{0.181651in}}%
\pgfpathlineto{\pgfqpoint{1.218909in}{0.181651in}}%
\pgfpathlineto{\pgfqpoint{1.230956in}{0.181651in}}%
\pgfpathlineto{\pgfqpoint{1.243002in}{0.181651in}}%
\pgfpathlineto{\pgfqpoint{1.255048in}{0.181651in}}%
\pgfpathlineto{\pgfqpoint{1.267094in}{0.181651in}}%
\pgfpathlineto{\pgfqpoint{1.279140in}{0.181651in}}%
\pgfpathlineto{\pgfqpoint{1.291186in}{0.181651in}}%
\pgfpathlineto{\pgfqpoint{1.303232in}{0.181651in}}%
\pgfpathlineto{\pgfqpoint{1.315278in}{0.181651in}}%
\pgfpathlineto{\pgfqpoint{1.327325in}{0.181651in}}%
\pgfpathlineto{\pgfqpoint{1.339371in}{0.181651in}}%
\pgfpathlineto{\pgfqpoint{1.351417in}{0.181651in}}%
\pgfpathlineto{\pgfqpoint{1.363463in}{0.181651in}}%
\pgfpathlineto{\pgfqpoint{1.375509in}{0.181651in}}%
\pgfpathlineto{\pgfqpoint{1.387555in}{0.181651in}}%
\pgfpathlineto{\pgfqpoint{1.399601in}{0.181651in}}%
\pgfpathlineto{\pgfqpoint{1.411647in}{0.181651in}}%
\pgfpathlineto{\pgfqpoint{1.423693in}{0.181651in}}%
\pgfpathlineto{\pgfqpoint{1.435740in}{0.181651in}}%
\pgfpathlineto{\pgfqpoint{1.447786in}{0.181651in}}%
\pgfpathlineto{\pgfqpoint{1.459832in}{0.181651in}}%
\pgfpathlineto{\pgfqpoint{1.471878in}{0.181651in}}%
\pgfpathlineto{\pgfqpoint{1.483924in}{0.181651in}}%
\pgfpathlineto{\pgfqpoint{1.495970in}{0.181651in}}%
\pgfpathlineto{\pgfqpoint{1.508016in}{0.181651in}}%
\pgfpathlineto{\pgfqpoint{1.520062in}{0.181651in}}%
\pgfpathlineto{\pgfqpoint{1.532108in}{0.181651in}}%
\pgfpathlineto{\pgfqpoint{1.544155in}{0.181651in}}%
\pgfpathlineto{\pgfqpoint{1.556201in}{0.181651in}}%
\pgfpathlineto{\pgfqpoint{1.568247in}{0.181651in}}%
\pgfpathlineto{\pgfqpoint{1.580293in}{0.181651in}}%
\pgfpathlineto{\pgfqpoint{1.592339in}{0.181651in}}%
\pgfpathlineto{\pgfqpoint{1.604385in}{0.181651in}}%
\pgfpathlineto{\pgfqpoint{1.616431in}{0.181651in}}%
\pgfpathlineto{\pgfqpoint{1.628477in}{0.181651in}}%
\pgfpathlineto{\pgfqpoint{1.640524in}{0.181651in}}%
\pgfpathlineto{\pgfqpoint{1.652570in}{0.181651in}}%
\pgfpathlineto{\pgfqpoint{1.664616in}{0.181651in}}%
\pgfpathlineto{\pgfqpoint{1.676662in}{0.181651in}}%
\pgfpathlineto{\pgfqpoint{1.688708in}{0.181651in}}%
\pgfpathlineto{\pgfqpoint{1.700754in}{0.181651in}}%
\pgfpathlineto{\pgfqpoint{1.712800in}{0.181651in}}%
\pgfpathlineto{\pgfqpoint{1.724846in}{0.181651in}}%
\pgfpathlineto{\pgfqpoint{1.736892in}{0.181651in}}%
\pgfpathlineto{\pgfqpoint{1.748939in}{0.181651in}}%
\pgfpathlineto{\pgfqpoint{1.760985in}{0.181651in}}%
\pgfpathlineto{\pgfqpoint{1.773031in}{0.181651in}}%
\pgfpathlineto{\pgfqpoint{1.785077in}{0.181651in}}%
\pgfpathlineto{\pgfqpoint{1.797123in}{0.181651in}}%
\pgfpathlineto{\pgfqpoint{1.809169in}{0.181651in}}%
\pgfpathlineto{\pgfqpoint{1.821215in}{0.181651in}}%
\pgfpathlineto{\pgfqpoint{1.833261in}{0.181651in}}%
\pgfpathlineto{\pgfqpoint{1.845307in}{0.181651in}}%
\pgfpathlineto{\pgfqpoint{1.857354in}{0.181651in}}%
\pgfpathlineto{\pgfqpoint{1.869400in}{0.181651in}}%
\pgfpathlineto{\pgfqpoint{1.881446in}{0.181651in}}%
\pgfpathlineto{\pgfqpoint{1.893492in}{0.181651in}}%
\pgfpathlineto{\pgfqpoint{1.905538in}{0.181651in}}%
\pgfpathlineto{\pgfqpoint{1.917584in}{0.181651in}}%
\pgfpathlineto{\pgfqpoint{1.929630in}{0.181651in}}%
\pgfpathlineto{\pgfqpoint{1.941676in}{0.181651in}}%
\pgfpathlineto{\pgfqpoint{1.953723in}{0.181651in}}%
\pgfpathlineto{\pgfqpoint{1.965769in}{0.181651in}}%
\pgfpathlineto{\pgfqpoint{1.965769in}{0.193697in}}%
\pgfpathlineto{\pgfqpoint{1.965769in}{0.205743in}}%
\pgfpathlineto{\pgfqpoint{1.965769in}{0.217790in}}%
\pgfpathlineto{\pgfqpoint{1.965769in}{0.229836in}}%
\pgfpathlineto{\pgfqpoint{1.965769in}{0.241882in}}%
\pgfpathlineto{\pgfqpoint{1.965769in}{0.253928in}}%
\pgfpathlineto{\pgfqpoint{1.965769in}{0.265974in}}%
\pgfpathlineto{\pgfqpoint{1.965769in}{0.278020in}}%
\pgfpathlineto{\pgfqpoint{1.965769in}{0.290066in}}%
\pgfpathlineto{\pgfqpoint{1.965769in}{0.302112in}}%
\pgfpathlineto{\pgfqpoint{1.965769in}{0.314159in}}%
\pgfpathlineto{\pgfqpoint{1.965769in}{0.326205in}}%
\pgfpathlineto{\pgfqpoint{1.965769in}{0.338251in}}%
\pgfpathlineto{\pgfqpoint{1.965769in}{0.350297in}}%
\pgfpathlineto{\pgfqpoint{1.965769in}{0.362343in}}%
\pgfpathlineto{\pgfqpoint{1.965769in}{0.374389in}}%
\pgfpathlineto{\pgfqpoint{1.965769in}{0.386435in}}%
\pgfpathlineto{\pgfqpoint{1.965769in}{0.398481in}}%
\pgfpathlineto{\pgfqpoint{1.965769in}{0.410527in}}%
\pgfpathlineto{\pgfqpoint{1.965769in}{0.422574in}}%
\pgfpathlineto{\pgfqpoint{1.965769in}{0.434620in}}%
\pgfpathlineto{\pgfqpoint{1.965769in}{0.446666in}}%
\pgfpathlineto{\pgfqpoint{1.965769in}{0.458712in}}%
\pgfpathlineto{\pgfqpoint{1.965769in}{0.470758in}}%
\pgfpathlineto{\pgfqpoint{1.965769in}{0.482804in}}%
\pgfpathlineto{\pgfqpoint{1.965769in}{0.494850in}}%
\pgfpathlineto{\pgfqpoint{1.965769in}{0.506896in}}%
\pgfpathlineto{\pgfqpoint{1.965769in}{0.518942in}}%
\pgfpathlineto{\pgfqpoint{1.965769in}{0.530989in}}%
\pgfpathlineto{\pgfqpoint{1.965769in}{0.543035in}}%
\pgfpathlineto{\pgfqpoint{1.965769in}{0.555081in}}%
\pgfpathlineto{\pgfqpoint{1.965769in}{0.567127in}}%
\pgfpathlineto{\pgfqpoint{1.965769in}{0.579173in}}%
\pgfpathlineto{\pgfqpoint{1.965769in}{0.591219in}}%
\pgfpathlineto{\pgfqpoint{1.965769in}{0.603265in}}%
\pgfpathlineto{\pgfqpoint{1.965769in}{0.615311in}}%
\pgfpathlineto{\pgfqpoint{1.965769in}{0.627358in}}%
\pgfpathlineto{\pgfqpoint{1.965769in}{0.639404in}}%
\pgfpathlineto{\pgfqpoint{1.965769in}{0.651450in}}%
\pgfpathlineto{\pgfqpoint{1.965769in}{0.663496in}}%
\pgfpathlineto{\pgfqpoint{1.965769in}{0.675542in}}%
\pgfpathlineto{\pgfqpoint{1.965769in}{0.687588in}}%
\pgfpathlineto{\pgfqpoint{1.965769in}{0.699634in}}%
\pgfpathlineto{\pgfqpoint{1.965769in}{0.711680in}}%
\pgfpathlineto{\pgfqpoint{1.965769in}{0.723726in}}%
\pgfpathlineto{\pgfqpoint{1.965769in}{0.735773in}}%
\pgfpathlineto{\pgfqpoint{1.965769in}{0.747819in}}%
\pgfpathlineto{\pgfqpoint{1.965769in}{0.759865in}}%
\pgfpathlineto{\pgfqpoint{1.965769in}{0.771911in}}%
\pgfpathlineto{\pgfqpoint{1.965769in}{0.783957in}}%
\pgfpathlineto{\pgfqpoint{1.965769in}{0.796003in}}%
\pgfpathlineto{\pgfqpoint{1.965769in}{0.808049in}}%
\pgfpathlineto{\pgfqpoint{1.965769in}{0.820095in}}%
\pgfpathlineto{\pgfqpoint{1.965769in}{0.832141in}}%
\pgfpathlineto{\pgfqpoint{1.965769in}{0.844188in}}%
\pgfpathlineto{\pgfqpoint{1.965769in}{0.856234in}}%
\pgfpathlineto{\pgfqpoint{1.965769in}{0.868280in}}%
\pgfpathlineto{\pgfqpoint{1.965769in}{0.880326in}}%
\pgfpathlineto{\pgfqpoint{1.965769in}{0.892372in}}%
\pgfpathlineto{\pgfqpoint{1.965769in}{0.904418in}}%
\pgfpathlineto{\pgfqpoint{1.965769in}{0.916464in}}%
\pgfpathlineto{\pgfqpoint{1.965769in}{0.928510in}}%
\pgfpathlineto{\pgfqpoint{1.965769in}{0.940557in}}%
\pgfpathlineto{\pgfqpoint{1.965769in}{0.952603in}}%
\pgfpathlineto{\pgfqpoint{1.965769in}{0.964649in}}%
\pgfpathlineto{\pgfqpoint{1.965769in}{0.976695in}}%
\pgfpathlineto{\pgfqpoint{1.965769in}{0.988741in}}%
\pgfpathlineto{\pgfqpoint{1.965769in}{1.000787in}}%
\pgfpathlineto{\pgfqpoint{1.965769in}{1.012833in}}%
\pgfpathlineto{\pgfqpoint{1.965769in}{1.024879in}}%
\pgfpathlineto{\pgfqpoint{1.965769in}{1.036925in}}%
\pgfpathlineto{\pgfqpoint{1.965769in}{1.048972in}}%
\pgfpathlineto{\pgfqpoint{1.965769in}{1.061018in}}%
\pgfpathlineto{\pgfqpoint{1.965769in}{1.073064in}}%
\pgfpathlineto{\pgfqpoint{1.965769in}{1.085110in}}%
\pgfpathlineto{\pgfqpoint{1.965769in}{1.097156in}}%
\pgfpathlineto{\pgfqpoint{1.965769in}{1.109202in}}%
\pgfpathlineto{\pgfqpoint{1.965769in}{1.121248in}}%
\pgfpathlineto{\pgfqpoint{1.965769in}{1.133294in}}%
\pgfpathlineto{\pgfqpoint{1.965769in}{1.145340in}}%
\pgfpathlineto{\pgfqpoint{1.965769in}{1.157387in}}%
\pgfpathlineto{\pgfqpoint{1.965769in}{1.169433in}}%
\pgfpathlineto{\pgfqpoint{1.965769in}{1.181479in}}%
\pgfpathlineto{\pgfqpoint{1.965769in}{1.193525in}}%
\pgfpathlineto{\pgfqpoint{1.965769in}{1.205571in}}%
\pgfpathlineto{\pgfqpoint{1.965769in}{1.217617in}}%
\pgfpathlineto{\pgfqpoint{1.965769in}{1.229663in}}%
\pgfpathlineto{\pgfqpoint{1.965769in}{1.241709in}}%
\pgfpathlineto{\pgfqpoint{1.965769in}{1.253756in}}%
\pgfpathlineto{\pgfqpoint{1.965769in}{1.265802in}}%
\pgfpathlineto{\pgfqpoint{1.965769in}{1.277848in}}%
\pgfpathlineto{\pgfqpoint{1.965769in}{1.289894in}}%
\pgfpathlineto{\pgfqpoint{1.965769in}{1.301940in}}%
\pgfpathlineto{\pgfqpoint{1.965769in}{1.313986in}}%
\pgfpathlineto{\pgfqpoint{1.965769in}{1.326032in}}%
\pgfpathlineto{\pgfqpoint{1.965769in}{1.338078in}}%
\pgfpathlineto{\pgfqpoint{1.965769in}{1.350124in}}%
\pgfpathlineto{\pgfqpoint{1.965769in}{1.362171in}}%
\pgfpathlineto{\pgfqpoint{1.965769in}{1.374217in}}%
\pgfpathlineto{\pgfqpoint{1.965769in}{1.386263in}}%
\pgfpathlineto{\pgfqpoint{1.965769in}{1.398309in}}%
\pgfpathlineto{\pgfqpoint{1.965769in}{1.410355in}}%
\pgfpathlineto{\pgfqpoint{1.965769in}{1.422401in}}%
\pgfpathlineto{\pgfqpoint{1.965769in}{1.434447in}}%
\pgfpathlineto{\pgfqpoint{1.965769in}{1.446493in}}%
\pgfpathlineto{\pgfqpoint{1.965769in}{1.458539in}}%
\pgfpathlineto{\pgfqpoint{1.965769in}{1.470586in}}%
\pgfpathlineto{\pgfqpoint{1.965769in}{1.482632in}}%
\pgfpathlineto{\pgfqpoint{1.965769in}{1.494678in}}%
\pgfpathlineto{\pgfqpoint{1.965769in}{1.506724in}}%
\pgfpathlineto{\pgfqpoint{1.965769in}{1.518770in}}%
\pgfpathlineto{\pgfqpoint{1.965769in}{1.530816in}}%
\pgfpathlineto{\pgfqpoint{1.965769in}{1.542862in}}%
\pgfpathlineto{\pgfqpoint{1.965769in}{1.554908in}}%
\pgfpathlineto{\pgfqpoint{1.965769in}{1.566955in}}%
\pgfpathlineto{\pgfqpoint{1.965769in}{1.579001in}}%
\pgfpathlineto{\pgfqpoint{1.965769in}{1.591047in}}%
\pgfpathlineto{\pgfqpoint{1.965769in}{1.603093in}}%
\pgfpathlineto{\pgfqpoint{1.965769in}{1.615139in}}%
\pgfpathlineto{\pgfqpoint{1.965769in}{1.627185in}}%
\pgfpathlineto{\pgfqpoint{1.965769in}{1.639231in}}%
\pgfpathlineto{\pgfqpoint{1.965769in}{1.651277in}}%
\pgfpathlineto{\pgfqpoint{1.965769in}{1.663323in}}%
\pgfpathlineto{\pgfqpoint{1.965769in}{1.675370in}}%
\pgfpathlineto{\pgfqpoint{1.965769in}{1.687416in}}%
\pgfpathlineto{\pgfqpoint{1.965769in}{1.699462in}}%
\pgfpathlineto{\pgfqpoint{1.965769in}{1.711508in}}%
\pgfpathlineto{\pgfqpoint{1.965769in}{1.723554in}}%
\pgfpathlineto{\pgfqpoint{1.965769in}{1.735600in}}%
\pgfpathlineto{\pgfqpoint{1.965769in}{1.747646in}}%
\pgfpathlineto{\pgfqpoint{1.965769in}{1.759692in}}%
\pgfpathlineto{\pgfqpoint{1.965769in}{1.771738in}}%
\pgfpathlineto{\pgfqpoint{1.965769in}{1.783785in}}%
\pgfpathlineto{\pgfqpoint{1.965769in}{1.795831in}}%
\pgfpathlineto{\pgfqpoint{1.965769in}{1.807877in}}%
\pgfpathlineto{\pgfqpoint{1.965769in}{1.819923in}}%
\pgfpathlineto{\pgfqpoint{1.965769in}{1.831969in}}%
\pgfpathlineto{\pgfqpoint{1.965769in}{1.844015in}}%
\pgfpathlineto{\pgfqpoint{1.965769in}{1.856061in}}%
\pgfpathlineto{\pgfqpoint{1.965769in}{1.868107in}}%
\pgfpathlineto{\pgfqpoint{1.965769in}{1.880154in}}%
\pgfpathlineto{\pgfqpoint{1.965769in}{1.892200in}}%
\pgfpathlineto{\pgfqpoint{1.965769in}{1.904246in}}%
\pgfpathlineto{\pgfqpoint{1.965769in}{1.916292in}}%
\pgfpathlineto{\pgfqpoint{1.965769in}{1.928338in}}%
\pgfpathlineto{\pgfqpoint{1.965769in}{1.940384in}}%
\pgfpathlineto{\pgfqpoint{1.965769in}{1.952430in}}%
\pgfpathlineto{\pgfqpoint{1.965769in}{1.964476in}}%
\pgfpathlineto{\pgfqpoint{1.965769in}{1.976522in}}%
\pgfpathlineto{\pgfqpoint{1.953723in}{1.976522in}}%
\pgfpathlineto{\pgfqpoint{1.941676in}{1.976522in}}%
\pgfpathlineto{\pgfqpoint{1.929630in}{1.976522in}}%
\pgfpathlineto{\pgfqpoint{1.917584in}{1.976522in}}%
\pgfpathlineto{\pgfqpoint{1.905538in}{1.976522in}}%
\pgfpathlineto{\pgfqpoint{1.893492in}{1.976522in}}%
\pgfpathlineto{\pgfqpoint{1.881446in}{1.976522in}}%
\pgfpathlineto{\pgfqpoint{1.869400in}{1.976522in}}%
\pgfpathlineto{\pgfqpoint{1.857354in}{1.976522in}}%
\pgfpathlineto{\pgfqpoint{1.845307in}{1.976522in}}%
\pgfpathlineto{\pgfqpoint{1.833261in}{1.976522in}}%
\pgfpathlineto{\pgfqpoint{1.821215in}{1.976522in}}%
\pgfpathlineto{\pgfqpoint{1.809169in}{1.976522in}}%
\pgfpathlineto{\pgfqpoint{1.797123in}{1.976522in}}%
\pgfpathlineto{\pgfqpoint{1.785077in}{1.976522in}}%
\pgfpathlineto{\pgfqpoint{1.773031in}{1.976522in}}%
\pgfpathlineto{\pgfqpoint{1.760985in}{1.976522in}}%
\pgfpathlineto{\pgfqpoint{1.748939in}{1.976522in}}%
\pgfpathlineto{\pgfqpoint{1.736892in}{1.976522in}}%
\pgfpathlineto{\pgfqpoint{1.724846in}{1.976522in}}%
\pgfpathlineto{\pgfqpoint{1.712800in}{1.976522in}}%
\pgfpathlineto{\pgfqpoint{1.700754in}{1.976522in}}%
\pgfpathlineto{\pgfqpoint{1.688708in}{1.976522in}}%
\pgfpathlineto{\pgfqpoint{1.676662in}{1.976522in}}%
\pgfpathlineto{\pgfqpoint{1.664616in}{1.976522in}}%
\pgfpathlineto{\pgfqpoint{1.652570in}{1.976522in}}%
\pgfpathlineto{\pgfqpoint{1.640524in}{1.976522in}}%
\pgfpathlineto{\pgfqpoint{1.628477in}{1.976522in}}%
\pgfpathlineto{\pgfqpoint{1.616431in}{1.976522in}}%
\pgfpathlineto{\pgfqpoint{1.604385in}{1.976522in}}%
\pgfpathlineto{\pgfqpoint{1.592339in}{1.976522in}}%
\pgfpathlineto{\pgfqpoint{1.580293in}{1.976522in}}%
\pgfpathlineto{\pgfqpoint{1.568247in}{1.976522in}}%
\pgfpathlineto{\pgfqpoint{1.556201in}{1.976522in}}%
\pgfpathlineto{\pgfqpoint{1.544155in}{1.976522in}}%
\pgfpathlineto{\pgfqpoint{1.532108in}{1.976522in}}%
\pgfpathlineto{\pgfqpoint{1.520062in}{1.976522in}}%
\pgfpathlineto{\pgfqpoint{1.508016in}{1.976522in}}%
\pgfpathlineto{\pgfqpoint{1.495970in}{1.976522in}}%
\pgfpathlineto{\pgfqpoint{1.483924in}{1.976522in}}%
\pgfpathlineto{\pgfqpoint{1.471878in}{1.976522in}}%
\pgfpathlineto{\pgfqpoint{1.459832in}{1.976522in}}%
\pgfpathlineto{\pgfqpoint{1.447786in}{1.976522in}}%
\pgfpathlineto{\pgfqpoint{1.435740in}{1.976522in}}%
\pgfpathlineto{\pgfqpoint{1.423693in}{1.976522in}}%
\pgfpathlineto{\pgfqpoint{1.411647in}{1.976522in}}%
\pgfpathlineto{\pgfqpoint{1.399601in}{1.976522in}}%
\pgfpathlineto{\pgfqpoint{1.387555in}{1.976522in}}%
\pgfpathlineto{\pgfqpoint{1.375509in}{1.976522in}}%
\pgfpathlineto{\pgfqpoint{1.363463in}{1.976522in}}%
\pgfpathlineto{\pgfqpoint{1.351417in}{1.976522in}}%
\pgfpathlineto{\pgfqpoint{1.339371in}{1.976522in}}%
\pgfpathlineto{\pgfqpoint{1.327325in}{1.976522in}}%
\pgfpathlineto{\pgfqpoint{1.315278in}{1.976522in}}%
\pgfpathlineto{\pgfqpoint{1.303232in}{1.976522in}}%
\pgfpathlineto{\pgfqpoint{1.291186in}{1.976522in}}%
\pgfpathlineto{\pgfqpoint{1.279140in}{1.976522in}}%
\pgfpathlineto{\pgfqpoint{1.267094in}{1.976522in}}%
\pgfpathlineto{\pgfqpoint{1.255048in}{1.976522in}}%
\pgfpathlineto{\pgfqpoint{1.243002in}{1.976522in}}%
\pgfpathlineto{\pgfqpoint{1.230956in}{1.976522in}}%
\pgfpathlineto{\pgfqpoint{1.218909in}{1.976522in}}%
\pgfpathlineto{\pgfqpoint{1.206863in}{1.976522in}}%
\pgfpathlineto{\pgfqpoint{1.194817in}{1.976522in}}%
\pgfpathlineto{\pgfqpoint{1.182771in}{1.976522in}}%
\pgfpathlineto{\pgfqpoint{1.170725in}{1.976522in}}%
\pgfpathlineto{\pgfqpoint{1.158679in}{1.976522in}}%
\pgfpathlineto{\pgfqpoint{1.146633in}{1.976522in}}%
\pgfpathlineto{\pgfqpoint{1.134587in}{1.976522in}}%
\pgfpathlineto{\pgfqpoint{1.122541in}{1.976522in}}%
\pgfpathlineto{\pgfqpoint{1.110494in}{1.976522in}}%
\pgfpathlineto{\pgfqpoint{1.098448in}{1.976522in}}%
\pgfpathlineto{\pgfqpoint{1.086402in}{1.976522in}}%
\pgfpathlineto{\pgfqpoint{1.074356in}{1.976522in}}%
\pgfpathlineto{\pgfqpoint{1.062310in}{1.976522in}}%
\pgfpathlineto{\pgfqpoint{1.050264in}{1.976522in}}%
\pgfpathlineto{\pgfqpoint{1.038218in}{1.976522in}}%
\pgfpathlineto{\pgfqpoint{1.026172in}{1.976522in}}%
\pgfpathlineto{\pgfqpoint{1.014126in}{1.976522in}}%
\pgfpathlineto{\pgfqpoint{1.002079in}{1.976522in}}%
\pgfpathlineto{\pgfqpoint{0.990033in}{1.976522in}}%
\pgfpathlineto{\pgfqpoint{0.977987in}{1.976522in}}%
\pgfpathlineto{\pgfqpoint{0.965941in}{1.976522in}}%
\pgfpathlineto{\pgfqpoint{0.953895in}{1.976522in}}%
\pgfpathlineto{\pgfqpoint{0.941849in}{1.976522in}}%
\pgfpathlineto{\pgfqpoint{0.929803in}{1.976522in}}%
\pgfpathlineto{\pgfqpoint{0.917757in}{1.976522in}}%
\pgfpathlineto{\pgfqpoint{0.905710in}{1.976522in}}%
\pgfpathlineto{\pgfqpoint{0.893664in}{1.976522in}}%
\pgfpathlineto{\pgfqpoint{0.881618in}{1.976522in}}%
\pgfpathlineto{\pgfqpoint{0.869572in}{1.976522in}}%
\pgfpathlineto{\pgfqpoint{0.857526in}{1.976522in}}%
\pgfpathlineto{\pgfqpoint{0.845480in}{1.976522in}}%
\pgfpathlineto{\pgfqpoint{0.833434in}{1.976522in}}%
\pgfpathlineto{\pgfqpoint{0.821388in}{1.976522in}}%
\pgfpathlineto{\pgfqpoint{0.809342in}{1.976522in}}%
\pgfpathlineto{\pgfqpoint{0.797295in}{1.976522in}}%
\pgfpathlineto{\pgfqpoint{0.785249in}{1.976522in}}%
\pgfpathlineto{\pgfqpoint{0.773203in}{1.976522in}}%
\pgfpathlineto{\pgfqpoint{0.761157in}{1.976522in}}%
\pgfpathlineto{\pgfqpoint{0.749111in}{1.976522in}}%
\pgfpathlineto{\pgfqpoint{0.737065in}{1.976522in}}%
\pgfpathlineto{\pgfqpoint{0.725019in}{1.976522in}}%
\pgfpathlineto{\pgfqpoint{0.712973in}{1.976522in}}%
\pgfpathlineto{\pgfqpoint{0.700927in}{1.976522in}}%
\pgfpathlineto{\pgfqpoint{0.688880in}{1.976522in}}%
\pgfpathlineto{\pgfqpoint{0.676834in}{1.976522in}}%
\pgfpathlineto{\pgfqpoint{0.664788in}{1.976522in}}%
\pgfpathlineto{\pgfqpoint{0.652742in}{1.976522in}}%
\pgfpathlineto{\pgfqpoint{0.640696in}{1.976522in}}%
\pgfpathlineto{\pgfqpoint{0.628650in}{1.976522in}}%
\pgfpathlineto{\pgfqpoint{0.616604in}{1.976522in}}%
\pgfpathlineto{\pgfqpoint{0.604558in}{1.976522in}}%
\pgfpathlineto{\pgfqpoint{0.592511in}{1.976522in}}%
\pgfpathlineto{\pgfqpoint{0.580465in}{1.976522in}}%
\pgfpathlineto{\pgfqpoint{0.568419in}{1.976522in}}%
\pgfpathlineto{\pgfqpoint{0.556373in}{1.976522in}}%
\pgfpathlineto{\pgfqpoint{0.544327in}{1.976522in}}%
\pgfpathlineto{\pgfqpoint{0.532281in}{1.976522in}}%
\pgfpathlineto{\pgfqpoint{0.520235in}{1.976522in}}%
\pgfpathlineto{\pgfqpoint{0.508189in}{1.976522in}}%
\pgfpathlineto{\pgfqpoint{0.496143in}{1.976522in}}%
\pgfpathlineto{\pgfqpoint{0.484096in}{1.976522in}}%
\pgfpathlineto{\pgfqpoint{0.472050in}{1.976522in}}%
\pgfpathlineto{\pgfqpoint{0.460004in}{1.976522in}}%
\pgfpathlineto{\pgfqpoint{0.447958in}{1.976522in}}%
\pgfpathlineto{\pgfqpoint{0.435912in}{1.976522in}}%
\pgfpathlineto{\pgfqpoint{0.423866in}{1.976522in}}%
\pgfpathlineto{\pgfqpoint{0.411820in}{1.976522in}}%
\pgfpathlineto{\pgfqpoint{0.399774in}{1.976522in}}%
\pgfpathlineto{\pgfqpoint{0.387728in}{1.976522in}}%
\pgfpathlineto{\pgfqpoint{0.375681in}{1.976522in}}%
\pgfpathlineto{\pgfqpoint{0.363635in}{1.976522in}}%
\pgfpathlineto{\pgfqpoint{0.351589in}{1.976522in}}%
\pgfpathlineto{\pgfqpoint{0.339543in}{1.976522in}}%
\pgfpathlineto{\pgfqpoint{0.327497in}{1.976522in}}%
\pgfpathlineto{\pgfqpoint{0.315451in}{1.976522in}}%
\pgfpathlineto{\pgfqpoint{0.303405in}{1.976522in}}%
\pgfpathlineto{\pgfqpoint{0.291359in}{1.976522in}}%
\pgfpathlineto{\pgfqpoint{0.279312in}{1.976522in}}%
\pgfpathlineto{\pgfqpoint{0.267266in}{1.976522in}}%
\pgfpathlineto{\pgfqpoint{0.255220in}{1.976522in}}%
\pgfpathlineto{\pgfqpoint{0.243174in}{1.976522in}}%
\pgfpathlineto{\pgfqpoint{0.231128in}{1.976522in}}%
\pgfpathlineto{\pgfqpoint{0.219082in}{1.976522in}}%
\pgfpathlineto{\pgfqpoint{0.207036in}{1.976522in}}%
\pgfpathlineto{\pgfqpoint{0.194990in}{1.976522in}}%
\pgfpathlineto{\pgfqpoint{0.182944in}{1.976522in}}%
\pgfpathlineto{\pgfqpoint{0.170897in}{1.976522in}}%
\pgfpathlineto{\pgfqpoint{0.170897in}{1.964476in}}%
\pgfpathlineto{\pgfqpoint{0.170897in}{1.952430in}}%
\pgfpathlineto{\pgfqpoint{0.170897in}{1.940384in}}%
\pgfpathlineto{\pgfqpoint{0.170897in}{1.928338in}}%
\pgfpathlineto{\pgfqpoint{0.170897in}{1.916292in}}%
\pgfpathlineto{\pgfqpoint{0.170897in}{1.904246in}}%
\pgfpathlineto{\pgfqpoint{0.170897in}{1.892200in}}%
\pgfpathlineto{\pgfqpoint{0.170897in}{1.880154in}}%
\pgfpathlineto{\pgfqpoint{0.170897in}{1.868107in}}%
\pgfpathlineto{\pgfqpoint{0.170897in}{1.856061in}}%
\pgfpathlineto{\pgfqpoint{0.170897in}{1.844015in}}%
\pgfpathlineto{\pgfqpoint{0.170897in}{1.831969in}}%
\pgfpathlineto{\pgfqpoint{0.170897in}{1.819923in}}%
\pgfpathlineto{\pgfqpoint{0.170897in}{1.807877in}}%
\pgfpathlineto{\pgfqpoint{0.170897in}{1.795831in}}%
\pgfpathlineto{\pgfqpoint{0.170897in}{1.783785in}}%
\pgfpathlineto{\pgfqpoint{0.170897in}{1.771738in}}%
\pgfpathlineto{\pgfqpoint{0.170897in}{1.759692in}}%
\pgfpathlineto{\pgfqpoint{0.170897in}{1.747646in}}%
\pgfpathlineto{\pgfqpoint{0.170897in}{1.735600in}}%
\pgfpathlineto{\pgfqpoint{0.170897in}{1.723554in}}%
\pgfpathlineto{\pgfqpoint{0.170897in}{1.711508in}}%
\pgfpathlineto{\pgfqpoint{0.170897in}{1.699462in}}%
\pgfpathlineto{\pgfqpoint{0.170897in}{1.687416in}}%
\pgfpathlineto{\pgfqpoint{0.170897in}{1.675370in}}%
\pgfpathlineto{\pgfqpoint{0.170897in}{1.663323in}}%
\pgfpathlineto{\pgfqpoint{0.170897in}{1.651277in}}%
\pgfpathlineto{\pgfqpoint{0.170897in}{1.639231in}}%
\pgfpathlineto{\pgfqpoint{0.170897in}{1.627185in}}%
\pgfpathlineto{\pgfqpoint{0.170897in}{1.615139in}}%
\pgfpathlineto{\pgfqpoint{0.170897in}{1.603093in}}%
\pgfpathlineto{\pgfqpoint{0.170897in}{1.591047in}}%
\pgfpathlineto{\pgfqpoint{0.170897in}{1.579001in}}%
\pgfpathlineto{\pgfqpoint{0.170897in}{1.566955in}}%
\pgfpathlineto{\pgfqpoint{0.170897in}{1.554908in}}%
\pgfpathlineto{\pgfqpoint{0.170897in}{1.542862in}}%
\pgfpathlineto{\pgfqpoint{0.170897in}{1.530816in}}%
\pgfpathlineto{\pgfqpoint{0.170897in}{1.518770in}}%
\pgfpathlineto{\pgfqpoint{0.170897in}{1.506724in}}%
\pgfpathlineto{\pgfqpoint{0.170897in}{1.494678in}}%
\pgfpathlineto{\pgfqpoint{0.170897in}{1.482632in}}%
\pgfpathlineto{\pgfqpoint{0.170897in}{1.470586in}}%
\pgfpathlineto{\pgfqpoint{0.170897in}{1.458539in}}%
\pgfpathlineto{\pgfqpoint{0.170897in}{1.446493in}}%
\pgfpathlineto{\pgfqpoint{0.170897in}{1.434447in}}%
\pgfpathlineto{\pgfqpoint{0.170897in}{1.422401in}}%
\pgfpathlineto{\pgfqpoint{0.170897in}{1.410355in}}%
\pgfpathlineto{\pgfqpoint{0.170897in}{1.398309in}}%
\pgfpathlineto{\pgfqpoint{0.170897in}{1.386263in}}%
\pgfpathlineto{\pgfqpoint{0.170897in}{1.374217in}}%
\pgfpathlineto{\pgfqpoint{0.170897in}{1.362171in}}%
\pgfpathlineto{\pgfqpoint{0.170897in}{1.350124in}}%
\pgfpathlineto{\pgfqpoint{0.170897in}{1.338078in}}%
\pgfpathlineto{\pgfqpoint{0.170897in}{1.326032in}}%
\pgfpathlineto{\pgfqpoint{0.170897in}{1.313986in}}%
\pgfpathlineto{\pgfqpoint{0.170897in}{1.301940in}}%
\pgfpathlineto{\pgfqpoint{0.170897in}{1.289894in}}%
\pgfpathlineto{\pgfqpoint{0.170897in}{1.277848in}}%
\pgfpathlineto{\pgfqpoint{0.170897in}{1.265802in}}%
\pgfpathlineto{\pgfqpoint{0.170897in}{1.253756in}}%
\pgfpathlineto{\pgfqpoint{0.170897in}{1.241709in}}%
\pgfpathlineto{\pgfqpoint{0.170897in}{1.229663in}}%
\pgfpathlineto{\pgfqpoint{0.170897in}{1.217617in}}%
\pgfpathlineto{\pgfqpoint{0.170897in}{1.205571in}}%
\pgfpathlineto{\pgfqpoint{0.170897in}{1.193525in}}%
\pgfpathlineto{\pgfqpoint{0.170897in}{1.181479in}}%
\pgfpathlineto{\pgfqpoint{0.170897in}{1.169433in}}%
\pgfpathlineto{\pgfqpoint{0.170897in}{1.157387in}}%
\pgfpathlineto{\pgfqpoint{0.170897in}{1.145340in}}%
\pgfpathlineto{\pgfqpoint{0.170897in}{1.133294in}}%
\pgfpathlineto{\pgfqpoint{0.170897in}{1.121248in}}%
\pgfpathlineto{\pgfqpoint{0.170897in}{1.109202in}}%
\pgfpathlineto{\pgfqpoint{0.170897in}{1.097156in}}%
\pgfpathlineto{\pgfqpoint{0.170897in}{1.085110in}}%
\pgfpathlineto{\pgfqpoint{0.170897in}{1.073064in}}%
\pgfpathlineto{\pgfqpoint{0.170897in}{1.061018in}}%
\pgfpathlineto{\pgfqpoint{0.170897in}{1.048972in}}%
\pgfpathlineto{\pgfqpoint{0.170897in}{1.036925in}}%
\pgfpathlineto{\pgfqpoint{0.170897in}{1.024879in}}%
\pgfpathlineto{\pgfqpoint{0.170897in}{1.012833in}}%
\pgfpathlineto{\pgfqpoint{0.170897in}{1.000787in}}%
\pgfpathlineto{\pgfqpoint{0.170897in}{0.988741in}}%
\pgfpathlineto{\pgfqpoint{0.170897in}{0.976695in}}%
\pgfpathlineto{\pgfqpoint{0.170897in}{0.964649in}}%
\pgfpathlineto{\pgfqpoint{0.170897in}{0.952603in}}%
\pgfpathlineto{\pgfqpoint{0.170897in}{0.940557in}}%
\pgfpathlineto{\pgfqpoint{0.170897in}{0.928510in}}%
\pgfpathlineto{\pgfqpoint{0.170897in}{0.916464in}}%
\pgfpathlineto{\pgfqpoint{0.170897in}{0.904418in}}%
\pgfpathlineto{\pgfqpoint{0.170897in}{0.892372in}}%
\pgfpathlineto{\pgfqpoint{0.170897in}{0.880326in}}%
\pgfpathlineto{\pgfqpoint{0.170897in}{0.868280in}}%
\pgfpathlineto{\pgfqpoint{0.170897in}{0.856234in}}%
\pgfpathlineto{\pgfqpoint{0.170897in}{0.844188in}}%
\pgfpathlineto{\pgfqpoint{0.170897in}{0.832141in}}%
\pgfpathlineto{\pgfqpoint{0.170897in}{0.820095in}}%
\pgfpathlineto{\pgfqpoint{0.170897in}{0.808049in}}%
\pgfpathlineto{\pgfqpoint{0.170897in}{0.796003in}}%
\pgfpathlineto{\pgfqpoint{0.170897in}{0.783957in}}%
\pgfpathlineto{\pgfqpoint{0.170897in}{0.771911in}}%
\pgfpathlineto{\pgfqpoint{0.170897in}{0.759865in}}%
\pgfpathlineto{\pgfqpoint{0.170897in}{0.747819in}}%
\pgfpathlineto{\pgfqpoint{0.170897in}{0.735773in}}%
\pgfpathlineto{\pgfqpoint{0.170897in}{0.723726in}}%
\pgfpathlineto{\pgfqpoint{0.170897in}{0.711680in}}%
\pgfpathlineto{\pgfqpoint{0.170897in}{0.699634in}}%
\pgfpathlineto{\pgfqpoint{0.170897in}{0.687588in}}%
\pgfpathlineto{\pgfqpoint{0.170897in}{0.675542in}}%
\pgfpathlineto{\pgfqpoint{0.170897in}{0.663496in}}%
\pgfpathlineto{\pgfqpoint{0.170897in}{0.651450in}}%
\pgfpathlineto{\pgfqpoint{0.170897in}{0.639404in}}%
\pgfpathlineto{\pgfqpoint{0.170897in}{0.627358in}}%
\pgfpathlineto{\pgfqpoint{0.170897in}{0.615311in}}%
\pgfpathlineto{\pgfqpoint{0.170897in}{0.603265in}}%
\pgfpathlineto{\pgfqpoint{0.170897in}{0.591219in}}%
\pgfpathlineto{\pgfqpoint{0.170897in}{0.579173in}}%
\pgfpathlineto{\pgfqpoint{0.170897in}{0.567127in}}%
\pgfpathlineto{\pgfqpoint{0.170897in}{0.555081in}}%
\pgfpathlineto{\pgfqpoint{0.170897in}{0.543035in}}%
\pgfpathlineto{\pgfqpoint{0.170897in}{0.530989in}}%
\pgfpathlineto{\pgfqpoint{0.170897in}{0.518942in}}%
\pgfpathlineto{\pgfqpoint{0.170897in}{0.506896in}}%
\pgfpathlineto{\pgfqpoint{0.170897in}{0.494850in}}%
\pgfpathlineto{\pgfqpoint{0.170897in}{0.482804in}}%
\pgfpathlineto{\pgfqpoint{0.170897in}{0.470758in}}%
\pgfpathlineto{\pgfqpoint{0.170897in}{0.458712in}}%
\pgfpathlineto{\pgfqpoint{0.170897in}{0.446666in}}%
\pgfpathlineto{\pgfqpoint{0.170897in}{0.434620in}}%
\pgfpathlineto{\pgfqpoint{0.170897in}{0.422574in}}%
\pgfpathlineto{\pgfqpoint{0.170897in}{0.410527in}}%
\pgfpathlineto{\pgfqpoint{0.170897in}{0.398481in}}%
\pgfpathlineto{\pgfqpoint{0.170897in}{0.386435in}}%
\pgfpathlineto{\pgfqpoint{0.170897in}{0.374389in}}%
\pgfpathlineto{\pgfqpoint{0.170897in}{0.362343in}}%
\pgfpathlineto{\pgfqpoint{0.170897in}{0.350297in}}%
\pgfpathlineto{\pgfqpoint{0.170897in}{0.338251in}}%
\pgfpathlineto{\pgfqpoint{0.170897in}{0.326205in}}%
\pgfpathlineto{\pgfqpoint{0.170897in}{0.314159in}}%
\pgfpathlineto{\pgfqpoint{0.170897in}{0.302112in}}%
\pgfpathlineto{\pgfqpoint{0.170897in}{0.290066in}}%
\pgfpathlineto{\pgfqpoint{0.170897in}{0.278020in}}%
\pgfpathlineto{\pgfqpoint{0.170897in}{0.265974in}}%
\pgfpathlineto{\pgfqpoint{0.170897in}{0.253928in}}%
\pgfpathlineto{\pgfqpoint{0.170897in}{0.241882in}}%
\pgfpathlineto{\pgfqpoint{0.170897in}{0.229836in}}%
\pgfpathlineto{\pgfqpoint{0.170897in}{0.217790in}}%
\pgfpathlineto{\pgfqpoint{0.170897in}{0.205743in}}%
\pgfpathlineto{\pgfqpoint{0.170897in}{0.193697in}}%
\pgfpathlineto{\pgfqpoint{0.170897in}{0.181651in}}%
\pgfpathclose%
\pgfpathmoveto{\pgfqpoint{0.921894in}{0.193697in}}%
\pgfpathlineto{\pgfqpoint{0.917757in}{0.194374in}}%
\pgfpathlineto{\pgfqpoint{0.905710in}{0.196509in}}%
\pgfpathlineto{\pgfqpoint{0.893664in}{0.198814in}}%
\pgfpathlineto{\pgfqpoint{0.881618in}{0.201290in}}%
\pgfpathlineto{\pgfqpoint{0.869572in}{0.203939in}}%
\pgfpathlineto{\pgfqpoint{0.861862in}{0.205743in}}%
\pgfpathlineto{\pgfqpoint{0.857526in}{0.206762in}}%
\pgfpathlineto{\pgfqpoint{0.845480in}{0.209762in}}%
\pgfpathlineto{\pgfqpoint{0.833434in}{0.212940in}}%
\pgfpathlineto{\pgfqpoint{0.821388in}{0.216296in}}%
\pgfpathlineto{\pgfqpoint{0.816291in}{0.217790in}}%
\pgfpathlineto{\pgfqpoint{0.809342in}{0.219836in}}%
\pgfpathlineto{\pgfqpoint{0.797295in}{0.223560in}}%
\pgfpathlineto{\pgfqpoint{0.785249in}{0.227470in}}%
\pgfpathlineto{\pgfqpoint{0.778279in}{0.229836in}}%
\pgfpathlineto{\pgfqpoint{0.773203in}{0.231569in}}%
\pgfpathlineto{\pgfqpoint{0.761157in}{0.235861in}}%
\pgfpathlineto{\pgfqpoint{0.749111in}{0.240346in}}%
\pgfpathlineto{\pgfqpoint{0.745145in}{0.241882in}}%
\pgfpathlineto{\pgfqpoint{0.737065in}{0.245032in}}%
\pgfpathlineto{\pgfqpoint{0.725019in}{0.249918in}}%
\pgfpathlineto{\pgfqpoint{0.715512in}{0.253928in}}%
\pgfpathlineto{\pgfqpoint{0.712973in}{0.255007in}}%
\pgfpathlineto{\pgfqpoint{0.700927in}{0.260309in}}%
\pgfpathlineto{\pgfqpoint{0.688880in}{0.265818in}}%
\pgfpathlineto{\pgfqpoint{0.688550in}{0.265974in}}%
\pgfpathlineto{\pgfqpoint{0.676834in}{0.271552in}}%
\pgfpathlineto{\pgfqpoint{0.664788in}{0.277500in}}%
\pgfpathlineto{\pgfqpoint{0.663767in}{0.278020in}}%
\pgfpathlineto{\pgfqpoint{0.652742in}{0.283683in}}%
\pgfpathlineto{\pgfqpoint{0.640739in}{0.290066in}}%
\pgfpathlineto{\pgfqpoint{0.640696in}{0.290089in}}%
\pgfpathlineto{\pgfqpoint{0.628650in}{0.296744in}}%
\pgfpathlineto{\pgfqpoint{0.619242in}{0.302112in}}%
\pgfpathlineto{\pgfqpoint{0.616604in}{0.303634in}}%
\pgfpathlineto{\pgfqpoint{0.604558in}{0.310780in}}%
\pgfpathlineto{\pgfqpoint{0.599026in}{0.314159in}}%
\pgfpathlineto{\pgfqpoint{0.592511in}{0.318183in}}%
\pgfpathlineto{\pgfqpoint{0.580465in}{0.325844in}}%
\pgfpathlineto{\pgfqpoint{0.579913in}{0.326205in}}%
\pgfpathlineto{\pgfqpoint{0.568419in}{0.333791in}}%
\pgfpathlineto{\pgfqpoint{0.561847in}{0.338251in}}%
\pgfpathlineto{\pgfqpoint{0.556373in}{0.342013in}}%
\pgfpathlineto{\pgfqpoint{0.544641in}{0.350297in}}%
\pgfpathlineto{\pgfqpoint{0.544327in}{0.350522in}}%
\pgfpathlineto{\pgfqpoint{0.532281in}{0.359344in}}%
\pgfpathlineto{\pgfqpoint{0.528288in}{0.362343in}}%
\pgfpathlineto{\pgfqpoint{0.520235in}{0.368478in}}%
\pgfpathlineto{\pgfqpoint{0.512662in}{0.374389in}}%
\pgfpathlineto{\pgfqpoint{0.508189in}{0.377933in}}%
\pgfpathlineto{\pgfqpoint{0.497707in}{0.386435in}}%
\pgfpathlineto{\pgfqpoint{0.496143in}{0.387724in}}%
\pgfpathlineto{\pgfqpoint{0.484096in}{0.397871in}}%
\pgfpathlineto{\pgfqpoint{0.483387in}{0.398481in}}%
\pgfpathlineto{\pgfqpoint{0.472050in}{0.408394in}}%
\pgfpathlineto{\pgfqpoint{0.469663in}{0.410527in}}%
\pgfpathlineto{\pgfqpoint{0.460004in}{0.419304in}}%
\pgfpathlineto{\pgfqpoint{0.456479in}{0.422574in}}%
\pgfpathlineto{\pgfqpoint{0.447958in}{0.430619in}}%
\pgfpathlineto{\pgfqpoint{0.443805in}{0.434620in}}%
\pgfpathlineto{\pgfqpoint{0.435912in}{0.442363in}}%
\pgfpathlineto{\pgfqpoint{0.431609in}{0.446666in}}%
\pgfpathlineto{\pgfqpoint{0.423866in}{0.454558in}}%
\pgfpathlineto{\pgfqpoint{0.419865in}{0.458712in}}%
\pgfpathlineto{\pgfqpoint{0.411820in}{0.467233in}}%
\pgfpathlineto{\pgfqpoint{0.408550in}{0.470758in}}%
\pgfpathlineto{\pgfqpoint{0.399774in}{0.480416in}}%
\pgfpathlineto{\pgfqpoint{0.397641in}{0.482804in}}%
\pgfpathlineto{\pgfqpoint{0.387728in}{0.494141in}}%
\pgfpathlineto{\pgfqpoint{0.387117in}{0.494850in}}%
\pgfpathlineto{\pgfqpoint{0.376970in}{0.506896in}}%
\pgfpathlineto{\pgfqpoint{0.375681in}{0.508460in}}%
\pgfpathlineto{\pgfqpoint{0.367179in}{0.518942in}}%
\pgfpathlineto{\pgfqpoint{0.363635in}{0.523415in}}%
\pgfpathlineto{\pgfqpoint{0.357724in}{0.530989in}}%
\pgfpathlineto{\pgfqpoint{0.351589in}{0.539042in}}%
\pgfpathlineto{\pgfqpoint{0.348591in}{0.543035in}}%
\pgfpathlineto{\pgfqpoint{0.339768in}{0.555081in}}%
\pgfpathlineto{\pgfqpoint{0.339543in}{0.555395in}}%
\pgfpathlineto{\pgfqpoint{0.331259in}{0.567127in}}%
\pgfpathlineto{\pgfqpoint{0.327497in}{0.572601in}}%
\pgfpathlineto{\pgfqpoint{0.323037in}{0.579173in}}%
\pgfpathlineto{\pgfqpoint{0.315451in}{0.590666in}}%
\pgfpathlineto{\pgfqpoint{0.315090in}{0.591219in}}%
\pgfpathlineto{\pgfqpoint{0.307429in}{0.603265in}}%
\pgfpathlineto{\pgfqpoint{0.303405in}{0.609780in}}%
\pgfpathlineto{\pgfqpoint{0.300026in}{0.615311in}}%
\pgfpathlineto{\pgfqpoint{0.292880in}{0.627358in}}%
\pgfpathlineto{\pgfqpoint{0.291359in}{0.629996in}}%
\pgfpathlineto{\pgfqpoint{0.285990in}{0.639404in}}%
\pgfpathlineto{\pgfqpoint{0.279335in}{0.651450in}}%
\pgfpathlineto{\pgfqpoint{0.279312in}{0.651493in}}%
\pgfpathlineto{\pgfqpoint{0.272929in}{0.663496in}}%
\pgfpathlineto{\pgfqpoint{0.267266in}{0.674521in}}%
\pgfpathlineto{\pgfqpoint{0.266746in}{0.675542in}}%
\pgfpathlineto{\pgfqpoint{0.260798in}{0.687588in}}%
\pgfpathlineto{\pgfqpoint{0.255220in}{0.699304in}}%
\pgfpathlineto{\pgfqpoint{0.255064in}{0.699634in}}%
\pgfpathlineto{\pgfqpoint{0.249555in}{0.711680in}}%
\pgfpathlineto{\pgfqpoint{0.244253in}{0.723726in}}%
\pgfpathlineto{\pgfqpoint{0.243174in}{0.726265in}}%
\pgfpathlineto{\pgfqpoint{0.239164in}{0.735773in}}%
\pgfpathlineto{\pgfqpoint{0.234278in}{0.747819in}}%
\pgfpathlineto{\pgfqpoint{0.231128in}{0.755899in}}%
\pgfpathlineto{\pgfqpoint{0.229592in}{0.759865in}}%
\pgfpathlineto{\pgfqpoint{0.225107in}{0.771911in}}%
\pgfpathlineto{\pgfqpoint{0.220815in}{0.783957in}}%
\pgfpathlineto{\pgfqpoint{0.219082in}{0.789033in}}%
\pgfpathlineto{\pgfqpoint{0.216716in}{0.796003in}}%
\pgfpathlineto{\pgfqpoint{0.212806in}{0.808049in}}%
\pgfpathlineto{\pgfqpoint{0.209082in}{0.820095in}}%
\pgfpathlineto{\pgfqpoint{0.207036in}{0.827045in}}%
\pgfpathlineto{\pgfqpoint{0.205543in}{0.832141in}}%
\pgfpathlineto{\pgfqpoint{0.202186in}{0.844188in}}%
\pgfpathlineto{\pgfqpoint{0.199008in}{0.856234in}}%
\pgfpathlineto{\pgfqpoint{0.196008in}{0.868280in}}%
\pgfpathlineto{\pgfqpoint{0.194990in}{0.872616in}}%
\pgfpathlineto{\pgfqpoint{0.193185in}{0.880326in}}%
\pgfpathlineto{\pgfqpoint{0.190536in}{0.892372in}}%
\pgfpathlineto{\pgfqpoint{0.188060in}{0.904418in}}%
\pgfpathlineto{\pgfqpoint{0.185755in}{0.916464in}}%
\pgfpathlineto{\pgfqpoint{0.183620in}{0.928510in}}%
\pgfpathlineto{\pgfqpoint{0.182944in}{0.932648in}}%
\pgfpathlineto{\pgfqpoint{0.181654in}{0.940557in}}%
\pgfpathlineto{\pgfqpoint{0.179856in}{0.952603in}}%
\pgfpathlineto{\pgfqpoint{0.178224in}{0.964649in}}%
\pgfpathlineto{\pgfqpoint{0.176758in}{0.976695in}}%
\pgfpathlineto{\pgfqpoint{0.175457in}{0.988741in}}%
\pgfpathlineto{\pgfqpoint{0.174320in}{1.000787in}}%
\pgfpathlineto{\pgfqpoint{0.173346in}{1.012833in}}%
\pgfpathlineto{\pgfqpoint{0.172536in}{1.024879in}}%
\pgfpathlineto{\pgfqpoint{0.171888in}{1.036925in}}%
\pgfpathlineto{\pgfqpoint{0.171403in}{1.048972in}}%
\pgfpathlineto{\pgfqpoint{0.171079in}{1.061018in}}%
\pgfpathlineto{\pgfqpoint{0.170918in}{1.073064in}}%
\pgfpathlineto{\pgfqpoint{0.170918in}{1.085110in}}%
\pgfpathlineto{\pgfqpoint{0.171079in}{1.097156in}}%
\pgfpathlineto{\pgfqpoint{0.171403in}{1.109202in}}%
\pgfpathlineto{\pgfqpoint{0.171888in}{1.121248in}}%
\pgfpathlineto{\pgfqpoint{0.172536in}{1.133294in}}%
\pgfpathlineto{\pgfqpoint{0.173346in}{1.145340in}}%
\pgfpathlineto{\pgfqpoint{0.174320in}{1.157387in}}%
\pgfpathlineto{\pgfqpoint{0.175457in}{1.169433in}}%
\pgfpathlineto{\pgfqpoint{0.176758in}{1.181479in}}%
\pgfpathlineto{\pgfqpoint{0.178224in}{1.193525in}}%
\pgfpathlineto{\pgfqpoint{0.179856in}{1.205571in}}%
\pgfpathlineto{\pgfqpoint{0.181654in}{1.217617in}}%
\pgfpathlineto{\pgfqpoint{0.182944in}{1.225525in}}%
\pgfpathlineto{\pgfqpoint{0.183620in}{1.229663in}}%
\pgfpathlineto{\pgfqpoint{0.185755in}{1.241709in}}%
\pgfpathlineto{\pgfqpoint{0.188060in}{1.253756in}}%
\pgfpathlineto{\pgfqpoint{0.190536in}{1.265802in}}%
\pgfpathlineto{\pgfqpoint{0.193185in}{1.277848in}}%
\pgfpathlineto{\pgfqpoint{0.194990in}{1.285558in}}%
\pgfpathlineto{\pgfqpoint{0.196008in}{1.289894in}}%
\pgfpathlineto{\pgfqpoint{0.199008in}{1.301940in}}%
\pgfpathlineto{\pgfqpoint{0.202186in}{1.313986in}}%
\pgfpathlineto{\pgfqpoint{0.205543in}{1.326032in}}%
\pgfpathlineto{\pgfqpoint{0.207036in}{1.331129in}}%
\pgfpathlineto{\pgfqpoint{0.209082in}{1.338078in}}%
\pgfpathlineto{\pgfqpoint{0.212806in}{1.350124in}}%
\pgfpathlineto{\pgfqpoint{0.216716in}{1.362171in}}%
\pgfpathlineto{\pgfqpoint{0.219082in}{1.369141in}}%
\pgfpathlineto{\pgfqpoint{0.220815in}{1.374217in}}%
\pgfpathlineto{\pgfqpoint{0.225107in}{1.386263in}}%
\pgfpathlineto{\pgfqpoint{0.229592in}{1.398309in}}%
\pgfpathlineto{\pgfqpoint{0.231128in}{1.402275in}}%
\pgfpathlineto{\pgfqpoint{0.234278in}{1.410355in}}%
\pgfpathlineto{\pgfqpoint{0.239164in}{1.422401in}}%
\pgfpathlineto{\pgfqpoint{0.243174in}{1.431908in}}%
\pgfpathlineto{\pgfqpoint{0.244253in}{1.434447in}}%
\pgfpathlineto{\pgfqpoint{0.249555in}{1.446493in}}%
\pgfpathlineto{\pgfqpoint{0.255064in}{1.458539in}}%
\pgfpathlineto{\pgfqpoint{0.255220in}{1.458870in}}%
\pgfpathlineto{\pgfqpoint{0.260798in}{1.470586in}}%
\pgfpathlineto{\pgfqpoint{0.266746in}{1.482632in}}%
\pgfpathlineto{\pgfqpoint{0.267266in}{1.483653in}}%
\pgfpathlineto{\pgfqpoint{0.272929in}{1.494678in}}%
\pgfpathlineto{\pgfqpoint{0.279312in}{1.506681in}}%
\pgfpathlineto{\pgfqpoint{0.279335in}{1.506724in}}%
\pgfpathlineto{\pgfqpoint{0.285990in}{1.518770in}}%
\pgfpathlineto{\pgfqpoint{0.291359in}{1.528178in}}%
\pgfpathlineto{\pgfqpoint{0.292880in}{1.530816in}}%
\pgfpathlineto{\pgfqpoint{0.300026in}{1.542862in}}%
\pgfpathlineto{\pgfqpoint{0.303405in}{1.548394in}}%
\pgfpathlineto{\pgfqpoint{0.307429in}{1.554908in}}%
\pgfpathlineto{\pgfqpoint{0.315090in}{1.566955in}}%
\pgfpathlineto{\pgfqpoint{0.315451in}{1.567507in}}%
\pgfpathlineto{\pgfqpoint{0.323037in}{1.579001in}}%
\pgfpathlineto{\pgfqpoint{0.327497in}{1.585573in}}%
\pgfpathlineto{\pgfqpoint{0.331259in}{1.591047in}}%
\pgfpathlineto{\pgfqpoint{0.339543in}{1.602779in}}%
\pgfpathlineto{\pgfqpoint{0.339768in}{1.603093in}}%
\pgfpathlineto{\pgfqpoint{0.348591in}{1.615139in}}%
\pgfpathlineto{\pgfqpoint{0.351589in}{1.619132in}}%
\pgfpathlineto{\pgfqpoint{0.357724in}{1.627185in}}%
\pgfpathlineto{\pgfqpoint{0.363635in}{1.634758in}}%
\pgfpathlineto{\pgfqpoint{0.367179in}{1.639231in}}%
\pgfpathlineto{\pgfqpoint{0.375681in}{1.649713in}}%
\pgfpathlineto{\pgfqpoint{0.376970in}{1.651277in}}%
\pgfpathlineto{\pgfqpoint{0.387117in}{1.663323in}}%
\pgfpathlineto{\pgfqpoint{0.387728in}{1.664033in}}%
\pgfpathlineto{\pgfqpoint{0.397641in}{1.675370in}}%
\pgfpathlineto{\pgfqpoint{0.399774in}{1.677757in}}%
\pgfpathlineto{\pgfqpoint{0.408550in}{1.687416in}}%
\pgfpathlineto{\pgfqpoint{0.411820in}{1.690941in}}%
\pgfpathlineto{\pgfqpoint{0.419865in}{1.699462in}}%
\pgfpathlineto{\pgfqpoint{0.423866in}{1.703615in}}%
\pgfpathlineto{\pgfqpoint{0.431609in}{1.711508in}}%
\pgfpathlineto{\pgfqpoint{0.435912in}{1.715811in}}%
\pgfpathlineto{\pgfqpoint{0.443805in}{1.723554in}}%
\pgfpathlineto{\pgfqpoint{0.447958in}{1.727554in}}%
\pgfpathlineto{\pgfqpoint{0.456479in}{1.735600in}}%
\pgfpathlineto{\pgfqpoint{0.460004in}{1.738870in}}%
\pgfpathlineto{\pgfqpoint{0.469663in}{1.747646in}}%
\pgfpathlineto{\pgfqpoint{0.472050in}{1.749779in}}%
\pgfpathlineto{\pgfqpoint{0.483387in}{1.759692in}}%
\pgfpathlineto{\pgfqpoint{0.484096in}{1.760303in}}%
\pgfpathlineto{\pgfqpoint{0.496143in}{1.770450in}}%
\pgfpathlineto{\pgfqpoint{0.497707in}{1.771738in}}%
\pgfpathlineto{\pgfqpoint{0.508189in}{1.780241in}}%
\pgfpathlineto{\pgfqpoint{0.512662in}{1.783785in}}%
\pgfpathlineto{\pgfqpoint{0.520235in}{1.789696in}}%
\pgfpathlineto{\pgfqpoint{0.528288in}{1.795831in}}%
\pgfpathlineto{\pgfqpoint{0.532281in}{1.798829in}}%
\pgfpathlineto{\pgfqpoint{0.544327in}{1.807652in}}%
\pgfpathlineto{\pgfqpoint{0.544641in}{1.807877in}}%
\pgfpathlineto{\pgfqpoint{0.556373in}{1.816160in}}%
\pgfpathlineto{\pgfqpoint{0.561847in}{1.819923in}}%
\pgfpathlineto{\pgfqpoint{0.568419in}{1.824383in}}%
\pgfpathlineto{\pgfqpoint{0.579913in}{1.831969in}}%
\pgfpathlineto{\pgfqpoint{0.580465in}{1.832330in}}%
\pgfpathlineto{\pgfqpoint{0.592511in}{1.839991in}}%
\pgfpathlineto{\pgfqpoint{0.599026in}{1.844015in}}%
\pgfpathlineto{\pgfqpoint{0.604558in}{1.847394in}}%
\pgfpathlineto{\pgfqpoint{0.616604in}{1.854540in}}%
\pgfpathlineto{\pgfqpoint{0.619242in}{1.856061in}}%
\pgfpathlineto{\pgfqpoint{0.628650in}{1.861430in}}%
\pgfpathlineto{\pgfqpoint{0.640696in}{1.868084in}}%
\pgfpathlineto{\pgfqpoint{0.640739in}{1.868107in}}%
\pgfpathlineto{\pgfqpoint{0.652742in}{1.874491in}}%
\pgfpathlineto{\pgfqpoint{0.663767in}{1.880154in}}%
\pgfpathlineto{\pgfqpoint{0.664788in}{1.880673in}}%
\pgfpathlineto{\pgfqpoint{0.676834in}{1.886622in}}%
\pgfpathlineto{\pgfqpoint{0.688550in}{1.892200in}}%
\pgfpathlineto{\pgfqpoint{0.688880in}{1.892356in}}%
\pgfpathlineto{\pgfqpoint{0.700927in}{1.897865in}}%
\pgfpathlineto{\pgfqpoint{0.712973in}{1.903167in}}%
\pgfpathlineto{\pgfqpoint{0.715512in}{1.904246in}}%
\pgfpathlineto{\pgfqpoint{0.725019in}{1.908256in}}%
\pgfpathlineto{\pgfqpoint{0.737065in}{1.913142in}}%
\pgfpathlineto{\pgfqpoint{0.745145in}{1.916292in}}%
\pgfpathlineto{\pgfqpoint{0.749111in}{1.917828in}}%
\pgfpathlineto{\pgfqpoint{0.761157in}{1.922312in}}%
\pgfpathlineto{\pgfqpoint{0.773203in}{1.926605in}}%
\pgfpathlineto{\pgfqpoint{0.778279in}{1.928338in}}%
\pgfpathlineto{\pgfqpoint{0.785249in}{1.930704in}}%
\pgfpathlineto{\pgfqpoint{0.797295in}{1.934614in}}%
\pgfpathlineto{\pgfqpoint{0.809342in}{1.938338in}}%
\pgfpathlineto{\pgfqpoint{0.816291in}{1.940384in}}%
\pgfpathlineto{\pgfqpoint{0.821388in}{1.941877in}}%
\pgfpathlineto{\pgfqpoint{0.833434in}{1.945234in}}%
\pgfpathlineto{\pgfqpoint{0.845480in}{1.948411in}}%
\pgfpathlineto{\pgfqpoint{0.857526in}{1.951412in}}%
\pgfpathlineto{\pgfqpoint{0.861862in}{1.952430in}}%
\pgfpathlineto{\pgfqpoint{0.869572in}{1.954235in}}%
\pgfpathlineto{\pgfqpoint{0.881618in}{1.956883in}}%
\pgfpathlineto{\pgfqpoint{0.893664in}{1.959360in}}%
\pgfpathlineto{\pgfqpoint{0.905710in}{1.961665in}}%
\pgfpathlineto{\pgfqpoint{0.917757in}{1.963800in}}%
\pgfpathlineto{\pgfqpoint{0.921894in}{1.964476in}}%
\pgfpathlineto{\pgfqpoint{0.929803in}{1.965766in}}%
\pgfpathlineto{\pgfqpoint{0.941849in}{1.967564in}}%
\pgfpathlineto{\pgfqpoint{0.953895in}{1.969196in}}%
\pgfpathlineto{\pgfqpoint{0.965941in}{1.970662in}}%
\pgfpathlineto{\pgfqpoint{0.977987in}{1.971963in}}%
\pgfpathlineto{\pgfqpoint{0.990033in}{1.973100in}}%
\pgfpathlineto{\pgfqpoint{1.002079in}{1.974073in}}%
\pgfpathlineto{\pgfqpoint{1.014126in}{1.974884in}}%
\pgfpathlineto{\pgfqpoint{1.026172in}{1.975532in}}%
\pgfpathlineto{\pgfqpoint{1.038218in}{1.976017in}}%
\pgfpathlineto{\pgfqpoint{1.050264in}{1.976341in}}%
\pgfpathlineto{\pgfqpoint{1.062310in}{1.976502in}}%
\pgfpathlineto{\pgfqpoint{1.074356in}{1.976502in}}%
\pgfpathlineto{\pgfqpoint{1.086402in}{1.976341in}}%
\pgfpathlineto{\pgfqpoint{1.098448in}{1.976017in}}%
\pgfpathlineto{\pgfqpoint{1.110494in}{1.975532in}}%
\pgfpathlineto{\pgfqpoint{1.122541in}{1.974884in}}%
\pgfpathlineto{\pgfqpoint{1.134587in}{1.974073in}}%
\pgfpathlineto{\pgfqpoint{1.146633in}{1.973100in}}%
\pgfpathlineto{\pgfqpoint{1.158679in}{1.971963in}}%
\pgfpathlineto{\pgfqpoint{1.170725in}{1.970662in}}%
\pgfpathlineto{\pgfqpoint{1.182771in}{1.969196in}}%
\pgfpathlineto{\pgfqpoint{1.194817in}{1.967564in}}%
\pgfpathlineto{\pgfqpoint{1.206863in}{1.965766in}}%
\pgfpathlineto{\pgfqpoint{1.214772in}{1.964476in}}%
\pgfpathlineto{\pgfqpoint{1.218909in}{1.963800in}}%
\pgfpathlineto{\pgfqpoint{1.230956in}{1.961665in}}%
\pgfpathlineto{\pgfqpoint{1.243002in}{1.959360in}}%
\pgfpathlineto{\pgfqpoint{1.255048in}{1.956883in}}%
\pgfpathlineto{\pgfqpoint{1.267094in}{1.954235in}}%
\pgfpathlineto{\pgfqpoint{1.274804in}{1.952430in}}%
\pgfpathlineto{\pgfqpoint{1.279140in}{1.951412in}}%
\pgfpathlineto{\pgfqpoint{1.291186in}{1.948411in}}%
\pgfpathlineto{\pgfqpoint{1.303232in}{1.945234in}}%
\pgfpathlineto{\pgfqpoint{1.315278in}{1.941877in}}%
\pgfpathlineto{\pgfqpoint{1.320375in}{1.940384in}}%
\pgfpathlineto{\pgfqpoint{1.327325in}{1.938338in}}%
\pgfpathlineto{\pgfqpoint{1.339371in}{1.934614in}}%
\pgfpathlineto{\pgfqpoint{1.351417in}{1.930704in}}%
\pgfpathlineto{\pgfqpoint{1.358387in}{1.928338in}}%
\pgfpathlineto{\pgfqpoint{1.363463in}{1.926605in}}%
\pgfpathlineto{\pgfqpoint{1.375509in}{1.922312in}}%
\pgfpathlineto{\pgfqpoint{1.387555in}{1.917828in}}%
\pgfpathlineto{\pgfqpoint{1.391521in}{1.916292in}}%
\pgfpathlineto{\pgfqpoint{1.399601in}{1.913142in}}%
\pgfpathlineto{\pgfqpoint{1.411647in}{1.908256in}}%
\pgfpathlineto{\pgfqpoint{1.421154in}{1.904246in}}%
\pgfpathlineto{\pgfqpoint{1.423693in}{1.903167in}}%
\pgfpathlineto{\pgfqpoint{1.435740in}{1.897865in}}%
\pgfpathlineto{\pgfqpoint{1.447786in}{1.892356in}}%
\pgfpathlineto{\pgfqpoint{1.448116in}{1.892200in}}%
\pgfpathlineto{\pgfqpoint{1.459832in}{1.886622in}}%
\pgfpathlineto{\pgfqpoint{1.471878in}{1.880673in}}%
\pgfpathlineto{\pgfqpoint{1.472899in}{1.880154in}}%
\pgfpathlineto{\pgfqpoint{1.483924in}{1.874491in}}%
\pgfpathlineto{\pgfqpoint{1.495927in}{1.868107in}}%
\pgfpathlineto{\pgfqpoint{1.495970in}{1.868084in}}%
\pgfpathlineto{\pgfqpoint{1.508016in}{1.861430in}}%
\pgfpathlineto{\pgfqpoint{1.517424in}{1.856061in}}%
\pgfpathlineto{\pgfqpoint{1.520062in}{1.854540in}}%
\pgfpathlineto{\pgfqpoint{1.532108in}{1.847394in}}%
\pgfpathlineto{\pgfqpoint{1.537640in}{1.844015in}}%
\pgfpathlineto{\pgfqpoint{1.544155in}{1.839991in}}%
\pgfpathlineto{\pgfqpoint{1.556201in}{1.832330in}}%
\pgfpathlineto{\pgfqpoint{1.556754in}{1.831969in}}%
\pgfpathlineto{\pgfqpoint{1.568247in}{1.824383in}}%
\pgfpathlineto{\pgfqpoint{1.574819in}{1.819923in}}%
\pgfpathlineto{\pgfqpoint{1.580293in}{1.816160in}}%
\pgfpathlineto{\pgfqpoint{1.592025in}{1.807877in}}%
\pgfpathlineto{\pgfqpoint{1.592339in}{1.807652in}}%
\pgfpathlineto{\pgfqpoint{1.604385in}{1.798829in}}%
\pgfpathlineto{\pgfqpoint{1.608378in}{1.795831in}}%
\pgfpathlineto{\pgfqpoint{1.616431in}{1.789696in}}%
\pgfpathlineto{\pgfqpoint{1.624005in}{1.783785in}}%
\pgfpathlineto{\pgfqpoint{1.628477in}{1.780241in}}%
\pgfpathlineto{\pgfqpoint{1.638959in}{1.771738in}}%
\pgfpathlineto{\pgfqpoint{1.640524in}{1.770450in}}%
\pgfpathlineto{\pgfqpoint{1.652570in}{1.760303in}}%
\pgfpathlineto{\pgfqpoint{1.653279in}{1.759692in}}%
\pgfpathlineto{\pgfqpoint{1.664616in}{1.749779in}}%
\pgfpathlineto{\pgfqpoint{1.667003in}{1.747646in}}%
\pgfpathlineto{\pgfqpoint{1.676662in}{1.738870in}}%
\pgfpathlineto{\pgfqpoint{1.680187in}{1.735600in}}%
\pgfpathlineto{\pgfqpoint{1.688708in}{1.727554in}}%
\pgfpathlineto{\pgfqpoint{1.692862in}{1.723554in}}%
\pgfpathlineto{\pgfqpoint{1.700754in}{1.715811in}}%
\pgfpathlineto{\pgfqpoint{1.705057in}{1.711508in}}%
\pgfpathlineto{\pgfqpoint{1.712800in}{1.703615in}}%
\pgfpathlineto{\pgfqpoint{1.716801in}{1.699462in}}%
\pgfpathlineto{\pgfqpoint{1.724846in}{1.690941in}}%
\pgfpathlineto{\pgfqpoint{1.728116in}{1.687416in}}%
\pgfpathlineto{\pgfqpoint{1.736892in}{1.677757in}}%
\pgfpathlineto{\pgfqpoint{1.739025in}{1.675370in}}%
\pgfpathlineto{\pgfqpoint{1.748939in}{1.664033in}}%
\pgfpathlineto{\pgfqpoint{1.749549in}{1.663323in}}%
\pgfpathlineto{\pgfqpoint{1.759696in}{1.651277in}}%
\pgfpathlineto{\pgfqpoint{1.760985in}{1.649713in}}%
\pgfpathlineto{\pgfqpoint{1.769487in}{1.639231in}}%
\pgfpathlineto{\pgfqpoint{1.773031in}{1.634758in}}%
\pgfpathlineto{\pgfqpoint{1.778942in}{1.627185in}}%
\pgfpathlineto{\pgfqpoint{1.785077in}{1.619132in}}%
\pgfpathlineto{\pgfqpoint{1.788075in}{1.615139in}}%
\pgfpathlineto{\pgfqpoint{1.796898in}{1.603093in}}%
\pgfpathlineto{\pgfqpoint{1.797123in}{1.602779in}}%
\pgfpathlineto{\pgfqpoint{1.805407in}{1.591047in}}%
\pgfpathlineto{\pgfqpoint{1.809169in}{1.585573in}}%
\pgfpathlineto{\pgfqpoint{1.813629in}{1.579001in}}%
\pgfpathlineto{\pgfqpoint{1.821215in}{1.567507in}}%
\pgfpathlineto{\pgfqpoint{1.821576in}{1.566955in}}%
\pgfpathlineto{\pgfqpoint{1.829237in}{1.554908in}}%
\pgfpathlineto{\pgfqpoint{1.833261in}{1.548394in}}%
\pgfpathlineto{\pgfqpoint{1.836640in}{1.542862in}}%
\pgfpathlineto{\pgfqpoint{1.843786in}{1.530816in}}%
\pgfpathlineto{\pgfqpoint{1.845307in}{1.528178in}}%
\pgfpathlineto{\pgfqpoint{1.850676in}{1.518770in}}%
\pgfpathlineto{\pgfqpoint{1.857331in}{1.506724in}}%
\pgfpathlineto{\pgfqpoint{1.857354in}{1.506681in}}%
\pgfpathlineto{\pgfqpoint{1.863737in}{1.494678in}}%
\pgfpathlineto{\pgfqpoint{1.869400in}{1.483653in}}%
\pgfpathlineto{\pgfqpoint{1.869920in}{1.482632in}}%
\pgfpathlineto{\pgfqpoint{1.875868in}{1.470586in}}%
\pgfpathlineto{\pgfqpoint{1.881446in}{1.458870in}}%
\pgfpathlineto{\pgfqpoint{1.881602in}{1.458539in}}%
\pgfpathlineto{\pgfqpoint{1.887111in}{1.446493in}}%
\pgfpathlineto{\pgfqpoint{1.892413in}{1.434447in}}%
\pgfpathlineto{\pgfqpoint{1.893492in}{1.431908in}}%
\pgfpathlineto{\pgfqpoint{1.897502in}{1.422401in}}%
\pgfpathlineto{\pgfqpoint{1.902388in}{1.410355in}}%
\pgfpathlineto{\pgfqpoint{1.905538in}{1.402275in}}%
\pgfpathlineto{\pgfqpoint{1.907074in}{1.398309in}}%
\pgfpathlineto{\pgfqpoint{1.911559in}{1.386263in}}%
\pgfpathlineto{\pgfqpoint{1.915851in}{1.374217in}}%
\pgfpathlineto{\pgfqpoint{1.917584in}{1.369141in}}%
\pgfpathlineto{\pgfqpoint{1.919950in}{1.362171in}}%
\pgfpathlineto{\pgfqpoint{1.923860in}{1.350124in}}%
\pgfpathlineto{\pgfqpoint{1.927584in}{1.338078in}}%
\pgfpathlineto{\pgfqpoint{1.929630in}{1.331129in}}%
\pgfpathlineto{\pgfqpoint{1.931124in}{1.326032in}}%
\pgfpathlineto{\pgfqpoint{1.934480in}{1.313986in}}%
\pgfpathlineto{\pgfqpoint{1.937658in}{1.301940in}}%
\pgfpathlineto{\pgfqpoint{1.940658in}{1.289894in}}%
\pgfpathlineto{\pgfqpoint{1.941676in}{1.285558in}}%
\pgfpathlineto{\pgfqpoint{1.943481in}{1.277848in}}%
\pgfpathlineto{\pgfqpoint{1.946130in}{1.265802in}}%
\pgfpathlineto{\pgfqpoint{1.948606in}{1.253756in}}%
\pgfpathlineto{\pgfqpoint{1.950911in}{1.241709in}}%
\pgfpathlineto{\pgfqpoint{1.953046in}{1.229663in}}%
\pgfpathlineto{\pgfqpoint{1.953723in}{1.225525in}}%
\pgfpathlineto{\pgfqpoint{1.955012in}{1.217617in}}%
\pgfpathlineto{\pgfqpoint{1.956810in}{1.205571in}}%
\pgfpathlineto{\pgfqpoint{1.958442in}{1.193525in}}%
\pgfpathlineto{\pgfqpoint{1.959908in}{1.181479in}}%
\pgfpathlineto{\pgfqpoint{1.961209in}{1.169433in}}%
\pgfpathlineto{\pgfqpoint{1.962346in}{1.157387in}}%
\pgfpathlineto{\pgfqpoint{1.963320in}{1.145340in}}%
\pgfpathlineto{\pgfqpoint{1.964130in}{1.133294in}}%
\pgfpathlineto{\pgfqpoint{1.964778in}{1.121248in}}%
\pgfpathlineto{\pgfqpoint{1.965263in}{1.109202in}}%
\pgfpathlineto{\pgfqpoint{1.965587in}{1.097156in}}%
\pgfpathlineto{\pgfqpoint{1.965748in}{1.085110in}}%
\pgfpathlineto{\pgfqpoint{1.965748in}{1.073064in}}%
\pgfpathlineto{\pgfqpoint{1.965587in}{1.061018in}}%
\pgfpathlineto{\pgfqpoint{1.965263in}{1.048972in}}%
\pgfpathlineto{\pgfqpoint{1.964778in}{1.036925in}}%
\pgfpathlineto{\pgfqpoint{1.964130in}{1.024879in}}%
\pgfpathlineto{\pgfqpoint{1.963320in}{1.012833in}}%
\pgfpathlineto{\pgfqpoint{1.962346in}{1.000787in}}%
\pgfpathlineto{\pgfqpoint{1.961209in}{0.988741in}}%
\pgfpathlineto{\pgfqpoint{1.959908in}{0.976695in}}%
\pgfpathlineto{\pgfqpoint{1.958442in}{0.964649in}}%
\pgfpathlineto{\pgfqpoint{1.956810in}{0.952603in}}%
\pgfpathlineto{\pgfqpoint{1.955012in}{0.940557in}}%
\pgfpathlineto{\pgfqpoint{1.953723in}{0.932648in}}%
\pgfpathlineto{\pgfqpoint{1.953046in}{0.928510in}}%
\pgfpathlineto{\pgfqpoint{1.950911in}{0.916464in}}%
\pgfpathlineto{\pgfqpoint{1.948606in}{0.904418in}}%
\pgfpathlineto{\pgfqpoint{1.946130in}{0.892372in}}%
\pgfpathlineto{\pgfqpoint{1.943481in}{0.880326in}}%
\pgfpathlineto{\pgfqpoint{1.941676in}{0.872616in}}%
\pgfpathlineto{\pgfqpoint{1.940658in}{0.868280in}}%
\pgfpathlineto{\pgfqpoint{1.937658in}{0.856234in}}%
\pgfpathlineto{\pgfqpoint{1.934480in}{0.844188in}}%
\pgfpathlineto{\pgfqpoint{1.931124in}{0.832141in}}%
\pgfpathlineto{\pgfqpoint{1.929630in}{0.827045in}}%
\pgfpathlineto{\pgfqpoint{1.927584in}{0.820095in}}%
\pgfpathlineto{\pgfqpoint{1.923860in}{0.808049in}}%
\pgfpathlineto{\pgfqpoint{1.919950in}{0.796003in}}%
\pgfpathlineto{\pgfqpoint{1.917584in}{0.789033in}}%
\pgfpathlineto{\pgfqpoint{1.915851in}{0.783957in}}%
\pgfpathlineto{\pgfqpoint{1.911559in}{0.771911in}}%
\pgfpathlineto{\pgfqpoint{1.907074in}{0.759865in}}%
\pgfpathlineto{\pgfqpoint{1.905538in}{0.755899in}}%
\pgfpathlineto{\pgfqpoint{1.902388in}{0.747819in}}%
\pgfpathlineto{\pgfqpoint{1.897502in}{0.735773in}}%
\pgfpathlineto{\pgfqpoint{1.893492in}{0.726265in}}%
\pgfpathlineto{\pgfqpoint{1.892413in}{0.723726in}}%
\pgfpathlineto{\pgfqpoint{1.887111in}{0.711680in}}%
\pgfpathlineto{\pgfqpoint{1.881602in}{0.699634in}}%
\pgfpathlineto{\pgfqpoint{1.881446in}{0.699304in}}%
\pgfpathlineto{\pgfqpoint{1.875868in}{0.687588in}}%
\pgfpathlineto{\pgfqpoint{1.869920in}{0.675542in}}%
\pgfpathlineto{\pgfqpoint{1.869400in}{0.674521in}}%
\pgfpathlineto{\pgfqpoint{1.863737in}{0.663496in}}%
\pgfpathlineto{\pgfqpoint{1.857354in}{0.651493in}}%
\pgfpathlineto{\pgfqpoint{1.857331in}{0.651450in}}%
\pgfpathlineto{\pgfqpoint{1.850676in}{0.639404in}}%
\pgfpathlineto{\pgfqpoint{1.845307in}{0.629996in}}%
\pgfpathlineto{\pgfqpoint{1.843786in}{0.627358in}}%
\pgfpathlineto{\pgfqpoint{1.836640in}{0.615311in}}%
\pgfpathlineto{\pgfqpoint{1.833261in}{0.609780in}}%
\pgfpathlineto{\pgfqpoint{1.829237in}{0.603265in}}%
\pgfpathlineto{\pgfqpoint{1.821576in}{0.591219in}}%
\pgfpathlineto{\pgfqpoint{1.821215in}{0.590666in}}%
\pgfpathlineto{\pgfqpoint{1.813629in}{0.579173in}}%
\pgfpathlineto{\pgfqpoint{1.809169in}{0.572601in}}%
\pgfpathlineto{\pgfqpoint{1.805407in}{0.567127in}}%
\pgfpathlineto{\pgfqpoint{1.797123in}{0.555395in}}%
\pgfpathlineto{\pgfqpoint{1.796898in}{0.555081in}}%
\pgfpathlineto{\pgfqpoint{1.788075in}{0.543035in}}%
\pgfpathlineto{\pgfqpoint{1.785077in}{0.539042in}}%
\pgfpathlineto{\pgfqpoint{1.778942in}{0.530989in}}%
\pgfpathlineto{\pgfqpoint{1.773031in}{0.523415in}}%
\pgfpathlineto{\pgfqpoint{1.769487in}{0.518942in}}%
\pgfpathlineto{\pgfqpoint{1.760985in}{0.508460in}}%
\pgfpathlineto{\pgfqpoint{1.759696in}{0.506896in}}%
\pgfpathlineto{\pgfqpoint{1.749549in}{0.494850in}}%
\pgfpathlineto{\pgfqpoint{1.748939in}{0.494141in}}%
\pgfpathlineto{\pgfqpoint{1.739025in}{0.482804in}}%
\pgfpathlineto{\pgfqpoint{1.736892in}{0.480416in}}%
\pgfpathlineto{\pgfqpoint{1.728116in}{0.470758in}}%
\pgfpathlineto{\pgfqpoint{1.724846in}{0.467233in}}%
\pgfpathlineto{\pgfqpoint{1.716801in}{0.458712in}}%
\pgfpathlineto{\pgfqpoint{1.712800in}{0.454558in}}%
\pgfpathlineto{\pgfqpoint{1.705057in}{0.446666in}}%
\pgfpathlineto{\pgfqpoint{1.700754in}{0.442363in}}%
\pgfpathlineto{\pgfqpoint{1.692862in}{0.434620in}}%
\pgfpathlineto{\pgfqpoint{1.688708in}{0.430619in}}%
\pgfpathlineto{\pgfqpoint{1.680187in}{0.422574in}}%
\pgfpathlineto{\pgfqpoint{1.676662in}{0.419304in}}%
\pgfpathlineto{\pgfqpoint{1.667003in}{0.410527in}}%
\pgfpathlineto{\pgfqpoint{1.664616in}{0.408394in}}%
\pgfpathlineto{\pgfqpoint{1.653279in}{0.398481in}}%
\pgfpathlineto{\pgfqpoint{1.652570in}{0.397871in}}%
\pgfpathlineto{\pgfqpoint{1.640524in}{0.387724in}}%
\pgfpathlineto{\pgfqpoint{1.638959in}{0.386435in}}%
\pgfpathlineto{\pgfqpoint{1.628477in}{0.377933in}}%
\pgfpathlineto{\pgfqpoint{1.624005in}{0.374389in}}%
\pgfpathlineto{\pgfqpoint{1.616431in}{0.368478in}}%
\pgfpathlineto{\pgfqpoint{1.608378in}{0.362343in}}%
\pgfpathlineto{\pgfqpoint{1.604385in}{0.359344in}}%
\pgfpathlineto{\pgfqpoint{1.592339in}{0.350522in}}%
\pgfpathlineto{\pgfqpoint{1.592025in}{0.350297in}}%
\pgfpathlineto{\pgfqpoint{1.580293in}{0.342013in}}%
\pgfpathlineto{\pgfqpoint{1.574819in}{0.338251in}}%
\pgfpathlineto{\pgfqpoint{1.568247in}{0.333791in}}%
\pgfpathlineto{\pgfqpoint{1.556754in}{0.326205in}}%
\pgfpathlineto{\pgfqpoint{1.556201in}{0.325844in}}%
\pgfpathlineto{\pgfqpoint{1.544155in}{0.318183in}}%
\pgfpathlineto{\pgfqpoint{1.537640in}{0.314159in}}%
\pgfpathlineto{\pgfqpoint{1.532108in}{0.310780in}}%
\pgfpathlineto{\pgfqpoint{1.520062in}{0.303634in}}%
\pgfpathlineto{\pgfqpoint{1.517424in}{0.302112in}}%
\pgfpathlineto{\pgfqpoint{1.508016in}{0.296744in}}%
\pgfpathlineto{\pgfqpoint{1.495970in}{0.290089in}}%
\pgfpathlineto{\pgfqpoint{1.495927in}{0.290066in}}%
\pgfpathlineto{\pgfqpoint{1.483924in}{0.283683in}}%
\pgfpathlineto{\pgfqpoint{1.472899in}{0.278020in}}%
\pgfpathlineto{\pgfqpoint{1.471878in}{0.277500in}}%
\pgfpathlineto{\pgfqpoint{1.459832in}{0.271552in}}%
\pgfpathlineto{\pgfqpoint{1.448116in}{0.265974in}}%
\pgfpathlineto{\pgfqpoint{1.447786in}{0.265818in}}%
\pgfpathlineto{\pgfqpoint{1.435740in}{0.260309in}}%
\pgfpathlineto{\pgfqpoint{1.423693in}{0.255007in}}%
\pgfpathlineto{\pgfqpoint{1.421154in}{0.253928in}}%
\pgfpathlineto{\pgfqpoint{1.411647in}{0.249918in}}%
\pgfpathlineto{\pgfqpoint{1.399601in}{0.245032in}}%
\pgfpathlineto{\pgfqpoint{1.391521in}{0.241882in}}%
\pgfpathlineto{\pgfqpoint{1.387555in}{0.240346in}}%
\pgfpathlineto{\pgfqpoint{1.375509in}{0.235861in}}%
\pgfpathlineto{\pgfqpoint{1.363463in}{0.231569in}}%
\pgfpathlineto{\pgfqpoint{1.358387in}{0.229836in}}%
\pgfpathlineto{\pgfqpoint{1.351417in}{0.227470in}}%
\pgfpathlineto{\pgfqpoint{1.339371in}{0.223560in}}%
\pgfpathlineto{\pgfqpoint{1.327325in}{0.219836in}}%
\pgfpathlineto{\pgfqpoint{1.320375in}{0.217790in}}%
\pgfpathlineto{\pgfqpoint{1.315278in}{0.216296in}}%
\pgfpathlineto{\pgfqpoint{1.303232in}{0.212940in}}%
\pgfpathlineto{\pgfqpoint{1.291186in}{0.209762in}}%
\pgfpathlineto{\pgfqpoint{1.279140in}{0.206762in}}%
\pgfpathlineto{\pgfqpoint{1.274804in}{0.205743in}}%
\pgfpathlineto{\pgfqpoint{1.267094in}{0.203939in}}%
\pgfpathlineto{\pgfqpoint{1.255048in}{0.201290in}}%
\pgfpathlineto{\pgfqpoint{1.243002in}{0.198814in}}%
\pgfpathlineto{\pgfqpoint{1.230956in}{0.196509in}}%
\pgfpathlineto{\pgfqpoint{1.218909in}{0.194374in}}%
\pgfpathlineto{\pgfqpoint{1.214772in}{0.193697in}}%
\pgfpathlineto{\pgfqpoint{1.206863in}{0.192408in}}%
\pgfpathlineto{\pgfqpoint{1.194817in}{0.190610in}}%
\pgfpathlineto{\pgfqpoint{1.182771in}{0.188978in}}%
\pgfpathlineto{\pgfqpoint{1.170725in}{0.187512in}}%
\pgfpathlineto{\pgfqpoint{1.158679in}{0.186211in}}%
\pgfpathlineto{\pgfqpoint{1.146633in}{0.185074in}}%
\pgfpathlineto{\pgfqpoint{1.134587in}{0.184100in}}%
\pgfpathlineto{\pgfqpoint{1.122541in}{0.183290in}}%
\pgfpathlineto{\pgfqpoint{1.110494in}{0.182642in}}%
\pgfpathlineto{\pgfqpoint{1.098448in}{0.182157in}}%
\pgfpathlineto{\pgfqpoint{1.086402in}{0.181833in}}%
\pgfpathlineto{\pgfqpoint{1.074356in}{0.181671in}}%
\pgfpathlineto{\pgfqpoint{1.062310in}{0.181671in}}%
\pgfpathlineto{\pgfqpoint{1.050264in}{0.181833in}}%
\pgfpathlineto{\pgfqpoint{1.038218in}{0.182157in}}%
\pgfpathlineto{\pgfqpoint{1.026172in}{0.182642in}}%
\pgfpathlineto{\pgfqpoint{1.014126in}{0.183290in}}%
\pgfpathlineto{\pgfqpoint{1.002079in}{0.184100in}}%
\pgfpathlineto{\pgfqpoint{0.990033in}{0.185074in}}%
\pgfpathlineto{\pgfqpoint{0.977987in}{0.186211in}}%
\pgfpathlineto{\pgfqpoint{0.965941in}{0.187512in}}%
\pgfpathlineto{\pgfqpoint{0.953895in}{0.188978in}}%
\pgfpathlineto{\pgfqpoint{0.941849in}{0.190610in}}%
\pgfpathlineto{\pgfqpoint{0.929803in}{0.192408in}}%
\pgfpathclose%
\pgfusepath{}%
\end{pgfscope}%
\begin{pgfscope}%
\pgfsetbuttcap%
\pgfsetroundjoin%
\definecolor{currentfill}{rgb}{0.000000,0.000000,0.000000}%
\pgfsetfillcolor{currentfill}%
\pgfsetlinewidth{0.803000pt}%
\definecolor{currentstroke}{rgb}{0.000000,0.000000,0.000000}%
\pgfsetstrokecolor{currentstroke}%
\pgfsetdash{}{0pt}%
\pgfsys@defobject{currentmarker}{\pgfqpoint{0.000000in}{-0.048611in}}{\pgfqpoint{0.000000in}{0.000000in}}{%
\pgfpathmoveto{\pgfqpoint{0.000000in}{0.000000in}}%
\pgfpathlineto{\pgfqpoint{0.000000in}{-0.048611in}}%
\pgfusepath{stroke,fill}%
}%
\begin{pgfscope}%
\pgfsys@transformshift{0.170897in}{1.079087in}%
\pgfsys@useobject{currentmarker}{}%
\end{pgfscope}%
\end{pgfscope}%
\begin{pgfscope}%
\pgftext[x=0.170897in,y=0.981865in,,top]{\sffamily\fontsize{10.000000}{12.000000}\selectfont -1}%
\end{pgfscope}%
\begin{pgfscope}%
\pgfsetbuttcap%
\pgfsetroundjoin%
\definecolor{currentfill}{rgb}{0.000000,0.000000,0.000000}%
\pgfsetfillcolor{currentfill}%
\pgfsetlinewidth{0.803000pt}%
\definecolor{currentstroke}{rgb}{0.000000,0.000000,0.000000}%
\pgfsetstrokecolor{currentstroke}%
\pgfsetdash{}{0pt}%
\pgfsys@defobject{currentmarker}{\pgfqpoint{0.000000in}{-0.048611in}}{\pgfqpoint{0.000000in}{0.000000in}}{%
\pgfpathmoveto{\pgfqpoint{0.000000in}{0.000000in}}%
\pgfpathlineto{\pgfqpoint{0.000000in}{-0.048611in}}%
\pgfusepath{stroke,fill}%
}%
\begin{pgfscope}%
\pgfsys@transformshift{0.619615in}{1.079087in}%
\pgfsys@useobject{currentmarker}{}%
\end{pgfscope}%
\end{pgfscope}%
\begin{pgfscope}%
\pgftext[x=0.619615in,y=0.981865in,,top]{\sffamily\fontsize{10.000000}{12.000000}\selectfont -0.5}%
\end{pgfscope}%
\begin{pgfscope}%
\pgfsetbuttcap%
\pgfsetroundjoin%
\definecolor{currentfill}{rgb}{0.000000,0.000000,0.000000}%
\pgfsetfillcolor{currentfill}%
\pgfsetlinewidth{0.803000pt}%
\definecolor{currentstroke}{rgb}{0.000000,0.000000,0.000000}%
\pgfsetstrokecolor{currentstroke}%
\pgfsetdash{}{0pt}%
\pgfsys@defobject{currentmarker}{\pgfqpoint{0.000000in}{-0.048611in}}{\pgfqpoint{0.000000in}{0.000000in}}{%
\pgfpathmoveto{\pgfqpoint{0.000000in}{0.000000in}}%
\pgfpathlineto{\pgfqpoint{0.000000in}{-0.048611in}}%
\pgfusepath{stroke,fill}%
}%
\begin{pgfscope}%
\pgfsys@transformshift{1.068333in}{1.079087in}%
\pgfsys@useobject{currentmarker}{}%
\end{pgfscope}%
\end{pgfscope}%
\begin{pgfscope}%
\pgfsetbuttcap%
\pgfsetroundjoin%
\definecolor{currentfill}{rgb}{0.000000,0.000000,0.000000}%
\pgfsetfillcolor{currentfill}%
\pgfsetlinewidth{0.803000pt}%
\definecolor{currentstroke}{rgb}{0.000000,0.000000,0.000000}%
\pgfsetstrokecolor{currentstroke}%
\pgfsetdash{}{0pt}%
\pgfsys@defobject{currentmarker}{\pgfqpoint{0.000000in}{-0.048611in}}{\pgfqpoint{0.000000in}{0.000000in}}{%
\pgfpathmoveto{\pgfqpoint{0.000000in}{0.000000in}}%
\pgfpathlineto{\pgfqpoint{0.000000in}{-0.048611in}}%
\pgfusepath{stroke,fill}%
}%
\begin{pgfscope}%
\pgfsys@transformshift{1.517051in}{1.079087in}%
\pgfsys@useobject{currentmarker}{}%
\end{pgfscope}%
\end{pgfscope}%
\begin{pgfscope}%
\pgftext[x=1.517051in,y=0.981865in,,top]{\sffamily\fontsize{10.000000}{12.000000}\selectfont 0.5}%
\end{pgfscope}%
\begin{pgfscope}%
\pgfsetbuttcap%
\pgfsetroundjoin%
\definecolor{currentfill}{rgb}{0.000000,0.000000,0.000000}%
\pgfsetfillcolor{currentfill}%
\pgfsetlinewidth{0.803000pt}%
\definecolor{currentstroke}{rgb}{0.000000,0.000000,0.000000}%
\pgfsetstrokecolor{currentstroke}%
\pgfsetdash{}{0pt}%
\pgfsys@defobject{currentmarker}{\pgfqpoint{0.000000in}{-0.048611in}}{\pgfqpoint{0.000000in}{0.000000in}}{%
\pgfpathmoveto{\pgfqpoint{0.000000in}{0.000000in}}%
\pgfpathlineto{\pgfqpoint{0.000000in}{-0.048611in}}%
\pgfusepath{stroke,fill}%
}%
\begin{pgfscope}%
\pgfsys@transformshift{1.965769in}{1.079087in}%
\pgfsys@useobject{currentmarker}{}%
\end{pgfscope}%
\end{pgfscope}%
\begin{pgfscope}%
\pgftext[x=1.965769in,y=0.981865in,,top]{\sffamily\fontsize{10.000000}{12.000000}\selectfont 1}%
\end{pgfscope}%
\begin{pgfscope}%
\pgfsetbuttcap%
\pgfsetroundjoin%
\definecolor{currentfill}{rgb}{0.000000,0.000000,0.000000}%
\pgfsetfillcolor{currentfill}%
\pgfsetlinewidth{0.602250pt}%
\definecolor{currentstroke}{rgb}{0.000000,0.000000,0.000000}%
\pgfsetstrokecolor{currentstroke}%
\pgfsetdash{}{0pt}%
\pgfsys@defobject{currentmarker}{\pgfqpoint{0.000000in}{-0.027778in}}{\pgfqpoint{0.000000in}{0.000000in}}{%
\pgfpathmoveto{\pgfqpoint{0.000000in}{0.000000in}}%
\pgfpathlineto{\pgfqpoint{0.000000in}{-0.027778in}}%
\pgfusepath{stroke,fill}%
}%
\begin{pgfscope}%
\pgfsys@transformshift{0.260641in}{1.079087in}%
\pgfsys@useobject{currentmarker}{}%
\end{pgfscope}%
\end{pgfscope}%
\begin{pgfscope}%
\pgfsetbuttcap%
\pgfsetroundjoin%
\definecolor{currentfill}{rgb}{0.000000,0.000000,0.000000}%
\pgfsetfillcolor{currentfill}%
\pgfsetlinewidth{0.602250pt}%
\definecolor{currentstroke}{rgb}{0.000000,0.000000,0.000000}%
\pgfsetstrokecolor{currentstroke}%
\pgfsetdash{}{0pt}%
\pgfsys@defobject{currentmarker}{\pgfqpoint{0.000000in}{-0.027778in}}{\pgfqpoint{0.000000in}{0.000000in}}{%
\pgfpathmoveto{\pgfqpoint{0.000000in}{0.000000in}}%
\pgfpathlineto{\pgfqpoint{0.000000in}{-0.027778in}}%
\pgfusepath{stroke,fill}%
}%
\begin{pgfscope}%
\pgfsys@transformshift{0.350385in}{1.079087in}%
\pgfsys@useobject{currentmarker}{}%
\end{pgfscope}%
\end{pgfscope}%
\begin{pgfscope}%
\pgfsetbuttcap%
\pgfsetroundjoin%
\definecolor{currentfill}{rgb}{0.000000,0.000000,0.000000}%
\pgfsetfillcolor{currentfill}%
\pgfsetlinewidth{0.602250pt}%
\definecolor{currentstroke}{rgb}{0.000000,0.000000,0.000000}%
\pgfsetstrokecolor{currentstroke}%
\pgfsetdash{}{0pt}%
\pgfsys@defobject{currentmarker}{\pgfqpoint{0.000000in}{-0.027778in}}{\pgfqpoint{0.000000in}{0.000000in}}{%
\pgfpathmoveto{\pgfqpoint{0.000000in}{0.000000in}}%
\pgfpathlineto{\pgfqpoint{0.000000in}{-0.027778in}}%
\pgfusepath{stroke,fill}%
}%
\begin{pgfscope}%
\pgfsys@transformshift{0.440128in}{1.079087in}%
\pgfsys@useobject{currentmarker}{}%
\end{pgfscope}%
\end{pgfscope}%
\begin{pgfscope}%
\pgfsetbuttcap%
\pgfsetroundjoin%
\definecolor{currentfill}{rgb}{0.000000,0.000000,0.000000}%
\pgfsetfillcolor{currentfill}%
\pgfsetlinewidth{0.602250pt}%
\definecolor{currentstroke}{rgb}{0.000000,0.000000,0.000000}%
\pgfsetstrokecolor{currentstroke}%
\pgfsetdash{}{0pt}%
\pgfsys@defobject{currentmarker}{\pgfqpoint{0.000000in}{-0.027778in}}{\pgfqpoint{0.000000in}{0.000000in}}{%
\pgfpathmoveto{\pgfqpoint{0.000000in}{0.000000in}}%
\pgfpathlineto{\pgfqpoint{0.000000in}{-0.027778in}}%
\pgfusepath{stroke,fill}%
}%
\begin{pgfscope}%
\pgfsys@transformshift{0.529872in}{1.079087in}%
\pgfsys@useobject{currentmarker}{}%
\end{pgfscope}%
\end{pgfscope}%
\begin{pgfscope}%
\pgfsetbuttcap%
\pgfsetroundjoin%
\definecolor{currentfill}{rgb}{0.000000,0.000000,0.000000}%
\pgfsetfillcolor{currentfill}%
\pgfsetlinewidth{0.602250pt}%
\definecolor{currentstroke}{rgb}{0.000000,0.000000,0.000000}%
\pgfsetstrokecolor{currentstroke}%
\pgfsetdash{}{0pt}%
\pgfsys@defobject{currentmarker}{\pgfqpoint{0.000000in}{-0.027778in}}{\pgfqpoint{0.000000in}{0.000000in}}{%
\pgfpathmoveto{\pgfqpoint{0.000000in}{0.000000in}}%
\pgfpathlineto{\pgfqpoint{0.000000in}{-0.027778in}}%
\pgfusepath{stroke,fill}%
}%
\begin{pgfscope}%
\pgfsys@transformshift{0.619615in}{1.079087in}%
\pgfsys@useobject{currentmarker}{}%
\end{pgfscope}%
\end{pgfscope}%
\begin{pgfscope}%
\pgfsetbuttcap%
\pgfsetroundjoin%
\definecolor{currentfill}{rgb}{0.000000,0.000000,0.000000}%
\pgfsetfillcolor{currentfill}%
\pgfsetlinewidth{0.602250pt}%
\definecolor{currentstroke}{rgb}{0.000000,0.000000,0.000000}%
\pgfsetstrokecolor{currentstroke}%
\pgfsetdash{}{0pt}%
\pgfsys@defobject{currentmarker}{\pgfqpoint{0.000000in}{-0.027778in}}{\pgfqpoint{0.000000in}{0.000000in}}{%
\pgfpathmoveto{\pgfqpoint{0.000000in}{0.000000in}}%
\pgfpathlineto{\pgfqpoint{0.000000in}{-0.027778in}}%
\pgfusepath{stroke,fill}%
}%
\begin{pgfscope}%
\pgfsys@transformshift{0.709359in}{1.079087in}%
\pgfsys@useobject{currentmarker}{}%
\end{pgfscope}%
\end{pgfscope}%
\begin{pgfscope}%
\pgfsetbuttcap%
\pgfsetroundjoin%
\definecolor{currentfill}{rgb}{0.000000,0.000000,0.000000}%
\pgfsetfillcolor{currentfill}%
\pgfsetlinewidth{0.602250pt}%
\definecolor{currentstroke}{rgb}{0.000000,0.000000,0.000000}%
\pgfsetstrokecolor{currentstroke}%
\pgfsetdash{}{0pt}%
\pgfsys@defobject{currentmarker}{\pgfqpoint{0.000000in}{-0.027778in}}{\pgfqpoint{0.000000in}{0.000000in}}{%
\pgfpathmoveto{\pgfqpoint{0.000000in}{0.000000in}}%
\pgfpathlineto{\pgfqpoint{0.000000in}{-0.027778in}}%
\pgfusepath{stroke,fill}%
}%
\begin{pgfscope}%
\pgfsys@transformshift{0.799102in}{1.079087in}%
\pgfsys@useobject{currentmarker}{}%
\end{pgfscope}%
\end{pgfscope}%
\begin{pgfscope}%
\pgfsetbuttcap%
\pgfsetroundjoin%
\definecolor{currentfill}{rgb}{0.000000,0.000000,0.000000}%
\pgfsetfillcolor{currentfill}%
\pgfsetlinewidth{0.602250pt}%
\definecolor{currentstroke}{rgb}{0.000000,0.000000,0.000000}%
\pgfsetstrokecolor{currentstroke}%
\pgfsetdash{}{0pt}%
\pgfsys@defobject{currentmarker}{\pgfqpoint{0.000000in}{-0.027778in}}{\pgfqpoint{0.000000in}{0.000000in}}{%
\pgfpathmoveto{\pgfqpoint{0.000000in}{0.000000in}}%
\pgfpathlineto{\pgfqpoint{0.000000in}{-0.027778in}}%
\pgfusepath{stroke,fill}%
}%
\begin{pgfscope}%
\pgfsys@transformshift{0.888846in}{1.079087in}%
\pgfsys@useobject{currentmarker}{}%
\end{pgfscope}%
\end{pgfscope}%
\begin{pgfscope}%
\pgfsetbuttcap%
\pgfsetroundjoin%
\definecolor{currentfill}{rgb}{0.000000,0.000000,0.000000}%
\pgfsetfillcolor{currentfill}%
\pgfsetlinewidth{0.602250pt}%
\definecolor{currentstroke}{rgb}{0.000000,0.000000,0.000000}%
\pgfsetstrokecolor{currentstroke}%
\pgfsetdash{}{0pt}%
\pgfsys@defobject{currentmarker}{\pgfqpoint{0.000000in}{-0.027778in}}{\pgfqpoint{0.000000in}{0.000000in}}{%
\pgfpathmoveto{\pgfqpoint{0.000000in}{0.000000in}}%
\pgfpathlineto{\pgfqpoint{0.000000in}{-0.027778in}}%
\pgfusepath{stroke,fill}%
}%
\begin{pgfscope}%
\pgfsys@transformshift{0.978589in}{1.079087in}%
\pgfsys@useobject{currentmarker}{}%
\end{pgfscope}%
\end{pgfscope}%
\begin{pgfscope}%
\pgfsetbuttcap%
\pgfsetroundjoin%
\definecolor{currentfill}{rgb}{0.000000,0.000000,0.000000}%
\pgfsetfillcolor{currentfill}%
\pgfsetlinewidth{0.602250pt}%
\definecolor{currentstroke}{rgb}{0.000000,0.000000,0.000000}%
\pgfsetstrokecolor{currentstroke}%
\pgfsetdash{}{0pt}%
\pgfsys@defobject{currentmarker}{\pgfqpoint{0.000000in}{-0.027778in}}{\pgfqpoint{0.000000in}{0.000000in}}{%
\pgfpathmoveto{\pgfqpoint{0.000000in}{0.000000in}}%
\pgfpathlineto{\pgfqpoint{0.000000in}{-0.027778in}}%
\pgfusepath{stroke,fill}%
}%
\begin{pgfscope}%
\pgfsys@transformshift{1.068333in}{1.079087in}%
\pgfsys@useobject{currentmarker}{}%
\end{pgfscope}%
\end{pgfscope}%
\begin{pgfscope}%
\pgfsetbuttcap%
\pgfsetroundjoin%
\definecolor{currentfill}{rgb}{0.000000,0.000000,0.000000}%
\pgfsetfillcolor{currentfill}%
\pgfsetlinewidth{0.602250pt}%
\definecolor{currentstroke}{rgb}{0.000000,0.000000,0.000000}%
\pgfsetstrokecolor{currentstroke}%
\pgfsetdash{}{0pt}%
\pgfsys@defobject{currentmarker}{\pgfqpoint{0.000000in}{-0.027778in}}{\pgfqpoint{0.000000in}{0.000000in}}{%
\pgfpathmoveto{\pgfqpoint{0.000000in}{0.000000in}}%
\pgfpathlineto{\pgfqpoint{0.000000in}{-0.027778in}}%
\pgfusepath{stroke,fill}%
}%
\begin{pgfscope}%
\pgfsys@transformshift{1.158077in}{1.079087in}%
\pgfsys@useobject{currentmarker}{}%
\end{pgfscope}%
\end{pgfscope}%
\begin{pgfscope}%
\pgfsetbuttcap%
\pgfsetroundjoin%
\definecolor{currentfill}{rgb}{0.000000,0.000000,0.000000}%
\pgfsetfillcolor{currentfill}%
\pgfsetlinewidth{0.602250pt}%
\definecolor{currentstroke}{rgb}{0.000000,0.000000,0.000000}%
\pgfsetstrokecolor{currentstroke}%
\pgfsetdash{}{0pt}%
\pgfsys@defobject{currentmarker}{\pgfqpoint{0.000000in}{-0.027778in}}{\pgfqpoint{0.000000in}{0.000000in}}{%
\pgfpathmoveto{\pgfqpoint{0.000000in}{0.000000in}}%
\pgfpathlineto{\pgfqpoint{0.000000in}{-0.027778in}}%
\pgfusepath{stroke,fill}%
}%
\begin{pgfscope}%
\pgfsys@transformshift{1.247820in}{1.079087in}%
\pgfsys@useobject{currentmarker}{}%
\end{pgfscope}%
\end{pgfscope}%
\begin{pgfscope}%
\pgfsetbuttcap%
\pgfsetroundjoin%
\definecolor{currentfill}{rgb}{0.000000,0.000000,0.000000}%
\pgfsetfillcolor{currentfill}%
\pgfsetlinewidth{0.602250pt}%
\definecolor{currentstroke}{rgb}{0.000000,0.000000,0.000000}%
\pgfsetstrokecolor{currentstroke}%
\pgfsetdash{}{0pt}%
\pgfsys@defobject{currentmarker}{\pgfqpoint{0.000000in}{-0.027778in}}{\pgfqpoint{0.000000in}{0.000000in}}{%
\pgfpathmoveto{\pgfqpoint{0.000000in}{0.000000in}}%
\pgfpathlineto{\pgfqpoint{0.000000in}{-0.027778in}}%
\pgfusepath{stroke,fill}%
}%
\begin{pgfscope}%
\pgfsys@transformshift{1.337564in}{1.079087in}%
\pgfsys@useobject{currentmarker}{}%
\end{pgfscope}%
\end{pgfscope}%
\begin{pgfscope}%
\pgfsetbuttcap%
\pgfsetroundjoin%
\definecolor{currentfill}{rgb}{0.000000,0.000000,0.000000}%
\pgfsetfillcolor{currentfill}%
\pgfsetlinewidth{0.602250pt}%
\definecolor{currentstroke}{rgb}{0.000000,0.000000,0.000000}%
\pgfsetstrokecolor{currentstroke}%
\pgfsetdash{}{0pt}%
\pgfsys@defobject{currentmarker}{\pgfqpoint{0.000000in}{-0.027778in}}{\pgfqpoint{0.000000in}{0.000000in}}{%
\pgfpathmoveto{\pgfqpoint{0.000000in}{0.000000in}}%
\pgfpathlineto{\pgfqpoint{0.000000in}{-0.027778in}}%
\pgfusepath{stroke,fill}%
}%
\begin{pgfscope}%
\pgfsys@transformshift{1.427307in}{1.079087in}%
\pgfsys@useobject{currentmarker}{}%
\end{pgfscope}%
\end{pgfscope}%
\begin{pgfscope}%
\pgfsetbuttcap%
\pgfsetroundjoin%
\definecolor{currentfill}{rgb}{0.000000,0.000000,0.000000}%
\pgfsetfillcolor{currentfill}%
\pgfsetlinewidth{0.602250pt}%
\definecolor{currentstroke}{rgb}{0.000000,0.000000,0.000000}%
\pgfsetstrokecolor{currentstroke}%
\pgfsetdash{}{0pt}%
\pgfsys@defobject{currentmarker}{\pgfqpoint{0.000000in}{-0.027778in}}{\pgfqpoint{0.000000in}{0.000000in}}{%
\pgfpathmoveto{\pgfqpoint{0.000000in}{0.000000in}}%
\pgfpathlineto{\pgfqpoint{0.000000in}{-0.027778in}}%
\pgfusepath{stroke,fill}%
}%
\begin{pgfscope}%
\pgfsys@transformshift{1.517051in}{1.079087in}%
\pgfsys@useobject{currentmarker}{}%
\end{pgfscope}%
\end{pgfscope}%
\begin{pgfscope}%
\pgfsetbuttcap%
\pgfsetroundjoin%
\definecolor{currentfill}{rgb}{0.000000,0.000000,0.000000}%
\pgfsetfillcolor{currentfill}%
\pgfsetlinewidth{0.602250pt}%
\definecolor{currentstroke}{rgb}{0.000000,0.000000,0.000000}%
\pgfsetstrokecolor{currentstroke}%
\pgfsetdash{}{0pt}%
\pgfsys@defobject{currentmarker}{\pgfqpoint{0.000000in}{-0.027778in}}{\pgfqpoint{0.000000in}{0.000000in}}{%
\pgfpathmoveto{\pgfqpoint{0.000000in}{0.000000in}}%
\pgfpathlineto{\pgfqpoint{0.000000in}{-0.027778in}}%
\pgfusepath{stroke,fill}%
}%
\begin{pgfscope}%
\pgfsys@transformshift{1.606794in}{1.079087in}%
\pgfsys@useobject{currentmarker}{}%
\end{pgfscope}%
\end{pgfscope}%
\begin{pgfscope}%
\pgfsetbuttcap%
\pgfsetroundjoin%
\definecolor{currentfill}{rgb}{0.000000,0.000000,0.000000}%
\pgfsetfillcolor{currentfill}%
\pgfsetlinewidth{0.602250pt}%
\definecolor{currentstroke}{rgb}{0.000000,0.000000,0.000000}%
\pgfsetstrokecolor{currentstroke}%
\pgfsetdash{}{0pt}%
\pgfsys@defobject{currentmarker}{\pgfqpoint{0.000000in}{-0.027778in}}{\pgfqpoint{0.000000in}{0.000000in}}{%
\pgfpathmoveto{\pgfqpoint{0.000000in}{0.000000in}}%
\pgfpathlineto{\pgfqpoint{0.000000in}{-0.027778in}}%
\pgfusepath{stroke,fill}%
}%
\begin{pgfscope}%
\pgfsys@transformshift{1.696538in}{1.079087in}%
\pgfsys@useobject{currentmarker}{}%
\end{pgfscope}%
\end{pgfscope}%
\begin{pgfscope}%
\pgfsetbuttcap%
\pgfsetroundjoin%
\definecolor{currentfill}{rgb}{0.000000,0.000000,0.000000}%
\pgfsetfillcolor{currentfill}%
\pgfsetlinewidth{0.602250pt}%
\definecolor{currentstroke}{rgb}{0.000000,0.000000,0.000000}%
\pgfsetstrokecolor{currentstroke}%
\pgfsetdash{}{0pt}%
\pgfsys@defobject{currentmarker}{\pgfqpoint{0.000000in}{-0.027778in}}{\pgfqpoint{0.000000in}{0.000000in}}{%
\pgfpathmoveto{\pgfqpoint{0.000000in}{0.000000in}}%
\pgfpathlineto{\pgfqpoint{0.000000in}{-0.027778in}}%
\pgfusepath{stroke,fill}%
}%
\begin{pgfscope}%
\pgfsys@transformshift{1.786282in}{1.079087in}%
\pgfsys@useobject{currentmarker}{}%
\end{pgfscope}%
\end{pgfscope}%
\begin{pgfscope}%
\pgfsetbuttcap%
\pgfsetroundjoin%
\definecolor{currentfill}{rgb}{0.000000,0.000000,0.000000}%
\pgfsetfillcolor{currentfill}%
\pgfsetlinewidth{0.602250pt}%
\definecolor{currentstroke}{rgb}{0.000000,0.000000,0.000000}%
\pgfsetstrokecolor{currentstroke}%
\pgfsetdash{}{0pt}%
\pgfsys@defobject{currentmarker}{\pgfqpoint{0.000000in}{-0.027778in}}{\pgfqpoint{0.000000in}{0.000000in}}{%
\pgfpathmoveto{\pgfqpoint{0.000000in}{0.000000in}}%
\pgfpathlineto{\pgfqpoint{0.000000in}{-0.027778in}}%
\pgfusepath{stroke,fill}%
}%
\begin{pgfscope}%
\pgfsys@transformshift{1.876025in}{1.079087in}%
\pgfsys@useobject{currentmarker}{}%
\end{pgfscope}%
\end{pgfscope}%
\begin{pgfscope}%
\pgfsetbuttcap%
\pgfsetroundjoin%
\definecolor{currentfill}{rgb}{0.000000,0.000000,0.000000}%
\pgfsetfillcolor{currentfill}%
\pgfsetlinewidth{0.602250pt}%
\definecolor{currentstroke}{rgb}{0.000000,0.000000,0.000000}%
\pgfsetstrokecolor{currentstroke}%
\pgfsetdash{}{0pt}%
\pgfsys@defobject{currentmarker}{\pgfqpoint{0.000000in}{-0.027778in}}{\pgfqpoint{0.000000in}{0.000000in}}{%
\pgfpathmoveto{\pgfqpoint{0.000000in}{0.000000in}}%
\pgfpathlineto{\pgfqpoint{0.000000in}{-0.027778in}}%
\pgfusepath{stroke,fill}%
}%
\begin{pgfscope}%
\pgfsys@transformshift{1.965769in}{1.079087in}%
\pgfsys@useobject{currentmarker}{}%
\end{pgfscope}%
\end{pgfscope}%
\begin{pgfscope}%
\pgfsetbuttcap%
\pgfsetroundjoin%
\definecolor{currentfill}{rgb}{0.000000,0.000000,0.000000}%
\pgfsetfillcolor{currentfill}%
\pgfsetlinewidth{0.803000pt}%
\definecolor{currentstroke}{rgb}{0.000000,0.000000,0.000000}%
\pgfsetstrokecolor{currentstroke}%
\pgfsetdash{}{0pt}%
\pgfsys@defobject{currentmarker}{\pgfqpoint{-0.048611in}{0.000000in}}{\pgfqpoint{0.000000in}{0.000000in}}{%
\pgfpathmoveto{\pgfqpoint{0.000000in}{0.000000in}}%
\pgfpathlineto{\pgfqpoint{-0.048611in}{0.000000in}}%
\pgfusepath{stroke,fill}%
}%
\begin{pgfscope}%
\pgfsys@transformshift{1.068333in}{0.181651in}%
\pgfsys@useobject{currentmarker}{}%
\end{pgfscope}%
\end{pgfscope}%
\begin{pgfscope}%
\pgftext[x=0.832629in,y=0.128890in,left,base]{\sffamily\fontsize{10.000000}{12.000000}\selectfont -1}%
\end{pgfscope}%
\begin{pgfscope}%
\pgfsetbuttcap%
\pgfsetroundjoin%
\definecolor{currentfill}{rgb}{0.000000,0.000000,0.000000}%
\pgfsetfillcolor{currentfill}%
\pgfsetlinewidth{0.803000pt}%
\definecolor{currentstroke}{rgb}{0.000000,0.000000,0.000000}%
\pgfsetstrokecolor{currentstroke}%
\pgfsetdash{}{0pt}%
\pgfsys@defobject{currentmarker}{\pgfqpoint{-0.048611in}{0.000000in}}{\pgfqpoint{0.000000in}{0.000000in}}{%
\pgfpathmoveto{\pgfqpoint{0.000000in}{0.000000in}}%
\pgfpathlineto{\pgfqpoint{-0.048611in}{0.000000in}}%
\pgfusepath{stroke,fill}%
}%
\begin{pgfscope}%
\pgfsys@transformshift{1.068333in}{0.630369in}%
\pgfsys@useobject{currentmarker}{}%
\end{pgfscope}%
\end{pgfscope}%
\begin{pgfscope}%
\pgftext[x=0.700115in,y=0.577608in,left,base]{\sffamily\fontsize{10.000000}{12.000000}\selectfont -0.5}%
\end{pgfscope}%
\begin{pgfscope}%
\pgfsetbuttcap%
\pgfsetroundjoin%
\definecolor{currentfill}{rgb}{0.000000,0.000000,0.000000}%
\pgfsetfillcolor{currentfill}%
\pgfsetlinewidth{0.803000pt}%
\definecolor{currentstroke}{rgb}{0.000000,0.000000,0.000000}%
\pgfsetstrokecolor{currentstroke}%
\pgfsetdash{}{0pt}%
\pgfsys@defobject{currentmarker}{\pgfqpoint{-0.048611in}{0.000000in}}{\pgfqpoint{0.000000in}{0.000000in}}{%
\pgfpathmoveto{\pgfqpoint{0.000000in}{0.000000in}}%
\pgfpathlineto{\pgfqpoint{-0.048611in}{0.000000in}}%
\pgfusepath{stroke,fill}%
}%
\begin{pgfscope}%
\pgfsys@transformshift{1.068333in}{1.079087in}%
\pgfsys@useobject{currentmarker}{}%
\end{pgfscope}%
\end{pgfscope}%
\begin{pgfscope}%
\pgfsetbuttcap%
\pgfsetroundjoin%
\definecolor{currentfill}{rgb}{0.000000,0.000000,0.000000}%
\pgfsetfillcolor{currentfill}%
\pgfsetlinewidth{0.803000pt}%
\definecolor{currentstroke}{rgb}{0.000000,0.000000,0.000000}%
\pgfsetstrokecolor{currentstroke}%
\pgfsetdash{}{0pt}%
\pgfsys@defobject{currentmarker}{\pgfqpoint{-0.048611in}{0.000000in}}{\pgfqpoint{0.000000in}{0.000000in}}{%
\pgfpathmoveto{\pgfqpoint{0.000000in}{0.000000in}}%
\pgfpathlineto{\pgfqpoint{-0.048611in}{0.000000in}}%
\pgfusepath{stroke,fill}%
}%
\begin{pgfscope}%
\pgfsys@transformshift{1.068333in}{1.527805in}%
\pgfsys@useobject{currentmarker}{}%
\end{pgfscope}%
\end{pgfscope}%
\begin{pgfscope}%
\pgftext[x=0.750231in,y=1.475043in,left,base]{\sffamily\fontsize{10.000000}{12.000000}\selectfont 0.5}%
\end{pgfscope}%
\begin{pgfscope}%
\pgfsetbuttcap%
\pgfsetroundjoin%
\definecolor{currentfill}{rgb}{0.000000,0.000000,0.000000}%
\pgfsetfillcolor{currentfill}%
\pgfsetlinewidth{0.803000pt}%
\definecolor{currentstroke}{rgb}{0.000000,0.000000,0.000000}%
\pgfsetstrokecolor{currentstroke}%
\pgfsetdash{}{0pt}%
\pgfsys@defobject{currentmarker}{\pgfqpoint{-0.048611in}{0.000000in}}{\pgfqpoint{0.000000in}{0.000000in}}{%
\pgfpathmoveto{\pgfqpoint{0.000000in}{0.000000in}}%
\pgfpathlineto{\pgfqpoint{-0.048611in}{0.000000in}}%
\pgfusepath{stroke,fill}%
}%
\begin{pgfscope}%
\pgfsys@transformshift{1.068333in}{1.976522in}%
\pgfsys@useobject{currentmarker}{}%
\end{pgfscope}%
\end{pgfscope}%
\begin{pgfscope}%
\pgftext[x=0.882746in,y=1.923761in,left,base]{\sffamily\fontsize{10.000000}{12.000000}\selectfont 1}%
\end{pgfscope}%
\begin{pgfscope}%
\pgfsetbuttcap%
\pgfsetroundjoin%
\definecolor{currentfill}{rgb}{0.000000,0.000000,0.000000}%
\pgfsetfillcolor{currentfill}%
\pgfsetlinewidth{0.602250pt}%
\definecolor{currentstroke}{rgb}{0.000000,0.000000,0.000000}%
\pgfsetstrokecolor{currentstroke}%
\pgfsetdash{}{0pt}%
\pgfsys@defobject{currentmarker}{\pgfqpoint{-0.027778in}{0.000000in}}{\pgfqpoint{0.000000in}{0.000000in}}{%
\pgfpathmoveto{\pgfqpoint{0.000000in}{0.000000in}}%
\pgfpathlineto{\pgfqpoint{-0.027778in}{0.000000in}}%
\pgfusepath{stroke,fill}%
}%
\begin{pgfscope}%
\pgfsys@transformshift{1.068333in}{0.271395in}%
\pgfsys@useobject{currentmarker}{}%
\end{pgfscope}%
\end{pgfscope}%
\begin{pgfscope}%
\pgfsetbuttcap%
\pgfsetroundjoin%
\definecolor{currentfill}{rgb}{0.000000,0.000000,0.000000}%
\pgfsetfillcolor{currentfill}%
\pgfsetlinewidth{0.602250pt}%
\definecolor{currentstroke}{rgb}{0.000000,0.000000,0.000000}%
\pgfsetstrokecolor{currentstroke}%
\pgfsetdash{}{0pt}%
\pgfsys@defobject{currentmarker}{\pgfqpoint{-0.027778in}{0.000000in}}{\pgfqpoint{0.000000in}{0.000000in}}{%
\pgfpathmoveto{\pgfqpoint{0.000000in}{0.000000in}}%
\pgfpathlineto{\pgfqpoint{-0.027778in}{0.000000in}}%
\pgfusepath{stroke,fill}%
}%
\begin{pgfscope}%
\pgfsys@transformshift{1.068333in}{0.361138in}%
\pgfsys@useobject{currentmarker}{}%
\end{pgfscope}%
\end{pgfscope}%
\begin{pgfscope}%
\pgfsetbuttcap%
\pgfsetroundjoin%
\definecolor{currentfill}{rgb}{0.000000,0.000000,0.000000}%
\pgfsetfillcolor{currentfill}%
\pgfsetlinewidth{0.602250pt}%
\definecolor{currentstroke}{rgb}{0.000000,0.000000,0.000000}%
\pgfsetstrokecolor{currentstroke}%
\pgfsetdash{}{0pt}%
\pgfsys@defobject{currentmarker}{\pgfqpoint{-0.027778in}{0.000000in}}{\pgfqpoint{0.000000in}{0.000000in}}{%
\pgfpathmoveto{\pgfqpoint{0.000000in}{0.000000in}}%
\pgfpathlineto{\pgfqpoint{-0.027778in}{0.000000in}}%
\pgfusepath{stroke,fill}%
}%
\begin{pgfscope}%
\pgfsys@transformshift{1.068333in}{0.450882in}%
\pgfsys@useobject{currentmarker}{}%
\end{pgfscope}%
\end{pgfscope}%
\begin{pgfscope}%
\pgfsetbuttcap%
\pgfsetroundjoin%
\definecolor{currentfill}{rgb}{0.000000,0.000000,0.000000}%
\pgfsetfillcolor{currentfill}%
\pgfsetlinewidth{0.602250pt}%
\definecolor{currentstroke}{rgb}{0.000000,0.000000,0.000000}%
\pgfsetstrokecolor{currentstroke}%
\pgfsetdash{}{0pt}%
\pgfsys@defobject{currentmarker}{\pgfqpoint{-0.027778in}{0.000000in}}{\pgfqpoint{0.000000in}{0.000000in}}{%
\pgfpathmoveto{\pgfqpoint{0.000000in}{0.000000in}}%
\pgfpathlineto{\pgfqpoint{-0.027778in}{0.000000in}}%
\pgfusepath{stroke,fill}%
}%
\begin{pgfscope}%
\pgfsys@transformshift{1.068333in}{0.540625in}%
\pgfsys@useobject{currentmarker}{}%
\end{pgfscope}%
\end{pgfscope}%
\begin{pgfscope}%
\pgfsetbuttcap%
\pgfsetroundjoin%
\definecolor{currentfill}{rgb}{0.000000,0.000000,0.000000}%
\pgfsetfillcolor{currentfill}%
\pgfsetlinewidth{0.602250pt}%
\definecolor{currentstroke}{rgb}{0.000000,0.000000,0.000000}%
\pgfsetstrokecolor{currentstroke}%
\pgfsetdash{}{0pt}%
\pgfsys@defobject{currentmarker}{\pgfqpoint{-0.027778in}{0.000000in}}{\pgfqpoint{0.000000in}{0.000000in}}{%
\pgfpathmoveto{\pgfqpoint{0.000000in}{0.000000in}}%
\pgfpathlineto{\pgfqpoint{-0.027778in}{0.000000in}}%
\pgfusepath{stroke,fill}%
}%
\begin{pgfscope}%
\pgfsys@transformshift{1.068333in}{0.630369in}%
\pgfsys@useobject{currentmarker}{}%
\end{pgfscope}%
\end{pgfscope}%
\begin{pgfscope}%
\pgfsetbuttcap%
\pgfsetroundjoin%
\definecolor{currentfill}{rgb}{0.000000,0.000000,0.000000}%
\pgfsetfillcolor{currentfill}%
\pgfsetlinewidth{0.602250pt}%
\definecolor{currentstroke}{rgb}{0.000000,0.000000,0.000000}%
\pgfsetstrokecolor{currentstroke}%
\pgfsetdash{}{0pt}%
\pgfsys@defobject{currentmarker}{\pgfqpoint{-0.027778in}{0.000000in}}{\pgfqpoint{0.000000in}{0.000000in}}{%
\pgfpathmoveto{\pgfqpoint{0.000000in}{0.000000in}}%
\pgfpathlineto{\pgfqpoint{-0.027778in}{0.000000in}}%
\pgfusepath{stroke,fill}%
}%
\begin{pgfscope}%
\pgfsys@transformshift{1.068333in}{0.720113in}%
\pgfsys@useobject{currentmarker}{}%
\end{pgfscope}%
\end{pgfscope}%
\begin{pgfscope}%
\pgfsetbuttcap%
\pgfsetroundjoin%
\definecolor{currentfill}{rgb}{0.000000,0.000000,0.000000}%
\pgfsetfillcolor{currentfill}%
\pgfsetlinewidth{0.602250pt}%
\definecolor{currentstroke}{rgb}{0.000000,0.000000,0.000000}%
\pgfsetstrokecolor{currentstroke}%
\pgfsetdash{}{0pt}%
\pgfsys@defobject{currentmarker}{\pgfqpoint{-0.027778in}{0.000000in}}{\pgfqpoint{0.000000in}{0.000000in}}{%
\pgfpathmoveto{\pgfqpoint{0.000000in}{0.000000in}}%
\pgfpathlineto{\pgfqpoint{-0.027778in}{0.000000in}}%
\pgfusepath{stroke,fill}%
}%
\begin{pgfscope}%
\pgfsys@transformshift{1.068333in}{0.809856in}%
\pgfsys@useobject{currentmarker}{}%
\end{pgfscope}%
\end{pgfscope}%
\begin{pgfscope}%
\pgfsetbuttcap%
\pgfsetroundjoin%
\definecolor{currentfill}{rgb}{0.000000,0.000000,0.000000}%
\pgfsetfillcolor{currentfill}%
\pgfsetlinewidth{0.602250pt}%
\definecolor{currentstroke}{rgb}{0.000000,0.000000,0.000000}%
\pgfsetstrokecolor{currentstroke}%
\pgfsetdash{}{0pt}%
\pgfsys@defobject{currentmarker}{\pgfqpoint{-0.027778in}{0.000000in}}{\pgfqpoint{0.000000in}{0.000000in}}{%
\pgfpathmoveto{\pgfqpoint{0.000000in}{0.000000in}}%
\pgfpathlineto{\pgfqpoint{-0.027778in}{0.000000in}}%
\pgfusepath{stroke,fill}%
}%
\begin{pgfscope}%
\pgfsys@transformshift{1.068333in}{0.899600in}%
\pgfsys@useobject{currentmarker}{}%
\end{pgfscope}%
\end{pgfscope}%
\begin{pgfscope}%
\pgfsetbuttcap%
\pgfsetroundjoin%
\definecolor{currentfill}{rgb}{0.000000,0.000000,0.000000}%
\pgfsetfillcolor{currentfill}%
\pgfsetlinewidth{0.602250pt}%
\definecolor{currentstroke}{rgb}{0.000000,0.000000,0.000000}%
\pgfsetstrokecolor{currentstroke}%
\pgfsetdash{}{0pt}%
\pgfsys@defobject{currentmarker}{\pgfqpoint{-0.027778in}{0.000000in}}{\pgfqpoint{0.000000in}{0.000000in}}{%
\pgfpathmoveto{\pgfqpoint{0.000000in}{0.000000in}}%
\pgfpathlineto{\pgfqpoint{-0.027778in}{0.000000in}}%
\pgfusepath{stroke,fill}%
}%
\begin{pgfscope}%
\pgfsys@transformshift{1.068333in}{0.989343in}%
\pgfsys@useobject{currentmarker}{}%
\end{pgfscope}%
\end{pgfscope}%
\begin{pgfscope}%
\pgfsetbuttcap%
\pgfsetroundjoin%
\definecolor{currentfill}{rgb}{0.000000,0.000000,0.000000}%
\pgfsetfillcolor{currentfill}%
\pgfsetlinewidth{0.602250pt}%
\definecolor{currentstroke}{rgb}{0.000000,0.000000,0.000000}%
\pgfsetstrokecolor{currentstroke}%
\pgfsetdash{}{0pt}%
\pgfsys@defobject{currentmarker}{\pgfqpoint{-0.027778in}{0.000000in}}{\pgfqpoint{0.000000in}{0.000000in}}{%
\pgfpathmoveto{\pgfqpoint{0.000000in}{0.000000in}}%
\pgfpathlineto{\pgfqpoint{-0.027778in}{0.000000in}}%
\pgfusepath{stroke,fill}%
}%
\begin{pgfscope}%
\pgfsys@transformshift{1.068333in}{1.079087in}%
\pgfsys@useobject{currentmarker}{}%
\end{pgfscope}%
\end{pgfscope}%
\begin{pgfscope}%
\pgfsetbuttcap%
\pgfsetroundjoin%
\definecolor{currentfill}{rgb}{0.000000,0.000000,0.000000}%
\pgfsetfillcolor{currentfill}%
\pgfsetlinewidth{0.602250pt}%
\definecolor{currentstroke}{rgb}{0.000000,0.000000,0.000000}%
\pgfsetstrokecolor{currentstroke}%
\pgfsetdash{}{0pt}%
\pgfsys@defobject{currentmarker}{\pgfqpoint{-0.027778in}{0.000000in}}{\pgfqpoint{0.000000in}{0.000000in}}{%
\pgfpathmoveto{\pgfqpoint{0.000000in}{0.000000in}}%
\pgfpathlineto{\pgfqpoint{-0.027778in}{0.000000in}}%
\pgfusepath{stroke,fill}%
}%
\begin{pgfscope}%
\pgfsys@transformshift{1.068333in}{1.168830in}%
\pgfsys@useobject{currentmarker}{}%
\end{pgfscope}%
\end{pgfscope}%
\begin{pgfscope}%
\pgfsetbuttcap%
\pgfsetroundjoin%
\definecolor{currentfill}{rgb}{0.000000,0.000000,0.000000}%
\pgfsetfillcolor{currentfill}%
\pgfsetlinewidth{0.602250pt}%
\definecolor{currentstroke}{rgb}{0.000000,0.000000,0.000000}%
\pgfsetstrokecolor{currentstroke}%
\pgfsetdash{}{0pt}%
\pgfsys@defobject{currentmarker}{\pgfqpoint{-0.027778in}{0.000000in}}{\pgfqpoint{0.000000in}{0.000000in}}{%
\pgfpathmoveto{\pgfqpoint{0.000000in}{0.000000in}}%
\pgfpathlineto{\pgfqpoint{-0.027778in}{0.000000in}}%
\pgfusepath{stroke,fill}%
}%
\begin{pgfscope}%
\pgfsys@transformshift{1.068333in}{1.258574in}%
\pgfsys@useobject{currentmarker}{}%
\end{pgfscope}%
\end{pgfscope}%
\begin{pgfscope}%
\pgfsetbuttcap%
\pgfsetroundjoin%
\definecolor{currentfill}{rgb}{0.000000,0.000000,0.000000}%
\pgfsetfillcolor{currentfill}%
\pgfsetlinewidth{0.602250pt}%
\definecolor{currentstroke}{rgb}{0.000000,0.000000,0.000000}%
\pgfsetstrokecolor{currentstroke}%
\pgfsetdash{}{0pt}%
\pgfsys@defobject{currentmarker}{\pgfqpoint{-0.027778in}{0.000000in}}{\pgfqpoint{0.000000in}{0.000000in}}{%
\pgfpathmoveto{\pgfqpoint{0.000000in}{0.000000in}}%
\pgfpathlineto{\pgfqpoint{-0.027778in}{0.000000in}}%
\pgfusepath{stroke,fill}%
}%
\begin{pgfscope}%
\pgfsys@transformshift{1.068333in}{1.348318in}%
\pgfsys@useobject{currentmarker}{}%
\end{pgfscope}%
\end{pgfscope}%
\begin{pgfscope}%
\pgfsetbuttcap%
\pgfsetroundjoin%
\definecolor{currentfill}{rgb}{0.000000,0.000000,0.000000}%
\pgfsetfillcolor{currentfill}%
\pgfsetlinewidth{0.602250pt}%
\definecolor{currentstroke}{rgb}{0.000000,0.000000,0.000000}%
\pgfsetstrokecolor{currentstroke}%
\pgfsetdash{}{0pt}%
\pgfsys@defobject{currentmarker}{\pgfqpoint{-0.027778in}{0.000000in}}{\pgfqpoint{0.000000in}{0.000000in}}{%
\pgfpathmoveto{\pgfqpoint{0.000000in}{0.000000in}}%
\pgfpathlineto{\pgfqpoint{-0.027778in}{0.000000in}}%
\pgfusepath{stroke,fill}%
}%
\begin{pgfscope}%
\pgfsys@transformshift{1.068333in}{1.438061in}%
\pgfsys@useobject{currentmarker}{}%
\end{pgfscope}%
\end{pgfscope}%
\begin{pgfscope}%
\pgfsetbuttcap%
\pgfsetroundjoin%
\definecolor{currentfill}{rgb}{0.000000,0.000000,0.000000}%
\pgfsetfillcolor{currentfill}%
\pgfsetlinewidth{0.602250pt}%
\definecolor{currentstroke}{rgb}{0.000000,0.000000,0.000000}%
\pgfsetstrokecolor{currentstroke}%
\pgfsetdash{}{0pt}%
\pgfsys@defobject{currentmarker}{\pgfqpoint{-0.027778in}{0.000000in}}{\pgfqpoint{0.000000in}{0.000000in}}{%
\pgfpathmoveto{\pgfqpoint{0.000000in}{0.000000in}}%
\pgfpathlineto{\pgfqpoint{-0.027778in}{0.000000in}}%
\pgfusepath{stroke,fill}%
}%
\begin{pgfscope}%
\pgfsys@transformshift{1.068333in}{1.527805in}%
\pgfsys@useobject{currentmarker}{}%
\end{pgfscope}%
\end{pgfscope}%
\begin{pgfscope}%
\pgfsetbuttcap%
\pgfsetroundjoin%
\definecolor{currentfill}{rgb}{0.000000,0.000000,0.000000}%
\pgfsetfillcolor{currentfill}%
\pgfsetlinewidth{0.602250pt}%
\definecolor{currentstroke}{rgb}{0.000000,0.000000,0.000000}%
\pgfsetstrokecolor{currentstroke}%
\pgfsetdash{}{0pt}%
\pgfsys@defobject{currentmarker}{\pgfqpoint{-0.027778in}{0.000000in}}{\pgfqpoint{0.000000in}{0.000000in}}{%
\pgfpathmoveto{\pgfqpoint{0.000000in}{0.000000in}}%
\pgfpathlineto{\pgfqpoint{-0.027778in}{0.000000in}}%
\pgfusepath{stroke,fill}%
}%
\begin{pgfscope}%
\pgfsys@transformshift{1.068333in}{1.617548in}%
\pgfsys@useobject{currentmarker}{}%
\end{pgfscope}%
\end{pgfscope}%
\begin{pgfscope}%
\pgfsetbuttcap%
\pgfsetroundjoin%
\definecolor{currentfill}{rgb}{0.000000,0.000000,0.000000}%
\pgfsetfillcolor{currentfill}%
\pgfsetlinewidth{0.602250pt}%
\definecolor{currentstroke}{rgb}{0.000000,0.000000,0.000000}%
\pgfsetstrokecolor{currentstroke}%
\pgfsetdash{}{0pt}%
\pgfsys@defobject{currentmarker}{\pgfqpoint{-0.027778in}{0.000000in}}{\pgfqpoint{0.000000in}{0.000000in}}{%
\pgfpathmoveto{\pgfqpoint{0.000000in}{0.000000in}}%
\pgfpathlineto{\pgfqpoint{-0.027778in}{0.000000in}}%
\pgfusepath{stroke,fill}%
}%
\begin{pgfscope}%
\pgfsys@transformshift{1.068333in}{1.707292in}%
\pgfsys@useobject{currentmarker}{}%
\end{pgfscope}%
\end{pgfscope}%
\begin{pgfscope}%
\pgfsetbuttcap%
\pgfsetroundjoin%
\definecolor{currentfill}{rgb}{0.000000,0.000000,0.000000}%
\pgfsetfillcolor{currentfill}%
\pgfsetlinewidth{0.602250pt}%
\definecolor{currentstroke}{rgb}{0.000000,0.000000,0.000000}%
\pgfsetstrokecolor{currentstroke}%
\pgfsetdash{}{0pt}%
\pgfsys@defobject{currentmarker}{\pgfqpoint{-0.027778in}{0.000000in}}{\pgfqpoint{0.000000in}{0.000000in}}{%
\pgfpathmoveto{\pgfqpoint{0.000000in}{0.000000in}}%
\pgfpathlineto{\pgfqpoint{-0.027778in}{0.000000in}}%
\pgfusepath{stroke,fill}%
}%
\begin{pgfscope}%
\pgfsys@transformshift{1.068333in}{1.797035in}%
\pgfsys@useobject{currentmarker}{}%
\end{pgfscope}%
\end{pgfscope}%
\begin{pgfscope}%
\pgfsetbuttcap%
\pgfsetroundjoin%
\definecolor{currentfill}{rgb}{0.000000,0.000000,0.000000}%
\pgfsetfillcolor{currentfill}%
\pgfsetlinewidth{0.602250pt}%
\definecolor{currentstroke}{rgb}{0.000000,0.000000,0.000000}%
\pgfsetstrokecolor{currentstroke}%
\pgfsetdash{}{0pt}%
\pgfsys@defobject{currentmarker}{\pgfqpoint{-0.027778in}{0.000000in}}{\pgfqpoint{0.000000in}{0.000000in}}{%
\pgfpathmoveto{\pgfqpoint{0.000000in}{0.000000in}}%
\pgfpathlineto{\pgfqpoint{-0.027778in}{0.000000in}}%
\pgfusepath{stroke,fill}%
}%
\begin{pgfscope}%
\pgfsys@transformshift{1.068333in}{1.886779in}%
\pgfsys@useobject{currentmarker}{}%
\end{pgfscope}%
\end{pgfscope}%
\begin{pgfscope}%
\pgfsetbuttcap%
\pgfsetroundjoin%
\definecolor{currentfill}{rgb}{0.000000,0.000000,0.000000}%
\pgfsetfillcolor{currentfill}%
\pgfsetlinewidth{0.602250pt}%
\definecolor{currentstroke}{rgb}{0.000000,0.000000,0.000000}%
\pgfsetstrokecolor{currentstroke}%
\pgfsetdash{}{0pt}%
\pgfsys@defobject{currentmarker}{\pgfqpoint{-0.027778in}{0.000000in}}{\pgfqpoint{0.000000in}{0.000000in}}{%
\pgfpathmoveto{\pgfqpoint{0.000000in}{0.000000in}}%
\pgfpathlineto{\pgfqpoint{-0.027778in}{0.000000in}}%
\pgfusepath{stroke,fill}%
}%
\begin{pgfscope}%
\pgfsys@transformshift{1.068333in}{1.976522in}%
\pgfsys@useobject{currentmarker}{}%
\end{pgfscope}%
\end{pgfscope}%
\begin{pgfscope}%
\pgfpathrectangle{\pgfqpoint{0.135000in}{0.145754in}}{\pgfqpoint{1.866666in}{1.866666in}} %
\pgfusepath{clip}%
\pgfsetbuttcap%
\pgfsetroundjoin%
\pgfsetlinewidth{1.505625pt}%
\definecolor{currentstroke}{rgb}{0.000000,0.000000,1.000000}%
\pgfsetstrokecolor{currentstroke}%
\pgfsetdash{}{0pt}%
\pgfpathmoveto{\pgfqpoint{0.929803in}{0.192408in}}%
\pgfpathlineto{\pgfqpoint{0.881618in}{0.201290in}}%
\pgfpathlineto{\pgfqpoint{0.833434in}{0.212940in}}%
\pgfpathlineto{\pgfqpoint{0.785249in}{0.227470in}}%
\pgfpathlineto{\pgfqpoint{0.745145in}{0.241882in}}%
\pgfpathlineto{\pgfqpoint{0.700927in}{0.260309in}}%
\pgfpathlineto{\pgfqpoint{0.663767in}{0.278020in}}%
\pgfpathlineto{\pgfqpoint{0.619242in}{0.302112in}}%
\pgfpathlineto{\pgfqpoint{0.592511in}{0.318183in}}%
\pgfpathlineto{\pgfqpoint{0.556373in}{0.342013in}}%
\pgfpathlineto{\pgfqpoint{0.520235in}{0.368478in}}%
\pgfpathlineto{\pgfqpoint{0.483387in}{0.398481in}}%
\pgfpathlineto{\pgfqpoint{0.447958in}{0.430619in}}%
\pgfpathlineto{\pgfqpoint{0.419865in}{0.458712in}}%
\pgfpathlineto{\pgfqpoint{0.387117in}{0.494850in}}%
\pgfpathlineto{\pgfqpoint{0.357724in}{0.530989in}}%
\pgfpathlineto{\pgfqpoint{0.331259in}{0.567127in}}%
\pgfpathlineto{\pgfqpoint{0.307429in}{0.603265in}}%
\pgfpathlineto{\pgfqpoint{0.285990in}{0.639404in}}%
\pgfpathlineto{\pgfqpoint{0.266746in}{0.675542in}}%
\pgfpathlineto{\pgfqpoint{0.249555in}{0.711680in}}%
\pgfpathlineto{\pgfqpoint{0.231128in}{0.755899in}}%
\pgfpathlineto{\pgfqpoint{0.216716in}{0.796003in}}%
\pgfpathlineto{\pgfqpoint{0.202186in}{0.844188in}}%
\pgfpathlineto{\pgfqpoint{0.190536in}{0.892372in}}%
\pgfpathlineto{\pgfqpoint{0.181654in}{0.940557in}}%
\pgfpathlineto{\pgfqpoint{0.175457in}{0.988741in}}%
\pgfpathlineto{\pgfqpoint{0.171888in}{1.036925in}}%
\pgfpathlineto{\pgfqpoint{0.170918in}{1.085110in}}%
\pgfpathlineto{\pgfqpoint{0.172536in}{1.133294in}}%
\pgfpathlineto{\pgfqpoint{0.176758in}{1.181479in}}%
\pgfpathlineto{\pgfqpoint{0.183620in}{1.229663in}}%
\pgfpathlineto{\pgfqpoint{0.193185in}{1.277848in}}%
\pgfpathlineto{\pgfqpoint{0.205543in}{1.326032in}}%
\pgfpathlineto{\pgfqpoint{0.220815in}{1.374217in}}%
\pgfpathlineto{\pgfqpoint{0.239164in}{1.422401in}}%
\pgfpathlineto{\pgfqpoint{0.255220in}{1.458870in}}%
\pgfpathlineto{\pgfqpoint{0.272929in}{1.494678in}}%
\pgfpathlineto{\pgfqpoint{0.292880in}{1.530816in}}%
\pgfpathlineto{\pgfqpoint{0.315451in}{1.567507in}}%
\pgfpathlineto{\pgfqpoint{0.339768in}{1.603093in}}%
\pgfpathlineto{\pgfqpoint{0.367179in}{1.639231in}}%
\pgfpathlineto{\pgfqpoint{0.399774in}{1.677757in}}%
\pgfpathlineto{\pgfqpoint{0.431609in}{1.711508in}}%
\pgfpathlineto{\pgfqpoint{0.460004in}{1.738870in}}%
\pgfpathlineto{\pgfqpoint{0.497707in}{1.771738in}}%
\pgfpathlineto{\pgfqpoint{0.532281in}{1.798829in}}%
\pgfpathlineto{\pgfqpoint{0.568419in}{1.824383in}}%
\pgfpathlineto{\pgfqpoint{0.604558in}{1.847394in}}%
\pgfpathlineto{\pgfqpoint{0.640739in}{1.868107in}}%
\pgfpathlineto{\pgfqpoint{0.676834in}{1.886622in}}%
\pgfpathlineto{\pgfqpoint{0.715512in}{1.904246in}}%
\pgfpathlineto{\pgfqpoint{0.761157in}{1.922312in}}%
\pgfpathlineto{\pgfqpoint{0.809342in}{1.938338in}}%
\pgfpathlineto{\pgfqpoint{0.857526in}{1.951412in}}%
\pgfpathlineto{\pgfqpoint{0.905710in}{1.961665in}}%
\pgfpathlineto{\pgfqpoint{0.953895in}{1.969196in}}%
\pgfpathlineto{\pgfqpoint{1.002079in}{1.974073in}}%
\pgfpathlineto{\pgfqpoint{1.050264in}{1.976341in}}%
\pgfpathlineto{\pgfqpoint{1.098448in}{1.976017in}}%
\pgfpathlineto{\pgfqpoint{1.146633in}{1.973100in}}%
\pgfpathlineto{\pgfqpoint{1.194817in}{1.967564in}}%
\pgfpathlineto{\pgfqpoint{1.243002in}{1.959360in}}%
\pgfpathlineto{\pgfqpoint{1.291186in}{1.948411in}}%
\pgfpathlineto{\pgfqpoint{1.339371in}{1.934614in}}%
\pgfpathlineto{\pgfqpoint{1.387555in}{1.917828in}}%
\pgfpathlineto{\pgfqpoint{1.423693in}{1.903167in}}%
\pgfpathlineto{\pgfqpoint{1.459832in}{1.886622in}}%
\pgfpathlineto{\pgfqpoint{1.495970in}{1.868084in}}%
\pgfpathlineto{\pgfqpoint{1.537640in}{1.844015in}}%
\pgfpathlineto{\pgfqpoint{1.574819in}{1.819923in}}%
\pgfpathlineto{\pgfqpoint{1.608378in}{1.795831in}}%
\pgfpathlineto{\pgfqpoint{1.640524in}{1.770450in}}%
\pgfpathlineto{\pgfqpoint{1.680187in}{1.735600in}}%
\pgfpathlineto{\pgfqpoint{1.712800in}{1.703615in}}%
\pgfpathlineto{\pgfqpoint{1.739025in}{1.675370in}}%
\pgfpathlineto{\pgfqpoint{1.769487in}{1.639231in}}%
\pgfpathlineto{\pgfqpoint{1.797123in}{1.602779in}}%
\pgfpathlineto{\pgfqpoint{1.821576in}{1.566955in}}%
\pgfpathlineto{\pgfqpoint{1.845307in}{1.528178in}}%
\pgfpathlineto{\pgfqpoint{1.863737in}{1.494678in}}%
\pgfpathlineto{\pgfqpoint{1.881602in}{1.458539in}}%
\pgfpathlineto{\pgfqpoint{1.897502in}{1.422401in}}%
\pgfpathlineto{\pgfqpoint{1.915851in}{1.374217in}}%
\pgfpathlineto{\pgfqpoint{1.929630in}{1.331129in}}%
\pgfpathlineto{\pgfqpoint{1.940658in}{1.289894in}}%
\pgfpathlineto{\pgfqpoint{1.950911in}{1.241709in}}%
\pgfpathlineto{\pgfqpoint{1.958442in}{1.193525in}}%
\pgfpathlineto{\pgfqpoint{1.963320in}{1.145340in}}%
\pgfpathlineto{\pgfqpoint{1.965587in}{1.097156in}}%
\pgfpathlineto{\pgfqpoint{1.965263in}{1.048972in}}%
\pgfpathlineto{\pgfqpoint{1.962346in}{1.000787in}}%
\pgfpathlineto{\pgfqpoint{1.956810in}{0.952603in}}%
\pgfpathlineto{\pgfqpoint{1.948606in}{0.904418in}}%
\pgfpathlineto{\pgfqpoint{1.937658in}{0.856234in}}%
\pgfpathlineto{\pgfqpoint{1.923860in}{0.808049in}}%
\pgfpathlineto{\pgfqpoint{1.907074in}{0.759865in}}%
\pgfpathlineto{\pgfqpoint{1.892413in}{0.723726in}}%
\pgfpathlineto{\pgfqpoint{1.875868in}{0.687588in}}%
\pgfpathlineto{\pgfqpoint{1.857331in}{0.651450in}}%
\pgfpathlineto{\pgfqpoint{1.833261in}{0.609780in}}%
\pgfpathlineto{\pgfqpoint{1.809169in}{0.572601in}}%
\pgfpathlineto{\pgfqpoint{1.785077in}{0.539042in}}%
\pgfpathlineto{\pgfqpoint{1.759696in}{0.506896in}}%
\pgfpathlineto{\pgfqpoint{1.724846in}{0.467233in}}%
\pgfpathlineto{\pgfqpoint{1.692862in}{0.434620in}}%
\pgfpathlineto{\pgfqpoint{1.664616in}{0.408394in}}%
\pgfpathlineto{\pgfqpoint{1.628477in}{0.377933in}}%
\pgfpathlineto{\pgfqpoint{1.592025in}{0.350297in}}%
\pgfpathlineto{\pgfqpoint{1.556201in}{0.325844in}}%
\pgfpathlineto{\pgfqpoint{1.517424in}{0.302112in}}%
\pgfpathlineto{\pgfqpoint{1.483924in}{0.283683in}}%
\pgfpathlineto{\pgfqpoint{1.447786in}{0.265818in}}%
\pgfpathlineto{\pgfqpoint{1.411647in}{0.249918in}}%
\pgfpathlineto{\pgfqpoint{1.363463in}{0.231569in}}%
\pgfpathlineto{\pgfqpoint{1.320375in}{0.217790in}}%
\pgfpathlineto{\pgfqpoint{1.279140in}{0.206762in}}%
\pgfpathlineto{\pgfqpoint{1.230956in}{0.196509in}}%
\pgfpathlineto{\pgfqpoint{1.182771in}{0.188978in}}%
\pgfpathlineto{\pgfqpoint{1.134587in}{0.184100in}}%
\pgfpathlineto{\pgfqpoint{1.086402in}{0.181833in}}%
\pgfpathlineto{\pgfqpoint{1.038218in}{0.182157in}}%
\pgfpathlineto{\pgfqpoint{0.990033in}{0.185074in}}%
\pgfpathlineto{\pgfqpoint{0.941849in}{0.190610in}}%
\pgfpathlineto{\pgfqpoint{0.929803in}{0.192408in}}%
\pgfpathlineto{\pgfqpoint{0.929803in}{0.192408in}}%
\pgfusepath{stroke}%
\end{pgfscope}%
\begin{pgfscope}%
\pgfpathrectangle{\pgfqpoint{0.135000in}{0.145754in}}{\pgfqpoint{1.866666in}{1.866666in}} %
\pgfusepath{clip}%
\pgfsetbuttcap%
\pgfsetroundjoin%
\pgfsetlinewidth{1.505625pt}%
\definecolor{currentstroke}{rgb}{1.000000,1.000000,1.000000}%
\pgfsetstrokecolor{currentstroke}%
\pgfsetdash{}{0pt}%
\pgfpathmoveto{\pgfqpoint{0.929803in}{0.192408in}}%
\pgfpathlineto{\pgfqpoint{0.881618in}{0.201290in}}%
\pgfpathlineto{\pgfqpoint{0.833434in}{0.212940in}}%
\pgfpathlineto{\pgfqpoint{0.785249in}{0.227470in}}%
\pgfpathlineto{\pgfqpoint{0.745145in}{0.241882in}}%
\pgfpathlineto{\pgfqpoint{0.700927in}{0.260309in}}%
\pgfpathlineto{\pgfqpoint{0.663767in}{0.278020in}}%
\pgfpathlineto{\pgfqpoint{0.619242in}{0.302112in}}%
\pgfpathlineto{\pgfqpoint{0.592511in}{0.318183in}}%
\pgfpathlineto{\pgfqpoint{0.556373in}{0.342013in}}%
\pgfpathlineto{\pgfqpoint{0.520235in}{0.368478in}}%
\pgfpathlineto{\pgfqpoint{0.483387in}{0.398481in}}%
\pgfpathlineto{\pgfqpoint{0.447958in}{0.430619in}}%
\pgfpathlineto{\pgfqpoint{0.419865in}{0.458712in}}%
\pgfpathlineto{\pgfqpoint{0.387117in}{0.494850in}}%
\pgfpathlineto{\pgfqpoint{0.357724in}{0.530989in}}%
\pgfpathlineto{\pgfqpoint{0.331259in}{0.567127in}}%
\pgfpathlineto{\pgfqpoint{0.307429in}{0.603265in}}%
\pgfpathlineto{\pgfqpoint{0.285990in}{0.639404in}}%
\pgfpathlineto{\pgfqpoint{0.266746in}{0.675542in}}%
\pgfpathlineto{\pgfqpoint{0.249555in}{0.711680in}}%
\pgfpathlineto{\pgfqpoint{0.231128in}{0.755899in}}%
\pgfpathlineto{\pgfqpoint{0.216716in}{0.796003in}}%
\pgfpathlineto{\pgfqpoint{0.202186in}{0.844188in}}%
\pgfpathlineto{\pgfqpoint{0.190536in}{0.892372in}}%
\pgfpathlineto{\pgfqpoint{0.181654in}{0.940557in}}%
\pgfpathlineto{\pgfqpoint{0.175457in}{0.988741in}}%
\pgfpathlineto{\pgfqpoint{0.171888in}{1.036925in}}%
\pgfpathlineto{\pgfqpoint{0.170918in}{1.085110in}}%
\pgfpathlineto{\pgfqpoint{0.172536in}{1.133294in}}%
\pgfpathlineto{\pgfqpoint{0.176758in}{1.181479in}}%
\pgfpathlineto{\pgfqpoint{0.183620in}{1.229663in}}%
\pgfpathlineto{\pgfqpoint{0.193185in}{1.277848in}}%
\pgfpathlineto{\pgfqpoint{0.205543in}{1.326032in}}%
\pgfpathlineto{\pgfqpoint{0.220815in}{1.374217in}}%
\pgfpathlineto{\pgfqpoint{0.239164in}{1.422401in}}%
\pgfpathlineto{\pgfqpoint{0.255220in}{1.458870in}}%
\pgfpathlineto{\pgfqpoint{0.272929in}{1.494678in}}%
\pgfpathlineto{\pgfqpoint{0.292880in}{1.530816in}}%
\pgfpathlineto{\pgfqpoint{0.315451in}{1.567507in}}%
\pgfpathlineto{\pgfqpoint{0.339768in}{1.603093in}}%
\pgfpathlineto{\pgfqpoint{0.367179in}{1.639231in}}%
\pgfpathlineto{\pgfqpoint{0.399774in}{1.677757in}}%
\pgfpathlineto{\pgfqpoint{0.431609in}{1.711508in}}%
\pgfpathlineto{\pgfqpoint{0.460004in}{1.738870in}}%
\pgfpathlineto{\pgfqpoint{0.497707in}{1.771738in}}%
\pgfpathlineto{\pgfqpoint{0.532281in}{1.798829in}}%
\pgfpathlineto{\pgfqpoint{0.568419in}{1.824383in}}%
\pgfpathlineto{\pgfqpoint{0.604558in}{1.847394in}}%
\pgfpathlineto{\pgfqpoint{0.640739in}{1.868107in}}%
\pgfpathlineto{\pgfqpoint{0.676834in}{1.886622in}}%
\pgfpathlineto{\pgfqpoint{0.715512in}{1.904246in}}%
\pgfpathlineto{\pgfqpoint{0.761157in}{1.922312in}}%
\pgfpathlineto{\pgfqpoint{0.809342in}{1.938338in}}%
\pgfpathlineto{\pgfqpoint{0.857526in}{1.951412in}}%
\pgfpathlineto{\pgfqpoint{0.905710in}{1.961665in}}%
\pgfpathlineto{\pgfqpoint{0.953895in}{1.969196in}}%
\pgfpathlineto{\pgfqpoint{1.002079in}{1.974073in}}%
\pgfpathlineto{\pgfqpoint{1.050264in}{1.976341in}}%
\pgfpathlineto{\pgfqpoint{1.098448in}{1.976017in}}%
\pgfpathlineto{\pgfqpoint{1.146633in}{1.973100in}}%
\pgfpathlineto{\pgfqpoint{1.194817in}{1.967564in}}%
\pgfpathlineto{\pgfqpoint{1.243002in}{1.959360in}}%
\pgfpathlineto{\pgfqpoint{1.291186in}{1.948411in}}%
\pgfpathlineto{\pgfqpoint{1.339371in}{1.934614in}}%
\pgfpathlineto{\pgfqpoint{1.387555in}{1.917828in}}%
\pgfpathlineto{\pgfqpoint{1.423693in}{1.903167in}}%
\pgfpathlineto{\pgfqpoint{1.459832in}{1.886622in}}%
\pgfpathlineto{\pgfqpoint{1.495970in}{1.868084in}}%
\pgfpathlineto{\pgfqpoint{1.537640in}{1.844015in}}%
\pgfpathlineto{\pgfqpoint{1.574819in}{1.819923in}}%
\pgfpathlineto{\pgfqpoint{1.608378in}{1.795831in}}%
\pgfpathlineto{\pgfqpoint{1.640524in}{1.770450in}}%
\pgfpathlineto{\pgfqpoint{1.680187in}{1.735600in}}%
\pgfpathlineto{\pgfqpoint{1.712800in}{1.703615in}}%
\pgfpathlineto{\pgfqpoint{1.739025in}{1.675370in}}%
\pgfpathlineto{\pgfqpoint{1.769487in}{1.639231in}}%
\pgfpathlineto{\pgfqpoint{1.797123in}{1.602779in}}%
\pgfpathlineto{\pgfqpoint{1.821576in}{1.566955in}}%
\pgfpathlineto{\pgfqpoint{1.845307in}{1.528178in}}%
\pgfpathlineto{\pgfqpoint{1.863737in}{1.494678in}}%
\pgfpathlineto{\pgfqpoint{1.881602in}{1.458539in}}%
\pgfpathlineto{\pgfqpoint{1.897502in}{1.422401in}}%
\pgfpathlineto{\pgfqpoint{1.915851in}{1.374217in}}%
\pgfpathlineto{\pgfqpoint{1.929630in}{1.331129in}}%
\pgfpathlineto{\pgfqpoint{1.940658in}{1.289894in}}%
\pgfpathlineto{\pgfqpoint{1.950911in}{1.241709in}}%
\pgfpathlineto{\pgfqpoint{1.958442in}{1.193525in}}%
\pgfpathlineto{\pgfqpoint{1.963320in}{1.145340in}}%
\pgfpathlineto{\pgfqpoint{1.965587in}{1.097156in}}%
\pgfpathlineto{\pgfqpoint{1.965263in}{1.048972in}}%
\pgfpathlineto{\pgfqpoint{1.962346in}{1.000787in}}%
\pgfpathlineto{\pgfqpoint{1.956810in}{0.952603in}}%
\pgfpathlineto{\pgfqpoint{1.948606in}{0.904418in}}%
\pgfpathlineto{\pgfqpoint{1.937658in}{0.856234in}}%
\pgfpathlineto{\pgfqpoint{1.923860in}{0.808049in}}%
\pgfpathlineto{\pgfqpoint{1.907074in}{0.759865in}}%
\pgfpathlineto{\pgfqpoint{1.892413in}{0.723726in}}%
\pgfpathlineto{\pgfqpoint{1.875868in}{0.687588in}}%
\pgfpathlineto{\pgfqpoint{1.857331in}{0.651450in}}%
\pgfpathlineto{\pgfqpoint{1.833261in}{0.609780in}}%
\pgfpathlineto{\pgfqpoint{1.809169in}{0.572601in}}%
\pgfpathlineto{\pgfqpoint{1.785077in}{0.539042in}}%
\pgfpathlineto{\pgfqpoint{1.759696in}{0.506896in}}%
\pgfpathlineto{\pgfqpoint{1.724846in}{0.467233in}}%
\pgfpathlineto{\pgfqpoint{1.692862in}{0.434620in}}%
\pgfpathlineto{\pgfqpoint{1.664616in}{0.408394in}}%
\pgfpathlineto{\pgfqpoint{1.628477in}{0.377933in}}%
\pgfpathlineto{\pgfqpoint{1.592025in}{0.350297in}}%
\pgfpathlineto{\pgfqpoint{1.556201in}{0.325844in}}%
\pgfpathlineto{\pgfqpoint{1.517424in}{0.302112in}}%
\pgfpathlineto{\pgfqpoint{1.483924in}{0.283683in}}%
\pgfpathlineto{\pgfqpoint{1.447786in}{0.265818in}}%
\pgfpathlineto{\pgfqpoint{1.411647in}{0.249918in}}%
\pgfpathlineto{\pgfqpoint{1.363463in}{0.231569in}}%
\pgfpathlineto{\pgfqpoint{1.320375in}{0.217790in}}%
\pgfpathlineto{\pgfqpoint{1.279140in}{0.206762in}}%
\pgfpathlineto{\pgfqpoint{1.230956in}{0.196509in}}%
\pgfpathlineto{\pgfqpoint{1.182771in}{0.188978in}}%
\pgfpathlineto{\pgfqpoint{1.134587in}{0.184100in}}%
\pgfpathlineto{\pgfqpoint{1.086402in}{0.181833in}}%
\pgfpathlineto{\pgfqpoint{1.038218in}{0.182157in}}%
\pgfpathlineto{\pgfqpoint{0.990033in}{0.185074in}}%
\pgfpathlineto{\pgfqpoint{0.941849in}{0.190610in}}%
\pgfpathlineto{\pgfqpoint{0.929803in}{0.192408in}}%
\pgfpathlineto{\pgfqpoint{0.929803in}{0.192408in}}%
\pgfusepath{stroke}%
\end{pgfscope}%
\begin{pgfscope}%
\pgfpathrectangle{\pgfqpoint{0.135000in}{0.145754in}}{\pgfqpoint{1.866666in}{1.866666in}} %
\pgfusepath{clip}%
\pgfsetbuttcap%
\pgfsetroundjoin%
\pgfsetlinewidth{1.505625pt}%
\definecolor{currentstroke}{rgb}{1.000000,1.000000,1.000000}%
\pgfsetstrokecolor{currentstroke}%
\pgfsetdash{}{0pt}%
\pgfpathmoveto{\pgfqpoint{0.929803in}{0.192408in}}%
\pgfpathlineto{\pgfqpoint{0.881618in}{0.201290in}}%
\pgfpathlineto{\pgfqpoint{0.833434in}{0.212940in}}%
\pgfpathlineto{\pgfqpoint{0.785249in}{0.227470in}}%
\pgfpathlineto{\pgfqpoint{0.745145in}{0.241882in}}%
\pgfpathlineto{\pgfqpoint{0.700927in}{0.260309in}}%
\pgfpathlineto{\pgfqpoint{0.663767in}{0.278020in}}%
\pgfpathlineto{\pgfqpoint{0.619242in}{0.302112in}}%
\pgfpathlineto{\pgfqpoint{0.592511in}{0.318183in}}%
\pgfpathlineto{\pgfqpoint{0.556373in}{0.342013in}}%
\pgfpathlineto{\pgfqpoint{0.520235in}{0.368478in}}%
\pgfpathlineto{\pgfqpoint{0.483387in}{0.398481in}}%
\pgfpathlineto{\pgfqpoint{0.447958in}{0.430619in}}%
\pgfpathlineto{\pgfqpoint{0.419865in}{0.458712in}}%
\pgfpathlineto{\pgfqpoint{0.387117in}{0.494850in}}%
\pgfpathlineto{\pgfqpoint{0.357724in}{0.530989in}}%
\pgfpathlineto{\pgfqpoint{0.331259in}{0.567127in}}%
\pgfpathlineto{\pgfqpoint{0.307429in}{0.603265in}}%
\pgfpathlineto{\pgfqpoint{0.285990in}{0.639404in}}%
\pgfpathlineto{\pgfqpoint{0.266746in}{0.675542in}}%
\pgfpathlineto{\pgfqpoint{0.249555in}{0.711680in}}%
\pgfpathlineto{\pgfqpoint{0.231128in}{0.755899in}}%
\pgfpathlineto{\pgfqpoint{0.216716in}{0.796003in}}%
\pgfpathlineto{\pgfqpoint{0.202186in}{0.844188in}}%
\pgfpathlineto{\pgfqpoint{0.190536in}{0.892372in}}%
\pgfpathlineto{\pgfqpoint{0.181654in}{0.940557in}}%
\pgfpathlineto{\pgfqpoint{0.175457in}{0.988741in}}%
\pgfpathlineto{\pgfqpoint{0.171888in}{1.036925in}}%
\pgfpathlineto{\pgfqpoint{0.170918in}{1.085110in}}%
\pgfpathlineto{\pgfqpoint{0.172536in}{1.133294in}}%
\pgfpathlineto{\pgfqpoint{0.176758in}{1.181479in}}%
\pgfpathlineto{\pgfqpoint{0.183620in}{1.229663in}}%
\pgfpathlineto{\pgfqpoint{0.193185in}{1.277848in}}%
\pgfpathlineto{\pgfqpoint{0.205543in}{1.326032in}}%
\pgfpathlineto{\pgfqpoint{0.220815in}{1.374217in}}%
\pgfpathlineto{\pgfqpoint{0.239164in}{1.422401in}}%
\pgfpathlineto{\pgfqpoint{0.255220in}{1.458870in}}%
\pgfpathlineto{\pgfqpoint{0.272929in}{1.494678in}}%
\pgfpathlineto{\pgfqpoint{0.292880in}{1.530816in}}%
\pgfpathlineto{\pgfqpoint{0.315451in}{1.567507in}}%
\pgfpathlineto{\pgfqpoint{0.339768in}{1.603093in}}%
\pgfpathlineto{\pgfqpoint{0.367179in}{1.639231in}}%
\pgfpathlineto{\pgfqpoint{0.399774in}{1.677757in}}%
\pgfpathlineto{\pgfqpoint{0.431609in}{1.711508in}}%
\pgfpathlineto{\pgfqpoint{0.460004in}{1.738870in}}%
\pgfpathlineto{\pgfqpoint{0.497707in}{1.771738in}}%
\pgfpathlineto{\pgfqpoint{0.532281in}{1.798829in}}%
\pgfpathlineto{\pgfqpoint{0.568419in}{1.824383in}}%
\pgfpathlineto{\pgfqpoint{0.604558in}{1.847394in}}%
\pgfpathlineto{\pgfqpoint{0.640739in}{1.868107in}}%
\pgfpathlineto{\pgfqpoint{0.676834in}{1.886622in}}%
\pgfpathlineto{\pgfqpoint{0.715512in}{1.904246in}}%
\pgfpathlineto{\pgfqpoint{0.761157in}{1.922312in}}%
\pgfpathlineto{\pgfqpoint{0.809342in}{1.938338in}}%
\pgfpathlineto{\pgfqpoint{0.857526in}{1.951412in}}%
\pgfpathlineto{\pgfqpoint{0.905710in}{1.961665in}}%
\pgfpathlineto{\pgfqpoint{0.953895in}{1.969196in}}%
\pgfpathlineto{\pgfqpoint{1.002079in}{1.974073in}}%
\pgfpathlineto{\pgfqpoint{1.050264in}{1.976341in}}%
\pgfpathlineto{\pgfqpoint{1.098448in}{1.976017in}}%
\pgfpathlineto{\pgfqpoint{1.146633in}{1.973100in}}%
\pgfpathlineto{\pgfqpoint{1.194817in}{1.967564in}}%
\pgfpathlineto{\pgfqpoint{1.243002in}{1.959360in}}%
\pgfpathlineto{\pgfqpoint{1.291186in}{1.948411in}}%
\pgfpathlineto{\pgfqpoint{1.339371in}{1.934614in}}%
\pgfpathlineto{\pgfqpoint{1.387555in}{1.917828in}}%
\pgfpathlineto{\pgfqpoint{1.423693in}{1.903167in}}%
\pgfpathlineto{\pgfqpoint{1.459832in}{1.886622in}}%
\pgfpathlineto{\pgfqpoint{1.495970in}{1.868084in}}%
\pgfpathlineto{\pgfqpoint{1.537640in}{1.844015in}}%
\pgfpathlineto{\pgfqpoint{1.574819in}{1.819923in}}%
\pgfpathlineto{\pgfqpoint{1.608378in}{1.795831in}}%
\pgfpathlineto{\pgfqpoint{1.640524in}{1.770450in}}%
\pgfpathlineto{\pgfqpoint{1.680187in}{1.735600in}}%
\pgfpathlineto{\pgfqpoint{1.712800in}{1.703615in}}%
\pgfpathlineto{\pgfqpoint{1.739025in}{1.675370in}}%
\pgfpathlineto{\pgfqpoint{1.769487in}{1.639231in}}%
\pgfpathlineto{\pgfqpoint{1.797123in}{1.602779in}}%
\pgfpathlineto{\pgfqpoint{1.821576in}{1.566955in}}%
\pgfpathlineto{\pgfqpoint{1.845307in}{1.528178in}}%
\pgfpathlineto{\pgfqpoint{1.863737in}{1.494678in}}%
\pgfpathlineto{\pgfqpoint{1.881602in}{1.458539in}}%
\pgfpathlineto{\pgfqpoint{1.897502in}{1.422401in}}%
\pgfpathlineto{\pgfqpoint{1.915851in}{1.374217in}}%
\pgfpathlineto{\pgfqpoint{1.929630in}{1.331129in}}%
\pgfpathlineto{\pgfqpoint{1.940658in}{1.289894in}}%
\pgfpathlineto{\pgfqpoint{1.950911in}{1.241709in}}%
\pgfpathlineto{\pgfqpoint{1.958442in}{1.193525in}}%
\pgfpathlineto{\pgfqpoint{1.963320in}{1.145340in}}%
\pgfpathlineto{\pgfqpoint{1.965587in}{1.097156in}}%
\pgfpathlineto{\pgfqpoint{1.965263in}{1.048972in}}%
\pgfpathlineto{\pgfqpoint{1.962346in}{1.000787in}}%
\pgfpathlineto{\pgfqpoint{1.956810in}{0.952603in}}%
\pgfpathlineto{\pgfqpoint{1.948606in}{0.904418in}}%
\pgfpathlineto{\pgfqpoint{1.937658in}{0.856234in}}%
\pgfpathlineto{\pgfqpoint{1.923860in}{0.808049in}}%
\pgfpathlineto{\pgfqpoint{1.907074in}{0.759865in}}%
\pgfpathlineto{\pgfqpoint{1.892413in}{0.723726in}}%
\pgfpathlineto{\pgfqpoint{1.875868in}{0.687588in}}%
\pgfpathlineto{\pgfqpoint{1.857331in}{0.651450in}}%
\pgfpathlineto{\pgfqpoint{1.833261in}{0.609780in}}%
\pgfpathlineto{\pgfqpoint{1.809169in}{0.572601in}}%
\pgfpathlineto{\pgfqpoint{1.785077in}{0.539042in}}%
\pgfpathlineto{\pgfqpoint{1.759696in}{0.506896in}}%
\pgfpathlineto{\pgfqpoint{1.724846in}{0.467233in}}%
\pgfpathlineto{\pgfqpoint{1.692862in}{0.434620in}}%
\pgfpathlineto{\pgfqpoint{1.664616in}{0.408394in}}%
\pgfpathlineto{\pgfqpoint{1.628477in}{0.377933in}}%
\pgfpathlineto{\pgfqpoint{1.592025in}{0.350297in}}%
\pgfpathlineto{\pgfqpoint{1.556201in}{0.325844in}}%
\pgfpathlineto{\pgfqpoint{1.517424in}{0.302112in}}%
\pgfpathlineto{\pgfqpoint{1.483924in}{0.283683in}}%
\pgfpathlineto{\pgfqpoint{1.447786in}{0.265818in}}%
\pgfpathlineto{\pgfqpoint{1.411647in}{0.249918in}}%
\pgfpathlineto{\pgfqpoint{1.363463in}{0.231569in}}%
\pgfpathlineto{\pgfqpoint{1.320375in}{0.217790in}}%
\pgfpathlineto{\pgfqpoint{1.279140in}{0.206762in}}%
\pgfpathlineto{\pgfqpoint{1.230956in}{0.196509in}}%
\pgfpathlineto{\pgfqpoint{1.182771in}{0.188978in}}%
\pgfpathlineto{\pgfqpoint{1.134587in}{0.184100in}}%
\pgfpathlineto{\pgfqpoint{1.086402in}{0.181833in}}%
\pgfpathlineto{\pgfqpoint{1.038218in}{0.182157in}}%
\pgfpathlineto{\pgfqpoint{0.990033in}{0.185074in}}%
\pgfpathlineto{\pgfqpoint{0.941849in}{0.190610in}}%
\pgfpathlineto{\pgfqpoint{0.929803in}{0.192408in}}%
\pgfpathlineto{\pgfqpoint{0.929803in}{0.192408in}}%
\pgfusepath{stroke}%
\end{pgfscope}%
\begin{pgfscope}%
\pgfpathrectangle{\pgfqpoint{0.135000in}{0.145754in}}{\pgfqpoint{1.866666in}{1.866666in}} %
\pgfusepath{clip}%
\pgfsetbuttcap%
\pgfsetroundjoin%
\pgfsetlinewidth{1.505625pt}%
\definecolor{currentstroke}{rgb}{0.000000,0.000000,1.000000}%
\pgfsetstrokecolor{currentstroke}%
\pgfsetdash{}{0pt}%
\pgfpathmoveto{\pgfqpoint{0.929803in}{0.192408in}}%
\pgfpathlineto{\pgfqpoint{0.881618in}{0.201290in}}%
\pgfpathlineto{\pgfqpoint{0.833434in}{0.212940in}}%
\pgfpathlineto{\pgfqpoint{0.785249in}{0.227470in}}%
\pgfpathlineto{\pgfqpoint{0.745145in}{0.241882in}}%
\pgfpathlineto{\pgfqpoint{0.700927in}{0.260309in}}%
\pgfpathlineto{\pgfqpoint{0.663767in}{0.278020in}}%
\pgfpathlineto{\pgfqpoint{0.619242in}{0.302112in}}%
\pgfpathlineto{\pgfqpoint{0.592511in}{0.318183in}}%
\pgfpathlineto{\pgfqpoint{0.556373in}{0.342013in}}%
\pgfpathlineto{\pgfqpoint{0.520235in}{0.368478in}}%
\pgfpathlineto{\pgfqpoint{0.483387in}{0.398481in}}%
\pgfpathlineto{\pgfqpoint{0.447958in}{0.430619in}}%
\pgfpathlineto{\pgfqpoint{0.419865in}{0.458712in}}%
\pgfpathlineto{\pgfqpoint{0.387117in}{0.494850in}}%
\pgfpathlineto{\pgfqpoint{0.357724in}{0.530989in}}%
\pgfpathlineto{\pgfqpoint{0.331259in}{0.567127in}}%
\pgfpathlineto{\pgfqpoint{0.307429in}{0.603265in}}%
\pgfpathlineto{\pgfqpoint{0.285990in}{0.639404in}}%
\pgfpathlineto{\pgfqpoint{0.266746in}{0.675542in}}%
\pgfpathlineto{\pgfqpoint{0.249555in}{0.711680in}}%
\pgfpathlineto{\pgfqpoint{0.231128in}{0.755899in}}%
\pgfpathlineto{\pgfqpoint{0.216716in}{0.796003in}}%
\pgfpathlineto{\pgfqpoint{0.202186in}{0.844188in}}%
\pgfpathlineto{\pgfqpoint{0.190536in}{0.892372in}}%
\pgfpathlineto{\pgfqpoint{0.181654in}{0.940557in}}%
\pgfpathlineto{\pgfqpoint{0.175457in}{0.988741in}}%
\pgfpathlineto{\pgfqpoint{0.171888in}{1.036925in}}%
\pgfpathlineto{\pgfqpoint{0.170918in}{1.085110in}}%
\pgfpathlineto{\pgfqpoint{0.172536in}{1.133294in}}%
\pgfpathlineto{\pgfqpoint{0.176758in}{1.181479in}}%
\pgfpathlineto{\pgfqpoint{0.183620in}{1.229663in}}%
\pgfpathlineto{\pgfqpoint{0.193185in}{1.277848in}}%
\pgfpathlineto{\pgfqpoint{0.205543in}{1.326032in}}%
\pgfpathlineto{\pgfqpoint{0.220815in}{1.374217in}}%
\pgfpathlineto{\pgfqpoint{0.239164in}{1.422401in}}%
\pgfpathlineto{\pgfqpoint{0.255220in}{1.458870in}}%
\pgfpathlineto{\pgfqpoint{0.272929in}{1.494678in}}%
\pgfpathlineto{\pgfqpoint{0.292880in}{1.530816in}}%
\pgfpathlineto{\pgfqpoint{0.315451in}{1.567507in}}%
\pgfpathlineto{\pgfqpoint{0.339768in}{1.603093in}}%
\pgfpathlineto{\pgfqpoint{0.367179in}{1.639231in}}%
\pgfpathlineto{\pgfqpoint{0.399774in}{1.677757in}}%
\pgfpathlineto{\pgfqpoint{0.431609in}{1.711508in}}%
\pgfpathlineto{\pgfqpoint{0.460004in}{1.738870in}}%
\pgfpathlineto{\pgfqpoint{0.497707in}{1.771738in}}%
\pgfpathlineto{\pgfqpoint{0.532281in}{1.798829in}}%
\pgfpathlineto{\pgfqpoint{0.568419in}{1.824383in}}%
\pgfpathlineto{\pgfqpoint{0.604558in}{1.847394in}}%
\pgfpathlineto{\pgfqpoint{0.640739in}{1.868107in}}%
\pgfpathlineto{\pgfqpoint{0.676834in}{1.886622in}}%
\pgfpathlineto{\pgfqpoint{0.715512in}{1.904246in}}%
\pgfpathlineto{\pgfqpoint{0.761157in}{1.922312in}}%
\pgfpathlineto{\pgfqpoint{0.809342in}{1.938338in}}%
\pgfpathlineto{\pgfqpoint{0.857526in}{1.951412in}}%
\pgfpathlineto{\pgfqpoint{0.905710in}{1.961665in}}%
\pgfpathlineto{\pgfqpoint{0.953895in}{1.969196in}}%
\pgfpathlineto{\pgfqpoint{1.002079in}{1.974073in}}%
\pgfpathlineto{\pgfqpoint{1.050264in}{1.976341in}}%
\pgfpathlineto{\pgfqpoint{1.098448in}{1.976017in}}%
\pgfpathlineto{\pgfqpoint{1.146633in}{1.973100in}}%
\pgfpathlineto{\pgfqpoint{1.194817in}{1.967564in}}%
\pgfpathlineto{\pgfqpoint{1.243002in}{1.959360in}}%
\pgfpathlineto{\pgfqpoint{1.291186in}{1.948411in}}%
\pgfpathlineto{\pgfqpoint{1.339371in}{1.934614in}}%
\pgfpathlineto{\pgfqpoint{1.387555in}{1.917828in}}%
\pgfpathlineto{\pgfqpoint{1.423693in}{1.903167in}}%
\pgfpathlineto{\pgfqpoint{1.459832in}{1.886622in}}%
\pgfpathlineto{\pgfqpoint{1.495970in}{1.868084in}}%
\pgfpathlineto{\pgfqpoint{1.537640in}{1.844015in}}%
\pgfpathlineto{\pgfqpoint{1.574819in}{1.819923in}}%
\pgfpathlineto{\pgfqpoint{1.608378in}{1.795831in}}%
\pgfpathlineto{\pgfqpoint{1.640524in}{1.770450in}}%
\pgfpathlineto{\pgfqpoint{1.680187in}{1.735600in}}%
\pgfpathlineto{\pgfqpoint{1.712800in}{1.703615in}}%
\pgfpathlineto{\pgfqpoint{1.739025in}{1.675370in}}%
\pgfpathlineto{\pgfqpoint{1.769487in}{1.639231in}}%
\pgfpathlineto{\pgfqpoint{1.797123in}{1.602779in}}%
\pgfpathlineto{\pgfqpoint{1.821576in}{1.566955in}}%
\pgfpathlineto{\pgfqpoint{1.845307in}{1.528178in}}%
\pgfpathlineto{\pgfqpoint{1.863737in}{1.494678in}}%
\pgfpathlineto{\pgfqpoint{1.881602in}{1.458539in}}%
\pgfpathlineto{\pgfqpoint{1.897502in}{1.422401in}}%
\pgfpathlineto{\pgfqpoint{1.915851in}{1.374217in}}%
\pgfpathlineto{\pgfqpoint{1.929630in}{1.331129in}}%
\pgfpathlineto{\pgfqpoint{1.940658in}{1.289894in}}%
\pgfpathlineto{\pgfqpoint{1.950911in}{1.241709in}}%
\pgfpathlineto{\pgfqpoint{1.958442in}{1.193525in}}%
\pgfpathlineto{\pgfqpoint{1.963320in}{1.145340in}}%
\pgfpathlineto{\pgfqpoint{1.965587in}{1.097156in}}%
\pgfpathlineto{\pgfqpoint{1.965263in}{1.048972in}}%
\pgfpathlineto{\pgfqpoint{1.962346in}{1.000787in}}%
\pgfpathlineto{\pgfqpoint{1.956810in}{0.952603in}}%
\pgfpathlineto{\pgfqpoint{1.948606in}{0.904418in}}%
\pgfpathlineto{\pgfqpoint{1.937658in}{0.856234in}}%
\pgfpathlineto{\pgfqpoint{1.923860in}{0.808049in}}%
\pgfpathlineto{\pgfqpoint{1.907074in}{0.759865in}}%
\pgfpathlineto{\pgfqpoint{1.892413in}{0.723726in}}%
\pgfpathlineto{\pgfqpoint{1.875868in}{0.687588in}}%
\pgfpathlineto{\pgfqpoint{1.857331in}{0.651450in}}%
\pgfpathlineto{\pgfqpoint{1.833261in}{0.609780in}}%
\pgfpathlineto{\pgfqpoint{1.809169in}{0.572601in}}%
\pgfpathlineto{\pgfqpoint{1.785077in}{0.539042in}}%
\pgfpathlineto{\pgfqpoint{1.759696in}{0.506896in}}%
\pgfpathlineto{\pgfqpoint{1.724846in}{0.467233in}}%
\pgfpathlineto{\pgfqpoint{1.692862in}{0.434620in}}%
\pgfpathlineto{\pgfqpoint{1.664616in}{0.408394in}}%
\pgfpathlineto{\pgfqpoint{1.628477in}{0.377933in}}%
\pgfpathlineto{\pgfqpoint{1.592025in}{0.350297in}}%
\pgfpathlineto{\pgfqpoint{1.556201in}{0.325844in}}%
\pgfpathlineto{\pgfqpoint{1.517424in}{0.302112in}}%
\pgfpathlineto{\pgfqpoint{1.483924in}{0.283683in}}%
\pgfpathlineto{\pgfqpoint{1.447786in}{0.265818in}}%
\pgfpathlineto{\pgfqpoint{1.411647in}{0.249918in}}%
\pgfpathlineto{\pgfqpoint{1.363463in}{0.231569in}}%
\pgfpathlineto{\pgfqpoint{1.320375in}{0.217790in}}%
\pgfpathlineto{\pgfqpoint{1.279140in}{0.206762in}}%
\pgfpathlineto{\pgfqpoint{1.230956in}{0.196509in}}%
\pgfpathlineto{\pgfqpoint{1.182771in}{0.188978in}}%
\pgfpathlineto{\pgfqpoint{1.134587in}{0.184100in}}%
\pgfpathlineto{\pgfqpoint{1.086402in}{0.181833in}}%
\pgfpathlineto{\pgfqpoint{1.038218in}{0.182157in}}%
\pgfpathlineto{\pgfqpoint{0.990033in}{0.185074in}}%
\pgfpathlineto{\pgfqpoint{0.941849in}{0.190610in}}%
\pgfpathlineto{\pgfqpoint{0.929803in}{0.192408in}}%
\pgfpathlineto{\pgfqpoint{0.929803in}{0.192408in}}%
\pgfusepath{stroke}%
\end{pgfscope}%
\begin{pgfscope}%
\pgfsetrectcap%
\pgfsetmiterjoin%
\pgfsetlinewidth{0.803000pt}%
\definecolor{currentstroke}{rgb}{0.000000,0.000000,0.000000}%
\pgfsetstrokecolor{currentstroke}%
\pgfsetdash{}{0pt}%
\pgfpathmoveto{\pgfqpoint{1.068333in}{0.145754in}}%
\pgfpathlineto{\pgfqpoint{1.068333in}{2.012420in}}%
\pgfusepath{stroke}%
\end{pgfscope}%
\begin{pgfscope}%
\pgfsetrectcap%
\pgfsetmiterjoin%
\pgfsetlinewidth{0.803000pt}%
\definecolor{currentstroke}{rgb}{0.000000,0.000000,0.000000}%
\pgfsetstrokecolor{currentstroke}%
\pgfsetdash{}{0pt}%
\pgfpathmoveto{\pgfqpoint{0.135000in}{1.079087in}}%
\pgfpathlineto{\pgfqpoint{2.001666in}{1.079087in}}%
\pgfusepath{stroke}%
\end{pgfscope}%
\end{pgfpicture}%
\makeatother%
\endgroup%

            \end{center}
        \item Sí que es una distancia
            \begin{center}
                %% Creator: Matplotlib, PGF backend
%%
%% To include the figure in your LaTeX document, write
%%   \input{<filename>.pgf}
%%
%% Make sure the required packages are loaded in your preamble
%%   \usepackage{pgf}
%%
%% Figures using additional raster images can only be included by \input if
%% they are in the same directory as the main LaTeX file. For loading figures
%% from other directories you can use the `import` package
%%   \usepackage{import}
%% and then include the figures with
%%   \import{<path to file>}{<filename>.pgf}
%%
%% Matplotlib used the following preamble
%%   \usepackage{fontspec}
%%   \setmainfont{DejaVu Serif}
%%   \setsansfont{DejaVu Sans}
%%   \setmonofont{DejaVu Sans Mono}
%%
\begingroup%
\makeatletter%
\begin{pgfpicture}%
\pgfpathrectangle{\pgfpointorigin}{\pgfqpoint{2.136666in}{2.147420in}}%
\pgfusepath{use as bounding box, clip}%
\begin{pgfscope}%
\pgfsetbuttcap%
\pgfsetmiterjoin%
\definecolor{currentfill}{rgb}{1.000000,1.000000,1.000000}%
\pgfsetfillcolor{currentfill}%
\pgfsetlinewidth{0.000000pt}%
\definecolor{currentstroke}{rgb}{1.000000,1.000000,1.000000}%
\pgfsetstrokecolor{currentstroke}%
\pgfsetdash{}{0pt}%
\pgfpathmoveto{\pgfqpoint{0.000000in}{0.000000in}}%
\pgfpathlineto{\pgfqpoint{2.136666in}{0.000000in}}%
\pgfpathlineto{\pgfqpoint{2.136666in}{2.147420in}}%
\pgfpathlineto{\pgfqpoint{0.000000in}{2.147420in}}%
\pgfpathclose%
\pgfusepath{fill}%
\end{pgfscope}%
\begin{pgfscope}%
\pgfsetbuttcap%
\pgfsetmiterjoin%
\definecolor{currentfill}{rgb}{1.000000,1.000000,1.000000}%
\pgfsetfillcolor{currentfill}%
\pgfsetlinewidth{0.000000pt}%
\definecolor{currentstroke}{rgb}{0.000000,0.000000,0.000000}%
\pgfsetstrokecolor{currentstroke}%
\pgfsetstrokeopacity{0.000000}%
\pgfsetdash{}{0pt}%
\pgfpathmoveto{\pgfqpoint{0.135000in}{0.145754in}}%
\pgfpathlineto{\pgfqpoint{2.001666in}{0.145754in}}%
\pgfpathlineto{\pgfqpoint{2.001666in}{2.012420in}}%
\pgfpathlineto{\pgfqpoint{0.135000in}{2.012420in}}%
\pgfpathclose%
\pgfusepath{fill}%
\end{pgfscope}%
\begin{pgfscope}%
\pgfpathrectangle{\pgfqpoint{0.135000in}{0.145754in}}{\pgfqpoint{1.866666in}{1.866666in}} %
\pgfusepath{clip}%
\pgfsetbuttcap%
\pgfsetroundjoin%
\definecolor{currentfill}{rgb}{0.000000,0.000000,1.000000}%
\pgfsetfillcolor{currentfill}%
\pgfsetfillopacity{0.300000}%
\pgfsetlinewidth{0.000000pt}%
\definecolor{currentstroke}{rgb}{0.000000,0.000000,0.000000}%
\pgfsetstrokecolor{currentstroke}%
\pgfsetdash{}{0pt}%
\pgfpathmoveto{\pgfqpoint{0.929803in}{0.192415in}}%
\pgfpathlineto{\pgfqpoint{0.941849in}{0.190625in}}%
\pgfpathlineto{\pgfqpoint{0.953895in}{0.188997in}}%
\pgfpathlineto{\pgfqpoint{0.965941in}{0.187532in}}%
\pgfpathlineto{\pgfqpoint{0.977987in}{0.186230in}}%
\pgfpathlineto{\pgfqpoint{0.990033in}{0.185090in}}%
\pgfpathlineto{\pgfqpoint{1.002079in}{0.184113in}}%
\pgfpathlineto{\pgfqpoint{1.014126in}{0.183299in}}%
\pgfpathlineto{\pgfqpoint{1.026172in}{0.182648in}}%
\pgfpathlineto{\pgfqpoint{1.038218in}{0.182160in}}%
\pgfpathlineto{\pgfqpoint{1.050264in}{0.181834in}}%
\pgfpathlineto{\pgfqpoint{1.062310in}{0.181672in}}%
\pgfpathlineto{\pgfqpoint{1.074356in}{0.181672in}}%
\pgfpathlineto{\pgfqpoint{1.086402in}{0.181834in}}%
\pgfpathlineto{\pgfqpoint{1.098448in}{0.182160in}}%
\pgfpathlineto{\pgfqpoint{1.110494in}{0.182648in}}%
\pgfpathlineto{\pgfqpoint{1.122541in}{0.183299in}}%
\pgfpathlineto{\pgfqpoint{1.134587in}{0.184113in}}%
\pgfpathlineto{\pgfqpoint{1.146633in}{0.185090in}}%
\pgfpathlineto{\pgfqpoint{1.158679in}{0.186230in}}%
\pgfpathlineto{\pgfqpoint{1.170725in}{0.187532in}}%
\pgfpathlineto{\pgfqpoint{1.182771in}{0.188997in}}%
\pgfpathlineto{\pgfqpoint{1.194817in}{0.190625in}}%
\pgfpathlineto{\pgfqpoint{1.206863in}{0.192415in}}%
\pgfpathlineto{\pgfqpoint{1.214769in}{0.193697in}}%
\pgfpathlineto{\pgfqpoint{1.218909in}{0.194378in}}%
\pgfpathlineto{\pgfqpoint{1.230956in}{0.196523in}}%
\pgfpathlineto{\pgfqpoint{1.243002in}{0.198833in}}%
\pgfpathlineto{\pgfqpoint{1.255048in}{0.201309in}}%
\pgfpathlineto{\pgfqpoint{1.267094in}{0.203949in}}%
\pgfpathlineto{\pgfqpoint{1.274800in}{0.205743in}}%
\pgfpathlineto{\pgfqpoint{1.279140in}{0.206768in}}%
\pgfpathlineto{\pgfqpoint{1.291186in}{0.209780in}}%
\pgfpathlineto{\pgfqpoint{1.303232in}{0.212959in}}%
\pgfpathlineto{\pgfqpoint{1.315278in}{0.216305in}}%
\pgfpathlineto{\pgfqpoint{1.320369in}{0.217790in}}%
\pgfpathlineto{\pgfqpoint{1.327325in}{0.219847in}}%
\pgfpathlineto{\pgfqpoint{1.339371in}{0.223579in}}%
\pgfpathlineto{\pgfqpoint{1.351417in}{0.227482in}}%
\pgfpathlineto{\pgfqpoint{1.358381in}{0.229836in}}%
\pgfpathlineto{\pgfqpoint{1.363463in}{0.231578in}}%
\pgfpathlineto{\pgfqpoint{1.375509in}{0.235880in}}%
\pgfpathlineto{\pgfqpoint{1.387555in}{0.240355in}}%
\pgfpathlineto{\pgfqpoint{1.391515in}{0.241882in}}%
\pgfpathlineto{\pgfqpoint{1.399601in}{0.245046in}}%
\pgfpathlineto{\pgfqpoint{1.411647in}{0.249934in}}%
\pgfpathlineto{\pgfqpoint{1.421149in}{0.253928in}}%
\pgfpathlineto{\pgfqpoint{1.423693in}{0.255013in}}%
\pgfpathlineto{\pgfqpoint{1.435740in}{0.260327in}}%
\pgfpathlineto{\pgfqpoint{1.447786in}{0.265819in}}%
\pgfpathlineto{\pgfqpoint{1.448115in}{0.265974in}}%
\pgfpathlineto{\pgfqpoint{1.459832in}{0.271570in}}%
\pgfpathlineto{\pgfqpoint{1.471878in}{0.277503in}}%
\pgfpathlineto{\pgfqpoint{1.472897in}{0.278020in}}%
\pgfpathlineto{\pgfqpoint{1.483924in}{0.283701in}}%
\pgfpathlineto{\pgfqpoint{1.495927in}{0.290066in}}%
\pgfpathlineto{\pgfqpoint{1.495970in}{0.290089in}}%
\pgfpathlineto{\pgfqpoint{1.508016in}{0.296761in}}%
\pgfpathlineto{\pgfqpoint{1.517417in}{0.302112in}}%
\pgfpathlineto{\pgfqpoint{1.520062in}{0.303642in}}%
\pgfpathlineto{\pgfqpoint{1.532108in}{0.310794in}}%
\pgfpathlineto{\pgfqpoint{1.537630in}{0.314159in}}%
\pgfpathlineto{\pgfqpoint{1.544155in}{0.318198in}}%
\pgfpathlineto{\pgfqpoint{1.556201in}{0.325846in}}%
\pgfpathlineto{\pgfqpoint{1.556752in}{0.326205in}}%
\pgfpathlineto{\pgfqpoint{1.568247in}{0.333806in}}%
\pgfpathlineto{\pgfqpoint{1.574808in}{0.338251in}}%
\pgfpathlineto{\pgfqpoint{1.580293in}{0.342028in}}%
\pgfpathlineto{\pgfqpoint{1.592024in}{0.350297in}}%
\pgfpathlineto{\pgfqpoint{1.592339in}{0.350523in}}%
\pgfpathlineto{\pgfqpoint{1.604385in}{0.359357in}}%
\pgfpathlineto{\pgfqpoint{1.608367in}{0.362343in}}%
\pgfpathlineto{\pgfqpoint{1.616431in}{0.368494in}}%
\pgfpathlineto{\pgfqpoint{1.623993in}{0.374389in}}%
\pgfpathlineto{\pgfqpoint{1.628477in}{0.377946in}}%
\pgfpathlineto{\pgfqpoint{1.638954in}{0.386435in}}%
\pgfpathlineto{\pgfqpoint{1.640524in}{0.387730in}}%
\pgfpathlineto{\pgfqpoint{1.652570in}{0.397874in}}%
\pgfpathlineto{\pgfqpoint{1.653276in}{0.398481in}}%
\pgfpathlineto{\pgfqpoint{1.664616in}{0.408403in}}%
\pgfpathlineto{\pgfqpoint{1.666995in}{0.410527in}}%
\pgfpathlineto{\pgfqpoint{1.676662in}{0.419316in}}%
\pgfpathlineto{\pgfqpoint{1.680175in}{0.422574in}}%
\pgfpathlineto{\pgfqpoint{1.688708in}{0.430632in}}%
\pgfpathlineto{\pgfqpoint{1.692849in}{0.434620in}}%
\pgfpathlineto{\pgfqpoint{1.700754in}{0.442376in}}%
\pgfpathlineto{\pgfqpoint{1.705044in}{0.446666in}}%
\pgfpathlineto{\pgfqpoint{1.712800in}{0.454571in}}%
\pgfpathlineto{\pgfqpoint{1.716788in}{0.458712in}}%
\pgfpathlineto{\pgfqpoint{1.724846in}{0.467245in}}%
\pgfpathlineto{\pgfqpoint{1.728104in}{0.470758in}}%
\pgfpathlineto{\pgfqpoint{1.736892in}{0.480425in}}%
\pgfpathlineto{\pgfqpoint{1.739017in}{0.482804in}}%
\pgfpathlineto{\pgfqpoint{1.748939in}{0.494143in}}%
\pgfpathlineto{\pgfqpoint{1.749546in}{0.494850in}}%
\pgfpathlineto{\pgfqpoint{1.759690in}{0.506896in}}%
\pgfpathlineto{\pgfqpoint{1.760985in}{0.508466in}}%
\pgfpathlineto{\pgfqpoint{1.769474in}{0.518942in}}%
\pgfpathlineto{\pgfqpoint{1.773031in}{0.523427in}}%
\pgfpathlineto{\pgfqpoint{1.778926in}{0.530989in}}%
\pgfpathlineto{\pgfqpoint{1.785077in}{0.539053in}}%
\pgfpathlineto{\pgfqpoint{1.788063in}{0.543035in}}%
\pgfpathlineto{\pgfqpoint{1.796897in}{0.555081in}}%
\pgfpathlineto{\pgfqpoint{1.797123in}{0.555396in}}%
\pgfpathlineto{\pgfqpoint{1.805392in}{0.567127in}}%
\pgfpathlineto{\pgfqpoint{1.809169in}{0.572612in}}%
\pgfpathlineto{\pgfqpoint{1.813614in}{0.579173in}}%
\pgfpathlineto{\pgfqpoint{1.821215in}{0.590668in}}%
\pgfpathlineto{\pgfqpoint{1.821574in}{0.591219in}}%
\pgfpathlineto{\pgfqpoint{1.829222in}{0.603265in}}%
\pgfpathlineto{\pgfqpoint{1.833261in}{0.609790in}}%
\pgfpathlineto{\pgfqpoint{1.836626in}{0.615311in}}%
\pgfpathlineto{\pgfqpoint{1.843778in}{0.627358in}}%
\pgfpathlineto{\pgfqpoint{1.845307in}{0.630003in}}%
\pgfpathlineto{\pgfqpoint{1.850659in}{0.639404in}}%
\pgfpathlineto{\pgfqpoint{1.857330in}{0.651450in}}%
\pgfpathlineto{\pgfqpoint{1.857354in}{0.651493in}}%
\pgfpathlineto{\pgfqpoint{1.863719in}{0.663496in}}%
\pgfpathlineto{\pgfqpoint{1.869400in}{0.674523in}}%
\pgfpathlineto{\pgfqpoint{1.869917in}{0.675542in}}%
\pgfpathlineto{\pgfqpoint{1.875850in}{0.687588in}}%
\pgfpathlineto{\pgfqpoint{1.881446in}{0.699305in}}%
\pgfpathlineto{\pgfqpoint{1.881601in}{0.699634in}}%
\pgfpathlineto{\pgfqpoint{1.887092in}{0.711680in}}%
\pgfpathlineto{\pgfqpoint{1.892407in}{0.723726in}}%
\pgfpathlineto{\pgfqpoint{1.893492in}{0.726271in}}%
\pgfpathlineto{\pgfqpoint{1.897485in}{0.735773in}}%
\pgfpathlineto{\pgfqpoint{1.902374in}{0.747819in}}%
\pgfpathlineto{\pgfqpoint{1.905538in}{0.755905in}}%
\pgfpathlineto{\pgfqpoint{1.907065in}{0.759865in}}%
\pgfpathlineto{\pgfqpoint{1.911540in}{0.771911in}}%
\pgfpathlineto{\pgfqpoint{1.915842in}{0.783957in}}%
\pgfpathlineto{\pgfqpoint{1.917584in}{0.789039in}}%
\pgfpathlineto{\pgfqpoint{1.919938in}{0.796003in}}%
\pgfpathlineto{\pgfqpoint{1.923841in}{0.808049in}}%
\pgfpathlineto{\pgfqpoint{1.927573in}{0.820095in}}%
\pgfpathlineto{\pgfqpoint{1.929630in}{0.827051in}}%
\pgfpathlineto{\pgfqpoint{1.931115in}{0.832141in}}%
\pgfpathlineto{\pgfqpoint{1.934461in}{0.844188in}}%
\pgfpathlineto{\pgfqpoint{1.937640in}{0.856234in}}%
\pgfpathlineto{\pgfqpoint{1.940652in}{0.868280in}}%
\pgfpathlineto{\pgfqpoint{1.941676in}{0.872620in}}%
\pgfpathlineto{\pgfqpoint{1.943471in}{0.880326in}}%
\pgfpathlineto{\pgfqpoint{1.946111in}{0.892372in}}%
\pgfpathlineto{\pgfqpoint{1.948586in}{0.904418in}}%
\pgfpathlineto{\pgfqpoint{1.950897in}{0.916464in}}%
\pgfpathlineto{\pgfqpoint{1.953042in}{0.928510in}}%
\pgfpathlineto{\pgfqpoint{1.953723in}{0.932651in}}%
\pgfpathlineto{\pgfqpoint{1.955004in}{0.940557in}}%
\pgfpathlineto{\pgfqpoint{1.956795in}{0.952603in}}%
\pgfpathlineto{\pgfqpoint{1.958423in}{0.964649in}}%
\pgfpathlineto{\pgfqpoint{1.959888in}{0.976695in}}%
\pgfpathlineto{\pgfqpoint{1.961190in}{0.988741in}}%
\pgfpathlineto{\pgfqpoint{1.962330in}{1.000787in}}%
\pgfpathlineto{\pgfqpoint{1.963307in}{1.012833in}}%
\pgfpathlineto{\pgfqpoint{1.964120in}{1.024879in}}%
\pgfpathlineto{\pgfqpoint{1.964772in}{1.036925in}}%
\pgfpathlineto{\pgfqpoint{1.965260in}{1.048972in}}%
\pgfpathlineto{\pgfqpoint{1.965585in}{1.061018in}}%
\pgfpathlineto{\pgfqpoint{1.965748in}{1.073064in}}%
\pgfpathlineto{\pgfqpoint{1.965748in}{1.085110in}}%
\pgfpathlineto{\pgfqpoint{1.965585in}{1.097156in}}%
\pgfpathlineto{\pgfqpoint{1.965260in}{1.109202in}}%
\pgfpathlineto{\pgfqpoint{1.964772in}{1.121248in}}%
\pgfpathlineto{\pgfqpoint{1.964120in}{1.133294in}}%
\pgfpathlineto{\pgfqpoint{1.963307in}{1.145340in}}%
\pgfpathlineto{\pgfqpoint{1.962330in}{1.157387in}}%
\pgfpathlineto{\pgfqpoint{1.961190in}{1.169433in}}%
\pgfpathlineto{\pgfqpoint{1.959888in}{1.181479in}}%
\pgfpathlineto{\pgfqpoint{1.958423in}{1.193525in}}%
\pgfpathlineto{\pgfqpoint{1.956795in}{1.205571in}}%
\pgfpathlineto{\pgfqpoint{1.955004in}{1.217617in}}%
\pgfpathlineto{\pgfqpoint{1.953723in}{1.225522in}}%
\pgfpathlineto{\pgfqpoint{1.953042in}{1.229663in}}%
\pgfpathlineto{\pgfqpoint{1.950897in}{1.241709in}}%
\pgfpathlineto{\pgfqpoint{1.948586in}{1.253756in}}%
\pgfpathlineto{\pgfqpoint{1.946111in}{1.265802in}}%
\pgfpathlineto{\pgfqpoint{1.943471in}{1.277848in}}%
\pgfpathlineto{\pgfqpoint{1.941676in}{1.285554in}}%
\pgfpathlineto{\pgfqpoint{1.940652in}{1.289894in}}%
\pgfpathlineto{\pgfqpoint{1.937640in}{1.301940in}}%
\pgfpathlineto{\pgfqpoint{1.934461in}{1.313986in}}%
\pgfpathlineto{\pgfqpoint{1.931115in}{1.326032in}}%
\pgfpathlineto{\pgfqpoint{1.929630in}{1.331123in}}%
\pgfpathlineto{\pgfqpoint{1.927573in}{1.338078in}}%
\pgfpathlineto{\pgfqpoint{1.923841in}{1.350124in}}%
\pgfpathlineto{\pgfqpoint{1.919938in}{1.362171in}}%
\pgfpathlineto{\pgfqpoint{1.917584in}{1.369135in}}%
\pgfpathlineto{\pgfqpoint{1.915842in}{1.374217in}}%
\pgfpathlineto{\pgfqpoint{1.911540in}{1.386263in}}%
\pgfpathlineto{\pgfqpoint{1.907065in}{1.398309in}}%
\pgfpathlineto{\pgfqpoint{1.905538in}{1.402269in}}%
\pgfpathlineto{\pgfqpoint{1.902374in}{1.410355in}}%
\pgfpathlineto{\pgfqpoint{1.897485in}{1.422401in}}%
\pgfpathlineto{\pgfqpoint{1.893492in}{1.431903in}}%
\pgfpathlineto{\pgfqpoint{1.892407in}{1.434447in}}%
\pgfpathlineto{\pgfqpoint{1.887092in}{1.446493in}}%
\pgfpathlineto{\pgfqpoint{1.881601in}{1.458539in}}%
\pgfpathlineto{\pgfqpoint{1.881446in}{1.458869in}}%
\pgfpathlineto{\pgfqpoint{1.875850in}{1.470586in}}%
\pgfpathlineto{\pgfqpoint{1.869917in}{1.482632in}}%
\pgfpathlineto{\pgfqpoint{1.869400in}{1.483650in}}%
\pgfpathlineto{\pgfqpoint{1.863719in}{1.494678in}}%
\pgfpathlineto{\pgfqpoint{1.857354in}{1.506681in}}%
\pgfpathlineto{\pgfqpoint{1.857330in}{1.506724in}}%
\pgfpathlineto{\pgfqpoint{1.850659in}{1.518770in}}%
\pgfpathlineto{\pgfqpoint{1.845307in}{1.528171in}}%
\pgfpathlineto{\pgfqpoint{1.843778in}{1.530816in}}%
\pgfpathlineto{\pgfqpoint{1.836626in}{1.542862in}}%
\pgfpathlineto{\pgfqpoint{1.833261in}{1.548383in}}%
\pgfpathlineto{\pgfqpoint{1.829222in}{1.554908in}}%
\pgfpathlineto{\pgfqpoint{1.821574in}{1.566955in}}%
\pgfpathlineto{\pgfqpoint{1.821215in}{1.567505in}}%
\pgfpathlineto{\pgfqpoint{1.813614in}{1.579001in}}%
\pgfpathlineto{\pgfqpoint{1.809169in}{1.585561in}}%
\pgfpathlineto{\pgfqpoint{1.805392in}{1.591047in}}%
\pgfpathlineto{\pgfqpoint{1.797123in}{1.602778in}}%
\pgfpathlineto{\pgfqpoint{1.796897in}{1.603093in}}%
\pgfpathlineto{\pgfqpoint{1.788063in}{1.615139in}}%
\pgfpathlineto{\pgfqpoint{1.785077in}{1.619121in}}%
\pgfpathlineto{\pgfqpoint{1.778926in}{1.627185in}}%
\pgfpathlineto{\pgfqpoint{1.773031in}{1.634747in}}%
\pgfpathlineto{\pgfqpoint{1.769474in}{1.639231in}}%
\pgfpathlineto{\pgfqpoint{1.760985in}{1.649708in}}%
\pgfpathlineto{\pgfqpoint{1.759690in}{1.651277in}}%
\pgfpathlineto{\pgfqpoint{1.749546in}{1.663323in}}%
\pgfpathlineto{\pgfqpoint{1.748939in}{1.664030in}}%
\pgfpathlineto{\pgfqpoint{1.739017in}{1.675370in}}%
\pgfpathlineto{\pgfqpoint{1.736892in}{1.677749in}}%
\pgfpathlineto{\pgfqpoint{1.728104in}{1.687416in}}%
\pgfpathlineto{\pgfqpoint{1.724846in}{1.690929in}}%
\pgfpathlineto{\pgfqpoint{1.716788in}{1.699462in}}%
\pgfpathlineto{\pgfqpoint{1.712800in}{1.703603in}}%
\pgfpathlineto{\pgfqpoint{1.705044in}{1.711508in}}%
\pgfpathlineto{\pgfqpoint{1.700754in}{1.715798in}}%
\pgfpathlineto{\pgfqpoint{1.692849in}{1.723554in}}%
\pgfpathlineto{\pgfqpoint{1.688708in}{1.727542in}}%
\pgfpathlineto{\pgfqpoint{1.680175in}{1.735600in}}%
\pgfpathlineto{\pgfqpoint{1.676662in}{1.738858in}}%
\pgfpathlineto{\pgfqpoint{1.666995in}{1.747646in}}%
\pgfpathlineto{\pgfqpoint{1.664616in}{1.749770in}}%
\pgfpathlineto{\pgfqpoint{1.653276in}{1.759692in}}%
\pgfpathlineto{\pgfqpoint{1.652570in}{1.760300in}}%
\pgfpathlineto{\pgfqpoint{1.640524in}{1.770444in}}%
\pgfpathlineto{\pgfqpoint{1.638954in}{1.771738in}}%
\pgfpathlineto{\pgfqpoint{1.628477in}{1.780228in}}%
\pgfpathlineto{\pgfqpoint{1.623993in}{1.783785in}}%
\pgfpathlineto{\pgfqpoint{1.616431in}{1.789680in}}%
\pgfpathlineto{\pgfqpoint{1.608367in}{1.795831in}}%
\pgfpathlineto{\pgfqpoint{1.604385in}{1.798817in}}%
\pgfpathlineto{\pgfqpoint{1.592339in}{1.807651in}}%
\pgfpathlineto{\pgfqpoint{1.592024in}{1.807877in}}%
\pgfpathlineto{\pgfqpoint{1.580293in}{1.816146in}}%
\pgfpathlineto{\pgfqpoint{1.574808in}{1.819923in}}%
\pgfpathlineto{\pgfqpoint{1.568247in}{1.824367in}}%
\pgfpathlineto{\pgfqpoint{1.556752in}{1.831969in}}%
\pgfpathlineto{\pgfqpoint{1.556201in}{1.832328in}}%
\pgfpathlineto{\pgfqpoint{1.544155in}{1.839976in}}%
\pgfpathlineto{\pgfqpoint{1.537630in}{1.844015in}}%
\pgfpathlineto{\pgfqpoint{1.532108in}{1.847380in}}%
\pgfpathlineto{\pgfqpoint{1.520062in}{1.854532in}}%
\pgfpathlineto{\pgfqpoint{1.517417in}{1.856061in}}%
\pgfpathlineto{\pgfqpoint{1.508016in}{1.861413in}}%
\pgfpathlineto{\pgfqpoint{1.495970in}{1.868084in}}%
\pgfpathlineto{\pgfqpoint{1.495927in}{1.868107in}}%
\pgfpathlineto{\pgfqpoint{1.483924in}{1.874473in}}%
\pgfpathlineto{\pgfqpoint{1.472897in}{1.880154in}}%
\pgfpathlineto{\pgfqpoint{1.471878in}{1.880670in}}%
\pgfpathlineto{\pgfqpoint{1.459832in}{1.886604in}}%
\pgfpathlineto{\pgfqpoint{1.448115in}{1.892200in}}%
\pgfpathlineto{\pgfqpoint{1.447786in}{1.892355in}}%
\pgfpathlineto{\pgfqpoint{1.435740in}{1.897846in}}%
\pgfpathlineto{\pgfqpoint{1.423693in}{1.903161in}}%
\pgfpathlineto{\pgfqpoint{1.421149in}{1.904246in}}%
\pgfpathlineto{\pgfqpoint{1.411647in}{1.908239in}}%
\pgfpathlineto{\pgfqpoint{1.399601in}{1.913128in}}%
\pgfpathlineto{\pgfqpoint{1.391515in}{1.916292in}}%
\pgfpathlineto{\pgfqpoint{1.387555in}{1.917819in}}%
\pgfpathlineto{\pgfqpoint{1.375509in}{1.922293in}}%
\pgfpathlineto{\pgfqpoint{1.363463in}{1.926596in}}%
\pgfpathlineto{\pgfqpoint{1.358381in}{1.928338in}}%
\pgfpathlineto{\pgfqpoint{1.351417in}{1.930692in}}%
\pgfpathlineto{\pgfqpoint{1.339371in}{1.934594in}}%
\pgfpathlineto{\pgfqpoint{1.327325in}{1.938327in}}%
\pgfpathlineto{\pgfqpoint{1.320369in}{1.940384in}}%
\pgfpathlineto{\pgfqpoint{1.315278in}{1.941869in}}%
\pgfpathlineto{\pgfqpoint{1.303232in}{1.945215in}}%
\pgfpathlineto{\pgfqpoint{1.291186in}{1.948394in}}%
\pgfpathlineto{\pgfqpoint{1.279140in}{1.951405in}}%
\pgfpathlineto{\pgfqpoint{1.274800in}{1.952430in}}%
\pgfpathlineto{\pgfqpoint{1.267094in}{1.954225in}}%
\pgfpathlineto{\pgfqpoint{1.255048in}{1.956865in}}%
\pgfpathlineto{\pgfqpoint{1.243002in}{1.959340in}}%
\pgfpathlineto{\pgfqpoint{1.230956in}{1.961650in}}%
\pgfpathlineto{\pgfqpoint{1.218909in}{1.963796in}}%
\pgfpathlineto{\pgfqpoint{1.214769in}{1.964476in}}%
\pgfpathlineto{\pgfqpoint{1.206863in}{1.965758in}}%
\pgfpathlineto{\pgfqpoint{1.194817in}{1.967549in}}%
\pgfpathlineto{\pgfqpoint{1.182771in}{1.969177in}}%
\pgfpathlineto{\pgfqpoint{1.170725in}{1.970642in}}%
\pgfpathlineto{\pgfqpoint{1.158679in}{1.971944in}}%
\pgfpathlineto{\pgfqpoint{1.146633in}{1.973084in}}%
\pgfpathlineto{\pgfqpoint{1.134587in}{1.974060in}}%
\pgfpathlineto{\pgfqpoint{1.122541in}{1.974874in}}%
\pgfpathlineto{\pgfqpoint{1.110494in}{1.975525in}}%
\pgfpathlineto{\pgfqpoint{1.098448in}{1.976014in}}%
\pgfpathlineto{\pgfqpoint{1.086402in}{1.976339in}}%
\pgfpathlineto{\pgfqpoint{1.074356in}{1.976502in}}%
\pgfpathlineto{\pgfqpoint{1.062310in}{1.976502in}}%
\pgfpathlineto{\pgfqpoint{1.050264in}{1.976339in}}%
\pgfpathlineto{\pgfqpoint{1.038218in}{1.976014in}}%
\pgfpathlineto{\pgfqpoint{1.026172in}{1.975525in}}%
\pgfpathlineto{\pgfqpoint{1.014126in}{1.974874in}}%
\pgfpathlineto{\pgfqpoint{1.002079in}{1.974060in}}%
\pgfpathlineto{\pgfqpoint{0.990033in}{1.973084in}}%
\pgfpathlineto{\pgfqpoint{0.977987in}{1.971944in}}%
\pgfpathlineto{\pgfqpoint{0.965941in}{1.970642in}}%
\pgfpathlineto{\pgfqpoint{0.953895in}{1.969177in}}%
\pgfpathlineto{\pgfqpoint{0.941849in}{1.967549in}}%
\pgfpathlineto{\pgfqpoint{0.929803in}{1.965758in}}%
\pgfpathlineto{\pgfqpoint{0.921897in}{1.964476in}}%
\pgfpathlineto{\pgfqpoint{0.917757in}{1.963796in}}%
\pgfpathlineto{\pgfqpoint{0.905710in}{1.961650in}}%
\pgfpathlineto{\pgfqpoint{0.893664in}{1.959340in}}%
\pgfpathlineto{\pgfqpoint{0.881618in}{1.956865in}}%
\pgfpathlineto{\pgfqpoint{0.869572in}{1.954225in}}%
\pgfpathlineto{\pgfqpoint{0.861866in}{1.952430in}}%
\pgfpathlineto{\pgfqpoint{0.857526in}{1.951405in}}%
\pgfpathlineto{\pgfqpoint{0.845480in}{1.948394in}}%
\pgfpathlineto{\pgfqpoint{0.833434in}{1.945215in}}%
\pgfpathlineto{\pgfqpoint{0.821388in}{1.941869in}}%
\pgfpathlineto{\pgfqpoint{0.816297in}{1.940384in}}%
\pgfpathlineto{\pgfqpoint{0.809342in}{1.938327in}}%
\pgfpathlineto{\pgfqpoint{0.797295in}{1.934594in}}%
\pgfpathlineto{\pgfqpoint{0.785249in}{1.930692in}}%
\pgfpathlineto{\pgfqpoint{0.778285in}{1.928338in}}%
\pgfpathlineto{\pgfqpoint{0.773203in}{1.926596in}}%
\pgfpathlineto{\pgfqpoint{0.761157in}{1.922293in}}%
\pgfpathlineto{\pgfqpoint{0.749111in}{1.917819in}}%
\pgfpathlineto{\pgfqpoint{0.745151in}{1.916292in}}%
\pgfpathlineto{\pgfqpoint{0.737065in}{1.913128in}}%
\pgfpathlineto{\pgfqpoint{0.725019in}{1.908239in}}%
\pgfpathlineto{\pgfqpoint{0.715517in}{1.904246in}}%
\pgfpathlineto{\pgfqpoint{0.712973in}{1.903161in}}%
\pgfpathlineto{\pgfqpoint{0.700927in}{1.897846in}}%
\pgfpathlineto{\pgfqpoint{0.688880in}{1.892355in}}%
\pgfpathlineto{\pgfqpoint{0.688551in}{1.892200in}}%
\pgfpathlineto{\pgfqpoint{0.676834in}{1.886604in}}%
\pgfpathlineto{\pgfqpoint{0.664788in}{1.880670in}}%
\pgfpathlineto{\pgfqpoint{0.663770in}{1.880154in}}%
\pgfpathlineto{\pgfqpoint{0.652742in}{1.874473in}}%
\pgfpathlineto{\pgfqpoint{0.640739in}{1.868107in}}%
\pgfpathlineto{\pgfqpoint{0.640696in}{1.868084in}}%
\pgfpathlineto{\pgfqpoint{0.628650in}{1.861413in}}%
\pgfpathlineto{\pgfqpoint{0.619249in}{1.856061in}}%
\pgfpathlineto{\pgfqpoint{0.616604in}{1.854532in}}%
\pgfpathlineto{\pgfqpoint{0.604558in}{1.847380in}}%
\pgfpathlineto{\pgfqpoint{0.599036in}{1.844015in}}%
\pgfpathlineto{\pgfqpoint{0.592511in}{1.839976in}}%
\pgfpathlineto{\pgfqpoint{0.580465in}{1.832328in}}%
\pgfpathlineto{\pgfqpoint{0.579914in}{1.831969in}}%
\pgfpathlineto{\pgfqpoint{0.568419in}{1.824367in}}%
\pgfpathlineto{\pgfqpoint{0.561858in}{1.819923in}}%
\pgfpathlineto{\pgfqpoint{0.556373in}{1.816146in}}%
\pgfpathlineto{\pgfqpoint{0.544642in}{1.807877in}}%
\pgfpathlineto{\pgfqpoint{0.544327in}{1.807651in}}%
\pgfpathlineto{\pgfqpoint{0.532281in}{1.798817in}}%
\pgfpathlineto{\pgfqpoint{0.528299in}{1.795831in}}%
\pgfpathlineto{\pgfqpoint{0.520235in}{1.789680in}}%
\pgfpathlineto{\pgfqpoint{0.512673in}{1.783785in}}%
\pgfpathlineto{\pgfqpoint{0.508189in}{1.780228in}}%
\pgfpathlineto{\pgfqpoint{0.497712in}{1.771738in}}%
\pgfpathlineto{\pgfqpoint{0.496143in}{1.770444in}}%
\pgfpathlineto{\pgfqpoint{0.484096in}{1.760300in}}%
\pgfpathlineto{\pgfqpoint{0.483390in}{1.759692in}}%
\pgfpathlineto{\pgfqpoint{0.472050in}{1.749770in}}%
\pgfpathlineto{\pgfqpoint{0.469671in}{1.747646in}}%
\pgfpathlineto{\pgfqpoint{0.460004in}{1.738858in}}%
\pgfpathlineto{\pgfqpoint{0.456491in}{1.735600in}}%
\pgfpathlineto{\pgfqpoint{0.447958in}{1.727542in}}%
\pgfpathlineto{\pgfqpoint{0.443817in}{1.723554in}}%
\pgfpathlineto{\pgfqpoint{0.435912in}{1.715798in}}%
\pgfpathlineto{\pgfqpoint{0.431622in}{1.711508in}}%
\pgfpathlineto{\pgfqpoint{0.423866in}{1.703603in}}%
\pgfpathlineto{\pgfqpoint{0.419878in}{1.699462in}}%
\pgfpathlineto{\pgfqpoint{0.411820in}{1.690929in}}%
\pgfpathlineto{\pgfqpoint{0.408562in}{1.687416in}}%
\pgfpathlineto{\pgfqpoint{0.399774in}{1.677749in}}%
\pgfpathlineto{\pgfqpoint{0.397649in}{1.675370in}}%
\pgfpathlineto{\pgfqpoint{0.387728in}{1.664030in}}%
\pgfpathlineto{\pgfqpoint{0.387120in}{1.663323in}}%
\pgfpathlineto{\pgfqpoint{0.376976in}{1.651277in}}%
\pgfpathlineto{\pgfqpoint{0.375681in}{1.649708in}}%
\pgfpathlineto{\pgfqpoint{0.367192in}{1.639231in}}%
\pgfpathlineto{\pgfqpoint{0.363635in}{1.634747in}}%
\pgfpathlineto{\pgfqpoint{0.357740in}{1.627185in}}%
\pgfpathlineto{\pgfqpoint{0.351589in}{1.619121in}}%
\pgfpathlineto{\pgfqpoint{0.348603in}{1.615139in}}%
\pgfpathlineto{\pgfqpoint{0.339769in}{1.603093in}}%
\pgfpathlineto{\pgfqpoint{0.339543in}{1.602778in}}%
\pgfpathlineto{\pgfqpoint{0.331274in}{1.591047in}}%
\pgfpathlineto{\pgfqpoint{0.327497in}{1.585561in}}%
\pgfpathlineto{\pgfqpoint{0.323052in}{1.579001in}}%
\pgfpathlineto{\pgfqpoint{0.315451in}{1.567505in}}%
\pgfpathlineto{\pgfqpoint{0.315092in}{1.566955in}}%
\pgfpathlineto{\pgfqpoint{0.307444in}{1.554908in}}%
\pgfpathlineto{\pgfqpoint{0.303405in}{1.548383in}}%
\pgfpathlineto{\pgfqpoint{0.300040in}{1.542862in}}%
\pgfpathlineto{\pgfqpoint{0.292888in}{1.530816in}}%
\pgfpathlineto{\pgfqpoint{0.291359in}{1.528171in}}%
\pgfpathlineto{\pgfqpoint{0.286007in}{1.518770in}}%
\pgfpathlineto{\pgfqpoint{0.279336in}{1.506724in}}%
\pgfpathlineto{\pgfqpoint{0.279312in}{1.506681in}}%
\pgfpathlineto{\pgfqpoint{0.272947in}{1.494678in}}%
\pgfpathlineto{\pgfqpoint{0.267266in}{1.483650in}}%
\pgfpathlineto{\pgfqpoint{0.266749in}{1.482632in}}%
\pgfpathlineto{\pgfqpoint{0.260816in}{1.470586in}}%
\pgfpathlineto{\pgfqpoint{0.255220in}{1.458869in}}%
\pgfpathlineto{\pgfqpoint{0.255065in}{1.458539in}}%
\pgfpathlineto{\pgfqpoint{0.249574in}{1.446493in}}%
\pgfpathlineto{\pgfqpoint{0.244259in}{1.434447in}}%
\pgfpathlineto{\pgfqpoint{0.243174in}{1.431903in}}%
\pgfpathlineto{\pgfqpoint{0.239181in}{1.422401in}}%
\pgfpathlineto{\pgfqpoint{0.234292in}{1.410355in}}%
\pgfpathlineto{\pgfqpoint{0.231128in}{1.402269in}}%
\pgfpathlineto{\pgfqpoint{0.229601in}{1.398309in}}%
\pgfpathlineto{\pgfqpoint{0.225126in}{1.386263in}}%
\pgfpathlineto{\pgfqpoint{0.220824in}{1.374217in}}%
\pgfpathlineto{\pgfqpoint{0.219082in}{1.369135in}}%
\pgfpathlineto{\pgfqpoint{0.216728in}{1.362171in}}%
\pgfpathlineto{\pgfqpoint{0.212826in}{1.350124in}}%
\pgfpathlineto{\pgfqpoint{0.209093in}{1.338078in}}%
\pgfpathlineto{\pgfqpoint{0.207036in}{1.331123in}}%
\pgfpathlineto{\pgfqpoint{0.205551in}{1.326032in}}%
\pgfpathlineto{\pgfqpoint{0.202205in}{1.313986in}}%
\pgfpathlineto{\pgfqpoint{0.199026in}{1.301940in}}%
\pgfpathlineto{\pgfqpoint{0.196014in}{1.289894in}}%
\pgfpathlineto{\pgfqpoint{0.194990in}{1.285554in}}%
\pgfpathlineto{\pgfqpoint{0.193195in}{1.277848in}}%
\pgfpathlineto{\pgfqpoint{0.190555in}{1.265802in}}%
\pgfpathlineto{\pgfqpoint{0.188080in}{1.253756in}}%
\pgfpathlineto{\pgfqpoint{0.185769in}{1.241709in}}%
\pgfpathlineto{\pgfqpoint{0.183624in}{1.229663in}}%
\pgfpathlineto{\pgfqpoint{0.182944in}{1.225522in}}%
\pgfpathlineto{\pgfqpoint{0.181662in}{1.217617in}}%
\pgfpathlineto{\pgfqpoint{0.179871in}{1.205571in}}%
\pgfpathlineto{\pgfqpoint{0.178243in}{1.193525in}}%
\pgfpathlineto{\pgfqpoint{0.176778in}{1.181479in}}%
\pgfpathlineto{\pgfqpoint{0.175476in}{1.169433in}}%
\pgfpathlineto{\pgfqpoint{0.174336in}{1.157387in}}%
\pgfpathlineto{\pgfqpoint{0.173360in}{1.145340in}}%
\pgfpathlineto{\pgfqpoint{0.172546in}{1.133294in}}%
\pgfpathlineto{\pgfqpoint{0.171894in}{1.121248in}}%
\pgfpathlineto{\pgfqpoint{0.171406in}{1.109202in}}%
\pgfpathlineto{\pgfqpoint{0.171081in}{1.097156in}}%
\pgfpathlineto{\pgfqpoint{0.170918in}{1.085110in}}%
\pgfpathlineto{\pgfqpoint{0.170918in}{1.073064in}}%
\pgfpathlineto{\pgfqpoint{0.171081in}{1.061018in}}%
\pgfpathlineto{\pgfqpoint{0.171406in}{1.048972in}}%
\pgfpathlineto{\pgfqpoint{0.171894in}{1.036925in}}%
\pgfpathlineto{\pgfqpoint{0.172546in}{1.024879in}}%
\pgfpathlineto{\pgfqpoint{0.173360in}{1.012833in}}%
\pgfpathlineto{\pgfqpoint{0.174336in}{1.000787in}}%
\pgfpathlineto{\pgfqpoint{0.175476in}{0.988741in}}%
\pgfpathlineto{\pgfqpoint{0.176778in}{0.976695in}}%
\pgfpathlineto{\pgfqpoint{0.178243in}{0.964649in}}%
\pgfpathlineto{\pgfqpoint{0.179871in}{0.952603in}}%
\pgfpathlineto{\pgfqpoint{0.181662in}{0.940557in}}%
\pgfpathlineto{\pgfqpoint{0.182944in}{0.932651in}}%
\pgfpathlineto{\pgfqpoint{0.183624in}{0.928510in}}%
\pgfpathlineto{\pgfqpoint{0.185769in}{0.916464in}}%
\pgfpathlineto{\pgfqpoint{0.188080in}{0.904418in}}%
\pgfpathlineto{\pgfqpoint{0.190555in}{0.892372in}}%
\pgfpathlineto{\pgfqpoint{0.193195in}{0.880326in}}%
\pgfpathlineto{\pgfqpoint{0.194990in}{0.872620in}}%
\pgfpathlineto{\pgfqpoint{0.196014in}{0.868280in}}%
\pgfpathlineto{\pgfqpoint{0.199026in}{0.856234in}}%
\pgfpathlineto{\pgfqpoint{0.202205in}{0.844188in}}%
\pgfpathlineto{\pgfqpoint{0.205551in}{0.832141in}}%
\pgfpathlineto{\pgfqpoint{0.207036in}{0.827051in}}%
\pgfpathlineto{\pgfqpoint{0.209093in}{0.820095in}}%
\pgfpathlineto{\pgfqpoint{0.212826in}{0.808049in}}%
\pgfpathlineto{\pgfqpoint{0.216728in}{0.796003in}}%
\pgfpathlineto{\pgfqpoint{0.219082in}{0.789039in}}%
\pgfpathlineto{\pgfqpoint{0.220824in}{0.783957in}}%
\pgfpathlineto{\pgfqpoint{0.225126in}{0.771911in}}%
\pgfpathlineto{\pgfqpoint{0.229601in}{0.759865in}}%
\pgfpathlineto{\pgfqpoint{0.231128in}{0.755905in}}%
\pgfpathlineto{\pgfqpoint{0.234292in}{0.747819in}}%
\pgfpathlineto{\pgfqpoint{0.239181in}{0.735773in}}%
\pgfpathlineto{\pgfqpoint{0.243174in}{0.726271in}}%
\pgfpathlineto{\pgfqpoint{0.244259in}{0.723726in}}%
\pgfpathlineto{\pgfqpoint{0.249574in}{0.711680in}}%
\pgfpathlineto{\pgfqpoint{0.255065in}{0.699634in}}%
\pgfpathlineto{\pgfqpoint{0.255220in}{0.699305in}}%
\pgfpathlineto{\pgfqpoint{0.260816in}{0.687588in}}%
\pgfpathlineto{\pgfqpoint{0.266749in}{0.675542in}}%
\pgfpathlineto{\pgfqpoint{0.267266in}{0.674523in}}%
\pgfpathlineto{\pgfqpoint{0.272947in}{0.663496in}}%
\pgfpathlineto{\pgfqpoint{0.279312in}{0.651493in}}%
\pgfpathlineto{\pgfqpoint{0.279336in}{0.651450in}}%
\pgfpathlineto{\pgfqpoint{0.286007in}{0.639404in}}%
\pgfpathlineto{\pgfqpoint{0.291359in}{0.630003in}}%
\pgfpathlineto{\pgfqpoint{0.292888in}{0.627358in}}%
\pgfpathlineto{\pgfqpoint{0.300040in}{0.615311in}}%
\pgfpathlineto{\pgfqpoint{0.303405in}{0.609790in}}%
\pgfpathlineto{\pgfqpoint{0.307444in}{0.603265in}}%
\pgfpathlineto{\pgfqpoint{0.315092in}{0.591219in}}%
\pgfpathlineto{\pgfqpoint{0.315451in}{0.590668in}}%
\pgfpathlineto{\pgfqpoint{0.323052in}{0.579173in}}%
\pgfpathlineto{\pgfqpoint{0.327497in}{0.572612in}}%
\pgfpathlineto{\pgfqpoint{0.331274in}{0.567127in}}%
\pgfpathlineto{\pgfqpoint{0.339543in}{0.555396in}}%
\pgfpathlineto{\pgfqpoint{0.339769in}{0.555081in}}%
\pgfpathlineto{\pgfqpoint{0.348603in}{0.543035in}}%
\pgfpathlineto{\pgfqpoint{0.351589in}{0.539053in}}%
\pgfpathlineto{\pgfqpoint{0.357740in}{0.530989in}}%
\pgfpathlineto{\pgfqpoint{0.363635in}{0.523427in}}%
\pgfpathlineto{\pgfqpoint{0.367192in}{0.518942in}}%
\pgfpathlineto{\pgfqpoint{0.375681in}{0.508466in}}%
\pgfpathlineto{\pgfqpoint{0.376976in}{0.506896in}}%
\pgfpathlineto{\pgfqpoint{0.387120in}{0.494850in}}%
\pgfpathlineto{\pgfqpoint{0.387728in}{0.494143in}}%
\pgfpathlineto{\pgfqpoint{0.397649in}{0.482804in}}%
\pgfpathlineto{\pgfqpoint{0.399774in}{0.480425in}}%
\pgfpathlineto{\pgfqpoint{0.408562in}{0.470758in}}%
\pgfpathlineto{\pgfqpoint{0.411820in}{0.467245in}}%
\pgfpathlineto{\pgfqpoint{0.419878in}{0.458712in}}%
\pgfpathlineto{\pgfqpoint{0.423866in}{0.454571in}}%
\pgfpathlineto{\pgfqpoint{0.431622in}{0.446666in}}%
\pgfpathlineto{\pgfqpoint{0.435912in}{0.442376in}}%
\pgfpathlineto{\pgfqpoint{0.443817in}{0.434620in}}%
\pgfpathlineto{\pgfqpoint{0.447958in}{0.430632in}}%
\pgfpathlineto{\pgfqpoint{0.456491in}{0.422574in}}%
\pgfpathlineto{\pgfqpoint{0.460004in}{0.419316in}}%
\pgfpathlineto{\pgfqpoint{0.469671in}{0.410527in}}%
\pgfpathlineto{\pgfqpoint{0.472050in}{0.408403in}}%
\pgfpathlineto{\pgfqpoint{0.483390in}{0.398481in}}%
\pgfpathlineto{\pgfqpoint{0.484096in}{0.397874in}}%
\pgfpathlineto{\pgfqpoint{0.496143in}{0.387730in}}%
\pgfpathlineto{\pgfqpoint{0.497712in}{0.386435in}}%
\pgfpathlineto{\pgfqpoint{0.508189in}{0.377946in}}%
\pgfpathlineto{\pgfqpoint{0.512673in}{0.374389in}}%
\pgfpathlineto{\pgfqpoint{0.520235in}{0.368494in}}%
\pgfpathlineto{\pgfqpoint{0.528299in}{0.362343in}}%
\pgfpathlineto{\pgfqpoint{0.532281in}{0.359357in}}%
\pgfpathlineto{\pgfqpoint{0.544327in}{0.350523in}}%
\pgfpathlineto{\pgfqpoint{0.544642in}{0.350297in}}%
\pgfpathlineto{\pgfqpoint{0.556373in}{0.342028in}}%
\pgfpathlineto{\pgfqpoint{0.561858in}{0.338251in}}%
\pgfpathlineto{\pgfqpoint{0.568419in}{0.333806in}}%
\pgfpathlineto{\pgfqpoint{0.579914in}{0.326205in}}%
\pgfpathlineto{\pgfqpoint{0.580465in}{0.325846in}}%
\pgfpathlineto{\pgfqpoint{0.592511in}{0.318198in}}%
\pgfpathlineto{\pgfqpoint{0.599036in}{0.314159in}}%
\pgfpathlineto{\pgfqpoint{0.604558in}{0.310794in}}%
\pgfpathlineto{\pgfqpoint{0.616604in}{0.303642in}}%
\pgfpathlineto{\pgfqpoint{0.619249in}{0.302112in}}%
\pgfpathlineto{\pgfqpoint{0.628650in}{0.296761in}}%
\pgfpathlineto{\pgfqpoint{0.640696in}{0.290089in}}%
\pgfpathlineto{\pgfqpoint{0.640739in}{0.290066in}}%
\pgfpathlineto{\pgfqpoint{0.652742in}{0.283701in}}%
\pgfpathlineto{\pgfqpoint{0.663770in}{0.278020in}}%
\pgfpathlineto{\pgfqpoint{0.664788in}{0.277503in}}%
\pgfpathlineto{\pgfqpoint{0.676834in}{0.271570in}}%
\pgfpathlineto{\pgfqpoint{0.688551in}{0.265974in}}%
\pgfpathlineto{\pgfqpoint{0.688880in}{0.265819in}}%
\pgfpathlineto{\pgfqpoint{0.700927in}{0.260327in}}%
\pgfpathlineto{\pgfqpoint{0.712973in}{0.255013in}}%
\pgfpathlineto{\pgfqpoint{0.715517in}{0.253928in}}%
\pgfpathlineto{\pgfqpoint{0.725019in}{0.249934in}}%
\pgfpathlineto{\pgfqpoint{0.737065in}{0.245046in}}%
\pgfpathlineto{\pgfqpoint{0.745151in}{0.241882in}}%
\pgfpathlineto{\pgfqpoint{0.749111in}{0.240355in}}%
\pgfpathlineto{\pgfqpoint{0.761157in}{0.235880in}}%
\pgfpathlineto{\pgfqpoint{0.773203in}{0.231578in}}%
\pgfpathlineto{\pgfqpoint{0.778285in}{0.229836in}}%
\pgfpathlineto{\pgfqpoint{0.785249in}{0.227482in}}%
\pgfpathlineto{\pgfqpoint{0.797295in}{0.223579in}}%
\pgfpathlineto{\pgfqpoint{0.809342in}{0.219847in}}%
\pgfpathlineto{\pgfqpoint{0.816297in}{0.217790in}}%
\pgfpathlineto{\pgfqpoint{0.821388in}{0.216305in}}%
\pgfpathlineto{\pgfqpoint{0.833434in}{0.212959in}}%
\pgfpathlineto{\pgfqpoint{0.845480in}{0.209780in}}%
\pgfpathlineto{\pgfqpoint{0.857526in}{0.206768in}}%
\pgfpathlineto{\pgfqpoint{0.861866in}{0.205743in}}%
\pgfpathlineto{\pgfqpoint{0.869572in}{0.203949in}}%
\pgfpathlineto{\pgfqpoint{0.881618in}{0.201309in}}%
\pgfpathlineto{\pgfqpoint{0.893664in}{0.198833in}}%
\pgfpathlineto{\pgfqpoint{0.905710in}{0.196523in}}%
\pgfpathlineto{\pgfqpoint{0.917757in}{0.194378in}}%
\pgfpathlineto{\pgfqpoint{0.921897in}{0.193697in}}%
\pgfpathclose%
\pgfusepath{fill}%
\end{pgfscope}%
\begin{pgfscope}%
\pgfpathrectangle{\pgfqpoint{0.135000in}{0.145754in}}{\pgfqpoint{1.866666in}{1.866666in}} %
\pgfusepath{clip}%
\pgfsetbuttcap%
\pgfsetroundjoin%
\definecolor{currentfill}{rgb}{0.000000,0.000000,1.000000}%
\pgfsetfillcolor{currentfill}%
\pgfsetlinewidth{0.000000pt}%
\definecolor{currentstroke}{rgb}{0.000000,0.000000,0.000000}%
\pgfsetstrokecolor{currentstroke}%
\pgfsetdash{}{0pt}%
\pgfpathmoveto{\pgfqpoint{0.929803in}{0.192415in}}%
\pgfpathlineto{\pgfqpoint{0.941849in}{0.190625in}}%
\pgfpathlineto{\pgfqpoint{0.953895in}{0.188997in}}%
\pgfpathlineto{\pgfqpoint{0.965941in}{0.187532in}}%
\pgfpathlineto{\pgfqpoint{0.977987in}{0.186230in}}%
\pgfpathlineto{\pgfqpoint{0.990033in}{0.185090in}}%
\pgfpathlineto{\pgfqpoint{1.002079in}{0.184113in}}%
\pgfpathlineto{\pgfqpoint{1.014126in}{0.183299in}}%
\pgfpathlineto{\pgfqpoint{1.026172in}{0.182648in}}%
\pgfpathlineto{\pgfqpoint{1.038218in}{0.182160in}}%
\pgfpathlineto{\pgfqpoint{1.050264in}{0.181834in}}%
\pgfpathlineto{\pgfqpoint{1.062310in}{0.181672in}}%
\pgfpathlineto{\pgfqpoint{1.074356in}{0.181672in}}%
\pgfpathlineto{\pgfqpoint{1.086402in}{0.181834in}}%
\pgfpathlineto{\pgfqpoint{1.098448in}{0.182160in}}%
\pgfpathlineto{\pgfqpoint{1.110494in}{0.182648in}}%
\pgfpathlineto{\pgfqpoint{1.122541in}{0.183299in}}%
\pgfpathlineto{\pgfqpoint{1.134587in}{0.184113in}}%
\pgfpathlineto{\pgfqpoint{1.146633in}{0.185090in}}%
\pgfpathlineto{\pgfqpoint{1.158679in}{0.186230in}}%
\pgfpathlineto{\pgfqpoint{1.170725in}{0.187532in}}%
\pgfpathlineto{\pgfqpoint{1.182771in}{0.188997in}}%
\pgfpathlineto{\pgfqpoint{1.194817in}{0.190625in}}%
\pgfpathlineto{\pgfqpoint{1.206863in}{0.192415in}}%
\pgfpathlineto{\pgfqpoint{1.214769in}{0.193697in}}%
\pgfpathlineto{\pgfqpoint{1.218909in}{0.194378in}}%
\pgfpathlineto{\pgfqpoint{1.230956in}{0.196523in}}%
\pgfpathlineto{\pgfqpoint{1.243002in}{0.198833in}}%
\pgfpathlineto{\pgfqpoint{1.255048in}{0.201309in}}%
\pgfpathlineto{\pgfqpoint{1.267094in}{0.203949in}}%
\pgfpathlineto{\pgfqpoint{1.274800in}{0.205743in}}%
\pgfpathlineto{\pgfqpoint{1.279140in}{0.206768in}}%
\pgfpathlineto{\pgfqpoint{1.291186in}{0.209780in}}%
\pgfpathlineto{\pgfqpoint{1.303232in}{0.212959in}}%
\pgfpathlineto{\pgfqpoint{1.315278in}{0.216305in}}%
\pgfpathlineto{\pgfqpoint{1.320369in}{0.217790in}}%
\pgfpathlineto{\pgfqpoint{1.327325in}{0.219847in}}%
\pgfpathlineto{\pgfqpoint{1.339371in}{0.223579in}}%
\pgfpathlineto{\pgfqpoint{1.351417in}{0.227482in}}%
\pgfpathlineto{\pgfqpoint{1.358381in}{0.229836in}}%
\pgfpathlineto{\pgfqpoint{1.363463in}{0.231578in}}%
\pgfpathlineto{\pgfqpoint{1.375509in}{0.235880in}}%
\pgfpathlineto{\pgfqpoint{1.387555in}{0.240355in}}%
\pgfpathlineto{\pgfqpoint{1.391515in}{0.241882in}}%
\pgfpathlineto{\pgfqpoint{1.399601in}{0.245046in}}%
\pgfpathlineto{\pgfqpoint{1.411647in}{0.249934in}}%
\pgfpathlineto{\pgfqpoint{1.421149in}{0.253928in}}%
\pgfpathlineto{\pgfqpoint{1.423693in}{0.255013in}}%
\pgfpathlineto{\pgfqpoint{1.435740in}{0.260327in}}%
\pgfpathlineto{\pgfqpoint{1.447786in}{0.265819in}}%
\pgfpathlineto{\pgfqpoint{1.448115in}{0.265974in}}%
\pgfpathlineto{\pgfqpoint{1.459832in}{0.271570in}}%
\pgfpathlineto{\pgfqpoint{1.471878in}{0.277503in}}%
\pgfpathlineto{\pgfqpoint{1.472897in}{0.278020in}}%
\pgfpathlineto{\pgfqpoint{1.483924in}{0.283701in}}%
\pgfpathlineto{\pgfqpoint{1.495927in}{0.290066in}}%
\pgfpathlineto{\pgfqpoint{1.495970in}{0.290089in}}%
\pgfpathlineto{\pgfqpoint{1.508016in}{0.296761in}}%
\pgfpathlineto{\pgfqpoint{1.517417in}{0.302112in}}%
\pgfpathlineto{\pgfqpoint{1.520062in}{0.303642in}}%
\pgfpathlineto{\pgfqpoint{1.532108in}{0.310794in}}%
\pgfpathlineto{\pgfqpoint{1.537630in}{0.314159in}}%
\pgfpathlineto{\pgfqpoint{1.544155in}{0.318198in}}%
\pgfpathlineto{\pgfqpoint{1.556201in}{0.325846in}}%
\pgfpathlineto{\pgfqpoint{1.556752in}{0.326205in}}%
\pgfpathlineto{\pgfqpoint{1.568247in}{0.333806in}}%
\pgfpathlineto{\pgfqpoint{1.574808in}{0.338251in}}%
\pgfpathlineto{\pgfqpoint{1.580293in}{0.342028in}}%
\pgfpathlineto{\pgfqpoint{1.592024in}{0.350297in}}%
\pgfpathlineto{\pgfqpoint{1.592339in}{0.350523in}}%
\pgfpathlineto{\pgfqpoint{1.604385in}{0.359357in}}%
\pgfpathlineto{\pgfqpoint{1.608367in}{0.362343in}}%
\pgfpathlineto{\pgfqpoint{1.616431in}{0.368494in}}%
\pgfpathlineto{\pgfqpoint{1.623993in}{0.374389in}}%
\pgfpathlineto{\pgfqpoint{1.628477in}{0.377946in}}%
\pgfpathlineto{\pgfqpoint{1.638954in}{0.386435in}}%
\pgfpathlineto{\pgfqpoint{1.640524in}{0.387730in}}%
\pgfpathlineto{\pgfqpoint{1.652570in}{0.397874in}}%
\pgfpathlineto{\pgfqpoint{1.653276in}{0.398481in}}%
\pgfpathlineto{\pgfqpoint{1.664616in}{0.408403in}}%
\pgfpathlineto{\pgfqpoint{1.666995in}{0.410527in}}%
\pgfpathlineto{\pgfqpoint{1.676662in}{0.419316in}}%
\pgfpathlineto{\pgfqpoint{1.680175in}{0.422574in}}%
\pgfpathlineto{\pgfqpoint{1.688708in}{0.430632in}}%
\pgfpathlineto{\pgfqpoint{1.692849in}{0.434620in}}%
\pgfpathlineto{\pgfqpoint{1.700754in}{0.442376in}}%
\pgfpathlineto{\pgfqpoint{1.705044in}{0.446666in}}%
\pgfpathlineto{\pgfqpoint{1.712800in}{0.454571in}}%
\pgfpathlineto{\pgfqpoint{1.716788in}{0.458712in}}%
\pgfpathlineto{\pgfqpoint{1.724846in}{0.467245in}}%
\pgfpathlineto{\pgfqpoint{1.728104in}{0.470758in}}%
\pgfpathlineto{\pgfqpoint{1.736892in}{0.480425in}}%
\pgfpathlineto{\pgfqpoint{1.739017in}{0.482804in}}%
\pgfpathlineto{\pgfqpoint{1.748939in}{0.494143in}}%
\pgfpathlineto{\pgfqpoint{1.749546in}{0.494850in}}%
\pgfpathlineto{\pgfqpoint{1.759690in}{0.506896in}}%
\pgfpathlineto{\pgfqpoint{1.760985in}{0.508466in}}%
\pgfpathlineto{\pgfqpoint{1.769474in}{0.518942in}}%
\pgfpathlineto{\pgfqpoint{1.773031in}{0.523427in}}%
\pgfpathlineto{\pgfqpoint{1.778926in}{0.530989in}}%
\pgfpathlineto{\pgfqpoint{1.785077in}{0.539053in}}%
\pgfpathlineto{\pgfqpoint{1.788063in}{0.543035in}}%
\pgfpathlineto{\pgfqpoint{1.796897in}{0.555081in}}%
\pgfpathlineto{\pgfqpoint{1.797123in}{0.555396in}}%
\pgfpathlineto{\pgfqpoint{1.805392in}{0.567127in}}%
\pgfpathlineto{\pgfqpoint{1.809169in}{0.572612in}}%
\pgfpathlineto{\pgfqpoint{1.813614in}{0.579173in}}%
\pgfpathlineto{\pgfqpoint{1.821215in}{0.590668in}}%
\pgfpathlineto{\pgfqpoint{1.821574in}{0.591219in}}%
\pgfpathlineto{\pgfqpoint{1.829222in}{0.603265in}}%
\pgfpathlineto{\pgfqpoint{1.833261in}{0.609790in}}%
\pgfpathlineto{\pgfqpoint{1.836626in}{0.615311in}}%
\pgfpathlineto{\pgfqpoint{1.843778in}{0.627358in}}%
\pgfpathlineto{\pgfqpoint{1.845307in}{0.630003in}}%
\pgfpathlineto{\pgfqpoint{1.850659in}{0.639404in}}%
\pgfpathlineto{\pgfqpoint{1.857330in}{0.651450in}}%
\pgfpathlineto{\pgfqpoint{1.857354in}{0.651493in}}%
\pgfpathlineto{\pgfqpoint{1.863719in}{0.663496in}}%
\pgfpathlineto{\pgfqpoint{1.869400in}{0.674523in}}%
\pgfpathlineto{\pgfqpoint{1.869917in}{0.675542in}}%
\pgfpathlineto{\pgfqpoint{1.875850in}{0.687588in}}%
\pgfpathlineto{\pgfqpoint{1.881446in}{0.699305in}}%
\pgfpathlineto{\pgfqpoint{1.881601in}{0.699634in}}%
\pgfpathlineto{\pgfqpoint{1.887092in}{0.711680in}}%
\pgfpathlineto{\pgfqpoint{1.892407in}{0.723726in}}%
\pgfpathlineto{\pgfqpoint{1.893492in}{0.726271in}}%
\pgfpathlineto{\pgfqpoint{1.897485in}{0.735773in}}%
\pgfpathlineto{\pgfqpoint{1.902374in}{0.747819in}}%
\pgfpathlineto{\pgfqpoint{1.905538in}{0.755905in}}%
\pgfpathlineto{\pgfqpoint{1.907065in}{0.759865in}}%
\pgfpathlineto{\pgfqpoint{1.911540in}{0.771911in}}%
\pgfpathlineto{\pgfqpoint{1.915842in}{0.783957in}}%
\pgfpathlineto{\pgfqpoint{1.917584in}{0.789039in}}%
\pgfpathlineto{\pgfqpoint{1.919938in}{0.796003in}}%
\pgfpathlineto{\pgfqpoint{1.923841in}{0.808049in}}%
\pgfpathlineto{\pgfqpoint{1.927573in}{0.820095in}}%
\pgfpathlineto{\pgfqpoint{1.929630in}{0.827051in}}%
\pgfpathlineto{\pgfqpoint{1.931115in}{0.832141in}}%
\pgfpathlineto{\pgfqpoint{1.934461in}{0.844188in}}%
\pgfpathlineto{\pgfqpoint{1.937640in}{0.856234in}}%
\pgfpathlineto{\pgfqpoint{1.940652in}{0.868280in}}%
\pgfpathlineto{\pgfqpoint{1.941676in}{0.872620in}}%
\pgfpathlineto{\pgfqpoint{1.943471in}{0.880326in}}%
\pgfpathlineto{\pgfqpoint{1.946111in}{0.892372in}}%
\pgfpathlineto{\pgfqpoint{1.948586in}{0.904418in}}%
\pgfpathlineto{\pgfqpoint{1.950897in}{0.916464in}}%
\pgfpathlineto{\pgfqpoint{1.953042in}{0.928510in}}%
\pgfpathlineto{\pgfqpoint{1.953723in}{0.932651in}}%
\pgfpathlineto{\pgfqpoint{1.955004in}{0.940557in}}%
\pgfpathlineto{\pgfqpoint{1.956795in}{0.952603in}}%
\pgfpathlineto{\pgfqpoint{1.958423in}{0.964649in}}%
\pgfpathlineto{\pgfqpoint{1.959888in}{0.976695in}}%
\pgfpathlineto{\pgfqpoint{1.961190in}{0.988741in}}%
\pgfpathlineto{\pgfqpoint{1.962330in}{1.000787in}}%
\pgfpathlineto{\pgfqpoint{1.963307in}{1.012833in}}%
\pgfpathlineto{\pgfqpoint{1.964120in}{1.024879in}}%
\pgfpathlineto{\pgfqpoint{1.964772in}{1.036925in}}%
\pgfpathlineto{\pgfqpoint{1.965260in}{1.048972in}}%
\pgfpathlineto{\pgfqpoint{1.965585in}{1.061018in}}%
\pgfpathlineto{\pgfqpoint{1.965748in}{1.073064in}}%
\pgfpathlineto{\pgfqpoint{1.965748in}{1.085110in}}%
\pgfpathlineto{\pgfqpoint{1.965585in}{1.097156in}}%
\pgfpathlineto{\pgfqpoint{1.965260in}{1.109202in}}%
\pgfpathlineto{\pgfqpoint{1.964772in}{1.121248in}}%
\pgfpathlineto{\pgfqpoint{1.964120in}{1.133294in}}%
\pgfpathlineto{\pgfqpoint{1.963307in}{1.145340in}}%
\pgfpathlineto{\pgfqpoint{1.962330in}{1.157387in}}%
\pgfpathlineto{\pgfqpoint{1.961190in}{1.169433in}}%
\pgfpathlineto{\pgfqpoint{1.959888in}{1.181479in}}%
\pgfpathlineto{\pgfqpoint{1.958423in}{1.193525in}}%
\pgfpathlineto{\pgfqpoint{1.956795in}{1.205571in}}%
\pgfpathlineto{\pgfqpoint{1.955004in}{1.217617in}}%
\pgfpathlineto{\pgfqpoint{1.953723in}{1.225522in}}%
\pgfpathlineto{\pgfqpoint{1.953042in}{1.229663in}}%
\pgfpathlineto{\pgfqpoint{1.950897in}{1.241709in}}%
\pgfpathlineto{\pgfqpoint{1.948586in}{1.253756in}}%
\pgfpathlineto{\pgfqpoint{1.946111in}{1.265802in}}%
\pgfpathlineto{\pgfqpoint{1.943471in}{1.277848in}}%
\pgfpathlineto{\pgfqpoint{1.941676in}{1.285554in}}%
\pgfpathlineto{\pgfqpoint{1.940652in}{1.289894in}}%
\pgfpathlineto{\pgfqpoint{1.937640in}{1.301940in}}%
\pgfpathlineto{\pgfqpoint{1.934461in}{1.313986in}}%
\pgfpathlineto{\pgfqpoint{1.931115in}{1.326032in}}%
\pgfpathlineto{\pgfqpoint{1.929630in}{1.331123in}}%
\pgfpathlineto{\pgfqpoint{1.927573in}{1.338078in}}%
\pgfpathlineto{\pgfqpoint{1.923841in}{1.350124in}}%
\pgfpathlineto{\pgfqpoint{1.919938in}{1.362171in}}%
\pgfpathlineto{\pgfqpoint{1.917584in}{1.369135in}}%
\pgfpathlineto{\pgfqpoint{1.915842in}{1.374217in}}%
\pgfpathlineto{\pgfqpoint{1.911540in}{1.386263in}}%
\pgfpathlineto{\pgfqpoint{1.907065in}{1.398309in}}%
\pgfpathlineto{\pgfqpoint{1.905538in}{1.402269in}}%
\pgfpathlineto{\pgfqpoint{1.902374in}{1.410355in}}%
\pgfpathlineto{\pgfqpoint{1.897485in}{1.422401in}}%
\pgfpathlineto{\pgfqpoint{1.893492in}{1.431903in}}%
\pgfpathlineto{\pgfqpoint{1.892407in}{1.434447in}}%
\pgfpathlineto{\pgfqpoint{1.887092in}{1.446493in}}%
\pgfpathlineto{\pgfqpoint{1.881601in}{1.458539in}}%
\pgfpathlineto{\pgfqpoint{1.881446in}{1.458869in}}%
\pgfpathlineto{\pgfqpoint{1.875850in}{1.470586in}}%
\pgfpathlineto{\pgfqpoint{1.869917in}{1.482632in}}%
\pgfpathlineto{\pgfqpoint{1.869400in}{1.483650in}}%
\pgfpathlineto{\pgfqpoint{1.863719in}{1.494678in}}%
\pgfpathlineto{\pgfqpoint{1.857354in}{1.506681in}}%
\pgfpathlineto{\pgfqpoint{1.857330in}{1.506724in}}%
\pgfpathlineto{\pgfqpoint{1.850659in}{1.518770in}}%
\pgfpathlineto{\pgfqpoint{1.845307in}{1.528171in}}%
\pgfpathlineto{\pgfqpoint{1.843778in}{1.530816in}}%
\pgfpathlineto{\pgfqpoint{1.836626in}{1.542862in}}%
\pgfpathlineto{\pgfqpoint{1.833261in}{1.548383in}}%
\pgfpathlineto{\pgfqpoint{1.829222in}{1.554908in}}%
\pgfpathlineto{\pgfqpoint{1.821574in}{1.566955in}}%
\pgfpathlineto{\pgfqpoint{1.821215in}{1.567505in}}%
\pgfpathlineto{\pgfqpoint{1.813614in}{1.579001in}}%
\pgfpathlineto{\pgfqpoint{1.809169in}{1.585561in}}%
\pgfpathlineto{\pgfqpoint{1.805392in}{1.591047in}}%
\pgfpathlineto{\pgfqpoint{1.797123in}{1.602778in}}%
\pgfpathlineto{\pgfqpoint{1.796897in}{1.603093in}}%
\pgfpathlineto{\pgfqpoint{1.788063in}{1.615139in}}%
\pgfpathlineto{\pgfqpoint{1.785077in}{1.619121in}}%
\pgfpathlineto{\pgfqpoint{1.778926in}{1.627185in}}%
\pgfpathlineto{\pgfqpoint{1.773031in}{1.634747in}}%
\pgfpathlineto{\pgfqpoint{1.769474in}{1.639231in}}%
\pgfpathlineto{\pgfqpoint{1.760985in}{1.649708in}}%
\pgfpathlineto{\pgfqpoint{1.759690in}{1.651277in}}%
\pgfpathlineto{\pgfqpoint{1.749546in}{1.663323in}}%
\pgfpathlineto{\pgfqpoint{1.748939in}{1.664030in}}%
\pgfpathlineto{\pgfqpoint{1.739017in}{1.675370in}}%
\pgfpathlineto{\pgfqpoint{1.736892in}{1.677749in}}%
\pgfpathlineto{\pgfqpoint{1.728104in}{1.687416in}}%
\pgfpathlineto{\pgfqpoint{1.724846in}{1.690929in}}%
\pgfpathlineto{\pgfqpoint{1.716788in}{1.699462in}}%
\pgfpathlineto{\pgfqpoint{1.712800in}{1.703603in}}%
\pgfpathlineto{\pgfqpoint{1.705044in}{1.711508in}}%
\pgfpathlineto{\pgfqpoint{1.700754in}{1.715798in}}%
\pgfpathlineto{\pgfqpoint{1.692849in}{1.723554in}}%
\pgfpathlineto{\pgfqpoint{1.688708in}{1.727542in}}%
\pgfpathlineto{\pgfqpoint{1.680175in}{1.735600in}}%
\pgfpathlineto{\pgfqpoint{1.676662in}{1.738858in}}%
\pgfpathlineto{\pgfqpoint{1.666995in}{1.747646in}}%
\pgfpathlineto{\pgfqpoint{1.664616in}{1.749770in}}%
\pgfpathlineto{\pgfqpoint{1.653276in}{1.759692in}}%
\pgfpathlineto{\pgfqpoint{1.652570in}{1.760300in}}%
\pgfpathlineto{\pgfqpoint{1.640524in}{1.770444in}}%
\pgfpathlineto{\pgfqpoint{1.638954in}{1.771738in}}%
\pgfpathlineto{\pgfqpoint{1.628477in}{1.780228in}}%
\pgfpathlineto{\pgfqpoint{1.623993in}{1.783785in}}%
\pgfpathlineto{\pgfqpoint{1.616431in}{1.789680in}}%
\pgfpathlineto{\pgfqpoint{1.608367in}{1.795831in}}%
\pgfpathlineto{\pgfqpoint{1.604385in}{1.798817in}}%
\pgfpathlineto{\pgfqpoint{1.592339in}{1.807651in}}%
\pgfpathlineto{\pgfqpoint{1.592024in}{1.807877in}}%
\pgfpathlineto{\pgfqpoint{1.580293in}{1.816146in}}%
\pgfpathlineto{\pgfqpoint{1.574808in}{1.819923in}}%
\pgfpathlineto{\pgfqpoint{1.568247in}{1.824367in}}%
\pgfpathlineto{\pgfqpoint{1.556752in}{1.831969in}}%
\pgfpathlineto{\pgfqpoint{1.556201in}{1.832328in}}%
\pgfpathlineto{\pgfqpoint{1.544155in}{1.839976in}}%
\pgfpathlineto{\pgfqpoint{1.537630in}{1.844015in}}%
\pgfpathlineto{\pgfqpoint{1.532108in}{1.847380in}}%
\pgfpathlineto{\pgfqpoint{1.520062in}{1.854532in}}%
\pgfpathlineto{\pgfqpoint{1.517417in}{1.856061in}}%
\pgfpathlineto{\pgfqpoint{1.508016in}{1.861413in}}%
\pgfpathlineto{\pgfqpoint{1.495970in}{1.868084in}}%
\pgfpathlineto{\pgfqpoint{1.495927in}{1.868107in}}%
\pgfpathlineto{\pgfqpoint{1.483924in}{1.874473in}}%
\pgfpathlineto{\pgfqpoint{1.472897in}{1.880154in}}%
\pgfpathlineto{\pgfqpoint{1.471878in}{1.880670in}}%
\pgfpathlineto{\pgfqpoint{1.459832in}{1.886604in}}%
\pgfpathlineto{\pgfqpoint{1.448115in}{1.892200in}}%
\pgfpathlineto{\pgfqpoint{1.447786in}{1.892355in}}%
\pgfpathlineto{\pgfqpoint{1.435740in}{1.897846in}}%
\pgfpathlineto{\pgfqpoint{1.423693in}{1.903161in}}%
\pgfpathlineto{\pgfqpoint{1.421149in}{1.904246in}}%
\pgfpathlineto{\pgfqpoint{1.411647in}{1.908239in}}%
\pgfpathlineto{\pgfqpoint{1.399601in}{1.913128in}}%
\pgfpathlineto{\pgfqpoint{1.391515in}{1.916292in}}%
\pgfpathlineto{\pgfqpoint{1.387555in}{1.917819in}}%
\pgfpathlineto{\pgfqpoint{1.375509in}{1.922293in}}%
\pgfpathlineto{\pgfqpoint{1.363463in}{1.926596in}}%
\pgfpathlineto{\pgfqpoint{1.358381in}{1.928338in}}%
\pgfpathlineto{\pgfqpoint{1.351417in}{1.930692in}}%
\pgfpathlineto{\pgfqpoint{1.339371in}{1.934594in}}%
\pgfpathlineto{\pgfqpoint{1.327325in}{1.938327in}}%
\pgfpathlineto{\pgfqpoint{1.320369in}{1.940384in}}%
\pgfpathlineto{\pgfqpoint{1.315278in}{1.941869in}}%
\pgfpathlineto{\pgfqpoint{1.303232in}{1.945215in}}%
\pgfpathlineto{\pgfqpoint{1.291186in}{1.948394in}}%
\pgfpathlineto{\pgfqpoint{1.279140in}{1.951405in}}%
\pgfpathlineto{\pgfqpoint{1.274800in}{1.952430in}}%
\pgfpathlineto{\pgfqpoint{1.267094in}{1.954225in}}%
\pgfpathlineto{\pgfqpoint{1.255048in}{1.956865in}}%
\pgfpathlineto{\pgfqpoint{1.243002in}{1.959340in}}%
\pgfpathlineto{\pgfqpoint{1.230956in}{1.961650in}}%
\pgfpathlineto{\pgfqpoint{1.218909in}{1.963796in}}%
\pgfpathlineto{\pgfqpoint{1.214769in}{1.964476in}}%
\pgfpathlineto{\pgfqpoint{1.206863in}{1.965758in}}%
\pgfpathlineto{\pgfqpoint{1.194817in}{1.967549in}}%
\pgfpathlineto{\pgfqpoint{1.182771in}{1.969177in}}%
\pgfpathlineto{\pgfqpoint{1.170725in}{1.970642in}}%
\pgfpathlineto{\pgfqpoint{1.158679in}{1.971944in}}%
\pgfpathlineto{\pgfqpoint{1.146633in}{1.973084in}}%
\pgfpathlineto{\pgfqpoint{1.134587in}{1.974060in}}%
\pgfpathlineto{\pgfqpoint{1.122541in}{1.974874in}}%
\pgfpathlineto{\pgfqpoint{1.110494in}{1.975525in}}%
\pgfpathlineto{\pgfqpoint{1.098448in}{1.976014in}}%
\pgfpathlineto{\pgfqpoint{1.086402in}{1.976339in}}%
\pgfpathlineto{\pgfqpoint{1.074356in}{1.976502in}}%
\pgfpathlineto{\pgfqpoint{1.062310in}{1.976502in}}%
\pgfpathlineto{\pgfqpoint{1.050264in}{1.976339in}}%
\pgfpathlineto{\pgfqpoint{1.038218in}{1.976014in}}%
\pgfpathlineto{\pgfqpoint{1.026172in}{1.975525in}}%
\pgfpathlineto{\pgfqpoint{1.014126in}{1.974874in}}%
\pgfpathlineto{\pgfqpoint{1.002079in}{1.974060in}}%
\pgfpathlineto{\pgfqpoint{0.990033in}{1.973084in}}%
\pgfpathlineto{\pgfqpoint{0.977987in}{1.971944in}}%
\pgfpathlineto{\pgfqpoint{0.965941in}{1.970642in}}%
\pgfpathlineto{\pgfqpoint{0.953895in}{1.969177in}}%
\pgfpathlineto{\pgfqpoint{0.941849in}{1.967549in}}%
\pgfpathlineto{\pgfqpoint{0.929803in}{1.965758in}}%
\pgfpathlineto{\pgfqpoint{0.921897in}{1.964476in}}%
\pgfpathlineto{\pgfqpoint{0.917757in}{1.963796in}}%
\pgfpathlineto{\pgfqpoint{0.905710in}{1.961650in}}%
\pgfpathlineto{\pgfqpoint{0.893664in}{1.959340in}}%
\pgfpathlineto{\pgfqpoint{0.881618in}{1.956865in}}%
\pgfpathlineto{\pgfqpoint{0.869572in}{1.954225in}}%
\pgfpathlineto{\pgfqpoint{0.861866in}{1.952430in}}%
\pgfpathlineto{\pgfqpoint{0.857526in}{1.951405in}}%
\pgfpathlineto{\pgfqpoint{0.845480in}{1.948394in}}%
\pgfpathlineto{\pgfqpoint{0.833434in}{1.945215in}}%
\pgfpathlineto{\pgfqpoint{0.821388in}{1.941869in}}%
\pgfpathlineto{\pgfqpoint{0.816297in}{1.940384in}}%
\pgfpathlineto{\pgfqpoint{0.809342in}{1.938327in}}%
\pgfpathlineto{\pgfqpoint{0.797295in}{1.934594in}}%
\pgfpathlineto{\pgfqpoint{0.785249in}{1.930692in}}%
\pgfpathlineto{\pgfqpoint{0.778285in}{1.928338in}}%
\pgfpathlineto{\pgfqpoint{0.773203in}{1.926596in}}%
\pgfpathlineto{\pgfqpoint{0.761157in}{1.922293in}}%
\pgfpathlineto{\pgfqpoint{0.749111in}{1.917819in}}%
\pgfpathlineto{\pgfqpoint{0.745151in}{1.916292in}}%
\pgfpathlineto{\pgfqpoint{0.737065in}{1.913128in}}%
\pgfpathlineto{\pgfqpoint{0.725019in}{1.908239in}}%
\pgfpathlineto{\pgfqpoint{0.715517in}{1.904246in}}%
\pgfpathlineto{\pgfqpoint{0.712973in}{1.903161in}}%
\pgfpathlineto{\pgfqpoint{0.700927in}{1.897846in}}%
\pgfpathlineto{\pgfqpoint{0.688880in}{1.892355in}}%
\pgfpathlineto{\pgfqpoint{0.688551in}{1.892200in}}%
\pgfpathlineto{\pgfqpoint{0.676834in}{1.886604in}}%
\pgfpathlineto{\pgfqpoint{0.664788in}{1.880670in}}%
\pgfpathlineto{\pgfqpoint{0.663770in}{1.880154in}}%
\pgfpathlineto{\pgfqpoint{0.652742in}{1.874473in}}%
\pgfpathlineto{\pgfqpoint{0.640739in}{1.868107in}}%
\pgfpathlineto{\pgfqpoint{0.640696in}{1.868084in}}%
\pgfpathlineto{\pgfqpoint{0.628650in}{1.861413in}}%
\pgfpathlineto{\pgfqpoint{0.619249in}{1.856061in}}%
\pgfpathlineto{\pgfqpoint{0.616604in}{1.854532in}}%
\pgfpathlineto{\pgfqpoint{0.604558in}{1.847380in}}%
\pgfpathlineto{\pgfqpoint{0.599036in}{1.844015in}}%
\pgfpathlineto{\pgfqpoint{0.592511in}{1.839976in}}%
\pgfpathlineto{\pgfqpoint{0.580465in}{1.832328in}}%
\pgfpathlineto{\pgfqpoint{0.579914in}{1.831969in}}%
\pgfpathlineto{\pgfqpoint{0.568419in}{1.824367in}}%
\pgfpathlineto{\pgfqpoint{0.561858in}{1.819923in}}%
\pgfpathlineto{\pgfqpoint{0.556373in}{1.816146in}}%
\pgfpathlineto{\pgfqpoint{0.544642in}{1.807877in}}%
\pgfpathlineto{\pgfqpoint{0.544327in}{1.807651in}}%
\pgfpathlineto{\pgfqpoint{0.532281in}{1.798817in}}%
\pgfpathlineto{\pgfqpoint{0.528299in}{1.795831in}}%
\pgfpathlineto{\pgfqpoint{0.520235in}{1.789680in}}%
\pgfpathlineto{\pgfqpoint{0.512673in}{1.783785in}}%
\pgfpathlineto{\pgfqpoint{0.508189in}{1.780228in}}%
\pgfpathlineto{\pgfqpoint{0.497712in}{1.771738in}}%
\pgfpathlineto{\pgfqpoint{0.496143in}{1.770444in}}%
\pgfpathlineto{\pgfqpoint{0.484096in}{1.760300in}}%
\pgfpathlineto{\pgfqpoint{0.483390in}{1.759692in}}%
\pgfpathlineto{\pgfqpoint{0.472050in}{1.749770in}}%
\pgfpathlineto{\pgfqpoint{0.469671in}{1.747646in}}%
\pgfpathlineto{\pgfqpoint{0.460004in}{1.738858in}}%
\pgfpathlineto{\pgfqpoint{0.456491in}{1.735600in}}%
\pgfpathlineto{\pgfqpoint{0.447958in}{1.727542in}}%
\pgfpathlineto{\pgfqpoint{0.443817in}{1.723554in}}%
\pgfpathlineto{\pgfqpoint{0.435912in}{1.715798in}}%
\pgfpathlineto{\pgfqpoint{0.431622in}{1.711508in}}%
\pgfpathlineto{\pgfqpoint{0.423866in}{1.703603in}}%
\pgfpathlineto{\pgfqpoint{0.419878in}{1.699462in}}%
\pgfpathlineto{\pgfqpoint{0.411820in}{1.690929in}}%
\pgfpathlineto{\pgfqpoint{0.408562in}{1.687416in}}%
\pgfpathlineto{\pgfqpoint{0.399774in}{1.677749in}}%
\pgfpathlineto{\pgfqpoint{0.397649in}{1.675370in}}%
\pgfpathlineto{\pgfqpoint{0.387728in}{1.664030in}}%
\pgfpathlineto{\pgfqpoint{0.387120in}{1.663323in}}%
\pgfpathlineto{\pgfqpoint{0.376976in}{1.651277in}}%
\pgfpathlineto{\pgfqpoint{0.375681in}{1.649708in}}%
\pgfpathlineto{\pgfqpoint{0.367192in}{1.639231in}}%
\pgfpathlineto{\pgfqpoint{0.363635in}{1.634747in}}%
\pgfpathlineto{\pgfqpoint{0.357740in}{1.627185in}}%
\pgfpathlineto{\pgfqpoint{0.351589in}{1.619121in}}%
\pgfpathlineto{\pgfqpoint{0.348603in}{1.615139in}}%
\pgfpathlineto{\pgfqpoint{0.339769in}{1.603093in}}%
\pgfpathlineto{\pgfqpoint{0.339543in}{1.602778in}}%
\pgfpathlineto{\pgfqpoint{0.331274in}{1.591047in}}%
\pgfpathlineto{\pgfqpoint{0.327497in}{1.585561in}}%
\pgfpathlineto{\pgfqpoint{0.323052in}{1.579001in}}%
\pgfpathlineto{\pgfqpoint{0.315451in}{1.567505in}}%
\pgfpathlineto{\pgfqpoint{0.315092in}{1.566955in}}%
\pgfpathlineto{\pgfqpoint{0.307444in}{1.554908in}}%
\pgfpathlineto{\pgfqpoint{0.303405in}{1.548383in}}%
\pgfpathlineto{\pgfqpoint{0.300040in}{1.542862in}}%
\pgfpathlineto{\pgfqpoint{0.292888in}{1.530816in}}%
\pgfpathlineto{\pgfqpoint{0.291359in}{1.528171in}}%
\pgfpathlineto{\pgfqpoint{0.286007in}{1.518770in}}%
\pgfpathlineto{\pgfqpoint{0.279336in}{1.506724in}}%
\pgfpathlineto{\pgfqpoint{0.279312in}{1.506681in}}%
\pgfpathlineto{\pgfqpoint{0.272947in}{1.494678in}}%
\pgfpathlineto{\pgfqpoint{0.267266in}{1.483650in}}%
\pgfpathlineto{\pgfqpoint{0.266749in}{1.482632in}}%
\pgfpathlineto{\pgfqpoint{0.260816in}{1.470586in}}%
\pgfpathlineto{\pgfqpoint{0.255220in}{1.458869in}}%
\pgfpathlineto{\pgfqpoint{0.255065in}{1.458539in}}%
\pgfpathlineto{\pgfqpoint{0.249574in}{1.446493in}}%
\pgfpathlineto{\pgfqpoint{0.244259in}{1.434447in}}%
\pgfpathlineto{\pgfqpoint{0.243174in}{1.431903in}}%
\pgfpathlineto{\pgfqpoint{0.239181in}{1.422401in}}%
\pgfpathlineto{\pgfqpoint{0.234292in}{1.410355in}}%
\pgfpathlineto{\pgfqpoint{0.231128in}{1.402269in}}%
\pgfpathlineto{\pgfqpoint{0.229601in}{1.398309in}}%
\pgfpathlineto{\pgfqpoint{0.225126in}{1.386263in}}%
\pgfpathlineto{\pgfqpoint{0.220824in}{1.374217in}}%
\pgfpathlineto{\pgfqpoint{0.219082in}{1.369135in}}%
\pgfpathlineto{\pgfqpoint{0.216728in}{1.362171in}}%
\pgfpathlineto{\pgfqpoint{0.212826in}{1.350124in}}%
\pgfpathlineto{\pgfqpoint{0.209093in}{1.338078in}}%
\pgfpathlineto{\pgfqpoint{0.207036in}{1.331123in}}%
\pgfpathlineto{\pgfqpoint{0.205551in}{1.326032in}}%
\pgfpathlineto{\pgfqpoint{0.202205in}{1.313986in}}%
\pgfpathlineto{\pgfqpoint{0.199026in}{1.301940in}}%
\pgfpathlineto{\pgfqpoint{0.196014in}{1.289894in}}%
\pgfpathlineto{\pgfqpoint{0.194990in}{1.285554in}}%
\pgfpathlineto{\pgfqpoint{0.193195in}{1.277848in}}%
\pgfpathlineto{\pgfqpoint{0.190555in}{1.265802in}}%
\pgfpathlineto{\pgfqpoint{0.188080in}{1.253756in}}%
\pgfpathlineto{\pgfqpoint{0.185769in}{1.241709in}}%
\pgfpathlineto{\pgfqpoint{0.183624in}{1.229663in}}%
\pgfpathlineto{\pgfqpoint{0.182944in}{1.225522in}}%
\pgfpathlineto{\pgfqpoint{0.181662in}{1.217617in}}%
\pgfpathlineto{\pgfqpoint{0.179871in}{1.205571in}}%
\pgfpathlineto{\pgfqpoint{0.178243in}{1.193525in}}%
\pgfpathlineto{\pgfqpoint{0.176778in}{1.181479in}}%
\pgfpathlineto{\pgfqpoint{0.175476in}{1.169433in}}%
\pgfpathlineto{\pgfqpoint{0.174336in}{1.157387in}}%
\pgfpathlineto{\pgfqpoint{0.173360in}{1.145340in}}%
\pgfpathlineto{\pgfqpoint{0.172546in}{1.133294in}}%
\pgfpathlineto{\pgfqpoint{0.171894in}{1.121248in}}%
\pgfpathlineto{\pgfqpoint{0.171406in}{1.109202in}}%
\pgfpathlineto{\pgfqpoint{0.171081in}{1.097156in}}%
\pgfpathlineto{\pgfqpoint{0.170918in}{1.085110in}}%
\pgfpathlineto{\pgfqpoint{0.170918in}{1.073064in}}%
\pgfpathlineto{\pgfqpoint{0.171081in}{1.061018in}}%
\pgfpathlineto{\pgfqpoint{0.171406in}{1.048972in}}%
\pgfpathlineto{\pgfqpoint{0.171894in}{1.036925in}}%
\pgfpathlineto{\pgfqpoint{0.172546in}{1.024879in}}%
\pgfpathlineto{\pgfqpoint{0.173360in}{1.012833in}}%
\pgfpathlineto{\pgfqpoint{0.174336in}{1.000787in}}%
\pgfpathlineto{\pgfqpoint{0.175476in}{0.988741in}}%
\pgfpathlineto{\pgfqpoint{0.176778in}{0.976695in}}%
\pgfpathlineto{\pgfqpoint{0.178243in}{0.964649in}}%
\pgfpathlineto{\pgfqpoint{0.179871in}{0.952603in}}%
\pgfpathlineto{\pgfqpoint{0.181662in}{0.940557in}}%
\pgfpathlineto{\pgfqpoint{0.182944in}{0.932651in}}%
\pgfpathlineto{\pgfqpoint{0.183624in}{0.928510in}}%
\pgfpathlineto{\pgfqpoint{0.185769in}{0.916464in}}%
\pgfpathlineto{\pgfqpoint{0.188080in}{0.904418in}}%
\pgfpathlineto{\pgfqpoint{0.190555in}{0.892372in}}%
\pgfpathlineto{\pgfqpoint{0.193195in}{0.880326in}}%
\pgfpathlineto{\pgfqpoint{0.194990in}{0.872620in}}%
\pgfpathlineto{\pgfqpoint{0.196014in}{0.868280in}}%
\pgfpathlineto{\pgfqpoint{0.199026in}{0.856234in}}%
\pgfpathlineto{\pgfqpoint{0.202205in}{0.844188in}}%
\pgfpathlineto{\pgfqpoint{0.205551in}{0.832141in}}%
\pgfpathlineto{\pgfqpoint{0.207036in}{0.827051in}}%
\pgfpathlineto{\pgfqpoint{0.209093in}{0.820095in}}%
\pgfpathlineto{\pgfqpoint{0.212826in}{0.808049in}}%
\pgfpathlineto{\pgfqpoint{0.216728in}{0.796003in}}%
\pgfpathlineto{\pgfqpoint{0.219082in}{0.789039in}}%
\pgfpathlineto{\pgfqpoint{0.220824in}{0.783957in}}%
\pgfpathlineto{\pgfqpoint{0.225126in}{0.771911in}}%
\pgfpathlineto{\pgfqpoint{0.229601in}{0.759865in}}%
\pgfpathlineto{\pgfqpoint{0.231128in}{0.755905in}}%
\pgfpathlineto{\pgfqpoint{0.234292in}{0.747819in}}%
\pgfpathlineto{\pgfqpoint{0.239181in}{0.735773in}}%
\pgfpathlineto{\pgfqpoint{0.243174in}{0.726271in}}%
\pgfpathlineto{\pgfqpoint{0.244259in}{0.723726in}}%
\pgfpathlineto{\pgfqpoint{0.249574in}{0.711680in}}%
\pgfpathlineto{\pgfqpoint{0.255065in}{0.699634in}}%
\pgfpathlineto{\pgfqpoint{0.255220in}{0.699305in}}%
\pgfpathlineto{\pgfqpoint{0.260816in}{0.687588in}}%
\pgfpathlineto{\pgfqpoint{0.266749in}{0.675542in}}%
\pgfpathlineto{\pgfqpoint{0.267266in}{0.674523in}}%
\pgfpathlineto{\pgfqpoint{0.272947in}{0.663496in}}%
\pgfpathlineto{\pgfqpoint{0.279312in}{0.651493in}}%
\pgfpathlineto{\pgfqpoint{0.279336in}{0.651450in}}%
\pgfpathlineto{\pgfqpoint{0.286007in}{0.639404in}}%
\pgfpathlineto{\pgfqpoint{0.291359in}{0.630003in}}%
\pgfpathlineto{\pgfqpoint{0.292888in}{0.627358in}}%
\pgfpathlineto{\pgfqpoint{0.300040in}{0.615311in}}%
\pgfpathlineto{\pgfqpoint{0.303405in}{0.609790in}}%
\pgfpathlineto{\pgfqpoint{0.307444in}{0.603265in}}%
\pgfpathlineto{\pgfqpoint{0.315092in}{0.591219in}}%
\pgfpathlineto{\pgfqpoint{0.315451in}{0.590668in}}%
\pgfpathlineto{\pgfqpoint{0.323052in}{0.579173in}}%
\pgfpathlineto{\pgfqpoint{0.327497in}{0.572612in}}%
\pgfpathlineto{\pgfqpoint{0.331274in}{0.567127in}}%
\pgfpathlineto{\pgfqpoint{0.339543in}{0.555396in}}%
\pgfpathlineto{\pgfqpoint{0.339769in}{0.555081in}}%
\pgfpathlineto{\pgfqpoint{0.348603in}{0.543035in}}%
\pgfpathlineto{\pgfqpoint{0.351589in}{0.539053in}}%
\pgfpathlineto{\pgfqpoint{0.357740in}{0.530989in}}%
\pgfpathlineto{\pgfqpoint{0.363635in}{0.523427in}}%
\pgfpathlineto{\pgfqpoint{0.367192in}{0.518942in}}%
\pgfpathlineto{\pgfqpoint{0.375681in}{0.508466in}}%
\pgfpathlineto{\pgfqpoint{0.376976in}{0.506896in}}%
\pgfpathlineto{\pgfqpoint{0.387120in}{0.494850in}}%
\pgfpathlineto{\pgfqpoint{0.387728in}{0.494143in}}%
\pgfpathlineto{\pgfqpoint{0.397649in}{0.482804in}}%
\pgfpathlineto{\pgfqpoint{0.399774in}{0.480425in}}%
\pgfpathlineto{\pgfqpoint{0.408562in}{0.470758in}}%
\pgfpathlineto{\pgfqpoint{0.411820in}{0.467245in}}%
\pgfpathlineto{\pgfqpoint{0.419878in}{0.458712in}}%
\pgfpathlineto{\pgfqpoint{0.423866in}{0.454571in}}%
\pgfpathlineto{\pgfqpoint{0.431622in}{0.446666in}}%
\pgfpathlineto{\pgfqpoint{0.435912in}{0.442376in}}%
\pgfpathlineto{\pgfqpoint{0.443817in}{0.434620in}}%
\pgfpathlineto{\pgfqpoint{0.447958in}{0.430632in}}%
\pgfpathlineto{\pgfqpoint{0.456491in}{0.422574in}}%
\pgfpathlineto{\pgfqpoint{0.460004in}{0.419316in}}%
\pgfpathlineto{\pgfqpoint{0.469671in}{0.410527in}}%
\pgfpathlineto{\pgfqpoint{0.472050in}{0.408403in}}%
\pgfpathlineto{\pgfqpoint{0.483390in}{0.398481in}}%
\pgfpathlineto{\pgfqpoint{0.484096in}{0.397874in}}%
\pgfpathlineto{\pgfqpoint{0.496143in}{0.387730in}}%
\pgfpathlineto{\pgfqpoint{0.497712in}{0.386435in}}%
\pgfpathlineto{\pgfqpoint{0.508189in}{0.377946in}}%
\pgfpathlineto{\pgfqpoint{0.512673in}{0.374389in}}%
\pgfpathlineto{\pgfqpoint{0.520235in}{0.368494in}}%
\pgfpathlineto{\pgfqpoint{0.528299in}{0.362343in}}%
\pgfpathlineto{\pgfqpoint{0.532281in}{0.359357in}}%
\pgfpathlineto{\pgfqpoint{0.544327in}{0.350523in}}%
\pgfpathlineto{\pgfqpoint{0.544642in}{0.350297in}}%
\pgfpathlineto{\pgfqpoint{0.556373in}{0.342028in}}%
\pgfpathlineto{\pgfqpoint{0.561858in}{0.338251in}}%
\pgfpathlineto{\pgfqpoint{0.568419in}{0.333806in}}%
\pgfpathlineto{\pgfqpoint{0.579914in}{0.326205in}}%
\pgfpathlineto{\pgfqpoint{0.580465in}{0.325846in}}%
\pgfpathlineto{\pgfqpoint{0.592511in}{0.318198in}}%
\pgfpathlineto{\pgfqpoint{0.599036in}{0.314159in}}%
\pgfpathlineto{\pgfqpoint{0.604558in}{0.310794in}}%
\pgfpathlineto{\pgfqpoint{0.616604in}{0.303642in}}%
\pgfpathlineto{\pgfqpoint{0.619249in}{0.302112in}}%
\pgfpathlineto{\pgfqpoint{0.628650in}{0.296761in}}%
\pgfpathlineto{\pgfqpoint{0.640696in}{0.290089in}}%
\pgfpathlineto{\pgfqpoint{0.640739in}{0.290066in}}%
\pgfpathlineto{\pgfqpoint{0.652742in}{0.283701in}}%
\pgfpathlineto{\pgfqpoint{0.663770in}{0.278020in}}%
\pgfpathlineto{\pgfqpoint{0.664788in}{0.277503in}}%
\pgfpathlineto{\pgfqpoint{0.676834in}{0.271570in}}%
\pgfpathlineto{\pgfqpoint{0.688551in}{0.265974in}}%
\pgfpathlineto{\pgfqpoint{0.688880in}{0.265819in}}%
\pgfpathlineto{\pgfqpoint{0.700927in}{0.260327in}}%
\pgfpathlineto{\pgfqpoint{0.712973in}{0.255013in}}%
\pgfpathlineto{\pgfqpoint{0.715517in}{0.253928in}}%
\pgfpathlineto{\pgfqpoint{0.725019in}{0.249934in}}%
\pgfpathlineto{\pgfqpoint{0.737065in}{0.245046in}}%
\pgfpathlineto{\pgfqpoint{0.745151in}{0.241882in}}%
\pgfpathlineto{\pgfqpoint{0.749111in}{0.240355in}}%
\pgfpathlineto{\pgfqpoint{0.761157in}{0.235880in}}%
\pgfpathlineto{\pgfqpoint{0.773203in}{0.231578in}}%
\pgfpathlineto{\pgfqpoint{0.778285in}{0.229836in}}%
\pgfpathlineto{\pgfqpoint{0.785249in}{0.227482in}}%
\pgfpathlineto{\pgfqpoint{0.797295in}{0.223579in}}%
\pgfpathlineto{\pgfqpoint{0.809342in}{0.219847in}}%
\pgfpathlineto{\pgfqpoint{0.816297in}{0.217790in}}%
\pgfpathlineto{\pgfqpoint{0.821388in}{0.216305in}}%
\pgfpathlineto{\pgfqpoint{0.833434in}{0.212959in}}%
\pgfpathlineto{\pgfqpoint{0.845480in}{0.209780in}}%
\pgfpathlineto{\pgfqpoint{0.857526in}{0.206768in}}%
\pgfpathlineto{\pgfqpoint{0.861866in}{0.205743in}}%
\pgfpathlineto{\pgfqpoint{0.869572in}{0.203949in}}%
\pgfpathlineto{\pgfqpoint{0.881618in}{0.201309in}}%
\pgfpathlineto{\pgfqpoint{0.893664in}{0.198833in}}%
\pgfpathlineto{\pgfqpoint{0.905710in}{0.196523in}}%
\pgfpathlineto{\pgfqpoint{0.917757in}{0.194378in}}%
\pgfpathlineto{\pgfqpoint{0.921897in}{0.193697in}}%
\pgfpathclose%
\pgfpathmoveto{\pgfqpoint{0.921897in}{0.193697in}}%
\pgfpathlineto{\pgfqpoint{0.917757in}{0.194378in}}%
\pgfpathlineto{\pgfqpoint{0.905710in}{0.196523in}}%
\pgfpathlineto{\pgfqpoint{0.893664in}{0.198833in}}%
\pgfpathlineto{\pgfqpoint{0.881618in}{0.201309in}}%
\pgfpathlineto{\pgfqpoint{0.869572in}{0.203949in}}%
\pgfpathlineto{\pgfqpoint{0.861866in}{0.205743in}}%
\pgfpathlineto{\pgfqpoint{0.857526in}{0.206768in}}%
\pgfpathlineto{\pgfqpoint{0.845480in}{0.209780in}}%
\pgfpathlineto{\pgfqpoint{0.833434in}{0.212959in}}%
\pgfpathlineto{\pgfqpoint{0.821388in}{0.216305in}}%
\pgfpathlineto{\pgfqpoint{0.816297in}{0.217790in}}%
\pgfpathlineto{\pgfqpoint{0.809342in}{0.219847in}}%
\pgfpathlineto{\pgfqpoint{0.797295in}{0.223579in}}%
\pgfpathlineto{\pgfqpoint{0.785249in}{0.227482in}}%
\pgfpathlineto{\pgfqpoint{0.778285in}{0.229836in}}%
\pgfpathlineto{\pgfqpoint{0.773203in}{0.231578in}}%
\pgfpathlineto{\pgfqpoint{0.761157in}{0.235880in}}%
\pgfpathlineto{\pgfqpoint{0.749111in}{0.240355in}}%
\pgfpathlineto{\pgfqpoint{0.745151in}{0.241882in}}%
\pgfpathlineto{\pgfqpoint{0.737065in}{0.245046in}}%
\pgfpathlineto{\pgfqpoint{0.725019in}{0.249934in}}%
\pgfpathlineto{\pgfqpoint{0.715517in}{0.253928in}}%
\pgfpathlineto{\pgfqpoint{0.712973in}{0.255013in}}%
\pgfpathlineto{\pgfqpoint{0.700927in}{0.260327in}}%
\pgfpathlineto{\pgfqpoint{0.688880in}{0.265819in}}%
\pgfpathlineto{\pgfqpoint{0.688551in}{0.265974in}}%
\pgfpathlineto{\pgfqpoint{0.676834in}{0.271570in}}%
\pgfpathlineto{\pgfqpoint{0.664788in}{0.277503in}}%
\pgfpathlineto{\pgfqpoint{0.663770in}{0.278020in}}%
\pgfpathlineto{\pgfqpoint{0.652742in}{0.283701in}}%
\pgfpathlineto{\pgfqpoint{0.640739in}{0.290066in}}%
\pgfpathlineto{\pgfqpoint{0.640696in}{0.290089in}}%
\pgfpathlineto{\pgfqpoint{0.628650in}{0.296761in}}%
\pgfpathlineto{\pgfqpoint{0.619249in}{0.302112in}}%
\pgfpathlineto{\pgfqpoint{0.616604in}{0.303642in}}%
\pgfpathlineto{\pgfqpoint{0.604558in}{0.310794in}}%
\pgfpathlineto{\pgfqpoint{0.599036in}{0.314159in}}%
\pgfpathlineto{\pgfqpoint{0.592511in}{0.318198in}}%
\pgfpathlineto{\pgfqpoint{0.580465in}{0.325846in}}%
\pgfpathlineto{\pgfqpoint{0.579914in}{0.326205in}}%
\pgfpathlineto{\pgfqpoint{0.568419in}{0.333806in}}%
\pgfpathlineto{\pgfqpoint{0.561858in}{0.338251in}}%
\pgfpathlineto{\pgfqpoint{0.556373in}{0.342028in}}%
\pgfpathlineto{\pgfqpoint{0.544642in}{0.350297in}}%
\pgfpathlineto{\pgfqpoint{0.544327in}{0.350523in}}%
\pgfpathlineto{\pgfqpoint{0.532281in}{0.359357in}}%
\pgfpathlineto{\pgfqpoint{0.528299in}{0.362343in}}%
\pgfpathlineto{\pgfqpoint{0.520235in}{0.368494in}}%
\pgfpathlineto{\pgfqpoint{0.512673in}{0.374389in}}%
\pgfpathlineto{\pgfqpoint{0.508189in}{0.377946in}}%
\pgfpathlineto{\pgfqpoint{0.497712in}{0.386435in}}%
\pgfpathlineto{\pgfqpoint{0.496143in}{0.387730in}}%
\pgfpathlineto{\pgfqpoint{0.484096in}{0.397874in}}%
\pgfpathlineto{\pgfqpoint{0.483390in}{0.398481in}}%
\pgfpathlineto{\pgfqpoint{0.472050in}{0.408403in}}%
\pgfpathlineto{\pgfqpoint{0.469671in}{0.410527in}}%
\pgfpathlineto{\pgfqpoint{0.460004in}{0.419316in}}%
\pgfpathlineto{\pgfqpoint{0.456491in}{0.422574in}}%
\pgfpathlineto{\pgfqpoint{0.447958in}{0.430632in}}%
\pgfpathlineto{\pgfqpoint{0.443817in}{0.434620in}}%
\pgfpathlineto{\pgfqpoint{0.435912in}{0.442376in}}%
\pgfpathlineto{\pgfqpoint{0.431622in}{0.446666in}}%
\pgfpathlineto{\pgfqpoint{0.423866in}{0.454571in}}%
\pgfpathlineto{\pgfqpoint{0.419878in}{0.458712in}}%
\pgfpathlineto{\pgfqpoint{0.411820in}{0.467245in}}%
\pgfpathlineto{\pgfqpoint{0.408562in}{0.470758in}}%
\pgfpathlineto{\pgfqpoint{0.399774in}{0.480425in}}%
\pgfpathlineto{\pgfqpoint{0.397649in}{0.482804in}}%
\pgfpathlineto{\pgfqpoint{0.387728in}{0.494143in}}%
\pgfpathlineto{\pgfqpoint{0.387120in}{0.494850in}}%
\pgfpathlineto{\pgfqpoint{0.376976in}{0.506896in}}%
\pgfpathlineto{\pgfqpoint{0.375681in}{0.508466in}}%
\pgfpathlineto{\pgfqpoint{0.367192in}{0.518942in}}%
\pgfpathlineto{\pgfqpoint{0.363635in}{0.523427in}}%
\pgfpathlineto{\pgfqpoint{0.357740in}{0.530989in}}%
\pgfpathlineto{\pgfqpoint{0.351589in}{0.539053in}}%
\pgfpathlineto{\pgfqpoint{0.348603in}{0.543035in}}%
\pgfpathlineto{\pgfqpoint{0.339769in}{0.555081in}}%
\pgfpathlineto{\pgfqpoint{0.339543in}{0.555396in}}%
\pgfpathlineto{\pgfqpoint{0.331274in}{0.567127in}}%
\pgfpathlineto{\pgfqpoint{0.327497in}{0.572612in}}%
\pgfpathlineto{\pgfqpoint{0.323052in}{0.579173in}}%
\pgfpathlineto{\pgfqpoint{0.315451in}{0.590668in}}%
\pgfpathlineto{\pgfqpoint{0.315092in}{0.591219in}}%
\pgfpathlineto{\pgfqpoint{0.307444in}{0.603265in}}%
\pgfpathlineto{\pgfqpoint{0.303405in}{0.609790in}}%
\pgfpathlineto{\pgfqpoint{0.300040in}{0.615311in}}%
\pgfpathlineto{\pgfqpoint{0.292888in}{0.627358in}}%
\pgfpathlineto{\pgfqpoint{0.291359in}{0.630003in}}%
\pgfpathlineto{\pgfqpoint{0.286007in}{0.639404in}}%
\pgfpathlineto{\pgfqpoint{0.279336in}{0.651450in}}%
\pgfpathlineto{\pgfqpoint{0.279312in}{0.651493in}}%
\pgfpathlineto{\pgfqpoint{0.272947in}{0.663496in}}%
\pgfpathlineto{\pgfqpoint{0.267266in}{0.674523in}}%
\pgfpathlineto{\pgfqpoint{0.266749in}{0.675542in}}%
\pgfpathlineto{\pgfqpoint{0.260816in}{0.687588in}}%
\pgfpathlineto{\pgfqpoint{0.255220in}{0.699305in}}%
\pgfpathlineto{\pgfqpoint{0.255065in}{0.699634in}}%
\pgfpathlineto{\pgfqpoint{0.249574in}{0.711680in}}%
\pgfpathlineto{\pgfqpoint{0.244259in}{0.723726in}}%
\pgfpathlineto{\pgfqpoint{0.243174in}{0.726271in}}%
\pgfpathlineto{\pgfqpoint{0.239181in}{0.735773in}}%
\pgfpathlineto{\pgfqpoint{0.234292in}{0.747819in}}%
\pgfpathlineto{\pgfqpoint{0.231128in}{0.755905in}}%
\pgfpathlineto{\pgfqpoint{0.229601in}{0.759865in}}%
\pgfpathlineto{\pgfqpoint{0.225126in}{0.771911in}}%
\pgfpathlineto{\pgfqpoint{0.220824in}{0.783957in}}%
\pgfpathlineto{\pgfqpoint{0.219082in}{0.789039in}}%
\pgfpathlineto{\pgfqpoint{0.216728in}{0.796003in}}%
\pgfpathlineto{\pgfqpoint{0.212826in}{0.808049in}}%
\pgfpathlineto{\pgfqpoint{0.209093in}{0.820095in}}%
\pgfpathlineto{\pgfqpoint{0.207036in}{0.827051in}}%
\pgfpathlineto{\pgfqpoint{0.205551in}{0.832141in}}%
\pgfpathlineto{\pgfqpoint{0.202205in}{0.844188in}}%
\pgfpathlineto{\pgfqpoint{0.199026in}{0.856234in}}%
\pgfpathlineto{\pgfqpoint{0.196014in}{0.868280in}}%
\pgfpathlineto{\pgfqpoint{0.194990in}{0.872620in}}%
\pgfpathlineto{\pgfqpoint{0.193195in}{0.880326in}}%
\pgfpathlineto{\pgfqpoint{0.190555in}{0.892372in}}%
\pgfpathlineto{\pgfqpoint{0.188080in}{0.904418in}}%
\pgfpathlineto{\pgfqpoint{0.185769in}{0.916464in}}%
\pgfpathlineto{\pgfqpoint{0.183624in}{0.928510in}}%
\pgfpathlineto{\pgfqpoint{0.182944in}{0.932651in}}%
\pgfpathlineto{\pgfqpoint{0.181662in}{0.940557in}}%
\pgfpathlineto{\pgfqpoint{0.179871in}{0.952603in}}%
\pgfpathlineto{\pgfqpoint{0.178243in}{0.964649in}}%
\pgfpathlineto{\pgfqpoint{0.176778in}{0.976695in}}%
\pgfpathlineto{\pgfqpoint{0.175476in}{0.988741in}}%
\pgfpathlineto{\pgfqpoint{0.174336in}{1.000787in}}%
\pgfpathlineto{\pgfqpoint{0.173360in}{1.012833in}}%
\pgfpathlineto{\pgfqpoint{0.172546in}{1.024879in}}%
\pgfpathlineto{\pgfqpoint{0.171894in}{1.036925in}}%
\pgfpathlineto{\pgfqpoint{0.171406in}{1.048972in}}%
\pgfpathlineto{\pgfqpoint{0.171081in}{1.061018in}}%
\pgfpathlineto{\pgfqpoint{0.170918in}{1.073064in}}%
\pgfpathlineto{\pgfqpoint{0.170918in}{1.085110in}}%
\pgfpathlineto{\pgfqpoint{0.171081in}{1.097156in}}%
\pgfpathlineto{\pgfqpoint{0.171406in}{1.109202in}}%
\pgfpathlineto{\pgfqpoint{0.171894in}{1.121248in}}%
\pgfpathlineto{\pgfqpoint{0.172546in}{1.133294in}}%
\pgfpathlineto{\pgfqpoint{0.173360in}{1.145340in}}%
\pgfpathlineto{\pgfqpoint{0.174336in}{1.157387in}}%
\pgfpathlineto{\pgfqpoint{0.175476in}{1.169433in}}%
\pgfpathlineto{\pgfqpoint{0.176778in}{1.181479in}}%
\pgfpathlineto{\pgfqpoint{0.178243in}{1.193525in}}%
\pgfpathlineto{\pgfqpoint{0.179871in}{1.205571in}}%
\pgfpathlineto{\pgfqpoint{0.181662in}{1.217617in}}%
\pgfpathlineto{\pgfqpoint{0.182944in}{1.225522in}}%
\pgfpathlineto{\pgfqpoint{0.183624in}{1.229663in}}%
\pgfpathlineto{\pgfqpoint{0.185769in}{1.241709in}}%
\pgfpathlineto{\pgfqpoint{0.188080in}{1.253756in}}%
\pgfpathlineto{\pgfqpoint{0.190555in}{1.265802in}}%
\pgfpathlineto{\pgfqpoint{0.193195in}{1.277848in}}%
\pgfpathlineto{\pgfqpoint{0.194990in}{1.285554in}}%
\pgfpathlineto{\pgfqpoint{0.196014in}{1.289894in}}%
\pgfpathlineto{\pgfqpoint{0.199026in}{1.301940in}}%
\pgfpathlineto{\pgfqpoint{0.202205in}{1.313986in}}%
\pgfpathlineto{\pgfqpoint{0.205551in}{1.326032in}}%
\pgfpathlineto{\pgfqpoint{0.207036in}{1.331123in}}%
\pgfpathlineto{\pgfqpoint{0.209093in}{1.338078in}}%
\pgfpathlineto{\pgfqpoint{0.212826in}{1.350124in}}%
\pgfpathlineto{\pgfqpoint{0.216728in}{1.362171in}}%
\pgfpathlineto{\pgfqpoint{0.219082in}{1.369135in}}%
\pgfpathlineto{\pgfqpoint{0.220824in}{1.374217in}}%
\pgfpathlineto{\pgfqpoint{0.225126in}{1.386263in}}%
\pgfpathlineto{\pgfqpoint{0.229601in}{1.398309in}}%
\pgfpathlineto{\pgfqpoint{0.231128in}{1.402269in}}%
\pgfpathlineto{\pgfqpoint{0.234292in}{1.410355in}}%
\pgfpathlineto{\pgfqpoint{0.239181in}{1.422401in}}%
\pgfpathlineto{\pgfqpoint{0.243174in}{1.431903in}}%
\pgfpathlineto{\pgfqpoint{0.244259in}{1.434447in}}%
\pgfpathlineto{\pgfqpoint{0.249574in}{1.446493in}}%
\pgfpathlineto{\pgfqpoint{0.255065in}{1.458539in}}%
\pgfpathlineto{\pgfqpoint{0.255220in}{1.458869in}}%
\pgfpathlineto{\pgfqpoint{0.260816in}{1.470586in}}%
\pgfpathlineto{\pgfqpoint{0.266749in}{1.482632in}}%
\pgfpathlineto{\pgfqpoint{0.267266in}{1.483650in}}%
\pgfpathlineto{\pgfqpoint{0.272947in}{1.494678in}}%
\pgfpathlineto{\pgfqpoint{0.279312in}{1.506681in}}%
\pgfpathlineto{\pgfqpoint{0.279336in}{1.506724in}}%
\pgfpathlineto{\pgfqpoint{0.286007in}{1.518770in}}%
\pgfpathlineto{\pgfqpoint{0.291359in}{1.528171in}}%
\pgfpathlineto{\pgfqpoint{0.292888in}{1.530816in}}%
\pgfpathlineto{\pgfqpoint{0.300040in}{1.542862in}}%
\pgfpathlineto{\pgfqpoint{0.303405in}{1.548383in}}%
\pgfpathlineto{\pgfqpoint{0.307444in}{1.554908in}}%
\pgfpathlineto{\pgfqpoint{0.315092in}{1.566955in}}%
\pgfpathlineto{\pgfqpoint{0.315451in}{1.567505in}}%
\pgfpathlineto{\pgfqpoint{0.323052in}{1.579001in}}%
\pgfpathlineto{\pgfqpoint{0.327497in}{1.585561in}}%
\pgfpathlineto{\pgfqpoint{0.331274in}{1.591047in}}%
\pgfpathlineto{\pgfqpoint{0.339543in}{1.602778in}}%
\pgfpathlineto{\pgfqpoint{0.339769in}{1.603093in}}%
\pgfpathlineto{\pgfqpoint{0.348603in}{1.615139in}}%
\pgfpathlineto{\pgfqpoint{0.351589in}{1.619121in}}%
\pgfpathlineto{\pgfqpoint{0.357740in}{1.627185in}}%
\pgfpathlineto{\pgfqpoint{0.363635in}{1.634747in}}%
\pgfpathlineto{\pgfqpoint{0.367192in}{1.639231in}}%
\pgfpathlineto{\pgfqpoint{0.375681in}{1.649708in}}%
\pgfpathlineto{\pgfqpoint{0.376976in}{1.651277in}}%
\pgfpathlineto{\pgfqpoint{0.387120in}{1.663323in}}%
\pgfpathlineto{\pgfqpoint{0.387728in}{1.664030in}}%
\pgfpathlineto{\pgfqpoint{0.397649in}{1.675370in}}%
\pgfpathlineto{\pgfqpoint{0.399774in}{1.677749in}}%
\pgfpathlineto{\pgfqpoint{0.408562in}{1.687416in}}%
\pgfpathlineto{\pgfqpoint{0.411820in}{1.690929in}}%
\pgfpathlineto{\pgfqpoint{0.419878in}{1.699462in}}%
\pgfpathlineto{\pgfqpoint{0.423866in}{1.703603in}}%
\pgfpathlineto{\pgfqpoint{0.431622in}{1.711508in}}%
\pgfpathlineto{\pgfqpoint{0.435912in}{1.715798in}}%
\pgfpathlineto{\pgfqpoint{0.443817in}{1.723554in}}%
\pgfpathlineto{\pgfqpoint{0.447958in}{1.727542in}}%
\pgfpathlineto{\pgfqpoint{0.456491in}{1.735600in}}%
\pgfpathlineto{\pgfqpoint{0.460004in}{1.738858in}}%
\pgfpathlineto{\pgfqpoint{0.469671in}{1.747646in}}%
\pgfpathlineto{\pgfqpoint{0.472050in}{1.749770in}}%
\pgfpathlineto{\pgfqpoint{0.483390in}{1.759692in}}%
\pgfpathlineto{\pgfqpoint{0.484096in}{1.760300in}}%
\pgfpathlineto{\pgfqpoint{0.496143in}{1.770444in}}%
\pgfpathlineto{\pgfqpoint{0.497712in}{1.771738in}}%
\pgfpathlineto{\pgfqpoint{0.508189in}{1.780228in}}%
\pgfpathlineto{\pgfqpoint{0.512673in}{1.783785in}}%
\pgfpathlineto{\pgfqpoint{0.520235in}{1.789680in}}%
\pgfpathlineto{\pgfqpoint{0.528299in}{1.795831in}}%
\pgfpathlineto{\pgfqpoint{0.532281in}{1.798817in}}%
\pgfpathlineto{\pgfqpoint{0.544327in}{1.807651in}}%
\pgfpathlineto{\pgfqpoint{0.544642in}{1.807877in}}%
\pgfpathlineto{\pgfqpoint{0.556373in}{1.816146in}}%
\pgfpathlineto{\pgfqpoint{0.561858in}{1.819923in}}%
\pgfpathlineto{\pgfqpoint{0.568419in}{1.824367in}}%
\pgfpathlineto{\pgfqpoint{0.579914in}{1.831969in}}%
\pgfpathlineto{\pgfqpoint{0.580465in}{1.832328in}}%
\pgfpathlineto{\pgfqpoint{0.592511in}{1.839976in}}%
\pgfpathlineto{\pgfqpoint{0.599036in}{1.844015in}}%
\pgfpathlineto{\pgfqpoint{0.604558in}{1.847380in}}%
\pgfpathlineto{\pgfqpoint{0.616604in}{1.854532in}}%
\pgfpathlineto{\pgfqpoint{0.619249in}{1.856061in}}%
\pgfpathlineto{\pgfqpoint{0.628650in}{1.861413in}}%
\pgfpathlineto{\pgfqpoint{0.640696in}{1.868084in}}%
\pgfpathlineto{\pgfqpoint{0.640739in}{1.868107in}}%
\pgfpathlineto{\pgfqpoint{0.652742in}{1.874473in}}%
\pgfpathlineto{\pgfqpoint{0.663770in}{1.880154in}}%
\pgfpathlineto{\pgfqpoint{0.664788in}{1.880670in}}%
\pgfpathlineto{\pgfqpoint{0.676834in}{1.886604in}}%
\pgfpathlineto{\pgfqpoint{0.688551in}{1.892200in}}%
\pgfpathlineto{\pgfqpoint{0.688880in}{1.892355in}}%
\pgfpathlineto{\pgfqpoint{0.700927in}{1.897846in}}%
\pgfpathlineto{\pgfqpoint{0.712973in}{1.903161in}}%
\pgfpathlineto{\pgfqpoint{0.715517in}{1.904246in}}%
\pgfpathlineto{\pgfqpoint{0.725019in}{1.908239in}}%
\pgfpathlineto{\pgfqpoint{0.737065in}{1.913128in}}%
\pgfpathlineto{\pgfqpoint{0.745151in}{1.916292in}}%
\pgfpathlineto{\pgfqpoint{0.749111in}{1.917819in}}%
\pgfpathlineto{\pgfqpoint{0.761157in}{1.922293in}}%
\pgfpathlineto{\pgfqpoint{0.773203in}{1.926596in}}%
\pgfpathlineto{\pgfqpoint{0.778285in}{1.928338in}}%
\pgfpathlineto{\pgfqpoint{0.785249in}{1.930692in}}%
\pgfpathlineto{\pgfqpoint{0.797295in}{1.934594in}}%
\pgfpathlineto{\pgfqpoint{0.809342in}{1.938327in}}%
\pgfpathlineto{\pgfqpoint{0.816297in}{1.940384in}}%
\pgfpathlineto{\pgfqpoint{0.821388in}{1.941869in}}%
\pgfpathlineto{\pgfqpoint{0.833434in}{1.945215in}}%
\pgfpathlineto{\pgfqpoint{0.845480in}{1.948394in}}%
\pgfpathlineto{\pgfqpoint{0.857526in}{1.951405in}}%
\pgfpathlineto{\pgfqpoint{0.861866in}{1.952430in}}%
\pgfpathlineto{\pgfqpoint{0.869572in}{1.954225in}}%
\pgfpathlineto{\pgfqpoint{0.881618in}{1.956865in}}%
\pgfpathlineto{\pgfqpoint{0.893664in}{1.959340in}}%
\pgfpathlineto{\pgfqpoint{0.905710in}{1.961650in}}%
\pgfpathlineto{\pgfqpoint{0.917757in}{1.963796in}}%
\pgfpathlineto{\pgfqpoint{0.921897in}{1.964476in}}%
\pgfpathlineto{\pgfqpoint{0.929803in}{1.965758in}}%
\pgfpathlineto{\pgfqpoint{0.941849in}{1.967549in}}%
\pgfpathlineto{\pgfqpoint{0.953895in}{1.969177in}}%
\pgfpathlineto{\pgfqpoint{0.965941in}{1.970642in}}%
\pgfpathlineto{\pgfqpoint{0.977987in}{1.971944in}}%
\pgfpathlineto{\pgfqpoint{0.990033in}{1.973084in}}%
\pgfpathlineto{\pgfqpoint{1.002079in}{1.974060in}}%
\pgfpathlineto{\pgfqpoint{1.014126in}{1.974874in}}%
\pgfpathlineto{\pgfqpoint{1.026172in}{1.975525in}}%
\pgfpathlineto{\pgfqpoint{1.038218in}{1.976014in}}%
\pgfpathlineto{\pgfqpoint{1.050264in}{1.976339in}}%
\pgfpathlineto{\pgfqpoint{1.062310in}{1.976502in}}%
\pgfpathlineto{\pgfqpoint{1.074356in}{1.976502in}}%
\pgfpathlineto{\pgfqpoint{1.086402in}{1.976339in}}%
\pgfpathlineto{\pgfqpoint{1.098448in}{1.976014in}}%
\pgfpathlineto{\pgfqpoint{1.110494in}{1.975525in}}%
\pgfpathlineto{\pgfqpoint{1.122541in}{1.974874in}}%
\pgfpathlineto{\pgfqpoint{1.134587in}{1.974060in}}%
\pgfpathlineto{\pgfqpoint{1.146633in}{1.973084in}}%
\pgfpathlineto{\pgfqpoint{1.158679in}{1.971944in}}%
\pgfpathlineto{\pgfqpoint{1.170725in}{1.970642in}}%
\pgfpathlineto{\pgfqpoint{1.182771in}{1.969177in}}%
\pgfpathlineto{\pgfqpoint{1.194817in}{1.967549in}}%
\pgfpathlineto{\pgfqpoint{1.206863in}{1.965758in}}%
\pgfpathlineto{\pgfqpoint{1.214769in}{1.964476in}}%
\pgfpathlineto{\pgfqpoint{1.218909in}{1.963796in}}%
\pgfpathlineto{\pgfqpoint{1.230956in}{1.961650in}}%
\pgfpathlineto{\pgfqpoint{1.243002in}{1.959340in}}%
\pgfpathlineto{\pgfqpoint{1.255048in}{1.956865in}}%
\pgfpathlineto{\pgfqpoint{1.267094in}{1.954225in}}%
\pgfpathlineto{\pgfqpoint{1.274800in}{1.952430in}}%
\pgfpathlineto{\pgfqpoint{1.279140in}{1.951405in}}%
\pgfpathlineto{\pgfqpoint{1.291186in}{1.948394in}}%
\pgfpathlineto{\pgfqpoint{1.303232in}{1.945215in}}%
\pgfpathlineto{\pgfqpoint{1.315278in}{1.941869in}}%
\pgfpathlineto{\pgfqpoint{1.320369in}{1.940384in}}%
\pgfpathlineto{\pgfqpoint{1.327325in}{1.938327in}}%
\pgfpathlineto{\pgfqpoint{1.339371in}{1.934594in}}%
\pgfpathlineto{\pgfqpoint{1.351417in}{1.930692in}}%
\pgfpathlineto{\pgfqpoint{1.358381in}{1.928338in}}%
\pgfpathlineto{\pgfqpoint{1.363463in}{1.926596in}}%
\pgfpathlineto{\pgfqpoint{1.375509in}{1.922293in}}%
\pgfpathlineto{\pgfqpoint{1.387555in}{1.917819in}}%
\pgfpathlineto{\pgfqpoint{1.391515in}{1.916292in}}%
\pgfpathlineto{\pgfqpoint{1.399601in}{1.913128in}}%
\pgfpathlineto{\pgfqpoint{1.411647in}{1.908239in}}%
\pgfpathlineto{\pgfqpoint{1.421149in}{1.904246in}}%
\pgfpathlineto{\pgfqpoint{1.423693in}{1.903161in}}%
\pgfpathlineto{\pgfqpoint{1.435740in}{1.897846in}}%
\pgfpathlineto{\pgfqpoint{1.447786in}{1.892355in}}%
\pgfpathlineto{\pgfqpoint{1.448115in}{1.892200in}}%
\pgfpathlineto{\pgfqpoint{1.459832in}{1.886604in}}%
\pgfpathlineto{\pgfqpoint{1.471878in}{1.880670in}}%
\pgfpathlineto{\pgfqpoint{1.472897in}{1.880154in}}%
\pgfpathlineto{\pgfqpoint{1.483924in}{1.874473in}}%
\pgfpathlineto{\pgfqpoint{1.495927in}{1.868107in}}%
\pgfpathlineto{\pgfqpoint{1.495970in}{1.868084in}}%
\pgfpathlineto{\pgfqpoint{1.508016in}{1.861413in}}%
\pgfpathlineto{\pgfqpoint{1.517417in}{1.856061in}}%
\pgfpathlineto{\pgfqpoint{1.520062in}{1.854532in}}%
\pgfpathlineto{\pgfqpoint{1.532108in}{1.847380in}}%
\pgfpathlineto{\pgfqpoint{1.537630in}{1.844015in}}%
\pgfpathlineto{\pgfqpoint{1.544155in}{1.839976in}}%
\pgfpathlineto{\pgfqpoint{1.556201in}{1.832328in}}%
\pgfpathlineto{\pgfqpoint{1.556752in}{1.831969in}}%
\pgfpathlineto{\pgfqpoint{1.568247in}{1.824367in}}%
\pgfpathlineto{\pgfqpoint{1.574808in}{1.819923in}}%
\pgfpathlineto{\pgfqpoint{1.580293in}{1.816146in}}%
\pgfpathlineto{\pgfqpoint{1.592024in}{1.807877in}}%
\pgfpathlineto{\pgfqpoint{1.592339in}{1.807651in}}%
\pgfpathlineto{\pgfqpoint{1.604385in}{1.798817in}}%
\pgfpathlineto{\pgfqpoint{1.608367in}{1.795831in}}%
\pgfpathlineto{\pgfqpoint{1.616431in}{1.789680in}}%
\pgfpathlineto{\pgfqpoint{1.623993in}{1.783785in}}%
\pgfpathlineto{\pgfqpoint{1.628477in}{1.780228in}}%
\pgfpathlineto{\pgfqpoint{1.638954in}{1.771738in}}%
\pgfpathlineto{\pgfqpoint{1.640524in}{1.770444in}}%
\pgfpathlineto{\pgfqpoint{1.652570in}{1.760300in}}%
\pgfpathlineto{\pgfqpoint{1.653276in}{1.759692in}}%
\pgfpathlineto{\pgfqpoint{1.664616in}{1.749770in}}%
\pgfpathlineto{\pgfqpoint{1.666995in}{1.747646in}}%
\pgfpathlineto{\pgfqpoint{1.676662in}{1.738858in}}%
\pgfpathlineto{\pgfqpoint{1.680175in}{1.735600in}}%
\pgfpathlineto{\pgfqpoint{1.688708in}{1.727542in}}%
\pgfpathlineto{\pgfqpoint{1.692849in}{1.723554in}}%
\pgfpathlineto{\pgfqpoint{1.700754in}{1.715798in}}%
\pgfpathlineto{\pgfqpoint{1.705044in}{1.711508in}}%
\pgfpathlineto{\pgfqpoint{1.712800in}{1.703603in}}%
\pgfpathlineto{\pgfqpoint{1.716788in}{1.699462in}}%
\pgfpathlineto{\pgfqpoint{1.724846in}{1.690929in}}%
\pgfpathlineto{\pgfqpoint{1.728104in}{1.687416in}}%
\pgfpathlineto{\pgfqpoint{1.736892in}{1.677749in}}%
\pgfpathlineto{\pgfqpoint{1.739017in}{1.675370in}}%
\pgfpathlineto{\pgfqpoint{1.748939in}{1.664030in}}%
\pgfpathlineto{\pgfqpoint{1.749546in}{1.663323in}}%
\pgfpathlineto{\pgfqpoint{1.759690in}{1.651277in}}%
\pgfpathlineto{\pgfqpoint{1.760985in}{1.649708in}}%
\pgfpathlineto{\pgfqpoint{1.769474in}{1.639231in}}%
\pgfpathlineto{\pgfqpoint{1.773031in}{1.634747in}}%
\pgfpathlineto{\pgfqpoint{1.778926in}{1.627185in}}%
\pgfpathlineto{\pgfqpoint{1.785077in}{1.619121in}}%
\pgfpathlineto{\pgfqpoint{1.788063in}{1.615139in}}%
\pgfpathlineto{\pgfqpoint{1.796897in}{1.603093in}}%
\pgfpathlineto{\pgfqpoint{1.797123in}{1.602778in}}%
\pgfpathlineto{\pgfqpoint{1.805392in}{1.591047in}}%
\pgfpathlineto{\pgfqpoint{1.809169in}{1.585561in}}%
\pgfpathlineto{\pgfqpoint{1.813614in}{1.579001in}}%
\pgfpathlineto{\pgfqpoint{1.821215in}{1.567505in}}%
\pgfpathlineto{\pgfqpoint{1.821574in}{1.566955in}}%
\pgfpathlineto{\pgfqpoint{1.829222in}{1.554908in}}%
\pgfpathlineto{\pgfqpoint{1.833261in}{1.548383in}}%
\pgfpathlineto{\pgfqpoint{1.836626in}{1.542862in}}%
\pgfpathlineto{\pgfqpoint{1.843778in}{1.530816in}}%
\pgfpathlineto{\pgfqpoint{1.845307in}{1.528171in}}%
\pgfpathlineto{\pgfqpoint{1.850659in}{1.518770in}}%
\pgfpathlineto{\pgfqpoint{1.857330in}{1.506724in}}%
\pgfpathlineto{\pgfqpoint{1.857354in}{1.506681in}}%
\pgfpathlineto{\pgfqpoint{1.863719in}{1.494678in}}%
\pgfpathlineto{\pgfqpoint{1.869400in}{1.483650in}}%
\pgfpathlineto{\pgfqpoint{1.869917in}{1.482632in}}%
\pgfpathlineto{\pgfqpoint{1.875850in}{1.470586in}}%
\pgfpathlineto{\pgfqpoint{1.881446in}{1.458869in}}%
\pgfpathlineto{\pgfqpoint{1.881601in}{1.458539in}}%
\pgfpathlineto{\pgfqpoint{1.887092in}{1.446493in}}%
\pgfpathlineto{\pgfqpoint{1.892407in}{1.434447in}}%
\pgfpathlineto{\pgfqpoint{1.893492in}{1.431903in}}%
\pgfpathlineto{\pgfqpoint{1.897485in}{1.422401in}}%
\pgfpathlineto{\pgfqpoint{1.902374in}{1.410355in}}%
\pgfpathlineto{\pgfqpoint{1.905538in}{1.402269in}}%
\pgfpathlineto{\pgfqpoint{1.907065in}{1.398309in}}%
\pgfpathlineto{\pgfqpoint{1.911540in}{1.386263in}}%
\pgfpathlineto{\pgfqpoint{1.915842in}{1.374217in}}%
\pgfpathlineto{\pgfqpoint{1.917584in}{1.369135in}}%
\pgfpathlineto{\pgfqpoint{1.919938in}{1.362171in}}%
\pgfpathlineto{\pgfqpoint{1.923841in}{1.350124in}}%
\pgfpathlineto{\pgfqpoint{1.927573in}{1.338078in}}%
\pgfpathlineto{\pgfqpoint{1.929630in}{1.331123in}}%
\pgfpathlineto{\pgfqpoint{1.931115in}{1.326032in}}%
\pgfpathlineto{\pgfqpoint{1.934461in}{1.313986in}}%
\pgfpathlineto{\pgfqpoint{1.937640in}{1.301940in}}%
\pgfpathlineto{\pgfqpoint{1.940652in}{1.289894in}}%
\pgfpathlineto{\pgfqpoint{1.941676in}{1.285554in}}%
\pgfpathlineto{\pgfqpoint{1.943471in}{1.277848in}}%
\pgfpathlineto{\pgfqpoint{1.946111in}{1.265802in}}%
\pgfpathlineto{\pgfqpoint{1.948586in}{1.253756in}}%
\pgfpathlineto{\pgfqpoint{1.950897in}{1.241709in}}%
\pgfpathlineto{\pgfqpoint{1.953042in}{1.229663in}}%
\pgfpathlineto{\pgfqpoint{1.953723in}{1.225522in}}%
\pgfpathlineto{\pgfqpoint{1.955004in}{1.217617in}}%
\pgfpathlineto{\pgfqpoint{1.956795in}{1.205571in}}%
\pgfpathlineto{\pgfqpoint{1.958423in}{1.193525in}}%
\pgfpathlineto{\pgfqpoint{1.959888in}{1.181479in}}%
\pgfpathlineto{\pgfqpoint{1.961190in}{1.169433in}}%
\pgfpathlineto{\pgfqpoint{1.962330in}{1.157387in}}%
\pgfpathlineto{\pgfqpoint{1.963307in}{1.145340in}}%
\pgfpathlineto{\pgfqpoint{1.964120in}{1.133294in}}%
\pgfpathlineto{\pgfqpoint{1.964772in}{1.121248in}}%
\pgfpathlineto{\pgfqpoint{1.965260in}{1.109202in}}%
\pgfpathlineto{\pgfqpoint{1.965585in}{1.097156in}}%
\pgfpathlineto{\pgfqpoint{1.965748in}{1.085110in}}%
\pgfpathlineto{\pgfqpoint{1.965748in}{1.073064in}}%
\pgfpathlineto{\pgfqpoint{1.965585in}{1.061018in}}%
\pgfpathlineto{\pgfqpoint{1.965260in}{1.048972in}}%
\pgfpathlineto{\pgfqpoint{1.964772in}{1.036925in}}%
\pgfpathlineto{\pgfqpoint{1.964120in}{1.024879in}}%
\pgfpathlineto{\pgfqpoint{1.963307in}{1.012833in}}%
\pgfpathlineto{\pgfqpoint{1.962330in}{1.000787in}}%
\pgfpathlineto{\pgfqpoint{1.961190in}{0.988741in}}%
\pgfpathlineto{\pgfqpoint{1.959888in}{0.976695in}}%
\pgfpathlineto{\pgfqpoint{1.958423in}{0.964649in}}%
\pgfpathlineto{\pgfqpoint{1.956795in}{0.952603in}}%
\pgfpathlineto{\pgfqpoint{1.955004in}{0.940557in}}%
\pgfpathlineto{\pgfqpoint{1.953723in}{0.932651in}}%
\pgfpathlineto{\pgfqpoint{1.953042in}{0.928510in}}%
\pgfpathlineto{\pgfqpoint{1.950897in}{0.916464in}}%
\pgfpathlineto{\pgfqpoint{1.948586in}{0.904418in}}%
\pgfpathlineto{\pgfqpoint{1.946111in}{0.892372in}}%
\pgfpathlineto{\pgfqpoint{1.943471in}{0.880326in}}%
\pgfpathlineto{\pgfqpoint{1.941676in}{0.872620in}}%
\pgfpathlineto{\pgfqpoint{1.940652in}{0.868280in}}%
\pgfpathlineto{\pgfqpoint{1.937640in}{0.856234in}}%
\pgfpathlineto{\pgfqpoint{1.934461in}{0.844188in}}%
\pgfpathlineto{\pgfqpoint{1.931115in}{0.832141in}}%
\pgfpathlineto{\pgfqpoint{1.929630in}{0.827051in}}%
\pgfpathlineto{\pgfqpoint{1.927573in}{0.820095in}}%
\pgfpathlineto{\pgfqpoint{1.923841in}{0.808049in}}%
\pgfpathlineto{\pgfqpoint{1.919938in}{0.796003in}}%
\pgfpathlineto{\pgfqpoint{1.917584in}{0.789039in}}%
\pgfpathlineto{\pgfqpoint{1.915842in}{0.783957in}}%
\pgfpathlineto{\pgfqpoint{1.911540in}{0.771911in}}%
\pgfpathlineto{\pgfqpoint{1.907065in}{0.759865in}}%
\pgfpathlineto{\pgfqpoint{1.905538in}{0.755905in}}%
\pgfpathlineto{\pgfqpoint{1.902374in}{0.747819in}}%
\pgfpathlineto{\pgfqpoint{1.897485in}{0.735773in}}%
\pgfpathlineto{\pgfqpoint{1.893492in}{0.726271in}}%
\pgfpathlineto{\pgfqpoint{1.892407in}{0.723726in}}%
\pgfpathlineto{\pgfqpoint{1.887092in}{0.711680in}}%
\pgfpathlineto{\pgfqpoint{1.881601in}{0.699634in}}%
\pgfpathlineto{\pgfqpoint{1.881446in}{0.699305in}}%
\pgfpathlineto{\pgfqpoint{1.875850in}{0.687588in}}%
\pgfpathlineto{\pgfqpoint{1.869917in}{0.675542in}}%
\pgfpathlineto{\pgfqpoint{1.869400in}{0.674523in}}%
\pgfpathlineto{\pgfqpoint{1.863719in}{0.663496in}}%
\pgfpathlineto{\pgfqpoint{1.857354in}{0.651493in}}%
\pgfpathlineto{\pgfqpoint{1.857330in}{0.651450in}}%
\pgfpathlineto{\pgfqpoint{1.850659in}{0.639404in}}%
\pgfpathlineto{\pgfqpoint{1.845307in}{0.630003in}}%
\pgfpathlineto{\pgfqpoint{1.843778in}{0.627358in}}%
\pgfpathlineto{\pgfqpoint{1.836626in}{0.615311in}}%
\pgfpathlineto{\pgfqpoint{1.833261in}{0.609790in}}%
\pgfpathlineto{\pgfqpoint{1.829222in}{0.603265in}}%
\pgfpathlineto{\pgfqpoint{1.821574in}{0.591219in}}%
\pgfpathlineto{\pgfqpoint{1.821215in}{0.590668in}}%
\pgfpathlineto{\pgfqpoint{1.813614in}{0.579173in}}%
\pgfpathlineto{\pgfqpoint{1.809169in}{0.572612in}}%
\pgfpathlineto{\pgfqpoint{1.805392in}{0.567127in}}%
\pgfpathlineto{\pgfqpoint{1.797123in}{0.555396in}}%
\pgfpathlineto{\pgfqpoint{1.796897in}{0.555081in}}%
\pgfpathlineto{\pgfqpoint{1.788063in}{0.543035in}}%
\pgfpathlineto{\pgfqpoint{1.785077in}{0.539053in}}%
\pgfpathlineto{\pgfqpoint{1.778926in}{0.530989in}}%
\pgfpathlineto{\pgfqpoint{1.773031in}{0.523427in}}%
\pgfpathlineto{\pgfqpoint{1.769474in}{0.518942in}}%
\pgfpathlineto{\pgfqpoint{1.760985in}{0.508466in}}%
\pgfpathlineto{\pgfqpoint{1.759690in}{0.506896in}}%
\pgfpathlineto{\pgfqpoint{1.749546in}{0.494850in}}%
\pgfpathlineto{\pgfqpoint{1.748939in}{0.494143in}}%
\pgfpathlineto{\pgfqpoint{1.739017in}{0.482804in}}%
\pgfpathlineto{\pgfqpoint{1.736892in}{0.480425in}}%
\pgfpathlineto{\pgfqpoint{1.728104in}{0.470758in}}%
\pgfpathlineto{\pgfqpoint{1.724846in}{0.467245in}}%
\pgfpathlineto{\pgfqpoint{1.716788in}{0.458712in}}%
\pgfpathlineto{\pgfqpoint{1.712800in}{0.454571in}}%
\pgfpathlineto{\pgfqpoint{1.705044in}{0.446666in}}%
\pgfpathlineto{\pgfqpoint{1.700754in}{0.442376in}}%
\pgfpathlineto{\pgfqpoint{1.692849in}{0.434620in}}%
\pgfpathlineto{\pgfqpoint{1.688708in}{0.430632in}}%
\pgfpathlineto{\pgfqpoint{1.680175in}{0.422574in}}%
\pgfpathlineto{\pgfqpoint{1.676662in}{0.419316in}}%
\pgfpathlineto{\pgfqpoint{1.666995in}{0.410527in}}%
\pgfpathlineto{\pgfqpoint{1.664616in}{0.408403in}}%
\pgfpathlineto{\pgfqpoint{1.653276in}{0.398481in}}%
\pgfpathlineto{\pgfqpoint{1.652570in}{0.397874in}}%
\pgfpathlineto{\pgfqpoint{1.640524in}{0.387730in}}%
\pgfpathlineto{\pgfqpoint{1.638954in}{0.386435in}}%
\pgfpathlineto{\pgfqpoint{1.628477in}{0.377946in}}%
\pgfpathlineto{\pgfqpoint{1.623993in}{0.374389in}}%
\pgfpathlineto{\pgfqpoint{1.616431in}{0.368494in}}%
\pgfpathlineto{\pgfqpoint{1.608367in}{0.362343in}}%
\pgfpathlineto{\pgfqpoint{1.604385in}{0.359357in}}%
\pgfpathlineto{\pgfqpoint{1.592339in}{0.350523in}}%
\pgfpathlineto{\pgfqpoint{1.592024in}{0.350297in}}%
\pgfpathlineto{\pgfqpoint{1.580293in}{0.342028in}}%
\pgfpathlineto{\pgfqpoint{1.574808in}{0.338251in}}%
\pgfpathlineto{\pgfqpoint{1.568247in}{0.333806in}}%
\pgfpathlineto{\pgfqpoint{1.556752in}{0.326205in}}%
\pgfpathlineto{\pgfqpoint{1.556201in}{0.325846in}}%
\pgfpathlineto{\pgfqpoint{1.544155in}{0.318198in}}%
\pgfpathlineto{\pgfqpoint{1.537630in}{0.314159in}}%
\pgfpathlineto{\pgfqpoint{1.532108in}{0.310794in}}%
\pgfpathlineto{\pgfqpoint{1.520062in}{0.303642in}}%
\pgfpathlineto{\pgfqpoint{1.517417in}{0.302112in}}%
\pgfpathlineto{\pgfqpoint{1.508016in}{0.296761in}}%
\pgfpathlineto{\pgfqpoint{1.495970in}{0.290089in}}%
\pgfpathlineto{\pgfqpoint{1.495927in}{0.290066in}}%
\pgfpathlineto{\pgfqpoint{1.483924in}{0.283701in}}%
\pgfpathlineto{\pgfqpoint{1.472897in}{0.278020in}}%
\pgfpathlineto{\pgfqpoint{1.471878in}{0.277503in}}%
\pgfpathlineto{\pgfqpoint{1.459832in}{0.271570in}}%
\pgfpathlineto{\pgfqpoint{1.448115in}{0.265974in}}%
\pgfpathlineto{\pgfqpoint{1.447786in}{0.265819in}}%
\pgfpathlineto{\pgfqpoint{1.435740in}{0.260327in}}%
\pgfpathlineto{\pgfqpoint{1.423693in}{0.255013in}}%
\pgfpathlineto{\pgfqpoint{1.421149in}{0.253928in}}%
\pgfpathlineto{\pgfqpoint{1.411647in}{0.249934in}}%
\pgfpathlineto{\pgfqpoint{1.399601in}{0.245046in}}%
\pgfpathlineto{\pgfqpoint{1.391515in}{0.241882in}}%
\pgfpathlineto{\pgfqpoint{1.387555in}{0.240355in}}%
\pgfpathlineto{\pgfqpoint{1.375509in}{0.235880in}}%
\pgfpathlineto{\pgfqpoint{1.363463in}{0.231578in}}%
\pgfpathlineto{\pgfqpoint{1.358381in}{0.229836in}}%
\pgfpathlineto{\pgfqpoint{1.351417in}{0.227482in}}%
\pgfpathlineto{\pgfqpoint{1.339371in}{0.223579in}}%
\pgfpathlineto{\pgfqpoint{1.327325in}{0.219847in}}%
\pgfpathlineto{\pgfqpoint{1.320369in}{0.217790in}}%
\pgfpathlineto{\pgfqpoint{1.315278in}{0.216305in}}%
\pgfpathlineto{\pgfqpoint{1.303232in}{0.212959in}}%
\pgfpathlineto{\pgfqpoint{1.291186in}{0.209780in}}%
\pgfpathlineto{\pgfqpoint{1.279140in}{0.206768in}}%
\pgfpathlineto{\pgfqpoint{1.274800in}{0.205743in}}%
\pgfpathlineto{\pgfqpoint{1.267094in}{0.203949in}}%
\pgfpathlineto{\pgfqpoint{1.255048in}{0.201309in}}%
\pgfpathlineto{\pgfqpoint{1.243002in}{0.198833in}}%
\pgfpathlineto{\pgfqpoint{1.230956in}{0.196523in}}%
\pgfpathlineto{\pgfqpoint{1.218909in}{0.194378in}}%
\pgfpathlineto{\pgfqpoint{1.214769in}{0.193697in}}%
\pgfpathlineto{\pgfqpoint{1.206863in}{0.192415in}}%
\pgfpathlineto{\pgfqpoint{1.194817in}{0.190625in}}%
\pgfpathlineto{\pgfqpoint{1.182771in}{0.188997in}}%
\pgfpathlineto{\pgfqpoint{1.170725in}{0.187532in}}%
\pgfpathlineto{\pgfqpoint{1.158679in}{0.186230in}}%
\pgfpathlineto{\pgfqpoint{1.146633in}{0.185090in}}%
\pgfpathlineto{\pgfqpoint{1.134587in}{0.184113in}}%
\pgfpathlineto{\pgfqpoint{1.122541in}{0.183299in}}%
\pgfpathlineto{\pgfqpoint{1.110494in}{0.182648in}}%
\pgfpathlineto{\pgfqpoint{1.098448in}{0.182160in}}%
\pgfpathlineto{\pgfqpoint{1.086402in}{0.181834in}}%
\pgfpathlineto{\pgfqpoint{1.074356in}{0.181672in}}%
\pgfpathlineto{\pgfqpoint{1.062310in}{0.181672in}}%
\pgfpathlineto{\pgfqpoint{1.050264in}{0.181834in}}%
\pgfpathlineto{\pgfqpoint{1.038218in}{0.182160in}}%
\pgfpathlineto{\pgfqpoint{1.026172in}{0.182648in}}%
\pgfpathlineto{\pgfqpoint{1.014126in}{0.183299in}}%
\pgfpathlineto{\pgfqpoint{1.002079in}{0.184113in}}%
\pgfpathlineto{\pgfqpoint{0.990033in}{0.185090in}}%
\pgfpathlineto{\pgfqpoint{0.977987in}{0.186230in}}%
\pgfpathlineto{\pgfqpoint{0.965941in}{0.187532in}}%
\pgfpathlineto{\pgfqpoint{0.953895in}{0.188997in}}%
\pgfpathlineto{\pgfqpoint{0.941849in}{0.190625in}}%
\pgfpathlineto{\pgfqpoint{0.929803in}{0.192415in}}%
\pgfpathclose%
\pgfusepath{fill}%
\end{pgfscope}%
\begin{pgfscope}%
\pgfpathrectangle{\pgfqpoint{0.135000in}{0.145754in}}{\pgfqpoint{1.866666in}{1.866666in}} %
\pgfusepath{clip}%
\pgfsetbuttcap%
\pgfsetroundjoin%
\definecolor{currentfill}{rgb}{1.000000,1.000000,1.000000}%
\pgfsetfillcolor{currentfill}%
\pgfsetlinewidth{0.000000pt}%
\definecolor{currentstroke}{rgb}{0.000000,0.000000,0.000000}%
\pgfsetstrokecolor{currentstroke}%
\pgfsetdash{}{0pt}%
\pgfpathmoveto{\pgfqpoint{0.929803in}{0.192415in}}%
\pgfpathlineto{\pgfqpoint{0.941849in}{0.190625in}}%
\pgfpathlineto{\pgfqpoint{0.953895in}{0.188997in}}%
\pgfpathlineto{\pgfqpoint{0.965941in}{0.187532in}}%
\pgfpathlineto{\pgfqpoint{0.977987in}{0.186230in}}%
\pgfpathlineto{\pgfqpoint{0.990033in}{0.185090in}}%
\pgfpathlineto{\pgfqpoint{1.002079in}{0.184113in}}%
\pgfpathlineto{\pgfqpoint{1.014126in}{0.183299in}}%
\pgfpathlineto{\pgfqpoint{1.026172in}{0.182648in}}%
\pgfpathlineto{\pgfqpoint{1.038218in}{0.182160in}}%
\pgfpathlineto{\pgfqpoint{1.050264in}{0.181834in}}%
\pgfpathlineto{\pgfqpoint{1.062310in}{0.181672in}}%
\pgfpathlineto{\pgfqpoint{1.074356in}{0.181672in}}%
\pgfpathlineto{\pgfqpoint{1.086402in}{0.181834in}}%
\pgfpathlineto{\pgfqpoint{1.098448in}{0.182160in}}%
\pgfpathlineto{\pgfqpoint{1.110494in}{0.182648in}}%
\pgfpathlineto{\pgfqpoint{1.122541in}{0.183299in}}%
\pgfpathlineto{\pgfqpoint{1.134587in}{0.184113in}}%
\pgfpathlineto{\pgfqpoint{1.146633in}{0.185090in}}%
\pgfpathlineto{\pgfqpoint{1.158679in}{0.186230in}}%
\pgfpathlineto{\pgfqpoint{1.170725in}{0.187532in}}%
\pgfpathlineto{\pgfqpoint{1.182771in}{0.188997in}}%
\pgfpathlineto{\pgfqpoint{1.194817in}{0.190625in}}%
\pgfpathlineto{\pgfqpoint{1.206863in}{0.192415in}}%
\pgfpathlineto{\pgfqpoint{1.214769in}{0.193697in}}%
\pgfpathlineto{\pgfqpoint{1.218909in}{0.194378in}}%
\pgfpathlineto{\pgfqpoint{1.230956in}{0.196523in}}%
\pgfpathlineto{\pgfqpoint{1.243002in}{0.198833in}}%
\pgfpathlineto{\pgfqpoint{1.255048in}{0.201309in}}%
\pgfpathlineto{\pgfqpoint{1.267094in}{0.203949in}}%
\pgfpathlineto{\pgfqpoint{1.274800in}{0.205743in}}%
\pgfpathlineto{\pgfqpoint{1.279140in}{0.206768in}}%
\pgfpathlineto{\pgfqpoint{1.291186in}{0.209780in}}%
\pgfpathlineto{\pgfqpoint{1.303232in}{0.212959in}}%
\pgfpathlineto{\pgfqpoint{1.315278in}{0.216305in}}%
\pgfpathlineto{\pgfqpoint{1.320369in}{0.217790in}}%
\pgfpathlineto{\pgfqpoint{1.327325in}{0.219847in}}%
\pgfpathlineto{\pgfqpoint{1.339371in}{0.223579in}}%
\pgfpathlineto{\pgfqpoint{1.351417in}{0.227482in}}%
\pgfpathlineto{\pgfqpoint{1.358381in}{0.229836in}}%
\pgfpathlineto{\pgfqpoint{1.363463in}{0.231578in}}%
\pgfpathlineto{\pgfqpoint{1.375509in}{0.235880in}}%
\pgfpathlineto{\pgfqpoint{1.387555in}{0.240355in}}%
\pgfpathlineto{\pgfqpoint{1.391515in}{0.241882in}}%
\pgfpathlineto{\pgfqpoint{1.399601in}{0.245046in}}%
\pgfpathlineto{\pgfqpoint{1.411647in}{0.249934in}}%
\pgfpathlineto{\pgfqpoint{1.421149in}{0.253928in}}%
\pgfpathlineto{\pgfqpoint{1.423693in}{0.255013in}}%
\pgfpathlineto{\pgfqpoint{1.435740in}{0.260327in}}%
\pgfpathlineto{\pgfqpoint{1.447786in}{0.265819in}}%
\pgfpathlineto{\pgfqpoint{1.448115in}{0.265974in}}%
\pgfpathlineto{\pgfqpoint{1.459832in}{0.271570in}}%
\pgfpathlineto{\pgfqpoint{1.471878in}{0.277503in}}%
\pgfpathlineto{\pgfqpoint{1.472897in}{0.278020in}}%
\pgfpathlineto{\pgfqpoint{1.483924in}{0.283701in}}%
\pgfpathlineto{\pgfqpoint{1.495927in}{0.290066in}}%
\pgfpathlineto{\pgfqpoint{1.495970in}{0.290089in}}%
\pgfpathlineto{\pgfqpoint{1.508016in}{0.296761in}}%
\pgfpathlineto{\pgfqpoint{1.517417in}{0.302112in}}%
\pgfpathlineto{\pgfqpoint{1.520062in}{0.303642in}}%
\pgfpathlineto{\pgfqpoint{1.532108in}{0.310794in}}%
\pgfpathlineto{\pgfqpoint{1.537630in}{0.314159in}}%
\pgfpathlineto{\pgfqpoint{1.544155in}{0.318198in}}%
\pgfpathlineto{\pgfqpoint{1.556201in}{0.325846in}}%
\pgfpathlineto{\pgfqpoint{1.556752in}{0.326205in}}%
\pgfpathlineto{\pgfqpoint{1.568247in}{0.333806in}}%
\pgfpathlineto{\pgfqpoint{1.574808in}{0.338251in}}%
\pgfpathlineto{\pgfqpoint{1.580293in}{0.342028in}}%
\pgfpathlineto{\pgfqpoint{1.592024in}{0.350297in}}%
\pgfpathlineto{\pgfqpoint{1.592339in}{0.350523in}}%
\pgfpathlineto{\pgfqpoint{1.604385in}{0.359357in}}%
\pgfpathlineto{\pgfqpoint{1.608367in}{0.362343in}}%
\pgfpathlineto{\pgfqpoint{1.616431in}{0.368494in}}%
\pgfpathlineto{\pgfqpoint{1.623993in}{0.374389in}}%
\pgfpathlineto{\pgfqpoint{1.628477in}{0.377946in}}%
\pgfpathlineto{\pgfqpoint{1.638954in}{0.386435in}}%
\pgfpathlineto{\pgfqpoint{1.640524in}{0.387730in}}%
\pgfpathlineto{\pgfqpoint{1.652570in}{0.397874in}}%
\pgfpathlineto{\pgfqpoint{1.653276in}{0.398481in}}%
\pgfpathlineto{\pgfqpoint{1.664616in}{0.408403in}}%
\pgfpathlineto{\pgfqpoint{1.666995in}{0.410527in}}%
\pgfpathlineto{\pgfqpoint{1.676662in}{0.419316in}}%
\pgfpathlineto{\pgfqpoint{1.680175in}{0.422574in}}%
\pgfpathlineto{\pgfqpoint{1.688708in}{0.430632in}}%
\pgfpathlineto{\pgfqpoint{1.692849in}{0.434620in}}%
\pgfpathlineto{\pgfqpoint{1.700754in}{0.442376in}}%
\pgfpathlineto{\pgfqpoint{1.705044in}{0.446666in}}%
\pgfpathlineto{\pgfqpoint{1.712800in}{0.454571in}}%
\pgfpathlineto{\pgfqpoint{1.716788in}{0.458712in}}%
\pgfpathlineto{\pgfqpoint{1.724846in}{0.467245in}}%
\pgfpathlineto{\pgfqpoint{1.728104in}{0.470758in}}%
\pgfpathlineto{\pgfqpoint{1.736892in}{0.480425in}}%
\pgfpathlineto{\pgfqpoint{1.739017in}{0.482804in}}%
\pgfpathlineto{\pgfqpoint{1.748939in}{0.494143in}}%
\pgfpathlineto{\pgfqpoint{1.749546in}{0.494850in}}%
\pgfpathlineto{\pgfqpoint{1.759690in}{0.506896in}}%
\pgfpathlineto{\pgfqpoint{1.760985in}{0.508466in}}%
\pgfpathlineto{\pgfqpoint{1.769474in}{0.518942in}}%
\pgfpathlineto{\pgfqpoint{1.773031in}{0.523427in}}%
\pgfpathlineto{\pgfqpoint{1.778926in}{0.530989in}}%
\pgfpathlineto{\pgfqpoint{1.785077in}{0.539053in}}%
\pgfpathlineto{\pgfqpoint{1.788063in}{0.543035in}}%
\pgfpathlineto{\pgfqpoint{1.796897in}{0.555081in}}%
\pgfpathlineto{\pgfqpoint{1.797123in}{0.555396in}}%
\pgfpathlineto{\pgfqpoint{1.805392in}{0.567127in}}%
\pgfpathlineto{\pgfqpoint{1.809169in}{0.572612in}}%
\pgfpathlineto{\pgfqpoint{1.813614in}{0.579173in}}%
\pgfpathlineto{\pgfqpoint{1.821215in}{0.590668in}}%
\pgfpathlineto{\pgfqpoint{1.821574in}{0.591219in}}%
\pgfpathlineto{\pgfqpoint{1.829222in}{0.603265in}}%
\pgfpathlineto{\pgfqpoint{1.833261in}{0.609790in}}%
\pgfpathlineto{\pgfqpoint{1.836626in}{0.615311in}}%
\pgfpathlineto{\pgfqpoint{1.843778in}{0.627358in}}%
\pgfpathlineto{\pgfqpoint{1.845307in}{0.630003in}}%
\pgfpathlineto{\pgfqpoint{1.850659in}{0.639404in}}%
\pgfpathlineto{\pgfqpoint{1.857330in}{0.651450in}}%
\pgfpathlineto{\pgfqpoint{1.857354in}{0.651493in}}%
\pgfpathlineto{\pgfqpoint{1.863719in}{0.663496in}}%
\pgfpathlineto{\pgfqpoint{1.869400in}{0.674523in}}%
\pgfpathlineto{\pgfqpoint{1.869917in}{0.675542in}}%
\pgfpathlineto{\pgfqpoint{1.875850in}{0.687588in}}%
\pgfpathlineto{\pgfqpoint{1.881446in}{0.699305in}}%
\pgfpathlineto{\pgfqpoint{1.881601in}{0.699634in}}%
\pgfpathlineto{\pgfqpoint{1.887092in}{0.711680in}}%
\pgfpathlineto{\pgfqpoint{1.892407in}{0.723726in}}%
\pgfpathlineto{\pgfqpoint{1.893492in}{0.726271in}}%
\pgfpathlineto{\pgfqpoint{1.897485in}{0.735773in}}%
\pgfpathlineto{\pgfqpoint{1.902374in}{0.747819in}}%
\pgfpathlineto{\pgfqpoint{1.905538in}{0.755905in}}%
\pgfpathlineto{\pgfqpoint{1.907065in}{0.759865in}}%
\pgfpathlineto{\pgfqpoint{1.911540in}{0.771911in}}%
\pgfpathlineto{\pgfqpoint{1.915842in}{0.783957in}}%
\pgfpathlineto{\pgfqpoint{1.917584in}{0.789039in}}%
\pgfpathlineto{\pgfqpoint{1.919938in}{0.796003in}}%
\pgfpathlineto{\pgfqpoint{1.923841in}{0.808049in}}%
\pgfpathlineto{\pgfqpoint{1.927573in}{0.820095in}}%
\pgfpathlineto{\pgfqpoint{1.929630in}{0.827051in}}%
\pgfpathlineto{\pgfqpoint{1.931115in}{0.832141in}}%
\pgfpathlineto{\pgfqpoint{1.934461in}{0.844188in}}%
\pgfpathlineto{\pgfqpoint{1.937640in}{0.856234in}}%
\pgfpathlineto{\pgfqpoint{1.940652in}{0.868280in}}%
\pgfpathlineto{\pgfqpoint{1.941676in}{0.872620in}}%
\pgfpathlineto{\pgfqpoint{1.943471in}{0.880326in}}%
\pgfpathlineto{\pgfqpoint{1.946111in}{0.892372in}}%
\pgfpathlineto{\pgfqpoint{1.948586in}{0.904418in}}%
\pgfpathlineto{\pgfqpoint{1.950897in}{0.916464in}}%
\pgfpathlineto{\pgfqpoint{1.953042in}{0.928510in}}%
\pgfpathlineto{\pgfqpoint{1.953723in}{0.932651in}}%
\pgfpathlineto{\pgfqpoint{1.955004in}{0.940557in}}%
\pgfpathlineto{\pgfqpoint{1.956795in}{0.952603in}}%
\pgfpathlineto{\pgfqpoint{1.958423in}{0.964649in}}%
\pgfpathlineto{\pgfqpoint{1.959888in}{0.976695in}}%
\pgfpathlineto{\pgfqpoint{1.961190in}{0.988741in}}%
\pgfpathlineto{\pgfqpoint{1.962330in}{1.000787in}}%
\pgfpathlineto{\pgfqpoint{1.963307in}{1.012833in}}%
\pgfpathlineto{\pgfqpoint{1.964120in}{1.024879in}}%
\pgfpathlineto{\pgfqpoint{1.964772in}{1.036925in}}%
\pgfpathlineto{\pgfqpoint{1.965260in}{1.048972in}}%
\pgfpathlineto{\pgfqpoint{1.965585in}{1.061018in}}%
\pgfpathlineto{\pgfqpoint{1.965748in}{1.073064in}}%
\pgfpathlineto{\pgfqpoint{1.965748in}{1.085110in}}%
\pgfpathlineto{\pgfqpoint{1.965585in}{1.097156in}}%
\pgfpathlineto{\pgfqpoint{1.965260in}{1.109202in}}%
\pgfpathlineto{\pgfqpoint{1.964772in}{1.121248in}}%
\pgfpathlineto{\pgfqpoint{1.964120in}{1.133294in}}%
\pgfpathlineto{\pgfqpoint{1.963307in}{1.145340in}}%
\pgfpathlineto{\pgfqpoint{1.962330in}{1.157387in}}%
\pgfpathlineto{\pgfqpoint{1.961190in}{1.169433in}}%
\pgfpathlineto{\pgfqpoint{1.959888in}{1.181479in}}%
\pgfpathlineto{\pgfqpoint{1.958423in}{1.193525in}}%
\pgfpathlineto{\pgfqpoint{1.956795in}{1.205571in}}%
\pgfpathlineto{\pgfqpoint{1.955004in}{1.217617in}}%
\pgfpathlineto{\pgfqpoint{1.953723in}{1.225522in}}%
\pgfpathlineto{\pgfqpoint{1.953042in}{1.229663in}}%
\pgfpathlineto{\pgfqpoint{1.950897in}{1.241709in}}%
\pgfpathlineto{\pgfqpoint{1.948586in}{1.253756in}}%
\pgfpathlineto{\pgfqpoint{1.946111in}{1.265802in}}%
\pgfpathlineto{\pgfqpoint{1.943471in}{1.277848in}}%
\pgfpathlineto{\pgfqpoint{1.941676in}{1.285554in}}%
\pgfpathlineto{\pgfqpoint{1.940652in}{1.289894in}}%
\pgfpathlineto{\pgfqpoint{1.937640in}{1.301940in}}%
\pgfpathlineto{\pgfqpoint{1.934461in}{1.313986in}}%
\pgfpathlineto{\pgfqpoint{1.931115in}{1.326032in}}%
\pgfpathlineto{\pgfqpoint{1.929630in}{1.331123in}}%
\pgfpathlineto{\pgfqpoint{1.927573in}{1.338078in}}%
\pgfpathlineto{\pgfqpoint{1.923841in}{1.350124in}}%
\pgfpathlineto{\pgfqpoint{1.919938in}{1.362171in}}%
\pgfpathlineto{\pgfqpoint{1.917584in}{1.369135in}}%
\pgfpathlineto{\pgfqpoint{1.915842in}{1.374217in}}%
\pgfpathlineto{\pgfqpoint{1.911540in}{1.386263in}}%
\pgfpathlineto{\pgfqpoint{1.907065in}{1.398309in}}%
\pgfpathlineto{\pgfqpoint{1.905538in}{1.402269in}}%
\pgfpathlineto{\pgfqpoint{1.902374in}{1.410355in}}%
\pgfpathlineto{\pgfqpoint{1.897485in}{1.422401in}}%
\pgfpathlineto{\pgfqpoint{1.893492in}{1.431903in}}%
\pgfpathlineto{\pgfqpoint{1.892407in}{1.434447in}}%
\pgfpathlineto{\pgfqpoint{1.887092in}{1.446493in}}%
\pgfpathlineto{\pgfqpoint{1.881601in}{1.458539in}}%
\pgfpathlineto{\pgfqpoint{1.881446in}{1.458869in}}%
\pgfpathlineto{\pgfqpoint{1.875850in}{1.470586in}}%
\pgfpathlineto{\pgfqpoint{1.869917in}{1.482632in}}%
\pgfpathlineto{\pgfqpoint{1.869400in}{1.483650in}}%
\pgfpathlineto{\pgfqpoint{1.863719in}{1.494678in}}%
\pgfpathlineto{\pgfqpoint{1.857354in}{1.506681in}}%
\pgfpathlineto{\pgfqpoint{1.857330in}{1.506724in}}%
\pgfpathlineto{\pgfqpoint{1.850659in}{1.518770in}}%
\pgfpathlineto{\pgfqpoint{1.845307in}{1.528171in}}%
\pgfpathlineto{\pgfqpoint{1.843778in}{1.530816in}}%
\pgfpathlineto{\pgfqpoint{1.836626in}{1.542862in}}%
\pgfpathlineto{\pgfqpoint{1.833261in}{1.548383in}}%
\pgfpathlineto{\pgfqpoint{1.829222in}{1.554908in}}%
\pgfpathlineto{\pgfqpoint{1.821574in}{1.566955in}}%
\pgfpathlineto{\pgfqpoint{1.821215in}{1.567505in}}%
\pgfpathlineto{\pgfqpoint{1.813614in}{1.579001in}}%
\pgfpathlineto{\pgfqpoint{1.809169in}{1.585561in}}%
\pgfpathlineto{\pgfqpoint{1.805392in}{1.591047in}}%
\pgfpathlineto{\pgfqpoint{1.797123in}{1.602778in}}%
\pgfpathlineto{\pgfqpoint{1.796897in}{1.603093in}}%
\pgfpathlineto{\pgfqpoint{1.788063in}{1.615139in}}%
\pgfpathlineto{\pgfqpoint{1.785077in}{1.619121in}}%
\pgfpathlineto{\pgfqpoint{1.778926in}{1.627185in}}%
\pgfpathlineto{\pgfqpoint{1.773031in}{1.634747in}}%
\pgfpathlineto{\pgfqpoint{1.769474in}{1.639231in}}%
\pgfpathlineto{\pgfqpoint{1.760985in}{1.649708in}}%
\pgfpathlineto{\pgfqpoint{1.759690in}{1.651277in}}%
\pgfpathlineto{\pgfqpoint{1.749546in}{1.663323in}}%
\pgfpathlineto{\pgfqpoint{1.748939in}{1.664030in}}%
\pgfpathlineto{\pgfqpoint{1.739017in}{1.675370in}}%
\pgfpathlineto{\pgfqpoint{1.736892in}{1.677749in}}%
\pgfpathlineto{\pgfqpoint{1.728104in}{1.687416in}}%
\pgfpathlineto{\pgfqpoint{1.724846in}{1.690929in}}%
\pgfpathlineto{\pgfqpoint{1.716788in}{1.699462in}}%
\pgfpathlineto{\pgfqpoint{1.712800in}{1.703603in}}%
\pgfpathlineto{\pgfqpoint{1.705044in}{1.711508in}}%
\pgfpathlineto{\pgfqpoint{1.700754in}{1.715798in}}%
\pgfpathlineto{\pgfqpoint{1.692849in}{1.723554in}}%
\pgfpathlineto{\pgfqpoint{1.688708in}{1.727542in}}%
\pgfpathlineto{\pgfqpoint{1.680175in}{1.735600in}}%
\pgfpathlineto{\pgfqpoint{1.676662in}{1.738858in}}%
\pgfpathlineto{\pgfqpoint{1.666995in}{1.747646in}}%
\pgfpathlineto{\pgfqpoint{1.664616in}{1.749770in}}%
\pgfpathlineto{\pgfqpoint{1.653276in}{1.759692in}}%
\pgfpathlineto{\pgfqpoint{1.652570in}{1.760300in}}%
\pgfpathlineto{\pgfqpoint{1.640524in}{1.770444in}}%
\pgfpathlineto{\pgfqpoint{1.638954in}{1.771738in}}%
\pgfpathlineto{\pgfqpoint{1.628477in}{1.780228in}}%
\pgfpathlineto{\pgfqpoint{1.623993in}{1.783785in}}%
\pgfpathlineto{\pgfqpoint{1.616431in}{1.789680in}}%
\pgfpathlineto{\pgfqpoint{1.608367in}{1.795831in}}%
\pgfpathlineto{\pgfqpoint{1.604385in}{1.798817in}}%
\pgfpathlineto{\pgfqpoint{1.592339in}{1.807651in}}%
\pgfpathlineto{\pgfqpoint{1.592024in}{1.807877in}}%
\pgfpathlineto{\pgfqpoint{1.580293in}{1.816146in}}%
\pgfpathlineto{\pgfqpoint{1.574808in}{1.819923in}}%
\pgfpathlineto{\pgfqpoint{1.568247in}{1.824367in}}%
\pgfpathlineto{\pgfqpoint{1.556752in}{1.831969in}}%
\pgfpathlineto{\pgfqpoint{1.556201in}{1.832328in}}%
\pgfpathlineto{\pgfqpoint{1.544155in}{1.839976in}}%
\pgfpathlineto{\pgfqpoint{1.537630in}{1.844015in}}%
\pgfpathlineto{\pgfqpoint{1.532108in}{1.847380in}}%
\pgfpathlineto{\pgfqpoint{1.520062in}{1.854532in}}%
\pgfpathlineto{\pgfqpoint{1.517417in}{1.856061in}}%
\pgfpathlineto{\pgfqpoint{1.508016in}{1.861413in}}%
\pgfpathlineto{\pgfqpoint{1.495970in}{1.868084in}}%
\pgfpathlineto{\pgfqpoint{1.495927in}{1.868107in}}%
\pgfpathlineto{\pgfqpoint{1.483924in}{1.874473in}}%
\pgfpathlineto{\pgfqpoint{1.472897in}{1.880154in}}%
\pgfpathlineto{\pgfqpoint{1.471878in}{1.880670in}}%
\pgfpathlineto{\pgfqpoint{1.459832in}{1.886604in}}%
\pgfpathlineto{\pgfqpoint{1.448115in}{1.892200in}}%
\pgfpathlineto{\pgfqpoint{1.447786in}{1.892355in}}%
\pgfpathlineto{\pgfqpoint{1.435740in}{1.897846in}}%
\pgfpathlineto{\pgfqpoint{1.423693in}{1.903161in}}%
\pgfpathlineto{\pgfqpoint{1.421149in}{1.904246in}}%
\pgfpathlineto{\pgfqpoint{1.411647in}{1.908239in}}%
\pgfpathlineto{\pgfqpoint{1.399601in}{1.913128in}}%
\pgfpathlineto{\pgfqpoint{1.391515in}{1.916292in}}%
\pgfpathlineto{\pgfqpoint{1.387555in}{1.917819in}}%
\pgfpathlineto{\pgfqpoint{1.375509in}{1.922293in}}%
\pgfpathlineto{\pgfqpoint{1.363463in}{1.926596in}}%
\pgfpathlineto{\pgfqpoint{1.358381in}{1.928338in}}%
\pgfpathlineto{\pgfqpoint{1.351417in}{1.930692in}}%
\pgfpathlineto{\pgfqpoint{1.339371in}{1.934594in}}%
\pgfpathlineto{\pgfqpoint{1.327325in}{1.938327in}}%
\pgfpathlineto{\pgfqpoint{1.320369in}{1.940384in}}%
\pgfpathlineto{\pgfqpoint{1.315278in}{1.941869in}}%
\pgfpathlineto{\pgfqpoint{1.303232in}{1.945215in}}%
\pgfpathlineto{\pgfqpoint{1.291186in}{1.948394in}}%
\pgfpathlineto{\pgfqpoint{1.279140in}{1.951405in}}%
\pgfpathlineto{\pgfqpoint{1.274800in}{1.952430in}}%
\pgfpathlineto{\pgfqpoint{1.267094in}{1.954225in}}%
\pgfpathlineto{\pgfqpoint{1.255048in}{1.956865in}}%
\pgfpathlineto{\pgfqpoint{1.243002in}{1.959340in}}%
\pgfpathlineto{\pgfqpoint{1.230956in}{1.961650in}}%
\pgfpathlineto{\pgfqpoint{1.218909in}{1.963796in}}%
\pgfpathlineto{\pgfqpoint{1.214769in}{1.964476in}}%
\pgfpathlineto{\pgfqpoint{1.206863in}{1.965758in}}%
\pgfpathlineto{\pgfqpoint{1.194817in}{1.967549in}}%
\pgfpathlineto{\pgfqpoint{1.182771in}{1.969177in}}%
\pgfpathlineto{\pgfqpoint{1.170725in}{1.970642in}}%
\pgfpathlineto{\pgfqpoint{1.158679in}{1.971944in}}%
\pgfpathlineto{\pgfqpoint{1.146633in}{1.973084in}}%
\pgfpathlineto{\pgfqpoint{1.134587in}{1.974060in}}%
\pgfpathlineto{\pgfqpoint{1.122541in}{1.974874in}}%
\pgfpathlineto{\pgfqpoint{1.110494in}{1.975525in}}%
\pgfpathlineto{\pgfqpoint{1.098448in}{1.976014in}}%
\pgfpathlineto{\pgfqpoint{1.086402in}{1.976339in}}%
\pgfpathlineto{\pgfqpoint{1.074356in}{1.976502in}}%
\pgfpathlineto{\pgfqpoint{1.062310in}{1.976502in}}%
\pgfpathlineto{\pgfqpoint{1.050264in}{1.976339in}}%
\pgfpathlineto{\pgfqpoint{1.038218in}{1.976014in}}%
\pgfpathlineto{\pgfqpoint{1.026172in}{1.975525in}}%
\pgfpathlineto{\pgfqpoint{1.014126in}{1.974874in}}%
\pgfpathlineto{\pgfqpoint{1.002079in}{1.974060in}}%
\pgfpathlineto{\pgfqpoint{0.990033in}{1.973084in}}%
\pgfpathlineto{\pgfqpoint{0.977987in}{1.971944in}}%
\pgfpathlineto{\pgfqpoint{0.965941in}{1.970642in}}%
\pgfpathlineto{\pgfqpoint{0.953895in}{1.969177in}}%
\pgfpathlineto{\pgfqpoint{0.941849in}{1.967549in}}%
\pgfpathlineto{\pgfqpoint{0.929803in}{1.965758in}}%
\pgfpathlineto{\pgfqpoint{0.921897in}{1.964476in}}%
\pgfpathlineto{\pgfqpoint{0.917757in}{1.963796in}}%
\pgfpathlineto{\pgfqpoint{0.905710in}{1.961650in}}%
\pgfpathlineto{\pgfqpoint{0.893664in}{1.959340in}}%
\pgfpathlineto{\pgfqpoint{0.881618in}{1.956865in}}%
\pgfpathlineto{\pgfqpoint{0.869572in}{1.954225in}}%
\pgfpathlineto{\pgfqpoint{0.861866in}{1.952430in}}%
\pgfpathlineto{\pgfqpoint{0.857526in}{1.951405in}}%
\pgfpathlineto{\pgfqpoint{0.845480in}{1.948394in}}%
\pgfpathlineto{\pgfqpoint{0.833434in}{1.945215in}}%
\pgfpathlineto{\pgfqpoint{0.821388in}{1.941869in}}%
\pgfpathlineto{\pgfqpoint{0.816297in}{1.940384in}}%
\pgfpathlineto{\pgfqpoint{0.809342in}{1.938327in}}%
\pgfpathlineto{\pgfqpoint{0.797295in}{1.934594in}}%
\pgfpathlineto{\pgfqpoint{0.785249in}{1.930692in}}%
\pgfpathlineto{\pgfqpoint{0.778285in}{1.928338in}}%
\pgfpathlineto{\pgfqpoint{0.773203in}{1.926596in}}%
\pgfpathlineto{\pgfqpoint{0.761157in}{1.922293in}}%
\pgfpathlineto{\pgfqpoint{0.749111in}{1.917819in}}%
\pgfpathlineto{\pgfqpoint{0.745151in}{1.916292in}}%
\pgfpathlineto{\pgfqpoint{0.737065in}{1.913128in}}%
\pgfpathlineto{\pgfqpoint{0.725019in}{1.908239in}}%
\pgfpathlineto{\pgfqpoint{0.715517in}{1.904246in}}%
\pgfpathlineto{\pgfqpoint{0.712973in}{1.903161in}}%
\pgfpathlineto{\pgfqpoint{0.700927in}{1.897846in}}%
\pgfpathlineto{\pgfqpoint{0.688880in}{1.892355in}}%
\pgfpathlineto{\pgfqpoint{0.688551in}{1.892200in}}%
\pgfpathlineto{\pgfqpoint{0.676834in}{1.886604in}}%
\pgfpathlineto{\pgfqpoint{0.664788in}{1.880670in}}%
\pgfpathlineto{\pgfqpoint{0.663770in}{1.880154in}}%
\pgfpathlineto{\pgfqpoint{0.652742in}{1.874473in}}%
\pgfpathlineto{\pgfqpoint{0.640739in}{1.868107in}}%
\pgfpathlineto{\pgfqpoint{0.640696in}{1.868084in}}%
\pgfpathlineto{\pgfqpoint{0.628650in}{1.861413in}}%
\pgfpathlineto{\pgfqpoint{0.619249in}{1.856061in}}%
\pgfpathlineto{\pgfqpoint{0.616604in}{1.854532in}}%
\pgfpathlineto{\pgfqpoint{0.604558in}{1.847380in}}%
\pgfpathlineto{\pgfqpoint{0.599036in}{1.844015in}}%
\pgfpathlineto{\pgfqpoint{0.592511in}{1.839976in}}%
\pgfpathlineto{\pgfqpoint{0.580465in}{1.832328in}}%
\pgfpathlineto{\pgfqpoint{0.579914in}{1.831969in}}%
\pgfpathlineto{\pgfqpoint{0.568419in}{1.824367in}}%
\pgfpathlineto{\pgfqpoint{0.561858in}{1.819923in}}%
\pgfpathlineto{\pgfqpoint{0.556373in}{1.816146in}}%
\pgfpathlineto{\pgfqpoint{0.544642in}{1.807877in}}%
\pgfpathlineto{\pgfqpoint{0.544327in}{1.807651in}}%
\pgfpathlineto{\pgfqpoint{0.532281in}{1.798817in}}%
\pgfpathlineto{\pgfqpoint{0.528299in}{1.795831in}}%
\pgfpathlineto{\pgfqpoint{0.520235in}{1.789680in}}%
\pgfpathlineto{\pgfqpoint{0.512673in}{1.783785in}}%
\pgfpathlineto{\pgfqpoint{0.508189in}{1.780228in}}%
\pgfpathlineto{\pgfqpoint{0.497712in}{1.771738in}}%
\pgfpathlineto{\pgfqpoint{0.496143in}{1.770444in}}%
\pgfpathlineto{\pgfqpoint{0.484096in}{1.760300in}}%
\pgfpathlineto{\pgfqpoint{0.483390in}{1.759692in}}%
\pgfpathlineto{\pgfqpoint{0.472050in}{1.749770in}}%
\pgfpathlineto{\pgfqpoint{0.469671in}{1.747646in}}%
\pgfpathlineto{\pgfqpoint{0.460004in}{1.738858in}}%
\pgfpathlineto{\pgfqpoint{0.456491in}{1.735600in}}%
\pgfpathlineto{\pgfqpoint{0.447958in}{1.727542in}}%
\pgfpathlineto{\pgfqpoint{0.443817in}{1.723554in}}%
\pgfpathlineto{\pgfqpoint{0.435912in}{1.715798in}}%
\pgfpathlineto{\pgfqpoint{0.431622in}{1.711508in}}%
\pgfpathlineto{\pgfqpoint{0.423866in}{1.703603in}}%
\pgfpathlineto{\pgfqpoint{0.419878in}{1.699462in}}%
\pgfpathlineto{\pgfqpoint{0.411820in}{1.690929in}}%
\pgfpathlineto{\pgfqpoint{0.408562in}{1.687416in}}%
\pgfpathlineto{\pgfqpoint{0.399774in}{1.677749in}}%
\pgfpathlineto{\pgfqpoint{0.397649in}{1.675370in}}%
\pgfpathlineto{\pgfqpoint{0.387728in}{1.664030in}}%
\pgfpathlineto{\pgfqpoint{0.387120in}{1.663323in}}%
\pgfpathlineto{\pgfqpoint{0.376976in}{1.651277in}}%
\pgfpathlineto{\pgfqpoint{0.375681in}{1.649708in}}%
\pgfpathlineto{\pgfqpoint{0.367192in}{1.639231in}}%
\pgfpathlineto{\pgfqpoint{0.363635in}{1.634747in}}%
\pgfpathlineto{\pgfqpoint{0.357740in}{1.627185in}}%
\pgfpathlineto{\pgfqpoint{0.351589in}{1.619121in}}%
\pgfpathlineto{\pgfqpoint{0.348603in}{1.615139in}}%
\pgfpathlineto{\pgfqpoint{0.339769in}{1.603093in}}%
\pgfpathlineto{\pgfqpoint{0.339543in}{1.602778in}}%
\pgfpathlineto{\pgfqpoint{0.331274in}{1.591047in}}%
\pgfpathlineto{\pgfqpoint{0.327497in}{1.585561in}}%
\pgfpathlineto{\pgfqpoint{0.323052in}{1.579001in}}%
\pgfpathlineto{\pgfqpoint{0.315451in}{1.567505in}}%
\pgfpathlineto{\pgfqpoint{0.315092in}{1.566955in}}%
\pgfpathlineto{\pgfqpoint{0.307444in}{1.554908in}}%
\pgfpathlineto{\pgfqpoint{0.303405in}{1.548383in}}%
\pgfpathlineto{\pgfqpoint{0.300040in}{1.542862in}}%
\pgfpathlineto{\pgfqpoint{0.292888in}{1.530816in}}%
\pgfpathlineto{\pgfqpoint{0.291359in}{1.528171in}}%
\pgfpathlineto{\pgfqpoint{0.286007in}{1.518770in}}%
\pgfpathlineto{\pgfqpoint{0.279336in}{1.506724in}}%
\pgfpathlineto{\pgfqpoint{0.279312in}{1.506681in}}%
\pgfpathlineto{\pgfqpoint{0.272947in}{1.494678in}}%
\pgfpathlineto{\pgfqpoint{0.267266in}{1.483650in}}%
\pgfpathlineto{\pgfqpoint{0.266749in}{1.482632in}}%
\pgfpathlineto{\pgfqpoint{0.260816in}{1.470586in}}%
\pgfpathlineto{\pgfqpoint{0.255220in}{1.458869in}}%
\pgfpathlineto{\pgfqpoint{0.255065in}{1.458539in}}%
\pgfpathlineto{\pgfqpoint{0.249574in}{1.446493in}}%
\pgfpathlineto{\pgfqpoint{0.244259in}{1.434447in}}%
\pgfpathlineto{\pgfqpoint{0.243174in}{1.431903in}}%
\pgfpathlineto{\pgfqpoint{0.239181in}{1.422401in}}%
\pgfpathlineto{\pgfqpoint{0.234292in}{1.410355in}}%
\pgfpathlineto{\pgfqpoint{0.231128in}{1.402269in}}%
\pgfpathlineto{\pgfqpoint{0.229601in}{1.398309in}}%
\pgfpathlineto{\pgfqpoint{0.225126in}{1.386263in}}%
\pgfpathlineto{\pgfqpoint{0.220824in}{1.374217in}}%
\pgfpathlineto{\pgfqpoint{0.219082in}{1.369135in}}%
\pgfpathlineto{\pgfqpoint{0.216728in}{1.362171in}}%
\pgfpathlineto{\pgfqpoint{0.212826in}{1.350124in}}%
\pgfpathlineto{\pgfqpoint{0.209093in}{1.338078in}}%
\pgfpathlineto{\pgfqpoint{0.207036in}{1.331123in}}%
\pgfpathlineto{\pgfqpoint{0.205551in}{1.326032in}}%
\pgfpathlineto{\pgfqpoint{0.202205in}{1.313986in}}%
\pgfpathlineto{\pgfqpoint{0.199026in}{1.301940in}}%
\pgfpathlineto{\pgfqpoint{0.196014in}{1.289894in}}%
\pgfpathlineto{\pgfqpoint{0.194990in}{1.285554in}}%
\pgfpathlineto{\pgfqpoint{0.193195in}{1.277848in}}%
\pgfpathlineto{\pgfqpoint{0.190555in}{1.265802in}}%
\pgfpathlineto{\pgfqpoint{0.188080in}{1.253756in}}%
\pgfpathlineto{\pgfqpoint{0.185769in}{1.241709in}}%
\pgfpathlineto{\pgfqpoint{0.183624in}{1.229663in}}%
\pgfpathlineto{\pgfqpoint{0.182944in}{1.225522in}}%
\pgfpathlineto{\pgfqpoint{0.181662in}{1.217617in}}%
\pgfpathlineto{\pgfqpoint{0.179871in}{1.205571in}}%
\pgfpathlineto{\pgfqpoint{0.178243in}{1.193525in}}%
\pgfpathlineto{\pgfqpoint{0.176778in}{1.181479in}}%
\pgfpathlineto{\pgfqpoint{0.175476in}{1.169433in}}%
\pgfpathlineto{\pgfqpoint{0.174336in}{1.157387in}}%
\pgfpathlineto{\pgfqpoint{0.173360in}{1.145340in}}%
\pgfpathlineto{\pgfqpoint{0.172546in}{1.133294in}}%
\pgfpathlineto{\pgfqpoint{0.171894in}{1.121248in}}%
\pgfpathlineto{\pgfqpoint{0.171406in}{1.109202in}}%
\pgfpathlineto{\pgfqpoint{0.171081in}{1.097156in}}%
\pgfpathlineto{\pgfqpoint{0.170918in}{1.085110in}}%
\pgfpathlineto{\pgfqpoint{0.170918in}{1.073064in}}%
\pgfpathlineto{\pgfqpoint{0.171081in}{1.061018in}}%
\pgfpathlineto{\pgfqpoint{0.171406in}{1.048972in}}%
\pgfpathlineto{\pgfqpoint{0.171894in}{1.036925in}}%
\pgfpathlineto{\pgfqpoint{0.172546in}{1.024879in}}%
\pgfpathlineto{\pgfqpoint{0.173360in}{1.012833in}}%
\pgfpathlineto{\pgfqpoint{0.174336in}{1.000787in}}%
\pgfpathlineto{\pgfqpoint{0.175476in}{0.988741in}}%
\pgfpathlineto{\pgfqpoint{0.176778in}{0.976695in}}%
\pgfpathlineto{\pgfqpoint{0.178243in}{0.964649in}}%
\pgfpathlineto{\pgfqpoint{0.179871in}{0.952603in}}%
\pgfpathlineto{\pgfqpoint{0.181662in}{0.940557in}}%
\pgfpathlineto{\pgfqpoint{0.182944in}{0.932651in}}%
\pgfpathlineto{\pgfqpoint{0.183624in}{0.928510in}}%
\pgfpathlineto{\pgfqpoint{0.185769in}{0.916464in}}%
\pgfpathlineto{\pgfqpoint{0.188080in}{0.904418in}}%
\pgfpathlineto{\pgfqpoint{0.190555in}{0.892372in}}%
\pgfpathlineto{\pgfqpoint{0.193195in}{0.880326in}}%
\pgfpathlineto{\pgfqpoint{0.194990in}{0.872620in}}%
\pgfpathlineto{\pgfqpoint{0.196014in}{0.868280in}}%
\pgfpathlineto{\pgfqpoint{0.199026in}{0.856234in}}%
\pgfpathlineto{\pgfqpoint{0.202205in}{0.844188in}}%
\pgfpathlineto{\pgfqpoint{0.205551in}{0.832141in}}%
\pgfpathlineto{\pgfqpoint{0.207036in}{0.827051in}}%
\pgfpathlineto{\pgfqpoint{0.209093in}{0.820095in}}%
\pgfpathlineto{\pgfqpoint{0.212826in}{0.808049in}}%
\pgfpathlineto{\pgfqpoint{0.216728in}{0.796003in}}%
\pgfpathlineto{\pgfqpoint{0.219082in}{0.789039in}}%
\pgfpathlineto{\pgfqpoint{0.220824in}{0.783957in}}%
\pgfpathlineto{\pgfqpoint{0.225126in}{0.771911in}}%
\pgfpathlineto{\pgfqpoint{0.229601in}{0.759865in}}%
\pgfpathlineto{\pgfqpoint{0.231128in}{0.755905in}}%
\pgfpathlineto{\pgfqpoint{0.234292in}{0.747819in}}%
\pgfpathlineto{\pgfqpoint{0.239181in}{0.735773in}}%
\pgfpathlineto{\pgfqpoint{0.243174in}{0.726271in}}%
\pgfpathlineto{\pgfqpoint{0.244259in}{0.723726in}}%
\pgfpathlineto{\pgfqpoint{0.249574in}{0.711680in}}%
\pgfpathlineto{\pgfqpoint{0.255065in}{0.699634in}}%
\pgfpathlineto{\pgfqpoint{0.255220in}{0.699305in}}%
\pgfpathlineto{\pgfqpoint{0.260816in}{0.687588in}}%
\pgfpathlineto{\pgfqpoint{0.266749in}{0.675542in}}%
\pgfpathlineto{\pgfqpoint{0.267266in}{0.674523in}}%
\pgfpathlineto{\pgfqpoint{0.272947in}{0.663496in}}%
\pgfpathlineto{\pgfqpoint{0.279312in}{0.651493in}}%
\pgfpathlineto{\pgfqpoint{0.279336in}{0.651450in}}%
\pgfpathlineto{\pgfqpoint{0.286007in}{0.639404in}}%
\pgfpathlineto{\pgfqpoint{0.291359in}{0.630003in}}%
\pgfpathlineto{\pgfqpoint{0.292888in}{0.627358in}}%
\pgfpathlineto{\pgfqpoint{0.300040in}{0.615311in}}%
\pgfpathlineto{\pgfqpoint{0.303405in}{0.609790in}}%
\pgfpathlineto{\pgfqpoint{0.307444in}{0.603265in}}%
\pgfpathlineto{\pgfqpoint{0.315092in}{0.591219in}}%
\pgfpathlineto{\pgfqpoint{0.315451in}{0.590668in}}%
\pgfpathlineto{\pgfqpoint{0.323052in}{0.579173in}}%
\pgfpathlineto{\pgfqpoint{0.327497in}{0.572612in}}%
\pgfpathlineto{\pgfqpoint{0.331274in}{0.567127in}}%
\pgfpathlineto{\pgfqpoint{0.339543in}{0.555396in}}%
\pgfpathlineto{\pgfqpoint{0.339769in}{0.555081in}}%
\pgfpathlineto{\pgfqpoint{0.348603in}{0.543035in}}%
\pgfpathlineto{\pgfqpoint{0.351589in}{0.539053in}}%
\pgfpathlineto{\pgfqpoint{0.357740in}{0.530989in}}%
\pgfpathlineto{\pgfqpoint{0.363635in}{0.523427in}}%
\pgfpathlineto{\pgfqpoint{0.367192in}{0.518942in}}%
\pgfpathlineto{\pgfqpoint{0.375681in}{0.508466in}}%
\pgfpathlineto{\pgfqpoint{0.376976in}{0.506896in}}%
\pgfpathlineto{\pgfqpoint{0.387120in}{0.494850in}}%
\pgfpathlineto{\pgfqpoint{0.387728in}{0.494143in}}%
\pgfpathlineto{\pgfqpoint{0.397649in}{0.482804in}}%
\pgfpathlineto{\pgfqpoint{0.399774in}{0.480425in}}%
\pgfpathlineto{\pgfqpoint{0.408562in}{0.470758in}}%
\pgfpathlineto{\pgfqpoint{0.411820in}{0.467245in}}%
\pgfpathlineto{\pgfqpoint{0.419878in}{0.458712in}}%
\pgfpathlineto{\pgfqpoint{0.423866in}{0.454571in}}%
\pgfpathlineto{\pgfqpoint{0.431622in}{0.446666in}}%
\pgfpathlineto{\pgfqpoint{0.435912in}{0.442376in}}%
\pgfpathlineto{\pgfqpoint{0.443817in}{0.434620in}}%
\pgfpathlineto{\pgfqpoint{0.447958in}{0.430632in}}%
\pgfpathlineto{\pgfqpoint{0.456491in}{0.422574in}}%
\pgfpathlineto{\pgfqpoint{0.460004in}{0.419316in}}%
\pgfpathlineto{\pgfqpoint{0.469671in}{0.410527in}}%
\pgfpathlineto{\pgfqpoint{0.472050in}{0.408403in}}%
\pgfpathlineto{\pgfqpoint{0.483390in}{0.398481in}}%
\pgfpathlineto{\pgfqpoint{0.484096in}{0.397874in}}%
\pgfpathlineto{\pgfqpoint{0.496143in}{0.387730in}}%
\pgfpathlineto{\pgfqpoint{0.497712in}{0.386435in}}%
\pgfpathlineto{\pgfqpoint{0.508189in}{0.377946in}}%
\pgfpathlineto{\pgfqpoint{0.512673in}{0.374389in}}%
\pgfpathlineto{\pgfqpoint{0.520235in}{0.368494in}}%
\pgfpathlineto{\pgfqpoint{0.528299in}{0.362343in}}%
\pgfpathlineto{\pgfqpoint{0.532281in}{0.359357in}}%
\pgfpathlineto{\pgfqpoint{0.544327in}{0.350523in}}%
\pgfpathlineto{\pgfqpoint{0.544642in}{0.350297in}}%
\pgfpathlineto{\pgfqpoint{0.556373in}{0.342028in}}%
\pgfpathlineto{\pgfqpoint{0.561858in}{0.338251in}}%
\pgfpathlineto{\pgfqpoint{0.568419in}{0.333806in}}%
\pgfpathlineto{\pgfqpoint{0.579914in}{0.326205in}}%
\pgfpathlineto{\pgfqpoint{0.580465in}{0.325846in}}%
\pgfpathlineto{\pgfqpoint{0.592511in}{0.318198in}}%
\pgfpathlineto{\pgfqpoint{0.599036in}{0.314159in}}%
\pgfpathlineto{\pgfqpoint{0.604558in}{0.310794in}}%
\pgfpathlineto{\pgfqpoint{0.616604in}{0.303642in}}%
\pgfpathlineto{\pgfqpoint{0.619249in}{0.302112in}}%
\pgfpathlineto{\pgfqpoint{0.628650in}{0.296761in}}%
\pgfpathlineto{\pgfqpoint{0.640696in}{0.290089in}}%
\pgfpathlineto{\pgfqpoint{0.640739in}{0.290066in}}%
\pgfpathlineto{\pgfqpoint{0.652742in}{0.283701in}}%
\pgfpathlineto{\pgfqpoint{0.663770in}{0.278020in}}%
\pgfpathlineto{\pgfqpoint{0.664788in}{0.277503in}}%
\pgfpathlineto{\pgfqpoint{0.676834in}{0.271570in}}%
\pgfpathlineto{\pgfqpoint{0.688551in}{0.265974in}}%
\pgfpathlineto{\pgfqpoint{0.688880in}{0.265819in}}%
\pgfpathlineto{\pgfqpoint{0.700927in}{0.260327in}}%
\pgfpathlineto{\pgfqpoint{0.712973in}{0.255013in}}%
\pgfpathlineto{\pgfqpoint{0.715517in}{0.253928in}}%
\pgfpathlineto{\pgfqpoint{0.725019in}{0.249934in}}%
\pgfpathlineto{\pgfqpoint{0.737065in}{0.245046in}}%
\pgfpathlineto{\pgfqpoint{0.745151in}{0.241882in}}%
\pgfpathlineto{\pgfqpoint{0.749111in}{0.240355in}}%
\pgfpathlineto{\pgfqpoint{0.761157in}{0.235880in}}%
\pgfpathlineto{\pgfqpoint{0.773203in}{0.231578in}}%
\pgfpathlineto{\pgfqpoint{0.778285in}{0.229836in}}%
\pgfpathlineto{\pgfqpoint{0.785249in}{0.227482in}}%
\pgfpathlineto{\pgfqpoint{0.797295in}{0.223579in}}%
\pgfpathlineto{\pgfqpoint{0.809342in}{0.219847in}}%
\pgfpathlineto{\pgfqpoint{0.816297in}{0.217790in}}%
\pgfpathlineto{\pgfqpoint{0.821388in}{0.216305in}}%
\pgfpathlineto{\pgfqpoint{0.833434in}{0.212959in}}%
\pgfpathlineto{\pgfqpoint{0.845480in}{0.209780in}}%
\pgfpathlineto{\pgfqpoint{0.857526in}{0.206768in}}%
\pgfpathlineto{\pgfqpoint{0.861866in}{0.205743in}}%
\pgfpathlineto{\pgfqpoint{0.869572in}{0.203949in}}%
\pgfpathlineto{\pgfqpoint{0.881618in}{0.201309in}}%
\pgfpathlineto{\pgfqpoint{0.893664in}{0.198833in}}%
\pgfpathlineto{\pgfqpoint{0.905710in}{0.196523in}}%
\pgfpathlineto{\pgfqpoint{0.917757in}{0.194378in}}%
\pgfpathlineto{\pgfqpoint{0.921897in}{0.193697in}}%
\pgfpathclose%
\pgfpathmoveto{\pgfqpoint{0.921897in}{0.193697in}}%
\pgfpathlineto{\pgfqpoint{0.917757in}{0.194378in}}%
\pgfpathlineto{\pgfqpoint{0.905710in}{0.196523in}}%
\pgfpathlineto{\pgfqpoint{0.893664in}{0.198833in}}%
\pgfpathlineto{\pgfqpoint{0.881618in}{0.201309in}}%
\pgfpathlineto{\pgfqpoint{0.869572in}{0.203949in}}%
\pgfpathlineto{\pgfqpoint{0.861866in}{0.205743in}}%
\pgfpathlineto{\pgfqpoint{0.857526in}{0.206768in}}%
\pgfpathlineto{\pgfqpoint{0.845480in}{0.209780in}}%
\pgfpathlineto{\pgfqpoint{0.833434in}{0.212959in}}%
\pgfpathlineto{\pgfqpoint{0.821388in}{0.216305in}}%
\pgfpathlineto{\pgfqpoint{0.816297in}{0.217790in}}%
\pgfpathlineto{\pgfqpoint{0.809342in}{0.219847in}}%
\pgfpathlineto{\pgfqpoint{0.797295in}{0.223579in}}%
\pgfpathlineto{\pgfqpoint{0.785249in}{0.227482in}}%
\pgfpathlineto{\pgfqpoint{0.778285in}{0.229836in}}%
\pgfpathlineto{\pgfqpoint{0.773203in}{0.231578in}}%
\pgfpathlineto{\pgfqpoint{0.761157in}{0.235880in}}%
\pgfpathlineto{\pgfqpoint{0.749111in}{0.240355in}}%
\pgfpathlineto{\pgfqpoint{0.745151in}{0.241882in}}%
\pgfpathlineto{\pgfqpoint{0.737065in}{0.245046in}}%
\pgfpathlineto{\pgfqpoint{0.725019in}{0.249934in}}%
\pgfpathlineto{\pgfqpoint{0.715517in}{0.253928in}}%
\pgfpathlineto{\pgfqpoint{0.712973in}{0.255013in}}%
\pgfpathlineto{\pgfqpoint{0.700927in}{0.260327in}}%
\pgfpathlineto{\pgfqpoint{0.688880in}{0.265819in}}%
\pgfpathlineto{\pgfqpoint{0.688551in}{0.265974in}}%
\pgfpathlineto{\pgfqpoint{0.676834in}{0.271570in}}%
\pgfpathlineto{\pgfqpoint{0.664788in}{0.277503in}}%
\pgfpathlineto{\pgfqpoint{0.663770in}{0.278020in}}%
\pgfpathlineto{\pgfqpoint{0.652742in}{0.283701in}}%
\pgfpathlineto{\pgfqpoint{0.640739in}{0.290066in}}%
\pgfpathlineto{\pgfqpoint{0.640696in}{0.290089in}}%
\pgfpathlineto{\pgfqpoint{0.628650in}{0.296761in}}%
\pgfpathlineto{\pgfqpoint{0.619249in}{0.302112in}}%
\pgfpathlineto{\pgfqpoint{0.616604in}{0.303642in}}%
\pgfpathlineto{\pgfqpoint{0.604558in}{0.310794in}}%
\pgfpathlineto{\pgfqpoint{0.599036in}{0.314159in}}%
\pgfpathlineto{\pgfqpoint{0.592511in}{0.318198in}}%
\pgfpathlineto{\pgfqpoint{0.580465in}{0.325846in}}%
\pgfpathlineto{\pgfqpoint{0.579914in}{0.326205in}}%
\pgfpathlineto{\pgfqpoint{0.568419in}{0.333806in}}%
\pgfpathlineto{\pgfqpoint{0.561858in}{0.338251in}}%
\pgfpathlineto{\pgfqpoint{0.556373in}{0.342028in}}%
\pgfpathlineto{\pgfqpoint{0.544642in}{0.350297in}}%
\pgfpathlineto{\pgfqpoint{0.544327in}{0.350523in}}%
\pgfpathlineto{\pgfqpoint{0.532281in}{0.359357in}}%
\pgfpathlineto{\pgfqpoint{0.528299in}{0.362343in}}%
\pgfpathlineto{\pgfqpoint{0.520235in}{0.368494in}}%
\pgfpathlineto{\pgfqpoint{0.512673in}{0.374389in}}%
\pgfpathlineto{\pgfqpoint{0.508189in}{0.377946in}}%
\pgfpathlineto{\pgfqpoint{0.497712in}{0.386435in}}%
\pgfpathlineto{\pgfqpoint{0.496143in}{0.387730in}}%
\pgfpathlineto{\pgfqpoint{0.484096in}{0.397874in}}%
\pgfpathlineto{\pgfqpoint{0.483390in}{0.398481in}}%
\pgfpathlineto{\pgfqpoint{0.472050in}{0.408403in}}%
\pgfpathlineto{\pgfqpoint{0.469671in}{0.410527in}}%
\pgfpathlineto{\pgfqpoint{0.460004in}{0.419316in}}%
\pgfpathlineto{\pgfqpoint{0.456491in}{0.422574in}}%
\pgfpathlineto{\pgfqpoint{0.447958in}{0.430632in}}%
\pgfpathlineto{\pgfqpoint{0.443817in}{0.434620in}}%
\pgfpathlineto{\pgfqpoint{0.435912in}{0.442376in}}%
\pgfpathlineto{\pgfqpoint{0.431622in}{0.446666in}}%
\pgfpathlineto{\pgfqpoint{0.423866in}{0.454571in}}%
\pgfpathlineto{\pgfqpoint{0.419878in}{0.458712in}}%
\pgfpathlineto{\pgfqpoint{0.411820in}{0.467245in}}%
\pgfpathlineto{\pgfqpoint{0.408562in}{0.470758in}}%
\pgfpathlineto{\pgfqpoint{0.399774in}{0.480425in}}%
\pgfpathlineto{\pgfqpoint{0.397649in}{0.482804in}}%
\pgfpathlineto{\pgfqpoint{0.387728in}{0.494143in}}%
\pgfpathlineto{\pgfqpoint{0.387120in}{0.494850in}}%
\pgfpathlineto{\pgfqpoint{0.376976in}{0.506896in}}%
\pgfpathlineto{\pgfqpoint{0.375681in}{0.508466in}}%
\pgfpathlineto{\pgfqpoint{0.367192in}{0.518942in}}%
\pgfpathlineto{\pgfqpoint{0.363635in}{0.523427in}}%
\pgfpathlineto{\pgfqpoint{0.357740in}{0.530989in}}%
\pgfpathlineto{\pgfqpoint{0.351589in}{0.539053in}}%
\pgfpathlineto{\pgfqpoint{0.348603in}{0.543035in}}%
\pgfpathlineto{\pgfqpoint{0.339769in}{0.555081in}}%
\pgfpathlineto{\pgfqpoint{0.339543in}{0.555396in}}%
\pgfpathlineto{\pgfqpoint{0.331274in}{0.567127in}}%
\pgfpathlineto{\pgfqpoint{0.327497in}{0.572612in}}%
\pgfpathlineto{\pgfqpoint{0.323052in}{0.579173in}}%
\pgfpathlineto{\pgfqpoint{0.315451in}{0.590668in}}%
\pgfpathlineto{\pgfqpoint{0.315092in}{0.591219in}}%
\pgfpathlineto{\pgfqpoint{0.307444in}{0.603265in}}%
\pgfpathlineto{\pgfqpoint{0.303405in}{0.609790in}}%
\pgfpathlineto{\pgfqpoint{0.300040in}{0.615311in}}%
\pgfpathlineto{\pgfqpoint{0.292888in}{0.627358in}}%
\pgfpathlineto{\pgfqpoint{0.291359in}{0.630003in}}%
\pgfpathlineto{\pgfqpoint{0.286007in}{0.639404in}}%
\pgfpathlineto{\pgfqpoint{0.279336in}{0.651450in}}%
\pgfpathlineto{\pgfqpoint{0.279312in}{0.651493in}}%
\pgfpathlineto{\pgfqpoint{0.272947in}{0.663496in}}%
\pgfpathlineto{\pgfqpoint{0.267266in}{0.674523in}}%
\pgfpathlineto{\pgfqpoint{0.266749in}{0.675542in}}%
\pgfpathlineto{\pgfqpoint{0.260816in}{0.687588in}}%
\pgfpathlineto{\pgfqpoint{0.255220in}{0.699305in}}%
\pgfpathlineto{\pgfqpoint{0.255065in}{0.699634in}}%
\pgfpathlineto{\pgfqpoint{0.249574in}{0.711680in}}%
\pgfpathlineto{\pgfqpoint{0.244259in}{0.723726in}}%
\pgfpathlineto{\pgfqpoint{0.243174in}{0.726271in}}%
\pgfpathlineto{\pgfqpoint{0.239181in}{0.735773in}}%
\pgfpathlineto{\pgfqpoint{0.234292in}{0.747819in}}%
\pgfpathlineto{\pgfqpoint{0.231128in}{0.755905in}}%
\pgfpathlineto{\pgfqpoint{0.229601in}{0.759865in}}%
\pgfpathlineto{\pgfqpoint{0.225126in}{0.771911in}}%
\pgfpathlineto{\pgfqpoint{0.220824in}{0.783957in}}%
\pgfpathlineto{\pgfqpoint{0.219082in}{0.789039in}}%
\pgfpathlineto{\pgfqpoint{0.216728in}{0.796003in}}%
\pgfpathlineto{\pgfqpoint{0.212826in}{0.808049in}}%
\pgfpathlineto{\pgfqpoint{0.209093in}{0.820095in}}%
\pgfpathlineto{\pgfqpoint{0.207036in}{0.827051in}}%
\pgfpathlineto{\pgfqpoint{0.205551in}{0.832141in}}%
\pgfpathlineto{\pgfqpoint{0.202205in}{0.844188in}}%
\pgfpathlineto{\pgfqpoint{0.199026in}{0.856234in}}%
\pgfpathlineto{\pgfqpoint{0.196014in}{0.868280in}}%
\pgfpathlineto{\pgfqpoint{0.194990in}{0.872620in}}%
\pgfpathlineto{\pgfqpoint{0.193195in}{0.880326in}}%
\pgfpathlineto{\pgfqpoint{0.190555in}{0.892372in}}%
\pgfpathlineto{\pgfqpoint{0.188080in}{0.904418in}}%
\pgfpathlineto{\pgfqpoint{0.185769in}{0.916464in}}%
\pgfpathlineto{\pgfqpoint{0.183624in}{0.928510in}}%
\pgfpathlineto{\pgfqpoint{0.182944in}{0.932651in}}%
\pgfpathlineto{\pgfqpoint{0.181662in}{0.940557in}}%
\pgfpathlineto{\pgfqpoint{0.179871in}{0.952603in}}%
\pgfpathlineto{\pgfqpoint{0.178243in}{0.964649in}}%
\pgfpathlineto{\pgfqpoint{0.176778in}{0.976695in}}%
\pgfpathlineto{\pgfqpoint{0.175476in}{0.988741in}}%
\pgfpathlineto{\pgfqpoint{0.174336in}{1.000787in}}%
\pgfpathlineto{\pgfqpoint{0.173360in}{1.012833in}}%
\pgfpathlineto{\pgfqpoint{0.172546in}{1.024879in}}%
\pgfpathlineto{\pgfqpoint{0.171894in}{1.036925in}}%
\pgfpathlineto{\pgfqpoint{0.171406in}{1.048972in}}%
\pgfpathlineto{\pgfqpoint{0.171081in}{1.061018in}}%
\pgfpathlineto{\pgfqpoint{0.170918in}{1.073064in}}%
\pgfpathlineto{\pgfqpoint{0.170918in}{1.085110in}}%
\pgfpathlineto{\pgfqpoint{0.171081in}{1.097156in}}%
\pgfpathlineto{\pgfqpoint{0.171406in}{1.109202in}}%
\pgfpathlineto{\pgfqpoint{0.171894in}{1.121248in}}%
\pgfpathlineto{\pgfqpoint{0.172546in}{1.133294in}}%
\pgfpathlineto{\pgfqpoint{0.173360in}{1.145340in}}%
\pgfpathlineto{\pgfqpoint{0.174336in}{1.157387in}}%
\pgfpathlineto{\pgfqpoint{0.175476in}{1.169433in}}%
\pgfpathlineto{\pgfqpoint{0.176778in}{1.181479in}}%
\pgfpathlineto{\pgfqpoint{0.178243in}{1.193525in}}%
\pgfpathlineto{\pgfqpoint{0.179871in}{1.205571in}}%
\pgfpathlineto{\pgfqpoint{0.181662in}{1.217617in}}%
\pgfpathlineto{\pgfqpoint{0.182944in}{1.225522in}}%
\pgfpathlineto{\pgfqpoint{0.183624in}{1.229663in}}%
\pgfpathlineto{\pgfqpoint{0.185769in}{1.241709in}}%
\pgfpathlineto{\pgfqpoint{0.188080in}{1.253756in}}%
\pgfpathlineto{\pgfqpoint{0.190555in}{1.265802in}}%
\pgfpathlineto{\pgfqpoint{0.193195in}{1.277848in}}%
\pgfpathlineto{\pgfqpoint{0.194990in}{1.285554in}}%
\pgfpathlineto{\pgfqpoint{0.196014in}{1.289894in}}%
\pgfpathlineto{\pgfqpoint{0.199026in}{1.301940in}}%
\pgfpathlineto{\pgfqpoint{0.202205in}{1.313986in}}%
\pgfpathlineto{\pgfqpoint{0.205551in}{1.326032in}}%
\pgfpathlineto{\pgfqpoint{0.207036in}{1.331123in}}%
\pgfpathlineto{\pgfqpoint{0.209093in}{1.338078in}}%
\pgfpathlineto{\pgfqpoint{0.212826in}{1.350124in}}%
\pgfpathlineto{\pgfqpoint{0.216728in}{1.362171in}}%
\pgfpathlineto{\pgfqpoint{0.219082in}{1.369135in}}%
\pgfpathlineto{\pgfqpoint{0.220824in}{1.374217in}}%
\pgfpathlineto{\pgfqpoint{0.225126in}{1.386263in}}%
\pgfpathlineto{\pgfqpoint{0.229601in}{1.398309in}}%
\pgfpathlineto{\pgfqpoint{0.231128in}{1.402269in}}%
\pgfpathlineto{\pgfqpoint{0.234292in}{1.410355in}}%
\pgfpathlineto{\pgfqpoint{0.239181in}{1.422401in}}%
\pgfpathlineto{\pgfqpoint{0.243174in}{1.431903in}}%
\pgfpathlineto{\pgfqpoint{0.244259in}{1.434447in}}%
\pgfpathlineto{\pgfqpoint{0.249574in}{1.446493in}}%
\pgfpathlineto{\pgfqpoint{0.255065in}{1.458539in}}%
\pgfpathlineto{\pgfqpoint{0.255220in}{1.458869in}}%
\pgfpathlineto{\pgfqpoint{0.260816in}{1.470586in}}%
\pgfpathlineto{\pgfqpoint{0.266749in}{1.482632in}}%
\pgfpathlineto{\pgfqpoint{0.267266in}{1.483650in}}%
\pgfpathlineto{\pgfqpoint{0.272947in}{1.494678in}}%
\pgfpathlineto{\pgfqpoint{0.279312in}{1.506681in}}%
\pgfpathlineto{\pgfqpoint{0.279336in}{1.506724in}}%
\pgfpathlineto{\pgfqpoint{0.286007in}{1.518770in}}%
\pgfpathlineto{\pgfqpoint{0.291359in}{1.528171in}}%
\pgfpathlineto{\pgfqpoint{0.292888in}{1.530816in}}%
\pgfpathlineto{\pgfqpoint{0.300040in}{1.542862in}}%
\pgfpathlineto{\pgfqpoint{0.303405in}{1.548383in}}%
\pgfpathlineto{\pgfqpoint{0.307444in}{1.554908in}}%
\pgfpathlineto{\pgfqpoint{0.315092in}{1.566955in}}%
\pgfpathlineto{\pgfqpoint{0.315451in}{1.567505in}}%
\pgfpathlineto{\pgfqpoint{0.323052in}{1.579001in}}%
\pgfpathlineto{\pgfqpoint{0.327497in}{1.585561in}}%
\pgfpathlineto{\pgfqpoint{0.331274in}{1.591047in}}%
\pgfpathlineto{\pgfqpoint{0.339543in}{1.602778in}}%
\pgfpathlineto{\pgfqpoint{0.339769in}{1.603093in}}%
\pgfpathlineto{\pgfqpoint{0.348603in}{1.615139in}}%
\pgfpathlineto{\pgfqpoint{0.351589in}{1.619121in}}%
\pgfpathlineto{\pgfqpoint{0.357740in}{1.627185in}}%
\pgfpathlineto{\pgfqpoint{0.363635in}{1.634747in}}%
\pgfpathlineto{\pgfqpoint{0.367192in}{1.639231in}}%
\pgfpathlineto{\pgfqpoint{0.375681in}{1.649708in}}%
\pgfpathlineto{\pgfqpoint{0.376976in}{1.651277in}}%
\pgfpathlineto{\pgfqpoint{0.387120in}{1.663323in}}%
\pgfpathlineto{\pgfqpoint{0.387728in}{1.664030in}}%
\pgfpathlineto{\pgfqpoint{0.397649in}{1.675370in}}%
\pgfpathlineto{\pgfqpoint{0.399774in}{1.677749in}}%
\pgfpathlineto{\pgfqpoint{0.408562in}{1.687416in}}%
\pgfpathlineto{\pgfqpoint{0.411820in}{1.690929in}}%
\pgfpathlineto{\pgfqpoint{0.419878in}{1.699462in}}%
\pgfpathlineto{\pgfqpoint{0.423866in}{1.703603in}}%
\pgfpathlineto{\pgfqpoint{0.431622in}{1.711508in}}%
\pgfpathlineto{\pgfqpoint{0.435912in}{1.715798in}}%
\pgfpathlineto{\pgfqpoint{0.443817in}{1.723554in}}%
\pgfpathlineto{\pgfqpoint{0.447958in}{1.727542in}}%
\pgfpathlineto{\pgfqpoint{0.456491in}{1.735600in}}%
\pgfpathlineto{\pgfqpoint{0.460004in}{1.738858in}}%
\pgfpathlineto{\pgfqpoint{0.469671in}{1.747646in}}%
\pgfpathlineto{\pgfqpoint{0.472050in}{1.749770in}}%
\pgfpathlineto{\pgfqpoint{0.483390in}{1.759692in}}%
\pgfpathlineto{\pgfqpoint{0.484096in}{1.760300in}}%
\pgfpathlineto{\pgfqpoint{0.496143in}{1.770444in}}%
\pgfpathlineto{\pgfqpoint{0.497712in}{1.771738in}}%
\pgfpathlineto{\pgfqpoint{0.508189in}{1.780228in}}%
\pgfpathlineto{\pgfqpoint{0.512673in}{1.783785in}}%
\pgfpathlineto{\pgfqpoint{0.520235in}{1.789680in}}%
\pgfpathlineto{\pgfqpoint{0.528299in}{1.795831in}}%
\pgfpathlineto{\pgfqpoint{0.532281in}{1.798817in}}%
\pgfpathlineto{\pgfqpoint{0.544327in}{1.807651in}}%
\pgfpathlineto{\pgfqpoint{0.544642in}{1.807877in}}%
\pgfpathlineto{\pgfqpoint{0.556373in}{1.816146in}}%
\pgfpathlineto{\pgfqpoint{0.561858in}{1.819923in}}%
\pgfpathlineto{\pgfqpoint{0.568419in}{1.824367in}}%
\pgfpathlineto{\pgfqpoint{0.579914in}{1.831969in}}%
\pgfpathlineto{\pgfqpoint{0.580465in}{1.832328in}}%
\pgfpathlineto{\pgfqpoint{0.592511in}{1.839976in}}%
\pgfpathlineto{\pgfqpoint{0.599036in}{1.844015in}}%
\pgfpathlineto{\pgfqpoint{0.604558in}{1.847380in}}%
\pgfpathlineto{\pgfqpoint{0.616604in}{1.854532in}}%
\pgfpathlineto{\pgfqpoint{0.619249in}{1.856061in}}%
\pgfpathlineto{\pgfqpoint{0.628650in}{1.861413in}}%
\pgfpathlineto{\pgfqpoint{0.640696in}{1.868084in}}%
\pgfpathlineto{\pgfqpoint{0.640739in}{1.868107in}}%
\pgfpathlineto{\pgfqpoint{0.652742in}{1.874473in}}%
\pgfpathlineto{\pgfqpoint{0.663770in}{1.880154in}}%
\pgfpathlineto{\pgfqpoint{0.664788in}{1.880670in}}%
\pgfpathlineto{\pgfqpoint{0.676834in}{1.886604in}}%
\pgfpathlineto{\pgfqpoint{0.688551in}{1.892200in}}%
\pgfpathlineto{\pgfqpoint{0.688880in}{1.892355in}}%
\pgfpathlineto{\pgfqpoint{0.700927in}{1.897846in}}%
\pgfpathlineto{\pgfqpoint{0.712973in}{1.903161in}}%
\pgfpathlineto{\pgfqpoint{0.715517in}{1.904246in}}%
\pgfpathlineto{\pgfqpoint{0.725019in}{1.908239in}}%
\pgfpathlineto{\pgfqpoint{0.737065in}{1.913128in}}%
\pgfpathlineto{\pgfqpoint{0.745151in}{1.916292in}}%
\pgfpathlineto{\pgfqpoint{0.749111in}{1.917819in}}%
\pgfpathlineto{\pgfqpoint{0.761157in}{1.922293in}}%
\pgfpathlineto{\pgfqpoint{0.773203in}{1.926596in}}%
\pgfpathlineto{\pgfqpoint{0.778285in}{1.928338in}}%
\pgfpathlineto{\pgfqpoint{0.785249in}{1.930692in}}%
\pgfpathlineto{\pgfqpoint{0.797295in}{1.934594in}}%
\pgfpathlineto{\pgfqpoint{0.809342in}{1.938327in}}%
\pgfpathlineto{\pgfqpoint{0.816297in}{1.940384in}}%
\pgfpathlineto{\pgfqpoint{0.821388in}{1.941869in}}%
\pgfpathlineto{\pgfqpoint{0.833434in}{1.945215in}}%
\pgfpathlineto{\pgfqpoint{0.845480in}{1.948394in}}%
\pgfpathlineto{\pgfqpoint{0.857526in}{1.951405in}}%
\pgfpathlineto{\pgfqpoint{0.861866in}{1.952430in}}%
\pgfpathlineto{\pgfqpoint{0.869572in}{1.954225in}}%
\pgfpathlineto{\pgfqpoint{0.881618in}{1.956865in}}%
\pgfpathlineto{\pgfqpoint{0.893664in}{1.959340in}}%
\pgfpathlineto{\pgfqpoint{0.905710in}{1.961650in}}%
\pgfpathlineto{\pgfqpoint{0.917757in}{1.963796in}}%
\pgfpathlineto{\pgfqpoint{0.921897in}{1.964476in}}%
\pgfpathlineto{\pgfqpoint{0.929803in}{1.965758in}}%
\pgfpathlineto{\pgfqpoint{0.941849in}{1.967549in}}%
\pgfpathlineto{\pgfqpoint{0.953895in}{1.969177in}}%
\pgfpathlineto{\pgfqpoint{0.965941in}{1.970642in}}%
\pgfpathlineto{\pgfqpoint{0.977987in}{1.971944in}}%
\pgfpathlineto{\pgfqpoint{0.990033in}{1.973084in}}%
\pgfpathlineto{\pgfqpoint{1.002079in}{1.974060in}}%
\pgfpathlineto{\pgfqpoint{1.014126in}{1.974874in}}%
\pgfpathlineto{\pgfqpoint{1.026172in}{1.975525in}}%
\pgfpathlineto{\pgfqpoint{1.038218in}{1.976014in}}%
\pgfpathlineto{\pgfqpoint{1.050264in}{1.976339in}}%
\pgfpathlineto{\pgfqpoint{1.062310in}{1.976502in}}%
\pgfpathlineto{\pgfqpoint{1.074356in}{1.976502in}}%
\pgfpathlineto{\pgfqpoint{1.086402in}{1.976339in}}%
\pgfpathlineto{\pgfqpoint{1.098448in}{1.976014in}}%
\pgfpathlineto{\pgfqpoint{1.110494in}{1.975525in}}%
\pgfpathlineto{\pgfqpoint{1.122541in}{1.974874in}}%
\pgfpathlineto{\pgfqpoint{1.134587in}{1.974060in}}%
\pgfpathlineto{\pgfqpoint{1.146633in}{1.973084in}}%
\pgfpathlineto{\pgfqpoint{1.158679in}{1.971944in}}%
\pgfpathlineto{\pgfqpoint{1.170725in}{1.970642in}}%
\pgfpathlineto{\pgfqpoint{1.182771in}{1.969177in}}%
\pgfpathlineto{\pgfqpoint{1.194817in}{1.967549in}}%
\pgfpathlineto{\pgfqpoint{1.206863in}{1.965758in}}%
\pgfpathlineto{\pgfqpoint{1.214769in}{1.964476in}}%
\pgfpathlineto{\pgfqpoint{1.218909in}{1.963796in}}%
\pgfpathlineto{\pgfqpoint{1.230956in}{1.961650in}}%
\pgfpathlineto{\pgfqpoint{1.243002in}{1.959340in}}%
\pgfpathlineto{\pgfqpoint{1.255048in}{1.956865in}}%
\pgfpathlineto{\pgfqpoint{1.267094in}{1.954225in}}%
\pgfpathlineto{\pgfqpoint{1.274800in}{1.952430in}}%
\pgfpathlineto{\pgfqpoint{1.279140in}{1.951405in}}%
\pgfpathlineto{\pgfqpoint{1.291186in}{1.948394in}}%
\pgfpathlineto{\pgfqpoint{1.303232in}{1.945215in}}%
\pgfpathlineto{\pgfqpoint{1.315278in}{1.941869in}}%
\pgfpathlineto{\pgfqpoint{1.320369in}{1.940384in}}%
\pgfpathlineto{\pgfqpoint{1.327325in}{1.938327in}}%
\pgfpathlineto{\pgfqpoint{1.339371in}{1.934594in}}%
\pgfpathlineto{\pgfqpoint{1.351417in}{1.930692in}}%
\pgfpathlineto{\pgfqpoint{1.358381in}{1.928338in}}%
\pgfpathlineto{\pgfqpoint{1.363463in}{1.926596in}}%
\pgfpathlineto{\pgfqpoint{1.375509in}{1.922293in}}%
\pgfpathlineto{\pgfqpoint{1.387555in}{1.917819in}}%
\pgfpathlineto{\pgfqpoint{1.391515in}{1.916292in}}%
\pgfpathlineto{\pgfqpoint{1.399601in}{1.913128in}}%
\pgfpathlineto{\pgfqpoint{1.411647in}{1.908239in}}%
\pgfpathlineto{\pgfqpoint{1.421149in}{1.904246in}}%
\pgfpathlineto{\pgfqpoint{1.423693in}{1.903161in}}%
\pgfpathlineto{\pgfqpoint{1.435740in}{1.897846in}}%
\pgfpathlineto{\pgfqpoint{1.447786in}{1.892355in}}%
\pgfpathlineto{\pgfqpoint{1.448115in}{1.892200in}}%
\pgfpathlineto{\pgfqpoint{1.459832in}{1.886604in}}%
\pgfpathlineto{\pgfqpoint{1.471878in}{1.880670in}}%
\pgfpathlineto{\pgfqpoint{1.472897in}{1.880154in}}%
\pgfpathlineto{\pgfqpoint{1.483924in}{1.874473in}}%
\pgfpathlineto{\pgfqpoint{1.495927in}{1.868107in}}%
\pgfpathlineto{\pgfqpoint{1.495970in}{1.868084in}}%
\pgfpathlineto{\pgfqpoint{1.508016in}{1.861413in}}%
\pgfpathlineto{\pgfqpoint{1.517417in}{1.856061in}}%
\pgfpathlineto{\pgfqpoint{1.520062in}{1.854532in}}%
\pgfpathlineto{\pgfqpoint{1.532108in}{1.847380in}}%
\pgfpathlineto{\pgfqpoint{1.537630in}{1.844015in}}%
\pgfpathlineto{\pgfqpoint{1.544155in}{1.839976in}}%
\pgfpathlineto{\pgfqpoint{1.556201in}{1.832328in}}%
\pgfpathlineto{\pgfqpoint{1.556752in}{1.831969in}}%
\pgfpathlineto{\pgfqpoint{1.568247in}{1.824367in}}%
\pgfpathlineto{\pgfqpoint{1.574808in}{1.819923in}}%
\pgfpathlineto{\pgfqpoint{1.580293in}{1.816146in}}%
\pgfpathlineto{\pgfqpoint{1.592024in}{1.807877in}}%
\pgfpathlineto{\pgfqpoint{1.592339in}{1.807651in}}%
\pgfpathlineto{\pgfqpoint{1.604385in}{1.798817in}}%
\pgfpathlineto{\pgfqpoint{1.608367in}{1.795831in}}%
\pgfpathlineto{\pgfqpoint{1.616431in}{1.789680in}}%
\pgfpathlineto{\pgfqpoint{1.623993in}{1.783785in}}%
\pgfpathlineto{\pgfqpoint{1.628477in}{1.780228in}}%
\pgfpathlineto{\pgfqpoint{1.638954in}{1.771738in}}%
\pgfpathlineto{\pgfqpoint{1.640524in}{1.770444in}}%
\pgfpathlineto{\pgfqpoint{1.652570in}{1.760300in}}%
\pgfpathlineto{\pgfqpoint{1.653276in}{1.759692in}}%
\pgfpathlineto{\pgfqpoint{1.664616in}{1.749770in}}%
\pgfpathlineto{\pgfqpoint{1.666995in}{1.747646in}}%
\pgfpathlineto{\pgfqpoint{1.676662in}{1.738858in}}%
\pgfpathlineto{\pgfqpoint{1.680175in}{1.735600in}}%
\pgfpathlineto{\pgfqpoint{1.688708in}{1.727542in}}%
\pgfpathlineto{\pgfqpoint{1.692849in}{1.723554in}}%
\pgfpathlineto{\pgfqpoint{1.700754in}{1.715798in}}%
\pgfpathlineto{\pgfqpoint{1.705044in}{1.711508in}}%
\pgfpathlineto{\pgfqpoint{1.712800in}{1.703603in}}%
\pgfpathlineto{\pgfqpoint{1.716788in}{1.699462in}}%
\pgfpathlineto{\pgfqpoint{1.724846in}{1.690929in}}%
\pgfpathlineto{\pgfqpoint{1.728104in}{1.687416in}}%
\pgfpathlineto{\pgfqpoint{1.736892in}{1.677749in}}%
\pgfpathlineto{\pgfqpoint{1.739017in}{1.675370in}}%
\pgfpathlineto{\pgfqpoint{1.748939in}{1.664030in}}%
\pgfpathlineto{\pgfqpoint{1.749546in}{1.663323in}}%
\pgfpathlineto{\pgfqpoint{1.759690in}{1.651277in}}%
\pgfpathlineto{\pgfqpoint{1.760985in}{1.649708in}}%
\pgfpathlineto{\pgfqpoint{1.769474in}{1.639231in}}%
\pgfpathlineto{\pgfqpoint{1.773031in}{1.634747in}}%
\pgfpathlineto{\pgfqpoint{1.778926in}{1.627185in}}%
\pgfpathlineto{\pgfqpoint{1.785077in}{1.619121in}}%
\pgfpathlineto{\pgfqpoint{1.788063in}{1.615139in}}%
\pgfpathlineto{\pgfqpoint{1.796897in}{1.603093in}}%
\pgfpathlineto{\pgfqpoint{1.797123in}{1.602778in}}%
\pgfpathlineto{\pgfqpoint{1.805392in}{1.591047in}}%
\pgfpathlineto{\pgfqpoint{1.809169in}{1.585561in}}%
\pgfpathlineto{\pgfqpoint{1.813614in}{1.579001in}}%
\pgfpathlineto{\pgfqpoint{1.821215in}{1.567505in}}%
\pgfpathlineto{\pgfqpoint{1.821574in}{1.566955in}}%
\pgfpathlineto{\pgfqpoint{1.829222in}{1.554908in}}%
\pgfpathlineto{\pgfqpoint{1.833261in}{1.548383in}}%
\pgfpathlineto{\pgfqpoint{1.836626in}{1.542862in}}%
\pgfpathlineto{\pgfqpoint{1.843778in}{1.530816in}}%
\pgfpathlineto{\pgfqpoint{1.845307in}{1.528171in}}%
\pgfpathlineto{\pgfqpoint{1.850659in}{1.518770in}}%
\pgfpathlineto{\pgfqpoint{1.857330in}{1.506724in}}%
\pgfpathlineto{\pgfqpoint{1.857354in}{1.506681in}}%
\pgfpathlineto{\pgfqpoint{1.863719in}{1.494678in}}%
\pgfpathlineto{\pgfqpoint{1.869400in}{1.483650in}}%
\pgfpathlineto{\pgfqpoint{1.869917in}{1.482632in}}%
\pgfpathlineto{\pgfqpoint{1.875850in}{1.470586in}}%
\pgfpathlineto{\pgfqpoint{1.881446in}{1.458869in}}%
\pgfpathlineto{\pgfqpoint{1.881601in}{1.458539in}}%
\pgfpathlineto{\pgfqpoint{1.887092in}{1.446493in}}%
\pgfpathlineto{\pgfqpoint{1.892407in}{1.434447in}}%
\pgfpathlineto{\pgfqpoint{1.893492in}{1.431903in}}%
\pgfpathlineto{\pgfqpoint{1.897485in}{1.422401in}}%
\pgfpathlineto{\pgfqpoint{1.902374in}{1.410355in}}%
\pgfpathlineto{\pgfqpoint{1.905538in}{1.402269in}}%
\pgfpathlineto{\pgfqpoint{1.907065in}{1.398309in}}%
\pgfpathlineto{\pgfqpoint{1.911540in}{1.386263in}}%
\pgfpathlineto{\pgfqpoint{1.915842in}{1.374217in}}%
\pgfpathlineto{\pgfqpoint{1.917584in}{1.369135in}}%
\pgfpathlineto{\pgfqpoint{1.919938in}{1.362171in}}%
\pgfpathlineto{\pgfqpoint{1.923841in}{1.350124in}}%
\pgfpathlineto{\pgfqpoint{1.927573in}{1.338078in}}%
\pgfpathlineto{\pgfqpoint{1.929630in}{1.331123in}}%
\pgfpathlineto{\pgfqpoint{1.931115in}{1.326032in}}%
\pgfpathlineto{\pgfqpoint{1.934461in}{1.313986in}}%
\pgfpathlineto{\pgfqpoint{1.937640in}{1.301940in}}%
\pgfpathlineto{\pgfqpoint{1.940652in}{1.289894in}}%
\pgfpathlineto{\pgfqpoint{1.941676in}{1.285554in}}%
\pgfpathlineto{\pgfqpoint{1.943471in}{1.277848in}}%
\pgfpathlineto{\pgfqpoint{1.946111in}{1.265802in}}%
\pgfpathlineto{\pgfqpoint{1.948586in}{1.253756in}}%
\pgfpathlineto{\pgfqpoint{1.950897in}{1.241709in}}%
\pgfpathlineto{\pgfqpoint{1.953042in}{1.229663in}}%
\pgfpathlineto{\pgfqpoint{1.953723in}{1.225522in}}%
\pgfpathlineto{\pgfqpoint{1.955004in}{1.217617in}}%
\pgfpathlineto{\pgfqpoint{1.956795in}{1.205571in}}%
\pgfpathlineto{\pgfqpoint{1.958423in}{1.193525in}}%
\pgfpathlineto{\pgfqpoint{1.959888in}{1.181479in}}%
\pgfpathlineto{\pgfqpoint{1.961190in}{1.169433in}}%
\pgfpathlineto{\pgfqpoint{1.962330in}{1.157387in}}%
\pgfpathlineto{\pgfqpoint{1.963307in}{1.145340in}}%
\pgfpathlineto{\pgfqpoint{1.964120in}{1.133294in}}%
\pgfpathlineto{\pgfqpoint{1.964772in}{1.121248in}}%
\pgfpathlineto{\pgfqpoint{1.965260in}{1.109202in}}%
\pgfpathlineto{\pgfqpoint{1.965585in}{1.097156in}}%
\pgfpathlineto{\pgfqpoint{1.965748in}{1.085110in}}%
\pgfpathlineto{\pgfqpoint{1.965748in}{1.073064in}}%
\pgfpathlineto{\pgfqpoint{1.965585in}{1.061018in}}%
\pgfpathlineto{\pgfqpoint{1.965260in}{1.048972in}}%
\pgfpathlineto{\pgfqpoint{1.964772in}{1.036925in}}%
\pgfpathlineto{\pgfqpoint{1.964120in}{1.024879in}}%
\pgfpathlineto{\pgfqpoint{1.963307in}{1.012833in}}%
\pgfpathlineto{\pgfqpoint{1.962330in}{1.000787in}}%
\pgfpathlineto{\pgfqpoint{1.961190in}{0.988741in}}%
\pgfpathlineto{\pgfqpoint{1.959888in}{0.976695in}}%
\pgfpathlineto{\pgfqpoint{1.958423in}{0.964649in}}%
\pgfpathlineto{\pgfqpoint{1.956795in}{0.952603in}}%
\pgfpathlineto{\pgfqpoint{1.955004in}{0.940557in}}%
\pgfpathlineto{\pgfqpoint{1.953723in}{0.932651in}}%
\pgfpathlineto{\pgfqpoint{1.953042in}{0.928510in}}%
\pgfpathlineto{\pgfqpoint{1.950897in}{0.916464in}}%
\pgfpathlineto{\pgfqpoint{1.948586in}{0.904418in}}%
\pgfpathlineto{\pgfqpoint{1.946111in}{0.892372in}}%
\pgfpathlineto{\pgfqpoint{1.943471in}{0.880326in}}%
\pgfpathlineto{\pgfqpoint{1.941676in}{0.872620in}}%
\pgfpathlineto{\pgfqpoint{1.940652in}{0.868280in}}%
\pgfpathlineto{\pgfqpoint{1.937640in}{0.856234in}}%
\pgfpathlineto{\pgfqpoint{1.934461in}{0.844188in}}%
\pgfpathlineto{\pgfqpoint{1.931115in}{0.832141in}}%
\pgfpathlineto{\pgfqpoint{1.929630in}{0.827051in}}%
\pgfpathlineto{\pgfqpoint{1.927573in}{0.820095in}}%
\pgfpathlineto{\pgfqpoint{1.923841in}{0.808049in}}%
\pgfpathlineto{\pgfqpoint{1.919938in}{0.796003in}}%
\pgfpathlineto{\pgfqpoint{1.917584in}{0.789039in}}%
\pgfpathlineto{\pgfqpoint{1.915842in}{0.783957in}}%
\pgfpathlineto{\pgfqpoint{1.911540in}{0.771911in}}%
\pgfpathlineto{\pgfqpoint{1.907065in}{0.759865in}}%
\pgfpathlineto{\pgfqpoint{1.905538in}{0.755905in}}%
\pgfpathlineto{\pgfqpoint{1.902374in}{0.747819in}}%
\pgfpathlineto{\pgfqpoint{1.897485in}{0.735773in}}%
\pgfpathlineto{\pgfqpoint{1.893492in}{0.726271in}}%
\pgfpathlineto{\pgfqpoint{1.892407in}{0.723726in}}%
\pgfpathlineto{\pgfqpoint{1.887092in}{0.711680in}}%
\pgfpathlineto{\pgfqpoint{1.881601in}{0.699634in}}%
\pgfpathlineto{\pgfqpoint{1.881446in}{0.699305in}}%
\pgfpathlineto{\pgfqpoint{1.875850in}{0.687588in}}%
\pgfpathlineto{\pgfqpoint{1.869917in}{0.675542in}}%
\pgfpathlineto{\pgfqpoint{1.869400in}{0.674523in}}%
\pgfpathlineto{\pgfqpoint{1.863719in}{0.663496in}}%
\pgfpathlineto{\pgfqpoint{1.857354in}{0.651493in}}%
\pgfpathlineto{\pgfqpoint{1.857330in}{0.651450in}}%
\pgfpathlineto{\pgfqpoint{1.850659in}{0.639404in}}%
\pgfpathlineto{\pgfqpoint{1.845307in}{0.630003in}}%
\pgfpathlineto{\pgfqpoint{1.843778in}{0.627358in}}%
\pgfpathlineto{\pgfqpoint{1.836626in}{0.615311in}}%
\pgfpathlineto{\pgfqpoint{1.833261in}{0.609790in}}%
\pgfpathlineto{\pgfqpoint{1.829222in}{0.603265in}}%
\pgfpathlineto{\pgfqpoint{1.821574in}{0.591219in}}%
\pgfpathlineto{\pgfqpoint{1.821215in}{0.590668in}}%
\pgfpathlineto{\pgfqpoint{1.813614in}{0.579173in}}%
\pgfpathlineto{\pgfqpoint{1.809169in}{0.572612in}}%
\pgfpathlineto{\pgfqpoint{1.805392in}{0.567127in}}%
\pgfpathlineto{\pgfqpoint{1.797123in}{0.555396in}}%
\pgfpathlineto{\pgfqpoint{1.796897in}{0.555081in}}%
\pgfpathlineto{\pgfqpoint{1.788063in}{0.543035in}}%
\pgfpathlineto{\pgfqpoint{1.785077in}{0.539053in}}%
\pgfpathlineto{\pgfqpoint{1.778926in}{0.530989in}}%
\pgfpathlineto{\pgfqpoint{1.773031in}{0.523427in}}%
\pgfpathlineto{\pgfqpoint{1.769474in}{0.518942in}}%
\pgfpathlineto{\pgfqpoint{1.760985in}{0.508466in}}%
\pgfpathlineto{\pgfqpoint{1.759690in}{0.506896in}}%
\pgfpathlineto{\pgfqpoint{1.749546in}{0.494850in}}%
\pgfpathlineto{\pgfqpoint{1.748939in}{0.494143in}}%
\pgfpathlineto{\pgfqpoint{1.739017in}{0.482804in}}%
\pgfpathlineto{\pgfqpoint{1.736892in}{0.480425in}}%
\pgfpathlineto{\pgfqpoint{1.728104in}{0.470758in}}%
\pgfpathlineto{\pgfqpoint{1.724846in}{0.467245in}}%
\pgfpathlineto{\pgfqpoint{1.716788in}{0.458712in}}%
\pgfpathlineto{\pgfqpoint{1.712800in}{0.454571in}}%
\pgfpathlineto{\pgfqpoint{1.705044in}{0.446666in}}%
\pgfpathlineto{\pgfqpoint{1.700754in}{0.442376in}}%
\pgfpathlineto{\pgfqpoint{1.692849in}{0.434620in}}%
\pgfpathlineto{\pgfqpoint{1.688708in}{0.430632in}}%
\pgfpathlineto{\pgfqpoint{1.680175in}{0.422574in}}%
\pgfpathlineto{\pgfqpoint{1.676662in}{0.419316in}}%
\pgfpathlineto{\pgfqpoint{1.666995in}{0.410527in}}%
\pgfpathlineto{\pgfqpoint{1.664616in}{0.408403in}}%
\pgfpathlineto{\pgfqpoint{1.653276in}{0.398481in}}%
\pgfpathlineto{\pgfqpoint{1.652570in}{0.397874in}}%
\pgfpathlineto{\pgfqpoint{1.640524in}{0.387730in}}%
\pgfpathlineto{\pgfqpoint{1.638954in}{0.386435in}}%
\pgfpathlineto{\pgfqpoint{1.628477in}{0.377946in}}%
\pgfpathlineto{\pgfqpoint{1.623993in}{0.374389in}}%
\pgfpathlineto{\pgfqpoint{1.616431in}{0.368494in}}%
\pgfpathlineto{\pgfqpoint{1.608367in}{0.362343in}}%
\pgfpathlineto{\pgfqpoint{1.604385in}{0.359357in}}%
\pgfpathlineto{\pgfqpoint{1.592339in}{0.350523in}}%
\pgfpathlineto{\pgfqpoint{1.592024in}{0.350297in}}%
\pgfpathlineto{\pgfqpoint{1.580293in}{0.342028in}}%
\pgfpathlineto{\pgfqpoint{1.574808in}{0.338251in}}%
\pgfpathlineto{\pgfqpoint{1.568247in}{0.333806in}}%
\pgfpathlineto{\pgfqpoint{1.556752in}{0.326205in}}%
\pgfpathlineto{\pgfqpoint{1.556201in}{0.325846in}}%
\pgfpathlineto{\pgfqpoint{1.544155in}{0.318198in}}%
\pgfpathlineto{\pgfqpoint{1.537630in}{0.314159in}}%
\pgfpathlineto{\pgfqpoint{1.532108in}{0.310794in}}%
\pgfpathlineto{\pgfqpoint{1.520062in}{0.303642in}}%
\pgfpathlineto{\pgfqpoint{1.517417in}{0.302112in}}%
\pgfpathlineto{\pgfqpoint{1.508016in}{0.296761in}}%
\pgfpathlineto{\pgfqpoint{1.495970in}{0.290089in}}%
\pgfpathlineto{\pgfqpoint{1.495927in}{0.290066in}}%
\pgfpathlineto{\pgfqpoint{1.483924in}{0.283701in}}%
\pgfpathlineto{\pgfqpoint{1.472897in}{0.278020in}}%
\pgfpathlineto{\pgfqpoint{1.471878in}{0.277503in}}%
\pgfpathlineto{\pgfqpoint{1.459832in}{0.271570in}}%
\pgfpathlineto{\pgfqpoint{1.448115in}{0.265974in}}%
\pgfpathlineto{\pgfqpoint{1.447786in}{0.265819in}}%
\pgfpathlineto{\pgfqpoint{1.435740in}{0.260327in}}%
\pgfpathlineto{\pgfqpoint{1.423693in}{0.255013in}}%
\pgfpathlineto{\pgfqpoint{1.421149in}{0.253928in}}%
\pgfpathlineto{\pgfqpoint{1.411647in}{0.249934in}}%
\pgfpathlineto{\pgfqpoint{1.399601in}{0.245046in}}%
\pgfpathlineto{\pgfqpoint{1.391515in}{0.241882in}}%
\pgfpathlineto{\pgfqpoint{1.387555in}{0.240355in}}%
\pgfpathlineto{\pgfqpoint{1.375509in}{0.235880in}}%
\pgfpathlineto{\pgfqpoint{1.363463in}{0.231578in}}%
\pgfpathlineto{\pgfqpoint{1.358381in}{0.229836in}}%
\pgfpathlineto{\pgfqpoint{1.351417in}{0.227482in}}%
\pgfpathlineto{\pgfqpoint{1.339371in}{0.223579in}}%
\pgfpathlineto{\pgfqpoint{1.327325in}{0.219847in}}%
\pgfpathlineto{\pgfqpoint{1.320369in}{0.217790in}}%
\pgfpathlineto{\pgfqpoint{1.315278in}{0.216305in}}%
\pgfpathlineto{\pgfqpoint{1.303232in}{0.212959in}}%
\pgfpathlineto{\pgfqpoint{1.291186in}{0.209780in}}%
\pgfpathlineto{\pgfqpoint{1.279140in}{0.206768in}}%
\pgfpathlineto{\pgfqpoint{1.274800in}{0.205743in}}%
\pgfpathlineto{\pgfqpoint{1.267094in}{0.203949in}}%
\pgfpathlineto{\pgfqpoint{1.255048in}{0.201309in}}%
\pgfpathlineto{\pgfqpoint{1.243002in}{0.198833in}}%
\pgfpathlineto{\pgfqpoint{1.230956in}{0.196523in}}%
\pgfpathlineto{\pgfqpoint{1.218909in}{0.194378in}}%
\pgfpathlineto{\pgfqpoint{1.214769in}{0.193697in}}%
\pgfpathlineto{\pgfqpoint{1.206863in}{0.192415in}}%
\pgfpathlineto{\pgfqpoint{1.194817in}{0.190625in}}%
\pgfpathlineto{\pgfqpoint{1.182771in}{0.188997in}}%
\pgfpathlineto{\pgfqpoint{1.170725in}{0.187532in}}%
\pgfpathlineto{\pgfqpoint{1.158679in}{0.186230in}}%
\pgfpathlineto{\pgfqpoint{1.146633in}{0.185090in}}%
\pgfpathlineto{\pgfqpoint{1.134587in}{0.184113in}}%
\pgfpathlineto{\pgfqpoint{1.122541in}{0.183299in}}%
\pgfpathlineto{\pgfqpoint{1.110494in}{0.182648in}}%
\pgfpathlineto{\pgfqpoint{1.098448in}{0.182160in}}%
\pgfpathlineto{\pgfqpoint{1.086402in}{0.181834in}}%
\pgfpathlineto{\pgfqpoint{1.074356in}{0.181672in}}%
\pgfpathlineto{\pgfqpoint{1.062310in}{0.181672in}}%
\pgfpathlineto{\pgfqpoint{1.050264in}{0.181834in}}%
\pgfpathlineto{\pgfqpoint{1.038218in}{0.182160in}}%
\pgfpathlineto{\pgfqpoint{1.026172in}{0.182648in}}%
\pgfpathlineto{\pgfqpoint{1.014126in}{0.183299in}}%
\pgfpathlineto{\pgfqpoint{1.002079in}{0.184113in}}%
\pgfpathlineto{\pgfqpoint{0.990033in}{0.185090in}}%
\pgfpathlineto{\pgfqpoint{0.977987in}{0.186230in}}%
\pgfpathlineto{\pgfqpoint{0.965941in}{0.187532in}}%
\pgfpathlineto{\pgfqpoint{0.953895in}{0.188997in}}%
\pgfpathlineto{\pgfqpoint{0.941849in}{0.190625in}}%
\pgfpathlineto{\pgfqpoint{0.929803in}{0.192415in}}%
\pgfpathclose%
\pgfusepath{fill}%
\end{pgfscope}%
\begin{pgfscope}%
\pgfpathrectangle{\pgfqpoint{0.135000in}{0.145754in}}{\pgfqpoint{1.866666in}{1.866666in}} %
\pgfusepath{clip}%
\pgfsetbuttcap%
\pgfsetroundjoin%
\pgfsetlinewidth{0.000000pt}%
\definecolor{currentstroke}{rgb}{0.000000,0.000000,0.000000}%
\pgfsetstrokecolor{currentstroke}%
\pgfsetdash{}{0pt}%
\pgfpathmoveto{\pgfqpoint{0.182944in}{0.181651in}}%
\pgfpathlineto{\pgfqpoint{0.194990in}{0.181651in}}%
\pgfpathlineto{\pgfqpoint{0.207036in}{0.181651in}}%
\pgfpathlineto{\pgfqpoint{0.219082in}{0.181651in}}%
\pgfpathlineto{\pgfqpoint{0.231128in}{0.181651in}}%
\pgfpathlineto{\pgfqpoint{0.243174in}{0.181651in}}%
\pgfpathlineto{\pgfqpoint{0.255220in}{0.181651in}}%
\pgfpathlineto{\pgfqpoint{0.267266in}{0.181651in}}%
\pgfpathlineto{\pgfqpoint{0.279312in}{0.181651in}}%
\pgfpathlineto{\pgfqpoint{0.291359in}{0.181651in}}%
\pgfpathlineto{\pgfqpoint{0.303405in}{0.181651in}}%
\pgfpathlineto{\pgfqpoint{0.315451in}{0.181651in}}%
\pgfpathlineto{\pgfqpoint{0.327497in}{0.181651in}}%
\pgfpathlineto{\pgfqpoint{0.339543in}{0.181651in}}%
\pgfpathlineto{\pgfqpoint{0.351589in}{0.181651in}}%
\pgfpathlineto{\pgfqpoint{0.363635in}{0.181651in}}%
\pgfpathlineto{\pgfqpoint{0.375681in}{0.181651in}}%
\pgfpathlineto{\pgfqpoint{0.387728in}{0.181651in}}%
\pgfpathlineto{\pgfqpoint{0.399774in}{0.181651in}}%
\pgfpathlineto{\pgfqpoint{0.411820in}{0.181651in}}%
\pgfpathlineto{\pgfqpoint{0.423866in}{0.181651in}}%
\pgfpathlineto{\pgfqpoint{0.435912in}{0.181651in}}%
\pgfpathlineto{\pgfqpoint{0.447958in}{0.181651in}}%
\pgfpathlineto{\pgfqpoint{0.460004in}{0.181651in}}%
\pgfpathlineto{\pgfqpoint{0.472050in}{0.181651in}}%
\pgfpathlineto{\pgfqpoint{0.484096in}{0.181651in}}%
\pgfpathlineto{\pgfqpoint{0.496143in}{0.181651in}}%
\pgfpathlineto{\pgfqpoint{0.508189in}{0.181651in}}%
\pgfpathlineto{\pgfqpoint{0.520235in}{0.181651in}}%
\pgfpathlineto{\pgfqpoint{0.532281in}{0.181651in}}%
\pgfpathlineto{\pgfqpoint{0.544327in}{0.181651in}}%
\pgfpathlineto{\pgfqpoint{0.556373in}{0.181651in}}%
\pgfpathlineto{\pgfqpoint{0.568419in}{0.181651in}}%
\pgfpathlineto{\pgfqpoint{0.580465in}{0.181651in}}%
\pgfpathlineto{\pgfqpoint{0.592511in}{0.181651in}}%
\pgfpathlineto{\pgfqpoint{0.604558in}{0.181651in}}%
\pgfpathlineto{\pgfqpoint{0.616604in}{0.181651in}}%
\pgfpathlineto{\pgfqpoint{0.628650in}{0.181651in}}%
\pgfpathlineto{\pgfqpoint{0.640696in}{0.181651in}}%
\pgfpathlineto{\pgfqpoint{0.652742in}{0.181651in}}%
\pgfpathlineto{\pgfqpoint{0.664788in}{0.181651in}}%
\pgfpathlineto{\pgfqpoint{0.676834in}{0.181651in}}%
\pgfpathlineto{\pgfqpoint{0.688880in}{0.181651in}}%
\pgfpathlineto{\pgfqpoint{0.700927in}{0.181651in}}%
\pgfpathlineto{\pgfqpoint{0.712973in}{0.181651in}}%
\pgfpathlineto{\pgfqpoint{0.725019in}{0.181651in}}%
\pgfpathlineto{\pgfqpoint{0.737065in}{0.181651in}}%
\pgfpathlineto{\pgfqpoint{0.749111in}{0.181651in}}%
\pgfpathlineto{\pgfqpoint{0.761157in}{0.181651in}}%
\pgfpathlineto{\pgfqpoint{0.773203in}{0.181651in}}%
\pgfpathlineto{\pgfqpoint{0.785249in}{0.181651in}}%
\pgfpathlineto{\pgfqpoint{0.797295in}{0.181651in}}%
\pgfpathlineto{\pgfqpoint{0.809342in}{0.181651in}}%
\pgfpathlineto{\pgfqpoint{0.821388in}{0.181651in}}%
\pgfpathlineto{\pgfqpoint{0.833434in}{0.181651in}}%
\pgfpathlineto{\pgfqpoint{0.845480in}{0.181651in}}%
\pgfpathlineto{\pgfqpoint{0.857526in}{0.181651in}}%
\pgfpathlineto{\pgfqpoint{0.869572in}{0.181651in}}%
\pgfpathlineto{\pgfqpoint{0.881618in}{0.181651in}}%
\pgfpathlineto{\pgfqpoint{0.893664in}{0.181651in}}%
\pgfpathlineto{\pgfqpoint{0.905710in}{0.181651in}}%
\pgfpathlineto{\pgfqpoint{0.917757in}{0.181651in}}%
\pgfpathlineto{\pgfqpoint{0.929803in}{0.181651in}}%
\pgfpathlineto{\pgfqpoint{0.941849in}{0.181651in}}%
\pgfpathlineto{\pgfqpoint{0.953895in}{0.181651in}}%
\pgfpathlineto{\pgfqpoint{0.965941in}{0.181651in}}%
\pgfpathlineto{\pgfqpoint{0.977987in}{0.181651in}}%
\pgfpathlineto{\pgfqpoint{0.990033in}{0.181651in}}%
\pgfpathlineto{\pgfqpoint{1.002079in}{0.181651in}}%
\pgfpathlineto{\pgfqpoint{1.014126in}{0.181651in}}%
\pgfpathlineto{\pgfqpoint{1.026172in}{0.181651in}}%
\pgfpathlineto{\pgfqpoint{1.038218in}{0.181651in}}%
\pgfpathlineto{\pgfqpoint{1.050264in}{0.181651in}}%
\pgfpathlineto{\pgfqpoint{1.062310in}{0.181651in}}%
\pgfpathlineto{\pgfqpoint{1.074356in}{0.181651in}}%
\pgfpathlineto{\pgfqpoint{1.086402in}{0.181651in}}%
\pgfpathlineto{\pgfqpoint{1.098448in}{0.181651in}}%
\pgfpathlineto{\pgfqpoint{1.110494in}{0.181651in}}%
\pgfpathlineto{\pgfqpoint{1.122541in}{0.181651in}}%
\pgfpathlineto{\pgfqpoint{1.134587in}{0.181651in}}%
\pgfpathlineto{\pgfqpoint{1.146633in}{0.181651in}}%
\pgfpathlineto{\pgfqpoint{1.158679in}{0.181651in}}%
\pgfpathlineto{\pgfqpoint{1.170725in}{0.181651in}}%
\pgfpathlineto{\pgfqpoint{1.182771in}{0.181651in}}%
\pgfpathlineto{\pgfqpoint{1.194817in}{0.181651in}}%
\pgfpathlineto{\pgfqpoint{1.206863in}{0.181651in}}%
\pgfpathlineto{\pgfqpoint{1.218909in}{0.181651in}}%
\pgfpathlineto{\pgfqpoint{1.230956in}{0.181651in}}%
\pgfpathlineto{\pgfqpoint{1.243002in}{0.181651in}}%
\pgfpathlineto{\pgfqpoint{1.255048in}{0.181651in}}%
\pgfpathlineto{\pgfqpoint{1.267094in}{0.181651in}}%
\pgfpathlineto{\pgfqpoint{1.279140in}{0.181651in}}%
\pgfpathlineto{\pgfqpoint{1.291186in}{0.181651in}}%
\pgfpathlineto{\pgfqpoint{1.303232in}{0.181651in}}%
\pgfpathlineto{\pgfqpoint{1.315278in}{0.181651in}}%
\pgfpathlineto{\pgfqpoint{1.327325in}{0.181651in}}%
\pgfpathlineto{\pgfqpoint{1.339371in}{0.181651in}}%
\pgfpathlineto{\pgfqpoint{1.351417in}{0.181651in}}%
\pgfpathlineto{\pgfqpoint{1.363463in}{0.181651in}}%
\pgfpathlineto{\pgfqpoint{1.375509in}{0.181651in}}%
\pgfpathlineto{\pgfqpoint{1.387555in}{0.181651in}}%
\pgfpathlineto{\pgfqpoint{1.399601in}{0.181651in}}%
\pgfpathlineto{\pgfqpoint{1.411647in}{0.181651in}}%
\pgfpathlineto{\pgfqpoint{1.423693in}{0.181651in}}%
\pgfpathlineto{\pgfqpoint{1.435740in}{0.181651in}}%
\pgfpathlineto{\pgfqpoint{1.447786in}{0.181651in}}%
\pgfpathlineto{\pgfqpoint{1.459832in}{0.181651in}}%
\pgfpathlineto{\pgfqpoint{1.471878in}{0.181651in}}%
\pgfpathlineto{\pgfqpoint{1.483924in}{0.181651in}}%
\pgfpathlineto{\pgfqpoint{1.495970in}{0.181651in}}%
\pgfpathlineto{\pgfqpoint{1.508016in}{0.181651in}}%
\pgfpathlineto{\pgfqpoint{1.520062in}{0.181651in}}%
\pgfpathlineto{\pgfqpoint{1.532108in}{0.181651in}}%
\pgfpathlineto{\pgfqpoint{1.544155in}{0.181651in}}%
\pgfpathlineto{\pgfqpoint{1.556201in}{0.181651in}}%
\pgfpathlineto{\pgfqpoint{1.568247in}{0.181651in}}%
\pgfpathlineto{\pgfqpoint{1.580293in}{0.181651in}}%
\pgfpathlineto{\pgfqpoint{1.592339in}{0.181651in}}%
\pgfpathlineto{\pgfqpoint{1.604385in}{0.181651in}}%
\pgfpathlineto{\pgfqpoint{1.616431in}{0.181651in}}%
\pgfpathlineto{\pgfqpoint{1.628477in}{0.181651in}}%
\pgfpathlineto{\pgfqpoint{1.640524in}{0.181651in}}%
\pgfpathlineto{\pgfqpoint{1.652570in}{0.181651in}}%
\pgfpathlineto{\pgfqpoint{1.664616in}{0.181651in}}%
\pgfpathlineto{\pgfqpoint{1.676662in}{0.181651in}}%
\pgfpathlineto{\pgfqpoint{1.688708in}{0.181651in}}%
\pgfpathlineto{\pgfqpoint{1.700754in}{0.181651in}}%
\pgfpathlineto{\pgfqpoint{1.712800in}{0.181651in}}%
\pgfpathlineto{\pgfqpoint{1.724846in}{0.181651in}}%
\pgfpathlineto{\pgfqpoint{1.736892in}{0.181651in}}%
\pgfpathlineto{\pgfqpoint{1.748939in}{0.181651in}}%
\pgfpathlineto{\pgfqpoint{1.760985in}{0.181651in}}%
\pgfpathlineto{\pgfqpoint{1.773031in}{0.181651in}}%
\pgfpathlineto{\pgfqpoint{1.785077in}{0.181651in}}%
\pgfpathlineto{\pgfqpoint{1.797123in}{0.181651in}}%
\pgfpathlineto{\pgfqpoint{1.809169in}{0.181651in}}%
\pgfpathlineto{\pgfqpoint{1.821215in}{0.181651in}}%
\pgfpathlineto{\pgfqpoint{1.833261in}{0.181651in}}%
\pgfpathlineto{\pgfqpoint{1.845307in}{0.181651in}}%
\pgfpathlineto{\pgfqpoint{1.857354in}{0.181651in}}%
\pgfpathlineto{\pgfqpoint{1.869400in}{0.181651in}}%
\pgfpathlineto{\pgfqpoint{1.881446in}{0.181651in}}%
\pgfpathlineto{\pgfqpoint{1.893492in}{0.181651in}}%
\pgfpathlineto{\pgfqpoint{1.905538in}{0.181651in}}%
\pgfpathlineto{\pgfqpoint{1.917584in}{0.181651in}}%
\pgfpathlineto{\pgfqpoint{1.929630in}{0.181651in}}%
\pgfpathlineto{\pgfqpoint{1.941676in}{0.181651in}}%
\pgfpathlineto{\pgfqpoint{1.953723in}{0.181651in}}%
\pgfpathlineto{\pgfqpoint{1.965769in}{0.181651in}}%
\pgfpathlineto{\pgfqpoint{1.965769in}{0.193697in}}%
\pgfpathlineto{\pgfqpoint{1.965769in}{0.205743in}}%
\pgfpathlineto{\pgfqpoint{1.965769in}{0.217790in}}%
\pgfpathlineto{\pgfqpoint{1.965769in}{0.229836in}}%
\pgfpathlineto{\pgfqpoint{1.965769in}{0.241882in}}%
\pgfpathlineto{\pgfqpoint{1.965769in}{0.253928in}}%
\pgfpathlineto{\pgfqpoint{1.965769in}{0.265974in}}%
\pgfpathlineto{\pgfqpoint{1.965769in}{0.278020in}}%
\pgfpathlineto{\pgfqpoint{1.965769in}{0.290066in}}%
\pgfpathlineto{\pgfqpoint{1.965769in}{0.302112in}}%
\pgfpathlineto{\pgfqpoint{1.965769in}{0.314159in}}%
\pgfpathlineto{\pgfqpoint{1.965769in}{0.326205in}}%
\pgfpathlineto{\pgfqpoint{1.965769in}{0.338251in}}%
\pgfpathlineto{\pgfqpoint{1.965769in}{0.350297in}}%
\pgfpathlineto{\pgfqpoint{1.965769in}{0.362343in}}%
\pgfpathlineto{\pgfqpoint{1.965769in}{0.374389in}}%
\pgfpathlineto{\pgfqpoint{1.965769in}{0.386435in}}%
\pgfpathlineto{\pgfqpoint{1.965769in}{0.398481in}}%
\pgfpathlineto{\pgfqpoint{1.965769in}{0.410527in}}%
\pgfpathlineto{\pgfqpoint{1.965769in}{0.422574in}}%
\pgfpathlineto{\pgfqpoint{1.965769in}{0.434620in}}%
\pgfpathlineto{\pgfqpoint{1.965769in}{0.446666in}}%
\pgfpathlineto{\pgfqpoint{1.965769in}{0.458712in}}%
\pgfpathlineto{\pgfqpoint{1.965769in}{0.470758in}}%
\pgfpathlineto{\pgfqpoint{1.965769in}{0.482804in}}%
\pgfpathlineto{\pgfqpoint{1.965769in}{0.494850in}}%
\pgfpathlineto{\pgfqpoint{1.965769in}{0.506896in}}%
\pgfpathlineto{\pgfqpoint{1.965769in}{0.518942in}}%
\pgfpathlineto{\pgfqpoint{1.965769in}{0.530989in}}%
\pgfpathlineto{\pgfqpoint{1.965769in}{0.543035in}}%
\pgfpathlineto{\pgfqpoint{1.965769in}{0.555081in}}%
\pgfpathlineto{\pgfqpoint{1.965769in}{0.567127in}}%
\pgfpathlineto{\pgfqpoint{1.965769in}{0.579173in}}%
\pgfpathlineto{\pgfqpoint{1.965769in}{0.591219in}}%
\pgfpathlineto{\pgfqpoint{1.965769in}{0.603265in}}%
\pgfpathlineto{\pgfqpoint{1.965769in}{0.615311in}}%
\pgfpathlineto{\pgfqpoint{1.965769in}{0.627358in}}%
\pgfpathlineto{\pgfqpoint{1.965769in}{0.639404in}}%
\pgfpathlineto{\pgfqpoint{1.965769in}{0.651450in}}%
\pgfpathlineto{\pgfqpoint{1.965769in}{0.663496in}}%
\pgfpathlineto{\pgfqpoint{1.965769in}{0.675542in}}%
\pgfpathlineto{\pgfqpoint{1.965769in}{0.687588in}}%
\pgfpathlineto{\pgfqpoint{1.965769in}{0.699634in}}%
\pgfpathlineto{\pgfqpoint{1.965769in}{0.711680in}}%
\pgfpathlineto{\pgfqpoint{1.965769in}{0.723726in}}%
\pgfpathlineto{\pgfqpoint{1.965769in}{0.735773in}}%
\pgfpathlineto{\pgfqpoint{1.965769in}{0.747819in}}%
\pgfpathlineto{\pgfqpoint{1.965769in}{0.759865in}}%
\pgfpathlineto{\pgfqpoint{1.965769in}{0.771911in}}%
\pgfpathlineto{\pgfqpoint{1.965769in}{0.783957in}}%
\pgfpathlineto{\pgfqpoint{1.965769in}{0.796003in}}%
\pgfpathlineto{\pgfqpoint{1.965769in}{0.808049in}}%
\pgfpathlineto{\pgfqpoint{1.965769in}{0.820095in}}%
\pgfpathlineto{\pgfqpoint{1.965769in}{0.832141in}}%
\pgfpathlineto{\pgfqpoint{1.965769in}{0.844188in}}%
\pgfpathlineto{\pgfqpoint{1.965769in}{0.856234in}}%
\pgfpathlineto{\pgfqpoint{1.965769in}{0.868280in}}%
\pgfpathlineto{\pgfqpoint{1.965769in}{0.880326in}}%
\pgfpathlineto{\pgfqpoint{1.965769in}{0.892372in}}%
\pgfpathlineto{\pgfqpoint{1.965769in}{0.904418in}}%
\pgfpathlineto{\pgfqpoint{1.965769in}{0.916464in}}%
\pgfpathlineto{\pgfqpoint{1.965769in}{0.928510in}}%
\pgfpathlineto{\pgfqpoint{1.965769in}{0.940557in}}%
\pgfpathlineto{\pgfqpoint{1.965769in}{0.952603in}}%
\pgfpathlineto{\pgfqpoint{1.965769in}{0.964649in}}%
\pgfpathlineto{\pgfqpoint{1.965769in}{0.976695in}}%
\pgfpathlineto{\pgfqpoint{1.965769in}{0.988741in}}%
\pgfpathlineto{\pgfqpoint{1.965769in}{1.000787in}}%
\pgfpathlineto{\pgfqpoint{1.965769in}{1.012833in}}%
\pgfpathlineto{\pgfqpoint{1.965769in}{1.024879in}}%
\pgfpathlineto{\pgfqpoint{1.965769in}{1.036925in}}%
\pgfpathlineto{\pgfqpoint{1.965769in}{1.048972in}}%
\pgfpathlineto{\pgfqpoint{1.965769in}{1.061018in}}%
\pgfpathlineto{\pgfqpoint{1.965769in}{1.073064in}}%
\pgfpathlineto{\pgfqpoint{1.965769in}{1.085110in}}%
\pgfpathlineto{\pgfqpoint{1.965769in}{1.097156in}}%
\pgfpathlineto{\pgfqpoint{1.965769in}{1.109202in}}%
\pgfpathlineto{\pgfqpoint{1.965769in}{1.121248in}}%
\pgfpathlineto{\pgfqpoint{1.965769in}{1.133294in}}%
\pgfpathlineto{\pgfqpoint{1.965769in}{1.145340in}}%
\pgfpathlineto{\pgfqpoint{1.965769in}{1.157387in}}%
\pgfpathlineto{\pgfqpoint{1.965769in}{1.169433in}}%
\pgfpathlineto{\pgfqpoint{1.965769in}{1.181479in}}%
\pgfpathlineto{\pgfqpoint{1.965769in}{1.193525in}}%
\pgfpathlineto{\pgfqpoint{1.965769in}{1.205571in}}%
\pgfpathlineto{\pgfqpoint{1.965769in}{1.217617in}}%
\pgfpathlineto{\pgfqpoint{1.965769in}{1.229663in}}%
\pgfpathlineto{\pgfqpoint{1.965769in}{1.241709in}}%
\pgfpathlineto{\pgfqpoint{1.965769in}{1.253756in}}%
\pgfpathlineto{\pgfqpoint{1.965769in}{1.265802in}}%
\pgfpathlineto{\pgfqpoint{1.965769in}{1.277848in}}%
\pgfpathlineto{\pgfqpoint{1.965769in}{1.289894in}}%
\pgfpathlineto{\pgfqpoint{1.965769in}{1.301940in}}%
\pgfpathlineto{\pgfqpoint{1.965769in}{1.313986in}}%
\pgfpathlineto{\pgfqpoint{1.965769in}{1.326032in}}%
\pgfpathlineto{\pgfqpoint{1.965769in}{1.338078in}}%
\pgfpathlineto{\pgfqpoint{1.965769in}{1.350124in}}%
\pgfpathlineto{\pgfqpoint{1.965769in}{1.362171in}}%
\pgfpathlineto{\pgfqpoint{1.965769in}{1.374217in}}%
\pgfpathlineto{\pgfqpoint{1.965769in}{1.386263in}}%
\pgfpathlineto{\pgfqpoint{1.965769in}{1.398309in}}%
\pgfpathlineto{\pgfqpoint{1.965769in}{1.410355in}}%
\pgfpathlineto{\pgfqpoint{1.965769in}{1.422401in}}%
\pgfpathlineto{\pgfqpoint{1.965769in}{1.434447in}}%
\pgfpathlineto{\pgfqpoint{1.965769in}{1.446493in}}%
\pgfpathlineto{\pgfqpoint{1.965769in}{1.458539in}}%
\pgfpathlineto{\pgfqpoint{1.965769in}{1.470586in}}%
\pgfpathlineto{\pgfqpoint{1.965769in}{1.482632in}}%
\pgfpathlineto{\pgfqpoint{1.965769in}{1.494678in}}%
\pgfpathlineto{\pgfqpoint{1.965769in}{1.506724in}}%
\pgfpathlineto{\pgfqpoint{1.965769in}{1.518770in}}%
\pgfpathlineto{\pgfqpoint{1.965769in}{1.530816in}}%
\pgfpathlineto{\pgfqpoint{1.965769in}{1.542862in}}%
\pgfpathlineto{\pgfqpoint{1.965769in}{1.554908in}}%
\pgfpathlineto{\pgfqpoint{1.965769in}{1.566955in}}%
\pgfpathlineto{\pgfqpoint{1.965769in}{1.579001in}}%
\pgfpathlineto{\pgfqpoint{1.965769in}{1.591047in}}%
\pgfpathlineto{\pgfqpoint{1.965769in}{1.603093in}}%
\pgfpathlineto{\pgfqpoint{1.965769in}{1.615139in}}%
\pgfpathlineto{\pgfqpoint{1.965769in}{1.627185in}}%
\pgfpathlineto{\pgfqpoint{1.965769in}{1.639231in}}%
\pgfpathlineto{\pgfqpoint{1.965769in}{1.651277in}}%
\pgfpathlineto{\pgfqpoint{1.965769in}{1.663323in}}%
\pgfpathlineto{\pgfqpoint{1.965769in}{1.675370in}}%
\pgfpathlineto{\pgfqpoint{1.965769in}{1.687416in}}%
\pgfpathlineto{\pgfqpoint{1.965769in}{1.699462in}}%
\pgfpathlineto{\pgfqpoint{1.965769in}{1.711508in}}%
\pgfpathlineto{\pgfqpoint{1.965769in}{1.723554in}}%
\pgfpathlineto{\pgfqpoint{1.965769in}{1.735600in}}%
\pgfpathlineto{\pgfqpoint{1.965769in}{1.747646in}}%
\pgfpathlineto{\pgfqpoint{1.965769in}{1.759692in}}%
\pgfpathlineto{\pgfqpoint{1.965769in}{1.771738in}}%
\pgfpathlineto{\pgfqpoint{1.965769in}{1.783785in}}%
\pgfpathlineto{\pgfqpoint{1.965769in}{1.795831in}}%
\pgfpathlineto{\pgfqpoint{1.965769in}{1.807877in}}%
\pgfpathlineto{\pgfqpoint{1.965769in}{1.819923in}}%
\pgfpathlineto{\pgfqpoint{1.965769in}{1.831969in}}%
\pgfpathlineto{\pgfqpoint{1.965769in}{1.844015in}}%
\pgfpathlineto{\pgfqpoint{1.965769in}{1.856061in}}%
\pgfpathlineto{\pgfqpoint{1.965769in}{1.868107in}}%
\pgfpathlineto{\pgfqpoint{1.965769in}{1.880154in}}%
\pgfpathlineto{\pgfqpoint{1.965769in}{1.892200in}}%
\pgfpathlineto{\pgfqpoint{1.965769in}{1.904246in}}%
\pgfpathlineto{\pgfqpoint{1.965769in}{1.916292in}}%
\pgfpathlineto{\pgfqpoint{1.965769in}{1.928338in}}%
\pgfpathlineto{\pgfqpoint{1.965769in}{1.940384in}}%
\pgfpathlineto{\pgfqpoint{1.965769in}{1.952430in}}%
\pgfpathlineto{\pgfqpoint{1.965769in}{1.964476in}}%
\pgfpathlineto{\pgfqpoint{1.965769in}{1.976522in}}%
\pgfpathlineto{\pgfqpoint{1.953723in}{1.976522in}}%
\pgfpathlineto{\pgfqpoint{1.941676in}{1.976522in}}%
\pgfpathlineto{\pgfqpoint{1.929630in}{1.976522in}}%
\pgfpathlineto{\pgfqpoint{1.917584in}{1.976522in}}%
\pgfpathlineto{\pgfqpoint{1.905538in}{1.976522in}}%
\pgfpathlineto{\pgfqpoint{1.893492in}{1.976522in}}%
\pgfpathlineto{\pgfqpoint{1.881446in}{1.976522in}}%
\pgfpathlineto{\pgfqpoint{1.869400in}{1.976522in}}%
\pgfpathlineto{\pgfqpoint{1.857354in}{1.976522in}}%
\pgfpathlineto{\pgfqpoint{1.845307in}{1.976522in}}%
\pgfpathlineto{\pgfqpoint{1.833261in}{1.976522in}}%
\pgfpathlineto{\pgfqpoint{1.821215in}{1.976522in}}%
\pgfpathlineto{\pgfqpoint{1.809169in}{1.976522in}}%
\pgfpathlineto{\pgfqpoint{1.797123in}{1.976522in}}%
\pgfpathlineto{\pgfqpoint{1.785077in}{1.976522in}}%
\pgfpathlineto{\pgfqpoint{1.773031in}{1.976522in}}%
\pgfpathlineto{\pgfqpoint{1.760985in}{1.976522in}}%
\pgfpathlineto{\pgfqpoint{1.748939in}{1.976522in}}%
\pgfpathlineto{\pgfqpoint{1.736892in}{1.976522in}}%
\pgfpathlineto{\pgfqpoint{1.724846in}{1.976522in}}%
\pgfpathlineto{\pgfqpoint{1.712800in}{1.976522in}}%
\pgfpathlineto{\pgfqpoint{1.700754in}{1.976522in}}%
\pgfpathlineto{\pgfqpoint{1.688708in}{1.976522in}}%
\pgfpathlineto{\pgfqpoint{1.676662in}{1.976522in}}%
\pgfpathlineto{\pgfqpoint{1.664616in}{1.976522in}}%
\pgfpathlineto{\pgfqpoint{1.652570in}{1.976522in}}%
\pgfpathlineto{\pgfqpoint{1.640524in}{1.976522in}}%
\pgfpathlineto{\pgfqpoint{1.628477in}{1.976522in}}%
\pgfpathlineto{\pgfqpoint{1.616431in}{1.976522in}}%
\pgfpathlineto{\pgfqpoint{1.604385in}{1.976522in}}%
\pgfpathlineto{\pgfqpoint{1.592339in}{1.976522in}}%
\pgfpathlineto{\pgfqpoint{1.580293in}{1.976522in}}%
\pgfpathlineto{\pgfqpoint{1.568247in}{1.976522in}}%
\pgfpathlineto{\pgfqpoint{1.556201in}{1.976522in}}%
\pgfpathlineto{\pgfqpoint{1.544155in}{1.976522in}}%
\pgfpathlineto{\pgfqpoint{1.532108in}{1.976522in}}%
\pgfpathlineto{\pgfqpoint{1.520062in}{1.976522in}}%
\pgfpathlineto{\pgfqpoint{1.508016in}{1.976522in}}%
\pgfpathlineto{\pgfqpoint{1.495970in}{1.976522in}}%
\pgfpathlineto{\pgfqpoint{1.483924in}{1.976522in}}%
\pgfpathlineto{\pgfqpoint{1.471878in}{1.976522in}}%
\pgfpathlineto{\pgfqpoint{1.459832in}{1.976522in}}%
\pgfpathlineto{\pgfqpoint{1.447786in}{1.976522in}}%
\pgfpathlineto{\pgfqpoint{1.435740in}{1.976522in}}%
\pgfpathlineto{\pgfqpoint{1.423693in}{1.976522in}}%
\pgfpathlineto{\pgfqpoint{1.411647in}{1.976522in}}%
\pgfpathlineto{\pgfqpoint{1.399601in}{1.976522in}}%
\pgfpathlineto{\pgfqpoint{1.387555in}{1.976522in}}%
\pgfpathlineto{\pgfqpoint{1.375509in}{1.976522in}}%
\pgfpathlineto{\pgfqpoint{1.363463in}{1.976522in}}%
\pgfpathlineto{\pgfqpoint{1.351417in}{1.976522in}}%
\pgfpathlineto{\pgfqpoint{1.339371in}{1.976522in}}%
\pgfpathlineto{\pgfqpoint{1.327325in}{1.976522in}}%
\pgfpathlineto{\pgfqpoint{1.315278in}{1.976522in}}%
\pgfpathlineto{\pgfqpoint{1.303232in}{1.976522in}}%
\pgfpathlineto{\pgfqpoint{1.291186in}{1.976522in}}%
\pgfpathlineto{\pgfqpoint{1.279140in}{1.976522in}}%
\pgfpathlineto{\pgfqpoint{1.267094in}{1.976522in}}%
\pgfpathlineto{\pgfqpoint{1.255048in}{1.976522in}}%
\pgfpathlineto{\pgfqpoint{1.243002in}{1.976522in}}%
\pgfpathlineto{\pgfqpoint{1.230956in}{1.976522in}}%
\pgfpathlineto{\pgfqpoint{1.218909in}{1.976522in}}%
\pgfpathlineto{\pgfqpoint{1.206863in}{1.976522in}}%
\pgfpathlineto{\pgfqpoint{1.194817in}{1.976522in}}%
\pgfpathlineto{\pgfqpoint{1.182771in}{1.976522in}}%
\pgfpathlineto{\pgfqpoint{1.170725in}{1.976522in}}%
\pgfpathlineto{\pgfqpoint{1.158679in}{1.976522in}}%
\pgfpathlineto{\pgfqpoint{1.146633in}{1.976522in}}%
\pgfpathlineto{\pgfqpoint{1.134587in}{1.976522in}}%
\pgfpathlineto{\pgfqpoint{1.122541in}{1.976522in}}%
\pgfpathlineto{\pgfqpoint{1.110494in}{1.976522in}}%
\pgfpathlineto{\pgfqpoint{1.098448in}{1.976522in}}%
\pgfpathlineto{\pgfqpoint{1.086402in}{1.976522in}}%
\pgfpathlineto{\pgfqpoint{1.074356in}{1.976522in}}%
\pgfpathlineto{\pgfqpoint{1.062310in}{1.976522in}}%
\pgfpathlineto{\pgfqpoint{1.050264in}{1.976522in}}%
\pgfpathlineto{\pgfqpoint{1.038218in}{1.976522in}}%
\pgfpathlineto{\pgfqpoint{1.026172in}{1.976522in}}%
\pgfpathlineto{\pgfqpoint{1.014126in}{1.976522in}}%
\pgfpathlineto{\pgfqpoint{1.002079in}{1.976522in}}%
\pgfpathlineto{\pgfqpoint{0.990033in}{1.976522in}}%
\pgfpathlineto{\pgfqpoint{0.977987in}{1.976522in}}%
\pgfpathlineto{\pgfqpoint{0.965941in}{1.976522in}}%
\pgfpathlineto{\pgfqpoint{0.953895in}{1.976522in}}%
\pgfpathlineto{\pgfqpoint{0.941849in}{1.976522in}}%
\pgfpathlineto{\pgfqpoint{0.929803in}{1.976522in}}%
\pgfpathlineto{\pgfqpoint{0.917757in}{1.976522in}}%
\pgfpathlineto{\pgfqpoint{0.905710in}{1.976522in}}%
\pgfpathlineto{\pgfqpoint{0.893664in}{1.976522in}}%
\pgfpathlineto{\pgfqpoint{0.881618in}{1.976522in}}%
\pgfpathlineto{\pgfqpoint{0.869572in}{1.976522in}}%
\pgfpathlineto{\pgfqpoint{0.857526in}{1.976522in}}%
\pgfpathlineto{\pgfqpoint{0.845480in}{1.976522in}}%
\pgfpathlineto{\pgfqpoint{0.833434in}{1.976522in}}%
\pgfpathlineto{\pgfqpoint{0.821388in}{1.976522in}}%
\pgfpathlineto{\pgfqpoint{0.809342in}{1.976522in}}%
\pgfpathlineto{\pgfqpoint{0.797295in}{1.976522in}}%
\pgfpathlineto{\pgfqpoint{0.785249in}{1.976522in}}%
\pgfpathlineto{\pgfqpoint{0.773203in}{1.976522in}}%
\pgfpathlineto{\pgfqpoint{0.761157in}{1.976522in}}%
\pgfpathlineto{\pgfqpoint{0.749111in}{1.976522in}}%
\pgfpathlineto{\pgfqpoint{0.737065in}{1.976522in}}%
\pgfpathlineto{\pgfqpoint{0.725019in}{1.976522in}}%
\pgfpathlineto{\pgfqpoint{0.712973in}{1.976522in}}%
\pgfpathlineto{\pgfqpoint{0.700927in}{1.976522in}}%
\pgfpathlineto{\pgfqpoint{0.688880in}{1.976522in}}%
\pgfpathlineto{\pgfqpoint{0.676834in}{1.976522in}}%
\pgfpathlineto{\pgfqpoint{0.664788in}{1.976522in}}%
\pgfpathlineto{\pgfqpoint{0.652742in}{1.976522in}}%
\pgfpathlineto{\pgfqpoint{0.640696in}{1.976522in}}%
\pgfpathlineto{\pgfqpoint{0.628650in}{1.976522in}}%
\pgfpathlineto{\pgfqpoint{0.616604in}{1.976522in}}%
\pgfpathlineto{\pgfqpoint{0.604558in}{1.976522in}}%
\pgfpathlineto{\pgfqpoint{0.592511in}{1.976522in}}%
\pgfpathlineto{\pgfqpoint{0.580465in}{1.976522in}}%
\pgfpathlineto{\pgfqpoint{0.568419in}{1.976522in}}%
\pgfpathlineto{\pgfqpoint{0.556373in}{1.976522in}}%
\pgfpathlineto{\pgfqpoint{0.544327in}{1.976522in}}%
\pgfpathlineto{\pgfqpoint{0.532281in}{1.976522in}}%
\pgfpathlineto{\pgfqpoint{0.520235in}{1.976522in}}%
\pgfpathlineto{\pgfqpoint{0.508189in}{1.976522in}}%
\pgfpathlineto{\pgfqpoint{0.496143in}{1.976522in}}%
\pgfpathlineto{\pgfqpoint{0.484096in}{1.976522in}}%
\pgfpathlineto{\pgfqpoint{0.472050in}{1.976522in}}%
\pgfpathlineto{\pgfqpoint{0.460004in}{1.976522in}}%
\pgfpathlineto{\pgfqpoint{0.447958in}{1.976522in}}%
\pgfpathlineto{\pgfqpoint{0.435912in}{1.976522in}}%
\pgfpathlineto{\pgfqpoint{0.423866in}{1.976522in}}%
\pgfpathlineto{\pgfqpoint{0.411820in}{1.976522in}}%
\pgfpathlineto{\pgfqpoint{0.399774in}{1.976522in}}%
\pgfpathlineto{\pgfqpoint{0.387728in}{1.976522in}}%
\pgfpathlineto{\pgfqpoint{0.375681in}{1.976522in}}%
\pgfpathlineto{\pgfqpoint{0.363635in}{1.976522in}}%
\pgfpathlineto{\pgfqpoint{0.351589in}{1.976522in}}%
\pgfpathlineto{\pgfqpoint{0.339543in}{1.976522in}}%
\pgfpathlineto{\pgfqpoint{0.327497in}{1.976522in}}%
\pgfpathlineto{\pgfqpoint{0.315451in}{1.976522in}}%
\pgfpathlineto{\pgfqpoint{0.303405in}{1.976522in}}%
\pgfpathlineto{\pgfqpoint{0.291359in}{1.976522in}}%
\pgfpathlineto{\pgfqpoint{0.279312in}{1.976522in}}%
\pgfpathlineto{\pgfqpoint{0.267266in}{1.976522in}}%
\pgfpathlineto{\pgfqpoint{0.255220in}{1.976522in}}%
\pgfpathlineto{\pgfqpoint{0.243174in}{1.976522in}}%
\pgfpathlineto{\pgfqpoint{0.231128in}{1.976522in}}%
\pgfpathlineto{\pgfqpoint{0.219082in}{1.976522in}}%
\pgfpathlineto{\pgfqpoint{0.207036in}{1.976522in}}%
\pgfpathlineto{\pgfqpoint{0.194990in}{1.976522in}}%
\pgfpathlineto{\pgfqpoint{0.182944in}{1.976522in}}%
\pgfpathlineto{\pgfqpoint{0.170897in}{1.976522in}}%
\pgfpathlineto{\pgfqpoint{0.170897in}{1.964476in}}%
\pgfpathlineto{\pgfqpoint{0.170897in}{1.952430in}}%
\pgfpathlineto{\pgfqpoint{0.170897in}{1.940384in}}%
\pgfpathlineto{\pgfqpoint{0.170897in}{1.928338in}}%
\pgfpathlineto{\pgfqpoint{0.170897in}{1.916292in}}%
\pgfpathlineto{\pgfqpoint{0.170897in}{1.904246in}}%
\pgfpathlineto{\pgfqpoint{0.170897in}{1.892200in}}%
\pgfpathlineto{\pgfqpoint{0.170897in}{1.880154in}}%
\pgfpathlineto{\pgfqpoint{0.170897in}{1.868107in}}%
\pgfpathlineto{\pgfqpoint{0.170897in}{1.856061in}}%
\pgfpathlineto{\pgfqpoint{0.170897in}{1.844015in}}%
\pgfpathlineto{\pgfqpoint{0.170897in}{1.831969in}}%
\pgfpathlineto{\pgfqpoint{0.170897in}{1.819923in}}%
\pgfpathlineto{\pgfqpoint{0.170897in}{1.807877in}}%
\pgfpathlineto{\pgfqpoint{0.170897in}{1.795831in}}%
\pgfpathlineto{\pgfqpoint{0.170897in}{1.783785in}}%
\pgfpathlineto{\pgfqpoint{0.170897in}{1.771738in}}%
\pgfpathlineto{\pgfqpoint{0.170897in}{1.759692in}}%
\pgfpathlineto{\pgfqpoint{0.170897in}{1.747646in}}%
\pgfpathlineto{\pgfqpoint{0.170897in}{1.735600in}}%
\pgfpathlineto{\pgfqpoint{0.170897in}{1.723554in}}%
\pgfpathlineto{\pgfqpoint{0.170897in}{1.711508in}}%
\pgfpathlineto{\pgfqpoint{0.170897in}{1.699462in}}%
\pgfpathlineto{\pgfqpoint{0.170897in}{1.687416in}}%
\pgfpathlineto{\pgfqpoint{0.170897in}{1.675370in}}%
\pgfpathlineto{\pgfqpoint{0.170897in}{1.663323in}}%
\pgfpathlineto{\pgfqpoint{0.170897in}{1.651277in}}%
\pgfpathlineto{\pgfqpoint{0.170897in}{1.639231in}}%
\pgfpathlineto{\pgfqpoint{0.170897in}{1.627185in}}%
\pgfpathlineto{\pgfqpoint{0.170897in}{1.615139in}}%
\pgfpathlineto{\pgfqpoint{0.170897in}{1.603093in}}%
\pgfpathlineto{\pgfqpoint{0.170897in}{1.591047in}}%
\pgfpathlineto{\pgfqpoint{0.170897in}{1.579001in}}%
\pgfpathlineto{\pgfqpoint{0.170897in}{1.566955in}}%
\pgfpathlineto{\pgfqpoint{0.170897in}{1.554908in}}%
\pgfpathlineto{\pgfqpoint{0.170897in}{1.542862in}}%
\pgfpathlineto{\pgfqpoint{0.170897in}{1.530816in}}%
\pgfpathlineto{\pgfqpoint{0.170897in}{1.518770in}}%
\pgfpathlineto{\pgfqpoint{0.170897in}{1.506724in}}%
\pgfpathlineto{\pgfqpoint{0.170897in}{1.494678in}}%
\pgfpathlineto{\pgfqpoint{0.170897in}{1.482632in}}%
\pgfpathlineto{\pgfqpoint{0.170897in}{1.470586in}}%
\pgfpathlineto{\pgfqpoint{0.170897in}{1.458539in}}%
\pgfpathlineto{\pgfqpoint{0.170897in}{1.446493in}}%
\pgfpathlineto{\pgfqpoint{0.170897in}{1.434447in}}%
\pgfpathlineto{\pgfqpoint{0.170897in}{1.422401in}}%
\pgfpathlineto{\pgfqpoint{0.170897in}{1.410355in}}%
\pgfpathlineto{\pgfqpoint{0.170897in}{1.398309in}}%
\pgfpathlineto{\pgfqpoint{0.170897in}{1.386263in}}%
\pgfpathlineto{\pgfqpoint{0.170897in}{1.374217in}}%
\pgfpathlineto{\pgfqpoint{0.170897in}{1.362171in}}%
\pgfpathlineto{\pgfqpoint{0.170897in}{1.350124in}}%
\pgfpathlineto{\pgfqpoint{0.170897in}{1.338078in}}%
\pgfpathlineto{\pgfqpoint{0.170897in}{1.326032in}}%
\pgfpathlineto{\pgfqpoint{0.170897in}{1.313986in}}%
\pgfpathlineto{\pgfqpoint{0.170897in}{1.301940in}}%
\pgfpathlineto{\pgfqpoint{0.170897in}{1.289894in}}%
\pgfpathlineto{\pgfqpoint{0.170897in}{1.277848in}}%
\pgfpathlineto{\pgfqpoint{0.170897in}{1.265802in}}%
\pgfpathlineto{\pgfqpoint{0.170897in}{1.253756in}}%
\pgfpathlineto{\pgfqpoint{0.170897in}{1.241709in}}%
\pgfpathlineto{\pgfqpoint{0.170897in}{1.229663in}}%
\pgfpathlineto{\pgfqpoint{0.170897in}{1.217617in}}%
\pgfpathlineto{\pgfqpoint{0.170897in}{1.205571in}}%
\pgfpathlineto{\pgfqpoint{0.170897in}{1.193525in}}%
\pgfpathlineto{\pgfqpoint{0.170897in}{1.181479in}}%
\pgfpathlineto{\pgfqpoint{0.170897in}{1.169433in}}%
\pgfpathlineto{\pgfqpoint{0.170897in}{1.157387in}}%
\pgfpathlineto{\pgfqpoint{0.170897in}{1.145340in}}%
\pgfpathlineto{\pgfqpoint{0.170897in}{1.133294in}}%
\pgfpathlineto{\pgfqpoint{0.170897in}{1.121248in}}%
\pgfpathlineto{\pgfqpoint{0.170897in}{1.109202in}}%
\pgfpathlineto{\pgfqpoint{0.170897in}{1.097156in}}%
\pgfpathlineto{\pgfqpoint{0.170897in}{1.085110in}}%
\pgfpathlineto{\pgfqpoint{0.170897in}{1.073064in}}%
\pgfpathlineto{\pgfqpoint{0.170897in}{1.061018in}}%
\pgfpathlineto{\pgfqpoint{0.170897in}{1.048972in}}%
\pgfpathlineto{\pgfqpoint{0.170897in}{1.036925in}}%
\pgfpathlineto{\pgfqpoint{0.170897in}{1.024879in}}%
\pgfpathlineto{\pgfqpoint{0.170897in}{1.012833in}}%
\pgfpathlineto{\pgfqpoint{0.170897in}{1.000787in}}%
\pgfpathlineto{\pgfqpoint{0.170897in}{0.988741in}}%
\pgfpathlineto{\pgfqpoint{0.170897in}{0.976695in}}%
\pgfpathlineto{\pgfqpoint{0.170897in}{0.964649in}}%
\pgfpathlineto{\pgfqpoint{0.170897in}{0.952603in}}%
\pgfpathlineto{\pgfqpoint{0.170897in}{0.940557in}}%
\pgfpathlineto{\pgfqpoint{0.170897in}{0.928510in}}%
\pgfpathlineto{\pgfqpoint{0.170897in}{0.916464in}}%
\pgfpathlineto{\pgfqpoint{0.170897in}{0.904418in}}%
\pgfpathlineto{\pgfqpoint{0.170897in}{0.892372in}}%
\pgfpathlineto{\pgfqpoint{0.170897in}{0.880326in}}%
\pgfpathlineto{\pgfqpoint{0.170897in}{0.868280in}}%
\pgfpathlineto{\pgfqpoint{0.170897in}{0.856234in}}%
\pgfpathlineto{\pgfqpoint{0.170897in}{0.844188in}}%
\pgfpathlineto{\pgfqpoint{0.170897in}{0.832141in}}%
\pgfpathlineto{\pgfqpoint{0.170897in}{0.820095in}}%
\pgfpathlineto{\pgfqpoint{0.170897in}{0.808049in}}%
\pgfpathlineto{\pgfqpoint{0.170897in}{0.796003in}}%
\pgfpathlineto{\pgfqpoint{0.170897in}{0.783957in}}%
\pgfpathlineto{\pgfqpoint{0.170897in}{0.771911in}}%
\pgfpathlineto{\pgfqpoint{0.170897in}{0.759865in}}%
\pgfpathlineto{\pgfqpoint{0.170897in}{0.747819in}}%
\pgfpathlineto{\pgfqpoint{0.170897in}{0.735773in}}%
\pgfpathlineto{\pgfqpoint{0.170897in}{0.723726in}}%
\pgfpathlineto{\pgfqpoint{0.170897in}{0.711680in}}%
\pgfpathlineto{\pgfqpoint{0.170897in}{0.699634in}}%
\pgfpathlineto{\pgfqpoint{0.170897in}{0.687588in}}%
\pgfpathlineto{\pgfqpoint{0.170897in}{0.675542in}}%
\pgfpathlineto{\pgfqpoint{0.170897in}{0.663496in}}%
\pgfpathlineto{\pgfqpoint{0.170897in}{0.651450in}}%
\pgfpathlineto{\pgfqpoint{0.170897in}{0.639404in}}%
\pgfpathlineto{\pgfqpoint{0.170897in}{0.627358in}}%
\pgfpathlineto{\pgfqpoint{0.170897in}{0.615311in}}%
\pgfpathlineto{\pgfqpoint{0.170897in}{0.603265in}}%
\pgfpathlineto{\pgfqpoint{0.170897in}{0.591219in}}%
\pgfpathlineto{\pgfqpoint{0.170897in}{0.579173in}}%
\pgfpathlineto{\pgfqpoint{0.170897in}{0.567127in}}%
\pgfpathlineto{\pgfqpoint{0.170897in}{0.555081in}}%
\pgfpathlineto{\pgfqpoint{0.170897in}{0.543035in}}%
\pgfpathlineto{\pgfqpoint{0.170897in}{0.530989in}}%
\pgfpathlineto{\pgfqpoint{0.170897in}{0.518942in}}%
\pgfpathlineto{\pgfqpoint{0.170897in}{0.506896in}}%
\pgfpathlineto{\pgfqpoint{0.170897in}{0.494850in}}%
\pgfpathlineto{\pgfqpoint{0.170897in}{0.482804in}}%
\pgfpathlineto{\pgfqpoint{0.170897in}{0.470758in}}%
\pgfpathlineto{\pgfqpoint{0.170897in}{0.458712in}}%
\pgfpathlineto{\pgfqpoint{0.170897in}{0.446666in}}%
\pgfpathlineto{\pgfqpoint{0.170897in}{0.434620in}}%
\pgfpathlineto{\pgfqpoint{0.170897in}{0.422574in}}%
\pgfpathlineto{\pgfqpoint{0.170897in}{0.410527in}}%
\pgfpathlineto{\pgfqpoint{0.170897in}{0.398481in}}%
\pgfpathlineto{\pgfqpoint{0.170897in}{0.386435in}}%
\pgfpathlineto{\pgfqpoint{0.170897in}{0.374389in}}%
\pgfpathlineto{\pgfqpoint{0.170897in}{0.362343in}}%
\pgfpathlineto{\pgfqpoint{0.170897in}{0.350297in}}%
\pgfpathlineto{\pgfqpoint{0.170897in}{0.338251in}}%
\pgfpathlineto{\pgfqpoint{0.170897in}{0.326205in}}%
\pgfpathlineto{\pgfqpoint{0.170897in}{0.314159in}}%
\pgfpathlineto{\pgfqpoint{0.170897in}{0.302112in}}%
\pgfpathlineto{\pgfqpoint{0.170897in}{0.290066in}}%
\pgfpathlineto{\pgfqpoint{0.170897in}{0.278020in}}%
\pgfpathlineto{\pgfqpoint{0.170897in}{0.265974in}}%
\pgfpathlineto{\pgfqpoint{0.170897in}{0.253928in}}%
\pgfpathlineto{\pgfqpoint{0.170897in}{0.241882in}}%
\pgfpathlineto{\pgfqpoint{0.170897in}{0.229836in}}%
\pgfpathlineto{\pgfqpoint{0.170897in}{0.217790in}}%
\pgfpathlineto{\pgfqpoint{0.170897in}{0.205743in}}%
\pgfpathlineto{\pgfqpoint{0.170897in}{0.193697in}}%
\pgfpathlineto{\pgfqpoint{0.170897in}{0.181651in}}%
\pgfpathclose%
\pgfpathmoveto{\pgfqpoint{0.921897in}{0.193697in}}%
\pgfpathlineto{\pgfqpoint{0.917757in}{0.194378in}}%
\pgfpathlineto{\pgfqpoint{0.905710in}{0.196523in}}%
\pgfpathlineto{\pgfqpoint{0.893664in}{0.198833in}}%
\pgfpathlineto{\pgfqpoint{0.881618in}{0.201309in}}%
\pgfpathlineto{\pgfqpoint{0.869572in}{0.203949in}}%
\pgfpathlineto{\pgfqpoint{0.861866in}{0.205743in}}%
\pgfpathlineto{\pgfqpoint{0.857526in}{0.206768in}}%
\pgfpathlineto{\pgfqpoint{0.845480in}{0.209780in}}%
\pgfpathlineto{\pgfqpoint{0.833434in}{0.212959in}}%
\pgfpathlineto{\pgfqpoint{0.821388in}{0.216305in}}%
\pgfpathlineto{\pgfqpoint{0.816297in}{0.217790in}}%
\pgfpathlineto{\pgfqpoint{0.809342in}{0.219847in}}%
\pgfpathlineto{\pgfqpoint{0.797295in}{0.223579in}}%
\pgfpathlineto{\pgfqpoint{0.785249in}{0.227482in}}%
\pgfpathlineto{\pgfqpoint{0.778285in}{0.229836in}}%
\pgfpathlineto{\pgfqpoint{0.773203in}{0.231578in}}%
\pgfpathlineto{\pgfqpoint{0.761157in}{0.235880in}}%
\pgfpathlineto{\pgfqpoint{0.749111in}{0.240355in}}%
\pgfpathlineto{\pgfqpoint{0.745151in}{0.241882in}}%
\pgfpathlineto{\pgfqpoint{0.737065in}{0.245046in}}%
\pgfpathlineto{\pgfqpoint{0.725019in}{0.249934in}}%
\pgfpathlineto{\pgfqpoint{0.715517in}{0.253928in}}%
\pgfpathlineto{\pgfqpoint{0.712973in}{0.255013in}}%
\pgfpathlineto{\pgfqpoint{0.700927in}{0.260327in}}%
\pgfpathlineto{\pgfqpoint{0.688880in}{0.265819in}}%
\pgfpathlineto{\pgfqpoint{0.688551in}{0.265974in}}%
\pgfpathlineto{\pgfqpoint{0.676834in}{0.271570in}}%
\pgfpathlineto{\pgfqpoint{0.664788in}{0.277503in}}%
\pgfpathlineto{\pgfqpoint{0.663770in}{0.278020in}}%
\pgfpathlineto{\pgfqpoint{0.652742in}{0.283701in}}%
\pgfpathlineto{\pgfqpoint{0.640739in}{0.290066in}}%
\pgfpathlineto{\pgfqpoint{0.640696in}{0.290089in}}%
\pgfpathlineto{\pgfqpoint{0.628650in}{0.296761in}}%
\pgfpathlineto{\pgfqpoint{0.619249in}{0.302112in}}%
\pgfpathlineto{\pgfqpoint{0.616604in}{0.303642in}}%
\pgfpathlineto{\pgfqpoint{0.604558in}{0.310794in}}%
\pgfpathlineto{\pgfqpoint{0.599036in}{0.314159in}}%
\pgfpathlineto{\pgfqpoint{0.592511in}{0.318198in}}%
\pgfpathlineto{\pgfqpoint{0.580465in}{0.325846in}}%
\pgfpathlineto{\pgfqpoint{0.579914in}{0.326205in}}%
\pgfpathlineto{\pgfqpoint{0.568419in}{0.333806in}}%
\pgfpathlineto{\pgfqpoint{0.561858in}{0.338251in}}%
\pgfpathlineto{\pgfqpoint{0.556373in}{0.342028in}}%
\pgfpathlineto{\pgfqpoint{0.544642in}{0.350297in}}%
\pgfpathlineto{\pgfqpoint{0.544327in}{0.350523in}}%
\pgfpathlineto{\pgfqpoint{0.532281in}{0.359357in}}%
\pgfpathlineto{\pgfqpoint{0.528299in}{0.362343in}}%
\pgfpathlineto{\pgfqpoint{0.520235in}{0.368494in}}%
\pgfpathlineto{\pgfqpoint{0.512673in}{0.374389in}}%
\pgfpathlineto{\pgfqpoint{0.508189in}{0.377946in}}%
\pgfpathlineto{\pgfqpoint{0.497712in}{0.386435in}}%
\pgfpathlineto{\pgfqpoint{0.496143in}{0.387730in}}%
\pgfpathlineto{\pgfqpoint{0.484096in}{0.397874in}}%
\pgfpathlineto{\pgfqpoint{0.483390in}{0.398481in}}%
\pgfpathlineto{\pgfqpoint{0.472050in}{0.408403in}}%
\pgfpathlineto{\pgfqpoint{0.469671in}{0.410527in}}%
\pgfpathlineto{\pgfqpoint{0.460004in}{0.419316in}}%
\pgfpathlineto{\pgfqpoint{0.456491in}{0.422574in}}%
\pgfpathlineto{\pgfqpoint{0.447958in}{0.430632in}}%
\pgfpathlineto{\pgfqpoint{0.443817in}{0.434620in}}%
\pgfpathlineto{\pgfqpoint{0.435912in}{0.442376in}}%
\pgfpathlineto{\pgfqpoint{0.431622in}{0.446666in}}%
\pgfpathlineto{\pgfqpoint{0.423866in}{0.454571in}}%
\pgfpathlineto{\pgfqpoint{0.419878in}{0.458712in}}%
\pgfpathlineto{\pgfqpoint{0.411820in}{0.467245in}}%
\pgfpathlineto{\pgfqpoint{0.408562in}{0.470758in}}%
\pgfpathlineto{\pgfqpoint{0.399774in}{0.480425in}}%
\pgfpathlineto{\pgfqpoint{0.397649in}{0.482804in}}%
\pgfpathlineto{\pgfqpoint{0.387728in}{0.494143in}}%
\pgfpathlineto{\pgfqpoint{0.387120in}{0.494850in}}%
\pgfpathlineto{\pgfqpoint{0.376976in}{0.506896in}}%
\pgfpathlineto{\pgfqpoint{0.375681in}{0.508466in}}%
\pgfpathlineto{\pgfqpoint{0.367192in}{0.518942in}}%
\pgfpathlineto{\pgfqpoint{0.363635in}{0.523427in}}%
\pgfpathlineto{\pgfqpoint{0.357740in}{0.530989in}}%
\pgfpathlineto{\pgfqpoint{0.351589in}{0.539053in}}%
\pgfpathlineto{\pgfqpoint{0.348603in}{0.543035in}}%
\pgfpathlineto{\pgfqpoint{0.339769in}{0.555081in}}%
\pgfpathlineto{\pgfqpoint{0.339543in}{0.555396in}}%
\pgfpathlineto{\pgfqpoint{0.331274in}{0.567127in}}%
\pgfpathlineto{\pgfqpoint{0.327497in}{0.572612in}}%
\pgfpathlineto{\pgfqpoint{0.323052in}{0.579173in}}%
\pgfpathlineto{\pgfqpoint{0.315451in}{0.590668in}}%
\pgfpathlineto{\pgfqpoint{0.315092in}{0.591219in}}%
\pgfpathlineto{\pgfqpoint{0.307444in}{0.603265in}}%
\pgfpathlineto{\pgfqpoint{0.303405in}{0.609790in}}%
\pgfpathlineto{\pgfqpoint{0.300040in}{0.615311in}}%
\pgfpathlineto{\pgfqpoint{0.292888in}{0.627358in}}%
\pgfpathlineto{\pgfqpoint{0.291359in}{0.630003in}}%
\pgfpathlineto{\pgfqpoint{0.286007in}{0.639404in}}%
\pgfpathlineto{\pgfqpoint{0.279336in}{0.651450in}}%
\pgfpathlineto{\pgfqpoint{0.279312in}{0.651493in}}%
\pgfpathlineto{\pgfqpoint{0.272947in}{0.663496in}}%
\pgfpathlineto{\pgfqpoint{0.267266in}{0.674523in}}%
\pgfpathlineto{\pgfqpoint{0.266749in}{0.675542in}}%
\pgfpathlineto{\pgfqpoint{0.260816in}{0.687588in}}%
\pgfpathlineto{\pgfqpoint{0.255220in}{0.699305in}}%
\pgfpathlineto{\pgfqpoint{0.255065in}{0.699634in}}%
\pgfpathlineto{\pgfqpoint{0.249574in}{0.711680in}}%
\pgfpathlineto{\pgfqpoint{0.244259in}{0.723726in}}%
\pgfpathlineto{\pgfqpoint{0.243174in}{0.726271in}}%
\pgfpathlineto{\pgfqpoint{0.239181in}{0.735773in}}%
\pgfpathlineto{\pgfqpoint{0.234292in}{0.747819in}}%
\pgfpathlineto{\pgfqpoint{0.231128in}{0.755905in}}%
\pgfpathlineto{\pgfqpoint{0.229601in}{0.759865in}}%
\pgfpathlineto{\pgfqpoint{0.225126in}{0.771911in}}%
\pgfpathlineto{\pgfqpoint{0.220824in}{0.783957in}}%
\pgfpathlineto{\pgfqpoint{0.219082in}{0.789039in}}%
\pgfpathlineto{\pgfqpoint{0.216728in}{0.796003in}}%
\pgfpathlineto{\pgfqpoint{0.212826in}{0.808049in}}%
\pgfpathlineto{\pgfqpoint{0.209093in}{0.820095in}}%
\pgfpathlineto{\pgfqpoint{0.207036in}{0.827051in}}%
\pgfpathlineto{\pgfqpoint{0.205551in}{0.832141in}}%
\pgfpathlineto{\pgfqpoint{0.202205in}{0.844188in}}%
\pgfpathlineto{\pgfqpoint{0.199026in}{0.856234in}}%
\pgfpathlineto{\pgfqpoint{0.196014in}{0.868280in}}%
\pgfpathlineto{\pgfqpoint{0.194990in}{0.872620in}}%
\pgfpathlineto{\pgfqpoint{0.193195in}{0.880326in}}%
\pgfpathlineto{\pgfqpoint{0.190555in}{0.892372in}}%
\pgfpathlineto{\pgfqpoint{0.188080in}{0.904418in}}%
\pgfpathlineto{\pgfqpoint{0.185769in}{0.916464in}}%
\pgfpathlineto{\pgfqpoint{0.183624in}{0.928510in}}%
\pgfpathlineto{\pgfqpoint{0.182944in}{0.932651in}}%
\pgfpathlineto{\pgfqpoint{0.181662in}{0.940557in}}%
\pgfpathlineto{\pgfqpoint{0.179871in}{0.952603in}}%
\pgfpathlineto{\pgfqpoint{0.178243in}{0.964649in}}%
\pgfpathlineto{\pgfqpoint{0.176778in}{0.976695in}}%
\pgfpathlineto{\pgfqpoint{0.175476in}{0.988741in}}%
\pgfpathlineto{\pgfqpoint{0.174336in}{1.000787in}}%
\pgfpathlineto{\pgfqpoint{0.173360in}{1.012833in}}%
\pgfpathlineto{\pgfqpoint{0.172546in}{1.024879in}}%
\pgfpathlineto{\pgfqpoint{0.171894in}{1.036925in}}%
\pgfpathlineto{\pgfqpoint{0.171406in}{1.048972in}}%
\pgfpathlineto{\pgfqpoint{0.171081in}{1.061018in}}%
\pgfpathlineto{\pgfqpoint{0.170918in}{1.073064in}}%
\pgfpathlineto{\pgfqpoint{0.170918in}{1.085110in}}%
\pgfpathlineto{\pgfqpoint{0.171081in}{1.097156in}}%
\pgfpathlineto{\pgfqpoint{0.171406in}{1.109202in}}%
\pgfpathlineto{\pgfqpoint{0.171894in}{1.121248in}}%
\pgfpathlineto{\pgfqpoint{0.172546in}{1.133294in}}%
\pgfpathlineto{\pgfqpoint{0.173360in}{1.145340in}}%
\pgfpathlineto{\pgfqpoint{0.174336in}{1.157387in}}%
\pgfpathlineto{\pgfqpoint{0.175476in}{1.169433in}}%
\pgfpathlineto{\pgfqpoint{0.176778in}{1.181479in}}%
\pgfpathlineto{\pgfqpoint{0.178243in}{1.193525in}}%
\pgfpathlineto{\pgfqpoint{0.179871in}{1.205571in}}%
\pgfpathlineto{\pgfqpoint{0.181662in}{1.217617in}}%
\pgfpathlineto{\pgfqpoint{0.182944in}{1.225522in}}%
\pgfpathlineto{\pgfqpoint{0.183624in}{1.229663in}}%
\pgfpathlineto{\pgfqpoint{0.185769in}{1.241709in}}%
\pgfpathlineto{\pgfqpoint{0.188080in}{1.253756in}}%
\pgfpathlineto{\pgfqpoint{0.190555in}{1.265802in}}%
\pgfpathlineto{\pgfqpoint{0.193195in}{1.277848in}}%
\pgfpathlineto{\pgfqpoint{0.194990in}{1.285554in}}%
\pgfpathlineto{\pgfqpoint{0.196014in}{1.289894in}}%
\pgfpathlineto{\pgfqpoint{0.199026in}{1.301940in}}%
\pgfpathlineto{\pgfqpoint{0.202205in}{1.313986in}}%
\pgfpathlineto{\pgfqpoint{0.205551in}{1.326032in}}%
\pgfpathlineto{\pgfqpoint{0.207036in}{1.331123in}}%
\pgfpathlineto{\pgfqpoint{0.209093in}{1.338078in}}%
\pgfpathlineto{\pgfqpoint{0.212826in}{1.350124in}}%
\pgfpathlineto{\pgfqpoint{0.216728in}{1.362171in}}%
\pgfpathlineto{\pgfqpoint{0.219082in}{1.369135in}}%
\pgfpathlineto{\pgfqpoint{0.220824in}{1.374217in}}%
\pgfpathlineto{\pgfqpoint{0.225126in}{1.386263in}}%
\pgfpathlineto{\pgfqpoint{0.229601in}{1.398309in}}%
\pgfpathlineto{\pgfqpoint{0.231128in}{1.402269in}}%
\pgfpathlineto{\pgfqpoint{0.234292in}{1.410355in}}%
\pgfpathlineto{\pgfqpoint{0.239181in}{1.422401in}}%
\pgfpathlineto{\pgfqpoint{0.243174in}{1.431903in}}%
\pgfpathlineto{\pgfqpoint{0.244259in}{1.434447in}}%
\pgfpathlineto{\pgfqpoint{0.249574in}{1.446493in}}%
\pgfpathlineto{\pgfqpoint{0.255065in}{1.458539in}}%
\pgfpathlineto{\pgfqpoint{0.255220in}{1.458869in}}%
\pgfpathlineto{\pgfqpoint{0.260816in}{1.470586in}}%
\pgfpathlineto{\pgfqpoint{0.266749in}{1.482632in}}%
\pgfpathlineto{\pgfqpoint{0.267266in}{1.483650in}}%
\pgfpathlineto{\pgfqpoint{0.272947in}{1.494678in}}%
\pgfpathlineto{\pgfqpoint{0.279312in}{1.506681in}}%
\pgfpathlineto{\pgfqpoint{0.279336in}{1.506724in}}%
\pgfpathlineto{\pgfqpoint{0.286007in}{1.518770in}}%
\pgfpathlineto{\pgfqpoint{0.291359in}{1.528171in}}%
\pgfpathlineto{\pgfqpoint{0.292888in}{1.530816in}}%
\pgfpathlineto{\pgfqpoint{0.300040in}{1.542862in}}%
\pgfpathlineto{\pgfqpoint{0.303405in}{1.548383in}}%
\pgfpathlineto{\pgfqpoint{0.307444in}{1.554908in}}%
\pgfpathlineto{\pgfqpoint{0.315092in}{1.566955in}}%
\pgfpathlineto{\pgfqpoint{0.315451in}{1.567505in}}%
\pgfpathlineto{\pgfqpoint{0.323052in}{1.579001in}}%
\pgfpathlineto{\pgfqpoint{0.327497in}{1.585561in}}%
\pgfpathlineto{\pgfqpoint{0.331274in}{1.591047in}}%
\pgfpathlineto{\pgfqpoint{0.339543in}{1.602778in}}%
\pgfpathlineto{\pgfqpoint{0.339769in}{1.603093in}}%
\pgfpathlineto{\pgfqpoint{0.348603in}{1.615139in}}%
\pgfpathlineto{\pgfqpoint{0.351589in}{1.619121in}}%
\pgfpathlineto{\pgfqpoint{0.357740in}{1.627185in}}%
\pgfpathlineto{\pgfqpoint{0.363635in}{1.634747in}}%
\pgfpathlineto{\pgfqpoint{0.367192in}{1.639231in}}%
\pgfpathlineto{\pgfqpoint{0.375681in}{1.649708in}}%
\pgfpathlineto{\pgfqpoint{0.376976in}{1.651277in}}%
\pgfpathlineto{\pgfqpoint{0.387120in}{1.663323in}}%
\pgfpathlineto{\pgfqpoint{0.387728in}{1.664030in}}%
\pgfpathlineto{\pgfqpoint{0.397649in}{1.675370in}}%
\pgfpathlineto{\pgfqpoint{0.399774in}{1.677749in}}%
\pgfpathlineto{\pgfqpoint{0.408562in}{1.687416in}}%
\pgfpathlineto{\pgfqpoint{0.411820in}{1.690929in}}%
\pgfpathlineto{\pgfqpoint{0.419878in}{1.699462in}}%
\pgfpathlineto{\pgfqpoint{0.423866in}{1.703603in}}%
\pgfpathlineto{\pgfqpoint{0.431622in}{1.711508in}}%
\pgfpathlineto{\pgfqpoint{0.435912in}{1.715798in}}%
\pgfpathlineto{\pgfqpoint{0.443817in}{1.723554in}}%
\pgfpathlineto{\pgfqpoint{0.447958in}{1.727542in}}%
\pgfpathlineto{\pgfqpoint{0.456491in}{1.735600in}}%
\pgfpathlineto{\pgfqpoint{0.460004in}{1.738858in}}%
\pgfpathlineto{\pgfqpoint{0.469671in}{1.747646in}}%
\pgfpathlineto{\pgfqpoint{0.472050in}{1.749770in}}%
\pgfpathlineto{\pgfqpoint{0.483390in}{1.759692in}}%
\pgfpathlineto{\pgfqpoint{0.484096in}{1.760300in}}%
\pgfpathlineto{\pgfqpoint{0.496143in}{1.770444in}}%
\pgfpathlineto{\pgfqpoint{0.497712in}{1.771738in}}%
\pgfpathlineto{\pgfqpoint{0.508189in}{1.780228in}}%
\pgfpathlineto{\pgfqpoint{0.512673in}{1.783785in}}%
\pgfpathlineto{\pgfqpoint{0.520235in}{1.789680in}}%
\pgfpathlineto{\pgfqpoint{0.528299in}{1.795831in}}%
\pgfpathlineto{\pgfqpoint{0.532281in}{1.798817in}}%
\pgfpathlineto{\pgfqpoint{0.544327in}{1.807651in}}%
\pgfpathlineto{\pgfqpoint{0.544642in}{1.807877in}}%
\pgfpathlineto{\pgfqpoint{0.556373in}{1.816146in}}%
\pgfpathlineto{\pgfqpoint{0.561858in}{1.819923in}}%
\pgfpathlineto{\pgfqpoint{0.568419in}{1.824367in}}%
\pgfpathlineto{\pgfqpoint{0.579914in}{1.831969in}}%
\pgfpathlineto{\pgfqpoint{0.580465in}{1.832328in}}%
\pgfpathlineto{\pgfqpoint{0.592511in}{1.839976in}}%
\pgfpathlineto{\pgfqpoint{0.599036in}{1.844015in}}%
\pgfpathlineto{\pgfqpoint{0.604558in}{1.847380in}}%
\pgfpathlineto{\pgfqpoint{0.616604in}{1.854532in}}%
\pgfpathlineto{\pgfqpoint{0.619249in}{1.856061in}}%
\pgfpathlineto{\pgfqpoint{0.628650in}{1.861413in}}%
\pgfpathlineto{\pgfqpoint{0.640696in}{1.868084in}}%
\pgfpathlineto{\pgfqpoint{0.640739in}{1.868107in}}%
\pgfpathlineto{\pgfqpoint{0.652742in}{1.874473in}}%
\pgfpathlineto{\pgfqpoint{0.663770in}{1.880154in}}%
\pgfpathlineto{\pgfqpoint{0.664788in}{1.880670in}}%
\pgfpathlineto{\pgfqpoint{0.676834in}{1.886604in}}%
\pgfpathlineto{\pgfqpoint{0.688551in}{1.892200in}}%
\pgfpathlineto{\pgfqpoint{0.688880in}{1.892355in}}%
\pgfpathlineto{\pgfqpoint{0.700927in}{1.897846in}}%
\pgfpathlineto{\pgfqpoint{0.712973in}{1.903161in}}%
\pgfpathlineto{\pgfqpoint{0.715517in}{1.904246in}}%
\pgfpathlineto{\pgfqpoint{0.725019in}{1.908239in}}%
\pgfpathlineto{\pgfqpoint{0.737065in}{1.913128in}}%
\pgfpathlineto{\pgfqpoint{0.745151in}{1.916292in}}%
\pgfpathlineto{\pgfqpoint{0.749111in}{1.917819in}}%
\pgfpathlineto{\pgfqpoint{0.761157in}{1.922293in}}%
\pgfpathlineto{\pgfqpoint{0.773203in}{1.926596in}}%
\pgfpathlineto{\pgfqpoint{0.778285in}{1.928338in}}%
\pgfpathlineto{\pgfqpoint{0.785249in}{1.930692in}}%
\pgfpathlineto{\pgfqpoint{0.797295in}{1.934594in}}%
\pgfpathlineto{\pgfqpoint{0.809342in}{1.938327in}}%
\pgfpathlineto{\pgfqpoint{0.816297in}{1.940384in}}%
\pgfpathlineto{\pgfqpoint{0.821388in}{1.941869in}}%
\pgfpathlineto{\pgfqpoint{0.833434in}{1.945215in}}%
\pgfpathlineto{\pgfqpoint{0.845480in}{1.948394in}}%
\pgfpathlineto{\pgfqpoint{0.857526in}{1.951405in}}%
\pgfpathlineto{\pgfqpoint{0.861866in}{1.952430in}}%
\pgfpathlineto{\pgfqpoint{0.869572in}{1.954225in}}%
\pgfpathlineto{\pgfqpoint{0.881618in}{1.956865in}}%
\pgfpathlineto{\pgfqpoint{0.893664in}{1.959340in}}%
\pgfpathlineto{\pgfqpoint{0.905710in}{1.961650in}}%
\pgfpathlineto{\pgfqpoint{0.917757in}{1.963796in}}%
\pgfpathlineto{\pgfqpoint{0.921897in}{1.964476in}}%
\pgfpathlineto{\pgfqpoint{0.929803in}{1.965758in}}%
\pgfpathlineto{\pgfqpoint{0.941849in}{1.967549in}}%
\pgfpathlineto{\pgfqpoint{0.953895in}{1.969177in}}%
\pgfpathlineto{\pgfqpoint{0.965941in}{1.970642in}}%
\pgfpathlineto{\pgfqpoint{0.977987in}{1.971944in}}%
\pgfpathlineto{\pgfqpoint{0.990033in}{1.973084in}}%
\pgfpathlineto{\pgfqpoint{1.002079in}{1.974060in}}%
\pgfpathlineto{\pgfqpoint{1.014126in}{1.974874in}}%
\pgfpathlineto{\pgfqpoint{1.026172in}{1.975525in}}%
\pgfpathlineto{\pgfqpoint{1.038218in}{1.976014in}}%
\pgfpathlineto{\pgfqpoint{1.050264in}{1.976339in}}%
\pgfpathlineto{\pgfqpoint{1.062310in}{1.976502in}}%
\pgfpathlineto{\pgfqpoint{1.074356in}{1.976502in}}%
\pgfpathlineto{\pgfqpoint{1.086402in}{1.976339in}}%
\pgfpathlineto{\pgfqpoint{1.098448in}{1.976014in}}%
\pgfpathlineto{\pgfqpoint{1.110494in}{1.975525in}}%
\pgfpathlineto{\pgfqpoint{1.122541in}{1.974874in}}%
\pgfpathlineto{\pgfqpoint{1.134587in}{1.974060in}}%
\pgfpathlineto{\pgfqpoint{1.146633in}{1.973084in}}%
\pgfpathlineto{\pgfqpoint{1.158679in}{1.971944in}}%
\pgfpathlineto{\pgfqpoint{1.170725in}{1.970642in}}%
\pgfpathlineto{\pgfqpoint{1.182771in}{1.969177in}}%
\pgfpathlineto{\pgfqpoint{1.194817in}{1.967549in}}%
\pgfpathlineto{\pgfqpoint{1.206863in}{1.965758in}}%
\pgfpathlineto{\pgfqpoint{1.214769in}{1.964476in}}%
\pgfpathlineto{\pgfqpoint{1.218909in}{1.963796in}}%
\pgfpathlineto{\pgfqpoint{1.230956in}{1.961650in}}%
\pgfpathlineto{\pgfqpoint{1.243002in}{1.959340in}}%
\pgfpathlineto{\pgfqpoint{1.255048in}{1.956865in}}%
\pgfpathlineto{\pgfqpoint{1.267094in}{1.954225in}}%
\pgfpathlineto{\pgfqpoint{1.274800in}{1.952430in}}%
\pgfpathlineto{\pgfqpoint{1.279140in}{1.951405in}}%
\pgfpathlineto{\pgfqpoint{1.291186in}{1.948394in}}%
\pgfpathlineto{\pgfqpoint{1.303232in}{1.945215in}}%
\pgfpathlineto{\pgfqpoint{1.315278in}{1.941869in}}%
\pgfpathlineto{\pgfqpoint{1.320369in}{1.940384in}}%
\pgfpathlineto{\pgfqpoint{1.327325in}{1.938327in}}%
\pgfpathlineto{\pgfqpoint{1.339371in}{1.934594in}}%
\pgfpathlineto{\pgfqpoint{1.351417in}{1.930692in}}%
\pgfpathlineto{\pgfqpoint{1.358381in}{1.928338in}}%
\pgfpathlineto{\pgfqpoint{1.363463in}{1.926596in}}%
\pgfpathlineto{\pgfqpoint{1.375509in}{1.922293in}}%
\pgfpathlineto{\pgfqpoint{1.387555in}{1.917819in}}%
\pgfpathlineto{\pgfqpoint{1.391515in}{1.916292in}}%
\pgfpathlineto{\pgfqpoint{1.399601in}{1.913128in}}%
\pgfpathlineto{\pgfqpoint{1.411647in}{1.908239in}}%
\pgfpathlineto{\pgfqpoint{1.421149in}{1.904246in}}%
\pgfpathlineto{\pgfqpoint{1.423693in}{1.903161in}}%
\pgfpathlineto{\pgfqpoint{1.435740in}{1.897846in}}%
\pgfpathlineto{\pgfqpoint{1.447786in}{1.892355in}}%
\pgfpathlineto{\pgfqpoint{1.448115in}{1.892200in}}%
\pgfpathlineto{\pgfqpoint{1.459832in}{1.886604in}}%
\pgfpathlineto{\pgfqpoint{1.471878in}{1.880670in}}%
\pgfpathlineto{\pgfqpoint{1.472897in}{1.880154in}}%
\pgfpathlineto{\pgfqpoint{1.483924in}{1.874473in}}%
\pgfpathlineto{\pgfqpoint{1.495927in}{1.868107in}}%
\pgfpathlineto{\pgfqpoint{1.495970in}{1.868084in}}%
\pgfpathlineto{\pgfqpoint{1.508016in}{1.861413in}}%
\pgfpathlineto{\pgfqpoint{1.517417in}{1.856061in}}%
\pgfpathlineto{\pgfqpoint{1.520062in}{1.854532in}}%
\pgfpathlineto{\pgfqpoint{1.532108in}{1.847380in}}%
\pgfpathlineto{\pgfqpoint{1.537630in}{1.844015in}}%
\pgfpathlineto{\pgfqpoint{1.544155in}{1.839976in}}%
\pgfpathlineto{\pgfqpoint{1.556201in}{1.832328in}}%
\pgfpathlineto{\pgfqpoint{1.556752in}{1.831969in}}%
\pgfpathlineto{\pgfqpoint{1.568247in}{1.824367in}}%
\pgfpathlineto{\pgfqpoint{1.574808in}{1.819923in}}%
\pgfpathlineto{\pgfqpoint{1.580293in}{1.816146in}}%
\pgfpathlineto{\pgfqpoint{1.592024in}{1.807877in}}%
\pgfpathlineto{\pgfqpoint{1.592339in}{1.807651in}}%
\pgfpathlineto{\pgfqpoint{1.604385in}{1.798817in}}%
\pgfpathlineto{\pgfqpoint{1.608367in}{1.795831in}}%
\pgfpathlineto{\pgfqpoint{1.616431in}{1.789680in}}%
\pgfpathlineto{\pgfqpoint{1.623993in}{1.783785in}}%
\pgfpathlineto{\pgfqpoint{1.628477in}{1.780228in}}%
\pgfpathlineto{\pgfqpoint{1.638954in}{1.771738in}}%
\pgfpathlineto{\pgfqpoint{1.640524in}{1.770444in}}%
\pgfpathlineto{\pgfqpoint{1.652570in}{1.760300in}}%
\pgfpathlineto{\pgfqpoint{1.653276in}{1.759692in}}%
\pgfpathlineto{\pgfqpoint{1.664616in}{1.749770in}}%
\pgfpathlineto{\pgfqpoint{1.666995in}{1.747646in}}%
\pgfpathlineto{\pgfqpoint{1.676662in}{1.738858in}}%
\pgfpathlineto{\pgfqpoint{1.680175in}{1.735600in}}%
\pgfpathlineto{\pgfqpoint{1.688708in}{1.727542in}}%
\pgfpathlineto{\pgfqpoint{1.692849in}{1.723554in}}%
\pgfpathlineto{\pgfqpoint{1.700754in}{1.715798in}}%
\pgfpathlineto{\pgfqpoint{1.705044in}{1.711508in}}%
\pgfpathlineto{\pgfqpoint{1.712800in}{1.703603in}}%
\pgfpathlineto{\pgfqpoint{1.716788in}{1.699462in}}%
\pgfpathlineto{\pgfqpoint{1.724846in}{1.690929in}}%
\pgfpathlineto{\pgfqpoint{1.728104in}{1.687416in}}%
\pgfpathlineto{\pgfqpoint{1.736892in}{1.677749in}}%
\pgfpathlineto{\pgfqpoint{1.739017in}{1.675370in}}%
\pgfpathlineto{\pgfqpoint{1.748939in}{1.664030in}}%
\pgfpathlineto{\pgfqpoint{1.749546in}{1.663323in}}%
\pgfpathlineto{\pgfqpoint{1.759690in}{1.651277in}}%
\pgfpathlineto{\pgfqpoint{1.760985in}{1.649708in}}%
\pgfpathlineto{\pgfqpoint{1.769474in}{1.639231in}}%
\pgfpathlineto{\pgfqpoint{1.773031in}{1.634747in}}%
\pgfpathlineto{\pgfqpoint{1.778926in}{1.627185in}}%
\pgfpathlineto{\pgfqpoint{1.785077in}{1.619121in}}%
\pgfpathlineto{\pgfqpoint{1.788063in}{1.615139in}}%
\pgfpathlineto{\pgfqpoint{1.796897in}{1.603093in}}%
\pgfpathlineto{\pgfqpoint{1.797123in}{1.602778in}}%
\pgfpathlineto{\pgfqpoint{1.805392in}{1.591047in}}%
\pgfpathlineto{\pgfqpoint{1.809169in}{1.585561in}}%
\pgfpathlineto{\pgfqpoint{1.813614in}{1.579001in}}%
\pgfpathlineto{\pgfqpoint{1.821215in}{1.567505in}}%
\pgfpathlineto{\pgfqpoint{1.821574in}{1.566955in}}%
\pgfpathlineto{\pgfqpoint{1.829222in}{1.554908in}}%
\pgfpathlineto{\pgfqpoint{1.833261in}{1.548383in}}%
\pgfpathlineto{\pgfqpoint{1.836626in}{1.542862in}}%
\pgfpathlineto{\pgfqpoint{1.843778in}{1.530816in}}%
\pgfpathlineto{\pgfqpoint{1.845307in}{1.528171in}}%
\pgfpathlineto{\pgfqpoint{1.850659in}{1.518770in}}%
\pgfpathlineto{\pgfqpoint{1.857330in}{1.506724in}}%
\pgfpathlineto{\pgfqpoint{1.857354in}{1.506681in}}%
\pgfpathlineto{\pgfqpoint{1.863719in}{1.494678in}}%
\pgfpathlineto{\pgfqpoint{1.869400in}{1.483650in}}%
\pgfpathlineto{\pgfqpoint{1.869917in}{1.482632in}}%
\pgfpathlineto{\pgfqpoint{1.875850in}{1.470586in}}%
\pgfpathlineto{\pgfqpoint{1.881446in}{1.458869in}}%
\pgfpathlineto{\pgfqpoint{1.881601in}{1.458539in}}%
\pgfpathlineto{\pgfqpoint{1.887092in}{1.446493in}}%
\pgfpathlineto{\pgfqpoint{1.892407in}{1.434447in}}%
\pgfpathlineto{\pgfqpoint{1.893492in}{1.431903in}}%
\pgfpathlineto{\pgfqpoint{1.897485in}{1.422401in}}%
\pgfpathlineto{\pgfqpoint{1.902374in}{1.410355in}}%
\pgfpathlineto{\pgfqpoint{1.905538in}{1.402269in}}%
\pgfpathlineto{\pgfqpoint{1.907065in}{1.398309in}}%
\pgfpathlineto{\pgfqpoint{1.911540in}{1.386263in}}%
\pgfpathlineto{\pgfqpoint{1.915842in}{1.374217in}}%
\pgfpathlineto{\pgfqpoint{1.917584in}{1.369135in}}%
\pgfpathlineto{\pgfqpoint{1.919938in}{1.362171in}}%
\pgfpathlineto{\pgfqpoint{1.923841in}{1.350124in}}%
\pgfpathlineto{\pgfqpoint{1.927573in}{1.338078in}}%
\pgfpathlineto{\pgfqpoint{1.929630in}{1.331123in}}%
\pgfpathlineto{\pgfqpoint{1.931115in}{1.326032in}}%
\pgfpathlineto{\pgfqpoint{1.934461in}{1.313986in}}%
\pgfpathlineto{\pgfqpoint{1.937640in}{1.301940in}}%
\pgfpathlineto{\pgfqpoint{1.940652in}{1.289894in}}%
\pgfpathlineto{\pgfqpoint{1.941676in}{1.285554in}}%
\pgfpathlineto{\pgfqpoint{1.943471in}{1.277848in}}%
\pgfpathlineto{\pgfqpoint{1.946111in}{1.265802in}}%
\pgfpathlineto{\pgfqpoint{1.948586in}{1.253756in}}%
\pgfpathlineto{\pgfqpoint{1.950897in}{1.241709in}}%
\pgfpathlineto{\pgfqpoint{1.953042in}{1.229663in}}%
\pgfpathlineto{\pgfqpoint{1.953723in}{1.225522in}}%
\pgfpathlineto{\pgfqpoint{1.955004in}{1.217617in}}%
\pgfpathlineto{\pgfqpoint{1.956795in}{1.205571in}}%
\pgfpathlineto{\pgfqpoint{1.958423in}{1.193525in}}%
\pgfpathlineto{\pgfqpoint{1.959888in}{1.181479in}}%
\pgfpathlineto{\pgfqpoint{1.961190in}{1.169433in}}%
\pgfpathlineto{\pgfqpoint{1.962330in}{1.157387in}}%
\pgfpathlineto{\pgfqpoint{1.963307in}{1.145340in}}%
\pgfpathlineto{\pgfqpoint{1.964120in}{1.133294in}}%
\pgfpathlineto{\pgfqpoint{1.964772in}{1.121248in}}%
\pgfpathlineto{\pgfqpoint{1.965260in}{1.109202in}}%
\pgfpathlineto{\pgfqpoint{1.965585in}{1.097156in}}%
\pgfpathlineto{\pgfqpoint{1.965748in}{1.085110in}}%
\pgfpathlineto{\pgfqpoint{1.965748in}{1.073064in}}%
\pgfpathlineto{\pgfqpoint{1.965585in}{1.061018in}}%
\pgfpathlineto{\pgfqpoint{1.965260in}{1.048972in}}%
\pgfpathlineto{\pgfqpoint{1.964772in}{1.036925in}}%
\pgfpathlineto{\pgfqpoint{1.964120in}{1.024879in}}%
\pgfpathlineto{\pgfqpoint{1.963307in}{1.012833in}}%
\pgfpathlineto{\pgfqpoint{1.962330in}{1.000787in}}%
\pgfpathlineto{\pgfqpoint{1.961190in}{0.988741in}}%
\pgfpathlineto{\pgfqpoint{1.959888in}{0.976695in}}%
\pgfpathlineto{\pgfqpoint{1.958423in}{0.964649in}}%
\pgfpathlineto{\pgfqpoint{1.956795in}{0.952603in}}%
\pgfpathlineto{\pgfqpoint{1.955004in}{0.940557in}}%
\pgfpathlineto{\pgfqpoint{1.953723in}{0.932651in}}%
\pgfpathlineto{\pgfqpoint{1.953042in}{0.928510in}}%
\pgfpathlineto{\pgfqpoint{1.950897in}{0.916464in}}%
\pgfpathlineto{\pgfqpoint{1.948586in}{0.904418in}}%
\pgfpathlineto{\pgfqpoint{1.946111in}{0.892372in}}%
\pgfpathlineto{\pgfqpoint{1.943471in}{0.880326in}}%
\pgfpathlineto{\pgfqpoint{1.941676in}{0.872620in}}%
\pgfpathlineto{\pgfqpoint{1.940652in}{0.868280in}}%
\pgfpathlineto{\pgfqpoint{1.937640in}{0.856234in}}%
\pgfpathlineto{\pgfqpoint{1.934461in}{0.844188in}}%
\pgfpathlineto{\pgfqpoint{1.931115in}{0.832141in}}%
\pgfpathlineto{\pgfqpoint{1.929630in}{0.827051in}}%
\pgfpathlineto{\pgfqpoint{1.927573in}{0.820095in}}%
\pgfpathlineto{\pgfqpoint{1.923841in}{0.808049in}}%
\pgfpathlineto{\pgfqpoint{1.919938in}{0.796003in}}%
\pgfpathlineto{\pgfqpoint{1.917584in}{0.789039in}}%
\pgfpathlineto{\pgfqpoint{1.915842in}{0.783957in}}%
\pgfpathlineto{\pgfqpoint{1.911540in}{0.771911in}}%
\pgfpathlineto{\pgfqpoint{1.907065in}{0.759865in}}%
\pgfpathlineto{\pgfqpoint{1.905538in}{0.755905in}}%
\pgfpathlineto{\pgfqpoint{1.902374in}{0.747819in}}%
\pgfpathlineto{\pgfqpoint{1.897485in}{0.735773in}}%
\pgfpathlineto{\pgfqpoint{1.893492in}{0.726271in}}%
\pgfpathlineto{\pgfqpoint{1.892407in}{0.723726in}}%
\pgfpathlineto{\pgfqpoint{1.887092in}{0.711680in}}%
\pgfpathlineto{\pgfqpoint{1.881601in}{0.699634in}}%
\pgfpathlineto{\pgfqpoint{1.881446in}{0.699305in}}%
\pgfpathlineto{\pgfqpoint{1.875850in}{0.687588in}}%
\pgfpathlineto{\pgfqpoint{1.869917in}{0.675542in}}%
\pgfpathlineto{\pgfqpoint{1.869400in}{0.674523in}}%
\pgfpathlineto{\pgfqpoint{1.863719in}{0.663496in}}%
\pgfpathlineto{\pgfqpoint{1.857354in}{0.651493in}}%
\pgfpathlineto{\pgfqpoint{1.857330in}{0.651450in}}%
\pgfpathlineto{\pgfqpoint{1.850659in}{0.639404in}}%
\pgfpathlineto{\pgfqpoint{1.845307in}{0.630003in}}%
\pgfpathlineto{\pgfqpoint{1.843778in}{0.627358in}}%
\pgfpathlineto{\pgfqpoint{1.836626in}{0.615311in}}%
\pgfpathlineto{\pgfqpoint{1.833261in}{0.609790in}}%
\pgfpathlineto{\pgfqpoint{1.829222in}{0.603265in}}%
\pgfpathlineto{\pgfqpoint{1.821574in}{0.591219in}}%
\pgfpathlineto{\pgfqpoint{1.821215in}{0.590668in}}%
\pgfpathlineto{\pgfqpoint{1.813614in}{0.579173in}}%
\pgfpathlineto{\pgfqpoint{1.809169in}{0.572612in}}%
\pgfpathlineto{\pgfqpoint{1.805392in}{0.567127in}}%
\pgfpathlineto{\pgfqpoint{1.797123in}{0.555396in}}%
\pgfpathlineto{\pgfqpoint{1.796897in}{0.555081in}}%
\pgfpathlineto{\pgfqpoint{1.788063in}{0.543035in}}%
\pgfpathlineto{\pgfqpoint{1.785077in}{0.539053in}}%
\pgfpathlineto{\pgfqpoint{1.778926in}{0.530989in}}%
\pgfpathlineto{\pgfqpoint{1.773031in}{0.523427in}}%
\pgfpathlineto{\pgfqpoint{1.769474in}{0.518942in}}%
\pgfpathlineto{\pgfqpoint{1.760985in}{0.508466in}}%
\pgfpathlineto{\pgfqpoint{1.759690in}{0.506896in}}%
\pgfpathlineto{\pgfqpoint{1.749546in}{0.494850in}}%
\pgfpathlineto{\pgfqpoint{1.748939in}{0.494143in}}%
\pgfpathlineto{\pgfqpoint{1.739017in}{0.482804in}}%
\pgfpathlineto{\pgfqpoint{1.736892in}{0.480425in}}%
\pgfpathlineto{\pgfqpoint{1.728104in}{0.470758in}}%
\pgfpathlineto{\pgfqpoint{1.724846in}{0.467245in}}%
\pgfpathlineto{\pgfqpoint{1.716788in}{0.458712in}}%
\pgfpathlineto{\pgfqpoint{1.712800in}{0.454571in}}%
\pgfpathlineto{\pgfqpoint{1.705044in}{0.446666in}}%
\pgfpathlineto{\pgfqpoint{1.700754in}{0.442376in}}%
\pgfpathlineto{\pgfqpoint{1.692849in}{0.434620in}}%
\pgfpathlineto{\pgfqpoint{1.688708in}{0.430632in}}%
\pgfpathlineto{\pgfqpoint{1.680175in}{0.422574in}}%
\pgfpathlineto{\pgfqpoint{1.676662in}{0.419316in}}%
\pgfpathlineto{\pgfqpoint{1.666995in}{0.410527in}}%
\pgfpathlineto{\pgfqpoint{1.664616in}{0.408403in}}%
\pgfpathlineto{\pgfqpoint{1.653276in}{0.398481in}}%
\pgfpathlineto{\pgfqpoint{1.652570in}{0.397874in}}%
\pgfpathlineto{\pgfqpoint{1.640524in}{0.387730in}}%
\pgfpathlineto{\pgfqpoint{1.638954in}{0.386435in}}%
\pgfpathlineto{\pgfqpoint{1.628477in}{0.377946in}}%
\pgfpathlineto{\pgfqpoint{1.623993in}{0.374389in}}%
\pgfpathlineto{\pgfqpoint{1.616431in}{0.368494in}}%
\pgfpathlineto{\pgfqpoint{1.608367in}{0.362343in}}%
\pgfpathlineto{\pgfqpoint{1.604385in}{0.359357in}}%
\pgfpathlineto{\pgfqpoint{1.592339in}{0.350523in}}%
\pgfpathlineto{\pgfqpoint{1.592024in}{0.350297in}}%
\pgfpathlineto{\pgfqpoint{1.580293in}{0.342028in}}%
\pgfpathlineto{\pgfqpoint{1.574808in}{0.338251in}}%
\pgfpathlineto{\pgfqpoint{1.568247in}{0.333806in}}%
\pgfpathlineto{\pgfqpoint{1.556752in}{0.326205in}}%
\pgfpathlineto{\pgfqpoint{1.556201in}{0.325846in}}%
\pgfpathlineto{\pgfqpoint{1.544155in}{0.318198in}}%
\pgfpathlineto{\pgfqpoint{1.537630in}{0.314159in}}%
\pgfpathlineto{\pgfqpoint{1.532108in}{0.310794in}}%
\pgfpathlineto{\pgfqpoint{1.520062in}{0.303642in}}%
\pgfpathlineto{\pgfqpoint{1.517417in}{0.302112in}}%
\pgfpathlineto{\pgfqpoint{1.508016in}{0.296761in}}%
\pgfpathlineto{\pgfqpoint{1.495970in}{0.290089in}}%
\pgfpathlineto{\pgfqpoint{1.495927in}{0.290066in}}%
\pgfpathlineto{\pgfqpoint{1.483924in}{0.283701in}}%
\pgfpathlineto{\pgfqpoint{1.472897in}{0.278020in}}%
\pgfpathlineto{\pgfqpoint{1.471878in}{0.277503in}}%
\pgfpathlineto{\pgfqpoint{1.459832in}{0.271570in}}%
\pgfpathlineto{\pgfqpoint{1.448115in}{0.265974in}}%
\pgfpathlineto{\pgfqpoint{1.447786in}{0.265819in}}%
\pgfpathlineto{\pgfqpoint{1.435740in}{0.260327in}}%
\pgfpathlineto{\pgfqpoint{1.423693in}{0.255013in}}%
\pgfpathlineto{\pgfqpoint{1.421149in}{0.253928in}}%
\pgfpathlineto{\pgfqpoint{1.411647in}{0.249934in}}%
\pgfpathlineto{\pgfqpoint{1.399601in}{0.245046in}}%
\pgfpathlineto{\pgfqpoint{1.391515in}{0.241882in}}%
\pgfpathlineto{\pgfqpoint{1.387555in}{0.240355in}}%
\pgfpathlineto{\pgfqpoint{1.375509in}{0.235880in}}%
\pgfpathlineto{\pgfqpoint{1.363463in}{0.231578in}}%
\pgfpathlineto{\pgfqpoint{1.358381in}{0.229836in}}%
\pgfpathlineto{\pgfqpoint{1.351417in}{0.227482in}}%
\pgfpathlineto{\pgfqpoint{1.339371in}{0.223579in}}%
\pgfpathlineto{\pgfqpoint{1.327325in}{0.219847in}}%
\pgfpathlineto{\pgfqpoint{1.320369in}{0.217790in}}%
\pgfpathlineto{\pgfqpoint{1.315278in}{0.216305in}}%
\pgfpathlineto{\pgfqpoint{1.303232in}{0.212959in}}%
\pgfpathlineto{\pgfqpoint{1.291186in}{0.209780in}}%
\pgfpathlineto{\pgfqpoint{1.279140in}{0.206768in}}%
\pgfpathlineto{\pgfqpoint{1.274800in}{0.205743in}}%
\pgfpathlineto{\pgfqpoint{1.267094in}{0.203949in}}%
\pgfpathlineto{\pgfqpoint{1.255048in}{0.201309in}}%
\pgfpathlineto{\pgfqpoint{1.243002in}{0.198833in}}%
\pgfpathlineto{\pgfqpoint{1.230956in}{0.196523in}}%
\pgfpathlineto{\pgfqpoint{1.218909in}{0.194378in}}%
\pgfpathlineto{\pgfqpoint{1.214769in}{0.193697in}}%
\pgfpathlineto{\pgfqpoint{1.206863in}{0.192415in}}%
\pgfpathlineto{\pgfqpoint{1.194817in}{0.190625in}}%
\pgfpathlineto{\pgfqpoint{1.182771in}{0.188997in}}%
\pgfpathlineto{\pgfqpoint{1.170725in}{0.187532in}}%
\pgfpathlineto{\pgfqpoint{1.158679in}{0.186230in}}%
\pgfpathlineto{\pgfqpoint{1.146633in}{0.185090in}}%
\pgfpathlineto{\pgfqpoint{1.134587in}{0.184113in}}%
\pgfpathlineto{\pgfqpoint{1.122541in}{0.183299in}}%
\pgfpathlineto{\pgfqpoint{1.110494in}{0.182648in}}%
\pgfpathlineto{\pgfqpoint{1.098448in}{0.182160in}}%
\pgfpathlineto{\pgfqpoint{1.086402in}{0.181834in}}%
\pgfpathlineto{\pgfqpoint{1.074356in}{0.181672in}}%
\pgfpathlineto{\pgfqpoint{1.062310in}{0.181672in}}%
\pgfpathlineto{\pgfqpoint{1.050264in}{0.181834in}}%
\pgfpathlineto{\pgfqpoint{1.038218in}{0.182160in}}%
\pgfpathlineto{\pgfqpoint{1.026172in}{0.182648in}}%
\pgfpathlineto{\pgfqpoint{1.014126in}{0.183299in}}%
\pgfpathlineto{\pgfqpoint{1.002079in}{0.184113in}}%
\pgfpathlineto{\pgfqpoint{0.990033in}{0.185090in}}%
\pgfpathlineto{\pgfqpoint{0.977987in}{0.186230in}}%
\pgfpathlineto{\pgfqpoint{0.965941in}{0.187532in}}%
\pgfpathlineto{\pgfqpoint{0.953895in}{0.188997in}}%
\pgfpathlineto{\pgfqpoint{0.941849in}{0.190625in}}%
\pgfpathlineto{\pgfqpoint{0.929803in}{0.192415in}}%
\pgfpathclose%
\pgfusepath{}%
\end{pgfscope}%
\begin{pgfscope}%
\pgfsetbuttcap%
\pgfsetroundjoin%
\definecolor{currentfill}{rgb}{0.000000,0.000000,0.000000}%
\pgfsetfillcolor{currentfill}%
\pgfsetlinewidth{0.803000pt}%
\definecolor{currentstroke}{rgb}{0.000000,0.000000,0.000000}%
\pgfsetstrokecolor{currentstroke}%
\pgfsetdash{}{0pt}%
\pgfsys@defobject{currentmarker}{\pgfqpoint{0.000000in}{-0.048611in}}{\pgfqpoint{0.000000in}{0.000000in}}{%
\pgfpathmoveto{\pgfqpoint{0.000000in}{0.000000in}}%
\pgfpathlineto{\pgfqpoint{0.000000in}{-0.048611in}}%
\pgfusepath{stroke,fill}%
}%
\begin{pgfscope}%
\pgfsys@transformshift{0.170897in}{1.079087in}%
\pgfsys@useobject{currentmarker}{}%
\end{pgfscope}%
\end{pgfscope}%
\begin{pgfscope}%
\pgftext[x=0.170897in,y=0.981865in,,top]{\sffamily\fontsize{10.000000}{12.000000}\selectfont -1}%
\end{pgfscope}%
\begin{pgfscope}%
\pgfsetbuttcap%
\pgfsetroundjoin%
\definecolor{currentfill}{rgb}{0.000000,0.000000,0.000000}%
\pgfsetfillcolor{currentfill}%
\pgfsetlinewidth{0.803000pt}%
\definecolor{currentstroke}{rgb}{0.000000,0.000000,0.000000}%
\pgfsetstrokecolor{currentstroke}%
\pgfsetdash{}{0pt}%
\pgfsys@defobject{currentmarker}{\pgfqpoint{0.000000in}{-0.048611in}}{\pgfqpoint{0.000000in}{0.000000in}}{%
\pgfpathmoveto{\pgfqpoint{0.000000in}{0.000000in}}%
\pgfpathlineto{\pgfqpoint{0.000000in}{-0.048611in}}%
\pgfusepath{stroke,fill}%
}%
\begin{pgfscope}%
\pgfsys@transformshift{0.619615in}{1.079087in}%
\pgfsys@useobject{currentmarker}{}%
\end{pgfscope}%
\end{pgfscope}%
\begin{pgfscope}%
\pgftext[x=0.619615in,y=0.981865in,,top]{\sffamily\fontsize{10.000000}{12.000000}\selectfont -0.5}%
\end{pgfscope}%
\begin{pgfscope}%
\pgfsetbuttcap%
\pgfsetroundjoin%
\definecolor{currentfill}{rgb}{0.000000,0.000000,0.000000}%
\pgfsetfillcolor{currentfill}%
\pgfsetlinewidth{0.803000pt}%
\definecolor{currentstroke}{rgb}{0.000000,0.000000,0.000000}%
\pgfsetstrokecolor{currentstroke}%
\pgfsetdash{}{0pt}%
\pgfsys@defobject{currentmarker}{\pgfqpoint{0.000000in}{-0.048611in}}{\pgfqpoint{0.000000in}{0.000000in}}{%
\pgfpathmoveto{\pgfqpoint{0.000000in}{0.000000in}}%
\pgfpathlineto{\pgfqpoint{0.000000in}{-0.048611in}}%
\pgfusepath{stroke,fill}%
}%
\begin{pgfscope}%
\pgfsys@transformshift{1.068333in}{1.079087in}%
\pgfsys@useobject{currentmarker}{}%
\end{pgfscope}%
\end{pgfscope}%
\begin{pgfscope}%
\pgfsetbuttcap%
\pgfsetroundjoin%
\definecolor{currentfill}{rgb}{0.000000,0.000000,0.000000}%
\pgfsetfillcolor{currentfill}%
\pgfsetlinewidth{0.803000pt}%
\definecolor{currentstroke}{rgb}{0.000000,0.000000,0.000000}%
\pgfsetstrokecolor{currentstroke}%
\pgfsetdash{}{0pt}%
\pgfsys@defobject{currentmarker}{\pgfqpoint{0.000000in}{-0.048611in}}{\pgfqpoint{0.000000in}{0.000000in}}{%
\pgfpathmoveto{\pgfqpoint{0.000000in}{0.000000in}}%
\pgfpathlineto{\pgfqpoint{0.000000in}{-0.048611in}}%
\pgfusepath{stroke,fill}%
}%
\begin{pgfscope}%
\pgfsys@transformshift{1.517051in}{1.079087in}%
\pgfsys@useobject{currentmarker}{}%
\end{pgfscope}%
\end{pgfscope}%
\begin{pgfscope}%
\pgftext[x=1.517051in,y=0.981865in,,top]{\sffamily\fontsize{10.000000}{12.000000}\selectfont 0.5}%
\end{pgfscope}%
\begin{pgfscope}%
\pgfsetbuttcap%
\pgfsetroundjoin%
\definecolor{currentfill}{rgb}{0.000000,0.000000,0.000000}%
\pgfsetfillcolor{currentfill}%
\pgfsetlinewidth{0.803000pt}%
\definecolor{currentstroke}{rgb}{0.000000,0.000000,0.000000}%
\pgfsetstrokecolor{currentstroke}%
\pgfsetdash{}{0pt}%
\pgfsys@defobject{currentmarker}{\pgfqpoint{0.000000in}{-0.048611in}}{\pgfqpoint{0.000000in}{0.000000in}}{%
\pgfpathmoveto{\pgfqpoint{0.000000in}{0.000000in}}%
\pgfpathlineto{\pgfqpoint{0.000000in}{-0.048611in}}%
\pgfusepath{stroke,fill}%
}%
\begin{pgfscope}%
\pgfsys@transformshift{1.965769in}{1.079087in}%
\pgfsys@useobject{currentmarker}{}%
\end{pgfscope}%
\end{pgfscope}%
\begin{pgfscope}%
\pgftext[x=1.965769in,y=0.981865in,,top]{\sffamily\fontsize{10.000000}{12.000000}\selectfont 1}%
\end{pgfscope}%
\begin{pgfscope}%
\pgfsetbuttcap%
\pgfsetroundjoin%
\definecolor{currentfill}{rgb}{0.000000,0.000000,0.000000}%
\pgfsetfillcolor{currentfill}%
\pgfsetlinewidth{0.602250pt}%
\definecolor{currentstroke}{rgb}{0.000000,0.000000,0.000000}%
\pgfsetstrokecolor{currentstroke}%
\pgfsetdash{}{0pt}%
\pgfsys@defobject{currentmarker}{\pgfqpoint{0.000000in}{-0.027778in}}{\pgfqpoint{0.000000in}{0.000000in}}{%
\pgfpathmoveto{\pgfqpoint{0.000000in}{0.000000in}}%
\pgfpathlineto{\pgfqpoint{0.000000in}{-0.027778in}}%
\pgfusepath{stroke,fill}%
}%
\begin{pgfscope}%
\pgfsys@transformshift{0.260641in}{1.079087in}%
\pgfsys@useobject{currentmarker}{}%
\end{pgfscope}%
\end{pgfscope}%
\begin{pgfscope}%
\pgfsetbuttcap%
\pgfsetroundjoin%
\definecolor{currentfill}{rgb}{0.000000,0.000000,0.000000}%
\pgfsetfillcolor{currentfill}%
\pgfsetlinewidth{0.602250pt}%
\definecolor{currentstroke}{rgb}{0.000000,0.000000,0.000000}%
\pgfsetstrokecolor{currentstroke}%
\pgfsetdash{}{0pt}%
\pgfsys@defobject{currentmarker}{\pgfqpoint{0.000000in}{-0.027778in}}{\pgfqpoint{0.000000in}{0.000000in}}{%
\pgfpathmoveto{\pgfqpoint{0.000000in}{0.000000in}}%
\pgfpathlineto{\pgfqpoint{0.000000in}{-0.027778in}}%
\pgfusepath{stroke,fill}%
}%
\begin{pgfscope}%
\pgfsys@transformshift{0.350385in}{1.079087in}%
\pgfsys@useobject{currentmarker}{}%
\end{pgfscope}%
\end{pgfscope}%
\begin{pgfscope}%
\pgfsetbuttcap%
\pgfsetroundjoin%
\definecolor{currentfill}{rgb}{0.000000,0.000000,0.000000}%
\pgfsetfillcolor{currentfill}%
\pgfsetlinewidth{0.602250pt}%
\definecolor{currentstroke}{rgb}{0.000000,0.000000,0.000000}%
\pgfsetstrokecolor{currentstroke}%
\pgfsetdash{}{0pt}%
\pgfsys@defobject{currentmarker}{\pgfqpoint{0.000000in}{-0.027778in}}{\pgfqpoint{0.000000in}{0.000000in}}{%
\pgfpathmoveto{\pgfqpoint{0.000000in}{0.000000in}}%
\pgfpathlineto{\pgfqpoint{0.000000in}{-0.027778in}}%
\pgfusepath{stroke,fill}%
}%
\begin{pgfscope}%
\pgfsys@transformshift{0.440128in}{1.079087in}%
\pgfsys@useobject{currentmarker}{}%
\end{pgfscope}%
\end{pgfscope}%
\begin{pgfscope}%
\pgfsetbuttcap%
\pgfsetroundjoin%
\definecolor{currentfill}{rgb}{0.000000,0.000000,0.000000}%
\pgfsetfillcolor{currentfill}%
\pgfsetlinewidth{0.602250pt}%
\definecolor{currentstroke}{rgb}{0.000000,0.000000,0.000000}%
\pgfsetstrokecolor{currentstroke}%
\pgfsetdash{}{0pt}%
\pgfsys@defobject{currentmarker}{\pgfqpoint{0.000000in}{-0.027778in}}{\pgfqpoint{0.000000in}{0.000000in}}{%
\pgfpathmoveto{\pgfqpoint{0.000000in}{0.000000in}}%
\pgfpathlineto{\pgfqpoint{0.000000in}{-0.027778in}}%
\pgfusepath{stroke,fill}%
}%
\begin{pgfscope}%
\pgfsys@transformshift{0.529872in}{1.079087in}%
\pgfsys@useobject{currentmarker}{}%
\end{pgfscope}%
\end{pgfscope}%
\begin{pgfscope}%
\pgfsetbuttcap%
\pgfsetroundjoin%
\definecolor{currentfill}{rgb}{0.000000,0.000000,0.000000}%
\pgfsetfillcolor{currentfill}%
\pgfsetlinewidth{0.602250pt}%
\definecolor{currentstroke}{rgb}{0.000000,0.000000,0.000000}%
\pgfsetstrokecolor{currentstroke}%
\pgfsetdash{}{0pt}%
\pgfsys@defobject{currentmarker}{\pgfqpoint{0.000000in}{-0.027778in}}{\pgfqpoint{0.000000in}{0.000000in}}{%
\pgfpathmoveto{\pgfqpoint{0.000000in}{0.000000in}}%
\pgfpathlineto{\pgfqpoint{0.000000in}{-0.027778in}}%
\pgfusepath{stroke,fill}%
}%
\begin{pgfscope}%
\pgfsys@transformshift{0.619615in}{1.079087in}%
\pgfsys@useobject{currentmarker}{}%
\end{pgfscope}%
\end{pgfscope}%
\begin{pgfscope}%
\pgfsetbuttcap%
\pgfsetroundjoin%
\definecolor{currentfill}{rgb}{0.000000,0.000000,0.000000}%
\pgfsetfillcolor{currentfill}%
\pgfsetlinewidth{0.602250pt}%
\definecolor{currentstroke}{rgb}{0.000000,0.000000,0.000000}%
\pgfsetstrokecolor{currentstroke}%
\pgfsetdash{}{0pt}%
\pgfsys@defobject{currentmarker}{\pgfqpoint{0.000000in}{-0.027778in}}{\pgfqpoint{0.000000in}{0.000000in}}{%
\pgfpathmoveto{\pgfqpoint{0.000000in}{0.000000in}}%
\pgfpathlineto{\pgfqpoint{0.000000in}{-0.027778in}}%
\pgfusepath{stroke,fill}%
}%
\begin{pgfscope}%
\pgfsys@transformshift{0.709359in}{1.079087in}%
\pgfsys@useobject{currentmarker}{}%
\end{pgfscope}%
\end{pgfscope}%
\begin{pgfscope}%
\pgfsetbuttcap%
\pgfsetroundjoin%
\definecolor{currentfill}{rgb}{0.000000,0.000000,0.000000}%
\pgfsetfillcolor{currentfill}%
\pgfsetlinewidth{0.602250pt}%
\definecolor{currentstroke}{rgb}{0.000000,0.000000,0.000000}%
\pgfsetstrokecolor{currentstroke}%
\pgfsetdash{}{0pt}%
\pgfsys@defobject{currentmarker}{\pgfqpoint{0.000000in}{-0.027778in}}{\pgfqpoint{0.000000in}{0.000000in}}{%
\pgfpathmoveto{\pgfqpoint{0.000000in}{0.000000in}}%
\pgfpathlineto{\pgfqpoint{0.000000in}{-0.027778in}}%
\pgfusepath{stroke,fill}%
}%
\begin{pgfscope}%
\pgfsys@transformshift{0.799102in}{1.079087in}%
\pgfsys@useobject{currentmarker}{}%
\end{pgfscope}%
\end{pgfscope}%
\begin{pgfscope}%
\pgfsetbuttcap%
\pgfsetroundjoin%
\definecolor{currentfill}{rgb}{0.000000,0.000000,0.000000}%
\pgfsetfillcolor{currentfill}%
\pgfsetlinewidth{0.602250pt}%
\definecolor{currentstroke}{rgb}{0.000000,0.000000,0.000000}%
\pgfsetstrokecolor{currentstroke}%
\pgfsetdash{}{0pt}%
\pgfsys@defobject{currentmarker}{\pgfqpoint{0.000000in}{-0.027778in}}{\pgfqpoint{0.000000in}{0.000000in}}{%
\pgfpathmoveto{\pgfqpoint{0.000000in}{0.000000in}}%
\pgfpathlineto{\pgfqpoint{0.000000in}{-0.027778in}}%
\pgfusepath{stroke,fill}%
}%
\begin{pgfscope}%
\pgfsys@transformshift{0.888846in}{1.079087in}%
\pgfsys@useobject{currentmarker}{}%
\end{pgfscope}%
\end{pgfscope}%
\begin{pgfscope}%
\pgfsetbuttcap%
\pgfsetroundjoin%
\definecolor{currentfill}{rgb}{0.000000,0.000000,0.000000}%
\pgfsetfillcolor{currentfill}%
\pgfsetlinewidth{0.602250pt}%
\definecolor{currentstroke}{rgb}{0.000000,0.000000,0.000000}%
\pgfsetstrokecolor{currentstroke}%
\pgfsetdash{}{0pt}%
\pgfsys@defobject{currentmarker}{\pgfqpoint{0.000000in}{-0.027778in}}{\pgfqpoint{0.000000in}{0.000000in}}{%
\pgfpathmoveto{\pgfqpoint{0.000000in}{0.000000in}}%
\pgfpathlineto{\pgfqpoint{0.000000in}{-0.027778in}}%
\pgfusepath{stroke,fill}%
}%
\begin{pgfscope}%
\pgfsys@transformshift{0.978589in}{1.079087in}%
\pgfsys@useobject{currentmarker}{}%
\end{pgfscope}%
\end{pgfscope}%
\begin{pgfscope}%
\pgfsetbuttcap%
\pgfsetroundjoin%
\definecolor{currentfill}{rgb}{0.000000,0.000000,0.000000}%
\pgfsetfillcolor{currentfill}%
\pgfsetlinewidth{0.602250pt}%
\definecolor{currentstroke}{rgb}{0.000000,0.000000,0.000000}%
\pgfsetstrokecolor{currentstroke}%
\pgfsetdash{}{0pt}%
\pgfsys@defobject{currentmarker}{\pgfqpoint{0.000000in}{-0.027778in}}{\pgfqpoint{0.000000in}{0.000000in}}{%
\pgfpathmoveto{\pgfqpoint{0.000000in}{0.000000in}}%
\pgfpathlineto{\pgfqpoint{0.000000in}{-0.027778in}}%
\pgfusepath{stroke,fill}%
}%
\begin{pgfscope}%
\pgfsys@transformshift{1.068333in}{1.079087in}%
\pgfsys@useobject{currentmarker}{}%
\end{pgfscope}%
\end{pgfscope}%
\begin{pgfscope}%
\pgfsetbuttcap%
\pgfsetroundjoin%
\definecolor{currentfill}{rgb}{0.000000,0.000000,0.000000}%
\pgfsetfillcolor{currentfill}%
\pgfsetlinewidth{0.602250pt}%
\definecolor{currentstroke}{rgb}{0.000000,0.000000,0.000000}%
\pgfsetstrokecolor{currentstroke}%
\pgfsetdash{}{0pt}%
\pgfsys@defobject{currentmarker}{\pgfqpoint{0.000000in}{-0.027778in}}{\pgfqpoint{0.000000in}{0.000000in}}{%
\pgfpathmoveto{\pgfqpoint{0.000000in}{0.000000in}}%
\pgfpathlineto{\pgfqpoint{0.000000in}{-0.027778in}}%
\pgfusepath{stroke,fill}%
}%
\begin{pgfscope}%
\pgfsys@transformshift{1.158077in}{1.079087in}%
\pgfsys@useobject{currentmarker}{}%
\end{pgfscope}%
\end{pgfscope}%
\begin{pgfscope}%
\pgfsetbuttcap%
\pgfsetroundjoin%
\definecolor{currentfill}{rgb}{0.000000,0.000000,0.000000}%
\pgfsetfillcolor{currentfill}%
\pgfsetlinewidth{0.602250pt}%
\definecolor{currentstroke}{rgb}{0.000000,0.000000,0.000000}%
\pgfsetstrokecolor{currentstroke}%
\pgfsetdash{}{0pt}%
\pgfsys@defobject{currentmarker}{\pgfqpoint{0.000000in}{-0.027778in}}{\pgfqpoint{0.000000in}{0.000000in}}{%
\pgfpathmoveto{\pgfqpoint{0.000000in}{0.000000in}}%
\pgfpathlineto{\pgfqpoint{0.000000in}{-0.027778in}}%
\pgfusepath{stroke,fill}%
}%
\begin{pgfscope}%
\pgfsys@transformshift{1.247820in}{1.079087in}%
\pgfsys@useobject{currentmarker}{}%
\end{pgfscope}%
\end{pgfscope}%
\begin{pgfscope}%
\pgfsetbuttcap%
\pgfsetroundjoin%
\definecolor{currentfill}{rgb}{0.000000,0.000000,0.000000}%
\pgfsetfillcolor{currentfill}%
\pgfsetlinewidth{0.602250pt}%
\definecolor{currentstroke}{rgb}{0.000000,0.000000,0.000000}%
\pgfsetstrokecolor{currentstroke}%
\pgfsetdash{}{0pt}%
\pgfsys@defobject{currentmarker}{\pgfqpoint{0.000000in}{-0.027778in}}{\pgfqpoint{0.000000in}{0.000000in}}{%
\pgfpathmoveto{\pgfqpoint{0.000000in}{0.000000in}}%
\pgfpathlineto{\pgfqpoint{0.000000in}{-0.027778in}}%
\pgfusepath{stroke,fill}%
}%
\begin{pgfscope}%
\pgfsys@transformshift{1.337564in}{1.079087in}%
\pgfsys@useobject{currentmarker}{}%
\end{pgfscope}%
\end{pgfscope}%
\begin{pgfscope}%
\pgfsetbuttcap%
\pgfsetroundjoin%
\definecolor{currentfill}{rgb}{0.000000,0.000000,0.000000}%
\pgfsetfillcolor{currentfill}%
\pgfsetlinewidth{0.602250pt}%
\definecolor{currentstroke}{rgb}{0.000000,0.000000,0.000000}%
\pgfsetstrokecolor{currentstroke}%
\pgfsetdash{}{0pt}%
\pgfsys@defobject{currentmarker}{\pgfqpoint{0.000000in}{-0.027778in}}{\pgfqpoint{0.000000in}{0.000000in}}{%
\pgfpathmoveto{\pgfqpoint{0.000000in}{0.000000in}}%
\pgfpathlineto{\pgfqpoint{0.000000in}{-0.027778in}}%
\pgfusepath{stroke,fill}%
}%
\begin{pgfscope}%
\pgfsys@transformshift{1.427307in}{1.079087in}%
\pgfsys@useobject{currentmarker}{}%
\end{pgfscope}%
\end{pgfscope}%
\begin{pgfscope}%
\pgfsetbuttcap%
\pgfsetroundjoin%
\definecolor{currentfill}{rgb}{0.000000,0.000000,0.000000}%
\pgfsetfillcolor{currentfill}%
\pgfsetlinewidth{0.602250pt}%
\definecolor{currentstroke}{rgb}{0.000000,0.000000,0.000000}%
\pgfsetstrokecolor{currentstroke}%
\pgfsetdash{}{0pt}%
\pgfsys@defobject{currentmarker}{\pgfqpoint{0.000000in}{-0.027778in}}{\pgfqpoint{0.000000in}{0.000000in}}{%
\pgfpathmoveto{\pgfqpoint{0.000000in}{0.000000in}}%
\pgfpathlineto{\pgfqpoint{0.000000in}{-0.027778in}}%
\pgfusepath{stroke,fill}%
}%
\begin{pgfscope}%
\pgfsys@transformshift{1.517051in}{1.079087in}%
\pgfsys@useobject{currentmarker}{}%
\end{pgfscope}%
\end{pgfscope}%
\begin{pgfscope}%
\pgfsetbuttcap%
\pgfsetroundjoin%
\definecolor{currentfill}{rgb}{0.000000,0.000000,0.000000}%
\pgfsetfillcolor{currentfill}%
\pgfsetlinewidth{0.602250pt}%
\definecolor{currentstroke}{rgb}{0.000000,0.000000,0.000000}%
\pgfsetstrokecolor{currentstroke}%
\pgfsetdash{}{0pt}%
\pgfsys@defobject{currentmarker}{\pgfqpoint{0.000000in}{-0.027778in}}{\pgfqpoint{0.000000in}{0.000000in}}{%
\pgfpathmoveto{\pgfqpoint{0.000000in}{0.000000in}}%
\pgfpathlineto{\pgfqpoint{0.000000in}{-0.027778in}}%
\pgfusepath{stroke,fill}%
}%
\begin{pgfscope}%
\pgfsys@transformshift{1.606794in}{1.079087in}%
\pgfsys@useobject{currentmarker}{}%
\end{pgfscope}%
\end{pgfscope}%
\begin{pgfscope}%
\pgfsetbuttcap%
\pgfsetroundjoin%
\definecolor{currentfill}{rgb}{0.000000,0.000000,0.000000}%
\pgfsetfillcolor{currentfill}%
\pgfsetlinewidth{0.602250pt}%
\definecolor{currentstroke}{rgb}{0.000000,0.000000,0.000000}%
\pgfsetstrokecolor{currentstroke}%
\pgfsetdash{}{0pt}%
\pgfsys@defobject{currentmarker}{\pgfqpoint{0.000000in}{-0.027778in}}{\pgfqpoint{0.000000in}{0.000000in}}{%
\pgfpathmoveto{\pgfqpoint{0.000000in}{0.000000in}}%
\pgfpathlineto{\pgfqpoint{0.000000in}{-0.027778in}}%
\pgfusepath{stroke,fill}%
}%
\begin{pgfscope}%
\pgfsys@transformshift{1.696538in}{1.079087in}%
\pgfsys@useobject{currentmarker}{}%
\end{pgfscope}%
\end{pgfscope}%
\begin{pgfscope}%
\pgfsetbuttcap%
\pgfsetroundjoin%
\definecolor{currentfill}{rgb}{0.000000,0.000000,0.000000}%
\pgfsetfillcolor{currentfill}%
\pgfsetlinewidth{0.602250pt}%
\definecolor{currentstroke}{rgb}{0.000000,0.000000,0.000000}%
\pgfsetstrokecolor{currentstroke}%
\pgfsetdash{}{0pt}%
\pgfsys@defobject{currentmarker}{\pgfqpoint{0.000000in}{-0.027778in}}{\pgfqpoint{0.000000in}{0.000000in}}{%
\pgfpathmoveto{\pgfqpoint{0.000000in}{0.000000in}}%
\pgfpathlineto{\pgfqpoint{0.000000in}{-0.027778in}}%
\pgfusepath{stroke,fill}%
}%
\begin{pgfscope}%
\pgfsys@transformshift{1.786282in}{1.079087in}%
\pgfsys@useobject{currentmarker}{}%
\end{pgfscope}%
\end{pgfscope}%
\begin{pgfscope}%
\pgfsetbuttcap%
\pgfsetroundjoin%
\definecolor{currentfill}{rgb}{0.000000,0.000000,0.000000}%
\pgfsetfillcolor{currentfill}%
\pgfsetlinewidth{0.602250pt}%
\definecolor{currentstroke}{rgb}{0.000000,0.000000,0.000000}%
\pgfsetstrokecolor{currentstroke}%
\pgfsetdash{}{0pt}%
\pgfsys@defobject{currentmarker}{\pgfqpoint{0.000000in}{-0.027778in}}{\pgfqpoint{0.000000in}{0.000000in}}{%
\pgfpathmoveto{\pgfqpoint{0.000000in}{0.000000in}}%
\pgfpathlineto{\pgfqpoint{0.000000in}{-0.027778in}}%
\pgfusepath{stroke,fill}%
}%
\begin{pgfscope}%
\pgfsys@transformshift{1.876025in}{1.079087in}%
\pgfsys@useobject{currentmarker}{}%
\end{pgfscope}%
\end{pgfscope}%
\begin{pgfscope}%
\pgfsetbuttcap%
\pgfsetroundjoin%
\definecolor{currentfill}{rgb}{0.000000,0.000000,0.000000}%
\pgfsetfillcolor{currentfill}%
\pgfsetlinewidth{0.602250pt}%
\definecolor{currentstroke}{rgb}{0.000000,0.000000,0.000000}%
\pgfsetstrokecolor{currentstroke}%
\pgfsetdash{}{0pt}%
\pgfsys@defobject{currentmarker}{\pgfqpoint{0.000000in}{-0.027778in}}{\pgfqpoint{0.000000in}{0.000000in}}{%
\pgfpathmoveto{\pgfqpoint{0.000000in}{0.000000in}}%
\pgfpathlineto{\pgfqpoint{0.000000in}{-0.027778in}}%
\pgfusepath{stroke,fill}%
}%
\begin{pgfscope}%
\pgfsys@transformshift{1.965769in}{1.079087in}%
\pgfsys@useobject{currentmarker}{}%
\end{pgfscope}%
\end{pgfscope}%
\begin{pgfscope}%
\pgfsetbuttcap%
\pgfsetroundjoin%
\definecolor{currentfill}{rgb}{0.000000,0.000000,0.000000}%
\pgfsetfillcolor{currentfill}%
\pgfsetlinewidth{0.803000pt}%
\definecolor{currentstroke}{rgb}{0.000000,0.000000,0.000000}%
\pgfsetstrokecolor{currentstroke}%
\pgfsetdash{}{0pt}%
\pgfsys@defobject{currentmarker}{\pgfqpoint{-0.048611in}{0.000000in}}{\pgfqpoint{0.000000in}{0.000000in}}{%
\pgfpathmoveto{\pgfqpoint{0.000000in}{0.000000in}}%
\pgfpathlineto{\pgfqpoint{-0.048611in}{0.000000in}}%
\pgfusepath{stroke,fill}%
}%
\begin{pgfscope}%
\pgfsys@transformshift{1.068333in}{0.181651in}%
\pgfsys@useobject{currentmarker}{}%
\end{pgfscope}%
\end{pgfscope}%
\begin{pgfscope}%
\pgftext[x=0.832629in,y=0.128890in,left,base]{\sffamily\fontsize{10.000000}{12.000000}\selectfont -1}%
\end{pgfscope}%
\begin{pgfscope}%
\pgfsetbuttcap%
\pgfsetroundjoin%
\definecolor{currentfill}{rgb}{0.000000,0.000000,0.000000}%
\pgfsetfillcolor{currentfill}%
\pgfsetlinewidth{0.803000pt}%
\definecolor{currentstroke}{rgb}{0.000000,0.000000,0.000000}%
\pgfsetstrokecolor{currentstroke}%
\pgfsetdash{}{0pt}%
\pgfsys@defobject{currentmarker}{\pgfqpoint{-0.048611in}{0.000000in}}{\pgfqpoint{0.000000in}{0.000000in}}{%
\pgfpathmoveto{\pgfqpoint{0.000000in}{0.000000in}}%
\pgfpathlineto{\pgfqpoint{-0.048611in}{0.000000in}}%
\pgfusepath{stroke,fill}%
}%
\begin{pgfscope}%
\pgfsys@transformshift{1.068333in}{0.630369in}%
\pgfsys@useobject{currentmarker}{}%
\end{pgfscope}%
\end{pgfscope}%
\begin{pgfscope}%
\pgftext[x=0.700115in,y=0.577608in,left,base]{\sffamily\fontsize{10.000000}{12.000000}\selectfont -0.5}%
\end{pgfscope}%
\begin{pgfscope}%
\pgfsetbuttcap%
\pgfsetroundjoin%
\definecolor{currentfill}{rgb}{0.000000,0.000000,0.000000}%
\pgfsetfillcolor{currentfill}%
\pgfsetlinewidth{0.803000pt}%
\definecolor{currentstroke}{rgb}{0.000000,0.000000,0.000000}%
\pgfsetstrokecolor{currentstroke}%
\pgfsetdash{}{0pt}%
\pgfsys@defobject{currentmarker}{\pgfqpoint{-0.048611in}{0.000000in}}{\pgfqpoint{0.000000in}{0.000000in}}{%
\pgfpathmoveto{\pgfqpoint{0.000000in}{0.000000in}}%
\pgfpathlineto{\pgfqpoint{-0.048611in}{0.000000in}}%
\pgfusepath{stroke,fill}%
}%
\begin{pgfscope}%
\pgfsys@transformshift{1.068333in}{1.079087in}%
\pgfsys@useobject{currentmarker}{}%
\end{pgfscope}%
\end{pgfscope}%
\begin{pgfscope}%
\pgfsetbuttcap%
\pgfsetroundjoin%
\definecolor{currentfill}{rgb}{0.000000,0.000000,0.000000}%
\pgfsetfillcolor{currentfill}%
\pgfsetlinewidth{0.803000pt}%
\definecolor{currentstroke}{rgb}{0.000000,0.000000,0.000000}%
\pgfsetstrokecolor{currentstroke}%
\pgfsetdash{}{0pt}%
\pgfsys@defobject{currentmarker}{\pgfqpoint{-0.048611in}{0.000000in}}{\pgfqpoint{0.000000in}{0.000000in}}{%
\pgfpathmoveto{\pgfqpoint{0.000000in}{0.000000in}}%
\pgfpathlineto{\pgfqpoint{-0.048611in}{0.000000in}}%
\pgfusepath{stroke,fill}%
}%
\begin{pgfscope}%
\pgfsys@transformshift{1.068333in}{1.527805in}%
\pgfsys@useobject{currentmarker}{}%
\end{pgfscope}%
\end{pgfscope}%
\begin{pgfscope}%
\pgftext[x=0.750231in,y=1.475043in,left,base]{\sffamily\fontsize{10.000000}{12.000000}\selectfont 0.5}%
\end{pgfscope}%
\begin{pgfscope}%
\pgfsetbuttcap%
\pgfsetroundjoin%
\definecolor{currentfill}{rgb}{0.000000,0.000000,0.000000}%
\pgfsetfillcolor{currentfill}%
\pgfsetlinewidth{0.803000pt}%
\definecolor{currentstroke}{rgb}{0.000000,0.000000,0.000000}%
\pgfsetstrokecolor{currentstroke}%
\pgfsetdash{}{0pt}%
\pgfsys@defobject{currentmarker}{\pgfqpoint{-0.048611in}{0.000000in}}{\pgfqpoint{0.000000in}{0.000000in}}{%
\pgfpathmoveto{\pgfqpoint{0.000000in}{0.000000in}}%
\pgfpathlineto{\pgfqpoint{-0.048611in}{0.000000in}}%
\pgfusepath{stroke,fill}%
}%
\begin{pgfscope}%
\pgfsys@transformshift{1.068333in}{1.976522in}%
\pgfsys@useobject{currentmarker}{}%
\end{pgfscope}%
\end{pgfscope}%
\begin{pgfscope}%
\pgftext[x=0.882746in,y=1.923761in,left,base]{\sffamily\fontsize{10.000000}{12.000000}\selectfont 1}%
\end{pgfscope}%
\begin{pgfscope}%
\pgfsetbuttcap%
\pgfsetroundjoin%
\definecolor{currentfill}{rgb}{0.000000,0.000000,0.000000}%
\pgfsetfillcolor{currentfill}%
\pgfsetlinewidth{0.602250pt}%
\definecolor{currentstroke}{rgb}{0.000000,0.000000,0.000000}%
\pgfsetstrokecolor{currentstroke}%
\pgfsetdash{}{0pt}%
\pgfsys@defobject{currentmarker}{\pgfqpoint{-0.027778in}{0.000000in}}{\pgfqpoint{0.000000in}{0.000000in}}{%
\pgfpathmoveto{\pgfqpoint{0.000000in}{0.000000in}}%
\pgfpathlineto{\pgfqpoint{-0.027778in}{0.000000in}}%
\pgfusepath{stroke,fill}%
}%
\begin{pgfscope}%
\pgfsys@transformshift{1.068333in}{0.271395in}%
\pgfsys@useobject{currentmarker}{}%
\end{pgfscope}%
\end{pgfscope}%
\begin{pgfscope}%
\pgfsetbuttcap%
\pgfsetroundjoin%
\definecolor{currentfill}{rgb}{0.000000,0.000000,0.000000}%
\pgfsetfillcolor{currentfill}%
\pgfsetlinewidth{0.602250pt}%
\definecolor{currentstroke}{rgb}{0.000000,0.000000,0.000000}%
\pgfsetstrokecolor{currentstroke}%
\pgfsetdash{}{0pt}%
\pgfsys@defobject{currentmarker}{\pgfqpoint{-0.027778in}{0.000000in}}{\pgfqpoint{0.000000in}{0.000000in}}{%
\pgfpathmoveto{\pgfqpoint{0.000000in}{0.000000in}}%
\pgfpathlineto{\pgfqpoint{-0.027778in}{0.000000in}}%
\pgfusepath{stroke,fill}%
}%
\begin{pgfscope}%
\pgfsys@transformshift{1.068333in}{0.361138in}%
\pgfsys@useobject{currentmarker}{}%
\end{pgfscope}%
\end{pgfscope}%
\begin{pgfscope}%
\pgfsetbuttcap%
\pgfsetroundjoin%
\definecolor{currentfill}{rgb}{0.000000,0.000000,0.000000}%
\pgfsetfillcolor{currentfill}%
\pgfsetlinewidth{0.602250pt}%
\definecolor{currentstroke}{rgb}{0.000000,0.000000,0.000000}%
\pgfsetstrokecolor{currentstroke}%
\pgfsetdash{}{0pt}%
\pgfsys@defobject{currentmarker}{\pgfqpoint{-0.027778in}{0.000000in}}{\pgfqpoint{0.000000in}{0.000000in}}{%
\pgfpathmoveto{\pgfqpoint{0.000000in}{0.000000in}}%
\pgfpathlineto{\pgfqpoint{-0.027778in}{0.000000in}}%
\pgfusepath{stroke,fill}%
}%
\begin{pgfscope}%
\pgfsys@transformshift{1.068333in}{0.450882in}%
\pgfsys@useobject{currentmarker}{}%
\end{pgfscope}%
\end{pgfscope}%
\begin{pgfscope}%
\pgfsetbuttcap%
\pgfsetroundjoin%
\definecolor{currentfill}{rgb}{0.000000,0.000000,0.000000}%
\pgfsetfillcolor{currentfill}%
\pgfsetlinewidth{0.602250pt}%
\definecolor{currentstroke}{rgb}{0.000000,0.000000,0.000000}%
\pgfsetstrokecolor{currentstroke}%
\pgfsetdash{}{0pt}%
\pgfsys@defobject{currentmarker}{\pgfqpoint{-0.027778in}{0.000000in}}{\pgfqpoint{0.000000in}{0.000000in}}{%
\pgfpathmoveto{\pgfqpoint{0.000000in}{0.000000in}}%
\pgfpathlineto{\pgfqpoint{-0.027778in}{0.000000in}}%
\pgfusepath{stroke,fill}%
}%
\begin{pgfscope}%
\pgfsys@transformshift{1.068333in}{0.540625in}%
\pgfsys@useobject{currentmarker}{}%
\end{pgfscope}%
\end{pgfscope}%
\begin{pgfscope}%
\pgfsetbuttcap%
\pgfsetroundjoin%
\definecolor{currentfill}{rgb}{0.000000,0.000000,0.000000}%
\pgfsetfillcolor{currentfill}%
\pgfsetlinewidth{0.602250pt}%
\definecolor{currentstroke}{rgb}{0.000000,0.000000,0.000000}%
\pgfsetstrokecolor{currentstroke}%
\pgfsetdash{}{0pt}%
\pgfsys@defobject{currentmarker}{\pgfqpoint{-0.027778in}{0.000000in}}{\pgfqpoint{0.000000in}{0.000000in}}{%
\pgfpathmoveto{\pgfqpoint{0.000000in}{0.000000in}}%
\pgfpathlineto{\pgfqpoint{-0.027778in}{0.000000in}}%
\pgfusepath{stroke,fill}%
}%
\begin{pgfscope}%
\pgfsys@transformshift{1.068333in}{0.630369in}%
\pgfsys@useobject{currentmarker}{}%
\end{pgfscope}%
\end{pgfscope}%
\begin{pgfscope}%
\pgfsetbuttcap%
\pgfsetroundjoin%
\definecolor{currentfill}{rgb}{0.000000,0.000000,0.000000}%
\pgfsetfillcolor{currentfill}%
\pgfsetlinewidth{0.602250pt}%
\definecolor{currentstroke}{rgb}{0.000000,0.000000,0.000000}%
\pgfsetstrokecolor{currentstroke}%
\pgfsetdash{}{0pt}%
\pgfsys@defobject{currentmarker}{\pgfqpoint{-0.027778in}{0.000000in}}{\pgfqpoint{0.000000in}{0.000000in}}{%
\pgfpathmoveto{\pgfqpoint{0.000000in}{0.000000in}}%
\pgfpathlineto{\pgfqpoint{-0.027778in}{0.000000in}}%
\pgfusepath{stroke,fill}%
}%
\begin{pgfscope}%
\pgfsys@transformshift{1.068333in}{0.720113in}%
\pgfsys@useobject{currentmarker}{}%
\end{pgfscope}%
\end{pgfscope}%
\begin{pgfscope}%
\pgfsetbuttcap%
\pgfsetroundjoin%
\definecolor{currentfill}{rgb}{0.000000,0.000000,0.000000}%
\pgfsetfillcolor{currentfill}%
\pgfsetlinewidth{0.602250pt}%
\definecolor{currentstroke}{rgb}{0.000000,0.000000,0.000000}%
\pgfsetstrokecolor{currentstroke}%
\pgfsetdash{}{0pt}%
\pgfsys@defobject{currentmarker}{\pgfqpoint{-0.027778in}{0.000000in}}{\pgfqpoint{0.000000in}{0.000000in}}{%
\pgfpathmoveto{\pgfqpoint{0.000000in}{0.000000in}}%
\pgfpathlineto{\pgfqpoint{-0.027778in}{0.000000in}}%
\pgfusepath{stroke,fill}%
}%
\begin{pgfscope}%
\pgfsys@transformshift{1.068333in}{0.809856in}%
\pgfsys@useobject{currentmarker}{}%
\end{pgfscope}%
\end{pgfscope}%
\begin{pgfscope}%
\pgfsetbuttcap%
\pgfsetroundjoin%
\definecolor{currentfill}{rgb}{0.000000,0.000000,0.000000}%
\pgfsetfillcolor{currentfill}%
\pgfsetlinewidth{0.602250pt}%
\definecolor{currentstroke}{rgb}{0.000000,0.000000,0.000000}%
\pgfsetstrokecolor{currentstroke}%
\pgfsetdash{}{0pt}%
\pgfsys@defobject{currentmarker}{\pgfqpoint{-0.027778in}{0.000000in}}{\pgfqpoint{0.000000in}{0.000000in}}{%
\pgfpathmoveto{\pgfqpoint{0.000000in}{0.000000in}}%
\pgfpathlineto{\pgfqpoint{-0.027778in}{0.000000in}}%
\pgfusepath{stroke,fill}%
}%
\begin{pgfscope}%
\pgfsys@transformshift{1.068333in}{0.899600in}%
\pgfsys@useobject{currentmarker}{}%
\end{pgfscope}%
\end{pgfscope}%
\begin{pgfscope}%
\pgfsetbuttcap%
\pgfsetroundjoin%
\definecolor{currentfill}{rgb}{0.000000,0.000000,0.000000}%
\pgfsetfillcolor{currentfill}%
\pgfsetlinewidth{0.602250pt}%
\definecolor{currentstroke}{rgb}{0.000000,0.000000,0.000000}%
\pgfsetstrokecolor{currentstroke}%
\pgfsetdash{}{0pt}%
\pgfsys@defobject{currentmarker}{\pgfqpoint{-0.027778in}{0.000000in}}{\pgfqpoint{0.000000in}{0.000000in}}{%
\pgfpathmoveto{\pgfqpoint{0.000000in}{0.000000in}}%
\pgfpathlineto{\pgfqpoint{-0.027778in}{0.000000in}}%
\pgfusepath{stroke,fill}%
}%
\begin{pgfscope}%
\pgfsys@transformshift{1.068333in}{0.989343in}%
\pgfsys@useobject{currentmarker}{}%
\end{pgfscope}%
\end{pgfscope}%
\begin{pgfscope}%
\pgfsetbuttcap%
\pgfsetroundjoin%
\definecolor{currentfill}{rgb}{0.000000,0.000000,0.000000}%
\pgfsetfillcolor{currentfill}%
\pgfsetlinewidth{0.602250pt}%
\definecolor{currentstroke}{rgb}{0.000000,0.000000,0.000000}%
\pgfsetstrokecolor{currentstroke}%
\pgfsetdash{}{0pt}%
\pgfsys@defobject{currentmarker}{\pgfqpoint{-0.027778in}{0.000000in}}{\pgfqpoint{0.000000in}{0.000000in}}{%
\pgfpathmoveto{\pgfqpoint{0.000000in}{0.000000in}}%
\pgfpathlineto{\pgfqpoint{-0.027778in}{0.000000in}}%
\pgfusepath{stroke,fill}%
}%
\begin{pgfscope}%
\pgfsys@transformshift{1.068333in}{1.079087in}%
\pgfsys@useobject{currentmarker}{}%
\end{pgfscope}%
\end{pgfscope}%
\begin{pgfscope}%
\pgfsetbuttcap%
\pgfsetroundjoin%
\definecolor{currentfill}{rgb}{0.000000,0.000000,0.000000}%
\pgfsetfillcolor{currentfill}%
\pgfsetlinewidth{0.602250pt}%
\definecolor{currentstroke}{rgb}{0.000000,0.000000,0.000000}%
\pgfsetstrokecolor{currentstroke}%
\pgfsetdash{}{0pt}%
\pgfsys@defobject{currentmarker}{\pgfqpoint{-0.027778in}{0.000000in}}{\pgfqpoint{0.000000in}{0.000000in}}{%
\pgfpathmoveto{\pgfqpoint{0.000000in}{0.000000in}}%
\pgfpathlineto{\pgfqpoint{-0.027778in}{0.000000in}}%
\pgfusepath{stroke,fill}%
}%
\begin{pgfscope}%
\pgfsys@transformshift{1.068333in}{1.168830in}%
\pgfsys@useobject{currentmarker}{}%
\end{pgfscope}%
\end{pgfscope}%
\begin{pgfscope}%
\pgfsetbuttcap%
\pgfsetroundjoin%
\definecolor{currentfill}{rgb}{0.000000,0.000000,0.000000}%
\pgfsetfillcolor{currentfill}%
\pgfsetlinewidth{0.602250pt}%
\definecolor{currentstroke}{rgb}{0.000000,0.000000,0.000000}%
\pgfsetstrokecolor{currentstroke}%
\pgfsetdash{}{0pt}%
\pgfsys@defobject{currentmarker}{\pgfqpoint{-0.027778in}{0.000000in}}{\pgfqpoint{0.000000in}{0.000000in}}{%
\pgfpathmoveto{\pgfqpoint{0.000000in}{0.000000in}}%
\pgfpathlineto{\pgfqpoint{-0.027778in}{0.000000in}}%
\pgfusepath{stroke,fill}%
}%
\begin{pgfscope}%
\pgfsys@transformshift{1.068333in}{1.258574in}%
\pgfsys@useobject{currentmarker}{}%
\end{pgfscope}%
\end{pgfscope}%
\begin{pgfscope}%
\pgfsetbuttcap%
\pgfsetroundjoin%
\definecolor{currentfill}{rgb}{0.000000,0.000000,0.000000}%
\pgfsetfillcolor{currentfill}%
\pgfsetlinewidth{0.602250pt}%
\definecolor{currentstroke}{rgb}{0.000000,0.000000,0.000000}%
\pgfsetstrokecolor{currentstroke}%
\pgfsetdash{}{0pt}%
\pgfsys@defobject{currentmarker}{\pgfqpoint{-0.027778in}{0.000000in}}{\pgfqpoint{0.000000in}{0.000000in}}{%
\pgfpathmoveto{\pgfqpoint{0.000000in}{0.000000in}}%
\pgfpathlineto{\pgfqpoint{-0.027778in}{0.000000in}}%
\pgfusepath{stroke,fill}%
}%
\begin{pgfscope}%
\pgfsys@transformshift{1.068333in}{1.348318in}%
\pgfsys@useobject{currentmarker}{}%
\end{pgfscope}%
\end{pgfscope}%
\begin{pgfscope}%
\pgfsetbuttcap%
\pgfsetroundjoin%
\definecolor{currentfill}{rgb}{0.000000,0.000000,0.000000}%
\pgfsetfillcolor{currentfill}%
\pgfsetlinewidth{0.602250pt}%
\definecolor{currentstroke}{rgb}{0.000000,0.000000,0.000000}%
\pgfsetstrokecolor{currentstroke}%
\pgfsetdash{}{0pt}%
\pgfsys@defobject{currentmarker}{\pgfqpoint{-0.027778in}{0.000000in}}{\pgfqpoint{0.000000in}{0.000000in}}{%
\pgfpathmoveto{\pgfqpoint{0.000000in}{0.000000in}}%
\pgfpathlineto{\pgfqpoint{-0.027778in}{0.000000in}}%
\pgfusepath{stroke,fill}%
}%
\begin{pgfscope}%
\pgfsys@transformshift{1.068333in}{1.438061in}%
\pgfsys@useobject{currentmarker}{}%
\end{pgfscope}%
\end{pgfscope}%
\begin{pgfscope}%
\pgfsetbuttcap%
\pgfsetroundjoin%
\definecolor{currentfill}{rgb}{0.000000,0.000000,0.000000}%
\pgfsetfillcolor{currentfill}%
\pgfsetlinewidth{0.602250pt}%
\definecolor{currentstroke}{rgb}{0.000000,0.000000,0.000000}%
\pgfsetstrokecolor{currentstroke}%
\pgfsetdash{}{0pt}%
\pgfsys@defobject{currentmarker}{\pgfqpoint{-0.027778in}{0.000000in}}{\pgfqpoint{0.000000in}{0.000000in}}{%
\pgfpathmoveto{\pgfqpoint{0.000000in}{0.000000in}}%
\pgfpathlineto{\pgfqpoint{-0.027778in}{0.000000in}}%
\pgfusepath{stroke,fill}%
}%
\begin{pgfscope}%
\pgfsys@transformshift{1.068333in}{1.527805in}%
\pgfsys@useobject{currentmarker}{}%
\end{pgfscope}%
\end{pgfscope}%
\begin{pgfscope}%
\pgfsetbuttcap%
\pgfsetroundjoin%
\definecolor{currentfill}{rgb}{0.000000,0.000000,0.000000}%
\pgfsetfillcolor{currentfill}%
\pgfsetlinewidth{0.602250pt}%
\definecolor{currentstroke}{rgb}{0.000000,0.000000,0.000000}%
\pgfsetstrokecolor{currentstroke}%
\pgfsetdash{}{0pt}%
\pgfsys@defobject{currentmarker}{\pgfqpoint{-0.027778in}{0.000000in}}{\pgfqpoint{0.000000in}{0.000000in}}{%
\pgfpathmoveto{\pgfqpoint{0.000000in}{0.000000in}}%
\pgfpathlineto{\pgfqpoint{-0.027778in}{0.000000in}}%
\pgfusepath{stroke,fill}%
}%
\begin{pgfscope}%
\pgfsys@transformshift{1.068333in}{1.617548in}%
\pgfsys@useobject{currentmarker}{}%
\end{pgfscope}%
\end{pgfscope}%
\begin{pgfscope}%
\pgfsetbuttcap%
\pgfsetroundjoin%
\definecolor{currentfill}{rgb}{0.000000,0.000000,0.000000}%
\pgfsetfillcolor{currentfill}%
\pgfsetlinewidth{0.602250pt}%
\definecolor{currentstroke}{rgb}{0.000000,0.000000,0.000000}%
\pgfsetstrokecolor{currentstroke}%
\pgfsetdash{}{0pt}%
\pgfsys@defobject{currentmarker}{\pgfqpoint{-0.027778in}{0.000000in}}{\pgfqpoint{0.000000in}{0.000000in}}{%
\pgfpathmoveto{\pgfqpoint{0.000000in}{0.000000in}}%
\pgfpathlineto{\pgfqpoint{-0.027778in}{0.000000in}}%
\pgfusepath{stroke,fill}%
}%
\begin{pgfscope}%
\pgfsys@transformshift{1.068333in}{1.707292in}%
\pgfsys@useobject{currentmarker}{}%
\end{pgfscope}%
\end{pgfscope}%
\begin{pgfscope}%
\pgfsetbuttcap%
\pgfsetroundjoin%
\definecolor{currentfill}{rgb}{0.000000,0.000000,0.000000}%
\pgfsetfillcolor{currentfill}%
\pgfsetlinewidth{0.602250pt}%
\definecolor{currentstroke}{rgb}{0.000000,0.000000,0.000000}%
\pgfsetstrokecolor{currentstroke}%
\pgfsetdash{}{0pt}%
\pgfsys@defobject{currentmarker}{\pgfqpoint{-0.027778in}{0.000000in}}{\pgfqpoint{0.000000in}{0.000000in}}{%
\pgfpathmoveto{\pgfqpoint{0.000000in}{0.000000in}}%
\pgfpathlineto{\pgfqpoint{-0.027778in}{0.000000in}}%
\pgfusepath{stroke,fill}%
}%
\begin{pgfscope}%
\pgfsys@transformshift{1.068333in}{1.797035in}%
\pgfsys@useobject{currentmarker}{}%
\end{pgfscope}%
\end{pgfscope}%
\begin{pgfscope}%
\pgfsetbuttcap%
\pgfsetroundjoin%
\definecolor{currentfill}{rgb}{0.000000,0.000000,0.000000}%
\pgfsetfillcolor{currentfill}%
\pgfsetlinewidth{0.602250pt}%
\definecolor{currentstroke}{rgb}{0.000000,0.000000,0.000000}%
\pgfsetstrokecolor{currentstroke}%
\pgfsetdash{}{0pt}%
\pgfsys@defobject{currentmarker}{\pgfqpoint{-0.027778in}{0.000000in}}{\pgfqpoint{0.000000in}{0.000000in}}{%
\pgfpathmoveto{\pgfqpoint{0.000000in}{0.000000in}}%
\pgfpathlineto{\pgfqpoint{-0.027778in}{0.000000in}}%
\pgfusepath{stroke,fill}%
}%
\begin{pgfscope}%
\pgfsys@transformshift{1.068333in}{1.886779in}%
\pgfsys@useobject{currentmarker}{}%
\end{pgfscope}%
\end{pgfscope}%
\begin{pgfscope}%
\pgfsetbuttcap%
\pgfsetroundjoin%
\definecolor{currentfill}{rgb}{0.000000,0.000000,0.000000}%
\pgfsetfillcolor{currentfill}%
\pgfsetlinewidth{0.602250pt}%
\definecolor{currentstroke}{rgb}{0.000000,0.000000,0.000000}%
\pgfsetstrokecolor{currentstroke}%
\pgfsetdash{}{0pt}%
\pgfsys@defobject{currentmarker}{\pgfqpoint{-0.027778in}{0.000000in}}{\pgfqpoint{0.000000in}{0.000000in}}{%
\pgfpathmoveto{\pgfqpoint{0.000000in}{0.000000in}}%
\pgfpathlineto{\pgfqpoint{-0.027778in}{0.000000in}}%
\pgfusepath{stroke,fill}%
}%
\begin{pgfscope}%
\pgfsys@transformshift{1.068333in}{1.976522in}%
\pgfsys@useobject{currentmarker}{}%
\end{pgfscope}%
\end{pgfscope}%
\begin{pgfscope}%
\pgfpathrectangle{\pgfqpoint{0.135000in}{0.145754in}}{\pgfqpoint{1.866666in}{1.866666in}} %
\pgfusepath{clip}%
\pgfsetbuttcap%
\pgfsetroundjoin%
\pgfsetlinewidth{1.505625pt}%
\definecolor{currentstroke}{rgb}{0.000000,0.000000,1.000000}%
\pgfsetstrokecolor{currentstroke}%
\pgfsetdash{}{0pt}%
\pgfpathmoveto{\pgfqpoint{0.929803in}{0.192415in}}%
\pgfpathlineto{\pgfqpoint{0.881618in}{0.201309in}}%
\pgfpathlineto{\pgfqpoint{0.833434in}{0.212959in}}%
\pgfpathlineto{\pgfqpoint{0.785249in}{0.227482in}}%
\pgfpathlineto{\pgfqpoint{0.745151in}{0.241882in}}%
\pgfpathlineto{\pgfqpoint{0.700927in}{0.260327in}}%
\pgfpathlineto{\pgfqpoint{0.663770in}{0.278020in}}%
\pgfpathlineto{\pgfqpoint{0.619249in}{0.302112in}}%
\pgfpathlineto{\pgfqpoint{0.592511in}{0.318198in}}%
\pgfpathlineto{\pgfqpoint{0.556373in}{0.342028in}}%
\pgfpathlineto{\pgfqpoint{0.520235in}{0.368494in}}%
\pgfpathlineto{\pgfqpoint{0.483390in}{0.398481in}}%
\pgfpathlineto{\pgfqpoint{0.447958in}{0.430632in}}%
\pgfpathlineto{\pgfqpoint{0.419878in}{0.458712in}}%
\pgfpathlineto{\pgfqpoint{0.387120in}{0.494850in}}%
\pgfpathlineto{\pgfqpoint{0.357740in}{0.530989in}}%
\pgfpathlineto{\pgfqpoint{0.331274in}{0.567127in}}%
\pgfpathlineto{\pgfqpoint{0.307444in}{0.603265in}}%
\pgfpathlineto{\pgfqpoint{0.286007in}{0.639404in}}%
\pgfpathlineto{\pgfqpoint{0.266749in}{0.675542in}}%
\pgfpathlineto{\pgfqpoint{0.249574in}{0.711680in}}%
\pgfpathlineto{\pgfqpoint{0.231128in}{0.755905in}}%
\pgfpathlineto{\pgfqpoint{0.216728in}{0.796003in}}%
\pgfpathlineto{\pgfqpoint{0.202205in}{0.844188in}}%
\pgfpathlineto{\pgfqpoint{0.190555in}{0.892372in}}%
\pgfpathlineto{\pgfqpoint{0.181662in}{0.940557in}}%
\pgfpathlineto{\pgfqpoint{0.175476in}{0.988741in}}%
\pgfpathlineto{\pgfqpoint{0.171894in}{1.036925in}}%
\pgfpathlineto{\pgfqpoint{0.170918in}{1.085110in}}%
\pgfpathlineto{\pgfqpoint{0.172546in}{1.133294in}}%
\pgfpathlineto{\pgfqpoint{0.176778in}{1.181479in}}%
\pgfpathlineto{\pgfqpoint{0.183624in}{1.229663in}}%
\pgfpathlineto{\pgfqpoint{0.193195in}{1.277848in}}%
\pgfpathlineto{\pgfqpoint{0.205551in}{1.326032in}}%
\pgfpathlineto{\pgfqpoint{0.219082in}{1.369135in}}%
\pgfpathlineto{\pgfqpoint{0.234292in}{1.410355in}}%
\pgfpathlineto{\pgfqpoint{0.249574in}{1.446493in}}%
\pgfpathlineto{\pgfqpoint{0.267266in}{1.483650in}}%
\pgfpathlineto{\pgfqpoint{0.291359in}{1.528171in}}%
\pgfpathlineto{\pgfqpoint{0.307444in}{1.554908in}}%
\pgfpathlineto{\pgfqpoint{0.331274in}{1.591047in}}%
\pgfpathlineto{\pgfqpoint{0.357740in}{1.627185in}}%
\pgfpathlineto{\pgfqpoint{0.387728in}{1.664030in}}%
\pgfpathlineto{\pgfqpoint{0.419878in}{1.699462in}}%
\pgfpathlineto{\pgfqpoint{0.447958in}{1.727542in}}%
\pgfpathlineto{\pgfqpoint{0.484096in}{1.760300in}}%
\pgfpathlineto{\pgfqpoint{0.520235in}{1.789680in}}%
\pgfpathlineto{\pgfqpoint{0.556373in}{1.816146in}}%
\pgfpathlineto{\pgfqpoint{0.592511in}{1.839976in}}%
\pgfpathlineto{\pgfqpoint{0.628650in}{1.861413in}}%
\pgfpathlineto{\pgfqpoint{0.664788in}{1.880670in}}%
\pgfpathlineto{\pgfqpoint{0.700927in}{1.897846in}}%
\pgfpathlineto{\pgfqpoint{0.745151in}{1.916292in}}%
\pgfpathlineto{\pgfqpoint{0.785249in}{1.930692in}}%
\pgfpathlineto{\pgfqpoint{0.833434in}{1.945215in}}%
\pgfpathlineto{\pgfqpoint{0.881618in}{1.956865in}}%
\pgfpathlineto{\pgfqpoint{0.929803in}{1.965758in}}%
\pgfpathlineto{\pgfqpoint{0.977987in}{1.971944in}}%
\pgfpathlineto{\pgfqpoint{1.026172in}{1.975525in}}%
\pgfpathlineto{\pgfqpoint{1.074356in}{1.976502in}}%
\pgfpathlineto{\pgfqpoint{1.122541in}{1.974874in}}%
\pgfpathlineto{\pgfqpoint{1.170725in}{1.970642in}}%
\pgfpathlineto{\pgfqpoint{1.218909in}{1.963796in}}%
\pgfpathlineto{\pgfqpoint{1.267094in}{1.954225in}}%
\pgfpathlineto{\pgfqpoint{1.315278in}{1.941869in}}%
\pgfpathlineto{\pgfqpoint{1.358381in}{1.928338in}}%
\pgfpathlineto{\pgfqpoint{1.399601in}{1.913128in}}%
\pgfpathlineto{\pgfqpoint{1.435740in}{1.897846in}}%
\pgfpathlineto{\pgfqpoint{1.472897in}{1.880154in}}%
\pgfpathlineto{\pgfqpoint{1.517417in}{1.856061in}}%
\pgfpathlineto{\pgfqpoint{1.544155in}{1.839976in}}%
\pgfpathlineto{\pgfqpoint{1.580293in}{1.816146in}}%
\pgfpathlineto{\pgfqpoint{1.616431in}{1.789680in}}%
\pgfpathlineto{\pgfqpoint{1.653276in}{1.759692in}}%
\pgfpathlineto{\pgfqpoint{1.688708in}{1.727542in}}%
\pgfpathlineto{\pgfqpoint{1.716788in}{1.699462in}}%
\pgfpathlineto{\pgfqpoint{1.749546in}{1.663323in}}%
\pgfpathlineto{\pgfqpoint{1.778926in}{1.627185in}}%
\pgfpathlineto{\pgfqpoint{1.805392in}{1.591047in}}%
\pgfpathlineto{\pgfqpoint{1.829222in}{1.554908in}}%
\pgfpathlineto{\pgfqpoint{1.850659in}{1.518770in}}%
\pgfpathlineto{\pgfqpoint{1.869917in}{1.482632in}}%
\pgfpathlineto{\pgfqpoint{1.887092in}{1.446493in}}%
\pgfpathlineto{\pgfqpoint{1.905538in}{1.402269in}}%
\pgfpathlineto{\pgfqpoint{1.919938in}{1.362171in}}%
\pgfpathlineto{\pgfqpoint{1.934461in}{1.313986in}}%
\pgfpathlineto{\pgfqpoint{1.946111in}{1.265802in}}%
\pgfpathlineto{\pgfqpoint{1.955004in}{1.217617in}}%
\pgfpathlineto{\pgfqpoint{1.961190in}{1.169433in}}%
\pgfpathlineto{\pgfqpoint{1.964772in}{1.121248in}}%
\pgfpathlineto{\pgfqpoint{1.965748in}{1.073064in}}%
\pgfpathlineto{\pgfqpoint{1.964120in}{1.024879in}}%
\pgfpathlineto{\pgfqpoint{1.959888in}{0.976695in}}%
\pgfpathlineto{\pgfqpoint{1.953042in}{0.928510in}}%
\pgfpathlineto{\pgfqpoint{1.943471in}{0.880326in}}%
\pgfpathlineto{\pgfqpoint{1.931115in}{0.832141in}}%
\pgfpathlineto{\pgfqpoint{1.917584in}{0.789039in}}%
\pgfpathlineto{\pgfqpoint{1.902374in}{0.747819in}}%
\pgfpathlineto{\pgfqpoint{1.887092in}{0.711680in}}%
\pgfpathlineto{\pgfqpoint{1.869400in}{0.674523in}}%
\pgfpathlineto{\pgfqpoint{1.845307in}{0.630003in}}%
\pgfpathlineto{\pgfqpoint{1.829222in}{0.603265in}}%
\pgfpathlineto{\pgfqpoint{1.805392in}{0.567127in}}%
\pgfpathlineto{\pgfqpoint{1.778926in}{0.530989in}}%
\pgfpathlineto{\pgfqpoint{1.748939in}{0.494143in}}%
\pgfpathlineto{\pgfqpoint{1.716788in}{0.458712in}}%
\pgfpathlineto{\pgfqpoint{1.688708in}{0.430632in}}%
\pgfpathlineto{\pgfqpoint{1.652570in}{0.397874in}}%
\pgfpathlineto{\pgfqpoint{1.616431in}{0.368494in}}%
\pgfpathlineto{\pgfqpoint{1.580293in}{0.342028in}}%
\pgfpathlineto{\pgfqpoint{1.544155in}{0.318198in}}%
\pgfpathlineto{\pgfqpoint{1.508016in}{0.296761in}}%
\pgfpathlineto{\pgfqpoint{1.471878in}{0.277503in}}%
\pgfpathlineto{\pgfqpoint{1.435740in}{0.260327in}}%
\pgfpathlineto{\pgfqpoint{1.391515in}{0.241882in}}%
\pgfpathlineto{\pgfqpoint{1.351417in}{0.227482in}}%
\pgfpathlineto{\pgfqpoint{1.303232in}{0.212959in}}%
\pgfpathlineto{\pgfqpoint{1.255048in}{0.201309in}}%
\pgfpathlineto{\pgfqpoint{1.206863in}{0.192415in}}%
\pgfpathlineto{\pgfqpoint{1.158679in}{0.186230in}}%
\pgfpathlineto{\pgfqpoint{1.110494in}{0.182648in}}%
\pgfpathlineto{\pgfqpoint{1.062310in}{0.181672in}}%
\pgfpathlineto{\pgfqpoint{1.014126in}{0.183299in}}%
\pgfpathlineto{\pgfqpoint{0.965941in}{0.187532in}}%
\pgfpathlineto{\pgfqpoint{0.929803in}{0.192415in}}%
\pgfpathlineto{\pgfqpoint{0.929803in}{0.192415in}}%
\pgfusepath{stroke}%
\end{pgfscope}%
\begin{pgfscope}%
\pgfpathrectangle{\pgfqpoint{0.135000in}{0.145754in}}{\pgfqpoint{1.866666in}{1.866666in}} %
\pgfusepath{clip}%
\pgfsetbuttcap%
\pgfsetroundjoin%
\pgfsetlinewidth{1.505625pt}%
\definecolor{currentstroke}{rgb}{1.000000,1.000000,1.000000}%
\pgfsetstrokecolor{currentstroke}%
\pgfsetdash{}{0pt}%
\pgfpathmoveto{\pgfqpoint{0.929803in}{0.192415in}}%
\pgfpathlineto{\pgfqpoint{0.881618in}{0.201309in}}%
\pgfpathlineto{\pgfqpoint{0.833434in}{0.212959in}}%
\pgfpathlineto{\pgfqpoint{0.785249in}{0.227482in}}%
\pgfpathlineto{\pgfqpoint{0.745151in}{0.241882in}}%
\pgfpathlineto{\pgfqpoint{0.700927in}{0.260327in}}%
\pgfpathlineto{\pgfqpoint{0.663770in}{0.278020in}}%
\pgfpathlineto{\pgfqpoint{0.619249in}{0.302112in}}%
\pgfpathlineto{\pgfqpoint{0.592511in}{0.318198in}}%
\pgfpathlineto{\pgfqpoint{0.556373in}{0.342028in}}%
\pgfpathlineto{\pgfqpoint{0.520235in}{0.368494in}}%
\pgfpathlineto{\pgfqpoint{0.483390in}{0.398481in}}%
\pgfpathlineto{\pgfqpoint{0.447958in}{0.430632in}}%
\pgfpathlineto{\pgfqpoint{0.419878in}{0.458712in}}%
\pgfpathlineto{\pgfqpoint{0.387120in}{0.494850in}}%
\pgfpathlineto{\pgfqpoint{0.357740in}{0.530989in}}%
\pgfpathlineto{\pgfqpoint{0.331274in}{0.567127in}}%
\pgfpathlineto{\pgfqpoint{0.307444in}{0.603265in}}%
\pgfpathlineto{\pgfqpoint{0.286007in}{0.639404in}}%
\pgfpathlineto{\pgfqpoint{0.266749in}{0.675542in}}%
\pgfpathlineto{\pgfqpoint{0.249574in}{0.711680in}}%
\pgfpathlineto{\pgfqpoint{0.231128in}{0.755905in}}%
\pgfpathlineto{\pgfqpoint{0.216728in}{0.796003in}}%
\pgfpathlineto{\pgfqpoint{0.202205in}{0.844188in}}%
\pgfpathlineto{\pgfqpoint{0.190555in}{0.892372in}}%
\pgfpathlineto{\pgfqpoint{0.181662in}{0.940557in}}%
\pgfpathlineto{\pgfqpoint{0.175476in}{0.988741in}}%
\pgfpathlineto{\pgfqpoint{0.171894in}{1.036925in}}%
\pgfpathlineto{\pgfqpoint{0.170918in}{1.085110in}}%
\pgfpathlineto{\pgfqpoint{0.172546in}{1.133294in}}%
\pgfpathlineto{\pgfqpoint{0.176778in}{1.181479in}}%
\pgfpathlineto{\pgfqpoint{0.183624in}{1.229663in}}%
\pgfpathlineto{\pgfqpoint{0.193195in}{1.277848in}}%
\pgfpathlineto{\pgfqpoint{0.205551in}{1.326032in}}%
\pgfpathlineto{\pgfqpoint{0.219082in}{1.369135in}}%
\pgfpathlineto{\pgfqpoint{0.234292in}{1.410355in}}%
\pgfpathlineto{\pgfqpoint{0.249574in}{1.446493in}}%
\pgfpathlineto{\pgfqpoint{0.267266in}{1.483650in}}%
\pgfpathlineto{\pgfqpoint{0.291359in}{1.528171in}}%
\pgfpathlineto{\pgfqpoint{0.307444in}{1.554908in}}%
\pgfpathlineto{\pgfqpoint{0.331274in}{1.591047in}}%
\pgfpathlineto{\pgfqpoint{0.357740in}{1.627185in}}%
\pgfpathlineto{\pgfqpoint{0.387728in}{1.664030in}}%
\pgfpathlineto{\pgfqpoint{0.419878in}{1.699462in}}%
\pgfpathlineto{\pgfqpoint{0.447958in}{1.727542in}}%
\pgfpathlineto{\pgfqpoint{0.484096in}{1.760300in}}%
\pgfpathlineto{\pgfqpoint{0.520235in}{1.789680in}}%
\pgfpathlineto{\pgfqpoint{0.556373in}{1.816146in}}%
\pgfpathlineto{\pgfqpoint{0.592511in}{1.839976in}}%
\pgfpathlineto{\pgfqpoint{0.628650in}{1.861413in}}%
\pgfpathlineto{\pgfqpoint{0.664788in}{1.880670in}}%
\pgfpathlineto{\pgfqpoint{0.700927in}{1.897846in}}%
\pgfpathlineto{\pgfqpoint{0.745151in}{1.916292in}}%
\pgfpathlineto{\pgfqpoint{0.785249in}{1.930692in}}%
\pgfpathlineto{\pgfqpoint{0.833434in}{1.945215in}}%
\pgfpathlineto{\pgfqpoint{0.881618in}{1.956865in}}%
\pgfpathlineto{\pgfqpoint{0.929803in}{1.965758in}}%
\pgfpathlineto{\pgfqpoint{0.977987in}{1.971944in}}%
\pgfpathlineto{\pgfqpoint{1.026172in}{1.975525in}}%
\pgfpathlineto{\pgfqpoint{1.074356in}{1.976502in}}%
\pgfpathlineto{\pgfqpoint{1.122541in}{1.974874in}}%
\pgfpathlineto{\pgfqpoint{1.170725in}{1.970642in}}%
\pgfpathlineto{\pgfqpoint{1.218909in}{1.963796in}}%
\pgfpathlineto{\pgfqpoint{1.267094in}{1.954225in}}%
\pgfpathlineto{\pgfqpoint{1.315278in}{1.941869in}}%
\pgfpathlineto{\pgfqpoint{1.358381in}{1.928338in}}%
\pgfpathlineto{\pgfqpoint{1.399601in}{1.913128in}}%
\pgfpathlineto{\pgfqpoint{1.435740in}{1.897846in}}%
\pgfpathlineto{\pgfqpoint{1.472897in}{1.880154in}}%
\pgfpathlineto{\pgfqpoint{1.517417in}{1.856061in}}%
\pgfpathlineto{\pgfqpoint{1.544155in}{1.839976in}}%
\pgfpathlineto{\pgfqpoint{1.580293in}{1.816146in}}%
\pgfpathlineto{\pgfqpoint{1.616431in}{1.789680in}}%
\pgfpathlineto{\pgfqpoint{1.653276in}{1.759692in}}%
\pgfpathlineto{\pgfqpoint{1.688708in}{1.727542in}}%
\pgfpathlineto{\pgfqpoint{1.716788in}{1.699462in}}%
\pgfpathlineto{\pgfqpoint{1.749546in}{1.663323in}}%
\pgfpathlineto{\pgfqpoint{1.778926in}{1.627185in}}%
\pgfpathlineto{\pgfqpoint{1.805392in}{1.591047in}}%
\pgfpathlineto{\pgfqpoint{1.829222in}{1.554908in}}%
\pgfpathlineto{\pgfqpoint{1.850659in}{1.518770in}}%
\pgfpathlineto{\pgfqpoint{1.869917in}{1.482632in}}%
\pgfpathlineto{\pgfqpoint{1.887092in}{1.446493in}}%
\pgfpathlineto{\pgfqpoint{1.905538in}{1.402269in}}%
\pgfpathlineto{\pgfqpoint{1.919938in}{1.362171in}}%
\pgfpathlineto{\pgfqpoint{1.934461in}{1.313986in}}%
\pgfpathlineto{\pgfqpoint{1.946111in}{1.265802in}}%
\pgfpathlineto{\pgfqpoint{1.955004in}{1.217617in}}%
\pgfpathlineto{\pgfqpoint{1.961190in}{1.169433in}}%
\pgfpathlineto{\pgfqpoint{1.964772in}{1.121248in}}%
\pgfpathlineto{\pgfqpoint{1.965748in}{1.073064in}}%
\pgfpathlineto{\pgfqpoint{1.964120in}{1.024879in}}%
\pgfpathlineto{\pgfqpoint{1.959888in}{0.976695in}}%
\pgfpathlineto{\pgfqpoint{1.953042in}{0.928510in}}%
\pgfpathlineto{\pgfqpoint{1.943471in}{0.880326in}}%
\pgfpathlineto{\pgfqpoint{1.931115in}{0.832141in}}%
\pgfpathlineto{\pgfqpoint{1.917584in}{0.789039in}}%
\pgfpathlineto{\pgfqpoint{1.902374in}{0.747819in}}%
\pgfpathlineto{\pgfqpoint{1.887092in}{0.711680in}}%
\pgfpathlineto{\pgfqpoint{1.869400in}{0.674523in}}%
\pgfpathlineto{\pgfqpoint{1.845307in}{0.630003in}}%
\pgfpathlineto{\pgfqpoint{1.829222in}{0.603265in}}%
\pgfpathlineto{\pgfqpoint{1.805392in}{0.567127in}}%
\pgfpathlineto{\pgfqpoint{1.778926in}{0.530989in}}%
\pgfpathlineto{\pgfqpoint{1.748939in}{0.494143in}}%
\pgfpathlineto{\pgfqpoint{1.716788in}{0.458712in}}%
\pgfpathlineto{\pgfqpoint{1.688708in}{0.430632in}}%
\pgfpathlineto{\pgfqpoint{1.652570in}{0.397874in}}%
\pgfpathlineto{\pgfqpoint{1.616431in}{0.368494in}}%
\pgfpathlineto{\pgfqpoint{1.580293in}{0.342028in}}%
\pgfpathlineto{\pgfqpoint{1.544155in}{0.318198in}}%
\pgfpathlineto{\pgfqpoint{1.508016in}{0.296761in}}%
\pgfpathlineto{\pgfqpoint{1.471878in}{0.277503in}}%
\pgfpathlineto{\pgfqpoint{1.435740in}{0.260327in}}%
\pgfpathlineto{\pgfqpoint{1.391515in}{0.241882in}}%
\pgfpathlineto{\pgfqpoint{1.351417in}{0.227482in}}%
\pgfpathlineto{\pgfqpoint{1.303232in}{0.212959in}}%
\pgfpathlineto{\pgfqpoint{1.255048in}{0.201309in}}%
\pgfpathlineto{\pgfqpoint{1.206863in}{0.192415in}}%
\pgfpathlineto{\pgfqpoint{1.158679in}{0.186230in}}%
\pgfpathlineto{\pgfqpoint{1.110494in}{0.182648in}}%
\pgfpathlineto{\pgfqpoint{1.062310in}{0.181672in}}%
\pgfpathlineto{\pgfqpoint{1.014126in}{0.183299in}}%
\pgfpathlineto{\pgfqpoint{0.965941in}{0.187532in}}%
\pgfpathlineto{\pgfqpoint{0.929803in}{0.192415in}}%
\pgfpathlineto{\pgfqpoint{0.929803in}{0.192415in}}%
\pgfusepath{stroke}%
\end{pgfscope}%
\begin{pgfscope}%
\pgfpathrectangle{\pgfqpoint{0.135000in}{0.145754in}}{\pgfqpoint{1.866666in}{1.866666in}} %
\pgfusepath{clip}%
\pgfsetbuttcap%
\pgfsetroundjoin%
\pgfsetlinewidth{1.505625pt}%
\definecolor{currentstroke}{rgb}{1.000000,1.000000,1.000000}%
\pgfsetstrokecolor{currentstroke}%
\pgfsetdash{}{0pt}%
\pgfpathmoveto{\pgfqpoint{0.929803in}{0.192415in}}%
\pgfpathlineto{\pgfqpoint{0.881618in}{0.201309in}}%
\pgfpathlineto{\pgfqpoint{0.833434in}{0.212959in}}%
\pgfpathlineto{\pgfqpoint{0.785249in}{0.227482in}}%
\pgfpathlineto{\pgfqpoint{0.745151in}{0.241882in}}%
\pgfpathlineto{\pgfqpoint{0.700927in}{0.260327in}}%
\pgfpathlineto{\pgfqpoint{0.663770in}{0.278020in}}%
\pgfpathlineto{\pgfqpoint{0.619249in}{0.302112in}}%
\pgfpathlineto{\pgfqpoint{0.592511in}{0.318198in}}%
\pgfpathlineto{\pgfqpoint{0.556373in}{0.342028in}}%
\pgfpathlineto{\pgfqpoint{0.520235in}{0.368494in}}%
\pgfpathlineto{\pgfqpoint{0.483390in}{0.398481in}}%
\pgfpathlineto{\pgfqpoint{0.447958in}{0.430632in}}%
\pgfpathlineto{\pgfqpoint{0.419878in}{0.458712in}}%
\pgfpathlineto{\pgfqpoint{0.387120in}{0.494850in}}%
\pgfpathlineto{\pgfqpoint{0.357740in}{0.530989in}}%
\pgfpathlineto{\pgfqpoint{0.331274in}{0.567127in}}%
\pgfpathlineto{\pgfqpoint{0.307444in}{0.603265in}}%
\pgfpathlineto{\pgfqpoint{0.286007in}{0.639404in}}%
\pgfpathlineto{\pgfqpoint{0.266749in}{0.675542in}}%
\pgfpathlineto{\pgfqpoint{0.249574in}{0.711680in}}%
\pgfpathlineto{\pgfqpoint{0.231128in}{0.755905in}}%
\pgfpathlineto{\pgfqpoint{0.216728in}{0.796003in}}%
\pgfpathlineto{\pgfqpoint{0.202205in}{0.844188in}}%
\pgfpathlineto{\pgfqpoint{0.190555in}{0.892372in}}%
\pgfpathlineto{\pgfqpoint{0.181662in}{0.940557in}}%
\pgfpathlineto{\pgfqpoint{0.175476in}{0.988741in}}%
\pgfpathlineto{\pgfqpoint{0.171894in}{1.036925in}}%
\pgfpathlineto{\pgfqpoint{0.170918in}{1.085110in}}%
\pgfpathlineto{\pgfqpoint{0.172546in}{1.133294in}}%
\pgfpathlineto{\pgfqpoint{0.176778in}{1.181479in}}%
\pgfpathlineto{\pgfqpoint{0.183624in}{1.229663in}}%
\pgfpathlineto{\pgfqpoint{0.193195in}{1.277848in}}%
\pgfpathlineto{\pgfqpoint{0.205551in}{1.326032in}}%
\pgfpathlineto{\pgfqpoint{0.219082in}{1.369135in}}%
\pgfpathlineto{\pgfqpoint{0.234292in}{1.410355in}}%
\pgfpathlineto{\pgfqpoint{0.249574in}{1.446493in}}%
\pgfpathlineto{\pgfqpoint{0.267266in}{1.483650in}}%
\pgfpathlineto{\pgfqpoint{0.291359in}{1.528171in}}%
\pgfpathlineto{\pgfqpoint{0.307444in}{1.554908in}}%
\pgfpathlineto{\pgfqpoint{0.331274in}{1.591047in}}%
\pgfpathlineto{\pgfqpoint{0.357740in}{1.627185in}}%
\pgfpathlineto{\pgfqpoint{0.387728in}{1.664030in}}%
\pgfpathlineto{\pgfqpoint{0.419878in}{1.699462in}}%
\pgfpathlineto{\pgfqpoint{0.447958in}{1.727542in}}%
\pgfpathlineto{\pgfqpoint{0.484096in}{1.760300in}}%
\pgfpathlineto{\pgfqpoint{0.520235in}{1.789680in}}%
\pgfpathlineto{\pgfqpoint{0.556373in}{1.816146in}}%
\pgfpathlineto{\pgfqpoint{0.592511in}{1.839976in}}%
\pgfpathlineto{\pgfqpoint{0.628650in}{1.861413in}}%
\pgfpathlineto{\pgfqpoint{0.664788in}{1.880670in}}%
\pgfpathlineto{\pgfqpoint{0.700927in}{1.897846in}}%
\pgfpathlineto{\pgfqpoint{0.745151in}{1.916292in}}%
\pgfpathlineto{\pgfqpoint{0.785249in}{1.930692in}}%
\pgfpathlineto{\pgfqpoint{0.833434in}{1.945215in}}%
\pgfpathlineto{\pgfqpoint{0.881618in}{1.956865in}}%
\pgfpathlineto{\pgfqpoint{0.929803in}{1.965758in}}%
\pgfpathlineto{\pgfqpoint{0.977987in}{1.971944in}}%
\pgfpathlineto{\pgfqpoint{1.026172in}{1.975525in}}%
\pgfpathlineto{\pgfqpoint{1.074356in}{1.976502in}}%
\pgfpathlineto{\pgfqpoint{1.122541in}{1.974874in}}%
\pgfpathlineto{\pgfqpoint{1.170725in}{1.970642in}}%
\pgfpathlineto{\pgfqpoint{1.218909in}{1.963796in}}%
\pgfpathlineto{\pgfqpoint{1.267094in}{1.954225in}}%
\pgfpathlineto{\pgfqpoint{1.315278in}{1.941869in}}%
\pgfpathlineto{\pgfqpoint{1.358381in}{1.928338in}}%
\pgfpathlineto{\pgfqpoint{1.399601in}{1.913128in}}%
\pgfpathlineto{\pgfqpoint{1.435740in}{1.897846in}}%
\pgfpathlineto{\pgfqpoint{1.472897in}{1.880154in}}%
\pgfpathlineto{\pgfqpoint{1.517417in}{1.856061in}}%
\pgfpathlineto{\pgfqpoint{1.544155in}{1.839976in}}%
\pgfpathlineto{\pgfqpoint{1.580293in}{1.816146in}}%
\pgfpathlineto{\pgfqpoint{1.616431in}{1.789680in}}%
\pgfpathlineto{\pgfqpoint{1.653276in}{1.759692in}}%
\pgfpathlineto{\pgfqpoint{1.688708in}{1.727542in}}%
\pgfpathlineto{\pgfqpoint{1.716788in}{1.699462in}}%
\pgfpathlineto{\pgfqpoint{1.749546in}{1.663323in}}%
\pgfpathlineto{\pgfqpoint{1.778926in}{1.627185in}}%
\pgfpathlineto{\pgfqpoint{1.805392in}{1.591047in}}%
\pgfpathlineto{\pgfqpoint{1.829222in}{1.554908in}}%
\pgfpathlineto{\pgfqpoint{1.850659in}{1.518770in}}%
\pgfpathlineto{\pgfqpoint{1.869917in}{1.482632in}}%
\pgfpathlineto{\pgfqpoint{1.887092in}{1.446493in}}%
\pgfpathlineto{\pgfqpoint{1.905538in}{1.402269in}}%
\pgfpathlineto{\pgfqpoint{1.919938in}{1.362171in}}%
\pgfpathlineto{\pgfqpoint{1.934461in}{1.313986in}}%
\pgfpathlineto{\pgfqpoint{1.946111in}{1.265802in}}%
\pgfpathlineto{\pgfqpoint{1.955004in}{1.217617in}}%
\pgfpathlineto{\pgfqpoint{1.961190in}{1.169433in}}%
\pgfpathlineto{\pgfqpoint{1.964772in}{1.121248in}}%
\pgfpathlineto{\pgfqpoint{1.965748in}{1.073064in}}%
\pgfpathlineto{\pgfqpoint{1.964120in}{1.024879in}}%
\pgfpathlineto{\pgfqpoint{1.959888in}{0.976695in}}%
\pgfpathlineto{\pgfqpoint{1.953042in}{0.928510in}}%
\pgfpathlineto{\pgfqpoint{1.943471in}{0.880326in}}%
\pgfpathlineto{\pgfqpoint{1.931115in}{0.832141in}}%
\pgfpathlineto{\pgfqpoint{1.917584in}{0.789039in}}%
\pgfpathlineto{\pgfqpoint{1.902374in}{0.747819in}}%
\pgfpathlineto{\pgfqpoint{1.887092in}{0.711680in}}%
\pgfpathlineto{\pgfqpoint{1.869400in}{0.674523in}}%
\pgfpathlineto{\pgfqpoint{1.845307in}{0.630003in}}%
\pgfpathlineto{\pgfqpoint{1.829222in}{0.603265in}}%
\pgfpathlineto{\pgfqpoint{1.805392in}{0.567127in}}%
\pgfpathlineto{\pgfqpoint{1.778926in}{0.530989in}}%
\pgfpathlineto{\pgfqpoint{1.748939in}{0.494143in}}%
\pgfpathlineto{\pgfqpoint{1.716788in}{0.458712in}}%
\pgfpathlineto{\pgfqpoint{1.688708in}{0.430632in}}%
\pgfpathlineto{\pgfqpoint{1.652570in}{0.397874in}}%
\pgfpathlineto{\pgfqpoint{1.616431in}{0.368494in}}%
\pgfpathlineto{\pgfqpoint{1.580293in}{0.342028in}}%
\pgfpathlineto{\pgfqpoint{1.544155in}{0.318198in}}%
\pgfpathlineto{\pgfqpoint{1.508016in}{0.296761in}}%
\pgfpathlineto{\pgfqpoint{1.471878in}{0.277503in}}%
\pgfpathlineto{\pgfqpoint{1.435740in}{0.260327in}}%
\pgfpathlineto{\pgfqpoint{1.391515in}{0.241882in}}%
\pgfpathlineto{\pgfqpoint{1.351417in}{0.227482in}}%
\pgfpathlineto{\pgfqpoint{1.303232in}{0.212959in}}%
\pgfpathlineto{\pgfqpoint{1.255048in}{0.201309in}}%
\pgfpathlineto{\pgfqpoint{1.206863in}{0.192415in}}%
\pgfpathlineto{\pgfqpoint{1.158679in}{0.186230in}}%
\pgfpathlineto{\pgfqpoint{1.110494in}{0.182648in}}%
\pgfpathlineto{\pgfqpoint{1.062310in}{0.181672in}}%
\pgfpathlineto{\pgfqpoint{1.014126in}{0.183299in}}%
\pgfpathlineto{\pgfqpoint{0.965941in}{0.187532in}}%
\pgfpathlineto{\pgfqpoint{0.929803in}{0.192415in}}%
\pgfpathlineto{\pgfqpoint{0.929803in}{0.192415in}}%
\pgfusepath{stroke}%
\end{pgfscope}%
\begin{pgfscope}%
\pgfpathrectangle{\pgfqpoint{0.135000in}{0.145754in}}{\pgfqpoint{1.866666in}{1.866666in}} %
\pgfusepath{clip}%
\pgfsetbuttcap%
\pgfsetroundjoin%
\pgfsetlinewidth{1.505625pt}%
\definecolor{currentstroke}{rgb}{0.000000,0.000000,1.000000}%
\pgfsetstrokecolor{currentstroke}%
\pgfsetdash{}{0pt}%
\pgfpathmoveto{\pgfqpoint{0.929803in}{0.192415in}}%
\pgfpathlineto{\pgfqpoint{0.881618in}{0.201309in}}%
\pgfpathlineto{\pgfqpoint{0.833434in}{0.212959in}}%
\pgfpathlineto{\pgfqpoint{0.785249in}{0.227482in}}%
\pgfpathlineto{\pgfqpoint{0.745151in}{0.241882in}}%
\pgfpathlineto{\pgfqpoint{0.700927in}{0.260327in}}%
\pgfpathlineto{\pgfqpoint{0.663770in}{0.278020in}}%
\pgfpathlineto{\pgfqpoint{0.619249in}{0.302112in}}%
\pgfpathlineto{\pgfqpoint{0.592511in}{0.318198in}}%
\pgfpathlineto{\pgfqpoint{0.556373in}{0.342028in}}%
\pgfpathlineto{\pgfqpoint{0.520235in}{0.368494in}}%
\pgfpathlineto{\pgfqpoint{0.483390in}{0.398481in}}%
\pgfpathlineto{\pgfqpoint{0.447958in}{0.430632in}}%
\pgfpathlineto{\pgfqpoint{0.419878in}{0.458712in}}%
\pgfpathlineto{\pgfqpoint{0.387120in}{0.494850in}}%
\pgfpathlineto{\pgfqpoint{0.357740in}{0.530989in}}%
\pgfpathlineto{\pgfqpoint{0.331274in}{0.567127in}}%
\pgfpathlineto{\pgfqpoint{0.307444in}{0.603265in}}%
\pgfpathlineto{\pgfqpoint{0.286007in}{0.639404in}}%
\pgfpathlineto{\pgfqpoint{0.266749in}{0.675542in}}%
\pgfpathlineto{\pgfqpoint{0.249574in}{0.711680in}}%
\pgfpathlineto{\pgfqpoint{0.231128in}{0.755905in}}%
\pgfpathlineto{\pgfqpoint{0.216728in}{0.796003in}}%
\pgfpathlineto{\pgfqpoint{0.202205in}{0.844188in}}%
\pgfpathlineto{\pgfqpoint{0.190555in}{0.892372in}}%
\pgfpathlineto{\pgfqpoint{0.181662in}{0.940557in}}%
\pgfpathlineto{\pgfqpoint{0.175476in}{0.988741in}}%
\pgfpathlineto{\pgfqpoint{0.171894in}{1.036925in}}%
\pgfpathlineto{\pgfqpoint{0.170918in}{1.085110in}}%
\pgfpathlineto{\pgfqpoint{0.172546in}{1.133294in}}%
\pgfpathlineto{\pgfqpoint{0.176778in}{1.181479in}}%
\pgfpathlineto{\pgfqpoint{0.183624in}{1.229663in}}%
\pgfpathlineto{\pgfqpoint{0.193195in}{1.277848in}}%
\pgfpathlineto{\pgfqpoint{0.205551in}{1.326032in}}%
\pgfpathlineto{\pgfqpoint{0.219082in}{1.369135in}}%
\pgfpathlineto{\pgfqpoint{0.234292in}{1.410355in}}%
\pgfpathlineto{\pgfqpoint{0.249574in}{1.446493in}}%
\pgfpathlineto{\pgfqpoint{0.267266in}{1.483650in}}%
\pgfpathlineto{\pgfqpoint{0.291359in}{1.528171in}}%
\pgfpathlineto{\pgfqpoint{0.307444in}{1.554908in}}%
\pgfpathlineto{\pgfqpoint{0.331274in}{1.591047in}}%
\pgfpathlineto{\pgfqpoint{0.357740in}{1.627185in}}%
\pgfpathlineto{\pgfqpoint{0.387728in}{1.664030in}}%
\pgfpathlineto{\pgfqpoint{0.419878in}{1.699462in}}%
\pgfpathlineto{\pgfqpoint{0.447958in}{1.727542in}}%
\pgfpathlineto{\pgfqpoint{0.484096in}{1.760300in}}%
\pgfpathlineto{\pgfqpoint{0.520235in}{1.789680in}}%
\pgfpathlineto{\pgfqpoint{0.556373in}{1.816146in}}%
\pgfpathlineto{\pgfqpoint{0.592511in}{1.839976in}}%
\pgfpathlineto{\pgfqpoint{0.628650in}{1.861413in}}%
\pgfpathlineto{\pgfqpoint{0.664788in}{1.880670in}}%
\pgfpathlineto{\pgfqpoint{0.700927in}{1.897846in}}%
\pgfpathlineto{\pgfqpoint{0.745151in}{1.916292in}}%
\pgfpathlineto{\pgfqpoint{0.785249in}{1.930692in}}%
\pgfpathlineto{\pgfqpoint{0.833434in}{1.945215in}}%
\pgfpathlineto{\pgfqpoint{0.881618in}{1.956865in}}%
\pgfpathlineto{\pgfqpoint{0.929803in}{1.965758in}}%
\pgfpathlineto{\pgfqpoint{0.977987in}{1.971944in}}%
\pgfpathlineto{\pgfqpoint{1.026172in}{1.975525in}}%
\pgfpathlineto{\pgfqpoint{1.074356in}{1.976502in}}%
\pgfpathlineto{\pgfqpoint{1.122541in}{1.974874in}}%
\pgfpathlineto{\pgfqpoint{1.170725in}{1.970642in}}%
\pgfpathlineto{\pgfqpoint{1.218909in}{1.963796in}}%
\pgfpathlineto{\pgfqpoint{1.267094in}{1.954225in}}%
\pgfpathlineto{\pgfqpoint{1.315278in}{1.941869in}}%
\pgfpathlineto{\pgfqpoint{1.358381in}{1.928338in}}%
\pgfpathlineto{\pgfqpoint{1.399601in}{1.913128in}}%
\pgfpathlineto{\pgfqpoint{1.435740in}{1.897846in}}%
\pgfpathlineto{\pgfqpoint{1.472897in}{1.880154in}}%
\pgfpathlineto{\pgfqpoint{1.517417in}{1.856061in}}%
\pgfpathlineto{\pgfqpoint{1.544155in}{1.839976in}}%
\pgfpathlineto{\pgfqpoint{1.580293in}{1.816146in}}%
\pgfpathlineto{\pgfqpoint{1.616431in}{1.789680in}}%
\pgfpathlineto{\pgfqpoint{1.653276in}{1.759692in}}%
\pgfpathlineto{\pgfqpoint{1.688708in}{1.727542in}}%
\pgfpathlineto{\pgfqpoint{1.716788in}{1.699462in}}%
\pgfpathlineto{\pgfqpoint{1.749546in}{1.663323in}}%
\pgfpathlineto{\pgfqpoint{1.778926in}{1.627185in}}%
\pgfpathlineto{\pgfqpoint{1.805392in}{1.591047in}}%
\pgfpathlineto{\pgfqpoint{1.829222in}{1.554908in}}%
\pgfpathlineto{\pgfqpoint{1.850659in}{1.518770in}}%
\pgfpathlineto{\pgfqpoint{1.869917in}{1.482632in}}%
\pgfpathlineto{\pgfqpoint{1.887092in}{1.446493in}}%
\pgfpathlineto{\pgfqpoint{1.905538in}{1.402269in}}%
\pgfpathlineto{\pgfqpoint{1.919938in}{1.362171in}}%
\pgfpathlineto{\pgfqpoint{1.934461in}{1.313986in}}%
\pgfpathlineto{\pgfqpoint{1.946111in}{1.265802in}}%
\pgfpathlineto{\pgfqpoint{1.955004in}{1.217617in}}%
\pgfpathlineto{\pgfqpoint{1.961190in}{1.169433in}}%
\pgfpathlineto{\pgfqpoint{1.964772in}{1.121248in}}%
\pgfpathlineto{\pgfqpoint{1.965748in}{1.073064in}}%
\pgfpathlineto{\pgfqpoint{1.964120in}{1.024879in}}%
\pgfpathlineto{\pgfqpoint{1.959888in}{0.976695in}}%
\pgfpathlineto{\pgfqpoint{1.953042in}{0.928510in}}%
\pgfpathlineto{\pgfqpoint{1.943471in}{0.880326in}}%
\pgfpathlineto{\pgfqpoint{1.931115in}{0.832141in}}%
\pgfpathlineto{\pgfqpoint{1.917584in}{0.789039in}}%
\pgfpathlineto{\pgfqpoint{1.902374in}{0.747819in}}%
\pgfpathlineto{\pgfqpoint{1.887092in}{0.711680in}}%
\pgfpathlineto{\pgfqpoint{1.869400in}{0.674523in}}%
\pgfpathlineto{\pgfqpoint{1.845307in}{0.630003in}}%
\pgfpathlineto{\pgfqpoint{1.829222in}{0.603265in}}%
\pgfpathlineto{\pgfqpoint{1.805392in}{0.567127in}}%
\pgfpathlineto{\pgfqpoint{1.778926in}{0.530989in}}%
\pgfpathlineto{\pgfqpoint{1.748939in}{0.494143in}}%
\pgfpathlineto{\pgfqpoint{1.716788in}{0.458712in}}%
\pgfpathlineto{\pgfqpoint{1.688708in}{0.430632in}}%
\pgfpathlineto{\pgfqpoint{1.652570in}{0.397874in}}%
\pgfpathlineto{\pgfqpoint{1.616431in}{0.368494in}}%
\pgfpathlineto{\pgfqpoint{1.580293in}{0.342028in}}%
\pgfpathlineto{\pgfqpoint{1.544155in}{0.318198in}}%
\pgfpathlineto{\pgfqpoint{1.508016in}{0.296761in}}%
\pgfpathlineto{\pgfqpoint{1.471878in}{0.277503in}}%
\pgfpathlineto{\pgfqpoint{1.435740in}{0.260327in}}%
\pgfpathlineto{\pgfqpoint{1.391515in}{0.241882in}}%
\pgfpathlineto{\pgfqpoint{1.351417in}{0.227482in}}%
\pgfpathlineto{\pgfqpoint{1.303232in}{0.212959in}}%
\pgfpathlineto{\pgfqpoint{1.255048in}{0.201309in}}%
\pgfpathlineto{\pgfqpoint{1.206863in}{0.192415in}}%
\pgfpathlineto{\pgfqpoint{1.158679in}{0.186230in}}%
\pgfpathlineto{\pgfqpoint{1.110494in}{0.182648in}}%
\pgfpathlineto{\pgfqpoint{1.062310in}{0.181672in}}%
\pgfpathlineto{\pgfqpoint{1.014126in}{0.183299in}}%
\pgfpathlineto{\pgfqpoint{0.965941in}{0.187532in}}%
\pgfpathlineto{\pgfqpoint{0.929803in}{0.192415in}}%
\pgfpathlineto{\pgfqpoint{0.929803in}{0.192415in}}%
\pgfusepath{stroke}%
\end{pgfscope}%
\begin{pgfscope}%
\pgfsetrectcap%
\pgfsetmiterjoin%
\pgfsetlinewidth{0.803000pt}%
\definecolor{currentstroke}{rgb}{0.000000,0.000000,0.000000}%
\pgfsetstrokecolor{currentstroke}%
\pgfsetdash{}{0pt}%
\pgfpathmoveto{\pgfqpoint{1.068333in}{0.145754in}}%
\pgfpathlineto{\pgfqpoint{1.068333in}{2.012420in}}%
\pgfusepath{stroke}%
\end{pgfscope}%
\begin{pgfscope}%
\pgfsetrectcap%
\pgfsetmiterjoin%
\pgfsetlinewidth{0.803000pt}%
\definecolor{currentstroke}{rgb}{0.000000,0.000000,0.000000}%
\pgfsetstrokecolor{currentstroke}%
\pgfsetdash{}{0pt}%
\pgfpathmoveto{\pgfqpoint{0.135000in}{1.079087in}}%
\pgfpathlineto{\pgfqpoint{2.001666in}{1.079087in}}%
\pgfusepath{stroke}%
\end{pgfscope}%
\end{pgfpicture}%
\makeatother%
\endgroup%

            \end{center}
        \item Sí que es una m\'etrica
            \begin{center}
                %% Creator: Matplotlib, PGF backend
%%
%% To include the figure in your LaTeX document, write
%%   \input{<filename>.pgf}
%%
%% Make sure the required packages are loaded in your preamble
%%   \usepackage{pgf}
%%
%% Figures using additional raster images can only be included by \input if
%% they are in the same directory as the main LaTeX file. For loading figures
%% from other directories you can use the `import` package
%%   \usepackage{import}
%% and then include the figures with
%%   \import{<path to file>}{<filename>.pgf}
%%
%% Matplotlib used the following preamble
%%   \usepackage{fontspec}
%%   \setmainfont{DejaVu Serif}
%%   \setsansfont{DejaVu Sans}
%%   \setmonofont{DejaVu Sans Mono}
%%
\begingroup%
\makeatletter%
\begin{pgfpicture}%
\pgfpathrectangle{\pgfpointorigin}{\pgfqpoint{2.136666in}{2.147420in}}%
\pgfusepath{use as bounding box, clip}%
\begin{pgfscope}%
\pgfsetbuttcap%
\pgfsetmiterjoin%
\definecolor{currentfill}{rgb}{1.000000,1.000000,1.000000}%
\pgfsetfillcolor{currentfill}%
\pgfsetlinewidth{0.000000pt}%
\definecolor{currentstroke}{rgb}{1.000000,1.000000,1.000000}%
\pgfsetstrokecolor{currentstroke}%
\pgfsetdash{}{0pt}%
\pgfpathmoveto{\pgfqpoint{0.000000in}{0.000000in}}%
\pgfpathlineto{\pgfqpoint{2.136666in}{0.000000in}}%
\pgfpathlineto{\pgfqpoint{2.136666in}{2.147420in}}%
\pgfpathlineto{\pgfqpoint{0.000000in}{2.147420in}}%
\pgfpathclose%
\pgfusepath{fill}%
\end{pgfscope}%
\begin{pgfscope}%
\pgfsetbuttcap%
\pgfsetmiterjoin%
\definecolor{currentfill}{rgb}{1.000000,1.000000,1.000000}%
\pgfsetfillcolor{currentfill}%
\pgfsetlinewidth{0.000000pt}%
\definecolor{currentstroke}{rgb}{0.000000,0.000000,0.000000}%
\pgfsetstrokecolor{currentstroke}%
\pgfsetstrokeopacity{0.000000}%
\pgfsetdash{}{0pt}%
\pgfpathmoveto{\pgfqpoint{0.135000in}{0.145754in}}%
\pgfpathlineto{\pgfqpoint{2.001666in}{0.145754in}}%
\pgfpathlineto{\pgfqpoint{2.001666in}{2.012420in}}%
\pgfpathlineto{\pgfqpoint{0.135000in}{2.012420in}}%
\pgfpathclose%
\pgfusepath{fill}%
\end{pgfscope}%
\begin{pgfscope}%
\pgfpathrectangle{\pgfqpoint{0.135000in}{0.145754in}}{\pgfqpoint{1.866666in}{1.866666in}} %
\pgfusepath{clip}%
\pgfsetbuttcap%
\pgfsetroundjoin%
\definecolor{currentfill}{rgb}{0.000000,0.000000,1.000000}%
\pgfsetfillcolor{currentfill}%
\pgfsetfillopacity{0.300000}%
\pgfsetlinewidth{0.000000pt}%
\definecolor{currentstroke}{rgb}{0.000000,0.000000,0.000000}%
\pgfsetstrokecolor{currentstroke}%
\pgfsetdash{}{0pt}%
\pgfpathmoveto{\pgfqpoint{1.062310in}{0.187674in}}%
\pgfpathlineto{\pgfqpoint{1.074356in}{0.187674in}}%
\pgfpathlineto{\pgfqpoint{1.080379in}{0.193697in}}%
\pgfpathlineto{\pgfqpoint{1.086402in}{0.199720in}}%
\pgfpathlineto{\pgfqpoint{1.092425in}{0.205743in}}%
\pgfpathlineto{\pgfqpoint{1.098448in}{0.211767in}}%
\pgfpathlineto{\pgfqpoint{1.104471in}{0.217790in}}%
\pgfpathlineto{\pgfqpoint{1.110494in}{0.223813in}}%
\pgfpathlineto{\pgfqpoint{1.116517in}{0.229836in}}%
\pgfpathlineto{\pgfqpoint{1.122541in}{0.235859in}}%
\pgfpathlineto{\pgfqpoint{1.128564in}{0.241882in}}%
\pgfpathlineto{\pgfqpoint{1.134587in}{0.247905in}}%
\pgfpathlineto{\pgfqpoint{1.140610in}{0.253928in}}%
\pgfpathlineto{\pgfqpoint{1.146633in}{0.259951in}}%
\pgfpathlineto{\pgfqpoint{1.152656in}{0.265974in}}%
\pgfpathlineto{\pgfqpoint{1.158679in}{0.271997in}}%
\pgfpathlineto{\pgfqpoint{1.164702in}{0.278020in}}%
\pgfpathlineto{\pgfqpoint{1.170725in}{0.284043in}}%
\pgfpathlineto{\pgfqpoint{1.176748in}{0.290066in}}%
\pgfpathlineto{\pgfqpoint{1.182771in}{0.296089in}}%
\pgfpathlineto{\pgfqpoint{1.188794in}{0.302112in}}%
\pgfpathlineto{\pgfqpoint{1.194817in}{0.308135in}}%
\pgfpathlineto{\pgfqpoint{1.200840in}{0.314159in}}%
\pgfpathlineto{\pgfqpoint{1.206863in}{0.320182in}}%
\pgfpathlineto{\pgfqpoint{1.212886in}{0.326205in}}%
\pgfpathlineto{\pgfqpoint{1.218909in}{0.332228in}}%
\pgfpathlineto{\pgfqpoint{1.224933in}{0.338251in}}%
\pgfpathlineto{\pgfqpoint{1.230956in}{0.344274in}}%
\pgfpathlineto{\pgfqpoint{1.236979in}{0.350297in}}%
\pgfpathlineto{\pgfqpoint{1.243002in}{0.356320in}}%
\pgfpathlineto{\pgfqpoint{1.249025in}{0.362343in}}%
\pgfpathlineto{\pgfqpoint{1.255048in}{0.368366in}}%
\pgfpathlineto{\pgfqpoint{1.261071in}{0.374389in}}%
\pgfpathlineto{\pgfqpoint{1.267094in}{0.380412in}}%
\pgfpathlineto{\pgfqpoint{1.273117in}{0.386435in}}%
\pgfpathlineto{\pgfqpoint{1.279140in}{0.392458in}}%
\pgfpathlineto{\pgfqpoint{1.285163in}{0.398481in}}%
\pgfpathlineto{\pgfqpoint{1.291186in}{0.404504in}}%
\pgfpathlineto{\pgfqpoint{1.297209in}{0.410527in}}%
\pgfpathlineto{\pgfqpoint{1.303232in}{0.416551in}}%
\pgfpathlineto{\pgfqpoint{1.309255in}{0.422574in}}%
\pgfpathlineto{\pgfqpoint{1.315278in}{0.428597in}}%
\pgfpathlineto{\pgfqpoint{1.321301in}{0.434620in}}%
\pgfpathlineto{\pgfqpoint{1.327325in}{0.440643in}}%
\pgfpathlineto{\pgfqpoint{1.333348in}{0.446666in}}%
\pgfpathlineto{\pgfqpoint{1.339371in}{0.452689in}}%
\pgfpathlineto{\pgfqpoint{1.345394in}{0.458712in}}%
\pgfpathlineto{\pgfqpoint{1.351417in}{0.464735in}}%
\pgfpathlineto{\pgfqpoint{1.357440in}{0.470758in}}%
\pgfpathlineto{\pgfqpoint{1.363463in}{0.476781in}}%
\pgfpathlineto{\pgfqpoint{1.369486in}{0.482804in}}%
\pgfpathlineto{\pgfqpoint{1.375509in}{0.488827in}}%
\pgfpathlineto{\pgfqpoint{1.381532in}{0.494850in}}%
\pgfpathlineto{\pgfqpoint{1.387555in}{0.500873in}}%
\pgfpathlineto{\pgfqpoint{1.393578in}{0.506896in}}%
\pgfpathlineto{\pgfqpoint{1.399601in}{0.512919in}}%
\pgfpathlineto{\pgfqpoint{1.405624in}{0.518942in}}%
\pgfpathlineto{\pgfqpoint{1.411647in}{0.524966in}}%
\pgfpathlineto{\pgfqpoint{1.417670in}{0.530989in}}%
\pgfpathlineto{\pgfqpoint{1.423693in}{0.537012in}}%
\pgfpathlineto{\pgfqpoint{1.429716in}{0.543035in}}%
\pgfpathlineto{\pgfqpoint{1.435740in}{0.549058in}}%
\pgfpathlineto{\pgfqpoint{1.441763in}{0.555081in}}%
\pgfpathlineto{\pgfqpoint{1.447786in}{0.561104in}}%
\pgfpathlineto{\pgfqpoint{1.453809in}{0.567127in}}%
\pgfpathlineto{\pgfqpoint{1.459832in}{0.573150in}}%
\pgfpathlineto{\pgfqpoint{1.465855in}{0.579173in}}%
\pgfpathlineto{\pgfqpoint{1.471878in}{0.585196in}}%
\pgfpathlineto{\pgfqpoint{1.477901in}{0.591219in}}%
\pgfpathlineto{\pgfqpoint{1.483924in}{0.597242in}}%
\pgfpathlineto{\pgfqpoint{1.489947in}{0.603265in}}%
\pgfpathlineto{\pgfqpoint{1.495970in}{0.609288in}}%
\pgfpathlineto{\pgfqpoint{1.501993in}{0.615311in}}%
\pgfpathlineto{\pgfqpoint{1.508016in}{0.621334in}}%
\pgfpathlineto{\pgfqpoint{1.514039in}{0.627358in}}%
\pgfpathlineto{\pgfqpoint{1.520062in}{0.633381in}}%
\pgfpathlineto{\pgfqpoint{1.526085in}{0.639404in}}%
\pgfpathlineto{\pgfqpoint{1.532108in}{0.645427in}}%
\pgfpathlineto{\pgfqpoint{1.538132in}{0.651450in}}%
\pgfpathlineto{\pgfqpoint{1.544155in}{0.657473in}}%
\pgfpathlineto{\pgfqpoint{1.550178in}{0.663496in}}%
\pgfpathlineto{\pgfqpoint{1.556201in}{0.669519in}}%
\pgfpathlineto{\pgfqpoint{1.562224in}{0.675542in}}%
\pgfpathlineto{\pgfqpoint{1.568247in}{0.681565in}}%
\pgfpathlineto{\pgfqpoint{1.574270in}{0.687588in}}%
\pgfpathlineto{\pgfqpoint{1.580293in}{0.693611in}}%
\pgfpathlineto{\pgfqpoint{1.586316in}{0.699634in}}%
\pgfpathlineto{\pgfqpoint{1.592339in}{0.705657in}}%
\pgfpathlineto{\pgfqpoint{1.598362in}{0.711680in}}%
\pgfpathlineto{\pgfqpoint{1.604385in}{0.717703in}}%
\pgfpathlineto{\pgfqpoint{1.610408in}{0.723726in}}%
\pgfpathlineto{\pgfqpoint{1.616431in}{0.729750in}}%
\pgfpathlineto{\pgfqpoint{1.622454in}{0.735773in}}%
\pgfpathlineto{\pgfqpoint{1.628477in}{0.741796in}}%
\pgfpathlineto{\pgfqpoint{1.634500in}{0.747819in}}%
\pgfpathlineto{\pgfqpoint{1.640524in}{0.753842in}}%
\pgfpathlineto{\pgfqpoint{1.646547in}{0.759865in}}%
\pgfpathlineto{\pgfqpoint{1.652570in}{0.765888in}}%
\pgfpathlineto{\pgfqpoint{1.658593in}{0.771911in}}%
\pgfpathlineto{\pgfqpoint{1.664616in}{0.777934in}}%
\pgfpathlineto{\pgfqpoint{1.670639in}{0.783957in}}%
\pgfpathlineto{\pgfqpoint{1.676662in}{0.789980in}}%
\pgfpathlineto{\pgfqpoint{1.682685in}{0.796003in}}%
\pgfpathlineto{\pgfqpoint{1.688708in}{0.802026in}}%
\pgfpathlineto{\pgfqpoint{1.694731in}{0.808049in}}%
\pgfpathlineto{\pgfqpoint{1.700754in}{0.814072in}}%
\pgfpathlineto{\pgfqpoint{1.706777in}{0.820095in}}%
\pgfpathlineto{\pgfqpoint{1.712800in}{0.826118in}}%
\pgfpathlineto{\pgfqpoint{1.718823in}{0.832141in}}%
\pgfpathlineto{\pgfqpoint{1.724846in}{0.838165in}}%
\pgfpathlineto{\pgfqpoint{1.730869in}{0.844188in}}%
\pgfpathlineto{\pgfqpoint{1.736892in}{0.850211in}}%
\pgfpathlineto{\pgfqpoint{1.742915in}{0.856234in}}%
\pgfpathlineto{\pgfqpoint{1.748939in}{0.862257in}}%
\pgfpathlineto{\pgfqpoint{1.754962in}{0.868280in}}%
\pgfpathlineto{\pgfqpoint{1.760985in}{0.874303in}}%
\pgfpathlineto{\pgfqpoint{1.767008in}{0.880326in}}%
\pgfpathlineto{\pgfqpoint{1.773031in}{0.886349in}}%
\pgfpathlineto{\pgfqpoint{1.779054in}{0.892372in}}%
\pgfpathlineto{\pgfqpoint{1.785077in}{0.898395in}}%
\pgfpathlineto{\pgfqpoint{1.791100in}{0.904418in}}%
\pgfpathlineto{\pgfqpoint{1.797123in}{0.910441in}}%
\pgfpathlineto{\pgfqpoint{1.803146in}{0.916464in}}%
\pgfpathlineto{\pgfqpoint{1.809169in}{0.922487in}}%
\pgfpathlineto{\pgfqpoint{1.815192in}{0.928510in}}%
\pgfpathlineto{\pgfqpoint{1.821215in}{0.934533in}}%
\pgfpathlineto{\pgfqpoint{1.827238in}{0.940557in}}%
\pgfpathlineto{\pgfqpoint{1.833261in}{0.946580in}}%
\pgfpathlineto{\pgfqpoint{1.839284in}{0.952603in}}%
\pgfpathlineto{\pgfqpoint{1.845307in}{0.958626in}}%
\pgfpathlineto{\pgfqpoint{1.851331in}{0.964649in}}%
\pgfpathlineto{\pgfqpoint{1.857354in}{0.970672in}}%
\pgfpathlineto{\pgfqpoint{1.863377in}{0.976695in}}%
\pgfpathlineto{\pgfqpoint{1.869400in}{0.982718in}}%
\pgfpathlineto{\pgfqpoint{1.875423in}{0.988741in}}%
\pgfpathlineto{\pgfqpoint{1.881446in}{0.994764in}}%
\pgfpathlineto{\pgfqpoint{1.887469in}{1.000787in}}%
\pgfpathlineto{\pgfqpoint{1.893492in}{1.006810in}}%
\pgfpathlineto{\pgfqpoint{1.899515in}{1.012833in}}%
\pgfpathlineto{\pgfqpoint{1.905538in}{1.018856in}}%
\pgfpathlineto{\pgfqpoint{1.911561in}{1.024879in}}%
\pgfpathlineto{\pgfqpoint{1.917584in}{1.030902in}}%
\pgfpathlineto{\pgfqpoint{1.923607in}{1.036925in}}%
\pgfpathlineto{\pgfqpoint{1.929630in}{1.042949in}}%
\pgfpathlineto{\pgfqpoint{1.935653in}{1.048972in}}%
\pgfpathlineto{\pgfqpoint{1.941676in}{1.054995in}}%
\pgfpathlineto{\pgfqpoint{1.947699in}{1.061018in}}%
\pgfpathlineto{\pgfqpoint{1.953723in}{1.067041in}}%
\pgfpathlineto{\pgfqpoint{1.959746in}{1.073064in}}%
\pgfpathlineto{\pgfqpoint{1.959746in}{1.085110in}}%
\pgfpathlineto{\pgfqpoint{1.953723in}{1.091133in}}%
\pgfpathlineto{\pgfqpoint{1.947699in}{1.097156in}}%
\pgfpathlineto{\pgfqpoint{1.941676in}{1.103179in}}%
\pgfpathlineto{\pgfqpoint{1.935653in}{1.109202in}}%
\pgfpathlineto{\pgfqpoint{1.929630in}{1.115225in}}%
\pgfpathlineto{\pgfqpoint{1.923607in}{1.121248in}}%
\pgfpathlineto{\pgfqpoint{1.917584in}{1.127271in}}%
\pgfpathlineto{\pgfqpoint{1.911561in}{1.133294in}}%
\pgfpathlineto{\pgfqpoint{1.905538in}{1.139317in}}%
\pgfpathlineto{\pgfqpoint{1.899515in}{1.145340in}}%
\pgfpathlineto{\pgfqpoint{1.893492in}{1.151364in}}%
\pgfpathlineto{\pgfqpoint{1.887469in}{1.157387in}}%
\pgfpathlineto{\pgfqpoint{1.881446in}{1.163410in}}%
\pgfpathlineto{\pgfqpoint{1.875423in}{1.169433in}}%
\pgfpathlineto{\pgfqpoint{1.869400in}{1.175456in}}%
\pgfpathlineto{\pgfqpoint{1.863377in}{1.181479in}}%
\pgfpathlineto{\pgfqpoint{1.857354in}{1.187502in}}%
\pgfpathlineto{\pgfqpoint{1.851331in}{1.193525in}}%
\pgfpathlineto{\pgfqpoint{1.845307in}{1.199548in}}%
\pgfpathlineto{\pgfqpoint{1.839284in}{1.205571in}}%
\pgfpathlineto{\pgfqpoint{1.833261in}{1.211594in}}%
\pgfpathlineto{\pgfqpoint{1.827238in}{1.217617in}}%
\pgfpathlineto{\pgfqpoint{1.821215in}{1.223640in}}%
\pgfpathlineto{\pgfqpoint{1.815192in}{1.229663in}}%
\pgfpathlineto{\pgfqpoint{1.809169in}{1.235686in}}%
\pgfpathlineto{\pgfqpoint{1.803146in}{1.241709in}}%
\pgfpathlineto{\pgfqpoint{1.797123in}{1.247732in}}%
\pgfpathlineto{\pgfqpoint{1.791100in}{1.253756in}}%
\pgfpathlineto{\pgfqpoint{1.785077in}{1.259779in}}%
\pgfpathlineto{\pgfqpoint{1.779054in}{1.265802in}}%
\pgfpathlineto{\pgfqpoint{1.773031in}{1.271825in}}%
\pgfpathlineto{\pgfqpoint{1.767008in}{1.277848in}}%
\pgfpathlineto{\pgfqpoint{1.760985in}{1.283871in}}%
\pgfpathlineto{\pgfqpoint{1.754962in}{1.289894in}}%
\pgfpathlineto{\pgfqpoint{1.748939in}{1.295917in}}%
\pgfpathlineto{\pgfqpoint{1.742915in}{1.301940in}}%
\pgfpathlineto{\pgfqpoint{1.736892in}{1.307963in}}%
\pgfpathlineto{\pgfqpoint{1.730869in}{1.313986in}}%
\pgfpathlineto{\pgfqpoint{1.724846in}{1.320009in}}%
\pgfpathlineto{\pgfqpoint{1.718823in}{1.326032in}}%
\pgfpathlineto{\pgfqpoint{1.712800in}{1.332055in}}%
\pgfpathlineto{\pgfqpoint{1.706777in}{1.338078in}}%
\pgfpathlineto{\pgfqpoint{1.700754in}{1.344101in}}%
\pgfpathlineto{\pgfqpoint{1.694731in}{1.350124in}}%
\pgfpathlineto{\pgfqpoint{1.688708in}{1.356148in}}%
\pgfpathlineto{\pgfqpoint{1.682685in}{1.362171in}}%
\pgfpathlineto{\pgfqpoint{1.676662in}{1.368194in}}%
\pgfpathlineto{\pgfqpoint{1.670639in}{1.374217in}}%
\pgfpathlineto{\pgfqpoint{1.664616in}{1.380240in}}%
\pgfpathlineto{\pgfqpoint{1.658593in}{1.386263in}}%
\pgfpathlineto{\pgfqpoint{1.652570in}{1.392286in}}%
\pgfpathlineto{\pgfqpoint{1.646547in}{1.398309in}}%
\pgfpathlineto{\pgfqpoint{1.640524in}{1.404332in}}%
\pgfpathlineto{\pgfqpoint{1.634500in}{1.410355in}}%
\pgfpathlineto{\pgfqpoint{1.628477in}{1.416378in}}%
\pgfpathlineto{\pgfqpoint{1.622454in}{1.422401in}}%
\pgfpathlineto{\pgfqpoint{1.616431in}{1.428424in}}%
\pgfpathlineto{\pgfqpoint{1.610408in}{1.434447in}}%
\pgfpathlineto{\pgfqpoint{1.604385in}{1.440470in}}%
\pgfpathlineto{\pgfqpoint{1.598362in}{1.446493in}}%
\pgfpathlineto{\pgfqpoint{1.592339in}{1.452516in}}%
\pgfpathlineto{\pgfqpoint{1.586316in}{1.458539in}}%
\pgfpathlineto{\pgfqpoint{1.580293in}{1.464563in}}%
\pgfpathlineto{\pgfqpoint{1.574270in}{1.470586in}}%
\pgfpathlineto{\pgfqpoint{1.568247in}{1.476609in}}%
\pgfpathlineto{\pgfqpoint{1.562224in}{1.482632in}}%
\pgfpathlineto{\pgfqpoint{1.556201in}{1.488655in}}%
\pgfpathlineto{\pgfqpoint{1.550178in}{1.494678in}}%
\pgfpathlineto{\pgfqpoint{1.544155in}{1.500701in}}%
\pgfpathlineto{\pgfqpoint{1.538132in}{1.506724in}}%
\pgfpathlineto{\pgfqpoint{1.532108in}{1.512747in}}%
\pgfpathlineto{\pgfqpoint{1.526085in}{1.518770in}}%
\pgfpathlineto{\pgfqpoint{1.520062in}{1.524793in}}%
\pgfpathlineto{\pgfqpoint{1.514039in}{1.530816in}}%
\pgfpathlineto{\pgfqpoint{1.508016in}{1.536839in}}%
\pgfpathlineto{\pgfqpoint{1.501993in}{1.542862in}}%
\pgfpathlineto{\pgfqpoint{1.495970in}{1.548885in}}%
\pgfpathlineto{\pgfqpoint{1.489947in}{1.554908in}}%
\pgfpathlineto{\pgfqpoint{1.483924in}{1.560931in}}%
\pgfpathlineto{\pgfqpoint{1.477901in}{1.566955in}}%
\pgfpathlineto{\pgfqpoint{1.471878in}{1.572978in}}%
\pgfpathlineto{\pgfqpoint{1.465855in}{1.579001in}}%
\pgfpathlineto{\pgfqpoint{1.459832in}{1.585024in}}%
\pgfpathlineto{\pgfqpoint{1.453809in}{1.591047in}}%
\pgfpathlineto{\pgfqpoint{1.447786in}{1.597070in}}%
\pgfpathlineto{\pgfqpoint{1.441763in}{1.603093in}}%
\pgfpathlineto{\pgfqpoint{1.435740in}{1.609116in}}%
\pgfpathlineto{\pgfqpoint{1.429716in}{1.615139in}}%
\pgfpathlineto{\pgfqpoint{1.423693in}{1.621162in}}%
\pgfpathlineto{\pgfqpoint{1.417670in}{1.627185in}}%
\pgfpathlineto{\pgfqpoint{1.411647in}{1.633208in}}%
\pgfpathlineto{\pgfqpoint{1.405624in}{1.639231in}}%
\pgfpathlineto{\pgfqpoint{1.399601in}{1.645254in}}%
\pgfpathlineto{\pgfqpoint{1.393578in}{1.651277in}}%
\pgfpathlineto{\pgfqpoint{1.387555in}{1.657300in}}%
\pgfpathlineto{\pgfqpoint{1.381532in}{1.663323in}}%
\pgfpathlineto{\pgfqpoint{1.375509in}{1.669347in}}%
\pgfpathlineto{\pgfqpoint{1.369486in}{1.675370in}}%
\pgfpathlineto{\pgfqpoint{1.363463in}{1.681393in}}%
\pgfpathlineto{\pgfqpoint{1.357440in}{1.687416in}}%
\pgfpathlineto{\pgfqpoint{1.351417in}{1.693439in}}%
\pgfpathlineto{\pgfqpoint{1.345394in}{1.699462in}}%
\pgfpathlineto{\pgfqpoint{1.339371in}{1.705485in}}%
\pgfpathlineto{\pgfqpoint{1.333348in}{1.711508in}}%
\pgfpathlineto{\pgfqpoint{1.327325in}{1.717531in}}%
\pgfpathlineto{\pgfqpoint{1.321301in}{1.723554in}}%
\pgfpathlineto{\pgfqpoint{1.315278in}{1.729577in}}%
\pgfpathlineto{\pgfqpoint{1.309255in}{1.735600in}}%
\pgfpathlineto{\pgfqpoint{1.303232in}{1.741623in}}%
\pgfpathlineto{\pgfqpoint{1.297209in}{1.747646in}}%
\pgfpathlineto{\pgfqpoint{1.291186in}{1.753669in}}%
\pgfpathlineto{\pgfqpoint{1.285163in}{1.759692in}}%
\pgfpathlineto{\pgfqpoint{1.279140in}{1.765715in}}%
\pgfpathlineto{\pgfqpoint{1.273117in}{1.771738in}}%
\pgfpathlineto{\pgfqpoint{1.267094in}{1.777762in}}%
\pgfpathlineto{\pgfqpoint{1.261071in}{1.783785in}}%
\pgfpathlineto{\pgfqpoint{1.255048in}{1.789808in}}%
\pgfpathlineto{\pgfqpoint{1.249025in}{1.795831in}}%
\pgfpathlineto{\pgfqpoint{1.243002in}{1.801854in}}%
\pgfpathlineto{\pgfqpoint{1.236979in}{1.807877in}}%
\pgfpathlineto{\pgfqpoint{1.230956in}{1.813900in}}%
\pgfpathlineto{\pgfqpoint{1.224933in}{1.819923in}}%
\pgfpathlineto{\pgfqpoint{1.218909in}{1.825946in}}%
\pgfpathlineto{\pgfqpoint{1.212886in}{1.831969in}}%
\pgfpathlineto{\pgfqpoint{1.206863in}{1.837992in}}%
\pgfpathlineto{\pgfqpoint{1.200840in}{1.844015in}}%
\pgfpathlineto{\pgfqpoint{1.194817in}{1.850038in}}%
\pgfpathlineto{\pgfqpoint{1.188794in}{1.856061in}}%
\pgfpathlineto{\pgfqpoint{1.182771in}{1.862084in}}%
\pgfpathlineto{\pgfqpoint{1.176748in}{1.868107in}}%
\pgfpathlineto{\pgfqpoint{1.170725in}{1.874130in}}%
\pgfpathlineto{\pgfqpoint{1.164702in}{1.880154in}}%
\pgfpathlineto{\pgfqpoint{1.158679in}{1.886177in}}%
\pgfpathlineto{\pgfqpoint{1.152656in}{1.892200in}}%
\pgfpathlineto{\pgfqpoint{1.146633in}{1.898223in}}%
\pgfpathlineto{\pgfqpoint{1.140610in}{1.904246in}}%
\pgfpathlineto{\pgfqpoint{1.134587in}{1.910269in}}%
\pgfpathlineto{\pgfqpoint{1.128564in}{1.916292in}}%
\pgfpathlineto{\pgfqpoint{1.122541in}{1.922315in}}%
\pgfpathlineto{\pgfqpoint{1.116517in}{1.928338in}}%
\pgfpathlineto{\pgfqpoint{1.110494in}{1.934361in}}%
\pgfpathlineto{\pgfqpoint{1.104471in}{1.940384in}}%
\pgfpathlineto{\pgfqpoint{1.098448in}{1.946407in}}%
\pgfpathlineto{\pgfqpoint{1.092425in}{1.952430in}}%
\pgfpathlineto{\pgfqpoint{1.086402in}{1.958453in}}%
\pgfpathlineto{\pgfqpoint{1.080379in}{1.964476in}}%
\pgfpathlineto{\pgfqpoint{1.074356in}{1.970499in}}%
\pgfpathlineto{\pgfqpoint{1.062310in}{1.970499in}}%
\pgfpathlineto{\pgfqpoint{1.056287in}{1.964476in}}%
\pgfpathlineto{\pgfqpoint{1.050264in}{1.958453in}}%
\pgfpathlineto{\pgfqpoint{1.044241in}{1.952430in}}%
\pgfpathlineto{\pgfqpoint{1.038218in}{1.946407in}}%
\pgfpathlineto{\pgfqpoint{1.032195in}{1.940384in}}%
\pgfpathlineto{\pgfqpoint{1.026172in}{1.934361in}}%
\pgfpathlineto{\pgfqpoint{1.020149in}{1.928338in}}%
\pgfpathlineto{\pgfqpoint{1.014126in}{1.922315in}}%
\pgfpathlineto{\pgfqpoint{1.008102in}{1.916292in}}%
\pgfpathlineto{\pgfqpoint{1.002079in}{1.910269in}}%
\pgfpathlineto{\pgfqpoint{0.996056in}{1.904246in}}%
\pgfpathlineto{\pgfqpoint{0.990033in}{1.898223in}}%
\pgfpathlineto{\pgfqpoint{0.984010in}{1.892200in}}%
\pgfpathlineto{\pgfqpoint{0.977987in}{1.886177in}}%
\pgfpathlineto{\pgfqpoint{0.971964in}{1.880154in}}%
\pgfpathlineto{\pgfqpoint{0.965941in}{1.874130in}}%
\pgfpathlineto{\pgfqpoint{0.959918in}{1.868107in}}%
\pgfpathlineto{\pgfqpoint{0.953895in}{1.862084in}}%
\pgfpathlineto{\pgfqpoint{0.947872in}{1.856061in}}%
\pgfpathlineto{\pgfqpoint{0.941849in}{1.850038in}}%
\pgfpathlineto{\pgfqpoint{0.935826in}{1.844015in}}%
\pgfpathlineto{\pgfqpoint{0.929803in}{1.837992in}}%
\pgfpathlineto{\pgfqpoint{0.923780in}{1.831969in}}%
\pgfpathlineto{\pgfqpoint{0.917757in}{1.825946in}}%
\pgfpathlineto{\pgfqpoint{0.911734in}{1.819923in}}%
\pgfpathlineto{\pgfqpoint{0.905710in}{1.813900in}}%
\pgfpathlineto{\pgfqpoint{0.899687in}{1.807877in}}%
\pgfpathlineto{\pgfqpoint{0.893664in}{1.801854in}}%
\pgfpathlineto{\pgfqpoint{0.887641in}{1.795831in}}%
\pgfpathlineto{\pgfqpoint{0.881618in}{1.789808in}}%
\pgfpathlineto{\pgfqpoint{0.875595in}{1.783785in}}%
\pgfpathlineto{\pgfqpoint{0.869572in}{1.777762in}}%
\pgfpathlineto{\pgfqpoint{0.863549in}{1.771738in}}%
\pgfpathlineto{\pgfqpoint{0.857526in}{1.765715in}}%
\pgfpathlineto{\pgfqpoint{0.851503in}{1.759692in}}%
\pgfpathlineto{\pgfqpoint{0.845480in}{1.753669in}}%
\pgfpathlineto{\pgfqpoint{0.839457in}{1.747646in}}%
\pgfpathlineto{\pgfqpoint{0.833434in}{1.741623in}}%
\pgfpathlineto{\pgfqpoint{0.827411in}{1.735600in}}%
\pgfpathlineto{\pgfqpoint{0.821388in}{1.729577in}}%
\pgfpathlineto{\pgfqpoint{0.815365in}{1.723554in}}%
\pgfpathlineto{\pgfqpoint{0.809342in}{1.717531in}}%
\pgfpathlineto{\pgfqpoint{0.803318in}{1.711508in}}%
\pgfpathlineto{\pgfqpoint{0.797295in}{1.705485in}}%
\pgfpathlineto{\pgfqpoint{0.791272in}{1.699462in}}%
\pgfpathlineto{\pgfqpoint{0.785249in}{1.693439in}}%
\pgfpathlineto{\pgfqpoint{0.779226in}{1.687416in}}%
\pgfpathlineto{\pgfqpoint{0.773203in}{1.681393in}}%
\pgfpathlineto{\pgfqpoint{0.767180in}{1.675370in}}%
\pgfpathlineto{\pgfqpoint{0.761157in}{1.669347in}}%
\pgfpathlineto{\pgfqpoint{0.755134in}{1.663323in}}%
\pgfpathlineto{\pgfqpoint{0.749111in}{1.657300in}}%
\pgfpathlineto{\pgfqpoint{0.743088in}{1.651277in}}%
\pgfpathlineto{\pgfqpoint{0.737065in}{1.645254in}}%
\pgfpathlineto{\pgfqpoint{0.731042in}{1.639231in}}%
\pgfpathlineto{\pgfqpoint{0.725019in}{1.633208in}}%
\pgfpathlineto{\pgfqpoint{0.718996in}{1.627185in}}%
\pgfpathlineto{\pgfqpoint{0.712973in}{1.621162in}}%
\pgfpathlineto{\pgfqpoint{0.706950in}{1.615139in}}%
\pgfpathlineto{\pgfqpoint{0.700927in}{1.609116in}}%
\pgfpathlineto{\pgfqpoint{0.694903in}{1.603093in}}%
\pgfpathlineto{\pgfqpoint{0.688880in}{1.597070in}}%
\pgfpathlineto{\pgfqpoint{0.682857in}{1.591047in}}%
\pgfpathlineto{\pgfqpoint{0.676834in}{1.585024in}}%
\pgfpathlineto{\pgfqpoint{0.670811in}{1.579001in}}%
\pgfpathlineto{\pgfqpoint{0.664788in}{1.572978in}}%
\pgfpathlineto{\pgfqpoint{0.658765in}{1.566955in}}%
\pgfpathlineto{\pgfqpoint{0.652742in}{1.560931in}}%
\pgfpathlineto{\pgfqpoint{0.646719in}{1.554908in}}%
\pgfpathlineto{\pgfqpoint{0.640696in}{1.548885in}}%
\pgfpathlineto{\pgfqpoint{0.634673in}{1.542862in}}%
\pgfpathlineto{\pgfqpoint{0.628650in}{1.536839in}}%
\pgfpathlineto{\pgfqpoint{0.622627in}{1.530816in}}%
\pgfpathlineto{\pgfqpoint{0.616604in}{1.524793in}}%
\pgfpathlineto{\pgfqpoint{0.610581in}{1.518770in}}%
\pgfpathlineto{\pgfqpoint{0.604558in}{1.512747in}}%
\pgfpathlineto{\pgfqpoint{0.598535in}{1.506724in}}%
\pgfpathlineto{\pgfqpoint{0.592511in}{1.500701in}}%
\pgfpathlineto{\pgfqpoint{0.586488in}{1.494678in}}%
\pgfpathlineto{\pgfqpoint{0.580465in}{1.488655in}}%
\pgfpathlineto{\pgfqpoint{0.574442in}{1.482632in}}%
\pgfpathlineto{\pgfqpoint{0.568419in}{1.476609in}}%
\pgfpathlineto{\pgfqpoint{0.562396in}{1.470586in}}%
\pgfpathlineto{\pgfqpoint{0.556373in}{1.464563in}}%
\pgfpathlineto{\pgfqpoint{0.550350in}{1.458539in}}%
\pgfpathlineto{\pgfqpoint{0.544327in}{1.452516in}}%
\pgfpathlineto{\pgfqpoint{0.538304in}{1.446493in}}%
\pgfpathlineto{\pgfqpoint{0.532281in}{1.440470in}}%
\pgfpathlineto{\pgfqpoint{0.526258in}{1.434447in}}%
\pgfpathlineto{\pgfqpoint{0.520235in}{1.428424in}}%
\pgfpathlineto{\pgfqpoint{0.514212in}{1.422401in}}%
\pgfpathlineto{\pgfqpoint{0.508189in}{1.416378in}}%
\pgfpathlineto{\pgfqpoint{0.502166in}{1.410355in}}%
\pgfpathlineto{\pgfqpoint{0.496143in}{1.404332in}}%
\pgfpathlineto{\pgfqpoint{0.490119in}{1.398309in}}%
\pgfpathlineto{\pgfqpoint{0.484096in}{1.392286in}}%
\pgfpathlineto{\pgfqpoint{0.478073in}{1.386263in}}%
\pgfpathlineto{\pgfqpoint{0.472050in}{1.380240in}}%
\pgfpathlineto{\pgfqpoint{0.466027in}{1.374217in}}%
\pgfpathlineto{\pgfqpoint{0.460004in}{1.368194in}}%
\pgfpathlineto{\pgfqpoint{0.453981in}{1.362171in}}%
\pgfpathlineto{\pgfqpoint{0.447958in}{1.356148in}}%
\pgfpathlineto{\pgfqpoint{0.441935in}{1.350124in}}%
\pgfpathlineto{\pgfqpoint{0.435912in}{1.344101in}}%
\pgfpathlineto{\pgfqpoint{0.429889in}{1.338078in}}%
\pgfpathlineto{\pgfqpoint{0.423866in}{1.332055in}}%
\pgfpathlineto{\pgfqpoint{0.417843in}{1.326032in}}%
\pgfpathlineto{\pgfqpoint{0.411820in}{1.320009in}}%
\pgfpathlineto{\pgfqpoint{0.405797in}{1.313986in}}%
\pgfpathlineto{\pgfqpoint{0.399774in}{1.307963in}}%
\pgfpathlineto{\pgfqpoint{0.393751in}{1.301940in}}%
\pgfpathlineto{\pgfqpoint{0.387728in}{1.295917in}}%
\pgfpathlineto{\pgfqpoint{0.381704in}{1.289894in}}%
\pgfpathlineto{\pgfqpoint{0.375681in}{1.283871in}}%
\pgfpathlineto{\pgfqpoint{0.369658in}{1.277848in}}%
\pgfpathlineto{\pgfqpoint{0.363635in}{1.271825in}}%
\pgfpathlineto{\pgfqpoint{0.357612in}{1.265802in}}%
\pgfpathlineto{\pgfqpoint{0.351589in}{1.259779in}}%
\pgfpathlineto{\pgfqpoint{0.345566in}{1.253756in}}%
\pgfpathlineto{\pgfqpoint{0.339543in}{1.247732in}}%
\pgfpathlineto{\pgfqpoint{0.333520in}{1.241709in}}%
\pgfpathlineto{\pgfqpoint{0.327497in}{1.235686in}}%
\pgfpathlineto{\pgfqpoint{0.321474in}{1.229663in}}%
\pgfpathlineto{\pgfqpoint{0.315451in}{1.223640in}}%
\pgfpathlineto{\pgfqpoint{0.309428in}{1.217617in}}%
\pgfpathlineto{\pgfqpoint{0.303405in}{1.211594in}}%
\pgfpathlineto{\pgfqpoint{0.297382in}{1.205571in}}%
\pgfpathlineto{\pgfqpoint{0.291359in}{1.199548in}}%
\pgfpathlineto{\pgfqpoint{0.285336in}{1.193525in}}%
\pgfpathlineto{\pgfqpoint{0.279312in}{1.187502in}}%
\pgfpathlineto{\pgfqpoint{0.273289in}{1.181479in}}%
\pgfpathlineto{\pgfqpoint{0.267266in}{1.175456in}}%
\pgfpathlineto{\pgfqpoint{0.261243in}{1.169433in}}%
\pgfpathlineto{\pgfqpoint{0.255220in}{1.163410in}}%
\pgfpathlineto{\pgfqpoint{0.249197in}{1.157387in}}%
\pgfpathlineto{\pgfqpoint{0.243174in}{1.151364in}}%
\pgfpathlineto{\pgfqpoint{0.237151in}{1.145340in}}%
\pgfpathlineto{\pgfqpoint{0.231128in}{1.139317in}}%
\pgfpathlineto{\pgfqpoint{0.225105in}{1.133294in}}%
\pgfpathlineto{\pgfqpoint{0.219082in}{1.127271in}}%
\pgfpathlineto{\pgfqpoint{0.213059in}{1.121248in}}%
\pgfpathlineto{\pgfqpoint{0.207036in}{1.115225in}}%
\pgfpathlineto{\pgfqpoint{0.201013in}{1.109202in}}%
\pgfpathlineto{\pgfqpoint{0.194990in}{1.103179in}}%
\pgfpathlineto{\pgfqpoint{0.188967in}{1.097156in}}%
\pgfpathlineto{\pgfqpoint{0.182944in}{1.091133in}}%
\pgfpathlineto{\pgfqpoint{0.176920in}{1.085110in}}%
\pgfpathlineto{\pgfqpoint{0.176920in}{1.073064in}}%
\pgfpathlineto{\pgfqpoint{0.182944in}{1.067041in}}%
\pgfpathlineto{\pgfqpoint{0.188967in}{1.061018in}}%
\pgfpathlineto{\pgfqpoint{0.194990in}{1.054995in}}%
\pgfpathlineto{\pgfqpoint{0.201013in}{1.048972in}}%
\pgfpathlineto{\pgfqpoint{0.207036in}{1.042949in}}%
\pgfpathlineto{\pgfqpoint{0.213059in}{1.036925in}}%
\pgfpathlineto{\pgfqpoint{0.219082in}{1.030902in}}%
\pgfpathlineto{\pgfqpoint{0.225105in}{1.024879in}}%
\pgfpathlineto{\pgfqpoint{0.231128in}{1.018856in}}%
\pgfpathlineto{\pgfqpoint{0.237151in}{1.012833in}}%
\pgfpathlineto{\pgfqpoint{0.243174in}{1.006810in}}%
\pgfpathlineto{\pgfqpoint{0.249197in}{1.000787in}}%
\pgfpathlineto{\pgfqpoint{0.255220in}{0.994764in}}%
\pgfpathlineto{\pgfqpoint{0.261243in}{0.988741in}}%
\pgfpathlineto{\pgfqpoint{0.267266in}{0.982718in}}%
\pgfpathlineto{\pgfqpoint{0.273289in}{0.976695in}}%
\pgfpathlineto{\pgfqpoint{0.279312in}{0.970672in}}%
\pgfpathlineto{\pgfqpoint{0.285336in}{0.964649in}}%
\pgfpathlineto{\pgfqpoint{0.291359in}{0.958626in}}%
\pgfpathlineto{\pgfqpoint{0.297382in}{0.952603in}}%
\pgfpathlineto{\pgfqpoint{0.303405in}{0.946580in}}%
\pgfpathlineto{\pgfqpoint{0.309428in}{0.940557in}}%
\pgfpathlineto{\pgfqpoint{0.315451in}{0.934533in}}%
\pgfpathlineto{\pgfqpoint{0.321474in}{0.928510in}}%
\pgfpathlineto{\pgfqpoint{0.327497in}{0.922487in}}%
\pgfpathlineto{\pgfqpoint{0.333520in}{0.916464in}}%
\pgfpathlineto{\pgfqpoint{0.339543in}{0.910441in}}%
\pgfpathlineto{\pgfqpoint{0.345566in}{0.904418in}}%
\pgfpathlineto{\pgfqpoint{0.351589in}{0.898395in}}%
\pgfpathlineto{\pgfqpoint{0.357612in}{0.892372in}}%
\pgfpathlineto{\pgfqpoint{0.363635in}{0.886349in}}%
\pgfpathlineto{\pgfqpoint{0.369658in}{0.880326in}}%
\pgfpathlineto{\pgfqpoint{0.375681in}{0.874303in}}%
\pgfpathlineto{\pgfqpoint{0.381704in}{0.868280in}}%
\pgfpathlineto{\pgfqpoint{0.387728in}{0.862257in}}%
\pgfpathlineto{\pgfqpoint{0.393751in}{0.856234in}}%
\pgfpathlineto{\pgfqpoint{0.399774in}{0.850211in}}%
\pgfpathlineto{\pgfqpoint{0.405797in}{0.844188in}}%
\pgfpathlineto{\pgfqpoint{0.411820in}{0.838165in}}%
\pgfpathlineto{\pgfqpoint{0.417843in}{0.832141in}}%
\pgfpathlineto{\pgfqpoint{0.423866in}{0.826118in}}%
\pgfpathlineto{\pgfqpoint{0.429889in}{0.820095in}}%
\pgfpathlineto{\pgfqpoint{0.435912in}{0.814072in}}%
\pgfpathlineto{\pgfqpoint{0.441935in}{0.808049in}}%
\pgfpathlineto{\pgfqpoint{0.447958in}{0.802026in}}%
\pgfpathlineto{\pgfqpoint{0.453981in}{0.796003in}}%
\pgfpathlineto{\pgfqpoint{0.460004in}{0.789980in}}%
\pgfpathlineto{\pgfqpoint{0.466027in}{0.783957in}}%
\pgfpathlineto{\pgfqpoint{0.472050in}{0.777934in}}%
\pgfpathlineto{\pgfqpoint{0.478073in}{0.771911in}}%
\pgfpathlineto{\pgfqpoint{0.484096in}{0.765888in}}%
\pgfpathlineto{\pgfqpoint{0.490119in}{0.759865in}}%
\pgfpathlineto{\pgfqpoint{0.496143in}{0.753842in}}%
\pgfpathlineto{\pgfqpoint{0.502166in}{0.747819in}}%
\pgfpathlineto{\pgfqpoint{0.508189in}{0.741796in}}%
\pgfpathlineto{\pgfqpoint{0.514212in}{0.735773in}}%
\pgfpathlineto{\pgfqpoint{0.520235in}{0.729750in}}%
\pgfpathlineto{\pgfqpoint{0.526258in}{0.723726in}}%
\pgfpathlineto{\pgfqpoint{0.532281in}{0.717703in}}%
\pgfpathlineto{\pgfqpoint{0.538304in}{0.711680in}}%
\pgfpathlineto{\pgfqpoint{0.544327in}{0.705657in}}%
\pgfpathlineto{\pgfqpoint{0.550350in}{0.699634in}}%
\pgfpathlineto{\pgfqpoint{0.556373in}{0.693611in}}%
\pgfpathlineto{\pgfqpoint{0.562396in}{0.687588in}}%
\pgfpathlineto{\pgfqpoint{0.568419in}{0.681565in}}%
\pgfpathlineto{\pgfqpoint{0.574442in}{0.675542in}}%
\pgfpathlineto{\pgfqpoint{0.580465in}{0.669519in}}%
\pgfpathlineto{\pgfqpoint{0.586488in}{0.663496in}}%
\pgfpathlineto{\pgfqpoint{0.592511in}{0.657473in}}%
\pgfpathlineto{\pgfqpoint{0.598535in}{0.651450in}}%
\pgfpathlineto{\pgfqpoint{0.604558in}{0.645427in}}%
\pgfpathlineto{\pgfqpoint{0.610581in}{0.639404in}}%
\pgfpathlineto{\pgfqpoint{0.616604in}{0.633381in}}%
\pgfpathlineto{\pgfqpoint{0.622627in}{0.627358in}}%
\pgfpathlineto{\pgfqpoint{0.628650in}{0.621334in}}%
\pgfpathlineto{\pgfqpoint{0.634673in}{0.615311in}}%
\pgfpathlineto{\pgfqpoint{0.640696in}{0.609288in}}%
\pgfpathlineto{\pgfqpoint{0.646719in}{0.603265in}}%
\pgfpathlineto{\pgfqpoint{0.652742in}{0.597242in}}%
\pgfpathlineto{\pgfqpoint{0.658765in}{0.591219in}}%
\pgfpathlineto{\pgfqpoint{0.664788in}{0.585196in}}%
\pgfpathlineto{\pgfqpoint{0.670811in}{0.579173in}}%
\pgfpathlineto{\pgfqpoint{0.676834in}{0.573150in}}%
\pgfpathlineto{\pgfqpoint{0.682857in}{0.567127in}}%
\pgfpathlineto{\pgfqpoint{0.688880in}{0.561104in}}%
\pgfpathlineto{\pgfqpoint{0.694903in}{0.555081in}}%
\pgfpathlineto{\pgfqpoint{0.700927in}{0.549058in}}%
\pgfpathlineto{\pgfqpoint{0.706950in}{0.543035in}}%
\pgfpathlineto{\pgfqpoint{0.712973in}{0.537012in}}%
\pgfpathlineto{\pgfqpoint{0.718996in}{0.530989in}}%
\pgfpathlineto{\pgfqpoint{0.725019in}{0.524966in}}%
\pgfpathlineto{\pgfqpoint{0.731042in}{0.518942in}}%
\pgfpathlineto{\pgfqpoint{0.737065in}{0.512919in}}%
\pgfpathlineto{\pgfqpoint{0.743088in}{0.506896in}}%
\pgfpathlineto{\pgfqpoint{0.749111in}{0.500873in}}%
\pgfpathlineto{\pgfqpoint{0.755134in}{0.494850in}}%
\pgfpathlineto{\pgfqpoint{0.761157in}{0.488827in}}%
\pgfpathlineto{\pgfqpoint{0.767180in}{0.482804in}}%
\pgfpathlineto{\pgfqpoint{0.773203in}{0.476781in}}%
\pgfpathlineto{\pgfqpoint{0.779226in}{0.470758in}}%
\pgfpathlineto{\pgfqpoint{0.785249in}{0.464735in}}%
\pgfpathlineto{\pgfqpoint{0.791272in}{0.458712in}}%
\pgfpathlineto{\pgfqpoint{0.797295in}{0.452689in}}%
\pgfpathlineto{\pgfqpoint{0.803318in}{0.446666in}}%
\pgfpathlineto{\pgfqpoint{0.809342in}{0.440643in}}%
\pgfpathlineto{\pgfqpoint{0.815365in}{0.434620in}}%
\pgfpathlineto{\pgfqpoint{0.821388in}{0.428597in}}%
\pgfpathlineto{\pgfqpoint{0.827411in}{0.422574in}}%
\pgfpathlineto{\pgfqpoint{0.833434in}{0.416551in}}%
\pgfpathlineto{\pgfqpoint{0.839457in}{0.410527in}}%
\pgfpathlineto{\pgfqpoint{0.845480in}{0.404504in}}%
\pgfpathlineto{\pgfqpoint{0.851503in}{0.398481in}}%
\pgfpathlineto{\pgfqpoint{0.857526in}{0.392458in}}%
\pgfpathlineto{\pgfqpoint{0.863549in}{0.386435in}}%
\pgfpathlineto{\pgfqpoint{0.869572in}{0.380412in}}%
\pgfpathlineto{\pgfqpoint{0.875595in}{0.374389in}}%
\pgfpathlineto{\pgfqpoint{0.881618in}{0.368366in}}%
\pgfpathlineto{\pgfqpoint{0.887641in}{0.362343in}}%
\pgfpathlineto{\pgfqpoint{0.893664in}{0.356320in}}%
\pgfpathlineto{\pgfqpoint{0.899687in}{0.350297in}}%
\pgfpathlineto{\pgfqpoint{0.905710in}{0.344274in}}%
\pgfpathlineto{\pgfqpoint{0.911734in}{0.338251in}}%
\pgfpathlineto{\pgfqpoint{0.917757in}{0.332228in}}%
\pgfpathlineto{\pgfqpoint{0.923780in}{0.326205in}}%
\pgfpathlineto{\pgfqpoint{0.929803in}{0.320182in}}%
\pgfpathlineto{\pgfqpoint{0.935826in}{0.314159in}}%
\pgfpathlineto{\pgfqpoint{0.941849in}{0.308135in}}%
\pgfpathlineto{\pgfqpoint{0.947872in}{0.302112in}}%
\pgfpathlineto{\pgfqpoint{0.953895in}{0.296089in}}%
\pgfpathlineto{\pgfqpoint{0.959918in}{0.290066in}}%
\pgfpathlineto{\pgfqpoint{0.965941in}{0.284043in}}%
\pgfpathlineto{\pgfqpoint{0.971964in}{0.278020in}}%
\pgfpathlineto{\pgfqpoint{0.977987in}{0.271997in}}%
\pgfpathlineto{\pgfqpoint{0.984010in}{0.265974in}}%
\pgfpathlineto{\pgfqpoint{0.990033in}{0.259951in}}%
\pgfpathlineto{\pgfqpoint{0.996056in}{0.253928in}}%
\pgfpathlineto{\pgfqpoint{1.002079in}{0.247905in}}%
\pgfpathlineto{\pgfqpoint{1.008102in}{0.241882in}}%
\pgfpathlineto{\pgfqpoint{1.014126in}{0.235859in}}%
\pgfpathlineto{\pgfqpoint{1.020149in}{0.229836in}}%
\pgfpathlineto{\pgfqpoint{1.026172in}{0.223813in}}%
\pgfpathlineto{\pgfqpoint{1.032195in}{0.217790in}}%
\pgfpathlineto{\pgfqpoint{1.038218in}{0.211767in}}%
\pgfpathlineto{\pgfqpoint{1.044241in}{0.205743in}}%
\pgfpathlineto{\pgfqpoint{1.050264in}{0.199720in}}%
\pgfpathlineto{\pgfqpoint{1.056287in}{0.193697in}}%
\pgfpathclose%
\pgfusepath{fill}%
\end{pgfscope}%
\begin{pgfscope}%
\pgfpathrectangle{\pgfqpoint{0.135000in}{0.145754in}}{\pgfqpoint{1.866666in}{1.866666in}} %
\pgfusepath{clip}%
\pgfsetbuttcap%
\pgfsetroundjoin%
\definecolor{currentfill}{rgb}{0.000000,0.000000,1.000000}%
\pgfsetfillcolor{currentfill}%
\pgfsetlinewidth{0.000000pt}%
\definecolor{currentstroke}{rgb}{0.000000,0.000000,0.000000}%
\pgfsetstrokecolor{currentstroke}%
\pgfsetdash{}{0pt}%
\pgfpathmoveto{\pgfqpoint{1.062310in}{0.187674in}}%
\pgfpathlineto{\pgfqpoint{1.074356in}{0.187674in}}%
\pgfpathlineto{\pgfqpoint{1.080379in}{0.193697in}}%
\pgfpathlineto{\pgfqpoint{1.086402in}{0.199720in}}%
\pgfpathlineto{\pgfqpoint{1.092425in}{0.205743in}}%
\pgfpathlineto{\pgfqpoint{1.098448in}{0.211767in}}%
\pgfpathlineto{\pgfqpoint{1.104471in}{0.217790in}}%
\pgfpathlineto{\pgfqpoint{1.110494in}{0.223813in}}%
\pgfpathlineto{\pgfqpoint{1.116517in}{0.229836in}}%
\pgfpathlineto{\pgfqpoint{1.122541in}{0.235859in}}%
\pgfpathlineto{\pgfqpoint{1.128564in}{0.241882in}}%
\pgfpathlineto{\pgfqpoint{1.134587in}{0.247905in}}%
\pgfpathlineto{\pgfqpoint{1.140610in}{0.253928in}}%
\pgfpathlineto{\pgfqpoint{1.146633in}{0.259951in}}%
\pgfpathlineto{\pgfqpoint{1.152656in}{0.265974in}}%
\pgfpathlineto{\pgfqpoint{1.158679in}{0.271997in}}%
\pgfpathlineto{\pgfqpoint{1.164702in}{0.278020in}}%
\pgfpathlineto{\pgfqpoint{1.170725in}{0.284043in}}%
\pgfpathlineto{\pgfqpoint{1.176748in}{0.290066in}}%
\pgfpathlineto{\pgfqpoint{1.182771in}{0.296089in}}%
\pgfpathlineto{\pgfqpoint{1.188794in}{0.302112in}}%
\pgfpathlineto{\pgfqpoint{1.194817in}{0.308135in}}%
\pgfpathlineto{\pgfqpoint{1.200840in}{0.314159in}}%
\pgfpathlineto{\pgfqpoint{1.206863in}{0.320182in}}%
\pgfpathlineto{\pgfqpoint{1.212886in}{0.326205in}}%
\pgfpathlineto{\pgfqpoint{1.218909in}{0.332228in}}%
\pgfpathlineto{\pgfqpoint{1.224933in}{0.338251in}}%
\pgfpathlineto{\pgfqpoint{1.230956in}{0.344274in}}%
\pgfpathlineto{\pgfqpoint{1.236979in}{0.350297in}}%
\pgfpathlineto{\pgfqpoint{1.243002in}{0.356320in}}%
\pgfpathlineto{\pgfqpoint{1.249025in}{0.362343in}}%
\pgfpathlineto{\pgfqpoint{1.255048in}{0.368366in}}%
\pgfpathlineto{\pgfqpoint{1.261071in}{0.374389in}}%
\pgfpathlineto{\pgfqpoint{1.267094in}{0.380412in}}%
\pgfpathlineto{\pgfqpoint{1.273117in}{0.386435in}}%
\pgfpathlineto{\pgfqpoint{1.279140in}{0.392458in}}%
\pgfpathlineto{\pgfqpoint{1.285163in}{0.398481in}}%
\pgfpathlineto{\pgfqpoint{1.291186in}{0.404504in}}%
\pgfpathlineto{\pgfqpoint{1.297209in}{0.410527in}}%
\pgfpathlineto{\pgfqpoint{1.303232in}{0.416551in}}%
\pgfpathlineto{\pgfqpoint{1.309255in}{0.422574in}}%
\pgfpathlineto{\pgfqpoint{1.315278in}{0.428597in}}%
\pgfpathlineto{\pgfqpoint{1.321301in}{0.434620in}}%
\pgfpathlineto{\pgfqpoint{1.327325in}{0.440643in}}%
\pgfpathlineto{\pgfqpoint{1.333348in}{0.446666in}}%
\pgfpathlineto{\pgfqpoint{1.339371in}{0.452689in}}%
\pgfpathlineto{\pgfqpoint{1.345394in}{0.458712in}}%
\pgfpathlineto{\pgfqpoint{1.351417in}{0.464735in}}%
\pgfpathlineto{\pgfqpoint{1.357440in}{0.470758in}}%
\pgfpathlineto{\pgfqpoint{1.363463in}{0.476781in}}%
\pgfpathlineto{\pgfqpoint{1.369486in}{0.482804in}}%
\pgfpathlineto{\pgfqpoint{1.375509in}{0.488827in}}%
\pgfpathlineto{\pgfqpoint{1.381532in}{0.494850in}}%
\pgfpathlineto{\pgfqpoint{1.387555in}{0.500873in}}%
\pgfpathlineto{\pgfqpoint{1.393578in}{0.506896in}}%
\pgfpathlineto{\pgfqpoint{1.399601in}{0.512919in}}%
\pgfpathlineto{\pgfqpoint{1.405624in}{0.518942in}}%
\pgfpathlineto{\pgfqpoint{1.411647in}{0.524966in}}%
\pgfpathlineto{\pgfqpoint{1.417670in}{0.530989in}}%
\pgfpathlineto{\pgfqpoint{1.423693in}{0.537012in}}%
\pgfpathlineto{\pgfqpoint{1.429716in}{0.543035in}}%
\pgfpathlineto{\pgfqpoint{1.435740in}{0.549058in}}%
\pgfpathlineto{\pgfqpoint{1.441763in}{0.555081in}}%
\pgfpathlineto{\pgfqpoint{1.447786in}{0.561104in}}%
\pgfpathlineto{\pgfqpoint{1.453809in}{0.567127in}}%
\pgfpathlineto{\pgfqpoint{1.459832in}{0.573150in}}%
\pgfpathlineto{\pgfqpoint{1.465855in}{0.579173in}}%
\pgfpathlineto{\pgfqpoint{1.471878in}{0.585196in}}%
\pgfpathlineto{\pgfqpoint{1.477901in}{0.591219in}}%
\pgfpathlineto{\pgfqpoint{1.483924in}{0.597242in}}%
\pgfpathlineto{\pgfqpoint{1.489947in}{0.603265in}}%
\pgfpathlineto{\pgfqpoint{1.495970in}{0.609288in}}%
\pgfpathlineto{\pgfqpoint{1.501993in}{0.615311in}}%
\pgfpathlineto{\pgfqpoint{1.508016in}{0.621334in}}%
\pgfpathlineto{\pgfqpoint{1.514039in}{0.627358in}}%
\pgfpathlineto{\pgfqpoint{1.520062in}{0.633381in}}%
\pgfpathlineto{\pgfqpoint{1.526085in}{0.639404in}}%
\pgfpathlineto{\pgfqpoint{1.532108in}{0.645427in}}%
\pgfpathlineto{\pgfqpoint{1.538132in}{0.651450in}}%
\pgfpathlineto{\pgfqpoint{1.544155in}{0.657473in}}%
\pgfpathlineto{\pgfqpoint{1.550178in}{0.663496in}}%
\pgfpathlineto{\pgfqpoint{1.556201in}{0.669519in}}%
\pgfpathlineto{\pgfqpoint{1.562224in}{0.675542in}}%
\pgfpathlineto{\pgfqpoint{1.568247in}{0.681565in}}%
\pgfpathlineto{\pgfqpoint{1.574270in}{0.687588in}}%
\pgfpathlineto{\pgfqpoint{1.580293in}{0.693611in}}%
\pgfpathlineto{\pgfqpoint{1.586316in}{0.699634in}}%
\pgfpathlineto{\pgfqpoint{1.592339in}{0.705657in}}%
\pgfpathlineto{\pgfqpoint{1.598362in}{0.711680in}}%
\pgfpathlineto{\pgfqpoint{1.604385in}{0.717703in}}%
\pgfpathlineto{\pgfqpoint{1.610408in}{0.723726in}}%
\pgfpathlineto{\pgfqpoint{1.616431in}{0.729750in}}%
\pgfpathlineto{\pgfqpoint{1.622454in}{0.735773in}}%
\pgfpathlineto{\pgfqpoint{1.628477in}{0.741796in}}%
\pgfpathlineto{\pgfqpoint{1.634500in}{0.747819in}}%
\pgfpathlineto{\pgfqpoint{1.640524in}{0.753842in}}%
\pgfpathlineto{\pgfqpoint{1.646547in}{0.759865in}}%
\pgfpathlineto{\pgfqpoint{1.652570in}{0.765888in}}%
\pgfpathlineto{\pgfqpoint{1.658593in}{0.771911in}}%
\pgfpathlineto{\pgfqpoint{1.664616in}{0.777934in}}%
\pgfpathlineto{\pgfqpoint{1.670639in}{0.783957in}}%
\pgfpathlineto{\pgfqpoint{1.676662in}{0.789980in}}%
\pgfpathlineto{\pgfqpoint{1.682685in}{0.796003in}}%
\pgfpathlineto{\pgfqpoint{1.688708in}{0.802026in}}%
\pgfpathlineto{\pgfqpoint{1.694731in}{0.808049in}}%
\pgfpathlineto{\pgfqpoint{1.700754in}{0.814072in}}%
\pgfpathlineto{\pgfqpoint{1.706777in}{0.820095in}}%
\pgfpathlineto{\pgfqpoint{1.712800in}{0.826118in}}%
\pgfpathlineto{\pgfqpoint{1.718823in}{0.832141in}}%
\pgfpathlineto{\pgfqpoint{1.724846in}{0.838165in}}%
\pgfpathlineto{\pgfqpoint{1.730869in}{0.844188in}}%
\pgfpathlineto{\pgfqpoint{1.736892in}{0.850211in}}%
\pgfpathlineto{\pgfqpoint{1.742915in}{0.856234in}}%
\pgfpathlineto{\pgfqpoint{1.748939in}{0.862257in}}%
\pgfpathlineto{\pgfqpoint{1.754962in}{0.868280in}}%
\pgfpathlineto{\pgfqpoint{1.760985in}{0.874303in}}%
\pgfpathlineto{\pgfqpoint{1.767008in}{0.880326in}}%
\pgfpathlineto{\pgfqpoint{1.773031in}{0.886349in}}%
\pgfpathlineto{\pgfqpoint{1.779054in}{0.892372in}}%
\pgfpathlineto{\pgfqpoint{1.785077in}{0.898395in}}%
\pgfpathlineto{\pgfqpoint{1.791100in}{0.904418in}}%
\pgfpathlineto{\pgfqpoint{1.797123in}{0.910441in}}%
\pgfpathlineto{\pgfqpoint{1.803146in}{0.916464in}}%
\pgfpathlineto{\pgfqpoint{1.809169in}{0.922487in}}%
\pgfpathlineto{\pgfqpoint{1.815192in}{0.928510in}}%
\pgfpathlineto{\pgfqpoint{1.821215in}{0.934533in}}%
\pgfpathlineto{\pgfqpoint{1.827238in}{0.940557in}}%
\pgfpathlineto{\pgfqpoint{1.833261in}{0.946580in}}%
\pgfpathlineto{\pgfqpoint{1.839284in}{0.952603in}}%
\pgfpathlineto{\pgfqpoint{1.845307in}{0.958626in}}%
\pgfpathlineto{\pgfqpoint{1.851331in}{0.964649in}}%
\pgfpathlineto{\pgfqpoint{1.857354in}{0.970672in}}%
\pgfpathlineto{\pgfqpoint{1.863377in}{0.976695in}}%
\pgfpathlineto{\pgfqpoint{1.869400in}{0.982718in}}%
\pgfpathlineto{\pgfqpoint{1.875423in}{0.988741in}}%
\pgfpathlineto{\pgfqpoint{1.881446in}{0.994764in}}%
\pgfpathlineto{\pgfqpoint{1.887469in}{1.000787in}}%
\pgfpathlineto{\pgfqpoint{1.893492in}{1.006810in}}%
\pgfpathlineto{\pgfqpoint{1.899515in}{1.012833in}}%
\pgfpathlineto{\pgfqpoint{1.905538in}{1.018856in}}%
\pgfpathlineto{\pgfqpoint{1.911561in}{1.024879in}}%
\pgfpathlineto{\pgfqpoint{1.917584in}{1.030902in}}%
\pgfpathlineto{\pgfqpoint{1.923607in}{1.036925in}}%
\pgfpathlineto{\pgfqpoint{1.929630in}{1.042949in}}%
\pgfpathlineto{\pgfqpoint{1.935653in}{1.048972in}}%
\pgfpathlineto{\pgfqpoint{1.941676in}{1.054995in}}%
\pgfpathlineto{\pgfqpoint{1.947699in}{1.061018in}}%
\pgfpathlineto{\pgfqpoint{1.953723in}{1.067041in}}%
\pgfpathlineto{\pgfqpoint{1.959746in}{1.073064in}}%
\pgfpathlineto{\pgfqpoint{1.959746in}{1.085110in}}%
\pgfpathlineto{\pgfqpoint{1.953723in}{1.091133in}}%
\pgfpathlineto{\pgfqpoint{1.947699in}{1.097156in}}%
\pgfpathlineto{\pgfqpoint{1.941676in}{1.103179in}}%
\pgfpathlineto{\pgfqpoint{1.935653in}{1.109202in}}%
\pgfpathlineto{\pgfqpoint{1.929630in}{1.115225in}}%
\pgfpathlineto{\pgfqpoint{1.923607in}{1.121248in}}%
\pgfpathlineto{\pgfqpoint{1.917584in}{1.127271in}}%
\pgfpathlineto{\pgfqpoint{1.911561in}{1.133294in}}%
\pgfpathlineto{\pgfqpoint{1.905538in}{1.139317in}}%
\pgfpathlineto{\pgfqpoint{1.899515in}{1.145340in}}%
\pgfpathlineto{\pgfqpoint{1.893492in}{1.151364in}}%
\pgfpathlineto{\pgfqpoint{1.887469in}{1.157387in}}%
\pgfpathlineto{\pgfqpoint{1.881446in}{1.163410in}}%
\pgfpathlineto{\pgfqpoint{1.875423in}{1.169433in}}%
\pgfpathlineto{\pgfqpoint{1.869400in}{1.175456in}}%
\pgfpathlineto{\pgfqpoint{1.863377in}{1.181479in}}%
\pgfpathlineto{\pgfqpoint{1.857354in}{1.187502in}}%
\pgfpathlineto{\pgfqpoint{1.851331in}{1.193525in}}%
\pgfpathlineto{\pgfqpoint{1.845307in}{1.199548in}}%
\pgfpathlineto{\pgfqpoint{1.839284in}{1.205571in}}%
\pgfpathlineto{\pgfqpoint{1.833261in}{1.211594in}}%
\pgfpathlineto{\pgfqpoint{1.827238in}{1.217617in}}%
\pgfpathlineto{\pgfqpoint{1.821215in}{1.223640in}}%
\pgfpathlineto{\pgfqpoint{1.815192in}{1.229663in}}%
\pgfpathlineto{\pgfqpoint{1.809169in}{1.235686in}}%
\pgfpathlineto{\pgfqpoint{1.803146in}{1.241709in}}%
\pgfpathlineto{\pgfqpoint{1.797123in}{1.247732in}}%
\pgfpathlineto{\pgfqpoint{1.791100in}{1.253756in}}%
\pgfpathlineto{\pgfqpoint{1.785077in}{1.259779in}}%
\pgfpathlineto{\pgfqpoint{1.779054in}{1.265802in}}%
\pgfpathlineto{\pgfqpoint{1.773031in}{1.271825in}}%
\pgfpathlineto{\pgfqpoint{1.767008in}{1.277848in}}%
\pgfpathlineto{\pgfqpoint{1.760985in}{1.283871in}}%
\pgfpathlineto{\pgfqpoint{1.754962in}{1.289894in}}%
\pgfpathlineto{\pgfqpoint{1.748939in}{1.295917in}}%
\pgfpathlineto{\pgfqpoint{1.742915in}{1.301940in}}%
\pgfpathlineto{\pgfqpoint{1.736892in}{1.307963in}}%
\pgfpathlineto{\pgfqpoint{1.730869in}{1.313986in}}%
\pgfpathlineto{\pgfqpoint{1.724846in}{1.320009in}}%
\pgfpathlineto{\pgfqpoint{1.718823in}{1.326032in}}%
\pgfpathlineto{\pgfqpoint{1.712800in}{1.332055in}}%
\pgfpathlineto{\pgfqpoint{1.706777in}{1.338078in}}%
\pgfpathlineto{\pgfqpoint{1.700754in}{1.344101in}}%
\pgfpathlineto{\pgfqpoint{1.694731in}{1.350124in}}%
\pgfpathlineto{\pgfqpoint{1.688708in}{1.356148in}}%
\pgfpathlineto{\pgfqpoint{1.682685in}{1.362171in}}%
\pgfpathlineto{\pgfqpoint{1.676662in}{1.368194in}}%
\pgfpathlineto{\pgfqpoint{1.670639in}{1.374217in}}%
\pgfpathlineto{\pgfqpoint{1.664616in}{1.380240in}}%
\pgfpathlineto{\pgfqpoint{1.658593in}{1.386263in}}%
\pgfpathlineto{\pgfqpoint{1.652570in}{1.392286in}}%
\pgfpathlineto{\pgfqpoint{1.646547in}{1.398309in}}%
\pgfpathlineto{\pgfqpoint{1.640524in}{1.404332in}}%
\pgfpathlineto{\pgfqpoint{1.634500in}{1.410355in}}%
\pgfpathlineto{\pgfqpoint{1.628477in}{1.416378in}}%
\pgfpathlineto{\pgfqpoint{1.622454in}{1.422401in}}%
\pgfpathlineto{\pgfqpoint{1.616431in}{1.428424in}}%
\pgfpathlineto{\pgfqpoint{1.610408in}{1.434447in}}%
\pgfpathlineto{\pgfqpoint{1.604385in}{1.440470in}}%
\pgfpathlineto{\pgfqpoint{1.598362in}{1.446493in}}%
\pgfpathlineto{\pgfqpoint{1.592339in}{1.452516in}}%
\pgfpathlineto{\pgfqpoint{1.586316in}{1.458539in}}%
\pgfpathlineto{\pgfqpoint{1.580293in}{1.464563in}}%
\pgfpathlineto{\pgfqpoint{1.574270in}{1.470586in}}%
\pgfpathlineto{\pgfqpoint{1.568247in}{1.476609in}}%
\pgfpathlineto{\pgfqpoint{1.562224in}{1.482632in}}%
\pgfpathlineto{\pgfqpoint{1.556201in}{1.488655in}}%
\pgfpathlineto{\pgfqpoint{1.550178in}{1.494678in}}%
\pgfpathlineto{\pgfqpoint{1.544155in}{1.500701in}}%
\pgfpathlineto{\pgfqpoint{1.538132in}{1.506724in}}%
\pgfpathlineto{\pgfqpoint{1.532108in}{1.512747in}}%
\pgfpathlineto{\pgfqpoint{1.526085in}{1.518770in}}%
\pgfpathlineto{\pgfqpoint{1.520062in}{1.524793in}}%
\pgfpathlineto{\pgfqpoint{1.514039in}{1.530816in}}%
\pgfpathlineto{\pgfqpoint{1.508016in}{1.536839in}}%
\pgfpathlineto{\pgfqpoint{1.501993in}{1.542862in}}%
\pgfpathlineto{\pgfqpoint{1.495970in}{1.548885in}}%
\pgfpathlineto{\pgfqpoint{1.489947in}{1.554908in}}%
\pgfpathlineto{\pgfqpoint{1.483924in}{1.560931in}}%
\pgfpathlineto{\pgfqpoint{1.477901in}{1.566955in}}%
\pgfpathlineto{\pgfqpoint{1.471878in}{1.572978in}}%
\pgfpathlineto{\pgfqpoint{1.465855in}{1.579001in}}%
\pgfpathlineto{\pgfqpoint{1.459832in}{1.585024in}}%
\pgfpathlineto{\pgfqpoint{1.453809in}{1.591047in}}%
\pgfpathlineto{\pgfqpoint{1.447786in}{1.597070in}}%
\pgfpathlineto{\pgfqpoint{1.441763in}{1.603093in}}%
\pgfpathlineto{\pgfqpoint{1.435740in}{1.609116in}}%
\pgfpathlineto{\pgfqpoint{1.429716in}{1.615139in}}%
\pgfpathlineto{\pgfqpoint{1.423693in}{1.621162in}}%
\pgfpathlineto{\pgfqpoint{1.417670in}{1.627185in}}%
\pgfpathlineto{\pgfqpoint{1.411647in}{1.633208in}}%
\pgfpathlineto{\pgfqpoint{1.405624in}{1.639231in}}%
\pgfpathlineto{\pgfqpoint{1.399601in}{1.645254in}}%
\pgfpathlineto{\pgfqpoint{1.393578in}{1.651277in}}%
\pgfpathlineto{\pgfqpoint{1.387555in}{1.657300in}}%
\pgfpathlineto{\pgfqpoint{1.381532in}{1.663323in}}%
\pgfpathlineto{\pgfqpoint{1.375509in}{1.669347in}}%
\pgfpathlineto{\pgfqpoint{1.369486in}{1.675370in}}%
\pgfpathlineto{\pgfqpoint{1.363463in}{1.681393in}}%
\pgfpathlineto{\pgfqpoint{1.357440in}{1.687416in}}%
\pgfpathlineto{\pgfqpoint{1.351417in}{1.693439in}}%
\pgfpathlineto{\pgfqpoint{1.345394in}{1.699462in}}%
\pgfpathlineto{\pgfqpoint{1.339371in}{1.705485in}}%
\pgfpathlineto{\pgfqpoint{1.333348in}{1.711508in}}%
\pgfpathlineto{\pgfqpoint{1.327325in}{1.717531in}}%
\pgfpathlineto{\pgfqpoint{1.321301in}{1.723554in}}%
\pgfpathlineto{\pgfqpoint{1.315278in}{1.729577in}}%
\pgfpathlineto{\pgfqpoint{1.309255in}{1.735600in}}%
\pgfpathlineto{\pgfqpoint{1.303232in}{1.741623in}}%
\pgfpathlineto{\pgfqpoint{1.297209in}{1.747646in}}%
\pgfpathlineto{\pgfqpoint{1.291186in}{1.753669in}}%
\pgfpathlineto{\pgfqpoint{1.285163in}{1.759692in}}%
\pgfpathlineto{\pgfqpoint{1.279140in}{1.765715in}}%
\pgfpathlineto{\pgfqpoint{1.273117in}{1.771738in}}%
\pgfpathlineto{\pgfqpoint{1.267094in}{1.777762in}}%
\pgfpathlineto{\pgfqpoint{1.261071in}{1.783785in}}%
\pgfpathlineto{\pgfqpoint{1.255048in}{1.789808in}}%
\pgfpathlineto{\pgfqpoint{1.249025in}{1.795831in}}%
\pgfpathlineto{\pgfqpoint{1.243002in}{1.801854in}}%
\pgfpathlineto{\pgfqpoint{1.236979in}{1.807877in}}%
\pgfpathlineto{\pgfqpoint{1.230956in}{1.813900in}}%
\pgfpathlineto{\pgfqpoint{1.224933in}{1.819923in}}%
\pgfpathlineto{\pgfqpoint{1.218909in}{1.825946in}}%
\pgfpathlineto{\pgfqpoint{1.212886in}{1.831969in}}%
\pgfpathlineto{\pgfqpoint{1.206863in}{1.837992in}}%
\pgfpathlineto{\pgfqpoint{1.200840in}{1.844015in}}%
\pgfpathlineto{\pgfqpoint{1.194817in}{1.850038in}}%
\pgfpathlineto{\pgfqpoint{1.188794in}{1.856061in}}%
\pgfpathlineto{\pgfqpoint{1.182771in}{1.862084in}}%
\pgfpathlineto{\pgfqpoint{1.176748in}{1.868107in}}%
\pgfpathlineto{\pgfqpoint{1.170725in}{1.874130in}}%
\pgfpathlineto{\pgfqpoint{1.164702in}{1.880154in}}%
\pgfpathlineto{\pgfqpoint{1.158679in}{1.886177in}}%
\pgfpathlineto{\pgfqpoint{1.152656in}{1.892200in}}%
\pgfpathlineto{\pgfqpoint{1.146633in}{1.898223in}}%
\pgfpathlineto{\pgfqpoint{1.140610in}{1.904246in}}%
\pgfpathlineto{\pgfqpoint{1.134587in}{1.910269in}}%
\pgfpathlineto{\pgfqpoint{1.128564in}{1.916292in}}%
\pgfpathlineto{\pgfqpoint{1.122541in}{1.922315in}}%
\pgfpathlineto{\pgfqpoint{1.116517in}{1.928338in}}%
\pgfpathlineto{\pgfqpoint{1.110494in}{1.934361in}}%
\pgfpathlineto{\pgfqpoint{1.104471in}{1.940384in}}%
\pgfpathlineto{\pgfqpoint{1.098448in}{1.946407in}}%
\pgfpathlineto{\pgfqpoint{1.092425in}{1.952430in}}%
\pgfpathlineto{\pgfqpoint{1.086402in}{1.958453in}}%
\pgfpathlineto{\pgfqpoint{1.080379in}{1.964476in}}%
\pgfpathlineto{\pgfqpoint{1.074356in}{1.970499in}}%
\pgfpathlineto{\pgfqpoint{1.062310in}{1.970499in}}%
\pgfpathlineto{\pgfqpoint{1.056287in}{1.964476in}}%
\pgfpathlineto{\pgfqpoint{1.050264in}{1.958453in}}%
\pgfpathlineto{\pgfqpoint{1.044241in}{1.952430in}}%
\pgfpathlineto{\pgfqpoint{1.038218in}{1.946407in}}%
\pgfpathlineto{\pgfqpoint{1.032195in}{1.940384in}}%
\pgfpathlineto{\pgfqpoint{1.026172in}{1.934361in}}%
\pgfpathlineto{\pgfqpoint{1.020149in}{1.928338in}}%
\pgfpathlineto{\pgfqpoint{1.014126in}{1.922315in}}%
\pgfpathlineto{\pgfqpoint{1.008102in}{1.916292in}}%
\pgfpathlineto{\pgfqpoint{1.002079in}{1.910269in}}%
\pgfpathlineto{\pgfqpoint{0.996056in}{1.904246in}}%
\pgfpathlineto{\pgfqpoint{0.990033in}{1.898223in}}%
\pgfpathlineto{\pgfqpoint{0.984010in}{1.892200in}}%
\pgfpathlineto{\pgfqpoint{0.977987in}{1.886177in}}%
\pgfpathlineto{\pgfqpoint{0.971964in}{1.880154in}}%
\pgfpathlineto{\pgfqpoint{0.965941in}{1.874130in}}%
\pgfpathlineto{\pgfqpoint{0.959918in}{1.868107in}}%
\pgfpathlineto{\pgfqpoint{0.953895in}{1.862084in}}%
\pgfpathlineto{\pgfqpoint{0.947872in}{1.856061in}}%
\pgfpathlineto{\pgfqpoint{0.941849in}{1.850038in}}%
\pgfpathlineto{\pgfqpoint{0.935826in}{1.844015in}}%
\pgfpathlineto{\pgfqpoint{0.929803in}{1.837992in}}%
\pgfpathlineto{\pgfqpoint{0.923780in}{1.831969in}}%
\pgfpathlineto{\pgfqpoint{0.917757in}{1.825946in}}%
\pgfpathlineto{\pgfqpoint{0.911734in}{1.819923in}}%
\pgfpathlineto{\pgfqpoint{0.905710in}{1.813900in}}%
\pgfpathlineto{\pgfqpoint{0.899687in}{1.807877in}}%
\pgfpathlineto{\pgfqpoint{0.893664in}{1.801854in}}%
\pgfpathlineto{\pgfqpoint{0.887641in}{1.795831in}}%
\pgfpathlineto{\pgfqpoint{0.881618in}{1.789808in}}%
\pgfpathlineto{\pgfqpoint{0.875595in}{1.783785in}}%
\pgfpathlineto{\pgfqpoint{0.869572in}{1.777762in}}%
\pgfpathlineto{\pgfqpoint{0.863549in}{1.771738in}}%
\pgfpathlineto{\pgfqpoint{0.857526in}{1.765715in}}%
\pgfpathlineto{\pgfqpoint{0.851503in}{1.759692in}}%
\pgfpathlineto{\pgfqpoint{0.845480in}{1.753669in}}%
\pgfpathlineto{\pgfqpoint{0.839457in}{1.747646in}}%
\pgfpathlineto{\pgfqpoint{0.833434in}{1.741623in}}%
\pgfpathlineto{\pgfqpoint{0.827411in}{1.735600in}}%
\pgfpathlineto{\pgfqpoint{0.821388in}{1.729577in}}%
\pgfpathlineto{\pgfqpoint{0.815365in}{1.723554in}}%
\pgfpathlineto{\pgfqpoint{0.809342in}{1.717531in}}%
\pgfpathlineto{\pgfqpoint{0.803318in}{1.711508in}}%
\pgfpathlineto{\pgfqpoint{0.797295in}{1.705485in}}%
\pgfpathlineto{\pgfqpoint{0.791272in}{1.699462in}}%
\pgfpathlineto{\pgfqpoint{0.785249in}{1.693439in}}%
\pgfpathlineto{\pgfqpoint{0.779226in}{1.687416in}}%
\pgfpathlineto{\pgfqpoint{0.773203in}{1.681393in}}%
\pgfpathlineto{\pgfqpoint{0.767180in}{1.675370in}}%
\pgfpathlineto{\pgfqpoint{0.761157in}{1.669347in}}%
\pgfpathlineto{\pgfqpoint{0.755134in}{1.663323in}}%
\pgfpathlineto{\pgfqpoint{0.749111in}{1.657300in}}%
\pgfpathlineto{\pgfqpoint{0.743088in}{1.651277in}}%
\pgfpathlineto{\pgfqpoint{0.737065in}{1.645254in}}%
\pgfpathlineto{\pgfqpoint{0.731042in}{1.639231in}}%
\pgfpathlineto{\pgfqpoint{0.725019in}{1.633208in}}%
\pgfpathlineto{\pgfqpoint{0.718996in}{1.627185in}}%
\pgfpathlineto{\pgfqpoint{0.712973in}{1.621162in}}%
\pgfpathlineto{\pgfqpoint{0.706950in}{1.615139in}}%
\pgfpathlineto{\pgfqpoint{0.700927in}{1.609116in}}%
\pgfpathlineto{\pgfqpoint{0.694903in}{1.603093in}}%
\pgfpathlineto{\pgfqpoint{0.688880in}{1.597070in}}%
\pgfpathlineto{\pgfqpoint{0.682857in}{1.591047in}}%
\pgfpathlineto{\pgfqpoint{0.676834in}{1.585024in}}%
\pgfpathlineto{\pgfqpoint{0.670811in}{1.579001in}}%
\pgfpathlineto{\pgfqpoint{0.664788in}{1.572978in}}%
\pgfpathlineto{\pgfqpoint{0.658765in}{1.566955in}}%
\pgfpathlineto{\pgfqpoint{0.652742in}{1.560931in}}%
\pgfpathlineto{\pgfqpoint{0.646719in}{1.554908in}}%
\pgfpathlineto{\pgfqpoint{0.640696in}{1.548885in}}%
\pgfpathlineto{\pgfqpoint{0.634673in}{1.542862in}}%
\pgfpathlineto{\pgfqpoint{0.628650in}{1.536839in}}%
\pgfpathlineto{\pgfqpoint{0.622627in}{1.530816in}}%
\pgfpathlineto{\pgfqpoint{0.616604in}{1.524793in}}%
\pgfpathlineto{\pgfqpoint{0.610581in}{1.518770in}}%
\pgfpathlineto{\pgfqpoint{0.604558in}{1.512747in}}%
\pgfpathlineto{\pgfqpoint{0.598535in}{1.506724in}}%
\pgfpathlineto{\pgfqpoint{0.592511in}{1.500701in}}%
\pgfpathlineto{\pgfqpoint{0.586488in}{1.494678in}}%
\pgfpathlineto{\pgfqpoint{0.580465in}{1.488655in}}%
\pgfpathlineto{\pgfqpoint{0.574442in}{1.482632in}}%
\pgfpathlineto{\pgfqpoint{0.568419in}{1.476609in}}%
\pgfpathlineto{\pgfqpoint{0.562396in}{1.470586in}}%
\pgfpathlineto{\pgfqpoint{0.556373in}{1.464563in}}%
\pgfpathlineto{\pgfqpoint{0.550350in}{1.458539in}}%
\pgfpathlineto{\pgfqpoint{0.544327in}{1.452516in}}%
\pgfpathlineto{\pgfqpoint{0.538304in}{1.446493in}}%
\pgfpathlineto{\pgfqpoint{0.532281in}{1.440470in}}%
\pgfpathlineto{\pgfqpoint{0.526258in}{1.434447in}}%
\pgfpathlineto{\pgfqpoint{0.520235in}{1.428424in}}%
\pgfpathlineto{\pgfqpoint{0.514212in}{1.422401in}}%
\pgfpathlineto{\pgfqpoint{0.508189in}{1.416378in}}%
\pgfpathlineto{\pgfqpoint{0.502166in}{1.410355in}}%
\pgfpathlineto{\pgfqpoint{0.496143in}{1.404332in}}%
\pgfpathlineto{\pgfqpoint{0.490119in}{1.398309in}}%
\pgfpathlineto{\pgfqpoint{0.484096in}{1.392286in}}%
\pgfpathlineto{\pgfqpoint{0.478073in}{1.386263in}}%
\pgfpathlineto{\pgfqpoint{0.472050in}{1.380240in}}%
\pgfpathlineto{\pgfqpoint{0.466027in}{1.374217in}}%
\pgfpathlineto{\pgfqpoint{0.460004in}{1.368194in}}%
\pgfpathlineto{\pgfqpoint{0.453981in}{1.362171in}}%
\pgfpathlineto{\pgfqpoint{0.447958in}{1.356148in}}%
\pgfpathlineto{\pgfqpoint{0.441935in}{1.350124in}}%
\pgfpathlineto{\pgfqpoint{0.435912in}{1.344101in}}%
\pgfpathlineto{\pgfqpoint{0.429889in}{1.338078in}}%
\pgfpathlineto{\pgfqpoint{0.423866in}{1.332055in}}%
\pgfpathlineto{\pgfqpoint{0.417843in}{1.326032in}}%
\pgfpathlineto{\pgfqpoint{0.411820in}{1.320009in}}%
\pgfpathlineto{\pgfqpoint{0.405797in}{1.313986in}}%
\pgfpathlineto{\pgfqpoint{0.399774in}{1.307963in}}%
\pgfpathlineto{\pgfqpoint{0.393751in}{1.301940in}}%
\pgfpathlineto{\pgfqpoint{0.387728in}{1.295917in}}%
\pgfpathlineto{\pgfqpoint{0.381704in}{1.289894in}}%
\pgfpathlineto{\pgfqpoint{0.375681in}{1.283871in}}%
\pgfpathlineto{\pgfqpoint{0.369658in}{1.277848in}}%
\pgfpathlineto{\pgfqpoint{0.363635in}{1.271825in}}%
\pgfpathlineto{\pgfqpoint{0.357612in}{1.265802in}}%
\pgfpathlineto{\pgfqpoint{0.351589in}{1.259779in}}%
\pgfpathlineto{\pgfqpoint{0.345566in}{1.253756in}}%
\pgfpathlineto{\pgfqpoint{0.339543in}{1.247732in}}%
\pgfpathlineto{\pgfqpoint{0.333520in}{1.241709in}}%
\pgfpathlineto{\pgfqpoint{0.327497in}{1.235686in}}%
\pgfpathlineto{\pgfqpoint{0.321474in}{1.229663in}}%
\pgfpathlineto{\pgfqpoint{0.315451in}{1.223640in}}%
\pgfpathlineto{\pgfqpoint{0.309428in}{1.217617in}}%
\pgfpathlineto{\pgfqpoint{0.303405in}{1.211594in}}%
\pgfpathlineto{\pgfqpoint{0.297382in}{1.205571in}}%
\pgfpathlineto{\pgfqpoint{0.291359in}{1.199548in}}%
\pgfpathlineto{\pgfqpoint{0.285336in}{1.193525in}}%
\pgfpathlineto{\pgfqpoint{0.279312in}{1.187502in}}%
\pgfpathlineto{\pgfqpoint{0.273289in}{1.181479in}}%
\pgfpathlineto{\pgfqpoint{0.267266in}{1.175456in}}%
\pgfpathlineto{\pgfqpoint{0.261243in}{1.169433in}}%
\pgfpathlineto{\pgfqpoint{0.255220in}{1.163410in}}%
\pgfpathlineto{\pgfqpoint{0.249197in}{1.157387in}}%
\pgfpathlineto{\pgfqpoint{0.243174in}{1.151364in}}%
\pgfpathlineto{\pgfqpoint{0.237151in}{1.145340in}}%
\pgfpathlineto{\pgfqpoint{0.231128in}{1.139317in}}%
\pgfpathlineto{\pgfqpoint{0.225105in}{1.133294in}}%
\pgfpathlineto{\pgfqpoint{0.219082in}{1.127271in}}%
\pgfpathlineto{\pgfqpoint{0.213059in}{1.121248in}}%
\pgfpathlineto{\pgfqpoint{0.207036in}{1.115225in}}%
\pgfpathlineto{\pgfqpoint{0.201013in}{1.109202in}}%
\pgfpathlineto{\pgfqpoint{0.194990in}{1.103179in}}%
\pgfpathlineto{\pgfqpoint{0.188967in}{1.097156in}}%
\pgfpathlineto{\pgfqpoint{0.182944in}{1.091133in}}%
\pgfpathlineto{\pgfqpoint{0.176920in}{1.085110in}}%
\pgfpathlineto{\pgfqpoint{0.176920in}{1.073064in}}%
\pgfpathlineto{\pgfqpoint{0.182944in}{1.067041in}}%
\pgfpathlineto{\pgfqpoint{0.188967in}{1.061018in}}%
\pgfpathlineto{\pgfqpoint{0.194990in}{1.054995in}}%
\pgfpathlineto{\pgfqpoint{0.201013in}{1.048972in}}%
\pgfpathlineto{\pgfqpoint{0.207036in}{1.042949in}}%
\pgfpathlineto{\pgfqpoint{0.213059in}{1.036925in}}%
\pgfpathlineto{\pgfqpoint{0.219082in}{1.030902in}}%
\pgfpathlineto{\pgfqpoint{0.225105in}{1.024879in}}%
\pgfpathlineto{\pgfqpoint{0.231128in}{1.018856in}}%
\pgfpathlineto{\pgfqpoint{0.237151in}{1.012833in}}%
\pgfpathlineto{\pgfqpoint{0.243174in}{1.006810in}}%
\pgfpathlineto{\pgfqpoint{0.249197in}{1.000787in}}%
\pgfpathlineto{\pgfqpoint{0.255220in}{0.994764in}}%
\pgfpathlineto{\pgfqpoint{0.261243in}{0.988741in}}%
\pgfpathlineto{\pgfqpoint{0.267266in}{0.982718in}}%
\pgfpathlineto{\pgfqpoint{0.273289in}{0.976695in}}%
\pgfpathlineto{\pgfqpoint{0.279312in}{0.970672in}}%
\pgfpathlineto{\pgfqpoint{0.285336in}{0.964649in}}%
\pgfpathlineto{\pgfqpoint{0.291359in}{0.958626in}}%
\pgfpathlineto{\pgfqpoint{0.297382in}{0.952603in}}%
\pgfpathlineto{\pgfqpoint{0.303405in}{0.946580in}}%
\pgfpathlineto{\pgfqpoint{0.309428in}{0.940557in}}%
\pgfpathlineto{\pgfqpoint{0.315451in}{0.934533in}}%
\pgfpathlineto{\pgfqpoint{0.321474in}{0.928510in}}%
\pgfpathlineto{\pgfqpoint{0.327497in}{0.922487in}}%
\pgfpathlineto{\pgfqpoint{0.333520in}{0.916464in}}%
\pgfpathlineto{\pgfqpoint{0.339543in}{0.910441in}}%
\pgfpathlineto{\pgfqpoint{0.345566in}{0.904418in}}%
\pgfpathlineto{\pgfqpoint{0.351589in}{0.898395in}}%
\pgfpathlineto{\pgfqpoint{0.357612in}{0.892372in}}%
\pgfpathlineto{\pgfqpoint{0.363635in}{0.886349in}}%
\pgfpathlineto{\pgfqpoint{0.369658in}{0.880326in}}%
\pgfpathlineto{\pgfqpoint{0.375681in}{0.874303in}}%
\pgfpathlineto{\pgfqpoint{0.381704in}{0.868280in}}%
\pgfpathlineto{\pgfqpoint{0.387728in}{0.862257in}}%
\pgfpathlineto{\pgfqpoint{0.393751in}{0.856234in}}%
\pgfpathlineto{\pgfqpoint{0.399774in}{0.850211in}}%
\pgfpathlineto{\pgfqpoint{0.405797in}{0.844188in}}%
\pgfpathlineto{\pgfqpoint{0.411820in}{0.838165in}}%
\pgfpathlineto{\pgfqpoint{0.417843in}{0.832141in}}%
\pgfpathlineto{\pgfqpoint{0.423866in}{0.826118in}}%
\pgfpathlineto{\pgfqpoint{0.429889in}{0.820095in}}%
\pgfpathlineto{\pgfqpoint{0.435912in}{0.814072in}}%
\pgfpathlineto{\pgfqpoint{0.441935in}{0.808049in}}%
\pgfpathlineto{\pgfqpoint{0.447958in}{0.802026in}}%
\pgfpathlineto{\pgfqpoint{0.453981in}{0.796003in}}%
\pgfpathlineto{\pgfqpoint{0.460004in}{0.789980in}}%
\pgfpathlineto{\pgfqpoint{0.466027in}{0.783957in}}%
\pgfpathlineto{\pgfqpoint{0.472050in}{0.777934in}}%
\pgfpathlineto{\pgfqpoint{0.478073in}{0.771911in}}%
\pgfpathlineto{\pgfqpoint{0.484096in}{0.765888in}}%
\pgfpathlineto{\pgfqpoint{0.490119in}{0.759865in}}%
\pgfpathlineto{\pgfqpoint{0.496143in}{0.753842in}}%
\pgfpathlineto{\pgfqpoint{0.502166in}{0.747819in}}%
\pgfpathlineto{\pgfqpoint{0.508189in}{0.741796in}}%
\pgfpathlineto{\pgfqpoint{0.514212in}{0.735773in}}%
\pgfpathlineto{\pgfqpoint{0.520235in}{0.729750in}}%
\pgfpathlineto{\pgfqpoint{0.526258in}{0.723726in}}%
\pgfpathlineto{\pgfqpoint{0.532281in}{0.717703in}}%
\pgfpathlineto{\pgfqpoint{0.538304in}{0.711680in}}%
\pgfpathlineto{\pgfqpoint{0.544327in}{0.705657in}}%
\pgfpathlineto{\pgfqpoint{0.550350in}{0.699634in}}%
\pgfpathlineto{\pgfqpoint{0.556373in}{0.693611in}}%
\pgfpathlineto{\pgfqpoint{0.562396in}{0.687588in}}%
\pgfpathlineto{\pgfqpoint{0.568419in}{0.681565in}}%
\pgfpathlineto{\pgfqpoint{0.574442in}{0.675542in}}%
\pgfpathlineto{\pgfqpoint{0.580465in}{0.669519in}}%
\pgfpathlineto{\pgfqpoint{0.586488in}{0.663496in}}%
\pgfpathlineto{\pgfqpoint{0.592511in}{0.657473in}}%
\pgfpathlineto{\pgfqpoint{0.598535in}{0.651450in}}%
\pgfpathlineto{\pgfqpoint{0.604558in}{0.645427in}}%
\pgfpathlineto{\pgfqpoint{0.610581in}{0.639404in}}%
\pgfpathlineto{\pgfqpoint{0.616604in}{0.633381in}}%
\pgfpathlineto{\pgfqpoint{0.622627in}{0.627358in}}%
\pgfpathlineto{\pgfqpoint{0.628650in}{0.621334in}}%
\pgfpathlineto{\pgfqpoint{0.634673in}{0.615311in}}%
\pgfpathlineto{\pgfqpoint{0.640696in}{0.609288in}}%
\pgfpathlineto{\pgfqpoint{0.646719in}{0.603265in}}%
\pgfpathlineto{\pgfqpoint{0.652742in}{0.597242in}}%
\pgfpathlineto{\pgfqpoint{0.658765in}{0.591219in}}%
\pgfpathlineto{\pgfqpoint{0.664788in}{0.585196in}}%
\pgfpathlineto{\pgfqpoint{0.670811in}{0.579173in}}%
\pgfpathlineto{\pgfqpoint{0.676834in}{0.573150in}}%
\pgfpathlineto{\pgfqpoint{0.682857in}{0.567127in}}%
\pgfpathlineto{\pgfqpoint{0.688880in}{0.561104in}}%
\pgfpathlineto{\pgfqpoint{0.694903in}{0.555081in}}%
\pgfpathlineto{\pgfqpoint{0.700927in}{0.549058in}}%
\pgfpathlineto{\pgfqpoint{0.706950in}{0.543035in}}%
\pgfpathlineto{\pgfqpoint{0.712973in}{0.537012in}}%
\pgfpathlineto{\pgfqpoint{0.718996in}{0.530989in}}%
\pgfpathlineto{\pgfqpoint{0.725019in}{0.524966in}}%
\pgfpathlineto{\pgfqpoint{0.731042in}{0.518942in}}%
\pgfpathlineto{\pgfqpoint{0.737065in}{0.512919in}}%
\pgfpathlineto{\pgfqpoint{0.743088in}{0.506896in}}%
\pgfpathlineto{\pgfqpoint{0.749111in}{0.500873in}}%
\pgfpathlineto{\pgfqpoint{0.755134in}{0.494850in}}%
\pgfpathlineto{\pgfqpoint{0.761157in}{0.488827in}}%
\pgfpathlineto{\pgfqpoint{0.767180in}{0.482804in}}%
\pgfpathlineto{\pgfqpoint{0.773203in}{0.476781in}}%
\pgfpathlineto{\pgfqpoint{0.779226in}{0.470758in}}%
\pgfpathlineto{\pgfqpoint{0.785249in}{0.464735in}}%
\pgfpathlineto{\pgfqpoint{0.791272in}{0.458712in}}%
\pgfpathlineto{\pgfqpoint{0.797295in}{0.452689in}}%
\pgfpathlineto{\pgfqpoint{0.803318in}{0.446666in}}%
\pgfpathlineto{\pgfqpoint{0.809342in}{0.440643in}}%
\pgfpathlineto{\pgfqpoint{0.815365in}{0.434620in}}%
\pgfpathlineto{\pgfqpoint{0.821388in}{0.428597in}}%
\pgfpathlineto{\pgfqpoint{0.827411in}{0.422574in}}%
\pgfpathlineto{\pgfqpoint{0.833434in}{0.416551in}}%
\pgfpathlineto{\pgfqpoint{0.839457in}{0.410527in}}%
\pgfpathlineto{\pgfqpoint{0.845480in}{0.404504in}}%
\pgfpathlineto{\pgfqpoint{0.851503in}{0.398481in}}%
\pgfpathlineto{\pgfqpoint{0.857526in}{0.392458in}}%
\pgfpathlineto{\pgfqpoint{0.863549in}{0.386435in}}%
\pgfpathlineto{\pgfqpoint{0.869572in}{0.380412in}}%
\pgfpathlineto{\pgfqpoint{0.875595in}{0.374389in}}%
\pgfpathlineto{\pgfqpoint{0.881618in}{0.368366in}}%
\pgfpathlineto{\pgfqpoint{0.887641in}{0.362343in}}%
\pgfpathlineto{\pgfqpoint{0.893664in}{0.356320in}}%
\pgfpathlineto{\pgfqpoint{0.899687in}{0.350297in}}%
\pgfpathlineto{\pgfqpoint{0.905710in}{0.344274in}}%
\pgfpathlineto{\pgfqpoint{0.911734in}{0.338251in}}%
\pgfpathlineto{\pgfqpoint{0.917757in}{0.332228in}}%
\pgfpathlineto{\pgfqpoint{0.923780in}{0.326205in}}%
\pgfpathlineto{\pgfqpoint{0.929803in}{0.320182in}}%
\pgfpathlineto{\pgfqpoint{0.935826in}{0.314159in}}%
\pgfpathlineto{\pgfqpoint{0.941849in}{0.308135in}}%
\pgfpathlineto{\pgfqpoint{0.947872in}{0.302112in}}%
\pgfpathlineto{\pgfqpoint{0.953895in}{0.296089in}}%
\pgfpathlineto{\pgfqpoint{0.959918in}{0.290066in}}%
\pgfpathlineto{\pgfqpoint{0.965941in}{0.284043in}}%
\pgfpathlineto{\pgfqpoint{0.971964in}{0.278020in}}%
\pgfpathlineto{\pgfqpoint{0.977987in}{0.271997in}}%
\pgfpathlineto{\pgfqpoint{0.984010in}{0.265974in}}%
\pgfpathlineto{\pgfqpoint{0.990033in}{0.259951in}}%
\pgfpathlineto{\pgfqpoint{0.996056in}{0.253928in}}%
\pgfpathlineto{\pgfqpoint{1.002079in}{0.247905in}}%
\pgfpathlineto{\pgfqpoint{1.008102in}{0.241882in}}%
\pgfpathlineto{\pgfqpoint{1.014126in}{0.235859in}}%
\pgfpathlineto{\pgfqpoint{1.020149in}{0.229836in}}%
\pgfpathlineto{\pgfqpoint{1.026172in}{0.223813in}}%
\pgfpathlineto{\pgfqpoint{1.032195in}{0.217790in}}%
\pgfpathlineto{\pgfqpoint{1.038218in}{0.211767in}}%
\pgfpathlineto{\pgfqpoint{1.044241in}{0.205743in}}%
\pgfpathlineto{\pgfqpoint{1.050264in}{0.199720in}}%
\pgfpathlineto{\pgfqpoint{1.056287in}{0.193697in}}%
\pgfpathclose%
\pgfpathmoveto{\pgfqpoint{1.056287in}{0.193697in}}%
\pgfpathlineto{\pgfqpoint{1.050264in}{0.199720in}}%
\pgfpathlineto{\pgfqpoint{1.044241in}{0.205743in}}%
\pgfpathlineto{\pgfqpoint{1.038218in}{0.211767in}}%
\pgfpathlineto{\pgfqpoint{1.032195in}{0.217790in}}%
\pgfpathlineto{\pgfqpoint{1.026172in}{0.223813in}}%
\pgfpathlineto{\pgfqpoint{1.020149in}{0.229836in}}%
\pgfpathlineto{\pgfqpoint{1.014126in}{0.235859in}}%
\pgfpathlineto{\pgfqpoint{1.008102in}{0.241882in}}%
\pgfpathlineto{\pgfqpoint{1.002079in}{0.247905in}}%
\pgfpathlineto{\pgfqpoint{0.996056in}{0.253928in}}%
\pgfpathlineto{\pgfqpoint{0.990033in}{0.259951in}}%
\pgfpathlineto{\pgfqpoint{0.984010in}{0.265974in}}%
\pgfpathlineto{\pgfqpoint{0.977987in}{0.271997in}}%
\pgfpathlineto{\pgfqpoint{0.971964in}{0.278020in}}%
\pgfpathlineto{\pgfqpoint{0.965941in}{0.284043in}}%
\pgfpathlineto{\pgfqpoint{0.959918in}{0.290066in}}%
\pgfpathlineto{\pgfqpoint{0.953895in}{0.296089in}}%
\pgfpathlineto{\pgfqpoint{0.947872in}{0.302112in}}%
\pgfpathlineto{\pgfqpoint{0.941849in}{0.308135in}}%
\pgfpathlineto{\pgfqpoint{0.935826in}{0.314159in}}%
\pgfpathlineto{\pgfqpoint{0.929803in}{0.320182in}}%
\pgfpathlineto{\pgfqpoint{0.923780in}{0.326205in}}%
\pgfpathlineto{\pgfqpoint{0.917757in}{0.332228in}}%
\pgfpathlineto{\pgfqpoint{0.911734in}{0.338251in}}%
\pgfpathlineto{\pgfqpoint{0.905710in}{0.344274in}}%
\pgfpathlineto{\pgfqpoint{0.899687in}{0.350297in}}%
\pgfpathlineto{\pgfqpoint{0.893664in}{0.356320in}}%
\pgfpathlineto{\pgfqpoint{0.887641in}{0.362343in}}%
\pgfpathlineto{\pgfqpoint{0.881618in}{0.368366in}}%
\pgfpathlineto{\pgfqpoint{0.875595in}{0.374389in}}%
\pgfpathlineto{\pgfqpoint{0.869572in}{0.380412in}}%
\pgfpathlineto{\pgfqpoint{0.863549in}{0.386435in}}%
\pgfpathlineto{\pgfqpoint{0.857526in}{0.392458in}}%
\pgfpathlineto{\pgfqpoint{0.851503in}{0.398481in}}%
\pgfpathlineto{\pgfqpoint{0.845480in}{0.404504in}}%
\pgfpathlineto{\pgfqpoint{0.839457in}{0.410527in}}%
\pgfpathlineto{\pgfqpoint{0.833434in}{0.416551in}}%
\pgfpathlineto{\pgfqpoint{0.827411in}{0.422574in}}%
\pgfpathlineto{\pgfqpoint{0.821388in}{0.428597in}}%
\pgfpathlineto{\pgfqpoint{0.815365in}{0.434620in}}%
\pgfpathlineto{\pgfqpoint{0.809342in}{0.440643in}}%
\pgfpathlineto{\pgfqpoint{0.803318in}{0.446666in}}%
\pgfpathlineto{\pgfqpoint{0.797295in}{0.452689in}}%
\pgfpathlineto{\pgfqpoint{0.791272in}{0.458712in}}%
\pgfpathlineto{\pgfqpoint{0.785249in}{0.464735in}}%
\pgfpathlineto{\pgfqpoint{0.779226in}{0.470758in}}%
\pgfpathlineto{\pgfqpoint{0.773203in}{0.476781in}}%
\pgfpathlineto{\pgfqpoint{0.767180in}{0.482804in}}%
\pgfpathlineto{\pgfqpoint{0.761157in}{0.488827in}}%
\pgfpathlineto{\pgfqpoint{0.755134in}{0.494850in}}%
\pgfpathlineto{\pgfqpoint{0.749111in}{0.500873in}}%
\pgfpathlineto{\pgfqpoint{0.743088in}{0.506896in}}%
\pgfpathlineto{\pgfqpoint{0.737065in}{0.512919in}}%
\pgfpathlineto{\pgfqpoint{0.731042in}{0.518942in}}%
\pgfpathlineto{\pgfqpoint{0.725019in}{0.524966in}}%
\pgfpathlineto{\pgfqpoint{0.718996in}{0.530989in}}%
\pgfpathlineto{\pgfqpoint{0.712973in}{0.537012in}}%
\pgfpathlineto{\pgfqpoint{0.706950in}{0.543035in}}%
\pgfpathlineto{\pgfqpoint{0.700927in}{0.549058in}}%
\pgfpathlineto{\pgfqpoint{0.694903in}{0.555081in}}%
\pgfpathlineto{\pgfqpoint{0.688880in}{0.561104in}}%
\pgfpathlineto{\pgfqpoint{0.682857in}{0.567127in}}%
\pgfpathlineto{\pgfqpoint{0.676834in}{0.573150in}}%
\pgfpathlineto{\pgfqpoint{0.670811in}{0.579173in}}%
\pgfpathlineto{\pgfqpoint{0.664788in}{0.585196in}}%
\pgfpathlineto{\pgfqpoint{0.658765in}{0.591219in}}%
\pgfpathlineto{\pgfqpoint{0.652742in}{0.597242in}}%
\pgfpathlineto{\pgfqpoint{0.646719in}{0.603265in}}%
\pgfpathlineto{\pgfqpoint{0.640696in}{0.609288in}}%
\pgfpathlineto{\pgfqpoint{0.634673in}{0.615311in}}%
\pgfpathlineto{\pgfqpoint{0.628650in}{0.621334in}}%
\pgfpathlineto{\pgfqpoint{0.622627in}{0.627358in}}%
\pgfpathlineto{\pgfqpoint{0.616604in}{0.633381in}}%
\pgfpathlineto{\pgfqpoint{0.610581in}{0.639404in}}%
\pgfpathlineto{\pgfqpoint{0.604558in}{0.645427in}}%
\pgfpathlineto{\pgfqpoint{0.598535in}{0.651450in}}%
\pgfpathlineto{\pgfqpoint{0.592511in}{0.657473in}}%
\pgfpathlineto{\pgfqpoint{0.586488in}{0.663496in}}%
\pgfpathlineto{\pgfqpoint{0.580465in}{0.669519in}}%
\pgfpathlineto{\pgfqpoint{0.574442in}{0.675542in}}%
\pgfpathlineto{\pgfqpoint{0.568419in}{0.681565in}}%
\pgfpathlineto{\pgfqpoint{0.562396in}{0.687588in}}%
\pgfpathlineto{\pgfqpoint{0.556373in}{0.693611in}}%
\pgfpathlineto{\pgfqpoint{0.550350in}{0.699634in}}%
\pgfpathlineto{\pgfqpoint{0.544327in}{0.705657in}}%
\pgfpathlineto{\pgfqpoint{0.538304in}{0.711680in}}%
\pgfpathlineto{\pgfqpoint{0.532281in}{0.717703in}}%
\pgfpathlineto{\pgfqpoint{0.526258in}{0.723726in}}%
\pgfpathlineto{\pgfqpoint{0.520235in}{0.729750in}}%
\pgfpathlineto{\pgfqpoint{0.514212in}{0.735773in}}%
\pgfpathlineto{\pgfqpoint{0.508189in}{0.741796in}}%
\pgfpathlineto{\pgfqpoint{0.502166in}{0.747819in}}%
\pgfpathlineto{\pgfqpoint{0.496143in}{0.753842in}}%
\pgfpathlineto{\pgfqpoint{0.490119in}{0.759865in}}%
\pgfpathlineto{\pgfqpoint{0.484096in}{0.765888in}}%
\pgfpathlineto{\pgfqpoint{0.478073in}{0.771911in}}%
\pgfpathlineto{\pgfqpoint{0.472050in}{0.777934in}}%
\pgfpathlineto{\pgfqpoint{0.466027in}{0.783957in}}%
\pgfpathlineto{\pgfqpoint{0.460004in}{0.789980in}}%
\pgfpathlineto{\pgfqpoint{0.453981in}{0.796003in}}%
\pgfpathlineto{\pgfqpoint{0.447958in}{0.802026in}}%
\pgfpathlineto{\pgfqpoint{0.441935in}{0.808049in}}%
\pgfpathlineto{\pgfqpoint{0.435912in}{0.814072in}}%
\pgfpathlineto{\pgfqpoint{0.429889in}{0.820095in}}%
\pgfpathlineto{\pgfqpoint{0.423866in}{0.826118in}}%
\pgfpathlineto{\pgfqpoint{0.417843in}{0.832141in}}%
\pgfpathlineto{\pgfqpoint{0.411820in}{0.838165in}}%
\pgfpathlineto{\pgfqpoint{0.405797in}{0.844188in}}%
\pgfpathlineto{\pgfqpoint{0.399774in}{0.850211in}}%
\pgfpathlineto{\pgfqpoint{0.393751in}{0.856234in}}%
\pgfpathlineto{\pgfqpoint{0.387728in}{0.862257in}}%
\pgfpathlineto{\pgfqpoint{0.381704in}{0.868280in}}%
\pgfpathlineto{\pgfqpoint{0.375681in}{0.874303in}}%
\pgfpathlineto{\pgfqpoint{0.369658in}{0.880326in}}%
\pgfpathlineto{\pgfqpoint{0.363635in}{0.886349in}}%
\pgfpathlineto{\pgfqpoint{0.357612in}{0.892372in}}%
\pgfpathlineto{\pgfqpoint{0.351589in}{0.898395in}}%
\pgfpathlineto{\pgfqpoint{0.345566in}{0.904418in}}%
\pgfpathlineto{\pgfqpoint{0.339543in}{0.910441in}}%
\pgfpathlineto{\pgfqpoint{0.333520in}{0.916464in}}%
\pgfpathlineto{\pgfqpoint{0.327497in}{0.922487in}}%
\pgfpathlineto{\pgfqpoint{0.321474in}{0.928510in}}%
\pgfpathlineto{\pgfqpoint{0.315451in}{0.934533in}}%
\pgfpathlineto{\pgfqpoint{0.309428in}{0.940557in}}%
\pgfpathlineto{\pgfqpoint{0.303405in}{0.946580in}}%
\pgfpathlineto{\pgfqpoint{0.297382in}{0.952603in}}%
\pgfpathlineto{\pgfqpoint{0.291359in}{0.958626in}}%
\pgfpathlineto{\pgfqpoint{0.285336in}{0.964649in}}%
\pgfpathlineto{\pgfqpoint{0.279312in}{0.970672in}}%
\pgfpathlineto{\pgfqpoint{0.273289in}{0.976695in}}%
\pgfpathlineto{\pgfqpoint{0.267266in}{0.982718in}}%
\pgfpathlineto{\pgfqpoint{0.261243in}{0.988741in}}%
\pgfpathlineto{\pgfqpoint{0.255220in}{0.994764in}}%
\pgfpathlineto{\pgfqpoint{0.249197in}{1.000787in}}%
\pgfpathlineto{\pgfqpoint{0.243174in}{1.006810in}}%
\pgfpathlineto{\pgfqpoint{0.237151in}{1.012833in}}%
\pgfpathlineto{\pgfqpoint{0.231128in}{1.018856in}}%
\pgfpathlineto{\pgfqpoint{0.225105in}{1.024879in}}%
\pgfpathlineto{\pgfqpoint{0.219082in}{1.030902in}}%
\pgfpathlineto{\pgfqpoint{0.213059in}{1.036925in}}%
\pgfpathlineto{\pgfqpoint{0.207036in}{1.042949in}}%
\pgfpathlineto{\pgfqpoint{0.201013in}{1.048972in}}%
\pgfpathlineto{\pgfqpoint{0.194990in}{1.054995in}}%
\pgfpathlineto{\pgfqpoint{0.188967in}{1.061018in}}%
\pgfpathlineto{\pgfqpoint{0.182944in}{1.067041in}}%
\pgfpathlineto{\pgfqpoint{0.176920in}{1.073064in}}%
\pgfpathlineto{\pgfqpoint{0.176920in}{1.085110in}}%
\pgfpathlineto{\pgfqpoint{0.182944in}{1.091133in}}%
\pgfpathlineto{\pgfqpoint{0.188967in}{1.097156in}}%
\pgfpathlineto{\pgfqpoint{0.194990in}{1.103179in}}%
\pgfpathlineto{\pgfqpoint{0.201013in}{1.109202in}}%
\pgfpathlineto{\pgfqpoint{0.207036in}{1.115225in}}%
\pgfpathlineto{\pgfqpoint{0.213059in}{1.121248in}}%
\pgfpathlineto{\pgfqpoint{0.219082in}{1.127271in}}%
\pgfpathlineto{\pgfqpoint{0.225105in}{1.133294in}}%
\pgfpathlineto{\pgfqpoint{0.231128in}{1.139317in}}%
\pgfpathlineto{\pgfqpoint{0.237151in}{1.145340in}}%
\pgfpathlineto{\pgfqpoint{0.243174in}{1.151364in}}%
\pgfpathlineto{\pgfqpoint{0.249197in}{1.157387in}}%
\pgfpathlineto{\pgfqpoint{0.255220in}{1.163410in}}%
\pgfpathlineto{\pgfqpoint{0.261243in}{1.169433in}}%
\pgfpathlineto{\pgfqpoint{0.267266in}{1.175456in}}%
\pgfpathlineto{\pgfqpoint{0.273289in}{1.181479in}}%
\pgfpathlineto{\pgfqpoint{0.279312in}{1.187502in}}%
\pgfpathlineto{\pgfqpoint{0.285336in}{1.193525in}}%
\pgfpathlineto{\pgfqpoint{0.291359in}{1.199548in}}%
\pgfpathlineto{\pgfqpoint{0.297382in}{1.205571in}}%
\pgfpathlineto{\pgfqpoint{0.303405in}{1.211594in}}%
\pgfpathlineto{\pgfqpoint{0.309428in}{1.217617in}}%
\pgfpathlineto{\pgfqpoint{0.315451in}{1.223640in}}%
\pgfpathlineto{\pgfqpoint{0.321474in}{1.229663in}}%
\pgfpathlineto{\pgfqpoint{0.327497in}{1.235686in}}%
\pgfpathlineto{\pgfqpoint{0.333520in}{1.241709in}}%
\pgfpathlineto{\pgfqpoint{0.339543in}{1.247732in}}%
\pgfpathlineto{\pgfqpoint{0.345566in}{1.253756in}}%
\pgfpathlineto{\pgfqpoint{0.351589in}{1.259779in}}%
\pgfpathlineto{\pgfqpoint{0.357612in}{1.265802in}}%
\pgfpathlineto{\pgfqpoint{0.363635in}{1.271825in}}%
\pgfpathlineto{\pgfqpoint{0.369658in}{1.277848in}}%
\pgfpathlineto{\pgfqpoint{0.375681in}{1.283871in}}%
\pgfpathlineto{\pgfqpoint{0.381704in}{1.289894in}}%
\pgfpathlineto{\pgfqpoint{0.387728in}{1.295917in}}%
\pgfpathlineto{\pgfqpoint{0.393751in}{1.301940in}}%
\pgfpathlineto{\pgfqpoint{0.399774in}{1.307963in}}%
\pgfpathlineto{\pgfqpoint{0.405797in}{1.313986in}}%
\pgfpathlineto{\pgfqpoint{0.411820in}{1.320009in}}%
\pgfpathlineto{\pgfqpoint{0.417843in}{1.326032in}}%
\pgfpathlineto{\pgfqpoint{0.423866in}{1.332055in}}%
\pgfpathlineto{\pgfqpoint{0.429889in}{1.338078in}}%
\pgfpathlineto{\pgfqpoint{0.435912in}{1.344101in}}%
\pgfpathlineto{\pgfqpoint{0.441935in}{1.350124in}}%
\pgfpathlineto{\pgfqpoint{0.447958in}{1.356148in}}%
\pgfpathlineto{\pgfqpoint{0.453981in}{1.362171in}}%
\pgfpathlineto{\pgfqpoint{0.460004in}{1.368194in}}%
\pgfpathlineto{\pgfqpoint{0.466027in}{1.374217in}}%
\pgfpathlineto{\pgfqpoint{0.472050in}{1.380240in}}%
\pgfpathlineto{\pgfqpoint{0.478073in}{1.386263in}}%
\pgfpathlineto{\pgfqpoint{0.484096in}{1.392286in}}%
\pgfpathlineto{\pgfqpoint{0.490119in}{1.398309in}}%
\pgfpathlineto{\pgfqpoint{0.496143in}{1.404332in}}%
\pgfpathlineto{\pgfqpoint{0.502166in}{1.410355in}}%
\pgfpathlineto{\pgfqpoint{0.508189in}{1.416378in}}%
\pgfpathlineto{\pgfqpoint{0.514212in}{1.422401in}}%
\pgfpathlineto{\pgfqpoint{0.520235in}{1.428424in}}%
\pgfpathlineto{\pgfqpoint{0.526258in}{1.434447in}}%
\pgfpathlineto{\pgfqpoint{0.532281in}{1.440470in}}%
\pgfpathlineto{\pgfqpoint{0.538304in}{1.446493in}}%
\pgfpathlineto{\pgfqpoint{0.544327in}{1.452516in}}%
\pgfpathlineto{\pgfqpoint{0.550350in}{1.458539in}}%
\pgfpathlineto{\pgfqpoint{0.556373in}{1.464563in}}%
\pgfpathlineto{\pgfqpoint{0.562396in}{1.470586in}}%
\pgfpathlineto{\pgfqpoint{0.568419in}{1.476609in}}%
\pgfpathlineto{\pgfqpoint{0.574442in}{1.482632in}}%
\pgfpathlineto{\pgfqpoint{0.580465in}{1.488655in}}%
\pgfpathlineto{\pgfqpoint{0.586488in}{1.494678in}}%
\pgfpathlineto{\pgfqpoint{0.592511in}{1.500701in}}%
\pgfpathlineto{\pgfqpoint{0.598535in}{1.506724in}}%
\pgfpathlineto{\pgfqpoint{0.604558in}{1.512747in}}%
\pgfpathlineto{\pgfqpoint{0.610581in}{1.518770in}}%
\pgfpathlineto{\pgfqpoint{0.616604in}{1.524793in}}%
\pgfpathlineto{\pgfqpoint{0.622627in}{1.530816in}}%
\pgfpathlineto{\pgfqpoint{0.628650in}{1.536839in}}%
\pgfpathlineto{\pgfqpoint{0.634673in}{1.542862in}}%
\pgfpathlineto{\pgfqpoint{0.640696in}{1.548885in}}%
\pgfpathlineto{\pgfqpoint{0.646719in}{1.554908in}}%
\pgfpathlineto{\pgfqpoint{0.652742in}{1.560931in}}%
\pgfpathlineto{\pgfqpoint{0.658765in}{1.566955in}}%
\pgfpathlineto{\pgfqpoint{0.664788in}{1.572978in}}%
\pgfpathlineto{\pgfqpoint{0.670811in}{1.579001in}}%
\pgfpathlineto{\pgfqpoint{0.676834in}{1.585024in}}%
\pgfpathlineto{\pgfqpoint{0.682857in}{1.591047in}}%
\pgfpathlineto{\pgfqpoint{0.688880in}{1.597070in}}%
\pgfpathlineto{\pgfqpoint{0.694903in}{1.603093in}}%
\pgfpathlineto{\pgfqpoint{0.700927in}{1.609116in}}%
\pgfpathlineto{\pgfqpoint{0.706950in}{1.615139in}}%
\pgfpathlineto{\pgfqpoint{0.712973in}{1.621162in}}%
\pgfpathlineto{\pgfqpoint{0.718996in}{1.627185in}}%
\pgfpathlineto{\pgfqpoint{0.725019in}{1.633208in}}%
\pgfpathlineto{\pgfqpoint{0.731042in}{1.639231in}}%
\pgfpathlineto{\pgfqpoint{0.737065in}{1.645254in}}%
\pgfpathlineto{\pgfqpoint{0.743088in}{1.651277in}}%
\pgfpathlineto{\pgfqpoint{0.749111in}{1.657300in}}%
\pgfpathlineto{\pgfqpoint{0.755134in}{1.663323in}}%
\pgfpathlineto{\pgfqpoint{0.761157in}{1.669347in}}%
\pgfpathlineto{\pgfqpoint{0.767180in}{1.675370in}}%
\pgfpathlineto{\pgfqpoint{0.773203in}{1.681393in}}%
\pgfpathlineto{\pgfqpoint{0.779226in}{1.687416in}}%
\pgfpathlineto{\pgfqpoint{0.785249in}{1.693439in}}%
\pgfpathlineto{\pgfqpoint{0.791272in}{1.699462in}}%
\pgfpathlineto{\pgfqpoint{0.797295in}{1.705485in}}%
\pgfpathlineto{\pgfqpoint{0.803318in}{1.711508in}}%
\pgfpathlineto{\pgfqpoint{0.809342in}{1.717531in}}%
\pgfpathlineto{\pgfqpoint{0.815365in}{1.723554in}}%
\pgfpathlineto{\pgfqpoint{0.821388in}{1.729577in}}%
\pgfpathlineto{\pgfqpoint{0.827411in}{1.735600in}}%
\pgfpathlineto{\pgfqpoint{0.833434in}{1.741623in}}%
\pgfpathlineto{\pgfqpoint{0.839457in}{1.747646in}}%
\pgfpathlineto{\pgfqpoint{0.845480in}{1.753669in}}%
\pgfpathlineto{\pgfqpoint{0.851503in}{1.759692in}}%
\pgfpathlineto{\pgfqpoint{0.857526in}{1.765715in}}%
\pgfpathlineto{\pgfqpoint{0.863549in}{1.771738in}}%
\pgfpathlineto{\pgfqpoint{0.869572in}{1.777762in}}%
\pgfpathlineto{\pgfqpoint{0.875595in}{1.783785in}}%
\pgfpathlineto{\pgfqpoint{0.881618in}{1.789808in}}%
\pgfpathlineto{\pgfqpoint{0.887641in}{1.795831in}}%
\pgfpathlineto{\pgfqpoint{0.893664in}{1.801854in}}%
\pgfpathlineto{\pgfqpoint{0.899687in}{1.807877in}}%
\pgfpathlineto{\pgfqpoint{0.905710in}{1.813900in}}%
\pgfpathlineto{\pgfqpoint{0.911734in}{1.819923in}}%
\pgfpathlineto{\pgfqpoint{0.917757in}{1.825946in}}%
\pgfpathlineto{\pgfqpoint{0.923780in}{1.831969in}}%
\pgfpathlineto{\pgfqpoint{0.929803in}{1.837992in}}%
\pgfpathlineto{\pgfqpoint{0.935826in}{1.844015in}}%
\pgfpathlineto{\pgfqpoint{0.941849in}{1.850038in}}%
\pgfpathlineto{\pgfqpoint{0.947872in}{1.856061in}}%
\pgfpathlineto{\pgfqpoint{0.953895in}{1.862084in}}%
\pgfpathlineto{\pgfqpoint{0.959918in}{1.868107in}}%
\pgfpathlineto{\pgfqpoint{0.965941in}{1.874130in}}%
\pgfpathlineto{\pgfqpoint{0.971964in}{1.880154in}}%
\pgfpathlineto{\pgfqpoint{0.977987in}{1.886177in}}%
\pgfpathlineto{\pgfqpoint{0.984010in}{1.892200in}}%
\pgfpathlineto{\pgfqpoint{0.990033in}{1.898223in}}%
\pgfpathlineto{\pgfqpoint{0.996056in}{1.904246in}}%
\pgfpathlineto{\pgfqpoint{1.002079in}{1.910269in}}%
\pgfpathlineto{\pgfqpoint{1.008102in}{1.916292in}}%
\pgfpathlineto{\pgfqpoint{1.014126in}{1.922315in}}%
\pgfpathlineto{\pgfqpoint{1.020149in}{1.928338in}}%
\pgfpathlineto{\pgfqpoint{1.026172in}{1.934361in}}%
\pgfpathlineto{\pgfqpoint{1.032195in}{1.940384in}}%
\pgfpathlineto{\pgfqpoint{1.038218in}{1.946407in}}%
\pgfpathlineto{\pgfqpoint{1.044241in}{1.952430in}}%
\pgfpathlineto{\pgfqpoint{1.050264in}{1.958453in}}%
\pgfpathlineto{\pgfqpoint{1.056287in}{1.964476in}}%
\pgfpathlineto{\pgfqpoint{1.062310in}{1.970499in}}%
\pgfpathlineto{\pgfqpoint{1.074356in}{1.970499in}}%
\pgfpathlineto{\pgfqpoint{1.080379in}{1.964476in}}%
\pgfpathlineto{\pgfqpoint{1.086402in}{1.958453in}}%
\pgfpathlineto{\pgfqpoint{1.092425in}{1.952430in}}%
\pgfpathlineto{\pgfqpoint{1.098448in}{1.946407in}}%
\pgfpathlineto{\pgfqpoint{1.104471in}{1.940384in}}%
\pgfpathlineto{\pgfqpoint{1.110494in}{1.934361in}}%
\pgfpathlineto{\pgfqpoint{1.116517in}{1.928338in}}%
\pgfpathlineto{\pgfqpoint{1.122541in}{1.922315in}}%
\pgfpathlineto{\pgfqpoint{1.128564in}{1.916292in}}%
\pgfpathlineto{\pgfqpoint{1.134587in}{1.910269in}}%
\pgfpathlineto{\pgfqpoint{1.140610in}{1.904246in}}%
\pgfpathlineto{\pgfqpoint{1.146633in}{1.898223in}}%
\pgfpathlineto{\pgfqpoint{1.152656in}{1.892200in}}%
\pgfpathlineto{\pgfqpoint{1.158679in}{1.886177in}}%
\pgfpathlineto{\pgfqpoint{1.164702in}{1.880154in}}%
\pgfpathlineto{\pgfqpoint{1.170725in}{1.874130in}}%
\pgfpathlineto{\pgfqpoint{1.176748in}{1.868107in}}%
\pgfpathlineto{\pgfqpoint{1.182771in}{1.862084in}}%
\pgfpathlineto{\pgfqpoint{1.188794in}{1.856061in}}%
\pgfpathlineto{\pgfqpoint{1.194817in}{1.850038in}}%
\pgfpathlineto{\pgfqpoint{1.200840in}{1.844015in}}%
\pgfpathlineto{\pgfqpoint{1.206863in}{1.837992in}}%
\pgfpathlineto{\pgfqpoint{1.212886in}{1.831969in}}%
\pgfpathlineto{\pgfqpoint{1.218909in}{1.825946in}}%
\pgfpathlineto{\pgfqpoint{1.224933in}{1.819923in}}%
\pgfpathlineto{\pgfqpoint{1.230956in}{1.813900in}}%
\pgfpathlineto{\pgfqpoint{1.236979in}{1.807877in}}%
\pgfpathlineto{\pgfqpoint{1.243002in}{1.801854in}}%
\pgfpathlineto{\pgfqpoint{1.249025in}{1.795831in}}%
\pgfpathlineto{\pgfqpoint{1.255048in}{1.789808in}}%
\pgfpathlineto{\pgfqpoint{1.261071in}{1.783785in}}%
\pgfpathlineto{\pgfqpoint{1.267094in}{1.777762in}}%
\pgfpathlineto{\pgfqpoint{1.273117in}{1.771738in}}%
\pgfpathlineto{\pgfqpoint{1.279140in}{1.765715in}}%
\pgfpathlineto{\pgfqpoint{1.285163in}{1.759692in}}%
\pgfpathlineto{\pgfqpoint{1.291186in}{1.753669in}}%
\pgfpathlineto{\pgfqpoint{1.297209in}{1.747646in}}%
\pgfpathlineto{\pgfqpoint{1.303232in}{1.741623in}}%
\pgfpathlineto{\pgfqpoint{1.309255in}{1.735600in}}%
\pgfpathlineto{\pgfqpoint{1.315278in}{1.729577in}}%
\pgfpathlineto{\pgfqpoint{1.321301in}{1.723554in}}%
\pgfpathlineto{\pgfqpoint{1.327325in}{1.717531in}}%
\pgfpathlineto{\pgfqpoint{1.333348in}{1.711508in}}%
\pgfpathlineto{\pgfqpoint{1.339371in}{1.705485in}}%
\pgfpathlineto{\pgfqpoint{1.345394in}{1.699462in}}%
\pgfpathlineto{\pgfqpoint{1.351417in}{1.693439in}}%
\pgfpathlineto{\pgfqpoint{1.357440in}{1.687416in}}%
\pgfpathlineto{\pgfqpoint{1.363463in}{1.681393in}}%
\pgfpathlineto{\pgfqpoint{1.369486in}{1.675370in}}%
\pgfpathlineto{\pgfqpoint{1.375509in}{1.669347in}}%
\pgfpathlineto{\pgfqpoint{1.381532in}{1.663323in}}%
\pgfpathlineto{\pgfqpoint{1.387555in}{1.657300in}}%
\pgfpathlineto{\pgfqpoint{1.393578in}{1.651277in}}%
\pgfpathlineto{\pgfqpoint{1.399601in}{1.645254in}}%
\pgfpathlineto{\pgfqpoint{1.405624in}{1.639231in}}%
\pgfpathlineto{\pgfqpoint{1.411647in}{1.633208in}}%
\pgfpathlineto{\pgfqpoint{1.417670in}{1.627185in}}%
\pgfpathlineto{\pgfqpoint{1.423693in}{1.621162in}}%
\pgfpathlineto{\pgfqpoint{1.429716in}{1.615139in}}%
\pgfpathlineto{\pgfqpoint{1.435740in}{1.609116in}}%
\pgfpathlineto{\pgfqpoint{1.441763in}{1.603093in}}%
\pgfpathlineto{\pgfqpoint{1.447786in}{1.597070in}}%
\pgfpathlineto{\pgfqpoint{1.453809in}{1.591047in}}%
\pgfpathlineto{\pgfqpoint{1.459832in}{1.585024in}}%
\pgfpathlineto{\pgfqpoint{1.465855in}{1.579001in}}%
\pgfpathlineto{\pgfqpoint{1.471878in}{1.572978in}}%
\pgfpathlineto{\pgfqpoint{1.477901in}{1.566955in}}%
\pgfpathlineto{\pgfqpoint{1.483924in}{1.560931in}}%
\pgfpathlineto{\pgfqpoint{1.489947in}{1.554908in}}%
\pgfpathlineto{\pgfqpoint{1.495970in}{1.548885in}}%
\pgfpathlineto{\pgfqpoint{1.501993in}{1.542862in}}%
\pgfpathlineto{\pgfqpoint{1.508016in}{1.536839in}}%
\pgfpathlineto{\pgfqpoint{1.514039in}{1.530816in}}%
\pgfpathlineto{\pgfqpoint{1.520062in}{1.524793in}}%
\pgfpathlineto{\pgfqpoint{1.526085in}{1.518770in}}%
\pgfpathlineto{\pgfqpoint{1.532108in}{1.512747in}}%
\pgfpathlineto{\pgfqpoint{1.538132in}{1.506724in}}%
\pgfpathlineto{\pgfqpoint{1.544155in}{1.500701in}}%
\pgfpathlineto{\pgfqpoint{1.550178in}{1.494678in}}%
\pgfpathlineto{\pgfqpoint{1.556201in}{1.488655in}}%
\pgfpathlineto{\pgfqpoint{1.562224in}{1.482632in}}%
\pgfpathlineto{\pgfqpoint{1.568247in}{1.476609in}}%
\pgfpathlineto{\pgfqpoint{1.574270in}{1.470586in}}%
\pgfpathlineto{\pgfqpoint{1.580293in}{1.464563in}}%
\pgfpathlineto{\pgfqpoint{1.586316in}{1.458539in}}%
\pgfpathlineto{\pgfqpoint{1.592339in}{1.452516in}}%
\pgfpathlineto{\pgfqpoint{1.598362in}{1.446493in}}%
\pgfpathlineto{\pgfqpoint{1.604385in}{1.440470in}}%
\pgfpathlineto{\pgfqpoint{1.610408in}{1.434447in}}%
\pgfpathlineto{\pgfqpoint{1.616431in}{1.428424in}}%
\pgfpathlineto{\pgfqpoint{1.622454in}{1.422401in}}%
\pgfpathlineto{\pgfqpoint{1.628477in}{1.416378in}}%
\pgfpathlineto{\pgfqpoint{1.634500in}{1.410355in}}%
\pgfpathlineto{\pgfqpoint{1.640524in}{1.404332in}}%
\pgfpathlineto{\pgfqpoint{1.646547in}{1.398309in}}%
\pgfpathlineto{\pgfqpoint{1.652570in}{1.392286in}}%
\pgfpathlineto{\pgfqpoint{1.658593in}{1.386263in}}%
\pgfpathlineto{\pgfqpoint{1.664616in}{1.380240in}}%
\pgfpathlineto{\pgfqpoint{1.670639in}{1.374217in}}%
\pgfpathlineto{\pgfqpoint{1.676662in}{1.368194in}}%
\pgfpathlineto{\pgfqpoint{1.682685in}{1.362171in}}%
\pgfpathlineto{\pgfqpoint{1.688708in}{1.356148in}}%
\pgfpathlineto{\pgfqpoint{1.694731in}{1.350124in}}%
\pgfpathlineto{\pgfqpoint{1.700754in}{1.344101in}}%
\pgfpathlineto{\pgfqpoint{1.706777in}{1.338078in}}%
\pgfpathlineto{\pgfqpoint{1.712800in}{1.332055in}}%
\pgfpathlineto{\pgfqpoint{1.718823in}{1.326032in}}%
\pgfpathlineto{\pgfqpoint{1.724846in}{1.320009in}}%
\pgfpathlineto{\pgfqpoint{1.730869in}{1.313986in}}%
\pgfpathlineto{\pgfqpoint{1.736892in}{1.307963in}}%
\pgfpathlineto{\pgfqpoint{1.742915in}{1.301940in}}%
\pgfpathlineto{\pgfqpoint{1.748939in}{1.295917in}}%
\pgfpathlineto{\pgfqpoint{1.754962in}{1.289894in}}%
\pgfpathlineto{\pgfqpoint{1.760985in}{1.283871in}}%
\pgfpathlineto{\pgfqpoint{1.767008in}{1.277848in}}%
\pgfpathlineto{\pgfqpoint{1.773031in}{1.271825in}}%
\pgfpathlineto{\pgfqpoint{1.779054in}{1.265802in}}%
\pgfpathlineto{\pgfqpoint{1.785077in}{1.259779in}}%
\pgfpathlineto{\pgfqpoint{1.791100in}{1.253756in}}%
\pgfpathlineto{\pgfqpoint{1.797123in}{1.247732in}}%
\pgfpathlineto{\pgfqpoint{1.803146in}{1.241709in}}%
\pgfpathlineto{\pgfqpoint{1.809169in}{1.235686in}}%
\pgfpathlineto{\pgfqpoint{1.815192in}{1.229663in}}%
\pgfpathlineto{\pgfqpoint{1.821215in}{1.223640in}}%
\pgfpathlineto{\pgfqpoint{1.827238in}{1.217617in}}%
\pgfpathlineto{\pgfqpoint{1.833261in}{1.211594in}}%
\pgfpathlineto{\pgfqpoint{1.839284in}{1.205571in}}%
\pgfpathlineto{\pgfqpoint{1.845307in}{1.199548in}}%
\pgfpathlineto{\pgfqpoint{1.851331in}{1.193525in}}%
\pgfpathlineto{\pgfqpoint{1.857354in}{1.187502in}}%
\pgfpathlineto{\pgfqpoint{1.863377in}{1.181479in}}%
\pgfpathlineto{\pgfqpoint{1.869400in}{1.175456in}}%
\pgfpathlineto{\pgfqpoint{1.875423in}{1.169433in}}%
\pgfpathlineto{\pgfqpoint{1.881446in}{1.163410in}}%
\pgfpathlineto{\pgfqpoint{1.887469in}{1.157387in}}%
\pgfpathlineto{\pgfqpoint{1.893492in}{1.151364in}}%
\pgfpathlineto{\pgfqpoint{1.899515in}{1.145340in}}%
\pgfpathlineto{\pgfqpoint{1.905538in}{1.139317in}}%
\pgfpathlineto{\pgfqpoint{1.911561in}{1.133294in}}%
\pgfpathlineto{\pgfqpoint{1.917584in}{1.127271in}}%
\pgfpathlineto{\pgfqpoint{1.923607in}{1.121248in}}%
\pgfpathlineto{\pgfqpoint{1.929630in}{1.115225in}}%
\pgfpathlineto{\pgfqpoint{1.935653in}{1.109202in}}%
\pgfpathlineto{\pgfqpoint{1.941676in}{1.103179in}}%
\pgfpathlineto{\pgfqpoint{1.947699in}{1.097156in}}%
\pgfpathlineto{\pgfqpoint{1.953723in}{1.091133in}}%
\pgfpathlineto{\pgfqpoint{1.959746in}{1.085110in}}%
\pgfpathlineto{\pgfqpoint{1.959746in}{1.073064in}}%
\pgfpathlineto{\pgfqpoint{1.953723in}{1.067041in}}%
\pgfpathlineto{\pgfqpoint{1.947699in}{1.061018in}}%
\pgfpathlineto{\pgfqpoint{1.941676in}{1.054995in}}%
\pgfpathlineto{\pgfqpoint{1.935653in}{1.048972in}}%
\pgfpathlineto{\pgfqpoint{1.929630in}{1.042949in}}%
\pgfpathlineto{\pgfqpoint{1.923607in}{1.036925in}}%
\pgfpathlineto{\pgfqpoint{1.917584in}{1.030902in}}%
\pgfpathlineto{\pgfqpoint{1.911561in}{1.024879in}}%
\pgfpathlineto{\pgfqpoint{1.905538in}{1.018856in}}%
\pgfpathlineto{\pgfqpoint{1.899515in}{1.012833in}}%
\pgfpathlineto{\pgfqpoint{1.893492in}{1.006810in}}%
\pgfpathlineto{\pgfqpoint{1.887469in}{1.000787in}}%
\pgfpathlineto{\pgfqpoint{1.881446in}{0.994764in}}%
\pgfpathlineto{\pgfqpoint{1.875423in}{0.988741in}}%
\pgfpathlineto{\pgfqpoint{1.869400in}{0.982718in}}%
\pgfpathlineto{\pgfqpoint{1.863377in}{0.976695in}}%
\pgfpathlineto{\pgfqpoint{1.857354in}{0.970672in}}%
\pgfpathlineto{\pgfqpoint{1.851331in}{0.964649in}}%
\pgfpathlineto{\pgfqpoint{1.845307in}{0.958626in}}%
\pgfpathlineto{\pgfqpoint{1.839284in}{0.952603in}}%
\pgfpathlineto{\pgfqpoint{1.833261in}{0.946580in}}%
\pgfpathlineto{\pgfqpoint{1.827238in}{0.940557in}}%
\pgfpathlineto{\pgfqpoint{1.821215in}{0.934533in}}%
\pgfpathlineto{\pgfqpoint{1.815192in}{0.928510in}}%
\pgfpathlineto{\pgfqpoint{1.809169in}{0.922487in}}%
\pgfpathlineto{\pgfqpoint{1.803146in}{0.916464in}}%
\pgfpathlineto{\pgfqpoint{1.797123in}{0.910441in}}%
\pgfpathlineto{\pgfqpoint{1.791100in}{0.904418in}}%
\pgfpathlineto{\pgfqpoint{1.785077in}{0.898395in}}%
\pgfpathlineto{\pgfqpoint{1.779054in}{0.892372in}}%
\pgfpathlineto{\pgfqpoint{1.773031in}{0.886349in}}%
\pgfpathlineto{\pgfqpoint{1.767008in}{0.880326in}}%
\pgfpathlineto{\pgfqpoint{1.760985in}{0.874303in}}%
\pgfpathlineto{\pgfqpoint{1.754962in}{0.868280in}}%
\pgfpathlineto{\pgfqpoint{1.748939in}{0.862257in}}%
\pgfpathlineto{\pgfqpoint{1.742915in}{0.856234in}}%
\pgfpathlineto{\pgfqpoint{1.736892in}{0.850211in}}%
\pgfpathlineto{\pgfqpoint{1.730869in}{0.844188in}}%
\pgfpathlineto{\pgfqpoint{1.724846in}{0.838165in}}%
\pgfpathlineto{\pgfqpoint{1.718823in}{0.832141in}}%
\pgfpathlineto{\pgfqpoint{1.712800in}{0.826118in}}%
\pgfpathlineto{\pgfqpoint{1.706777in}{0.820095in}}%
\pgfpathlineto{\pgfqpoint{1.700754in}{0.814072in}}%
\pgfpathlineto{\pgfqpoint{1.694731in}{0.808049in}}%
\pgfpathlineto{\pgfqpoint{1.688708in}{0.802026in}}%
\pgfpathlineto{\pgfqpoint{1.682685in}{0.796003in}}%
\pgfpathlineto{\pgfqpoint{1.676662in}{0.789980in}}%
\pgfpathlineto{\pgfqpoint{1.670639in}{0.783957in}}%
\pgfpathlineto{\pgfqpoint{1.664616in}{0.777934in}}%
\pgfpathlineto{\pgfqpoint{1.658593in}{0.771911in}}%
\pgfpathlineto{\pgfqpoint{1.652570in}{0.765888in}}%
\pgfpathlineto{\pgfqpoint{1.646547in}{0.759865in}}%
\pgfpathlineto{\pgfqpoint{1.640524in}{0.753842in}}%
\pgfpathlineto{\pgfqpoint{1.634500in}{0.747819in}}%
\pgfpathlineto{\pgfqpoint{1.628477in}{0.741796in}}%
\pgfpathlineto{\pgfqpoint{1.622454in}{0.735773in}}%
\pgfpathlineto{\pgfqpoint{1.616431in}{0.729750in}}%
\pgfpathlineto{\pgfqpoint{1.610408in}{0.723726in}}%
\pgfpathlineto{\pgfqpoint{1.604385in}{0.717703in}}%
\pgfpathlineto{\pgfqpoint{1.598362in}{0.711680in}}%
\pgfpathlineto{\pgfqpoint{1.592339in}{0.705657in}}%
\pgfpathlineto{\pgfqpoint{1.586316in}{0.699634in}}%
\pgfpathlineto{\pgfqpoint{1.580293in}{0.693611in}}%
\pgfpathlineto{\pgfqpoint{1.574270in}{0.687588in}}%
\pgfpathlineto{\pgfqpoint{1.568247in}{0.681565in}}%
\pgfpathlineto{\pgfqpoint{1.562224in}{0.675542in}}%
\pgfpathlineto{\pgfqpoint{1.556201in}{0.669519in}}%
\pgfpathlineto{\pgfqpoint{1.550178in}{0.663496in}}%
\pgfpathlineto{\pgfqpoint{1.544155in}{0.657473in}}%
\pgfpathlineto{\pgfqpoint{1.538132in}{0.651450in}}%
\pgfpathlineto{\pgfqpoint{1.532108in}{0.645427in}}%
\pgfpathlineto{\pgfqpoint{1.526085in}{0.639404in}}%
\pgfpathlineto{\pgfqpoint{1.520062in}{0.633381in}}%
\pgfpathlineto{\pgfqpoint{1.514039in}{0.627358in}}%
\pgfpathlineto{\pgfqpoint{1.508016in}{0.621334in}}%
\pgfpathlineto{\pgfqpoint{1.501993in}{0.615311in}}%
\pgfpathlineto{\pgfqpoint{1.495970in}{0.609288in}}%
\pgfpathlineto{\pgfqpoint{1.489947in}{0.603265in}}%
\pgfpathlineto{\pgfqpoint{1.483924in}{0.597242in}}%
\pgfpathlineto{\pgfqpoint{1.477901in}{0.591219in}}%
\pgfpathlineto{\pgfqpoint{1.471878in}{0.585196in}}%
\pgfpathlineto{\pgfqpoint{1.465855in}{0.579173in}}%
\pgfpathlineto{\pgfqpoint{1.459832in}{0.573150in}}%
\pgfpathlineto{\pgfqpoint{1.453809in}{0.567127in}}%
\pgfpathlineto{\pgfqpoint{1.447786in}{0.561104in}}%
\pgfpathlineto{\pgfqpoint{1.441763in}{0.555081in}}%
\pgfpathlineto{\pgfqpoint{1.435740in}{0.549058in}}%
\pgfpathlineto{\pgfqpoint{1.429716in}{0.543035in}}%
\pgfpathlineto{\pgfqpoint{1.423693in}{0.537012in}}%
\pgfpathlineto{\pgfqpoint{1.417670in}{0.530989in}}%
\pgfpathlineto{\pgfqpoint{1.411647in}{0.524966in}}%
\pgfpathlineto{\pgfqpoint{1.405624in}{0.518942in}}%
\pgfpathlineto{\pgfqpoint{1.399601in}{0.512919in}}%
\pgfpathlineto{\pgfqpoint{1.393578in}{0.506896in}}%
\pgfpathlineto{\pgfqpoint{1.387555in}{0.500873in}}%
\pgfpathlineto{\pgfqpoint{1.381532in}{0.494850in}}%
\pgfpathlineto{\pgfqpoint{1.375509in}{0.488827in}}%
\pgfpathlineto{\pgfqpoint{1.369486in}{0.482804in}}%
\pgfpathlineto{\pgfqpoint{1.363463in}{0.476781in}}%
\pgfpathlineto{\pgfqpoint{1.357440in}{0.470758in}}%
\pgfpathlineto{\pgfqpoint{1.351417in}{0.464735in}}%
\pgfpathlineto{\pgfqpoint{1.345394in}{0.458712in}}%
\pgfpathlineto{\pgfqpoint{1.339371in}{0.452689in}}%
\pgfpathlineto{\pgfqpoint{1.333348in}{0.446666in}}%
\pgfpathlineto{\pgfqpoint{1.327325in}{0.440643in}}%
\pgfpathlineto{\pgfqpoint{1.321301in}{0.434620in}}%
\pgfpathlineto{\pgfqpoint{1.315278in}{0.428597in}}%
\pgfpathlineto{\pgfqpoint{1.309255in}{0.422574in}}%
\pgfpathlineto{\pgfqpoint{1.303232in}{0.416551in}}%
\pgfpathlineto{\pgfqpoint{1.297209in}{0.410527in}}%
\pgfpathlineto{\pgfqpoint{1.291186in}{0.404504in}}%
\pgfpathlineto{\pgfqpoint{1.285163in}{0.398481in}}%
\pgfpathlineto{\pgfqpoint{1.279140in}{0.392458in}}%
\pgfpathlineto{\pgfqpoint{1.273117in}{0.386435in}}%
\pgfpathlineto{\pgfqpoint{1.267094in}{0.380412in}}%
\pgfpathlineto{\pgfqpoint{1.261071in}{0.374389in}}%
\pgfpathlineto{\pgfqpoint{1.255048in}{0.368366in}}%
\pgfpathlineto{\pgfqpoint{1.249025in}{0.362343in}}%
\pgfpathlineto{\pgfqpoint{1.243002in}{0.356320in}}%
\pgfpathlineto{\pgfqpoint{1.236979in}{0.350297in}}%
\pgfpathlineto{\pgfqpoint{1.230956in}{0.344274in}}%
\pgfpathlineto{\pgfqpoint{1.224933in}{0.338251in}}%
\pgfpathlineto{\pgfqpoint{1.218909in}{0.332228in}}%
\pgfpathlineto{\pgfqpoint{1.212886in}{0.326205in}}%
\pgfpathlineto{\pgfqpoint{1.206863in}{0.320182in}}%
\pgfpathlineto{\pgfqpoint{1.200840in}{0.314159in}}%
\pgfpathlineto{\pgfqpoint{1.194817in}{0.308135in}}%
\pgfpathlineto{\pgfqpoint{1.188794in}{0.302112in}}%
\pgfpathlineto{\pgfqpoint{1.182771in}{0.296089in}}%
\pgfpathlineto{\pgfqpoint{1.176748in}{0.290066in}}%
\pgfpathlineto{\pgfqpoint{1.170725in}{0.284043in}}%
\pgfpathlineto{\pgfqpoint{1.164702in}{0.278020in}}%
\pgfpathlineto{\pgfqpoint{1.158679in}{0.271997in}}%
\pgfpathlineto{\pgfqpoint{1.152656in}{0.265974in}}%
\pgfpathlineto{\pgfqpoint{1.146633in}{0.259951in}}%
\pgfpathlineto{\pgfqpoint{1.140610in}{0.253928in}}%
\pgfpathlineto{\pgfqpoint{1.134587in}{0.247905in}}%
\pgfpathlineto{\pgfqpoint{1.128564in}{0.241882in}}%
\pgfpathlineto{\pgfqpoint{1.122541in}{0.235859in}}%
\pgfpathlineto{\pgfqpoint{1.116517in}{0.229836in}}%
\pgfpathlineto{\pgfqpoint{1.110494in}{0.223813in}}%
\pgfpathlineto{\pgfqpoint{1.104471in}{0.217790in}}%
\pgfpathlineto{\pgfqpoint{1.098448in}{0.211767in}}%
\pgfpathlineto{\pgfqpoint{1.092425in}{0.205743in}}%
\pgfpathlineto{\pgfqpoint{1.086402in}{0.199720in}}%
\pgfpathlineto{\pgfqpoint{1.080379in}{0.193697in}}%
\pgfpathlineto{\pgfqpoint{1.074356in}{0.187674in}}%
\pgfpathlineto{\pgfqpoint{1.062310in}{0.187674in}}%
\pgfpathclose%
\pgfusepath{fill}%
\end{pgfscope}%
\begin{pgfscope}%
\pgfpathrectangle{\pgfqpoint{0.135000in}{0.145754in}}{\pgfqpoint{1.866666in}{1.866666in}} %
\pgfusepath{clip}%
\pgfsetbuttcap%
\pgfsetroundjoin%
\definecolor{currentfill}{rgb}{1.000000,1.000000,1.000000}%
\pgfsetfillcolor{currentfill}%
\pgfsetlinewidth{0.000000pt}%
\definecolor{currentstroke}{rgb}{0.000000,0.000000,0.000000}%
\pgfsetstrokecolor{currentstroke}%
\pgfsetdash{}{0pt}%
\pgfpathmoveto{\pgfqpoint{1.062310in}{0.187674in}}%
\pgfpathlineto{\pgfqpoint{1.074356in}{0.187674in}}%
\pgfpathlineto{\pgfqpoint{1.080379in}{0.193697in}}%
\pgfpathlineto{\pgfqpoint{1.086402in}{0.199720in}}%
\pgfpathlineto{\pgfqpoint{1.092425in}{0.205743in}}%
\pgfpathlineto{\pgfqpoint{1.098448in}{0.211767in}}%
\pgfpathlineto{\pgfqpoint{1.104471in}{0.217790in}}%
\pgfpathlineto{\pgfqpoint{1.110494in}{0.223813in}}%
\pgfpathlineto{\pgfqpoint{1.116517in}{0.229836in}}%
\pgfpathlineto{\pgfqpoint{1.122541in}{0.235859in}}%
\pgfpathlineto{\pgfqpoint{1.128564in}{0.241882in}}%
\pgfpathlineto{\pgfqpoint{1.134587in}{0.247905in}}%
\pgfpathlineto{\pgfqpoint{1.140610in}{0.253928in}}%
\pgfpathlineto{\pgfqpoint{1.146633in}{0.259951in}}%
\pgfpathlineto{\pgfqpoint{1.152656in}{0.265974in}}%
\pgfpathlineto{\pgfqpoint{1.158679in}{0.271997in}}%
\pgfpathlineto{\pgfqpoint{1.164702in}{0.278020in}}%
\pgfpathlineto{\pgfqpoint{1.170725in}{0.284043in}}%
\pgfpathlineto{\pgfqpoint{1.176748in}{0.290066in}}%
\pgfpathlineto{\pgfqpoint{1.182771in}{0.296089in}}%
\pgfpathlineto{\pgfqpoint{1.188794in}{0.302112in}}%
\pgfpathlineto{\pgfqpoint{1.194817in}{0.308135in}}%
\pgfpathlineto{\pgfqpoint{1.200840in}{0.314159in}}%
\pgfpathlineto{\pgfqpoint{1.206863in}{0.320182in}}%
\pgfpathlineto{\pgfqpoint{1.212886in}{0.326205in}}%
\pgfpathlineto{\pgfqpoint{1.218909in}{0.332228in}}%
\pgfpathlineto{\pgfqpoint{1.224933in}{0.338251in}}%
\pgfpathlineto{\pgfqpoint{1.230956in}{0.344274in}}%
\pgfpathlineto{\pgfqpoint{1.236979in}{0.350297in}}%
\pgfpathlineto{\pgfqpoint{1.243002in}{0.356320in}}%
\pgfpathlineto{\pgfqpoint{1.249025in}{0.362343in}}%
\pgfpathlineto{\pgfqpoint{1.255048in}{0.368366in}}%
\pgfpathlineto{\pgfqpoint{1.261071in}{0.374389in}}%
\pgfpathlineto{\pgfqpoint{1.267094in}{0.380412in}}%
\pgfpathlineto{\pgfqpoint{1.273117in}{0.386435in}}%
\pgfpathlineto{\pgfqpoint{1.279140in}{0.392458in}}%
\pgfpathlineto{\pgfqpoint{1.285163in}{0.398481in}}%
\pgfpathlineto{\pgfqpoint{1.291186in}{0.404504in}}%
\pgfpathlineto{\pgfqpoint{1.297209in}{0.410527in}}%
\pgfpathlineto{\pgfqpoint{1.303232in}{0.416551in}}%
\pgfpathlineto{\pgfqpoint{1.309255in}{0.422574in}}%
\pgfpathlineto{\pgfqpoint{1.315278in}{0.428597in}}%
\pgfpathlineto{\pgfqpoint{1.321301in}{0.434620in}}%
\pgfpathlineto{\pgfqpoint{1.327325in}{0.440643in}}%
\pgfpathlineto{\pgfqpoint{1.333348in}{0.446666in}}%
\pgfpathlineto{\pgfqpoint{1.339371in}{0.452689in}}%
\pgfpathlineto{\pgfqpoint{1.345394in}{0.458712in}}%
\pgfpathlineto{\pgfqpoint{1.351417in}{0.464735in}}%
\pgfpathlineto{\pgfqpoint{1.357440in}{0.470758in}}%
\pgfpathlineto{\pgfqpoint{1.363463in}{0.476781in}}%
\pgfpathlineto{\pgfqpoint{1.369486in}{0.482804in}}%
\pgfpathlineto{\pgfqpoint{1.375509in}{0.488827in}}%
\pgfpathlineto{\pgfqpoint{1.381532in}{0.494850in}}%
\pgfpathlineto{\pgfqpoint{1.387555in}{0.500873in}}%
\pgfpathlineto{\pgfqpoint{1.393578in}{0.506896in}}%
\pgfpathlineto{\pgfqpoint{1.399601in}{0.512919in}}%
\pgfpathlineto{\pgfqpoint{1.405624in}{0.518942in}}%
\pgfpathlineto{\pgfqpoint{1.411647in}{0.524966in}}%
\pgfpathlineto{\pgfqpoint{1.417670in}{0.530989in}}%
\pgfpathlineto{\pgfqpoint{1.423693in}{0.537012in}}%
\pgfpathlineto{\pgfqpoint{1.429716in}{0.543035in}}%
\pgfpathlineto{\pgfqpoint{1.435740in}{0.549058in}}%
\pgfpathlineto{\pgfqpoint{1.441763in}{0.555081in}}%
\pgfpathlineto{\pgfqpoint{1.447786in}{0.561104in}}%
\pgfpathlineto{\pgfqpoint{1.453809in}{0.567127in}}%
\pgfpathlineto{\pgfqpoint{1.459832in}{0.573150in}}%
\pgfpathlineto{\pgfqpoint{1.465855in}{0.579173in}}%
\pgfpathlineto{\pgfqpoint{1.471878in}{0.585196in}}%
\pgfpathlineto{\pgfqpoint{1.477901in}{0.591219in}}%
\pgfpathlineto{\pgfqpoint{1.483924in}{0.597242in}}%
\pgfpathlineto{\pgfqpoint{1.489947in}{0.603265in}}%
\pgfpathlineto{\pgfqpoint{1.495970in}{0.609288in}}%
\pgfpathlineto{\pgfqpoint{1.501993in}{0.615311in}}%
\pgfpathlineto{\pgfqpoint{1.508016in}{0.621334in}}%
\pgfpathlineto{\pgfqpoint{1.514039in}{0.627358in}}%
\pgfpathlineto{\pgfqpoint{1.520062in}{0.633381in}}%
\pgfpathlineto{\pgfqpoint{1.526085in}{0.639404in}}%
\pgfpathlineto{\pgfqpoint{1.532108in}{0.645427in}}%
\pgfpathlineto{\pgfqpoint{1.538132in}{0.651450in}}%
\pgfpathlineto{\pgfqpoint{1.544155in}{0.657473in}}%
\pgfpathlineto{\pgfqpoint{1.550178in}{0.663496in}}%
\pgfpathlineto{\pgfqpoint{1.556201in}{0.669519in}}%
\pgfpathlineto{\pgfqpoint{1.562224in}{0.675542in}}%
\pgfpathlineto{\pgfqpoint{1.568247in}{0.681565in}}%
\pgfpathlineto{\pgfqpoint{1.574270in}{0.687588in}}%
\pgfpathlineto{\pgfqpoint{1.580293in}{0.693611in}}%
\pgfpathlineto{\pgfqpoint{1.586316in}{0.699634in}}%
\pgfpathlineto{\pgfqpoint{1.592339in}{0.705657in}}%
\pgfpathlineto{\pgfqpoint{1.598362in}{0.711680in}}%
\pgfpathlineto{\pgfqpoint{1.604385in}{0.717703in}}%
\pgfpathlineto{\pgfqpoint{1.610408in}{0.723726in}}%
\pgfpathlineto{\pgfqpoint{1.616431in}{0.729750in}}%
\pgfpathlineto{\pgfqpoint{1.622454in}{0.735773in}}%
\pgfpathlineto{\pgfqpoint{1.628477in}{0.741796in}}%
\pgfpathlineto{\pgfqpoint{1.634500in}{0.747819in}}%
\pgfpathlineto{\pgfqpoint{1.640524in}{0.753842in}}%
\pgfpathlineto{\pgfqpoint{1.646547in}{0.759865in}}%
\pgfpathlineto{\pgfqpoint{1.652570in}{0.765888in}}%
\pgfpathlineto{\pgfqpoint{1.658593in}{0.771911in}}%
\pgfpathlineto{\pgfqpoint{1.664616in}{0.777934in}}%
\pgfpathlineto{\pgfqpoint{1.670639in}{0.783957in}}%
\pgfpathlineto{\pgfqpoint{1.676662in}{0.789980in}}%
\pgfpathlineto{\pgfqpoint{1.682685in}{0.796003in}}%
\pgfpathlineto{\pgfqpoint{1.688708in}{0.802026in}}%
\pgfpathlineto{\pgfqpoint{1.694731in}{0.808049in}}%
\pgfpathlineto{\pgfqpoint{1.700754in}{0.814072in}}%
\pgfpathlineto{\pgfqpoint{1.706777in}{0.820095in}}%
\pgfpathlineto{\pgfqpoint{1.712800in}{0.826118in}}%
\pgfpathlineto{\pgfqpoint{1.718823in}{0.832141in}}%
\pgfpathlineto{\pgfqpoint{1.724846in}{0.838165in}}%
\pgfpathlineto{\pgfqpoint{1.730869in}{0.844188in}}%
\pgfpathlineto{\pgfqpoint{1.736892in}{0.850211in}}%
\pgfpathlineto{\pgfqpoint{1.742915in}{0.856234in}}%
\pgfpathlineto{\pgfqpoint{1.748939in}{0.862257in}}%
\pgfpathlineto{\pgfqpoint{1.754962in}{0.868280in}}%
\pgfpathlineto{\pgfqpoint{1.760985in}{0.874303in}}%
\pgfpathlineto{\pgfqpoint{1.767008in}{0.880326in}}%
\pgfpathlineto{\pgfqpoint{1.773031in}{0.886349in}}%
\pgfpathlineto{\pgfqpoint{1.779054in}{0.892372in}}%
\pgfpathlineto{\pgfqpoint{1.785077in}{0.898395in}}%
\pgfpathlineto{\pgfqpoint{1.791100in}{0.904418in}}%
\pgfpathlineto{\pgfqpoint{1.797123in}{0.910441in}}%
\pgfpathlineto{\pgfqpoint{1.803146in}{0.916464in}}%
\pgfpathlineto{\pgfqpoint{1.809169in}{0.922487in}}%
\pgfpathlineto{\pgfqpoint{1.815192in}{0.928510in}}%
\pgfpathlineto{\pgfqpoint{1.821215in}{0.934533in}}%
\pgfpathlineto{\pgfqpoint{1.827238in}{0.940557in}}%
\pgfpathlineto{\pgfqpoint{1.833261in}{0.946580in}}%
\pgfpathlineto{\pgfqpoint{1.839284in}{0.952603in}}%
\pgfpathlineto{\pgfqpoint{1.845307in}{0.958626in}}%
\pgfpathlineto{\pgfqpoint{1.851331in}{0.964649in}}%
\pgfpathlineto{\pgfqpoint{1.857354in}{0.970672in}}%
\pgfpathlineto{\pgfqpoint{1.863377in}{0.976695in}}%
\pgfpathlineto{\pgfqpoint{1.869400in}{0.982718in}}%
\pgfpathlineto{\pgfqpoint{1.875423in}{0.988741in}}%
\pgfpathlineto{\pgfqpoint{1.881446in}{0.994764in}}%
\pgfpathlineto{\pgfqpoint{1.887469in}{1.000787in}}%
\pgfpathlineto{\pgfqpoint{1.893492in}{1.006810in}}%
\pgfpathlineto{\pgfqpoint{1.899515in}{1.012833in}}%
\pgfpathlineto{\pgfqpoint{1.905538in}{1.018856in}}%
\pgfpathlineto{\pgfqpoint{1.911561in}{1.024879in}}%
\pgfpathlineto{\pgfqpoint{1.917584in}{1.030902in}}%
\pgfpathlineto{\pgfqpoint{1.923607in}{1.036925in}}%
\pgfpathlineto{\pgfqpoint{1.929630in}{1.042949in}}%
\pgfpathlineto{\pgfqpoint{1.935653in}{1.048972in}}%
\pgfpathlineto{\pgfqpoint{1.941676in}{1.054995in}}%
\pgfpathlineto{\pgfqpoint{1.947699in}{1.061018in}}%
\pgfpathlineto{\pgfqpoint{1.953723in}{1.067041in}}%
\pgfpathlineto{\pgfqpoint{1.959746in}{1.073064in}}%
\pgfpathlineto{\pgfqpoint{1.959746in}{1.085110in}}%
\pgfpathlineto{\pgfqpoint{1.953723in}{1.091133in}}%
\pgfpathlineto{\pgfqpoint{1.947699in}{1.097156in}}%
\pgfpathlineto{\pgfqpoint{1.941676in}{1.103179in}}%
\pgfpathlineto{\pgfqpoint{1.935653in}{1.109202in}}%
\pgfpathlineto{\pgfqpoint{1.929630in}{1.115225in}}%
\pgfpathlineto{\pgfqpoint{1.923607in}{1.121248in}}%
\pgfpathlineto{\pgfqpoint{1.917584in}{1.127271in}}%
\pgfpathlineto{\pgfqpoint{1.911561in}{1.133294in}}%
\pgfpathlineto{\pgfqpoint{1.905538in}{1.139317in}}%
\pgfpathlineto{\pgfqpoint{1.899515in}{1.145340in}}%
\pgfpathlineto{\pgfqpoint{1.893492in}{1.151364in}}%
\pgfpathlineto{\pgfqpoint{1.887469in}{1.157387in}}%
\pgfpathlineto{\pgfqpoint{1.881446in}{1.163410in}}%
\pgfpathlineto{\pgfqpoint{1.875423in}{1.169433in}}%
\pgfpathlineto{\pgfqpoint{1.869400in}{1.175456in}}%
\pgfpathlineto{\pgfqpoint{1.863377in}{1.181479in}}%
\pgfpathlineto{\pgfqpoint{1.857354in}{1.187502in}}%
\pgfpathlineto{\pgfqpoint{1.851331in}{1.193525in}}%
\pgfpathlineto{\pgfqpoint{1.845307in}{1.199548in}}%
\pgfpathlineto{\pgfqpoint{1.839284in}{1.205571in}}%
\pgfpathlineto{\pgfqpoint{1.833261in}{1.211594in}}%
\pgfpathlineto{\pgfqpoint{1.827238in}{1.217617in}}%
\pgfpathlineto{\pgfqpoint{1.821215in}{1.223640in}}%
\pgfpathlineto{\pgfqpoint{1.815192in}{1.229663in}}%
\pgfpathlineto{\pgfqpoint{1.809169in}{1.235686in}}%
\pgfpathlineto{\pgfqpoint{1.803146in}{1.241709in}}%
\pgfpathlineto{\pgfqpoint{1.797123in}{1.247732in}}%
\pgfpathlineto{\pgfqpoint{1.791100in}{1.253756in}}%
\pgfpathlineto{\pgfqpoint{1.785077in}{1.259779in}}%
\pgfpathlineto{\pgfqpoint{1.779054in}{1.265802in}}%
\pgfpathlineto{\pgfqpoint{1.773031in}{1.271825in}}%
\pgfpathlineto{\pgfqpoint{1.767008in}{1.277848in}}%
\pgfpathlineto{\pgfqpoint{1.760985in}{1.283871in}}%
\pgfpathlineto{\pgfqpoint{1.754962in}{1.289894in}}%
\pgfpathlineto{\pgfqpoint{1.748939in}{1.295917in}}%
\pgfpathlineto{\pgfqpoint{1.742915in}{1.301940in}}%
\pgfpathlineto{\pgfqpoint{1.736892in}{1.307963in}}%
\pgfpathlineto{\pgfqpoint{1.730869in}{1.313986in}}%
\pgfpathlineto{\pgfqpoint{1.724846in}{1.320009in}}%
\pgfpathlineto{\pgfqpoint{1.718823in}{1.326032in}}%
\pgfpathlineto{\pgfqpoint{1.712800in}{1.332055in}}%
\pgfpathlineto{\pgfqpoint{1.706777in}{1.338078in}}%
\pgfpathlineto{\pgfqpoint{1.700754in}{1.344101in}}%
\pgfpathlineto{\pgfqpoint{1.694731in}{1.350124in}}%
\pgfpathlineto{\pgfqpoint{1.688708in}{1.356148in}}%
\pgfpathlineto{\pgfqpoint{1.682685in}{1.362171in}}%
\pgfpathlineto{\pgfqpoint{1.676662in}{1.368194in}}%
\pgfpathlineto{\pgfqpoint{1.670639in}{1.374217in}}%
\pgfpathlineto{\pgfqpoint{1.664616in}{1.380240in}}%
\pgfpathlineto{\pgfqpoint{1.658593in}{1.386263in}}%
\pgfpathlineto{\pgfqpoint{1.652570in}{1.392286in}}%
\pgfpathlineto{\pgfqpoint{1.646547in}{1.398309in}}%
\pgfpathlineto{\pgfqpoint{1.640524in}{1.404332in}}%
\pgfpathlineto{\pgfqpoint{1.634500in}{1.410355in}}%
\pgfpathlineto{\pgfqpoint{1.628477in}{1.416378in}}%
\pgfpathlineto{\pgfqpoint{1.622454in}{1.422401in}}%
\pgfpathlineto{\pgfqpoint{1.616431in}{1.428424in}}%
\pgfpathlineto{\pgfqpoint{1.610408in}{1.434447in}}%
\pgfpathlineto{\pgfqpoint{1.604385in}{1.440470in}}%
\pgfpathlineto{\pgfqpoint{1.598362in}{1.446493in}}%
\pgfpathlineto{\pgfqpoint{1.592339in}{1.452516in}}%
\pgfpathlineto{\pgfqpoint{1.586316in}{1.458539in}}%
\pgfpathlineto{\pgfqpoint{1.580293in}{1.464563in}}%
\pgfpathlineto{\pgfqpoint{1.574270in}{1.470586in}}%
\pgfpathlineto{\pgfqpoint{1.568247in}{1.476609in}}%
\pgfpathlineto{\pgfqpoint{1.562224in}{1.482632in}}%
\pgfpathlineto{\pgfqpoint{1.556201in}{1.488655in}}%
\pgfpathlineto{\pgfqpoint{1.550178in}{1.494678in}}%
\pgfpathlineto{\pgfqpoint{1.544155in}{1.500701in}}%
\pgfpathlineto{\pgfqpoint{1.538132in}{1.506724in}}%
\pgfpathlineto{\pgfqpoint{1.532108in}{1.512747in}}%
\pgfpathlineto{\pgfqpoint{1.526085in}{1.518770in}}%
\pgfpathlineto{\pgfqpoint{1.520062in}{1.524793in}}%
\pgfpathlineto{\pgfqpoint{1.514039in}{1.530816in}}%
\pgfpathlineto{\pgfqpoint{1.508016in}{1.536839in}}%
\pgfpathlineto{\pgfqpoint{1.501993in}{1.542862in}}%
\pgfpathlineto{\pgfqpoint{1.495970in}{1.548885in}}%
\pgfpathlineto{\pgfqpoint{1.489947in}{1.554908in}}%
\pgfpathlineto{\pgfqpoint{1.483924in}{1.560931in}}%
\pgfpathlineto{\pgfqpoint{1.477901in}{1.566955in}}%
\pgfpathlineto{\pgfqpoint{1.471878in}{1.572978in}}%
\pgfpathlineto{\pgfqpoint{1.465855in}{1.579001in}}%
\pgfpathlineto{\pgfqpoint{1.459832in}{1.585024in}}%
\pgfpathlineto{\pgfqpoint{1.453809in}{1.591047in}}%
\pgfpathlineto{\pgfqpoint{1.447786in}{1.597070in}}%
\pgfpathlineto{\pgfqpoint{1.441763in}{1.603093in}}%
\pgfpathlineto{\pgfqpoint{1.435740in}{1.609116in}}%
\pgfpathlineto{\pgfqpoint{1.429716in}{1.615139in}}%
\pgfpathlineto{\pgfqpoint{1.423693in}{1.621162in}}%
\pgfpathlineto{\pgfqpoint{1.417670in}{1.627185in}}%
\pgfpathlineto{\pgfqpoint{1.411647in}{1.633208in}}%
\pgfpathlineto{\pgfqpoint{1.405624in}{1.639231in}}%
\pgfpathlineto{\pgfqpoint{1.399601in}{1.645254in}}%
\pgfpathlineto{\pgfqpoint{1.393578in}{1.651277in}}%
\pgfpathlineto{\pgfqpoint{1.387555in}{1.657300in}}%
\pgfpathlineto{\pgfqpoint{1.381532in}{1.663323in}}%
\pgfpathlineto{\pgfqpoint{1.375509in}{1.669347in}}%
\pgfpathlineto{\pgfqpoint{1.369486in}{1.675370in}}%
\pgfpathlineto{\pgfqpoint{1.363463in}{1.681393in}}%
\pgfpathlineto{\pgfqpoint{1.357440in}{1.687416in}}%
\pgfpathlineto{\pgfqpoint{1.351417in}{1.693439in}}%
\pgfpathlineto{\pgfqpoint{1.345394in}{1.699462in}}%
\pgfpathlineto{\pgfqpoint{1.339371in}{1.705485in}}%
\pgfpathlineto{\pgfqpoint{1.333348in}{1.711508in}}%
\pgfpathlineto{\pgfqpoint{1.327325in}{1.717531in}}%
\pgfpathlineto{\pgfqpoint{1.321301in}{1.723554in}}%
\pgfpathlineto{\pgfqpoint{1.315278in}{1.729577in}}%
\pgfpathlineto{\pgfqpoint{1.309255in}{1.735600in}}%
\pgfpathlineto{\pgfqpoint{1.303232in}{1.741623in}}%
\pgfpathlineto{\pgfqpoint{1.297209in}{1.747646in}}%
\pgfpathlineto{\pgfqpoint{1.291186in}{1.753669in}}%
\pgfpathlineto{\pgfqpoint{1.285163in}{1.759692in}}%
\pgfpathlineto{\pgfqpoint{1.279140in}{1.765715in}}%
\pgfpathlineto{\pgfqpoint{1.273117in}{1.771738in}}%
\pgfpathlineto{\pgfqpoint{1.267094in}{1.777762in}}%
\pgfpathlineto{\pgfqpoint{1.261071in}{1.783785in}}%
\pgfpathlineto{\pgfqpoint{1.255048in}{1.789808in}}%
\pgfpathlineto{\pgfqpoint{1.249025in}{1.795831in}}%
\pgfpathlineto{\pgfqpoint{1.243002in}{1.801854in}}%
\pgfpathlineto{\pgfqpoint{1.236979in}{1.807877in}}%
\pgfpathlineto{\pgfqpoint{1.230956in}{1.813900in}}%
\pgfpathlineto{\pgfqpoint{1.224933in}{1.819923in}}%
\pgfpathlineto{\pgfqpoint{1.218909in}{1.825946in}}%
\pgfpathlineto{\pgfqpoint{1.212886in}{1.831969in}}%
\pgfpathlineto{\pgfqpoint{1.206863in}{1.837992in}}%
\pgfpathlineto{\pgfqpoint{1.200840in}{1.844015in}}%
\pgfpathlineto{\pgfqpoint{1.194817in}{1.850038in}}%
\pgfpathlineto{\pgfqpoint{1.188794in}{1.856061in}}%
\pgfpathlineto{\pgfqpoint{1.182771in}{1.862084in}}%
\pgfpathlineto{\pgfqpoint{1.176748in}{1.868107in}}%
\pgfpathlineto{\pgfqpoint{1.170725in}{1.874130in}}%
\pgfpathlineto{\pgfqpoint{1.164702in}{1.880154in}}%
\pgfpathlineto{\pgfqpoint{1.158679in}{1.886177in}}%
\pgfpathlineto{\pgfqpoint{1.152656in}{1.892200in}}%
\pgfpathlineto{\pgfqpoint{1.146633in}{1.898223in}}%
\pgfpathlineto{\pgfqpoint{1.140610in}{1.904246in}}%
\pgfpathlineto{\pgfqpoint{1.134587in}{1.910269in}}%
\pgfpathlineto{\pgfqpoint{1.128564in}{1.916292in}}%
\pgfpathlineto{\pgfqpoint{1.122541in}{1.922315in}}%
\pgfpathlineto{\pgfqpoint{1.116517in}{1.928338in}}%
\pgfpathlineto{\pgfqpoint{1.110494in}{1.934361in}}%
\pgfpathlineto{\pgfqpoint{1.104471in}{1.940384in}}%
\pgfpathlineto{\pgfqpoint{1.098448in}{1.946407in}}%
\pgfpathlineto{\pgfqpoint{1.092425in}{1.952430in}}%
\pgfpathlineto{\pgfqpoint{1.086402in}{1.958453in}}%
\pgfpathlineto{\pgfqpoint{1.080379in}{1.964476in}}%
\pgfpathlineto{\pgfqpoint{1.074356in}{1.970499in}}%
\pgfpathlineto{\pgfqpoint{1.062310in}{1.970499in}}%
\pgfpathlineto{\pgfqpoint{1.056287in}{1.964476in}}%
\pgfpathlineto{\pgfqpoint{1.050264in}{1.958453in}}%
\pgfpathlineto{\pgfqpoint{1.044241in}{1.952430in}}%
\pgfpathlineto{\pgfqpoint{1.038218in}{1.946407in}}%
\pgfpathlineto{\pgfqpoint{1.032195in}{1.940384in}}%
\pgfpathlineto{\pgfqpoint{1.026172in}{1.934361in}}%
\pgfpathlineto{\pgfqpoint{1.020149in}{1.928338in}}%
\pgfpathlineto{\pgfqpoint{1.014126in}{1.922315in}}%
\pgfpathlineto{\pgfqpoint{1.008102in}{1.916292in}}%
\pgfpathlineto{\pgfqpoint{1.002079in}{1.910269in}}%
\pgfpathlineto{\pgfqpoint{0.996056in}{1.904246in}}%
\pgfpathlineto{\pgfqpoint{0.990033in}{1.898223in}}%
\pgfpathlineto{\pgfqpoint{0.984010in}{1.892200in}}%
\pgfpathlineto{\pgfqpoint{0.977987in}{1.886177in}}%
\pgfpathlineto{\pgfqpoint{0.971964in}{1.880154in}}%
\pgfpathlineto{\pgfqpoint{0.965941in}{1.874130in}}%
\pgfpathlineto{\pgfqpoint{0.959918in}{1.868107in}}%
\pgfpathlineto{\pgfqpoint{0.953895in}{1.862084in}}%
\pgfpathlineto{\pgfqpoint{0.947872in}{1.856061in}}%
\pgfpathlineto{\pgfqpoint{0.941849in}{1.850038in}}%
\pgfpathlineto{\pgfqpoint{0.935826in}{1.844015in}}%
\pgfpathlineto{\pgfqpoint{0.929803in}{1.837992in}}%
\pgfpathlineto{\pgfqpoint{0.923780in}{1.831969in}}%
\pgfpathlineto{\pgfqpoint{0.917757in}{1.825946in}}%
\pgfpathlineto{\pgfqpoint{0.911734in}{1.819923in}}%
\pgfpathlineto{\pgfqpoint{0.905710in}{1.813900in}}%
\pgfpathlineto{\pgfqpoint{0.899687in}{1.807877in}}%
\pgfpathlineto{\pgfqpoint{0.893664in}{1.801854in}}%
\pgfpathlineto{\pgfqpoint{0.887641in}{1.795831in}}%
\pgfpathlineto{\pgfqpoint{0.881618in}{1.789808in}}%
\pgfpathlineto{\pgfqpoint{0.875595in}{1.783785in}}%
\pgfpathlineto{\pgfqpoint{0.869572in}{1.777762in}}%
\pgfpathlineto{\pgfqpoint{0.863549in}{1.771738in}}%
\pgfpathlineto{\pgfqpoint{0.857526in}{1.765715in}}%
\pgfpathlineto{\pgfqpoint{0.851503in}{1.759692in}}%
\pgfpathlineto{\pgfqpoint{0.845480in}{1.753669in}}%
\pgfpathlineto{\pgfqpoint{0.839457in}{1.747646in}}%
\pgfpathlineto{\pgfqpoint{0.833434in}{1.741623in}}%
\pgfpathlineto{\pgfqpoint{0.827411in}{1.735600in}}%
\pgfpathlineto{\pgfqpoint{0.821388in}{1.729577in}}%
\pgfpathlineto{\pgfqpoint{0.815365in}{1.723554in}}%
\pgfpathlineto{\pgfqpoint{0.809342in}{1.717531in}}%
\pgfpathlineto{\pgfqpoint{0.803318in}{1.711508in}}%
\pgfpathlineto{\pgfqpoint{0.797295in}{1.705485in}}%
\pgfpathlineto{\pgfqpoint{0.791272in}{1.699462in}}%
\pgfpathlineto{\pgfqpoint{0.785249in}{1.693439in}}%
\pgfpathlineto{\pgfqpoint{0.779226in}{1.687416in}}%
\pgfpathlineto{\pgfqpoint{0.773203in}{1.681393in}}%
\pgfpathlineto{\pgfqpoint{0.767180in}{1.675370in}}%
\pgfpathlineto{\pgfqpoint{0.761157in}{1.669347in}}%
\pgfpathlineto{\pgfqpoint{0.755134in}{1.663323in}}%
\pgfpathlineto{\pgfqpoint{0.749111in}{1.657300in}}%
\pgfpathlineto{\pgfqpoint{0.743088in}{1.651277in}}%
\pgfpathlineto{\pgfqpoint{0.737065in}{1.645254in}}%
\pgfpathlineto{\pgfqpoint{0.731042in}{1.639231in}}%
\pgfpathlineto{\pgfqpoint{0.725019in}{1.633208in}}%
\pgfpathlineto{\pgfqpoint{0.718996in}{1.627185in}}%
\pgfpathlineto{\pgfqpoint{0.712973in}{1.621162in}}%
\pgfpathlineto{\pgfqpoint{0.706950in}{1.615139in}}%
\pgfpathlineto{\pgfqpoint{0.700927in}{1.609116in}}%
\pgfpathlineto{\pgfqpoint{0.694903in}{1.603093in}}%
\pgfpathlineto{\pgfqpoint{0.688880in}{1.597070in}}%
\pgfpathlineto{\pgfqpoint{0.682857in}{1.591047in}}%
\pgfpathlineto{\pgfqpoint{0.676834in}{1.585024in}}%
\pgfpathlineto{\pgfqpoint{0.670811in}{1.579001in}}%
\pgfpathlineto{\pgfqpoint{0.664788in}{1.572978in}}%
\pgfpathlineto{\pgfqpoint{0.658765in}{1.566955in}}%
\pgfpathlineto{\pgfqpoint{0.652742in}{1.560931in}}%
\pgfpathlineto{\pgfqpoint{0.646719in}{1.554908in}}%
\pgfpathlineto{\pgfqpoint{0.640696in}{1.548885in}}%
\pgfpathlineto{\pgfqpoint{0.634673in}{1.542862in}}%
\pgfpathlineto{\pgfqpoint{0.628650in}{1.536839in}}%
\pgfpathlineto{\pgfqpoint{0.622627in}{1.530816in}}%
\pgfpathlineto{\pgfqpoint{0.616604in}{1.524793in}}%
\pgfpathlineto{\pgfqpoint{0.610581in}{1.518770in}}%
\pgfpathlineto{\pgfqpoint{0.604558in}{1.512747in}}%
\pgfpathlineto{\pgfqpoint{0.598535in}{1.506724in}}%
\pgfpathlineto{\pgfqpoint{0.592511in}{1.500701in}}%
\pgfpathlineto{\pgfqpoint{0.586488in}{1.494678in}}%
\pgfpathlineto{\pgfqpoint{0.580465in}{1.488655in}}%
\pgfpathlineto{\pgfqpoint{0.574442in}{1.482632in}}%
\pgfpathlineto{\pgfqpoint{0.568419in}{1.476609in}}%
\pgfpathlineto{\pgfqpoint{0.562396in}{1.470586in}}%
\pgfpathlineto{\pgfqpoint{0.556373in}{1.464563in}}%
\pgfpathlineto{\pgfqpoint{0.550350in}{1.458539in}}%
\pgfpathlineto{\pgfqpoint{0.544327in}{1.452516in}}%
\pgfpathlineto{\pgfqpoint{0.538304in}{1.446493in}}%
\pgfpathlineto{\pgfqpoint{0.532281in}{1.440470in}}%
\pgfpathlineto{\pgfqpoint{0.526258in}{1.434447in}}%
\pgfpathlineto{\pgfqpoint{0.520235in}{1.428424in}}%
\pgfpathlineto{\pgfqpoint{0.514212in}{1.422401in}}%
\pgfpathlineto{\pgfqpoint{0.508189in}{1.416378in}}%
\pgfpathlineto{\pgfqpoint{0.502166in}{1.410355in}}%
\pgfpathlineto{\pgfqpoint{0.496143in}{1.404332in}}%
\pgfpathlineto{\pgfqpoint{0.490119in}{1.398309in}}%
\pgfpathlineto{\pgfqpoint{0.484096in}{1.392286in}}%
\pgfpathlineto{\pgfqpoint{0.478073in}{1.386263in}}%
\pgfpathlineto{\pgfqpoint{0.472050in}{1.380240in}}%
\pgfpathlineto{\pgfqpoint{0.466027in}{1.374217in}}%
\pgfpathlineto{\pgfqpoint{0.460004in}{1.368194in}}%
\pgfpathlineto{\pgfqpoint{0.453981in}{1.362171in}}%
\pgfpathlineto{\pgfqpoint{0.447958in}{1.356148in}}%
\pgfpathlineto{\pgfqpoint{0.441935in}{1.350124in}}%
\pgfpathlineto{\pgfqpoint{0.435912in}{1.344101in}}%
\pgfpathlineto{\pgfqpoint{0.429889in}{1.338078in}}%
\pgfpathlineto{\pgfqpoint{0.423866in}{1.332055in}}%
\pgfpathlineto{\pgfqpoint{0.417843in}{1.326032in}}%
\pgfpathlineto{\pgfqpoint{0.411820in}{1.320009in}}%
\pgfpathlineto{\pgfqpoint{0.405797in}{1.313986in}}%
\pgfpathlineto{\pgfqpoint{0.399774in}{1.307963in}}%
\pgfpathlineto{\pgfqpoint{0.393751in}{1.301940in}}%
\pgfpathlineto{\pgfqpoint{0.387728in}{1.295917in}}%
\pgfpathlineto{\pgfqpoint{0.381704in}{1.289894in}}%
\pgfpathlineto{\pgfqpoint{0.375681in}{1.283871in}}%
\pgfpathlineto{\pgfqpoint{0.369658in}{1.277848in}}%
\pgfpathlineto{\pgfqpoint{0.363635in}{1.271825in}}%
\pgfpathlineto{\pgfqpoint{0.357612in}{1.265802in}}%
\pgfpathlineto{\pgfqpoint{0.351589in}{1.259779in}}%
\pgfpathlineto{\pgfqpoint{0.345566in}{1.253756in}}%
\pgfpathlineto{\pgfqpoint{0.339543in}{1.247732in}}%
\pgfpathlineto{\pgfqpoint{0.333520in}{1.241709in}}%
\pgfpathlineto{\pgfqpoint{0.327497in}{1.235686in}}%
\pgfpathlineto{\pgfqpoint{0.321474in}{1.229663in}}%
\pgfpathlineto{\pgfqpoint{0.315451in}{1.223640in}}%
\pgfpathlineto{\pgfqpoint{0.309428in}{1.217617in}}%
\pgfpathlineto{\pgfqpoint{0.303405in}{1.211594in}}%
\pgfpathlineto{\pgfqpoint{0.297382in}{1.205571in}}%
\pgfpathlineto{\pgfqpoint{0.291359in}{1.199548in}}%
\pgfpathlineto{\pgfqpoint{0.285336in}{1.193525in}}%
\pgfpathlineto{\pgfqpoint{0.279312in}{1.187502in}}%
\pgfpathlineto{\pgfqpoint{0.273289in}{1.181479in}}%
\pgfpathlineto{\pgfqpoint{0.267266in}{1.175456in}}%
\pgfpathlineto{\pgfqpoint{0.261243in}{1.169433in}}%
\pgfpathlineto{\pgfqpoint{0.255220in}{1.163410in}}%
\pgfpathlineto{\pgfqpoint{0.249197in}{1.157387in}}%
\pgfpathlineto{\pgfqpoint{0.243174in}{1.151364in}}%
\pgfpathlineto{\pgfqpoint{0.237151in}{1.145340in}}%
\pgfpathlineto{\pgfqpoint{0.231128in}{1.139317in}}%
\pgfpathlineto{\pgfqpoint{0.225105in}{1.133294in}}%
\pgfpathlineto{\pgfqpoint{0.219082in}{1.127271in}}%
\pgfpathlineto{\pgfqpoint{0.213059in}{1.121248in}}%
\pgfpathlineto{\pgfqpoint{0.207036in}{1.115225in}}%
\pgfpathlineto{\pgfqpoint{0.201013in}{1.109202in}}%
\pgfpathlineto{\pgfqpoint{0.194990in}{1.103179in}}%
\pgfpathlineto{\pgfqpoint{0.188967in}{1.097156in}}%
\pgfpathlineto{\pgfqpoint{0.182944in}{1.091133in}}%
\pgfpathlineto{\pgfqpoint{0.176920in}{1.085110in}}%
\pgfpathlineto{\pgfqpoint{0.176920in}{1.073064in}}%
\pgfpathlineto{\pgfqpoint{0.182944in}{1.067041in}}%
\pgfpathlineto{\pgfqpoint{0.188967in}{1.061018in}}%
\pgfpathlineto{\pgfqpoint{0.194990in}{1.054995in}}%
\pgfpathlineto{\pgfqpoint{0.201013in}{1.048972in}}%
\pgfpathlineto{\pgfqpoint{0.207036in}{1.042949in}}%
\pgfpathlineto{\pgfqpoint{0.213059in}{1.036925in}}%
\pgfpathlineto{\pgfqpoint{0.219082in}{1.030902in}}%
\pgfpathlineto{\pgfqpoint{0.225105in}{1.024879in}}%
\pgfpathlineto{\pgfqpoint{0.231128in}{1.018856in}}%
\pgfpathlineto{\pgfqpoint{0.237151in}{1.012833in}}%
\pgfpathlineto{\pgfqpoint{0.243174in}{1.006810in}}%
\pgfpathlineto{\pgfqpoint{0.249197in}{1.000787in}}%
\pgfpathlineto{\pgfqpoint{0.255220in}{0.994764in}}%
\pgfpathlineto{\pgfqpoint{0.261243in}{0.988741in}}%
\pgfpathlineto{\pgfqpoint{0.267266in}{0.982718in}}%
\pgfpathlineto{\pgfqpoint{0.273289in}{0.976695in}}%
\pgfpathlineto{\pgfqpoint{0.279312in}{0.970672in}}%
\pgfpathlineto{\pgfqpoint{0.285336in}{0.964649in}}%
\pgfpathlineto{\pgfqpoint{0.291359in}{0.958626in}}%
\pgfpathlineto{\pgfqpoint{0.297382in}{0.952603in}}%
\pgfpathlineto{\pgfqpoint{0.303405in}{0.946580in}}%
\pgfpathlineto{\pgfqpoint{0.309428in}{0.940557in}}%
\pgfpathlineto{\pgfqpoint{0.315451in}{0.934533in}}%
\pgfpathlineto{\pgfqpoint{0.321474in}{0.928510in}}%
\pgfpathlineto{\pgfqpoint{0.327497in}{0.922487in}}%
\pgfpathlineto{\pgfqpoint{0.333520in}{0.916464in}}%
\pgfpathlineto{\pgfqpoint{0.339543in}{0.910441in}}%
\pgfpathlineto{\pgfqpoint{0.345566in}{0.904418in}}%
\pgfpathlineto{\pgfqpoint{0.351589in}{0.898395in}}%
\pgfpathlineto{\pgfqpoint{0.357612in}{0.892372in}}%
\pgfpathlineto{\pgfqpoint{0.363635in}{0.886349in}}%
\pgfpathlineto{\pgfqpoint{0.369658in}{0.880326in}}%
\pgfpathlineto{\pgfqpoint{0.375681in}{0.874303in}}%
\pgfpathlineto{\pgfqpoint{0.381704in}{0.868280in}}%
\pgfpathlineto{\pgfqpoint{0.387728in}{0.862257in}}%
\pgfpathlineto{\pgfqpoint{0.393751in}{0.856234in}}%
\pgfpathlineto{\pgfqpoint{0.399774in}{0.850211in}}%
\pgfpathlineto{\pgfqpoint{0.405797in}{0.844188in}}%
\pgfpathlineto{\pgfqpoint{0.411820in}{0.838165in}}%
\pgfpathlineto{\pgfqpoint{0.417843in}{0.832141in}}%
\pgfpathlineto{\pgfqpoint{0.423866in}{0.826118in}}%
\pgfpathlineto{\pgfqpoint{0.429889in}{0.820095in}}%
\pgfpathlineto{\pgfqpoint{0.435912in}{0.814072in}}%
\pgfpathlineto{\pgfqpoint{0.441935in}{0.808049in}}%
\pgfpathlineto{\pgfqpoint{0.447958in}{0.802026in}}%
\pgfpathlineto{\pgfqpoint{0.453981in}{0.796003in}}%
\pgfpathlineto{\pgfqpoint{0.460004in}{0.789980in}}%
\pgfpathlineto{\pgfqpoint{0.466027in}{0.783957in}}%
\pgfpathlineto{\pgfqpoint{0.472050in}{0.777934in}}%
\pgfpathlineto{\pgfqpoint{0.478073in}{0.771911in}}%
\pgfpathlineto{\pgfqpoint{0.484096in}{0.765888in}}%
\pgfpathlineto{\pgfqpoint{0.490119in}{0.759865in}}%
\pgfpathlineto{\pgfqpoint{0.496143in}{0.753842in}}%
\pgfpathlineto{\pgfqpoint{0.502166in}{0.747819in}}%
\pgfpathlineto{\pgfqpoint{0.508189in}{0.741796in}}%
\pgfpathlineto{\pgfqpoint{0.514212in}{0.735773in}}%
\pgfpathlineto{\pgfqpoint{0.520235in}{0.729750in}}%
\pgfpathlineto{\pgfqpoint{0.526258in}{0.723726in}}%
\pgfpathlineto{\pgfqpoint{0.532281in}{0.717703in}}%
\pgfpathlineto{\pgfqpoint{0.538304in}{0.711680in}}%
\pgfpathlineto{\pgfqpoint{0.544327in}{0.705657in}}%
\pgfpathlineto{\pgfqpoint{0.550350in}{0.699634in}}%
\pgfpathlineto{\pgfqpoint{0.556373in}{0.693611in}}%
\pgfpathlineto{\pgfqpoint{0.562396in}{0.687588in}}%
\pgfpathlineto{\pgfqpoint{0.568419in}{0.681565in}}%
\pgfpathlineto{\pgfqpoint{0.574442in}{0.675542in}}%
\pgfpathlineto{\pgfqpoint{0.580465in}{0.669519in}}%
\pgfpathlineto{\pgfqpoint{0.586488in}{0.663496in}}%
\pgfpathlineto{\pgfqpoint{0.592511in}{0.657473in}}%
\pgfpathlineto{\pgfqpoint{0.598535in}{0.651450in}}%
\pgfpathlineto{\pgfqpoint{0.604558in}{0.645427in}}%
\pgfpathlineto{\pgfqpoint{0.610581in}{0.639404in}}%
\pgfpathlineto{\pgfqpoint{0.616604in}{0.633381in}}%
\pgfpathlineto{\pgfqpoint{0.622627in}{0.627358in}}%
\pgfpathlineto{\pgfqpoint{0.628650in}{0.621334in}}%
\pgfpathlineto{\pgfqpoint{0.634673in}{0.615311in}}%
\pgfpathlineto{\pgfqpoint{0.640696in}{0.609288in}}%
\pgfpathlineto{\pgfqpoint{0.646719in}{0.603265in}}%
\pgfpathlineto{\pgfqpoint{0.652742in}{0.597242in}}%
\pgfpathlineto{\pgfqpoint{0.658765in}{0.591219in}}%
\pgfpathlineto{\pgfqpoint{0.664788in}{0.585196in}}%
\pgfpathlineto{\pgfqpoint{0.670811in}{0.579173in}}%
\pgfpathlineto{\pgfqpoint{0.676834in}{0.573150in}}%
\pgfpathlineto{\pgfqpoint{0.682857in}{0.567127in}}%
\pgfpathlineto{\pgfqpoint{0.688880in}{0.561104in}}%
\pgfpathlineto{\pgfqpoint{0.694903in}{0.555081in}}%
\pgfpathlineto{\pgfqpoint{0.700927in}{0.549058in}}%
\pgfpathlineto{\pgfqpoint{0.706950in}{0.543035in}}%
\pgfpathlineto{\pgfqpoint{0.712973in}{0.537012in}}%
\pgfpathlineto{\pgfqpoint{0.718996in}{0.530989in}}%
\pgfpathlineto{\pgfqpoint{0.725019in}{0.524966in}}%
\pgfpathlineto{\pgfqpoint{0.731042in}{0.518942in}}%
\pgfpathlineto{\pgfqpoint{0.737065in}{0.512919in}}%
\pgfpathlineto{\pgfqpoint{0.743088in}{0.506896in}}%
\pgfpathlineto{\pgfqpoint{0.749111in}{0.500873in}}%
\pgfpathlineto{\pgfqpoint{0.755134in}{0.494850in}}%
\pgfpathlineto{\pgfqpoint{0.761157in}{0.488827in}}%
\pgfpathlineto{\pgfqpoint{0.767180in}{0.482804in}}%
\pgfpathlineto{\pgfqpoint{0.773203in}{0.476781in}}%
\pgfpathlineto{\pgfqpoint{0.779226in}{0.470758in}}%
\pgfpathlineto{\pgfqpoint{0.785249in}{0.464735in}}%
\pgfpathlineto{\pgfqpoint{0.791272in}{0.458712in}}%
\pgfpathlineto{\pgfqpoint{0.797295in}{0.452689in}}%
\pgfpathlineto{\pgfqpoint{0.803318in}{0.446666in}}%
\pgfpathlineto{\pgfqpoint{0.809342in}{0.440643in}}%
\pgfpathlineto{\pgfqpoint{0.815365in}{0.434620in}}%
\pgfpathlineto{\pgfqpoint{0.821388in}{0.428597in}}%
\pgfpathlineto{\pgfqpoint{0.827411in}{0.422574in}}%
\pgfpathlineto{\pgfqpoint{0.833434in}{0.416551in}}%
\pgfpathlineto{\pgfqpoint{0.839457in}{0.410527in}}%
\pgfpathlineto{\pgfqpoint{0.845480in}{0.404504in}}%
\pgfpathlineto{\pgfqpoint{0.851503in}{0.398481in}}%
\pgfpathlineto{\pgfqpoint{0.857526in}{0.392458in}}%
\pgfpathlineto{\pgfqpoint{0.863549in}{0.386435in}}%
\pgfpathlineto{\pgfqpoint{0.869572in}{0.380412in}}%
\pgfpathlineto{\pgfqpoint{0.875595in}{0.374389in}}%
\pgfpathlineto{\pgfqpoint{0.881618in}{0.368366in}}%
\pgfpathlineto{\pgfqpoint{0.887641in}{0.362343in}}%
\pgfpathlineto{\pgfqpoint{0.893664in}{0.356320in}}%
\pgfpathlineto{\pgfqpoint{0.899687in}{0.350297in}}%
\pgfpathlineto{\pgfqpoint{0.905710in}{0.344274in}}%
\pgfpathlineto{\pgfqpoint{0.911734in}{0.338251in}}%
\pgfpathlineto{\pgfqpoint{0.917757in}{0.332228in}}%
\pgfpathlineto{\pgfqpoint{0.923780in}{0.326205in}}%
\pgfpathlineto{\pgfqpoint{0.929803in}{0.320182in}}%
\pgfpathlineto{\pgfqpoint{0.935826in}{0.314159in}}%
\pgfpathlineto{\pgfqpoint{0.941849in}{0.308135in}}%
\pgfpathlineto{\pgfqpoint{0.947872in}{0.302112in}}%
\pgfpathlineto{\pgfqpoint{0.953895in}{0.296089in}}%
\pgfpathlineto{\pgfqpoint{0.959918in}{0.290066in}}%
\pgfpathlineto{\pgfqpoint{0.965941in}{0.284043in}}%
\pgfpathlineto{\pgfqpoint{0.971964in}{0.278020in}}%
\pgfpathlineto{\pgfqpoint{0.977987in}{0.271997in}}%
\pgfpathlineto{\pgfqpoint{0.984010in}{0.265974in}}%
\pgfpathlineto{\pgfqpoint{0.990033in}{0.259951in}}%
\pgfpathlineto{\pgfqpoint{0.996056in}{0.253928in}}%
\pgfpathlineto{\pgfqpoint{1.002079in}{0.247905in}}%
\pgfpathlineto{\pgfqpoint{1.008102in}{0.241882in}}%
\pgfpathlineto{\pgfqpoint{1.014126in}{0.235859in}}%
\pgfpathlineto{\pgfqpoint{1.020149in}{0.229836in}}%
\pgfpathlineto{\pgfqpoint{1.026172in}{0.223813in}}%
\pgfpathlineto{\pgfqpoint{1.032195in}{0.217790in}}%
\pgfpathlineto{\pgfqpoint{1.038218in}{0.211767in}}%
\pgfpathlineto{\pgfqpoint{1.044241in}{0.205743in}}%
\pgfpathlineto{\pgfqpoint{1.050264in}{0.199720in}}%
\pgfpathlineto{\pgfqpoint{1.056287in}{0.193697in}}%
\pgfpathclose%
\pgfpathmoveto{\pgfqpoint{1.056287in}{0.193697in}}%
\pgfpathlineto{\pgfqpoint{1.050264in}{0.199720in}}%
\pgfpathlineto{\pgfqpoint{1.044241in}{0.205743in}}%
\pgfpathlineto{\pgfqpoint{1.038218in}{0.211767in}}%
\pgfpathlineto{\pgfqpoint{1.032195in}{0.217790in}}%
\pgfpathlineto{\pgfqpoint{1.026172in}{0.223813in}}%
\pgfpathlineto{\pgfqpoint{1.020149in}{0.229836in}}%
\pgfpathlineto{\pgfqpoint{1.014126in}{0.235859in}}%
\pgfpathlineto{\pgfqpoint{1.008102in}{0.241882in}}%
\pgfpathlineto{\pgfqpoint{1.002079in}{0.247905in}}%
\pgfpathlineto{\pgfqpoint{0.996056in}{0.253928in}}%
\pgfpathlineto{\pgfqpoint{0.990033in}{0.259951in}}%
\pgfpathlineto{\pgfqpoint{0.984010in}{0.265974in}}%
\pgfpathlineto{\pgfqpoint{0.977987in}{0.271997in}}%
\pgfpathlineto{\pgfqpoint{0.971964in}{0.278020in}}%
\pgfpathlineto{\pgfqpoint{0.965941in}{0.284043in}}%
\pgfpathlineto{\pgfqpoint{0.959918in}{0.290066in}}%
\pgfpathlineto{\pgfqpoint{0.953895in}{0.296089in}}%
\pgfpathlineto{\pgfqpoint{0.947872in}{0.302112in}}%
\pgfpathlineto{\pgfqpoint{0.941849in}{0.308135in}}%
\pgfpathlineto{\pgfqpoint{0.935826in}{0.314159in}}%
\pgfpathlineto{\pgfqpoint{0.929803in}{0.320182in}}%
\pgfpathlineto{\pgfqpoint{0.923780in}{0.326205in}}%
\pgfpathlineto{\pgfqpoint{0.917757in}{0.332228in}}%
\pgfpathlineto{\pgfqpoint{0.911734in}{0.338251in}}%
\pgfpathlineto{\pgfqpoint{0.905710in}{0.344274in}}%
\pgfpathlineto{\pgfqpoint{0.899687in}{0.350297in}}%
\pgfpathlineto{\pgfqpoint{0.893664in}{0.356320in}}%
\pgfpathlineto{\pgfqpoint{0.887641in}{0.362343in}}%
\pgfpathlineto{\pgfqpoint{0.881618in}{0.368366in}}%
\pgfpathlineto{\pgfqpoint{0.875595in}{0.374389in}}%
\pgfpathlineto{\pgfqpoint{0.869572in}{0.380412in}}%
\pgfpathlineto{\pgfqpoint{0.863549in}{0.386435in}}%
\pgfpathlineto{\pgfqpoint{0.857526in}{0.392458in}}%
\pgfpathlineto{\pgfqpoint{0.851503in}{0.398481in}}%
\pgfpathlineto{\pgfqpoint{0.845480in}{0.404504in}}%
\pgfpathlineto{\pgfqpoint{0.839457in}{0.410527in}}%
\pgfpathlineto{\pgfqpoint{0.833434in}{0.416551in}}%
\pgfpathlineto{\pgfqpoint{0.827411in}{0.422574in}}%
\pgfpathlineto{\pgfqpoint{0.821388in}{0.428597in}}%
\pgfpathlineto{\pgfqpoint{0.815365in}{0.434620in}}%
\pgfpathlineto{\pgfqpoint{0.809342in}{0.440643in}}%
\pgfpathlineto{\pgfqpoint{0.803318in}{0.446666in}}%
\pgfpathlineto{\pgfqpoint{0.797295in}{0.452689in}}%
\pgfpathlineto{\pgfqpoint{0.791272in}{0.458712in}}%
\pgfpathlineto{\pgfqpoint{0.785249in}{0.464735in}}%
\pgfpathlineto{\pgfqpoint{0.779226in}{0.470758in}}%
\pgfpathlineto{\pgfqpoint{0.773203in}{0.476781in}}%
\pgfpathlineto{\pgfqpoint{0.767180in}{0.482804in}}%
\pgfpathlineto{\pgfqpoint{0.761157in}{0.488827in}}%
\pgfpathlineto{\pgfqpoint{0.755134in}{0.494850in}}%
\pgfpathlineto{\pgfqpoint{0.749111in}{0.500873in}}%
\pgfpathlineto{\pgfqpoint{0.743088in}{0.506896in}}%
\pgfpathlineto{\pgfqpoint{0.737065in}{0.512919in}}%
\pgfpathlineto{\pgfqpoint{0.731042in}{0.518942in}}%
\pgfpathlineto{\pgfqpoint{0.725019in}{0.524966in}}%
\pgfpathlineto{\pgfqpoint{0.718996in}{0.530989in}}%
\pgfpathlineto{\pgfqpoint{0.712973in}{0.537012in}}%
\pgfpathlineto{\pgfqpoint{0.706950in}{0.543035in}}%
\pgfpathlineto{\pgfqpoint{0.700927in}{0.549058in}}%
\pgfpathlineto{\pgfqpoint{0.694903in}{0.555081in}}%
\pgfpathlineto{\pgfqpoint{0.688880in}{0.561104in}}%
\pgfpathlineto{\pgfqpoint{0.682857in}{0.567127in}}%
\pgfpathlineto{\pgfqpoint{0.676834in}{0.573150in}}%
\pgfpathlineto{\pgfqpoint{0.670811in}{0.579173in}}%
\pgfpathlineto{\pgfqpoint{0.664788in}{0.585196in}}%
\pgfpathlineto{\pgfqpoint{0.658765in}{0.591219in}}%
\pgfpathlineto{\pgfqpoint{0.652742in}{0.597242in}}%
\pgfpathlineto{\pgfqpoint{0.646719in}{0.603265in}}%
\pgfpathlineto{\pgfqpoint{0.640696in}{0.609288in}}%
\pgfpathlineto{\pgfqpoint{0.634673in}{0.615311in}}%
\pgfpathlineto{\pgfqpoint{0.628650in}{0.621334in}}%
\pgfpathlineto{\pgfqpoint{0.622627in}{0.627358in}}%
\pgfpathlineto{\pgfqpoint{0.616604in}{0.633381in}}%
\pgfpathlineto{\pgfqpoint{0.610581in}{0.639404in}}%
\pgfpathlineto{\pgfqpoint{0.604558in}{0.645427in}}%
\pgfpathlineto{\pgfqpoint{0.598535in}{0.651450in}}%
\pgfpathlineto{\pgfqpoint{0.592511in}{0.657473in}}%
\pgfpathlineto{\pgfqpoint{0.586488in}{0.663496in}}%
\pgfpathlineto{\pgfqpoint{0.580465in}{0.669519in}}%
\pgfpathlineto{\pgfqpoint{0.574442in}{0.675542in}}%
\pgfpathlineto{\pgfqpoint{0.568419in}{0.681565in}}%
\pgfpathlineto{\pgfqpoint{0.562396in}{0.687588in}}%
\pgfpathlineto{\pgfqpoint{0.556373in}{0.693611in}}%
\pgfpathlineto{\pgfqpoint{0.550350in}{0.699634in}}%
\pgfpathlineto{\pgfqpoint{0.544327in}{0.705657in}}%
\pgfpathlineto{\pgfqpoint{0.538304in}{0.711680in}}%
\pgfpathlineto{\pgfqpoint{0.532281in}{0.717703in}}%
\pgfpathlineto{\pgfqpoint{0.526258in}{0.723726in}}%
\pgfpathlineto{\pgfqpoint{0.520235in}{0.729750in}}%
\pgfpathlineto{\pgfqpoint{0.514212in}{0.735773in}}%
\pgfpathlineto{\pgfqpoint{0.508189in}{0.741796in}}%
\pgfpathlineto{\pgfqpoint{0.502166in}{0.747819in}}%
\pgfpathlineto{\pgfqpoint{0.496143in}{0.753842in}}%
\pgfpathlineto{\pgfqpoint{0.490119in}{0.759865in}}%
\pgfpathlineto{\pgfqpoint{0.484096in}{0.765888in}}%
\pgfpathlineto{\pgfqpoint{0.478073in}{0.771911in}}%
\pgfpathlineto{\pgfqpoint{0.472050in}{0.777934in}}%
\pgfpathlineto{\pgfqpoint{0.466027in}{0.783957in}}%
\pgfpathlineto{\pgfqpoint{0.460004in}{0.789980in}}%
\pgfpathlineto{\pgfqpoint{0.453981in}{0.796003in}}%
\pgfpathlineto{\pgfqpoint{0.447958in}{0.802026in}}%
\pgfpathlineto{\pgfqpoint{0.441935in}{0.808049in}}%
\pgfpathlineto{\pgfqpoint{0.435912in}{0.814072in}}%
\pgfpathlineto{\pgfqpoint{0.429889in}{0.820095in}}%
\pgfpathlineto{\pgfqpoint{0.423866in}{0.826118in}}%
\pgfpathlineto{\pgfqpoint{0.417843in}{0.832141in}}%
\pgfpathlineto{\pgfqpoint{0.411820in}{0.838165in}}%
\pgfpathlineto{\pgfqpoint{0.405797in}{0.844188in}}%
\pgfpathlineto{\pgfqpoint{0.399774in}{0.850211in}}%
\pgfpathlineto{\pgfqpoint{0.393751in}{0.856234in}}%
\pgfpathlineto{\pgfqpoint{0.387728in}{0.862257in}}%
\pgfpathlineto{\pgfqpoint{0.381704in}{0.868280in}}%
\pgfpathlineto{\pgfqpoint{0.375681in}{0.874303in}}%
\pgfpathlineto{\pgfqpoint{0.369658in}{0.880326in}}%
\pgfpathlineto{\pgfqpoint{0.363635in}{0.886349in}}%
\pgfpathlineto{\pgfqpoint{0.357612in}{0.892372in}}%
\pgfpathlineto{\pgfqpoint{0.351589in}{0.898395in}}%
\pgfpathlineto{\pgfqpoint{0.345566in}{0.904418in}}%
\pgfpathlineto{\pgfqpoint{0.339543in}{0.910441in}}%
\pgfpathlineto{\pgfqpoint{0.333520in}{0.916464in}}%
\pgfpathlineto{\pgfqpoint{0.327497in}{0.922487in}}%
\pgfpathlineto{\pgfqpoint{0.321474in}{0.928510in}}%
\pgfpathlineto{\pgfqpoint{0.315451in}{0.934533in}}%
\pgfpathlineto{\pgfqpoint{0.309428in}{0.940557in}}%
\pgfpathlineto{\pgfqpoint{0.303405in}{0.946580in}}%
\pgfpathlineto{\pgfqpoint{0.297382in}{0.952603in}}%
\pgfpathlineto{\pgfqpoint{0.291359in}{0.958626in}}%
\pgfpathlineto{\pgfqpoint{0.285336in}{0.964649in}}%
\pgfpathlineto{\pgfqpoint{0.279312in}{0.970672in}}%
\pgfpathlineto{\pgfqpoint{0.273289in}{0.976695in}}%
\pgfpathlineto{\pgfqpoint{0.267266in}{0.982718in}}%
\pgfpathlineto{\pgfqpoint{0.261243in}{0.988741in}}%
\pgfpathlineto{\pgfqpoint{0.255220in}{0.994764in}}%
\pgfpathlineto{\pgfqpoint{0.249197in}{1.000787in}}%
\pgfpathlineto{\pgfqpoint{0.243174in}{1.006810in}}%
\pgfpathlineto{\pgfqpoint{0.237151in}{1.012833in}}%
\pgfpathlineto{\pgfqpoint{0.231128in}{1.018856in}}%
\pgfpathlineto{\pgfqpoint{0.225105in}{1.024879in}}%
\pgfpathlineto{\pgfqpoint{0.219082in}{1.030902in}}%
\pgfpathlineto{\pgfqpoint{0.213059in}{1.036925in}}%
\pgfpathlineto{\pgfqpoint{0.207036in}{1.042949in}}%
\pgfpathlineto{\pgfqpoint{0.201013in}{1.048972in}}%
\pgfpathlineto{\pgfqpoint{0.194990in}{1.054995in}}%
\pgfpathlineto{\pgfqpoint{0.188967in}{1.061018in}}%
\pgfpathlineto{\pgfqpoint{0.182944in}{1.067041in}}%
\pgfpathlineto{\pgfqpoint{0.176920in}{1.073064in}}%
\pgfpathlineto{\pgfqpoint{0.176920in}{1.085110in}}%
\pgfpathlineto{\pgfqpoint{0.182944in}{1.091133in}}%
\pgfpathlineto{\pgfqpoint{0.188967in}{1.097156in}}%
\pgfpathlineto{\pgfqpoint{0.194990in}{1.103179in}}%
\pgfpathlineto{\pgfqpoint{0.201013in}{1.109202in}}%
\pgfpathlineto{\pgfqpoint{0.207036in}{1.115225in}}%
\pgfpathlineto{\pgfqpoint{0.213059in}{1.121248in}}%
\pgfpathlineto{\pgfqpoint{0.219082in}{1.127271in}}%
\pgfpathlineto{\pgfqpoint{0.225105in}{1.133294in}}%
\pgfpathlineto{\pgfqpoint{0.231128in}{1.139317in}}%
\pgfpathlineto{\pgfqpoint{0.237151in}{1.145340in}}%
\pgfpathlineto{\pgfqpoint{0.243174in}{1.151364in}}%
\pgfpathlineto{\pgfqpoint{0.249197in}{1.157387in}}%
\pgfpathlineto{\pgfqpoint{0.255220in}{1.163410in}}%
\pgfpathlineto{\pgfqpoint{0.261243in}{1.169433in}}%
\pgfpathlineto{\pgfqpoint{0.267266in}{1.175456in}}%
\pgfpathlineto{\pgfqpoint{0.273289in}{1.181479in}}%
\pgfpathlineto{\pgfqpoint{0.279312in}{1.187502in}}%
\pgfpathlineto{\pgfqpoint{0.285336in}{1.193525in}}%
\pgfpathlineto{\pgfqpoint{0.291359in}{1.199548in}}%
\pgfpathlineto{\pgfqpoint{0.297382in}{1.205571in}}%
\pgfpathlineto{\pgfqpoint{0.303405in}{1.211594in}}%
\pgfpathlineto{\pgfqpoint{0.309428in}{1.217617in}}%
\pgfpathlineto{\pgfqpoint{0.315451in}{1.223640in}}%
\pgfpathlineto{\pgfqpoint{0.321474in}{1.229663in}}%
\pgfpathlineto{\pgfqpoint{0.327497in}{1.235686in}}%
\pgfpathlineto{\pgfqpoint{0.333520in}{1.241709in}}%
\pgfpathlineto{\pgfqpoint{0.339543in}{1.247732in}}%
\pgfpathlineto{\pgfqpoint{0.345566in}{1.253756in}}%
\pgfpathlineto{\pgfqpoint{0.351589in}{1.259779in}}%
\pgfpathlineto{\pgfqpoint{0.357612in}{1.265802in}}%
\pgfpathlineto{\pgfqpoint{0.363635in}{1.271825in}}%
\pgfpathlineto{\pgfqpoint{0.369658in}{1.277848in}}%
\pgfpathlineto{\pgfqpoint{0.375681in}{1.283871in}}%
\pgfpathlineto{\pgfqpoint{0.381704in}{1.289894in}}%
\pgfpathlineto{\pgfqpoint{0.387728in}{1.295917in}}%
\pgfpathlineto{\pgfqpoint{0.393751in}{1.301940in}}%
\pgfpathlineto{\pgfqpoint{0.399774in}{1.307963in}}%
\pgfpathlineto{\pgfqpoint{0.405797in}{1.313986in}}%
\pgfpathlineto{\pgfqpoint{0.411820in}{1.320009in}}%
\pgfpathlineto{\pgfqpoint{0.417843in}{1.326032in}}%
\pgfpathlineto{\pgfqpoint{0.423866in}{1.332055in}}%
\pgfpathlineto{\pgfqpoint{0.429889in}{1.338078in}}%
\pgfpathlineto{\pgfqpoint{0.435912in}{1.344101in}}%
\pgfpathlineto{\pgfqpoint{0.441935in}{1.350124in}}%
\pgfpathlineto{\pgfqpoint{0.447958in}{1.356148in}}%
\pgfpathlineto{\pgfqpoint{0.453981in}{1.362171in}}%
\pgfpathlineto{\pgfqpoint{0.460004in}{1.368194in}}%
\pgfpathlineto{\pgfqpoint{0.466027in}{1.374217in}}%
\pgfpathlineto{\pgfqpoint{0.472050in}{1.380240in}}%
\pgfpathlineto{\pgfqpoint{0.478073in}{1.386263in}}%
\pgfpathlineto{\pgfqpoint{0.484096in}{1.392286in}}%
\pgfpathlineto{\pgfqpoint{0.490119in}{1.398309in}}%
\pgfpathlineto{\pgfqpoint{0.496143in}{1.404332in}}%
\pgfpathlineto{\pgfqpoint{0.502166in}{1.410355in}}%
\pgfpathlineto{\pgfqpoint{0.508189in}{1.416378in}}%
\pgfpathlineto{\pgfqpoint{0.514212in}{1.422401in}}%
\pgfpathlineto{\pgfqpoint{0.520235in}{1.428424in}}%
\pgfpathlineto{\pgfqpoint{0.526258in}{1.434447in}}%
\pgfpathlineto{\pgfqpoint{0.532281in}{1.440470in}}%
\pgfpathlineto{\pgfqpoint{0.538304in}{1.446493in}}%
\pgfpathlineto{\pgfqpoint{0.544327in}{1.452516in}}%
\pgfpathlineto{\pgfqpoint{0.550350in}{1.458539in}}%
\pgfpathlineto{\pgfqpoint{0.556373in}{1.464563in}}%
\pgfpathlineto{\pgfqpoint{0.562396in}{1.470586in}}%
\pgfpathlineto{\pgfqpoint{0.568419in}{1.476609in}}%
\pgfpathlineto{\pgfqpoint{0.574442in}{1.482632in}}%
\pgfpathlineto{\pgfqpoint{0.580465in}{1.488655in}}%
\pgfpathlineto{\pgfqpoint{0.586488in}{1.494678in}}%
\pgfpathlineto{\pgfqpoint{0.592511in}{1.500701in}}%
\pgfpathlineto{\pgfqpoint{0.598535in}{1.506724in}}%
\pgfpathlineto{\pgfqpoint{0.604558in}{1.512747in}}%
\pgfpathlineto{\pgfqpoint{0.610581in}{1.518770in}}%
\pgfpathlineto{\pgfqpoint{0.616604in}{1.524793in}}%
\pgfpathlineto{\pgfqpoint{0.622627in}{1.530816in}}%
\pgfpathlineto{\pgfqpoint{0.628650in}{1.536839in}}%
\pgfpathlineto{\pgfqpoint{0.634673in}{1.542862in}}%
\pgfpathlineto{\pgfqpoint{0.640696in}{1.548885in}}%
\pgfpathlineto{\pgfqpoint{0.646719in}{1.554908in}}%
\pgfpathlineto{\pgfqpoint{0.652742in}{1.560931in}}%
\pgfpathlineto{\pgfqpoint{0.658765in}{1.566955in}}%
\pgfpathlineto{\pgfqpoint{0.664788in}{1.572978in}}%
\pgfpathlineto{\pgfqpoint{0.670811in}{1.579001in}}%
\pgfpathlineto{\pgfqpoint{0.676834in}{1.585024in}}%
\pgfpathlineto{\pgfqpoint{0.682857in}{1.591047in}}%
\pgfpathlineto{\pgfqpoint{0.688880in}{1.597070in}}%
\pgfpathlineto{\pgfqpoint{0.694903in}{1.603093in}}%
\pgfpathlineto{\pgfqpoint{0.700927in}{1.609116in}}%
\pgfpathlineto{\pgfqpoint{0.706950in}{1.615139in}}%
\pgfpathlineto{\pgfqpoint{0.712973in}{1.621162in}}%
\pgfpathlineto{\pgfqpoint{0.718996in}{1.627185in}}%
\pgfpathlineto{\pgfqpoint{0.725019in}{1.633208in}}%
\pgfpathlineto{\pgfqpoint{0.731042in}{1.639231in}}%
\pgfpathlineto{\pgfqpoint{0.737065in}{1.645254in}}%
\pgfpathlineto{\pgfqpoint{0.743088in}{1.651277in}}%
\pgfpathlineto{\pgfqpoint{0.749111in}{1.657300in}}%
\pgfpathlineto{\pgfqpoint{0.755134in}{1.663323in}}%
\pgfpathlineto{\pgfqpoint{0.761157in}{1.669347in}}%
\pgfpathlineto{\pgfqpoint{0.767180in}{1.675370in}}%
\pgfpathlineto{\pgfqpoint{0.773203in}{1.681393in}}%
\pgfpathlineto{\pgfqpoint{0.779226in}{1.687416in}}%
\pgfpathlineto{\pgfqpoint{0.785249in}{1.693439in}}%
\pgfpathlineto{\pgfqpoint{0.791272in}{1.699462in}}%
\pgfpathlineto{\pgfqpoint{0.797295in}{1.705485in}}%
\pgfpathlineto{\pgfqpoint{0.803318in}{1.711508in}}%
\pgfpathlineto{\pgfqpoint{0.809342in}{1.717531in}}%
\pgfpathlineto{\pgfqpoint{0.815365in}{1.723554in}}%
\pgfpathlineto{\pgfqpoint{0.821388in}{1.729577in}}%
\pgfpathlineto{\pgfqpoint{0.827411in}{1.735600in}}%
\pgfpathlineto{\pgfqpoint{0.833434in}{1.741623in}}%
\pgfpathlineto{\pgfqpoint{0.839457in}{1.747646in}}%
\pgfpathlineto{\pgfqpoint{0.845480in}{1.753669in}}%
\pgfpathlineto{\pgfqpoint{0.851503in}{1.759692in}}%
\pgfpathlineto{\pgfqpoint{0.857526in}{1.765715in}}%
\pgfpathlineto{\pgfqpoint{0.863549in}{1.771738in}}%
\pgfpathlineto{\pgfqpoint{0.869572in}{1.777762in}}%
\pgfpathlineto{\pgfqpoint{0.875595in}{1.783785in}}%
\pgfpathlineto{\pgfqpoint{0.881618in}{1.789808in}}%
\pgfpathlineto{\pgfqpoint{0.887641in}{1.795831in}}%
\pgfpathlineto{\pgfqpoint{0.893664in}{1.801854in}}%
\pgfpathlineto{\pgfqpoint{0.899687in}{1.807877in}}%
\pgfpathlineto{\pgfqpoint{0.905710in}{1.813900in}}%
\pgfpathlineto{\pgfqpoint{0.911734in}{1.819923in}}%
\pgfpathlineto{\pgfqpoint{0.917757in}{1.825946in}}%
\pgfpathlineto{\pgfqpoint{0.923780in}{1.831969in}}%
\pgfpathlineto{\pgfqpoint{0.929803in}{1.837992in}}%
\pgfpathlineto{\pgfqpoint{0.935826in}{1.844015in}}%
\pgfpathlineto{\pgfqpoint{0.941849in}{1.850038in}}%
\pgfpathlineto{\pgfqpoint{0.947872in}{1.856061in}}%
\pgfpathlineto{\pgfqpoint{0.953895in}{1.862084in}}%
\pgfpathlineto{\pgfqpoint{0.959918in}{1.868107in}}%
\pgfpathlineto{\pgfqpoint{0.965941in}{1.874130in}}%
\pgfpathlineto{\pgfqpoint{0.971964in}{1.880154in}}%
\pgfpathlineto{\pgfqpoint{0.977987in}{1.886177in}}%
\pgfpathlineto{\pgfqpoint{0.984010in}{1.892200in}}%
\pgfpathlineto{\pgfqpoint{0.990033in}{1.898223in}}%
\pgfpathlineto{\pgfqpoint{0.996056in}{1.904246in}}%
\pgfpathlineto{\pgfqpoint{1.002079in}{1.910269in}}%
\pgfpathlineto{\pgfqpoint{1.008102in}{1.916292in}}%
\pgfpathlineto{\pgfqpoint{1.014126in}{1.922315in}}%
\pgfpathlineto{\pgfqpoint{1.020149in}{1.928338in}}%
\pgfpathlineto{\pgfqpoint{1.026172in}{1.934361in}}%
\pgfpathlineto{\pgfqpoint{1.032195in}{1.940384in}}%
\pgfpathlineto{\pgfqpoint{1.038218in}{1.946407in}}%
\pgfpathlineto{\pgfqpoint{1.044241in}{1.952430in}}%
\pgfpathlineto{\pgfqpoint{1.050264in}{1.958453in}}%
\pgfpathlineto{\pgfqpoint{1.056287in}{1.964476in}}%
\pgfpathlineto{\pgfqpoint{1.062310in}{1.970499in}}%
\pgfpathlineto{\pgfqpoint{1.074356in}{1.970499in}}%
\pgfpathlineto{\pgfqpoint{1.080379in}{1.964476in}}%
\pgfpathlineto{\pgfqpoint{1.086402in}{1.958453in}}%
\pgfpathlineto{\pgfqpoint{1.092425in}{1.952430in}}%
\pgfpathlineto{\pgfqpoint{1.098448in}{1.946407in}}%
\pgfpathlineto{\pgfqpoint{1.104471in}{1.940384in}}%
\pgfpathlineto{\pgfqpoint{1.110494in}{1.934361in}}%
\pgfpathlineto{\pgfqpoint{1.116517in}{1.928338in}}%
\pgfpathlineto{\pgfqpoint{1.122541in}{1.922315in}}%
\pgfpathlineto{\pgfqpoint{1.128564in}{1.916292in}}%
\pgfpathlineto{\pgfqpoint{1.134587in}{1.910269in}}%
\pgfpathlineto{\pgfqpoint{1.140610in}{1.904246in}}%
\pgfpathlineto{\pgfqpoint{1.146633in}{1.898223in}}%
\pgfpathlineto{\pgfqpoint{1.152656in}{1.892200in}}%
\pgfpathlineto{\pgfqpoint{1.158679in}{1.886177in}}%
\pgfpathlineto{\pgfqpoint{1.164702in}{1.880154in}}%
\pgfpathlineto{\pgfqpoint{1.170725in}{1.874130in}}%
\pgfpathlineto{\pgfqpoint{1.176748in}{1.868107in}}%
\pgfpathlineto{\pgfqpoint{1.182771in}{1.862084in}}%
\pgfpathlineto{\pgfqpoint{1.188794in}{1.856061in}}%
\pgfpathlineto{\pgfqpoint{1.194817in}{1.850038in}}%
\pgfpathlineto{\pgfqpoint{1.200840in}{1.844015in}}%
\pgfpathlineto{\pgfqpoint{1.206863in}{1.837992in}}%
\pgfpathlineto{\pgfqpoint{1.212886in}{1.831969in}}%
\pgfpathlineto{\pgfqpoint{1.218909in}{1.825946in}}%
\pgfpathlineto{\pgfqpoint{1.224933in}{1.819923in}}%
\pgfpathlineto{\pgfqpoint{1.230956in}{1.813900in}}%
\pgfpathlineto{\pgfqpoint{1.236979in}{1.807877in}}%
\pgfpathlineto{\pgfqpoint{1.243002in}{1.801854in}}%
\pgfpathlineto{\pgfqpoint{1.249025in}{1.795831in}}%
\pgfpathlineto{\pgfqpoint{1.255048in}{1.789808in}}%
\pgfpathlineto{\pgfqpoint{1.261071in}{1.783785in}}%
\pgfpathlineto{\pgfqpoint{1.267094in}{1.777762in}}%
\pgfpathlineto{\pgfqpoint{1.273117in}{1.771738in}}%
\pgfpathlineto{\pgfqpoint{1.279140in}{1.765715in}}%
\pgfpathlineto{\pgfqpoint{1.285163in}{1.759692in}}%
\pgfpathlineto{\pgfqpoint{1.291186in}{1.753669in}}%
\pgfpathlineto{\pgfqpoint{1.297209in}{1.747646in}}%
\pgfpathlineto{\pgfqpoint{1.303232in}{1.741623in}}%
\pgfpathlineto{\pgfqpoint{1.309255in}{1.735600in}}%
\pgfpathlineto{\pgfqpoint{1.315278in}{1.729577in}}%
\pgfpathlineto{\pgfqpoint{1.321301in}{1.723554in}}%
\pgfpathlineto{\pgfqpoint{1.327325in}{1.717531in}}%
\pgfpathlineto{\pgfqpoint{1.333348in}{1.711508in}}%
\pgfpathlineto{\pgfqpoint{1.339371in}{1.705485in}}%
\pgfpathlineto{\pgfqpoint{1.345394in}{1.699462in}}%
\pgfpathlineto{\pgfqpoint{1.351417in}{1.693439in}}%
\pgfpathlineto{\pgfqpoint{1.357440in}{1.687416in}}%
\pgfpathlineto{\pgfqpoint{1.363463in}{1.681393in}}%
\pgfpathlineto{\pgfqpoint{1.369486in}{1.675370in}}%
\pgfpathlineto{\pgfqpoint{1.375509in}{1.669347in}}%
\pgfpathlineto{\pgfqpoint{1.381532in}{1.663323in}}%
\pgfpathlineto{\pgfqpoint{1.387555in}{1.657300in}}%
\pgfpathlineto{\pgfqpoint{1.393578in}{1.651277in}}%
\pgfpathlineto{\pgfqpoint{1.399601in}{1.645254in}}%
\pgfpathlineto{\pgfqpoint{1.405624in}{1.639231in}}%
\pgfpathlineto{\pgfqpoint{1.411647in}{1.633208in}}%
\pgfpathlineto{\pgfqpoint{1.417670in}{1.627185in}}%
\pgfpathlineto{\pgfqpoint{1.423693in}{1.621162in}}%
\pgfpathlineto{\pgfqpoint{1.429716in}{1.615139in}}%
\pgfpathlineto{\pgfqpoint{1.435740in}{1.609116in}}%
\pgfpathlineto{\pgfqpoint{1.441763in}{1.603093in}}%
\pgfpathlineto{\pgfqpoint{1.447786in}{1.597070in}}%
\pgfpathlineto{\pgfqpoint{1.453809in}{1.591047in}}%
\pgfpathlineto{\pgfqpoint{1.459832in}{1.585024in}}%
\pgfpathlineto{\pgfqpoint{1.465855in}{1.579001in}}%
\pgfpathlineto{\pgfqpoint{1.471878in}{1.572978in}}%
\pgfpathlineto{\pgfqpoint{1.477901in}{1.566955in}}%
\pgfpathlineto{\pgfqpoint{1.483924in}{1.560931in}}%
\pgfpathlineto{\pgfqpoint{1.489947in}{1.554908in}}%
\pgfpathlineto{\pgfqpoint{1.495970in}{1.548885in}}%
\pgfpathlineto{\pgfqpoint{1.501993in}{1.542862in}}%
\pgfpathlineto{\pgfqpoint{1.508016in}{1.536839in}}%
\pgfpathlineto{\pgfqpoint{1.514039in}{1.530816in}}%
\pgfpathlineto{\pgfqpoint{1.520062in}{1.524793in}}%
\pgfpathlineto{\pgfqpoint{1.526085in}{1.518770in}}%
\pgfpathlineto{\pgfqpoint{1.532108in}{1.512747in}}%
\pgfpathlineto{\pgfqpoint{1.538132in}{1.506724in}}%
\pgfpathlineto{\pgfqpoint{1.544155in}{1.500701in}}%
\pgfpathlineto{\pgfqpoint{1.550178in}{1.494678in}}%
\pgfpathlineto{\pgfqpoint{1.556201in}{1.488655in}}%
\pgfpathlineto{\pgfqpoint{1.562224in}{1.482632in}}%
\pgfpathlineto{\pgfqpoint{1.568247in}{1.476609in}}%
\pgfpathlineto{\pgfqpoint{1.574270in}{1.470586in}}%
\pgfpathlineto{\pgfqpoint{1.580293in}{1.464563in}}%
\pgfpathlineto{\pgfqpoint{1.586316in}{1.458539in}}%
\pgfpathlineto{\pgfqpoint{1.592339in}{1.452516in}}%
\pgfpathlineto{\pgfqpoint{1.598362in}{1.446493in}}%
\pgfpathlineto{\pgfqpoint{1.604385in}{1.440470in}}%
\pgfpathlineto{\pgfqpoint{1.610408in}{1.434447in}}%
\pgfpathlineto{\pgfqpoint{1.616431in}{1.428424in}}%
\pgfpathlineto{\pgfqpoint{1.622454in}{1.422401in}}%
\pgfpathlineto{\pgfqpoint{1.628477in}{1.416378in}}%
\pgfpathlineto{\pgfqpoint{1.634500in}{1.410355in}}%
\pgfpathlineto{\pgfqpoint{1.640524in}{1.404332in}}%
\pgfpathlineto{\pgfqpoint{1.646547in}{1.398309in}}%
\pgfpathlineto{\pgfqpoint{1.652570in}{1.392286in}}%
\pgfpathlineto{\pgfqpoint{1.658593in}{1.386263in}}%
\pgfpathlineto{\pgfqpoint{1.664616in}{1.380240in}}%
\pgfpathlineto{\pgfqpoint{1.670639in}{1.374217in}}%
\pgfpathlineto{\pgfqpoint{1.676662in}{1.368194in}}%
\pgfpathlineto{\pgfqpoint{1.682685in}{1.362171in}}%
\pgfpathlineto{\pgfqpoint{1.688708in}{1.356148in}}%
\pgfpathlineto{\pgfqpoint{1.694731in}{1.350124in}}%
\pgfpathlineto{\pgfqpoint{1.700754in}{1.344101in}}%
\pgfpathlineto{\pgfqpoint{1.706777in}{1.338078in}}%
\pgfpathlineto{\pgfqpoint{1.712800in}{1.332055in}}%
\pgfpathlineto{\pgfqpoint{1.718823in}{1.326032in}}%
\pgfpathlineto{\pgfqpoint{1.724846in}{1.320009in}}%
\pgfpathlineto{\pgfqpoint{1.730869in}{1.313986in}}%
\pgfpathlineto{\pgfqpoint{1.736892in}{1.307963in}}%
\pgfpathlineto{\pgfqpoint{1.742915in}{1.301940in}}%
\pgfpathlineto{\pgfqpoint{1.748939in}{1.295917in}}%
\pgfpathlineto{\pgfqpoint{1.754962in}{1.289894in}}%
\pgfpathlineto{\pgfqpoint{1.760985in}{1.283871in}}%
\pgfpathlineto{\pgfqpoint{1.767008in}{1.277848in}}%
\pgfpathlineto{\pgfqpoint{1.773031in}{1.271825in}}%
\pgfpathlineto{\pgfqpoint{1.779054in}{1.265802in}}%
\pgfpathlineto{\pgfqpoint{1.785077in}{1.259779in}}%
\pgfpathlineto{\pgfqpoint{1.791100in}{1.253756in}}%
\pgfpathlineto{\pgfqpoint{1.797123in}{1.247732in}}%
\pgfpathlineto{\pgfqpoint{1.803146in}{1.241709in}}%
\pgfpathlineto{\pgfqpoint{1.809169in}{1.235686in}}%
\pgfpathlineto{\pgfqpoint{1.815192in}{1.229663in}}%
\pgfpathlineto{\pgfqpoint{1.821215in}{1.223640in}}%
\pgfpathlineto{\pgfqpoint{1.827238in}{1.217617in}}%
\pgfpathlineto{\pgfqpoint{1.833261in}{1.211594in}}%
\pgfpathlineto{\pgfqpoint{1.839284in}{1.205571in}}%
\pgfpathlineto{\pgfqpoint{1.845307in}{1.199548in}}%
\pgfpathlineto{\pgfqpoint{1.851331in}{1.193525in}}%
\pgfpathlineto{\pgfqpoint{1.857354in}{1.187502in}}%
\pgfpathlineto{\pgfqpoint{1.863377in}{1.181479in}}%
\pgfpathlineto{\pgfqpoint{1.869400in}{1.175456in}}%
\pgfpathlineto{\pgfqpoint{1.875423in}{1.169433in}}%
\pgfpathlineto{\pgfqpoint{1.881446in}{1.163410in}}%
\pgfpathlineto{\pgfqpoint{1.887469in}{1.157387in}}%
\pgfpathlineto{\pgfqpoint{1.893492in}{1.151364in}}%
\pgfpathlineto{\pgfqpoint{1.899515in}{1.145340in}}%
\pgfpathlineto{\pgfqpoint{1.905538in}{1.139317in}}%
\pgfpathlineto{\pgfqpoint{1.911561in}{1.133294in}}%
\pgfpathlineto{\pgfqpoint{1.917584in}{1.127271in}}%
\pgfpathlineto{\pgfqpoint{1.923607in}{1.121248in}}%
\pgfpathlineto{\pgfqpoint{1.929630in}{1.115225in}}%
\pgfpathlineto{\pgfqpoint{1.935653in}{1.109202in}}%
\pgfpathlineto{\pgfqpoint{1.941676in}{1.103179in}}%
\pgfpathlineto{\pgfqpoint{1.947699in}{1.097156in}}%
\pgfpathlineto{\pgfqpoint{1.953723in}{1.091133in}}%
\pgfpathlineto{\pgfqpoint{1.959746in}{1.085110in}}%
\pgfpathlineto{\pgfqpoint{1.959746in}{1.073064in}}%
\pgfpathlineto{\pgfqpoint{1.953723in}{1.067041in}}%
\pgfpathlineto{\pgfqpoint{1.947699in}{1.061018in}}%
\pgfpathlineto{\pgfqpoint{1.941676in}{1.054995in}}%
\pgfpathlineto{\pgfqpoint{1.935653in}{1.048972in}}%
\pgfpathlineto{\pgfqpoint{1.929630in}{1.042949in}}%
\pgfpathlineto{\pgfqpoint{1.923607in}{1.036925in}}%
\pgfpathlineto{\pgfqpoint{1.917584in}{1.030902in}}%
\pgfpathlineto{\pgfqpoint{1.911561in}{1.024879in}}%
\pgfpathlineto{\pgfqpoint{1.905538in}{1.018856in}}%
\pgfpathlineto{\pgfqpoint{1.899515in}{1.012833in}}%
\pgfpathlineto{\pgfqpoint{1.893492in}{1.006810in}}%
\pgfpathlineto{\pgfqpoint{1.887469in}{1.000787in}}%
\pgfpathlineto{\pgfqpoint{1.881446in}{0.994764in}}%
\pgfpathlineto{\pgfqpoint{1.875423in}{0.988741in}}%
\pgfpathlineto{\pgfqpoint{1.869400in}{0.982718in}}%
\pgfpathlineto{\pgfqpoint{1.863377in}{0.976695in}}%
\pgfpathlineto{\pgfqpoint{1.857354in}{0.970672in}}%
\pgfpathlineto{\pgfqpoint{1.851331in}{0.964649in}}%
\pgfpathlineto{\pgfqpoint{1.845307in}{0.958626in}}%
\pgfpathlineto{\pgfqpoint{1.839284in}{0.952603in}}%
\pgfpathlineto{\pgfqpoint{1.833261in}{0.946580in}}%
\pgfpathlineto{\pgfqpoint{1.827238in}{0.940557in}}%
\pgfpathlineto{\pgfqpoint{1.821215in}{0.934533in}}%
\pgfpathlineto{\pgfqpoint{1.815192in}{0.928510in}}%
\pgfpathlineto{\pgfqpoint{1.809169in}{0.922487in}}%
\pgfpathlineto{\pgfqpoint{1.803146in}{0.916464in}}%
\pgfpathlineto{\pgfqpoint{1.797123in}{0.910441in}}%
\pgfpathlineto{\pgfqpoint{1.791100in}{0.904418in}}%
\pgfpathlineto{\pgfqpoint{1.785077in}{0.898395in}}%
\pgfpathlineto{\pgfqpoint{1.779054in}{0.892372in}}%
\pgfpathlineto{\pgfqpoint{1.773031in}{0.886349in}}%
\pgfpathlineto{\pgfqpoint{1.767008in}{0.880326in}}%
\pgfpathlineto{\pgfqpoint{1.760985in}{0.874303in}}%
\pgfpathlineto{\pgfqpoint{1.754962in}{0.868280in}}%
\pgfpathlineto{\pgfqpoint{1.748939in}{0.862257in}}%
\pgfpathlineto{\pgfqpoint{1.742915in}{0.856234in}}%
\pgfpathlineto{\pgfqpoint{1.736892in}{0.850211in}}%
\pgfpathlineto{\pgfqpoint{1.730869in}{0.844188in}}%
\pgfpathlineto{\pgfqpoint{1.724846in}{0.838165in}}%
\pgfpathlineto{\pgfqpoint{1.718823in}{0.832141in}}%
\pgfpathlineto{\pgfqpoint{1.712800in}{0.826118in}}%
\pgfpathlineto{\pgfqpoint{1.706777in}{0.820095in}}%
\pgfpathlineto{\pgfqpoint{1.700754in}{0.814072in}}%
\pgfpathlineto{\pgfqpoint{1.694731in}{0.808049in}}%
\pgfpathlineto{\pgfqpoint{1.688708in}{0.802026in}}%
\pgfpathlineto{\pgfqpoint{1.682685in}{0.796003in}}%
\pgfpathlineto{\pgfqpoint{1.676662in}{0.789980in}}%
\pgfpathlineto{\pgfqpoint{1.670639in}{0.783957in}}%
\pgfpathlineto{\pgfqpoint{1.664616in}{0.777934in}}%
\pgfpathlineto{\pgfqpoint{1.658593in}{0.771911in}}%
\pgfpathlineto{\pgfqpoint{1.652570in}{0.765888in}}%
\pgfpathlineto{\pgfqpoint{1.646547in}{0.759865in}}%
\pgfpathlineto{\pgfqpoint{1.640524in}{0.753842in}}%
\pgfpathlineto{\pgfqpoint{1.634500in}{0.747819in}}%
\pgfpathlineto{\pgfqpoint{1.628477in}{0.741796in}}%
\pgfpathlineto{\pgfqpoint{1.622454in}{0.735773in}}%
\pgfpathlineto{\pgfqpoint{1.616431in}{0.729750in}}%
\pgfpathlineto{\pgfqpoint{1.610408in}{0.723726in}}%
\pgfpathlineto{\pgfqpoint{1.604385in}{0.717703in}}%
\pgfpathlineto{\pgfqpoint{1.598362in}{0.711680in}}%
\pgfpathlineto{\pgfqpoint{1.592339in}{0.705657in}}%
\pgfpathlineto{\pgfqpoint{1.586316in}{0.699634in}}%
\pgfpathlineto{\pgfqpoint{1.580293in}{0.693611in}}%
\pgfpathlineto{\pgfqpoint{1.574270in}{0.687588in}}%
\pgfpathlineto{\pgfqpoint{1.568247in}{0.681565in}}%
\pgfpathlineto{\pgfqpoint{1.562224in}{0.675542in}}%
\pgfpathlineto{\pgfqpoint{1.556201in}{0.669519in}}%
\pgfpathlineto{\pgfqpoint{1.550178in}{0.663496in}}%
\pgfpathlineto{\pgfqpoint{1.544155in}{0.657473in}}%
\pgfpathlineto{\pgfqpoint{1.538132in}{0.651450in}}%
\pgfpathlineto{\pgfqpoint{1.532108in}{0.645427in}}%
\pgfpathlineto{\pgfqpoint{1.526085in}{0.639404in}}%
\pgfpathlineto{\pgfqpoint{1.520062in}{0.633381in}}%
\pgfpathlineto{\pgfqpoint{1.514039in}{0.627358in}}%
\pgfpathlineto{\pgfqpoint{1.508016in}{0.621334in}}%
\pgfpathlineto{\pgfqpoint{1.501993in}{0.615311in}}%
\pgfpathlineto{\pgfqpoint{1.495970in}{0.609288in}}%
\pgfpathlineto{\pgfqpoint{1.489947in}{0.603265in}}%
\pgfpathlineto{\pgfqpoint{1.483924in}{0.597242in}}%
\pgfpathlineto{\pgfqpoint{1.477901in}{0.591219in}}%
\pgfpathlineto{\pgfqpoint{1.471878in}{0.585196in}}%
\pgfpathlineto{\pgfqpoint{1.465855in}{0.579173in}}%
\pgfpathlineto{\pgfqpoint{1.459832in}{0.573150in}}%
\pgfpathlineto{\pgfqpoint{1.453809in}{0.567127in}}%
\pgfpathlineto{\pgfqpoint{1.447786in}{0.561104in}}%
\pgfpathlineto{\pgfqpoint{1.441763in}{0.555081in}}%
\pgfpathlineto{\pgfqpoint{1.435740in}{0.549058in}}%
\pgfpathlineto{\pgfqpoint{1.429716in}{0.543035in}}%
\pgfpathlineto{\pgfqpoint{1.423693in}{0.537012in}}%
\pgfpathlineto{\pgfqpoint{1.417670in}{0.530989in}}%
\pgfpathlineto{\pgfqpoint{1.411647in}{0.524966in}}%
\pgfpathlineto{\pgfqpoint{1.405624in}{0.518942in}}%
\pgfpathlineto{\pgfqpoint{1.399601in}{0.512919in}}%
\pgfpathlineto{\pgfqpoint{1.393578in}{0.506896in}}%
\pgfpathlineto{\pgfqpoint{1.387555in}{0.500873in}}%
\pgfpathlineto{\pgfqpoint{1.381532in}{0.494850in}}%
\pgfpathlineto{\pgfqpoint{1.375509in}{0.488827in}}%
\pgfpathlineto{\pgfqpoint{1.369486in}{0.482804in}}%
\pgfpathlineto{\pgfqpoint{1.363463in}{0.476781in}}%
\pgfpathlineto{\pgfqpoint{1.357440in}{0.470758in}}%
\pgfpathlineto{\pgfqpoint{1.351417in}{0.464735in}}%
\pgfpathlineto{\pgfqpoint{1.345394in}{0.458712in}}%
\pgfpathlineto{\pgfqpoint{1.339371in}{0.452689in}}%
\pgfpathlineto{\pgfqpoint{1.333348in}{0.446666in}}%
\pgfpathlineto{\pgfqpoint{1.327325in}{0.440643in}}%
\pgfpathlineto{\pgfqpoint{1.321301in}{0.434620in}}%
\pgfpathlineto{\pgfqpoint{1.315278in}{0.428597in}}%
\pgfpathlineto{\pgfqpoint{1.309255in}{0.422574in}}%
\pgfpathlineto{\pgfqpoint{1.303232in}{0.416551in}}%
\pgfpathlineto{\pgfqpoint{1.297209in}{0.410527in}}%
\pgfpathlineto{\pgfqpoint{1.291186in}{0.404504in}}%
\pgfpathlineto{\pgfqpoint{1.285163in}{0.398481in}}%
\pgfpathlineto{\pgfqpoint{1.279140in}{0.392458in}}%
\pgfpathlineto{\pgfqpoint{1.273117in}{0.386435in}}%
\pgfpathlineto{\pgfqpoint{1.267094in}{0.380412in}}%
\pgfpathlineto{\pgfqpoint{1.261071in}{0.374389in}}%
\pgfpathlineto{\pgfqpoint{1.255048in}{0.368366in}}%
\pgfpathlineto{\pgfqpoint{1.249025in}{0.362343in}}%
\pgfpathlineto{\pgfqpoint{1.243002in}{0.356320in}}%
\pgfpathlineto{\pgfqpoint{1.236979in}{0.350297in}}%
\pgfpathlineto{\pgfqpoint{1.230956in}{0.344274in}}%
\pgfpathlineto{\pgfqpoint{1.224933in}{0.338251in}}%
\pgfpathlineto{\pgfqpoint{1.218909in}{0.332228in}}%
\pgfpathlineto{\pgfqpoint{1.212886in}{0.326205in}}%
\pgfpathlineto{\pgfqpoint{1.206863in}{0.320182in}}%
\pgfpathlineto{\pgfqpoint{1.200840in}{0.314159in}}%
\pgfpathlineto{\pgfqpoint{1.194817in}{0.308135in}}%
\pgfpathlineto{\pgfqpoint{1.188794in}{0.302112in}}%
\pgfpathlineto{\pgfqpoint{1.182771in}{0.296089in}}%
\pgfpathlineto{\pgfqpoint{1.176748in}{0.290066in}}%
\pgfpathlineto{\pgfqpoint{1.170725in}{0.284043in}}%
\pgfpathlineto{\pgfqpoint{1.164702in}{0.278020in}}%
\pgfpathlineto{\pgfqpoint{1.158679in}{0.271997in}}%
\pgfpathlineto{\pgfqpoint{1.152656in}{0.265974in}}%
\pgfpathlineto{\pgfqpoint{1.146633in}{0.259951in}}%
\pgfpathlineto{\pgfqpoint{1.140610in}{0.253928in}}%
\pgfpathlineto{\pgfqpoint{1.134587in}{0.247905in}}%
\pgfpathlineto{\pgfqpoint{1.128564in}{0.241882in}}%
\pgfpathlineto{\pgfqpoint{1.122541in}{0.235859in}}%
\pgfpathlineto{\pgfqpoint{1.116517in}{0.229836in}}%
\pgfpathlineto{\pgfqpoint{1.110494in}{0.223813in}}%
\pgfpathlineto{\pgfqpoint{1.104471in}{0.217790in}}%
\pgfpathlineto{\pgfqpoint{1.098448in}{0.211767in}}%
\pgfpathlineto{\pgfqpoint{1.092425in}{0.205743in}}%
\pgfpathlineto{\pgfqpoint{1.086402in}{0.199720in}}%
\pgfpathlineto{\pgfqpoint{1.080379in}{0.193697in}}%
\pgfpathlineto{\pgfqpoint{1.074356in}{0.187674in}}%
\pgfpathlineto{\pgfqpoint{1.062310in}{0.187674in}}%
\pgfpathclose%
\pgfusepath{fill}%
\end{pgfscope}%
\begin{pgfscope}%
\pgfpathrectangle{\pgfqpoint{0.135000in}{0.145754in}}{\pgfqpoint{1.866666in}{1.866666in}} %
\pgfusepath{clip}%
\pgfsetbuttcap%
\pgfsetroundjoin%
\pgfsetlinewidth{0.000000pt}%
\definecolor{currentstroke}{rgb}{0.000000,0.000000,0.000000}%
\pgfsetstrokecolor{currentstroke}%
\pgfsetdash{}{0pt}%
\pgfpathmoveto{\pgfqpoint{0.182944in}{0.181651in}}%
\pgfpathlineto{\pgfqpoint{0.194990in}{0.181651in}}%
\pgfpathlineto{\pgfqpoint{0.207036in}{0.181651in}}%
\pgfpathlineto{\pgfqpoint{0.219082in}{0.181651in}}%
\pgfpathlineto{\pgfqpoint{0.231128in}{0.181651in}}%
\pgfpathlineto{\pgfqpoint{0.243174in}{0.181651in}}%
\pgfpathlineto{\pgfqpoint{0.255220in}{0.181651in}}%
\pgfpathlineto{\pgfqpoint{0.267266in}{0.181651in}}%
\pgfpathlineto{\pgfqpoint{0.279312in}{0.181651in}}%
\pgfpathlineto{\pgfqpoint{0.291359in}{0.181651in}}%
\pgfpathlineto{\pgfqpoint{0.303405in}{0.181651in}}%
\pgfpathlineto{\pgfqpoint{0.315451in}{0.181651in}}%
\pgfpathlineto{\pgfqpoint{0.327497in}{0.181651in}}%
\pgfpathlineto{\pgfqpoint{0.339543in}{0.181651in}}%
\pgfpathlineto{\pgfqpoint{0.351589in}{0.181651in}}%
\pgfpathlineto{\pgfqpoint{0.363635in}{0.181651in}}%
\pgfpathlineto{\pgfqpoint{0.375681in}{0.181651in}}%
\pgfpathlineto{\pgfqpoint{0.387728in}{0.181651in}}%
\pgfpathlineto{\pgfqpoint{0.399774in}{0.181651in}}%
\pgfpathlineto{\pgfqpoint{0.411820in}{0.181651in}}%
\pgfpathlineto{\pgfqpoint{0.423866in}{0.181651in}}%
\pgfpathlineto{\pgfqpoint{0.435912in}{0.181651in}}%
\pgfpathlineto{\pgfqpoint{0.447958in}{0.181651in}}%
\pgfpathlineto{\pgfqpoint{0.460004in}{0.181651in}}%
\pgfpathlineto{\pgfqpoint{0.472050in}{0.181651in}}%
\pgfpathlineto{\pgfqpoint{0.484096in}{0.181651in}}%
\pgfpathlineto{\pgfqpoint{0.496143in}{0.181651in}}%
\pgfpathlineto{\pgfqpoint{0.508189in}{0.181651in}}%
\pgfpathlineto{\pgfqpoint{0.520235in}{0.181651in}}%
\pgfpathlineto{\pgfqpoint{0.532281in}{0.181651in}}%
\pgfpathlineto{\pgfqpoint{0.544327in}{0.181651in}}%
\pgfpathlineto{\pgfqpoint{0.556373in}{0.181651in}}%
\pgfpathlineto{\pgfqpoint{0.568419in}{0.181651in}}%
\pgfpathlineto{\pgfqpoint{0.580465in}{0.181651in}}%
\pgfpathlineto{\pgfqpoint{0.592511in}{0.181651in}}%
\pgfpathlineto{\pgfqpoint{0.604558in}{0.181651in}}%
\pgfpathlineto{\pgfqpoint{0.616604in}{0.181651in}}%
\pgfpathlineto{\pgfqpoint{0.628650in}{0.181651in}}%
\pgfpathlineto{\pgfqpoint{0.640696in}{0.181651in}}%
\pgfpathlineto{\pgfqpoint{0.652742in}{0.181651in}}%
\pgfpathlineto{\pgfqpoint{0.664788in}{0.181651in}}%
\pgfpathlineto{\pgfqpoint{0.676834in}{0.181651in}}%
\pgfpathlineto{\pgfqpoint{0.688880in}{0.181651in}}%
\pgfpathlineto{\pgfqpoint{0.700927in}{0.181651in}}%
\pgfpathlineto{\pgfqpoint{0.712973in}{0.181651in}}%
\pgfpathlineto{\pgfqpoint{0.725019in}{0.181651in}}%
\pgfpathlineto{\pgfqpoint{0.737065in}{0.181651in}}%
\pgfpathlineto{\pgfqpoint{0.749111in}{0.181651in}}%
\pgfpathlineto{\pgfqpoint{0.761157in}{0.181651in}}%
\pgfpathlineto{\pgfqpoint{0.773203in}{0.181651in}}%
\pgfpathlineto{\pgfqpoint{0.785249in}{0.181651in}}%
\pgfpathlineto{\pgfqpoint{0.797295in}{0.181651in}}%
\pgfpathlineto{\pgfqpoint{0.809342in}{0.181651in}}%
\pgfpathlineto{\pgfqpoint{0.821388in}{0.181651in}}%
\pgfpathlineto{\pgfqpoint{0.833434in}{0.181651in}}%
\pgfpathlineto{\pgfqpoint{0.845480in}{0.181651in}}%
\pgfpathlineto{\pgfqpoint{0.857526in}{0.181651in}}%
\pgfpathlineto{\pgfqpoint{0.869572in}{0.181651in}}%
\pgfpathlineto{\pgfqpoint{0.881618in}{0.181651in}}%
\pgfpathlineto{\pgfqpoint{0.893664in}{0.181651in}}%
\pgfpathlineto{\pgfqpoint{0.905710in}{0.181651in}}%
\pgfpathlineto{\pgfqpoint{0.917757in}{0.181651in}}%
\pgfpathlineto{\pgfqpoint{0.929803in}{0.181651in}}%
\pgfpathlineto{\pgfqpoint{0.941849in}{0.181651in}}%
\pgfpathlineto{\pgfqpoint{0.953895in}{0.181651in}}%
\pgfpathlineto{\pgfqpoint{0.965941in}{0.181651in}}%
\pgfpathlineto{\pgfqpoint{0.977987in}{0.181651in}}%
\pgfpathlineto{\pgfqpoint{0.990033in}{0.181651in}}%
\pgfpathlineto{\pgfqpoint{1.002079in}{0.181651in}}%
\pgfpathlineto{\pgfqpoint{1.014126in}{0.181651in}}%
\pgfpathlineto{\pgfqpoint{1.026172in}{0.181651in}}%
\pgfpathlineto{\pgfqpoint{1.038218in}{0.181651in}}%
\pgfpathlineto{\pgfqpoint{1.050264in}{0.181651in}}%
\pgfpathlineto{\pgfqpoint{1.062310in}{0.181651in}}%
\pgfpathlineto{\pgfqpoint{1.074356in}{0.181651in}}%
\pgfpathlineto{\pgfqpoint{1.086402in}{0.181651in}}%
\pgfpathlineto{\pgfqpoint{1.098448in}{0.181651in}}%
\pgfpathlineto{\pgfqpoint{1.110494in}{0.181651in}}%
\pgfpathlineto{\pgfqpoint{1.122541in}{0.181651in}}%
\pgfpathlineto{\pgfqpoint{1.134587in}{0.181651in}}%
\pgfpathlineto{\pgfqpoint{1.146633in}{0.181651in}}%
\pgfpathlineto{\pgfqpoint{1.158679in}{0.181651in}}%
\pgfpathlineto{\pgfqpoint{1.170725in}{0.181651in}}%
\pgfpathlineto{\pgfqpoint{1.182771in}{0.181651in}}%
\pgfpathlineto{\pgfqpoint{1.194817in}{0.181651in}}%
\pgfpathlineto{\pgfqpoint{1.206863in}{0.181651in}}%
\pgfpathlineto{\pgfqpoint{1.218909in}{0.181651in}}%
\pgfpathlineto{\pgfqpoint{1.230956in}{0.181651in}}%
\pgfpathlineto{\pgfqpoint{1.243002in}{0.181651in}}%
\pgfpathlineto{\pgfqpoint{1.255048in}{0.181651in}}%
\pgfpathlineto{\pgfqpoint{1.267094in}{0.181651in}}%
\pgfpathlineto{\pgfqpoint{1.279140in}{0.181651in}}%
\pgfpathlineto{\pgfqpoint{1.291186in}{0.181651in}}%
\pgfpathlineto{\pgfqpoint{1.303232in}{0.181651in}}%
\pgfpathlineto{\pgfqpoint{1.315278in}{0.181651in}}%
\pgfpathlineto{\pgfqpoint{1.327325in}{0.181651in}}%
\pgfpathlineto{\pgfqpoint{1.339371in}{0.181651in}}%
\pgfpathlineto{\pgfqpoint{1.351417in}{0.181651in}}%
\pgfpathlineto{\pgfqpoint{1.363463in}{0.181651in}}%
\pgfpathlineto{\pgfqpoint{1.375509in}{0.181651in}}%
\pgfpathlineto{\pgfqpoint{1.387555in}{0.181651in}}%
\pgfpathlineto{\pgfqpoint{1.399601in}{0.181651in}}%
\pgfpathlineto{\pgfqpoint{1.411647in}{0.181651in}}%
\pgfpathlineto{\pgfqpoint{1.423693in}{0.181651in}}%
\pgfpathlineto{\pgfqpoint{1.435740in}{0.181651in}}%
\pgfpathlineto{\pgfqpoint{1.447786in}{0.181651in}}%
\pgfpathlineto{\pgfqpoint{1.459832in}{0.181651in}}%
\pgfpathlineto{\pgfqpoint{1.471878in}{0.181651in}}%
\pgfpathlineto{\pgfqpoint{1.483924in}{0.181651in}}%
\pgfpathlineto{\pgfqpoint{1.495970in}{0.181651in}}%
\pgfpathlineto{\pgfqpoint{1.508016in}{0.181651in}}%
\pgfpathlineto{\pgfqpoint{1.520062in}{0.181651in}}%
\pgfpathlineto{\pgfqpoint{1.532108in}{0.181651in}}%
\pgfpathlineto{\pgfqpoint{1.544155in}{0.181651in}}%
\pgfpathlineto{\pgfqpoint{1.556201in}{0.181651in}}%
\pgfpathlineto{\pgfqpoint{1.568247in}{0.181651in}}%
\pgfpathlineto{\pgfqpoint{1.580293in}{0.181651in}}%
\pgfpathlineto{\pgfqpoint{1.592339in}{0.181651in}}%
\pgfpathlineto{\pgfqpoint{1.604385in}{0.181651in}}%
\pgfpathlineto{\pgfqpoint{1.616431in}{0.181651in}}%
\pgfpathlineto{\pgfqpoint{1.628477in}{0.181651in}}%
\pgfpathlineto{\pgfqpoint{1.640524in}{0.181651in}}%
\pgfpathlineto{\pgfqpoint{1.652570in}{0.181651in}}%
\pgfpathlineto{\pgfqpoint{1.664616in}{0.181651in}}%
\pgfpathlineto{\pgfqpoint{1.676662in}{0.181651in}}%
\pgfpathlineto{\pgfqpoint{1.688708in}{0.181651in}}%
\pgfpathlineto{\pgfqpoint{1.700754in}{0.181651in}}%
\pgfpathlineto{\pgfqpoint{1.712800in}{0.181651in}}%
\pgfpathlineto{\pgfqpoint{1.724846in}{0.181651in}}%
\pgfpathlineto{\pgfqpoint{1.736892in}{0.181651in}}%
\pgfpathlineto{\pgfqpoint{1.748939in}{0.181651in}}%
\pgfpathlineto{\pgfqpoint{1.760985in}{0.181651in}}%
\pgfpathlineto{\pgfqpoint{1.773031in}{0.181651in}}%
\pgfpathlineto{\pgfqpoint{1.785077in}{0.181651in}}%
\pgfpathlineto{\pgfqpoint{1.797123in}{0.181651in}}%
\pgfpathlineto{\pgfqpoint{1.809169in}{0.181651in}}%
\pgfpathlineto{\pgfqpoint{1.821215in}{0.181651in}}%
\pgfpathlineto{\pgfqpoint{1.833261in}{0.181651in}}%
\pgfpathlineto{\pgfqpoint{1.845307in}{0.181651in}}%
\pgfpathlineto{\pgfqpoint{1.857354in}{0.181651in}}%
\pgfpathlineto{\pgfqpoint{1.869400in}{0.181651in}}%
\pgfpathlineto{\pgfqpoint{1.881446in}{0.181651in}}%
\pgfpathlineto{\pgfqpoint{1.893492in}{0.181651in}}%
\pgfpathlineto{\pgfqpoint{1.905538in}{0.181651in}}%
\pgfpathlineto{\pgfqpoint{1.917584in}{0.181651in}}%
\pgfpathlineto{\pgfqpoint{1.929630in}{0.181651in}}%
\pgfpathlineto{\pgfqpoint{1.941676in}{0.181651in}}%
\pgfpathlineto{\pgfqpoint{1.953723in}{0.181651in}}%
\pgfpathlineto{\pgfqpoint{1.965769in}{0.181651in}}%
\pgfpathlineto{\pgfqpoint{1.965769in}{0.193697in}}%
\pgfpathlineto{\pgfqpoint{1.965769in}{0.205743in}}%
\pgfpathlineto{\pgfqpoint{1.965769in}{0.217790in}}%
\pgfpathlineto{\pgfqpoint{1.965769in}{0.229836in}}%
\pgfpathlineto{\pgfqpoint{1.965769in}{0.241882in}}%
\pgfpathlineto{\pgfqpoint{1.965769in}{0.253928in}}%
\pgfpathlineto{\pgfqpoint{1.965769in}{0.265974in}}%
\pgfpathlineto{\pgfqpoint{1.965769in}{0.278020in}}%
\pgfpathlineto{\pgfqpoint{1.965769in}{0.290066in}}%
\pgfpathlineto{\pgfqpoint{1.965769in}{0.302112in}}%
\pgfpathlineto{\pgfqpoint{1.965769in}{0.314159in}}%
\pgfpathlineto{\pgfqpoint{1.965769in}{0.326205in}}%
\pgfpathlineto{\pgfqpoint{1.965769in}{0.338251in}}%
\pgfpathlineto{\pgfqpoint{1.965769in}{0.350297in}}%
\pgfpathlineto{\pgfqpoint{1.965769in}{0.362343in}}%
\pgfpathlineto{\pgfqpoint{1.965769in}{0.374389in}}%
\pgfpathlineto{\pgfqpoint{1.965769in}{0.386435in}}%
\pgfpathlineto{\pgfqpoint{1.965769in}{0.398481in}}%
\pgfpathlineto{\pgfqpoint{1.965769in}{0.410527in}}%
\pgfpathlineto{\pgfqpoint{1.965769in}{0.422574in}}%
\pgfpathlineto{\pgfqpoint{1.965769in}{0.434620in}}%
\pgfpathlineto{\pgfqpoint{1.965769in}{0.446666in}}%
\pgfpathlineto{\pgfqpoint{1.965769in}{0.458712in}}%
\pgfpathlineto{\pgfqpoint{1.965769in}{0.470758in}}%
\pgfpathlineto{\pgfqpoint{1.965769in}{0.482804in}}%
\pgfpathlineto{\pgfqpoint{1.965769in}{0.494850in}}%
\pgfpathlineto{\pgfqpoint{1.965769in}{0.506896in}}%
\pgfpathlineto{\pgfqpoint{1.965769in}{0.518942in}}%
\pgfpathlineto{\pgfqpoint{1.965769in}{0.530989in}}%
\pgfpathlineto{\pgfqpoint{1.965769in}{0.543035in}}%
\pgfpathlineto{\pgfqpoint{1.965769in}{0.555081in}}%
\pgfpathlineto{\pgfqpoint{1.965769in}{0.567127in}}%
\pgfpathlineto{\pgfqpoint{1.965769in}{0.579173in}}%
\pgfpathlineto{\pgfqpoint{1.965769in}{0.591219in}}%
\pgfpathlineto{\pgfqpoint{1.965769in}{0.603265in}}%
\pgfpathlineto{\pgfqpoint{1.965769in}{0.615311in}}%
\pgfpathlineto{\pgfqpoint{1.965769in}{0.627358in}}%
\pgfpathlineto{\pgfqpoint{1.965769in}{0.639404in}}%
\pgfpathlineto{\pgfqpoint{1.965769in}{0.651450in}}%
\pgfpathlineto{\pgfqpoint{1.965769in}{0.663496in}}%
\pgfpathlineto{\pgfqpoint{1.965769in}{0.675542in}}%
\pgfpathlineto{\pgfqpoint{1.965769in}{0.687588in}}%
\pgfpathlineto{\pgfqpoint{1.965769in}{0.699634in}}%
\pgfpathlineto{\pgfqpoint{1.965769in}{0.711680in}}%
\pgfpathlineto{\pgfqpoint{1.965769in}{0.723726in}}%
\pgfpathlineto{\pgfqpoint{1.965769in}{0.735773in}}%
\pgfpathlineto{\pgfqpoint{1.965769in}{0.747819in}}%
\pgfpathlineto{\pgfqpoint{1.965769in}{0.759865in}}%
\pgfpathlineto{\pgfqpoint{1.965769in}{0.771911in}}%
\pgfpathlineto{\pgfqpoint{1.965769in}{0.783957in}}%
\pgfpathlineto{\pgfqpoint{1.965769in}{0.796003in}}%
\pgfpathlineto{\pgfqpoint{1.965769in}{0.808049in}}%
\pgfpathlineto{\pgfqpoint{1.965769in}{0.820095in}}%
\pgfpathlineto{\pgfqpoint{1.965769in}{0.832141in}}%
\pgfpathlineto{\pgfqpoint{1.965769in}{0.844188in}}%
\pgfpathlineto{\pgfqpoint{1.965769in}{0.856234in}}%
\pgfpathlineto{\pgfqpoint{1.965769in}{0.868280in}}%
\pgfpathlineto{\pgfqpoint{1.965769in}{0.880326in}}%
\pgfpathlineto{\pgfqpoint{1.965769in}{0.892372in}}%
\pgfpathlineto{\pgfqpoint{1.965769in}{0.904418in}}%
\pgfpathlineto{\pgfqpoint{1.965769in}{0.916464in}}%
\pgfpathlineto{\pgfqpoint{1.965769in}{0.928510in}}%
\pgfpathlineto{\pgfqpoint{1.965769in}{0.940557in}}%
\pgfpathlineto{\pgfqpoint{1.965769in}{0.952603in}}%
\pgfpathlineto{\pgfqpoint{1.965769in}{0.964649in}}%
\pgfpathlineto{\pgfqpoint{1.965769in}{0.976695in}}%
\pgfpathlineto{\pgfqpoint{1.965769in}{0.988741in}}%
\pgfpathlineto{\pgfqpoint{1.965769in}{1.000787in}}%
\pgfpathlineto{\pgfqpoint{1.965769in}{1.012833in}}%
\pgfpathlineto{\pgfqpoint{1.965769in}{1.024879in}}%
\pgfpathlineto{\pgfqpoint{1.965769in}{1.036925in}}%
\pgfpathlineto{\pgfqpoint{1.965769in}{1.048972in}}%
\pgfpathlineto{\pgfqpoint{1.965769in}{1.061018in}}%
\pgfpathlineto{\pgfqpoint{1.965769in}{1.073064in}}%
\pgfpathlineto{\pgfqpoint{1.965769in}{1.085110in}}%
\pgfpathlineto{\pgfqpoint{1.965769in}{1.097156in}}%
\pgfpathlineto{\pgfqpoint{1.965769in}{1.109202in}}%
\pgfpathlineto{\pgfqpoint{1.965769in}{1.121248in}}%
\pgfpathlineto{\pgfqpoint{1.965769in}{1.133294in}}%
\pgfpathlineto{\pgfqpoint{1.965769in}{1.145340in}}%
\pgfpathlineto{\pgfqpoint{1.965769in}{1.157387in}}%
\pgfpathlineto{\pgfqpoint{1.965769in}{1.169433in}}%
\pgfpathlineto{\pgfqpoint{1.965769in}{1.181479in}}%
\pgfpathlineto{\pgfqpoint{1.965769in}{1.193525in}}%
\pgfpathlineto{\pgfqpoint{1.965769in}{1.205571in}}%
\pgfpathlineto{\pgfqpoint{1.965769in}{1.217617in}}%
\pgfpathlineto{\pgfqpoint{1.965769in}{1.229663in}}%
\pgfpathlineto{\pgfqpoint{1.965769in}{1.241709in}}%
\pgfpathlineto{\pgfqpoint{1.965769in}{1.253756in}}%
\pgfpathlineto{\pgfqpoint{1.965769in}{1.265802in}}%
\pgfpathlineto{\pgfqpoint{1.965769in}{1.277848in}}%
\pgfpathlineto{\pgfqpoint{1.965769in}{1.289894in}}%
\pgfpathlineto{\pgfqpoint{1.965769in}{1.301940in}}%
\pgfpathlineto{\pgfqpoint{1.965769in}{1.313986in}}%
\pgfpathlineto{\pgfqpoint{1.965769in}{1.326032in}}%
\pgfpathlineto{\pgfqpoint{1.965769in}{1.338078in}}%
\pgfpathlineto{\pgfqpoint{1.965769in}{1.350124in}}%
\pgfpathlineto{\pgfqpoint{1.965769in}{1.362171in}}%
\pgfpathlineto{\pgfqpoint{1.965769in}{1.374217in}}%
\pgfpathlineto{\pgfqpoint{1.965769in}{1.386263in}}%
\pgfpathlineto{\pgfqpoint{1.965769in}{1.398309in}}%
\pgfpathlineto{\pgfqpoint{1.965769in}{1.410355in}}%
\pgfpathlineto{\pgfqpoint{1.965769in}{1.422401in}}%
\pgfpathlineto{\pgfqpoint{1.965769in}{1.434447in}}%
\pgfpathlineto{\pgfqpoint{1.965769in}{1.446493in}}%
\pgfpathlineto{\pgfqpoint{1.965769in}{1.458539in}}%
\pgfpathlineto{\pgfqpoint{1.965769in}{1.470586in}}%
\pgfpathlineto{\pgfqpoint{1.965769in}{1.482632in}}%
\pgfpathlineto{\pgfqpoint{1.965769in}{1.494678in}}%
\pgfpathlineto{\pgfqpoint{1.965769in}{1.506724in}}%
\pgfpathlineto{\pgfqpoint{1.965769in}{1.518770in}}%
\pgfpathlineto{\pgfqpoint{1.965769in}{1.530816in}}%
\pgfpathlineto{\pgfqpoint{1.965769in}{1.542862in}}%
\pgfpathlineto{\pgfqpoint{1.965769in}{1.554908in}}%
\pgfpathlineto{\pgfqpoint{1.965769in}{1.566955in}}%
\pgfpathlineto{\pgfqpoint{1.965769in}{1.579001in}}%
\pgfpathlineto{\pgfqpoint{1.965769in}{1.591047in}}%
\pgfpathlineto{\pgfqpoint{1.965769in}{1.603093in}}%
\pgfpathlineto{\pgfqpoint{1.965769in}{1.615139in}}%
\pgfpathlineto{\pgfqpoint{1.965769in}{1.627185in}}%
\pgfpathlineto{\pgfqpoint{1.965769in}{1.639231in}}%
\pgfpathlineto{\pgfqpoint{1.965769in}{1.651277in}}%
\pgfpathlineto{\pgfqpoint{1.965769in}{1.663323in}}%
\pgfpathlineto{\pgfqpoint{1.965769in}{1.675370in}}%
\pgfpathlineto{\pgfqpoint{1.965769in}{1.687416in}}%
\pgfpathlineto{\pgfqpoint{1.965769in}{1.699462in}}%
\pgfpathlineto{\pgfqpoint{1.965769in}{1.711508in}}%
\pgfpathlineto{\pgfqpoint{1.965769in}{1.723554in}}%
\pgfpathlineto{\pgfqpoint{1.965769in}{1.735600in}}%
\pgfpathlineto{\pgfqpoint{1.965769in}{1.747646in}}%
\pgfpathlineto{\pgfqpoint{1.965769in}{1.759692in}}%
\pgfpathlineto{\pgfqpoint{1.965769in}{1.771738in}}%
\pgfpathlineto{\pgfqpoint{1.965769in}{1.783785in}}%
\pgfpathlineto{\pgfqpoint{1.965769in}{1.795831in}}%
\pgfpathlineto{\pgfqpoint{1.965769in}{1.807877in}}%
\pgfpathlineto{\pgfqpoint{1.965769in}{1.819923in}}%
\pgfpathlineto{\pgfqpoint{1.965769in}{1.831969in}}%
\pgfpathlineto{\pgfqpoint{1.965769in}{1.844015in}}%
\pgfpathlineto{\pgfqpoint{1.965769in}{1.856061in}}%
\pgfpathlineto{\pgfqpoint{1.965769in}{1.868107in}}%
\pgfpathlineto{\pgfqpoint{1.965769in}{1.880154in}}%
\pgfpathlineto{\pgfqpoint{1.965769in}{1.892200in}}%
\pgfpathlineto{\pgfqpoint{1.965769in}{1.904246in}}%
\pgfpathlineto{\pgfqpoint{1.965769in}{1.916292in}}%
\pgfpathlineto{\pgfqpoint{1.965769in}{1.928338in}}%
\pgfpathlineto{\pgfqpoint{1.965769in}{1.940384in}}%
\pgfpathlineto{\pgfqpoint{1.965769in}{1.952430in}}%
\pgfpathlineto{\pgfqpoint{1.965769in}{1.964476in}}%
\pgfpathlineto{\pgfqpoint{1.965769in}{1.976522in}}%
\pgfpathlineto{\pgfqpoint{1.953723in}{1.976522in}}%
\pgfpathlineto{\pgfqpoint{1.941676in}{1.976522in}}%
\pgfpathlineto{\pgfqpoint{1.929630in}{1.976522in}}%
\pgfpathlineto{\pgfqpoint{1.917584in}{1.976522in}}%
\pgfpathlineto{\pgfqpoint{1.905538in}{1.976522in}}%
\pgfpathlineto{\pgfqpoint{1.893492in}{1.976522in}}%
\pgfpathlineto{\pgfqpoint{1.881446in}{1.976522in}}%
\pgfpathlineto{\pgfqpoint{1.869400in}{1.976522in}}%
\pgfpathlineto{\pgfqpoint{1.857354in}{1.976522in}}%
\pgfpathlineto{\pgfqpoint{1.845307in}{1.976522in}}%
\pgfpathlineto{\pgfqpoint{1.833261in}{1.976522in}}%
\pgfpathlineto{\pgfqpoint{1.821215in}{1.976522in}}%
\pgfpathlineto{\pgfqpoint{1.809169in}{1.976522in}}%
\pgfpathlineto{\pgfqpoint{1.797123in}{1.976522in}}%
\pgfpathlineto{\pgfqpoint{1.785077in}{1.976522in}}%
\pgfpathlineto{\pgfqpoint{1.773031in}{1.976522in}}%
\pgfpathlineto{\pgfqpoint{1.760985in}{1.976522in}}%
\pgfpathlineto{\pgfqpoint{1.748939in}{1.976522in}}%
\pgfpathlineto{\pgfqpoint{1.736892in}{1.976522in}}%
\pgfpathlineto{\pgfqpoint{1.724846in}{1.976522in}}%
\pgfpathlineto{\pgfqpoint{1.712800in}{1.976522in}}%
\pgfpathlineto{\pgfqpoint{1.700754in}{1.976522in}}%
\pgfpathlineto{\pgfqpoint{1.688708in}{1.976522in}}%
\pgfpathlineto{\pgfqpoint{1.676662in}{1.976522in}}%
\pgfpathlineto{\pgfqpoint{1.664616in}{1.976522in}}%
\pgfpathlineto{\pgfqpoint{1.652570in}{1.976522in}}%
\pgfpathlineto{\pgfqpoint{1.640524in}{1.976522in}}%
\pgfpathlineto{\pgfqpoint{1.628477in}{1.976522in}}%
\pgfpathlineto{\pgfqpoint{1.616431in}{1.976522in}}%
\pgfpathlineto{\pgfqpoint{1.604385in}{1.976522in}}%
\pgfpathlineto{\pgfqpoint{1.592339in}{1.976522in}}%
\pgfpathlineto{\pgfqpoint{1.580293in}{1.976522in}}%
\pgfpathlineto{\pgfqpoint{1.568247in}{1.976522in}}%
\pgfpathlineto{\pgfqpoint{1.556201in}{1.976522in}}%
\pgfpathlineto{\pgfqpoint{1.544155in}{1.976522in}}%
\pgfpathlineto{\pgfqpoint{1.532108in}{1.976522in}}%
\pgfpathlineto{\pgfqpoint{1.520062in}{1.976522in}}%
\pgfpathlineto{\pgfqpoint{1.508016in}{1.976522in}}%
\pgfpathlineto{\pgfqpoint{1.495970in}{1.976522in}}%
\pgfpathlineto{\pgfqpoint{1.483924in}{1.976522in}}%
\pgfpathlineto{\pgfqpoint{1.471878in}{1.976522in}}%
\pgfpathlineto{\pgfqpoint{1.459832in}{1.976522in}}%
\pgfpathlineto{\pgfqpoint{1.447786in}{1.976522in}}%
\pgfpathlineto{\pgfqpoint{1.435740in}{1.976522in}}%
\pgfpathlineto{\pgfqpoint{1.423693in}{1.976522in}}%
\pgfpathlineto{\pgfqpoint{1.411647in}{1.976522in}}%
\pgfpathlineto{\pgfqpoint{1.399601in}{1.976522in}}%
\pgfpathlineto{\pgfqpoint{1.387555in}{1.976522in}}%
\pgfpathlineto{\pgfqpoint{1.375509in}{1.976522in}}%
\pgfpathlineto{\pgfqpoint{1.363463in}{1.976522in}}%
\pgfpathlineto{\pgfqpoint{1.351417in}{1.976522in}}%
\pgfpathlineto{\pgfqpoint{1.339371in}{1.976522in}}%
\pgfpathlineto{\pgfqpoint{1.327325in}{1.976522in}}%
\pgfpathlineto{\pgfqpoint{1.315278in}{1.976522in}}%
\pgfpathlineto{\pgfqpoint{1.303232in}{1.976522in}}%
\pgfpathlineto{\pgfqpoint{1.291186in}{1.976522in}}%
\pgfpathlineto{\pgfqpoint{1.279140in}{1.976522in}}%
\pgfpathlineto{\pgfqpoint{1.267094in}{1.976522in}}%
\pgfpathlineto{\pgfqpoint{1.255048in}{1.976522in}}%
\pgfpathlineto{\pgfqpoint{1.243002in}{1.976522in}}%
\pgfpathlineto{\pgfqpoint{1.230956in}{1.976522in}}%
\pgfpathlineto{\pgfqpoint{1.218909in}{1.976522in}}%
\pgfpathlineto{\pgfqpoint{1.206863in}{1.976522in}}%
\pgfpathlineto{\pgfqpoint{1.194817in}{1.976522in}}%
\pgfpathlineto{\pgfqpoint{1.182771in}{1.976522in}}%
\pgfpathlineto{\pgfqpoint{1.170725in}{1.976522in}}%
\pgfpathlineto{\pgfqpoint{1.158679in}{1.976522in}}%
\pgfpathlineto{\pgfqpoint{1.146633in}{1.976522in}}%
\pgfpathlineto{\pgfqpoint{1.134587in}{1.976522in}}%
\pgfpathlineto{\pgfqpoint{1.122541in}{1.976522in}}%
\pgfpathlineto{\pgfqpoint{1.110494in}{1.976522in}}%
\pgfpathlineto{\pgfqpoint{1.098448in}{1.976522in}}%
\pgfpathlineto{\pgfqpoint{1.086402in}{1.976522in}}%
\pgfpathlineto{\pgfqpoint{1.074356in}{1.976522in}}%
\pgfpathlineto{\pgfqpoint{1.062310in}{1.976522in}}%
\pgfpathlineto{\pgfqpoint{1.050264in}{1.976522in}}%
\pgfpathlineto{\pgfqpoint{1.038218in}{1.976522in}}%
\pgfpathlineto{\pgfqpoint{1.026172in}{1.976522in}}%
\pgfpathlineto{\pgfqpoint{1.014126in}{1.976522in}}%
\pgfpathlineto{\pgfqpoint{1.002079in}{1.976522in}}%
\pgfpathlineto{\pgfqpoint{0.990033in}{1.976522in}}%
\pgfpathlineto{\pgfqpoint{0.977987in}{1.976522in}}%
\pgfpathlineto{\pgfqpoint{0.965941in}{1.976522in}}%
\pgfpathlineto{\pgfqpoint{0.953895in}{1.976522in}}%
\pgfpathlineto{\pgfqpoint{0.941849in}{1.976522in}}%
\pgfpathlineto{\pgfqpoint{0.929803in}{1.976522in}}%
\pgfpathlineto{\pgfqpoint{0.917757in}{1.976522in}}%
\pgfpathlineto{\pgfqpoint{0.905710in}{1.976522in}}%
\pgfpathlineto{\pgfqpoint{0.893664in}{1.976522in}}%
\pgfpathlineto{\pgfqpoint{0.881618in}{1.976522in}}%
\pgfpathlineto{\pgfqpoint{0.869572in}{1.976522in}}%
\pgfpathlineto{\pgfqpoint{0.857526in}{1.976522in}}%
\pgfpathlineto{\pgfqpoint{0.845480in}{1.976522in}}%
\pgfpathlineto{\pgfqpoint{0.833434in}{1.976522in}}%
\pgfpathlineto{\pgfqpoint{0.821388in}{1.976522in}}%
\pgfpathlineto{\pgfqpoint{0.809342in}{1.976522in}}%
\pgfpathlineto{\pgfqpoint{0.797295in}{1.976522in}}%
\pgfpathlineto{\pgfqpoint{0.785249in}{1.976522in}}%
\pgfpathlineto{\pgfqpoint{0.773203in}{1.976522in}}%
\pgfpathlineto{\pgfqpoint{0.761157in}{1.976522in}}%
\pgfpathlineto{\pgfqpoint{0.749111in}{1.976522in}}%
\pgfpathlineto{\pgfqpoint{0.737065in}{1.976522in}}%
\pgfpathlineto{\pgfqpoint{0.725019in}{1.976522in}}%
\pgfpathlineto{\pgfqpoint{0.712973in}{1.976522in}}%
\pgfpathlineto{\pgfqpoint{0.700927in}{1.976522in}}%
\pgfpathlineto{\pgfqpoint{0.688880in}{1.976522in}}%
\pgfpathlineto{\pgfqpoint{0.676834in}{1.976522in}}%
\pgfpathlineto{\pgfqpoint{0.664788in}{1.976522in}}%
\pgfpathlineto{\pgfqpoint{0.652742in}{1.976522in}}%
\pgfpathlineto{\pgfqpoint{0.640696in}{1.976522in}}%
\pgfpathlineto{\pgfqpoint{0.628650in}{1.976522in}}%
\pgfpathlineto{\pgfqpoint{0.616604in}{1.976522in}}%
\pgfpathlineto{\pgfqpoint{0.604558in}{1.976522in}}%
\pgfpathlineto{\pgfqpoint{0.592511in}{1.976522in}}%
\pgfpathlineto{\pgfqpoint{0.580465in}{1.976522in}}%
\pgfpathlineto{\pgfqpoint{0.568419in}{1.976522in}}%
\pgfpathlineto{\pgfqpoint{0.556373in}{1.976522in}}%
\pgfpathlineto{\pgfqpoint{0.544327in}{1.976522in}}%
\pgfpathlineto{\pgfqpoint{0.532281in}{1.976522in}}%
\pgfpathlineto{\pgfqpoint{0.520235in}{1.976522in}}%
\pgfpathlineto{\pgfqpoint{0.508189in}{1.976522in}}%
\pgfpathlineto{\pgfqpoint{0.496143in}{1.976522in}}%
\pgfpathlineto{\pgfqpoint{0.484096in}{1.976522in}}%
\pgfpathlineto{\pgfqpoint{0.472050in}{1.976522in}}%
\pgfpathlineto{\pgfqpoint{0.460004in}{1.976522in}}%
\pgfpathlineto{\pgfqpoint{0.447958in}{1.976522in}}%
\pgfpathlineto{\pgfqpoint{0.435912in}{1.976522in}}%
\pgfpathlineto{\pgfqpoint{0.423866in}{1.976522in}}%
\pgfpathlineto{\pgfqpoint{0.411820in}{1.976522in}}%
\pgfpathlineto{\pgfqpoint{0.399774in}{1.976522in}}%
\pgfpathlineto{\pgfqpoint{0.387728in}{1.976522in}}%
\pgfpathlineto{\pgfqpoint{0.375681in}{1.976522in}}%
\pgfpathlineto{\pgfqpoint{0.363635in}{1.976522in}}%
\pgfpathlineto{\pgfqpoint{0.351589in}{1.976522in}}%
\pgfpathlineto{\pgfqpoint{0.339543in}{1.976522in}}%
\pgfpathlineto{\pgfqpoint{0.327497in}{1.976522in}}%
\pgfpathlineto{\pgfqpoint{0.315451in}{1.976522in}}%
\pgfpathlineto{\pgfqpoint{0.303405in}{1.976522in}}%
\pgfpathlineto{\pgfqpoint{0.291359in}{1.976522in}}%
\pgfpathlineto{\pgfqpoint{0.279312in}{1.976522in}}%
\pgfpathlineto{\pgfqpoint{0.267266in}{1.976522in}}%
\pgfpathlineto{\pgfqpoint{0.255220in}{1.976522in}}%
\pgfpathlineto{\pgfqpoint{0.243174in}{1.976522in}}%
\pgfpathlineto{\pgfqpoint{0.231128in}{1.976522in}}%
\pgfpathlineto{\pgfqpoint{0.219082in}{1.976522in}}%
\pgfpathlineto{\pgfqpoint{0.207036in}{1.976522in}}%
\pgfpathlineto{\pgfqpoint{0.194990in}{1.976522in}}%
\pgfpathlineto{\pgfqpoint{0.182944in}{1.976522in}}%
\pgfpathlineto{\pgfqpoint{0.170897in}{1.976522in}}%
\pgfpathlineto{\pgfqpoint{0.170897in}{1.964476in}}%
\pgfpathlineto{\pgfqpoint{0.170897in}{1.952430in}}%
\pgfpathlineto{\pgfqpoint{0.170897in}{1.940384in}}%
\pgfpathlineto{\pgfqpoint{0.170897in}{1.928338in}}%
\pgfpathlineto{\pgfqpoint{0.170897in}{1.916292in}}%
\pgfpathlineto{\pgfqpoint{0.170897in}{1.904246in}}%
\pgfpathlineto{\pgfqpoint{0.170897in}{1.892200in}}%
\pgfpathlineto{\pgfqpoint{0.170897in}{1.880154in}}%
\pgfpathlineto{\pgfqpoint{0.170897in}{1.868107in}}%
\pgfpathlineto{\pgfqpoint{0.170897in}{1.856061in}}%
\pgfpathlineto{\pgfqpoint{0.170897in}{1.844015in}}%
\pgfpathlineto{\pgfqpoint{0.170897in}{1.831969in}}%
\pgfpathlineto{\pgfqpoint{0.170897in}{1.819923in}}%
\pgfpathlineto{\pgfqpoint{0.170897in}{1.807877in}}%
\pgfpathlineto{\pgfqpoint{0.170897in}{1.795831in}}%
\pgfpathlineto{\pgfqpoint{0.170897in}{1.783785in}}%
\pgfpathlineto{\pgfqpoint{0.170897in}{1.771738in}}%
\pgfpathlineto{\pgfqpoint{0.170897in}{1.759692in}}%
\pgfpathlineto{\pgfqpoint{0.170897in}{1.747646in}}%
\pgfpathlineto{\pgfqpoint{0.170897in}{1.735600in}}%
\pgfpathlineto{\pgfqpoint{0.170897in}{1.723554in}}%
\pgfpathlineto{\pgfqpoint{0.170897in}{1.711508in}}%
\pgfpathlineto{\pgfqpoint{0.170897in}{1.699462in}}%
\pgfpathlineto{\pgfqpoint{0.170897in}{1.687416in}}%
\pgfpathlineto{\pgfqpoint{0.170897in}{1.675370in}}%
\pgfpathlineto{\pgfqpoint{0.170897in}{1.663323in}}%
\pgfpathlineto{\pgfqpoint{0.170897in}{1.651277in}}%
\pgfpathlineto{\pgfqpoint{0.170897in}{1.639231in}}%
\pgfpathlineto{\pgfqpoint{0.170897in}{1.627185in}}%
\pgfpathlineto{\pgfqpoint{0.170897in}{1.615139in}}%
\pgfpathlineto{\pgfqpoint{0.170897in}{1.603093in}}%
\pgfpathlineto{\pgfqpoint{0.170897in}{1.591047in}}%
\pgfpathlineto{\pgfqpoint{0.170897in}{1.579001in}}%
\pgfpathlineto{\pgfqpoint{0.170897in}{1.566955in}}%
\pgfpathlineto{\pgfqpoint{0.170897in}{1.554908in}}%
\pgfpathlineto{\pgfqpoint{0.170897in}{1.542862in}}%
\pgfpathlineto{\pgfqpoint{0.170897in}{1.530816in}}%
\pgfpathlineto{\pgfqpoint{0.170897in}{1.518770in}}%
\pgfpathlineto{\pgfqpoint{0.170897in}{1.506724in}}%
\pgfpathlineto{\pgfqpoint{0.170897in}{1.494678in}}%
\pgfpathlineto{\pgfqpoint{0.170897in}{1.482632in}}%
\pgfpathlineto{\pgfqpoint{0.170897in}{1.470586in}}%
\pgfpathlineto{\pgfqpoint{0.170897in}{1.458539in}}%
\pgfpathlineto{\pgfqpoint{0.170897in}{1.446493in}}%
\pgfpathlineto{\pgfqpoint{0.170897in}{1.434447in}}%
\pgfpathlineto{\pgfqpoint{0.170897in}{1.422401in}}%
\pgfpathlineto{\pgfqpoint{0.170897in}{1.410355in}}%
\pgfpathlineto{\pgfqpoint{0.170897in}{1.398309in}}%
\pgfpathlineto{\pgfqpoint{0.170897in}{1.386263in}}%
\pgfpathlineto{\pgfqpoint{0.170897in}{1.374217in}}%
\pgfpathlineto{\pgfqpoint{0.170897in}{1.362171in}}%
\pgfpathlineto{\pgfqpoint{0.170897in}{1.350124in}}%
\pgfpathlineto{\pgfqpoint{0.170897in}{1.338078in}}%
\pgfpathlineto{\pgfqpoint{0.170897in}{1.326032in}}%
\pgfpathlineto{\pgfqpoint{0.170897in}{1.313986in}}%
\pgfpathlineto{\pgfqpoint{0.170897in}{1.301940in}}%
\pgfpathlineto{\pgfqpoint{0.170897in}{1.289894in}}%
\pgfpathlineto{\pgfqpoint{0.170897in}{1.277848in}}%
\pgfpathlineto{\pgfqpoint{0.170897in}{1.265802in}}%
\pgfpathlineto{\pgfqpoint{0.170897in}{1.253756in}}%
\pgfpathlineto{\pgfqpoint{0.170897in}{1.241709in}}%
\pgfpathlineto{\pgfqpoint{0.170897in}{1.229663in}}%
\pgfpathlineto{\pgfqpoint{0.170897in}{1.217617in}}%
\pgfpathlineto{\pgfqpoint{0.170897in}{1.205571in}}%
\pgfpathlineto{\pgfqpoint{0.170897in}{1.193525in}}%
\pgfpathlineto{\pgfqpoint{0.170897in}{1.181479in}}%
\pgfpathlineto{\pgfqpoint{0.170897in}{1.169433in}}%
\pgfpathlineto{\pgfqpoint{0.170897in}{1.157387in}}%
\pgfpathlineto{\pgfqpoint{0.170897in}{1.145340in}}%
\pgfpathlineto{\pgfqpoint{0.170897in}{1.133294in}}%
\pgfpathlineto{\pgfqpoint{0.170897in}{1.121248in}}%
\pgfpathlineto{\pgfqpoint{0.170897in}{1.109202in}}%
\pgfpathlineto{\pgfqpoint{0.170897in}{1.097156in}}%
\pgfpathlineto{\pgfqpoint{0.170897in}{1.085110in}}%
\pgfpathlineto{\pgfqpoint{0.170897in}{1.073064in}}%
\pgfpathlineto{\pgfqpoint{0.170897in}{1.061018in}}%
\pgfpathlineto{\pgfqpoint{0.170897in}{1.048972in}}%
\pgfpathlineto{\pgfqpoint{0.170897in}{1.036925in}}%
\pgfpathlineto{\pgfqpoint{0.170897in}{1.024879in}}%
\pgfpathlineto{\pgfqpoint{0.170897in}{1.012833in}}%
\pgfpathlineto{\pgfqpoint{0.170897in}{1.000787in}}%
\pgfpathlineto{\pgfqpoint{0.170897in}{0.988741in}}%
\pgfpathlineto{\pgfqpoint{0.170897in}{0.976695in}}%
\pgfpathlineto{\pgfqpoint{0.170897in}{0.964649in}}%
\pgfpathlineto{\pgfqpoint{0.170897in}{0.952603in}}%
\pgfpathlineto{\pgfqpoint{0.170897in}{0.940557in}}%
\pgfpathlineto{\pgfqpoint{0.170897in}{0.928510in}}%
\pgfpathlineto{\pgfqpoint{0.170897in}{0.916464in}}%
\pgfpathlineto{\pgfqpoint{0.170897in}{0.904418in}}%
\pgfpathlineto{\pgfqpoint{0.170897in}{0.892372in}}%
\pgfpathlineto{\pgfqpoint{0.170897in}{0.880326in}}%
\pgfpathlineto{\pgfqpoint{0.170897in}{0.868280in}}%
\pgfpathlineto{\pgfqpoint{0.170897in}{0.856234in}}%
\pgfpathlineto{\pgfqpoint{0.170897in}{0.844188in}}%
\pgfpathlineto{\pgfqpoint{0.170897in}{0.832141in}}%
\pgfpathlineto{\pgfqpoint{0.170897in}{0.820095in}}%
\pgfpathlineto{\pgfqpoint{0.170897in}{0.808049in}}%
\pgfpathlineto{\pgfqpoint{0.170897in}{0.796003in}}%
\pgfpathlineto{\pgfqpoint{0.170897in}{0.783957in}}%
\pgfpathlineto{\pgfqpoint{0.170897in}{0.771911in}}%
\pgfpathlineto{\pgfqpoint{0.170897in}{0.759865in}}%
\pgfpathlineto{\pgfqpoint{0.170897in}{0.747819in}}%
\pgfpathlineto{\pgfqpoint{0.170897in}{0.735773in}}%
\pgfpathlineto{\pgfqpoint{0.170897in}{0.723726in}}%
\pgfpathlineto{\pgfqpoint{0.170897in}{0.711680in}}%
\pgfpathlineto{\pgfqpoint{0.170897in}{0.699634in}}%
\pgfpathlineto{\pgfqpoint{0.170897in}{0.687588in}}%
\pgfpathlineto{\pgfqpoint{0.170897in}{0.675542in}}%
\pgfpathlineto{\pgfqpoint{0.170897in}{0.663496in}}%
\pgfpathlineto{\pgfqpoint{0.170897in}{0.651450in}}%
\pgfpathlineto{\pgfqpoint{0.170897in}{0.639404in}}%
\pgfpathlineto{\pgfqpoint{0.170897in}{0.627358in}}%
\pgfpathlineto{\pgfqpoint{0.170897in}{0.615311in}}%
\pgfpathlineto{\pgfqpoint{0.170897in}{0.603265in}}%
\pgfpathlineto{\pgfqpoint{0.170897in}{0.591219in}}%
\pgfpathlineto{\pgfqpoint{0.170897in}{0.579173in}}%
\pgfpathlineto{\pgfqpoint{0.170897in}{0.567127in}}%
\pgfpathlineto{\pgfqpoint{0.170897in}{0.555081in}}%
\pgfpathlineto{\pgfqpoint{0.170897in}{0.543035in}}%
\pgfpathlineto{\pgfqpoint{0.170897in}{0.530989in}}%
\pgfpathlineto{\pgfqpoint{0.170897in}{0.518942in}}%
\pgfpathlineto{\pgfqpoint{0.170897in}{0.506896in}}%
\pgfpathlineto{\pgfqpoint{0.170897in}{0.494850in}}%
\pgfpathlineto{\pgfqpoint{0.170897in}{0.482804in}}%
\pgfpathlineto{\pgfqpoint{0.170897in}{0.470758in}}%
\pgfpathlineto{\pgfqpoint{0.170897in}{0.458712in}}%
\pgfpathlineto{\pgfqpoint{0.170897in}{0.446666in}}%
\pgfpathlineto{\pgfqpoint{0.170897in}{0.434620in}}%
\pgfpathlineto{\pgfqpoint{0.170897in}{0.422574in}}%
\pgfpathlineto{\pgfqpoint{0.170897in}{0.410527in}}%
\pgfpathlineto{\pgfqpoint{0.170897in}{0.398481in}}%
\pgfpathlineto{\pgfqpoint{0.170897in}{0.386435in}}%
\pgfpathlineto{\pgfqpoint{0.170897in}{0.374389in}}%
\pgfpathlineto{\pgfqpoint{0.170897in}{0.362343in}}%
\pgfpathlineto{\pgfqpoint{0.170897in}{0.350297in}}%
\pgfpathlineto{\pgfqpoint{0.170897in}{0.338251in}}%
\pgfpathlineto{\pgfqpoint{0.170897in}{0.326205in}}%
\pgfpathlineto{\pgfqpoint{0.170897in}{0.314159in}}%
\pgfpathlineto{\pgfqpoint{0.170897in}{0.302112in}}%
\pgfpathlineto{\pgfqpoint{0.170897in}{0.290066in}}%
\pgfpathlineto{\pgfqpoint{0.170897in}{0.278020in}}%
\pgfpathlineto{\pgfqpoint{0.170897in}{0.265974in}}%
\pgfpathlineto{\pgfqpoint{0.170897in}{0.253928in}}%
\pgfpathlineto{\pgfqpoint{0.170897in}{0.241882in}}%
\pgfpathlineto{\pgfqpoint{0.170897in}{0.229836in}}%
\pgfpathlineto{\pgfqpoint{0.170897in}{0.217790in}}%
\pgfpathlineto{\pgfqpoint{0.170897in}{0.205743in}}%
\pgfpathlineto{\pgfqpoint{0.170897in}{0.193697in}}%
\pgfpathlineto{\pgfqpoint{0.170897in}{0.181651in}}%
\pgfpathclose%
\pgfpathmoveto{\pgfqpoint{1.056287in}{0.193697in}}%
\pgfpathlineto{\pgfqpoint{1.050264in}{0.199720in}}%
\pgfpathlineto{\pgfqpoint{1.044241in}{0.205743in}}%
\pgfpathlineto{\pgfqpoint{1.038218in}{0.211767in}}%
\pgfpathlineto{\pgfqpoint{1.032195in}{0.217790in}}%
\pgfpathlineto{\pgfqpoint{1.026172in}{0.223813in}}%
\pgfpathlineto{\pgfqpoint{1.020149in}{0.229836in}}%
\pgfpathlineto{\pgfqpoint{1.014126in}{0.235859in}}%
\pgfpathlineto{\pgfqpoint{1.008102in}{0.241882in}}%
\pgfpathlineto{\pgfqpoint{1.002079in}{0.247905in}}%
\pgfpathlineto{\pgfqpoint{0.996056in}{0.253928in}}%
\pgfpathlineto{\pgfqpoint{0.990033in}{0.259951in}}%
\pgfpathlineto{\pgfqpoint{0.984010in}{0.265974in}}%
\pgfpathlineto{\pgfqpoint{0.977987in}{0.271997in}}%
\pgfpathlineto{\pgfqpoint{0.971964in}{0.278020in}}%
\pgfpathlineto{\pgfqpoint{0.965941in}{0.284043in}}%
\pgfpathlineto{\pgfqpoint{0.959918in}{0.290066in}}%
\pgfpathlineto{\pgfqpoint{0.953895in}{0.296089in}}%
\pgfpathlineto{\pgfqpoint{0.947872in}{0.302112in}}%
\pgfpathlineto{\pgfqpoint{0.941849in}{0.308135in}}%
\pgfpathlineto{\pgfqpoint{0.935826in}{0.314159in}}%
\pgfpathlineto{\pgfqpoint{0.929803in}{0.320182in}}%
\pgfpathlineto{\pgfqpoint{0.923780in}{0.326205in}}%
\pgfpathlineto{\pgfqpoint{0.917757in}{0.332228in}}%
\pgfpathlineto{\pgfqpoint{0.911734in}{0.338251in}}%
\pgfpathlineto{\pgfqpoint{0.905710in}{0.344274in}}%
\pgfpathlineto{\pgfqpoint{0.899687in}{0.350297in}}%
\pgfpathlineto{\pgfqpoint{0.893664in}{0.356320in}}%
\pgfpathlineto{\pgfqpoint{0.887641in}{0.362343in}}%
\pgfpathlineto{\pgfqpoint{0.881618in}{0.368366in}}%
\pgfpathlineto{\pgfqpoint{0.875595in}{0.374389in}}%
\pgfpathlineto{\pgfqpoint{0.869572in}{0.380412in}}%
\pgfpathlineto{\pgfqpoint{0.863549in}{0.386435in}}%
\pgfpathlineto{\pgfqpoint{0.857526in}{0.392458in}}%
\pgfpathlineto{\pgfqpoint{0.851503in}{0.398481in}}%
\pgfpathlineto{\pgfqpoint{0.845480in}{0.404504in}}%
\pgfpathlineto{\pgfqpoint{0.839457in}{0.410527in}}%
\pgfpathlineto{\pgfqpoint{0.833434in}{0.416551in}}%
\pgfpathlineto{\pgfqpoint{0.827411in}{0.422574in}}%
\pgfpathlineto{\pgfqpoint{0.821388in}{0.428597in}}%
\pgfpathlineto{\pgfqpoint{0.815365in}{0.434620in}}%
\pgfpathlineto{\pgfqpoint{0.809342in}{0.440643in}}%
\pgfpathlineto{\pgfqpoint{0.803318in}{0.446666in}}%
\pgfpathlineto{\pgfqpoint{0.797295in}{0.452689in}}%
\pgfpathlineto{\pgfqpoint{0.791272in}{0.458712in}}%
\pgfpathlineto{\pgfqpoint{0.785249in}{0.464735in}}%
\pgfpathlineto{\pgfqpoint{0.779226in}{0.470758in}}%
\pgfpathlineto{\pgfqpoint{0.773203in}{0.476781in}}%
\pgfpathlineto{\pgfqpoint{0.767180in}{0.482804in}}%
\pgfpathlineto{\pgfqpoint{0.761157in}{0.488827in}}%
\pgfpathlineto{\pgfqpoint{0.755134in}{0.494850in}}%
\pgfpathlineto{\pgfqpoint{0.749111in}{0.500873in}}%
\pgfpathlineto{\pgfqpoint{0.743088in}{0.506896in}}%
\pgfpathlineto{\pgfqpoint{0.737065in}{0.512919in}}%
\pgfpathlineto{\pgfqpoint{0.731042in}{0.518942in}}%
\pgfpathlineto{\pgfqpoint{0.725019in}{0.524966in}}%
\pgfpathlineto{\pgfqpoint{0.718996in}{0.530989in}}%
\pgfpathlineto{\pgfqpoint{0.712973in}{0.537012in}}%
\pgfpathlineto{\pgfqpoint{0.706950in}{0.543035in}}%
\pgfpathlineto{\pgfqpoint{0.700927in}{0.549058in}}%
\pgfpathlineto{\pgfqpoint{0.694903in}{0.555081in}}%
\pgfpathlineto{\pgfqpoint{0.688880in}{0.561104in}}%
\pgfpathlineto{\pgfqpoint{0.682857in}{0.567127in}}%
\pgfpathlineto{\pgfqpoint{0.676834in}{0.573150in}}%
\pgfpathlineto{\pgfqpoint{0.670811in}{0.579173in}}%
\pgfpathlineto{\pgfqpoint{0.664788in}{0.585196in}}%
\pgfpathlineto{\pgfqpoint{0.658765in}{0.591219in}}%
\pgfpathlineto{\pgfqpoint{0.652742in}{0.597242in}}%
\pgfpathlineto{\pgfqpoint{0.646719in}{0.603265in}}%
\pgfpathlineto{\pgfqpoint{0.640696in}{0.609288in}}%
\pgfpathlineto{\pgfqpoint{0.634673in}{0.615311in}}%
\pgfpathlineto{\pgfqpoint{0.628650in}{0.621334in}}%
\pgfpathlineto{\pgfqpoint{0.622627in}{0.627358in}}%
\pgfpathlineto{\pgfqpoint{0.616604in}{0.633381in}}%
\pgfpathlineto{\pgfqpoint{0.610581in}{0.639404in}}%
\pgfpathlineto{\pgfqpoint{0.604558in}{0.645427in}}%
\pgfpathlineto{\pgfqpoint{0.598535in}{0.651450in}}%
\pgfpathlineto{\pgfqpoint{0.592511in}{0.657473in}}%
\pgfpathlineto{\pgfqpoint{0.586488in}{0.663496in}}%
\pgfpathlineto{\pgfqpoint{0.580465in}{0.669519in}}%
\pgfpathlineto{\pgfqpoint{0.574442in}{0.675542in}}%
\pgfpathlineto{\pgfqpoint{0.568419in}{0.681565in}}%
\pgfpathlineto{\pgfqpoint{0.562396in}{0.687588in}}%
\pgfpathlineto{\pgfqpoint{0.556373in}{0.693611in}}%
\pgfpathlineto{\pgfqpoint{0.550350in}{0.699634in}}%
\pgfpathlineto{\pgfqpoint{0.544327in}{0.705657in}}%
\pgfpathlineto{\pgfqpoint{0.538304in}{0.711680in}}%
\pgfpathlineto{\pgfqpoint{0.532281in}{0.717703in}}%
\pgfpathlineto{\pgfqpoint{0.526258in}{0.723726in}}%
\pgfpathlineto{\pgfqpoint{0.520235in}{0.729750in}}%
\pgfpathlineto{\pgfqpoint{0.514212in}{0.735773in}}%
\pgfpathlineto{\pgfqpoint{0.508189in}{0.741796in}}%
\pgfpathlineto{\pgfqpoint{0.502166in}{0.747819in}}%
\pgfpathlineto{\pgfqpoint{0.496143in}{0.753842in}}%
\pgfpathlineto{\pgfqpoint{0.490119in}{0.759865in}}%
\pgfpathlineto{\pgfqpoint{0.484096in}{0.765888in}}%
\pgfpathlineto{\pgfqpoint{0.478073in}{0.771911in}}%
\pgfpathlineto{\pgfqpoint{0.472050in}{0.777934in}}%
\pgfpathlineto{\pgfqpoint{0.466027in}{0.783957in}}%
\pgfpathlineto{\pgfqpoint{0.460004in}{0.789980in}}%
\pgfpathlineto{\pgfqpoint{0.453981in}{0.796003in}}%
\pgfpathlineto{\pgfqpoint{0.447958in}{0.802026in}}%
\pgfpathlineto{\pgfqpoint{0.441935in}{0.808049in}}%
\pgfpathlineto{\pgfqpoint{0.435912in}{0.814072in}}%
\pgfpathlineto{\pgfqpoint{0.429889in}{0.820095in}}%
\pgfpathlineto{\pgfqpoint{0.423866in}{0.826118in}}%
\pgfpathlineto{\pgfqpoint{0.417843in}{0.832141in}}%
\pgfpathlineto{\pgfqpoint{0.411820in}{0.838165in}}%
\pgfpathlineto{\pgfqpoint{0.405797in}{0.844188in}}%
\pgfpathlineto{\pgfqpoint{0.399774in}{0.850211in}}%
\pgfpathlineto{\pgfqpoint{0.393751in}{0.856234in}}%
\pgfpathlineto{\pgfqpoint{0.387728in}{0.862257in}}%
\pgfpathlineto{\pgfqpoint{0.381704in}{0.868280in}}%
\pgfpathlineto{\pgfqpoint{0.375681in}{0.874303in}}%
\pgfpathlineto{\pgfqpoint{0.369658in}{0.880326in}}%
\pgfpathlineto{\pgfqpoint{0.363635in}{0.886349in}}%
\pgfpathlineto{\pgfqpoint{0.357612in}{0.892372in}}%
\pgfpathlineto{\pgfqpoint{0.351589in}{0.898395in}}%
\pgfpathlineto{\pgfqpoint{0.345566in}{0.904418in}}%
\pgfpathlineto{\pgfqpoint{0.339543in}{0.910441in}}%
\pgfpathlineto{\pgfqpoint{0.333520in}{0.916464in}}%
\pgfpathlineto{\pgfqpoint{0.327497in}{0.922487in}}%
\pgfpathlineto{\pgfqpoint{0.321474in}{0.928510in}}%
\pgfpathlineto{\pgfqpoint{0.315451in}{0.934533in}}%
\pgfpathlineto{\pgfqpoint{0.309428in}{0.940557in}}%
\pgfpathlineto{\pgfqpoint{0.303405in}{0.946580in}}%
\pgfpathlineto{\pgfqpoint{0.297382in}{0.952603in}}%
\pgfpathlineto{\pgfqpoint{0.291359in}{0.958626in}}%
\pgfpathlineto{\pgfqpoint{0.285336in}{0.964649in}}%
\pgfpathlineto{\pgfqpoint{0.279312in}{0.970672in}}%
\pgfpathlineto{\pgfqpoint{0.273289in}{0.976695in}}%
\pgfpathlineto{\pgfqpoint{0.267266in}{0.982718in}}%
\pgfpathlineto{\pgfqpoint{0.261243in}{0.988741in}}%
\pgfpathlineto{\pgfqpoint{0.255220in}{0.994764in}}%
\pgfpathlineto{\pgfqpoint{0.249197in}{1.000787in}}%
\pgfpathlineto{\pgfqpoint{0.243174in}{1.006810in}}%
\pgfpathlineto{\pgfqpoint{0.237151in}{1.012833in}}%
\pgfpathlineto{\pgfqpoint{0.231128in}{1.018856in}}%
\pgfpathlineto{\pgfqpoint{0.225105in}{1.024879in}}%
\pgfpathlineto{\pgfqpoint{0.219082in}{1.030902in}}%
\pgfpathlineto{\pgfqpoint{0.213059in}{1.036925in}}%
\pgfpathlineto{\pgfqpoint{0.207036in}{1.042949in}}%
\pgfpathlineto{\pgfqpoint{0.201013in}{1.048972in}}%
\pgfpathlineto{\pgfqpoint{0.194990in}{1.054995in}}%
\pgfpathlineto{\pgfqpoint{0.188967in}{1.061018in}}%
\pgfpathlineto{\pgfqpoint{0.182944in}{1.067041in}}%
\pgfpathlineto{\pgfqpoint{0.176920in}{1.073064in}}%
\pgfpathlineto{\pgfqpoint{0.176920in}{1.085110in}}%
\pgfpathlineto{\pgfqpoint{0.182944in}{1.091133in}}%
\pgfpathlineto{\pgfqpoint{0.188967in}{1.097156in}}%
\pgfpathlineto{\pgfqpoint{0.194990in}{1.103179in}}%
\pgfpathlineto{\pgfqpoint{0.201013in}{1.109202in}}%
\pgfpathlineto{\pgfqpoint{0.207036in}{1.115225in}}%
\pgfpathlineto{\pgfqpoint{0.213059in}{1.121248in}}%
\pgfpathlineto{\pgfqpoint{0.219082in}{1.127271in}}%
\pgfpathlineto{\pgfqpoint{0.225105in}{1.133294in}}%
\pgfpathlineto{\pgfqpoint{0.231128in}{1.139317in}}%
\pgfpathlineto{\pgfqpoint{0.237151in}{1.145340in}}%
\pgfpathlineto{\pgfqpoint{0.243174in}{1.151364in}}%
\pgfpathlineto{\pgfqpoint{0.249197in}{1.157387in}}%
\pgfpathlineto{\pgfqpoint{0.255220in}{1.163410in}}%
\pgfpathlineto{\pgfqpoint{0.261243in}{1.169433in}}%
\pgfpathlineto{\pgfqpoint{0.267266in}{1.175456in}}%
\pgfpathlineto{\pgfqpoint{0.273289in}{1.181479in}}%
\pgfpathlineto{\pgfqpoint{0.279312in}{1.187502in}}%
\pgfpathlineto{\pgfqpoint{0.285336in}{1.193525in}}%
\pgfpathlineto{\pgfqpoint{0.291359in}{1.199548in}}%
\pgfpathlineto{\pgfqpoint{0.297382in}{1.205571in}}%
\pgfpathlineto{\pgfqpoint{0.303405in}{1.211594in}}%
\pgfpathlineto{\pgfqpoint{0.309428in}{1.217617in}}%
\pgfpathlineto{\pgfqpoint{0.315451in}{1.223640in}}%
\pgfpathlineto{\pgfqpoint{0.321474in}{1.229663in}}%
\pgfpathlineto{\pgfqpoint{0.327497in}{1.235686in}}%
\pgfpathlineto{\pgfqpoint{0.333520in}{1.241709in}}%
\pgfpathlineto{\pgfqpoint{0.339543in}{1.247732in}}%
\pgfpathlineto{\pgfqpoint{0.345566in}{1.253756in}}%
\pgfpathlineto{\pgfqpoint{0.351589in}{1.259779in}}%
\pgfpathlineto{\pgfqpoint{0.357612in}{1.265802in}}%
\pgfpathlineto{\pgfqpoint{0.363635in}{1.271825in}}%
\pgfpathlineto{\pgfqpoint{0.369658in}{1.277848in}}%
\pgfpathlineto{\pgfqpoint{0.375681in}{1.283871in}}%
\pgfpathlineto{\pgfqpoint{0.381704in}{1.289894in}}%
\pgfpathlineto{\pgfqpoint{0.387728in}{1.295917in}}%
\pgfpathlineto{\pgfqpoint{0.393751in}{1.301940in}}%
\pgfpathlineto{\pgfqpoint{0.399774in}{1.307963in}}%
\pgfpathlineto{\pgfqpoint{0.405797in}{1.313986in}}%
\pgfpathlineto{\pgfqpoint{0.411820in}{1.320009in}}%
\pgfpathlineto{\pgfqpoint{0.417843in}{1.326032in}}%
\pgfpathlineto{\pgfqpoint{0.423866in}{1.332055in}}%
\pgfpathlineto{\pgfqpoint{0.429889in}{1.338078in}}%
\pgfpathlineto{\pgfqpoint{0.435912in}{1.344101in}}%
\pgfpathlineto{\pgfqpoint{0.441935in}{1.350124in}}%
\pgfpathlineto{\pgfqpoint{0.447958in}{1.356148in}}%
\pgfpathlineto{\pgfqpoint{0.453981in}{1.362171in}}%
\pgfpathlineto{\pgfqpoint{0.460004in}{1.368194in}}%
\pgfpathlineto{\pgfqpoint{0.466027in}{1.374217in}}%
\pgfpathlineto{\pgfqpoint{0.472050in}{1.380240in}}%
\pgfpathlineto{\pgfqpoint{0.478073in}{1.386263in}}%
\pgfpathlineto{\pgfqpoint{0.484096in}{1.392286in}}%
\pgfpathlineto{\pgfqpoint{0.490119in}{1.398309in}}%
\pgfpathlineto{\pgfqpoint{0.496143in}{1.404332in}}%
\pgfpathlineto{\pgfqpoint{0.502166in}{1.410355in}}%
\pgfpathlineto{\pgfqpoint{0.508189in}{1.416378in}}%
\pgfpathlineto{\pgfqpoint{0.514212in}{1.422401in}}%
\pgfpathlineto{\pgfqpoint{0.520235in}{1.428424in}}%
\pgfpathlineto{\pgfqpoint{0.526258in}{1.434447in}}%
\pgfpathlineto{\pgfqpoint{0.532281in}{1.440470in}}%
\pgfpathlineto{\pgfqpoint{0.538304in}{1.446493in}}%
\pgfpathlineto{\pgfqpoint{0.544327in}{1.452516in}}%
\pgfpathlineto{\pgfqpoint{0.550350in}{1.458539in}}%
\pgfpathlineto{\pgfqpoint{0.556373in}{1.464563in}}%
\pgfpathlineto{\pgfqpoint{0.562396in}{1.470586in}}%
\pgfpathlineto{\pgfqpoint{0.568419in}{1.476609in}}%
\pgfpathlineto{\pgfqpoint{0.574442in}{1.482632in}}%
\pgfpathlineto{\pgfqpoint{0.580465in}{1.488655in}}%
\pgfpathlineto{\pgfqpoint{0.586488in}{1.494678in}}%
\pgfpathlineto{\pgfqpoint{0.592511in}{1.500701in}}%
\pgfpathlineto{\pgfqpoint{0.598535in}{1.506724in}}%
\pgfpathlineto{\pgfqpoint{0.604558in}{1.512747in}}%
\pgfpathlineto{\pgfqpoint{0.610581in}{1.518770in}}%
\pgfpathlineto{\pgfqpoint{0.616604in}{1.524793in}}%
\pgfpathlineto{\pgfqpoint{0.622627in}{1.530816in}}%
\pgfpathlineto{\pgfqpoint{0.628650in}{1.536839in}}%
\pgfpathlineto{\pgfqpoint{0.634673in}{1.542862in}}%
\pgfpathlineto{\pgfqpoint{0.640696in}{1.548885in}}%
\pgfpathlineto{\pgfqpoint{0.646719in}{1.554908in}}%
\pgfpathlineto{\pgfqpoint{0.652742in}{1.560931in}}%
\pgfpathlineto{\pgfqpoint{0.658765in}{1.566955in}}%
\pgfpathlineto{\pgfqpoint{0.664788in}{1.572978in}}%
\pgfpathlineto{\pgfqpoint{0.670811in}{1.579001in}}%
\pgfpathlineto{\pgfqpoint{0.676834in}{1.585024in}}%
\pgfpathlineto{\pgfqpoint{0.682857in}{1.591047in}}%
\pgfpathlineto{\pgfqpoint{0.688880in}{1.597070in}}%
\pgfpathlineto{\pgfqpoint{0.694903in}{1.603093in}}%
\pgfpathlineto{\pgfqpoint{0.700927in}{1.609116in}}%
\pgfpathlineto{\pgfqpoint{0.706950in}{1.615139in}}%
\pgfpathlineto{\pgfqpoint{0.712973in}{1.621162in}}%
\pgfpathlineto{\pgfqpoint{0.718996in}{1.627185in}}%
\pgfpathlineto{\pgfqpoint{0.725019in}{1.633208in}}%
\pgfpathlineto{\pgfqpoint{0.731042in}{1.639231in}}%
\pgfpathlineto{\pgfqpoint{0.737065in}{1.645254in}}%
\pgfpathlineto{\pgfqpoint{0.743088in}{1.651277in}}%
\pgfpathlineto{\pgfqpoint{0.749111in}{1.657300in}}%
\pgfpathlineto{\pgfqpoint{0.755134in}{1.663323in}}%
\pgfpathlineto{\pgfqpoint{0.761157in}{1.669347in}}%
\pgfpathlineto{\pgfqpoint{0.767180in}{1.675370in}}%
\pgfpathlineto{\pgfqpoint{0.773203in}{1.681393in}}%
\pgfpathlineto{\pgfqpoint{0.779226in}{1.687416in}}%
\pgfpathlineto{\pgfqpoint{0.785249in}{1.693439in}}%
\pgfpathlineto{\pgfqpoint{0.791272in}{1.699462in}}%
\pgfpathlineto{\pgfqpoint{0.797295in}{1.705485in}}%
\pgfpathlineto{\pgfqpoint{0.803318in}{1.711508in}}%
\pgfpathlineto{\pgfqpoint{0.809342in}{1.717531in}}%
\pgfpathlineto{\pgfqpoint{0.815365in}{1.723554in}}%
\pgfpathlineto{\pgfqpoint{0.821388in}{1.729577in}}%
\pgfpathlineto{\pgfqpoint{0.827411in}{1.735600in}}%
\pgfpathlineto{\pgfqpoint{0.833434in}{1.741623in}}%
\pgfpathlineto{\pgfqpoint{0.839457in}{1.747646in}}%
\pgfpathlineto{\pgfqpoint{0.845480in}{1.753669in}}%
\pgfpathlineto{\pgfqpoint{0.851503in}{1.759692in}}%
\pgfpathlineto{\pgfqpoint{0.857526in}{1.765715in}}%
\pgfpathlineto{\pgfqpoint{0.863549in}{1.771738in}}%
\pgfpathlineto{\pgfqpoint{0.869572in}{1.777762in}}%
\pgfpathlineto{\pgfqpoint{0.875595in}{1.783785in}}%
\pgfpathlineto{\pgfqpoint{0.881618in}{1.789808in}}%
\pgfpathlineto{\pgfqpoint{0.887641in}{1.795831in}}%
\pgfpathlineto{\pgfqpoint{0.893664in}{1.801854in}}%
\pgfpathlineto{\pgfqpoint{0.899687in}{1.807877in}}%
\pgfpathlineto{\pgfqpoint{0.905710in}{1.813900in}}%
\pgfpathlineto{\pgfqpoint{0.911734in}{1.819923in}}%
\pgfpathlineto{\pgfqpoint{0.917757in}{1.825946in}}%
\pgfpathlineto{\pgfqpoint{0.923780in}{1.831969in}}%
\pgfpathlineto{\pgfqpoint{0.929803in}{1.837992in}}%
\pgfpathlineto{\pgfqpoint{0.935826in}{1.844015in}}%
\pgfpathlineto{\pgfqpoint{0.941849in}{1.850038in}}%
\pgfpathlineto{\pgfqpoint{0.947872in}{1.856061in}}%
\pgfpathlineto{\pgfqpoint{0.953895in}{1.862084in}}%
\pgfpathlineto{\pgfqpoint{0.959918in}{1.868107in}}%
\pgfpathlineto{\pgfqpoint{0.965941in}{1.874130in}}%
\pgfpathlineto{\pgfqpoint{0.971964in}{1.880154in}}%
\pgfpathlineto{\pgfqpoint{0.977987in}{1.886177in}}%
\pgfpathlineto{\pgfqpoint{0.984010in}{1.892200in}}%
\pgfpathlineto{\pgfqpoint{0.990033in}{1.898223in}}%
\pgfpathlineto{\pgfqpoint{0.996056in}{1.904246in}}%
\pgfpathlineto{\pgfqpoint{1.002079in}{1.910269in}}%
\pgfpathlineto{\pgfqpoint{1.008102in}{1.916292in}}%
\pgfpathlineto{\pgfqpoint{1.014126in}{1.922315in}}%
\pgfpathlineto{\pgfqpoint{1.020149in}{1.928338in}}%
\pgfpathlineto{\pgfqpoint{1.026172in}{1.934361in}}%
\pgfpathlineto{\pgfqpoint{1.032195in}{1.940384in}}%
\pgfpathlineto{\pgfqpoint{1.038218in}{1.946407in}}%
\pgfpathlineto{\pgfqpoint{1.044241in}{1.952430in}}%
\pgfpathlineto{\pgfqpoint{1.050264in}{1.958453in}}%
\pgfpathlineto{\pgfqpoint{1.056287in}{1.964476in}}%
\pgfpathlineto{\pgfqpoint{1.062310in}{1.970499in}}%
\pgfpathlineto{\pgfqpoint{1.074356in}{1.970499in}}%
\pgfpathlineto{\pgfqpoint{1.080379in}{1.964476in}}%
\pgfpathlineto{\pgfqpoint{1.086402in}{1.958453in}}%
\pgfpathlineto{\pgfqpoint{1.092425in}{1.952430in}}%
\pgfpathlineto{\pgfqpoint{1.098448in}{1.946407in}}%
\pgfpathlineto{\pgfqpoint{1.104471in}{1.940384in}}%
\pgfpathlineto{\pgfqpoint{1.110494in}{1.934361in}}%
\pgfpathlineto{\pgfqpoint{1.116517in}{1.928338in}}%
\pgfpathlineto{\pgfqpoint{1.122541in}{1.922315in}}%
\pgfpathlineto{\pgfqpoint{1.128564in}{1.916292in}}%
\pgfpathlineto{\pgfqpoint{1.134587in}{1.910269in}}%
\pgfpathlineto{\pgfqpoint{1.140610in}{1.904246in}}%
\pgfpathlineto{\pgfqpoint{1.146633in}{1.898223in}}%
\pgfpathlineto{\pgfqpoint{1.152656in}{1.892200in}}%
\pgfpathlineto{\pgfqpoint{1.158679in}{1.886177in}}%
\pgfpathlineto{\pgfqpoint{1.164702in}{1.880154in}}%
\pgfpathlineto{\pgfqpoint{1.170725in}{1.874130in}}%
\pgfpathlineto{\pgfqpoint{1.176748in}{1.868107in}}%
\pgfpathlineto{\pgfqpoint{1.182771in}{1.862084in}}%
\pgfpathlineto{\pgfqpoint{1.188794in}{1.856061in}}%
\pgfpathlineto{\pgfqpoint{1.194817in}{1.850038in}}%
\pgfpathlineto{\pgfqpoint{1.200840in}{1.844015in}}%
\pgfpathlineto{\pgfqpoint{1.206863in}{1.837992in}}%
\pgfpathlineto{\pgfqpoint{1.212886in}{1.831969in}}%
\pgfpathlineto{\pgfqpoint{1.218909in}{1.825946in}}%
\pgfpathlineto{\pgfqpoint{1.224933in}{1.819923in}}%
\pgfpathlineto{\pgfqpoint{1.230956in}{1.813900in}}%
\pgfpathlineto{\pgfqpoint{1.236979in}{1.807877in}}%
\pgfpathlineto{\pgfqpoint{1.243002in}{1.801854in}}%
\pgfpathlineto{\pgfqpoint{1.249025in}{1.795831in}}%
\pgfpathlineto{\pgfqpoint{1.255048in}{1.789808in}}%
\pgfpathlineto{\pgfqpoint{1.261071in}{1.783785in}}%
\pgfpathlineto{\pgfqpoint{1.267094in}{1.777762in}}%
\pgfpathlineto{\pgfqpoint{1.273117in}{1.771738in}}%
\pgfpathlineto{\pgfqpoint{1.279140in}{1.765715in}}%
\pgfpathlineto{\pgfqpoint{1.285163in}{1.759692in}}%
\pgfpathlineto{\pgfqpoint{1.291186in}{1.753669in}}%
\pgfpathlineto{\pgfqpoint{1.297209in}{1.747646in}}%
\pgfpathlineto{\pgfqpoint{1.303232in}{1.741623in}}%
\pgfpathlineto{\pgfqpoint{1.309255in}{1.735600in}}%
\pgfpathlineto{\pgfqpoint{1.315278in}{1.729577in}}%
\pgfpathlineto{\pgfqpoint{1.321301in}{1.723554in}}%
\pgfpathlineto{\pgfqpoint{1.327325in}{1.717531in}}%
\pgfpathlineto{\pgfqpoint{1.333348in}{1.711508in}}%
\pgfpathlineto{\pgfqpoint{1.339371in}{1.705485in}}%
\pgfpathlineto{\pgfqpoint{1.345394in}{1.699462in}}%
\pgfpathlineto{\pgfqpoint{1.351417in}{1.693439in}}%
\pgfpathlineto{\pgfqpoint{1.357440in}{1.687416in}}%
\pgfpathlineto{\pgfqpoint{1.363463in}{1.681393in}}%
\pgfpathlineto{\pgfqpoint{1.369486in}{1.675370in}}%
\pgfpathlineto{\pgfqpoint{1.375509in}{1.669347in}}%
\pgfpathlineto{\pgfqpoint{1.381532in}{1.663323in}}%
\pgfpathlineto{\pgfqpoint{1.387555in}{1.657300in}}%
\pgfpathlineto{\pgfqpoint{1.393578in}{1.651277in}}%
\pgfpathlineto{\pgfqpoint{1.399601in}{1.645254in}}%
\pgfpathlineto{\pgfqpoint{1.405624in}{1.639231in}}%
\pgfpathlineto{\pgfqpoint{1.411647in}{1.633208in}}%
\pgfpathlineto{\pgfqpoint{1.417670in}{1.627185in}}%
\pgfpathlineto{\pgfqpoint{1.423693in}{1.621162in}}%
\pgfpathlineto{\pgfqpoint{1.429716in}{1.615139in}}%
\pgfpathlineto{\pgfqpoint{1.435740in}{1.609116in}}%
\pgfpathlineto{\pgfqpoint{1.441763in}{1.603093in}}%
\pgfpathlineto{\pgfqpoint{1.447786in}{1.597070in}}%
\pgfpathlineto{\pgfqpoint{1.453809in}{1.591047in}}%
\pgfpathlineto{\pgfqpoint{1.459832in}{1.585024in}}%
\pgfpathlineto{\pgfqpoint{1.465855in}{1.579001in}}%
\pgfpathlineto{\pgfqpoint{1.471878in}{1.572978in}}%
\pgfpathlineto{\pgfqpoint{1.477901in}{1.566955in}}%
\pgfpathlineto{\pgfqpoint{1.483924in}{1.560931in}}%
\pgfpathlineto{\pgfqpoint{1.489947in}{1.554908in}}%
\pgfpathlineto{\pgfqpoint{1.495970in}{1.548885in}}%
\pgfpathlineto{\pgfqpoint{1.501993in}{1.542862in}}%
\pgfpathlineto{\pgfqpoint{1.508016in}{1.536839in}}%
\pgfpathlineto{\pgfqpoint{1.514039in}{1.530816in}}%
\pgfpathlineto{\pgfqpoint{1.520062in}{1.524793in}}%
\pgfpathlineto{\pgfqpoint{1.526085in}{1.518770in}}%
\pgfpathlineto{\pgfqpoint{1.532108in}{1.512747in}}%
\pgfpathlineto{\pgfqpoint{1.538132in}{1.506724in}}%
\pgfpathlineto{\pgfqpoint{1.544155in}{1.500701in}}%
\pgfpathlineto{\pgfqpoint{1.550178in}{1.494678in}}%
\pgfpathlineto{\pgfqpoint{1.556201in}{1.488655in}}%
\pgfpathlineto{\pgfqpoint{1.562224in}{1.482632in}}%
\pgfpathlineto{\pgfqpoint{1.568247in}{1.476609in}}%
\pgfpathlineto{\pgfqpoint{1.574270in}{1.470586in}}%
\pgfpathlineto{\pgfqpoint{1.580293in}{1.464563in}}%
\pgfpathlineto{\pgfqpoint{1.586316in}{1.458539in}}%
\pgfpathlineto{\pgfqpoint{1.592339in}{1.452516in}}%
\pgfpathlineto{\pgfqpoint{1.598362in}{1.446493in}}%
\pgfpathlineto{\pgfqpoint{1.604385in}{1.440470in}}%
\pgfpathlineto{\pgfqpoint{1.610408in}{1.434447in}}%
\pgfpathlineto{\pgfqpoint{1.616431in}{1.428424in}}%
\pgfpathlineto{\pgfqpoint{1.622454in}{1.422401in}}%
\pgfpathlineto{\pgfqpoint{1.628477in}{1.416378in}}%
\pgfpathlineto{\pgfqpoint{1.634500in}{1.410355in}}%
\pgfpathlineto{\pgfqpoint{1.640524in}{1.404332in}}%
\pgfpathlineto{\pgfqpoint{1.646547in}{1.398309in}}%
\pgfpathlineto{\pgfqpoint{1.652570in}{1.392286in}}%
\pgfpathlineto{\pgfqpoint{1.658593in}{1.386263in}}%
\pgfpathlineto{\pgfqpoint{1.664616in}{1.380240in}}%
\pgfpathlineto{\pgfqpoint{1.670639in}{1.374217in}}%
\pgfpathlineto{\pgfqpoint{1.676662in}{1.368194in}}%
\pgfpathlineto{\pgfqpoint{1.682685in}{1.362171in}}%
\pgfpathlineto{\pgfqpoint{1.688708in}{1.356148in}}%
\pgfpathlineto{\pgfqpoint{1.694731in}{1.350124in}}%
\pgfpathlineto{\pgfqpoint{1.700754in}{1.344101in}}%
\pgfpathlineto{\pgfqpoint{1.706777in}{1.338078in}}%
\pgfpathlineto{\pgfqpoint{1.712800in}{1.332055in}}%
\pgfpathlineto{\pgfqpoint{1.718823in}{1.326032in}}%
\pgfpathlineto{\pgfqpoint{1.724846in}{1.320009in}}%
\pgfpathlineto{\pgfqpoint{1.730869in}{1.313986in}}%
\pgfpathlineto{\pgfqpoint{1.736892in}{1.307963in}}%
\pgfpathlineto{\pgfqpoint{1.742915in}{1.301940in}}%
\pgfpathlineto{\pgfqpoint{1.748939in}{1.295917in}}%
\pgfpathlineto{\pgfqpoint{1.754962in}{1.289894in}}%
\pgfpathlineto{\pgfqpoint{1.760985in}{1.283871in}}%
\pgfpathlineto{\pgfqpoint{1.767008in}{1.277848in}}%
\pgfpathlineto{\pgfqpoint{1.773031in}{1.271825in}}%
\pgfpathlineto{\pgfqpoint{1.779054in}{1.265802in}}%
\pgfpathlineto{\pgfqpoint{1.785077in}{1.259779in}}%
\pgfpathlineto{\pgfqpoint{1.791100in}{1.253756in}}%
\pgfpathlineto{\pgfqpoint{1.797123in}{1.247732in}}%
\pgfpathlineto{\pgfqpoint{1.803146in}{1.241709in}}%
\pgfpathlineto{\pgfqpoint{1.809169in}{1.235686in}}%
\pgfpathlineto{\pgfqpoint{1.815192in}{1.229663in}}%
\pgfpathlineto{\pgfqpoint{1.821215in}{1.223640in}}%
\pgfpathlineto{\pgfqpoint{1.827238in}{1.217617in}}%
\pgfpathlineto{\pgfqpoint{1.833261in}{1.211594in}}%
\pgfpathlineto{\pgfqpoint{1.839284in}{1.205571in}}%
\pgfpathlineto{\pgfqpoint{1.845307in}{1.199548in}}%
\pgfpathlineto{\pgfqpoint{1.851331in}{1.193525in}}%
\pgfpathlineto{\pgfqpoint{1.857354in}{1.187502in}}%
\pgfpathlineto{\pgfqpoint{1.863377in}{1.181479in}}%
\pgfpathlineto{\pgfqpoint{1.869400in}{1.175456in}}%
\pgfpathlineto{\pgfqpoint{1.875423in}{1.169433in}}%
\pgfpathlineto{\pgfqpoint{1.881446in}{1.163410in}}%
\pgfpathlineto{\pgfqpoint{1.887469in}{1.157387in}}%
\pgfpathlineto{\pgfqpoint{1.893492in}{1.151364in}}%
\pgfpathlineto{\pgfqpoint{1.899515in}{1.145340in}}%
\pgfpathlineto{\pgfqpoint{1.905538in}{1.139317in}}%
\pgfpathlineto{\pgfqpoint{1.911561in}{1.133294in}}%
\pgfpathlineto{\pgfqpoint{1.917584in}{1.127271in}}%
\pgfpathlineto{\pgfqpoint{1.923607in}{1.121248in}}%
\pgfpathlineto{\pgfqpoint{1.929630in}{1.115225in}}%
\pgfpathlineto{\pgfqpoint{1.935653in}{1.109202in}}%
\pgfpathlineto{\pgfqpoint{1.941676in}{1.103179in}}%
\pgfpathlineto{\pgfqpoint{1.947699in}{1.097156in}}%
\pgfpathlineto{\pgfqpoint{1.953723in}{1.091133in}}%
\pgfpathlineto{\pgfqpoint{1.959746in}{1.085110in}}%
\pgfpathlineto{\pgfqpoint{1.959746in}{1.073064in}}%
\pgfpathlineto{\pgfqpoint{1.953723in}{1.067041in}}%
\pgfpathlineto{\pgfqpoint{1.947699in}{1.061018in}}%
\pgfpathlineto{\pgfqpoint{1.941676in}{1.054995in}}%
\pgfpathlineto{\pgfqpoint{1.935653in}{1.048972in}}%
\pgfpathlineto{\pgfqpoint{1.929630in}{1.042949in}}%
\pgfpathlineto{\pgfqpoint{1.923607in}{1.036925in}}%
\pgfpathlineto{\pgfqpoint{1.917584in}{1.030902in}}%
\pgfpathlineto{\pgfqpoint{1.911561in}{1.024879in}}%
\pgfpathlineto{\pgfqpoint{1.905538in}{1.018856in}}%
\pgfpathlineto{\pgfqpoint{1.899515in}{1.012833in}}%
\pgfpathlineto{\pgfqpoint{1.893492in}{1.006810in}}%
\pgfpathlineto{\pgfqpoint{1.887469in}{1.000787in}}%
\pgfpathlineto{\pgfqpoint{1.881446in}{0.994764in}}%
\pgfpathlineto{\pgfqpoint{1.875423in}{0.988741in}}%
\pgfpathlineto{\pgfqpoint{1.869400in}{0.982718in}}%
\pgfpathlineto{\pgfqpoint{1.863377in}{0.976695in}}%
\pgfpathlineto{\pgfqpoint{1.857354in}{0.970672in}}%
\pgfpathlineto{\pgfqpoint{1.851331in}{0.964649in}}%
\pgfpathlineto{\pgfqpoint{1.845307in}{0.958626in}}%
\pgfpathlineto{\pgfqpoint{1.839284in}{0.952603in}}%
\pgfpathlineto{\pgfqpoint{1.833261in}{0.946580in}}%
\pgfpathlineto{\pgfqpoint{1.827238in}{0.940557in}}%
\pgfpathlineto{\pgfqpoint{1.821215in}{0.934533in}}%
\pgfpathlineto{\pgfqpoint{1.815192in}{0.928510in}}%
\pgfpathlineto{\pgfqpoint{1.809169in}{0.922487in}}%
\pgfpathlineto{\pgfqpoint{1.803146in}{0.916464in}}%
\pgfpathlineto{\pgfqpoint{1.797123in}{0.910441in}}%
\pgfpathlineto{\pgfqpoint{1.791100in}{0.904418in}}%
\pgfpathlineto{\pgfqpoint{1.785077in}{0.898395in}}%
\pgfpathlineto{\pgfqpoint{1.779054in}{0.892372in}}%
\pgfpathlineto{\pgfqpoint{1.773031in}{0.886349in}}%
\pgfpathlineto{\pgfqpoint{1.767008in}{0.880326in}}%
\pgfpathlineto{\pgfqpoint{1.760985in}{0.874303in}}%
\pgfpathlineto{\pgfqpoint{1.754962in}{0.868280in}}%
\pgfpathlineto{\pgfqpoint{1.748939in}{0.862257in}}%
\pgfpathlineto{\pgfqpoint{1.742915in}{0.856234in}}%
\pgfpathlineto{\pgfqpoint{1.736892in}{0.850211in}}%
\pgfpathlineto{\pgfqpoint{1.730869in}{0.844188in}}%
\pgfpathlineto{\pgfqpoint{1.724846in}{0.838165in}}%
\pgfpathlineto{\pgfqpoint{1.718823in}{0.832141in}}%
\pgfpathlineto{\pgfqpoint{1.712800in}{0.826118in}}%
\pgfpathlineto{\pgfqpoint{1.706777in}{0.820095in}}%
\pgfpathlineto{\pgfqpoint{1.700754in}{0.814072in}}%
\pgfpathlineto{\pgfqpoint{1.694731in}{0.808049in}}%
\pgfpathlineto{\pgfqpoint{1.688708in}{0.802026in}}%
\pgfpathlineto{\pgfqpoint{1.682685in}{0.796003in}}%
\pgfpathlineto{\pgfqpoint{1.676662in}{0.789980in}}%
\pgfpathlineto{\pgfqpoint{1.670639in}{0.783957in}}%
\pgfpathlineto{\pgfqpoint{1.664616in}{0.777934in}}%
\pgfpathlineto{\pgfqpoint{1.658593in}{0.771911in}}%
\pgfpathlineto{\pgfqpoint{1.652570in}{0.765888in}}%
\pgfpathlineto{\pgfqpoint{1.646547in}{0.759865in}}%
\pgfpathlineto{\pgfqpoint{1.640524in}{0.753842in}}%
\pgfpathlineto{\pgfqpoint{1.634500in}{0.747819in}}%
\pgfpathlineto{\pgfqpoint{1.628477in}{0.741796in}}%
\pgfpathlineto{\pgfqpoint{1.622454in}{0.735773in}}%
\pgfpathlineto{\pgfqpoint{1.616431in}{0.729750in}}%
\pgfpathlineto{\pgfqpoint{1.610408in}{0.723726in}}%
\pgfpathlineto{\pgfqpoint{1.604385in}{0.717703in}}%
\pgfpathlineto{\pgfqpoint{1.598362in}{0.711680in}}%
\pgfpathlineto{\pgfqpoint{1.592339in}{0.705657in}}%
\pgfpathlineto{\pgfqpoint{1.586316in}{0.699634in}}%
\pgfpathlineto{\pgfqpoint{1.580293in}{0.693611in}}%
\pgfpathlineto{\pgfqpoint{1.574270in}{0.687588in}}%
\pgfpathlineto{\pgfqpoint{1.568247in}{0.681565in}}%
\pgfpathlineto{\pgfqpoint{1.562224in}{0.675542in}}%
\pgfpathlineto{\pgfqpoint{1.556201in}{0.669519in}}%
\pgfpathlineto{\pgfqpoint{1.550178in}{0.663496in}}%
\pgfpathlineto{\pgfqpoint{1.544155in}{0.657473in}}%
\pgfpathlineto{\pgfqpoint{1.538132in}{0.651450in}}%
\pgfpathlineto{\pgfqpoint{1.532108in}{0.645427in}}%
\pgfpathlineto{\pgfqpoint{1.526085in}{0.639404in}}%
\pgfpathlineto{\pgfqpoint{1.520062in}{0.633381in}}%
\pgfpathlineto{\pgfqpoint{1.514039in}{0.627358in}}%
\pgfpathlineto{\pgfqpoint{1.508016in}{0.621334in}}%
\pgfpathlineto{\pgfqpoint{1.501993in}{0.615311in}}%
\pgfpathlineto{\pgfqpoint{1.495970in}{0.609288in}}%
\pgfpathlineto{\pgfqpoint{1.489947in}{0.603265in}}%
\pgfpathlineto{\pgfqpoint{1.483924in}{0.597242in}}%
\pgfpathlineto{\pgfqpoint{1.477901in}{0.591219in}}%
\pgfpathlineto{\pgfqpoint{1.471878in}{0.585196in}}%
\pgfpathlineto{\pgfqpoint{1.465855in}{0.579173in}}%
\pgfpathlineto{\pgfqpoint{1.459832in}{0.573150in}}%
\pgfpathlineto{\pgfqpoint{1.453809in}{0.567127in}}%
\pgfpathlineto{\pgfqpoint{1.447786in}{0.561104in}}%
\pgfpathlineto{\pgfqpoint{1.441763in}{0.555081in}}%
\pgfpathlineto{\pgfqpoint{1.435740in}{0.549058in}}%
\pgfpathlineto{\pgfqpoint{1.429716in}{0.543035in}}%
\pgfpathlineto{\pgfqpoint{1.423693in}{0.537012in}}%
\pgfpathlineto{\pgfqpoint{1.417670in}{0.530989in}}%
\pgfpathlineto{\pgfqpoint{1.411647in}{0.524966in}}%
\pgfpathlineto{\pgfqpoint{1.405624in}{0.518942in}}%
\pgfpathlineto{\pgfqpoint{1.399601in}{0.512919in}}%
\pgfpathlineto{\pgfqpoint{1.393578in}{0.506896in}}%
\pgfpathlineto{\pgfqpoint{1.387555in}{0.500873in}}%
\pgfpathlineto{\pgfqpoint{1.381532in}{0.494850in}}%
\pgfpathlineto{\pgfqpoint{1.375509in}{0.488827in}}%
\pgfpathlineto{\pgfqpoint{1.369486in}{0.482804in}}%
\pgfpathlineto{\pgfqpoint{1.363463in}{0.476781in}}%
\pgfpathlineto{\pgfqpoint{1.357440in}{0.470758in}}%
\pgfpathlineto{\pgfqpoint{1.351417in}{0.464735in}}%
\pgfpathlineto{\pgfqpoint{1.345394in}{0.458712in}}%
\pgfpathlineto{\pgfqpoint{1.339371in}{0.452689in}}%
\pgfpathlineto{\pgfqpoint{1.333348in}{0.446666in}}%
\pgfpathlineto{\pgfqpoint{1.327325in}{0.440643in}}%
\pgfpathlineto{\pgfqpoint{1.321301in}{0.434620in}}%
\pgfpathlineto{\pgfqpoint{1.315278in}{0.428597in}}%
\pgfpathlineto{\pgfqpoint{1.309255in}{0.422574in}}%
\pgfpathlineto{\pgfqpoint{1.303232in}{0.416551in}}%
\pgfpathlineto{\pgfqpoint{1.297209in}{0.410527in}}%
\pgfpathlineto{\pgfqpoint{1.291186in}{0.404504in}}%
\pgfpathlineto{\pgfqpoint{1.285163in}{0.398481in}}%
\pgfpathlineto{\pgfqpoint{1.279140in}{0.392458in}}%
\pgfpathlineto{\pgfqpoint{1.273117in}{0.386435in}}%
\pgfpathlineto{\pgfqpoint{1.267094in}{0.380412in}}%
\pgfpathlineto{\pgfqpoint{1.261071in}{0.374389in}}%
\pgfpathlineto{\pgfqpoint{1.255048in}{0.368366in}}%
\pgfpathlineto{\pgfqpoint{1.249025in}{0.362343in}}%
\pgfpathlineto{\pgfqpoint{1.243002in}{0.356320in}}%
\pgfpathlineto{\pgfqpoint{1.236979in}{0.350297in}}%
\pgfpathlineto{\pgfqpoint{1.230956in}{0.344274in}}%
\pgfpathlineto{\pgfqpoint{1.224933in}{0.338251in}}%
\pgfpathlineto{\pgfqpoint{1.218909in}{0.332228in}}%
\pgfpathlineto{\pgfqpoint{1.212886in}{0.326205in}}%
\pgfpathlineto{\pgfqpoint{1.206863in}{0.320182in}}%
\pgfpathlineto{\pgfqpoint{1.200840in}{0.314159in}}%
\pgfpathlineto{\pgfqpoint{1.194817in}{0.308135in}}%
\pgfpathlineto{\pgfqpoint{1.188794in}{0.302112in}}%
\pgfpathlineto{\pgfqpoint{1.182771in}{0.296089in}}%
\pgfpathlineto{\pgfqpoint{1.176748in}{0.290066in}}%
\pgfpathlineto{\pgfqpoint{1.170725in}{0.284043in}}%
\pgfpathlineto{\pgfqpoint{1.164702in}{0.278020in}}%
\pgfpathlineto{\pgfqpoint{1.158679in}{0.271997in}}%
\pgfpathlineto{\pgfqpoint{1.152656in}{0.265974in}}%
\pgfpathlineto{\pgfqpoint{1.146633in}{0.259951in}}%
\pgfpathlineto{\pgfqpoint{1.140610in}{0.253928in}}%
\pgfpathlineto{\pgfqpoint{1.134587in}{0.247905in}}%
\pgfpathlineto{\pgfqpoint{1.128564in}{0.241882in}}%
\pgfpathlineto{\pgfqpoint{1.122541in}{0.235859in}}%
\pgfpathlineto{\pgfqpoint{1.116517in}{0.229836in}}%
\pgfpathlineto{\pgfqpoint{1.110494in}{0.223813in}}%
\pgfpathlineto{\pgfqpoint{1.104471in}{0.217790in}}%
\pgfpathlineto{\pgfqpoint{1.098448in}{0.211767in}}%
\pgfpathlineto{\pgfqpoint{1.092425in}{0.205743in}}%
\pgfpathlineto{\pgfqpoint{1.086402in}{0.199720in}}%
\pgfpathlineto{\pgfqpoint{1.080379in}{0.193697in}}%
\pgfpathlineto{\pgfqpoint{1.074356in}{0.187674in}}%
\pgfpathlineto{\pgfqpoint{1.062310in}{0.187674in}}%
\pgfpathclose%
\pgfusepath{}%
\end{pgfscope}%
\begin{pgfscope}%
\pgfsetbuttcap%
\pgfsetroundjoin%
\definecolor{currentfill}{rgb}{0.000000,0.000000,0.000000}%
\pgfsetfillcolor{currentfill}%
\pgfsetlinewidth{0.803000pt}%
\definecolor{currentstroke}{rgb}{0.000000,0.000000,0.000000}%
\pgfsetstrokecolor{currentstroke}%
\pgfsetdash{}{0pt}%
\pgfsys@defobject{currentmarker}{\pgfqpoint{0.000000in}{-0.048611in}}{\pgfqpoint{0.000000in}{0.000000in}}{%
\pgfpathmoveto{\pgfqpoint{0.000000in}{0.000000in}}%
\pgfpathlineto{\pgfqpoint{0.000000in}{-0.048611in}}%
\pgfusepath{stroke,fill}%
}%
\begin{pgfscope}%
\pgfsys@transformshift{0.170897in}{1.079087in}%
\pgfsys@useobject{currentmarker}{}%
\end{pgfscope}%
\end{pgfscope}%
\begin{pgfscope}%
\pgftext[x=0.170897in,y=0.981865in,,top]{\sffamily\fontsize{10.000000}{12.000000}\selectfont -1}%
\end{pgfscope}%
\begin{pgfscope}%
\pgfsetbuttcap%
\pgfsetroundjoin%
\definecolor{currentfill}{rgb}{0.000000,0.000000,0.000000}%
\pgfsetfillcolor{currentfill}%
\pgfsetlinewidth{0.803000pt}%
\definecolor{currentstroke}{rgb}{0.000000,0.000000,0.000000}%
\pgfsetstrokecolor{currentstroke}%
\pgfsetdash{}{0pt}%
\pgfsys@defobject{currentmarker}{\pgfqpoint{0.000000in}{-0.048611in}}{\pgfqpoint{0.000000in}{0.000000in}}{%
\pgfpathmoveto{\pgfqpoint{0.000000in}{0.000000in}}%
\pgfpathlineto{\pgfqpoint{0.000000in}{-0.048611in}}%
\pgfusepath{stroke,fill}%
}%
\begin{pgfscope}%
\pgfsys@transformshift{0.619615in}{1.079087in}%
\pgfsys@useobject{currentmarker}{}%
\end{pgfscope}%
\end{pgfscope}%
\begin{pgfscope}%
\pgftext[x=0.619615in,y=0.981865in,,top]{\sffamily\fontsize{10.000000}{12.000000}\selectfont -0.5}%
\end{pgfscope}%
\begin{pgfscope}%
\pgfsetbuttcap%
\pgfsetroundjoin%
\definecolor{currentfill}{rgb}{0.000000,0.000000,0.000000}%
\pgfsetfillcolor{currentfill}%
\pgfsetlinewidth{0.803000pt}%
\definecolor{currentstroke}{rgb}{0.000000,0.000000,0.000000}%
\pgfsetstrokecolor{currentstroke}%
\pgfsetdash{}{0pt}%
\pgfsys@defobject{currentmarker}{\pgfqpoint{0.000000in}{-0.048611in}}{\pgfqpoint{0.000000in}{0.000000in}}{%
\pgfpathmoveto{\pgfqpoint{0.000000in}{0.000000in}}%
\pgfpathlineto{\pgfqpoint{0.000000in}{-0.048611in}}%
\pgfusepath{stroke,fill}%
}%
\begin{pgfscope}%
\pgfsys@transformshift{1.068333in}{1.079087in}%
\pgfsys@useobject{currentmarker}{}%
\end{pgfscope}%
\end{pgfscope}%
\begin{pgfscope}%
\pgfsetbuttcap%
\pgfsetroundjoin%
\definecolor{currentfill}{rgb}{0.000000,0.000000,0.000000}%
\pgfsetfillcolor{currentfill}%
\pgfsetlinewidth{0.803000pt}%
\definecolor{currentstroke}{rgb}{0.000000,0.000000,0.000000}%
\pgfsetstrokecolor{currentstroke}%
\pgfsetdash{}{0pt}%
\pgfsys@defobject{currentmarker}{\pgfqpoint{0.000000in}{-0.048611in}}{\pgfqpoint{0.000000in}{0.000000in}}{%
\pgfpathmoveto{\pgfqpoint{0.000000in}{0.000000in}}%
\pgfpathlineto{\pgfqpoint{0.000000in}{-0.048611in}}%
\pgfusepath{stroke,fill}%
}%
\begin{pgfscope}%
\pgfsys@transformshift{1.517051in}{1.079087in}%
\pgfsys@useobject{currentmarker}{}%
\end{pgfscope}%
\end{pgfscope}%
\begin{pgfscope}%
\pgftext[x=1.517051in,y=0.981865in,,top]{\sffamily\fontsize{10.000000}{12.000000}\selectfont 0.5}%
\end{pgfscope}%
\begin{pgfscope}%
\pgfsetbuttcap%
\pgfsetroundjoin%
\definecolor{currentfill}{rgb}{0.000000,0.000000,0.000000}%
\pgfsetfillcolor{currentfill}%
\pgfsetlinewidth{0.803000pt}%
\definecolor{currentstroke}{rgb}{0.000000,0.000000,0.000000}%
\pgfsetstrokecolor{currentstroke}%
\pgfsetdash{}{0pt}%
\pgfsys@defobject{currentmarker}{\pgfqpoint{0.000000in}{-0.048611in}}{\pgfqpoint{0.000000in}{0.000000in}}{%
\pgfpathmoveto{\pgfqpoint{0.000000in}{0.000000in}}%
\pgfpathlineto{\pgfqpoint{0.000000in}{-0.048611in}}%
\pgfusepath{stroke,fill}%
}%
\begin{pgfscope}%
\pgfsys@transformshift{1.965769in}{1.079087in}%
\pgfsys@useobject{currentmarker}{}%
\end{pgfscope}%
\end{pgfscope}%
\begin{pgfscope}%
\pgftext[x=1.965769in,y=0.981865in,,top]{\sffamily\fontsize{10.000000}{12.000000}\selectfont 1}%
\end{pgfscope}%
\begin{pgfscope}%
\pgfsetbuttcap%
\pgfsetroundjoin%
\definecolor{currentfill}{rgb}{0.000000,0.000000,0.000000}%
\pgfsetfillcolor{currentfill}%
\pgfsetlinewidth{0.602250pt}%
\definecolor{currentstroke}{rgb}{0.000000,0.000000,0.000000}%
\pgfsetstrokecolor{currentstroke}%
\pgfsetdash{}{0pt}%
\pgfsys@defobject{currentmarker}{\pgfqpoint{0.000000in}{-0.027778in}}{\pgfqpoint{0.000000in}{0.000000in}}{%
\pgfpathmoveto{\pgfqpoint{0.000000in}{0.000000in}}%
\pgfpathlineto{\pgfqpoint{0.000000in}{-0.027778in}}%
\pgfusepath{stroke,fill}%
}%
\begin{pgfscope}%
\pgfsys@transformshift{0.260641in}{1.079087in}%
\pgfsys@useobject{currentmarker}{}%
\end{pgfscope}%
\end{pgfscope}%
\begin{pgfscope}%
\pgfsetbuttcap%
\pgfsetroundjoin%
\definecolor{currentfill}{rgb}{0.000000,0.000000,0.000000}%
\pgfsetfillcolor{currentfill}%
\pgfsetlinewidth{0.602250pt}%
\definecolor{currentstroke}{rgb}{0.000000,0.000000,0.000000}%
\pgfsetstrokecolor{currentstroke}%
\pgfsetdash{}{0pt}%
\pgfsys@defobject{currentmarker}{\pgfqpoint{0.000000in}{-0.027778in}}{\pgfqpoint{0.000000in}{0.000000in}}{%
\pgfpathmoveto{\pgfqpoint{0.000000in}{0.000000in}}%
\pgfpathlineto{\pgfqpoint{0.000000in}{-0.027778in}}%
\pgfusepath{stroke,fill}%
}%
\begin{pgfscope}%
\pgfsys@transformshift{0.350385in}{1.079087in}%
\pgfsys@useobject{currentmarker}{}%
\end{pgfscope}%
\end{pgfscope}%
\begin{pgfscope}%
\pgfsetbuttcap%
\pgfsetroundjoin%
\definecolor{currentfill}{rgb}{0.000000,0.000000,0.000000}%
\pgfsetfillcolor{currentfill}%
\pgfsetlinewidth{0.602250pt}%
\definecolor{currentstroke}{rgb}{0.000000,0.000000,0.000000}%
\pgfsetstrokecolor{currentstroke}%
\pgfsetdash{}{0pt}%
\pgfsys@defobject{currentmarker}{\pgfqpoint{0.000000in}{-0.027778in}}{\pgfqpoint{0.000000in}{0.000000in}}{%
\pgfpathmoveto{\pgfqpoint{0.000000in}{0.000000in}}%
\pgfpathlineto{\pgfqpoint{0.000000in}{-0.027778in}}%
\pgfusepath{stroke,fill}%
}%
\begin{pgfscope}%
\pgfsys@transformshift{0.440128in}{1.079087in}%
\pgfsys@useobject{currentmarker}{}%
\end{pgfscope}%
\end{pgfscope}%
\begin{pgfscope}%
\pgfsetbuttcap%
\pgfsetroundjoin%
\definecolor{currentfill}{rgb}{0.000000,0.000000,0.000000}%
\pgfsetfillcolor{currentfill}%
\pgfsetlinewidth{0.602250pt}%
\definecolor{currentstroke}{rgb}{0.000000,0.000000,0.000000}%
\pgfsetstrokecolor{currentstroke}%
\pgfsetdash{}{0pt}%
\pgfsys@defobject{currentmarker}{\pgfqpoint{0.000000in}{-0.027778in}}{\pgfqpoint{0.000000in}{0.000000in}}{%
\pgfpathmoveto{\pgfqpoint{0.000000in}{0.000000in}}%
\pgfpathlineto{\pgfqpoint{0.000000in}{-0.027778in}}%
\pgfusepath{stroke,fill}%
}%
\begin{pgfscope}%
\pgfsys@transformshift{0.529872in}{1.079087in}%
\pgfsys@useobject{currentmarker}{}%
\end{pgfscope}%
\end{pgfscope}%
\begin{pgfscope}%
\pgfsetbuttcap%
\pgfsetroundjoin%
\definecolor{currentfill}{rgb}{0.000000,0.000000,0.000000}%
\pgfsetfillcolor{currentfill}%
\pgfsetlinewidth{0.602250pt}%
\definecolor{currentstroke}{rgb}{0.000000,0.000000,0.000000}%
\pgfsetstrokecolor{currentstroke}%
\pgfsetdash{}{0pt}%
\pgfsys@defobject{currentmarker}{\pgfqpoint{0.000000in}{-0.027778in}}{\pgfqpoint{0.000000in}{0.000000in}}{%
\pgfpathmoveto{\pgfqpoint{0.000000in}{0.000000in}}%
\pgfpathlineto{\pgfqpoint{0.000000in}{-0.027778in}}%
\pgfusepath{stroke,fill}%
}%
\begin{pgfscope}%
\pgfsys@transformshift{0.619615in}{1.079087in}%
\pgfsys@useobject{currentmarker}{}%
\end{pgfscope}%
\end{pgfscope}%
\begin{pgfscope}%
\pgfsetbuttcap%
\pgfsetroundjoin%
\definecolor{currentfill}{rgb}{0.000000,0.000000,0.000000}%
\pgfsetfillcolor{currentfill}%
\pgfsetlinewidth{0.602250pt}%
\definecolor{currentstroke}{rgb}{0.000000,0.000000,0.000000}%
\pgfsetstrokecolor{currentstroke}%
\pgfsetdash{}{0pt}%
\pgfsys@defobject{currentmarker}{\pgfqpoint{0.000000in}{-0.027778in}}{\pgfqpoint{0.000000in}{0.000000in}}{%
\pgfpathmoveto{\pgfqpoint{0.000000in}{0.000000in}}%
\pgfpathlineto{\pgfqpoint{0.000000in}{-0.027778in}}%
\pgfusepath{stroke,fill}%
}%
\begin{pgfscope}%
\pgfsys@transformshift{0.709359in}{1.079087in}%
\pgfsys@useobject{currentmarker}{}%
\end{pgfscope}%
\end{pgfscope}%
\begin{pgfscope}%
\pgfsetbuttcap%
\pgfsetroundjoin%
\definecolor{currentfill}{rgb}{0.000000,0.000000,0.000000}%
\pgfsetfillcolor{currentfill}%
\pgfsetlinewidth{0.602250pt}%
\definecolor{currentstroke}{rgb}{0.000000,0.000000,0.000000}%
\pgfsetstrokecolor{currentstroke}%
\pgfsetdash{}{0pt}%
\pgfsys@defobject{currentmarker}{\pgfqpoint{0.000000in}{-0.027778in}}{\pgfqpoint{0.000000in}{0.000000in}}{%
\pgfpathmoveto{\pgfqpoint{0.000000in}{0.000000in}}%
\pgfpathlineto{\pgfqpoint{0.000000in}{-0.027778in}}%
\pgfusepath{stroke,fill}%
}%
\begin{pgfscope}%
\pgfsys@transformshift{0.799102in}{1.079087in}%
\pgfsys@useobject{currentmarker}{}%
\end{pgfscope}%
\end{pgfscope}%
\begin{pgfscope}%
\pgfsetbuttcap%
\pgfsetroundjoin%
\definecolor{currentfill}{rgb}{0.000000,0.000000,0.000000}%
\pgfsetfillcolor{currentfill}%
\pgfsetlinewidth{0.602250pt}%
\definecolor{currentstroke}{rgb}{0.000000,0.000000,0.000000}%
\pgfsetstrokecolor{currentstroke}%
\pgfsetdash{}{0pt}%
\pgfsys@defobject{currentmarker}{\pgfqpoint{0.000000in}{-0.027778in}}{\pgfqpoint{0.000000in}{0.000000in}}{%
\pgfpathmoveto{\pgfqpoint{0.000000in}{0.000000in}}%
\pgfpathlineto{\pgfqpoint{0.000000in}{-0.027778in}}%
\pgfusepath{stroke,fill}%
}%
\begin{pgfscope}%
\pgfsys@transformshift{0.888846in}{1.079087in}%
\pgfsys@useobject{currentmarker}{}%
\end{pgfscope}%
\end{pgfscope}%
\begin{pgfscope}%
\pgfsetbuttcap%
\pgfsetroundjoin%
\definecolor{currentfill}{rgb}{0.000000,0.000000,0.000000}%
\pgfsetfillcolor{currentfill}%
\pgfsetlinewidth{0.602250pt}%
\definecolor{currentstroke}{rgb}{0.000000,0.000000,0.000000}%
\pgfsetstrokecolor{currentstroke}%
\pgfsetdash{}{0pt}%
\pgfsys@defobject{currentmarker}{\pgfqpoint{0.000000in}{-0.027778in}}{\pgfqpoint{0.000000in}{0.000000in}}{%
\pgfpathmoveto{\pgfqpoint{0.000000in}{0.000000in}}%
\pgfpathlineto{\pgfqpoint{0.000000in}{-0.027778in}}%
\pgfusepath{stroke,fill}%
}%
\begin{pgfscope}%
\pgfsys@transformshift{0.978589in}{1.079087in}%
\pgfsys@useobject{currentmarker}{}%
\end{pgfscope}%
\end{pgfscope}%
\begin{pgfscope}%
\pgfsetbuttcap%
\pgfsetroundjoin%
\definecolor{currentfill}{rgb}{0.000000,0.000000,0.000000}%
\pgfsetfillcolor{currentfill}%
\pgfsetlinewidth{0.602250pt}%
\definecolor{currentstroke}{rgb}{0.000000,0.000000,0.000000}%
\pgfsetstrokecolor{currentstroke}%
\pgfsetdash{}{0pt}%
\pgfsys@defobject{currentmarker}{\pgfqpoint{0.000000in}{-0.027778in}}{\pgfqpoint{0.000000in}{0.000000in}}{%
\pgfpathmoveto{\pgfqpoint{0.000000in}{0.000000in}}%
\pgfpathlineto{\pgfqpoint{0.000000in}{-0.027778in}}%
\pgfusepath{stroke,fill}%
}%
\begin{pgfscope}%
\pgfsys@transformshift{1.068333in}{1.079087in}%
\pgfsys@useobject{currentmarker}{}%
\end{pgfscope}%
\end{pgfscope}%
\begin{pgfscope}%
\pgfsetbuttcap%
\pgfsetroundjoin%
\definecolor{currentfill}{rgb}{0.000000,0.000000,0.000000}%
\pgfsetfillcolor{currentfill}%
\pgfsetlinewidth{0.602250pt}%
\definecolor{currentstroke}{rgb}{0.000000,0.000000,0.000000}%
\pgfsetstrokecolor{currentstroke}%
\pgfsetdash{}{0pt}%
\pgfsys@defobject{currentmarker}{\pgfqpoint{0.000000in}{-0.027778in}}{\pgfqpoint{0.000000in}{0.000000in}}{%
\pgfpathmoveto{\pgfqpoint{0.000000in}{0.000000in}}%
\pgfpathlineto{\pgfqpoint{0.000000in}{-0.027778in}}%
\pgfusepath{stroke,fill}%
}%
\begin{pgfscope}%
\pgfsys@transformshift{1.158077in}{1.079087in}%
\pgfsys@useobject{currentmarker}{}%
\end{pgfscope}%
\end{pgfscope}%
\begin{pgfscope}%
\pgfsetbuttcap%
\pgfsetroundjoin%
\definecolor{currentfill}{rgb}{0.000000,0.000000,0.000000}%
\pgfsetfillcolor{currentfill}%
\pgfsetlinewidth{0.602250pt}%
\definecolor{currentstroke}{rgb}{0.000000,0.000000,0.000000}%
\pgfsetstrokecolor{currentstroke}%
\pgfsetdash{}{0pt}%
\pgfsys@defobject{currentmarker}{\pgfqpoint{0.000000in}{-0.027778in}}{\pgfqpoint{0.000000in}{0.000000in}}{%
\pgfpathmoveto{\pgfqpoint{0.000000in}{0.000000in}}%
\pgfpathlineto{\pgfqpoint{0.000000in}{-0.027778in}}%
\pgfusepath{stroke,fill}%
}%
\begin{pgfscope}%
\pgfsys@transformshift{1.247820in}{1.079087in}%
\pgfsys@useobject{currentmarker}{}%
\end{pgfscope}%
\end{pgfscope}%
\begin{pgfscope}%
\pgfsetbuttcap%
\pgfsetroundjoin%
\definecolor{currentfill}{rgb}{0.000000,0.000000,0.000000}%
\pgfsetfillcolor{currentfill}%
\pgfsetlinewidth{0.602250pt}%
\definecolor{currentstroke}{rgb}{0.000000,0.000000,0.000000}%
\pgfsetstrokecolor{currentstroke}%
\pgfsetdash{}{0pt}%
\pgfsys@defobject{currentmarker}{\pgfqpoint{0.000000in}{-0.027778in}}{\pgfqpoint{0.000000in}{0.000000in}}{%
\pgfpathmoveto{\pgfqpoint{0.000000in}{0.000000in}}%
\pgfpathlineto{\pgfqpoint{0.000000in}{-0.027778in}}%
\pgfusepath{stroke,fill}%
}%
\begin{pgfscope}%
\pgfsys@transformshift{1.337564in}{1.079087in}%
\pgfsys@useobject{currentmarker}{}%
\end{pgfscope}%
\end{pgfscope}%
\begin{pgfscope}%
\pgfsetbuttcap%
\pgfsetroundjoin%
\definecolor{currentfill}{rgb}{0.000000,0.000000,0.000000}%
\pgfsetfillcolor{currentfill}%
\pgfsetlinewidth{0.602250pt}%
\definecolor{currentstroke}{rgb}{0.000000,0.000000,0.000000}%
\pgfsetstrokecolor{currentstroke}%
\pgfsetdash{}{0pt}%
\pgfsys@defobject{currentmarker}{\pgfqpoint{0.000000in}{-0.027778in}}{\pgfqpoint{0.000000in}{0.000000in}}{%
\pgfpathmoveto{\pgfqpoint{0.000000in}{0.000000in}}%
\pgfpathlineto{\pgfqpoint{0.000000in}{-0.027778in}}%
\pgfusepath{stroke,fill}%
}%
\begin{pgfscope}%
\pgfsys@transformshift{1.427307in}{1.079087in}%
\pgfsys@useobject{currentmarker}{}%
\end{pgfscope}%
\end{pgfscope}%
\begin{pgfscope}%
\pgfsetbuttcap%
\pgfsetroundjoin%
\definecolor{currentfill}{rgb}{0.000000,0.000000,0.000000}%
\pgfsetfillcolor{currentfill}%
\pgfsetlinewidth{0.602250pt}%
\definecolor{currentstroke}{rgb}{0.000000,0.000000,0.000000}%
\pgfsetstrokecolor{currentstroke}%
\pgfsetdash{}{0pt}%
\pgfsys@defobject{currentmarker}{\pgfqpoint{0.000000in}{-0.027778in}}{\pgfqpoint{0.000000in}{0.000000in}}{%
\pgfpathmoveto{\pgfqpoint{0.000000in}{0.000000in}}%
\pgfpathlineto{\pgfqpoint{0.000000in}{-0.027778in}}%
\pgfusepath{stroke,fill}%
}%
\begin{pgfscope}%
\pgfsys@transformshift{1.517051in}{1.079087in}%
\pgfsys@useobject{currentmarker}{}%
\end{pgfscope}%
\end{pgfscope}%
\begin{pgfscope}%
\pgfsetbuttcap%
\pgfsetroundjoin%
\definecolor{currentfill}{rgb}{0.000000,0.000000,0.000000}%
\pgfsetfillcolor{currentfill}%
\pgfsetlinewidth{0.602250pt}%
\definecolor{currentstroke}{rgb}{0.000000,0.000000,0.000000}%
\pgfsetstrokecolor{currentstroke}%
\pgfsetdash{}{0pt}%
\pgfsys@defobject{currentmarker}{\pgfqpoint{0.000000in}{-0.027778in}}{\pgfqpoint{0.000000in}{0.000000in}}{%
\pgfpathmoveto{\pgfqpoint{0.000000in}{0.000000in}}%
\pgfpathlineto{\pgfqpoint{0.000000in}{-0.027778in}}%
\pgfusepath{stroke,fill}%
}%
\begin{pgfscope}%
\pgfsys@transformshift{1.606794in}{1.079087in}%
\pgfsys@useobject{currentmarker}{}%
\end{pgfscope}%
\end{pgfscope}%
\begin{pgfscope}%
\pgfsetbuttcap%
\pgfsetroundjoin%
\definecolor{currentfill}{rgb}{0.000000,0.000000,0.000000}%
\pgfsetfillcolor{currentfill}%
\pgfsetlinewidth{0.602250pt}%
\definecolor{currentstroke}{rgb}{0.000000,0.000000,0.000000}%
\pgfsetstrokecolor{currentstroke}%
\pgfsetdash{}{0pt}%
\pgfsys@defobject{currentmarker}{\pgfqpoint{0.000000in}{-0.027778in}}{\pgfqpoint{0.000000in}{0.000000in}}{%
\pgfpathmoveto{\pgfqpoint{0.000000in}{0.000000in}}%
\pgfpathlineto{\pgfqpoint{0.000000in}{-0.027778in}}%
\pgfusepath{stroke,fill}%
}%
\begin{pgfscope}%
\pgfsys@transformshift{1.696538in}{1.079087in}%
\pgfsys@useobject{currentmarker}{}%
\end{pgfscope}%
\end{pgfscope}%
\begin{pgfscope}%
\pgfsetbuttcap%
\pgfsetroundjoin%
\definecolor{currentfill}{rgb}{0.000000,0.000000,0.000000}%
\pgfsetfillcolor{currentfill}%
\pgfsetlinewidth{0.602250pt}%
\definecolor{currentstroke}{rgb}{0.000000,0.000000,0.000000}%
\pgfsetstrokecolor{currentstroke}%
\pgfsetdash{}{0pt}%
\pgfsys@defobject{currentmarker}{\pgfqpoint{0.000000in}{-0.027778in}}{\pgfqpoint{0.000000in}{0.000000in}}{%
\pgfpathmoveto{\pgfqpoint{0.000000in}{0.000000in}}%
\pgfpathlineto{\pgfqpoint{0.000000in}{-0.027778in}}%
\pgfusepath{stroke,fill}%
}%
\begin{pgfscope}%
\pgfsys@transformshift{1.786282in}{1.079087in}%
\pgfsys@useobject{currentmarker}{}%
\end{pgfscope}%
\end{pgfscope}%
\begin{pgfscope}%
\pgfsetbuttcap%
\pgfsetroundjoin%
\definecolor{currentfill}{rgb}{0.000000,0.000000,0.000000}%
\pgfsetfillcolor{currentfill}%
\pgfsetlinewidth{0.602250pt}%
\definecolor{currentstroke}{rgb}{0.000000,0.000000,0.000000}%
\pgfsetstrokecolor{currentstroke}%
\pgfsetdash{}{0pt}%
\pgfsys@defobject{currentmarker}{\pgfqpoint{0.000000in}{-0.027778in}}{\pgfqpoint{0.000000in}{0.000000in}}{%
\pgfpathmoveto{\pgfqpoint{0.000000in}{0.000000in}}%
\pgfpathlineto{\pgfqpoint{0.000000in}{-0.027778in}}%
\pgfusepath{stroke,fill}%
}%
\begin{pgfscope}%
\pgfsys@transformshift{1.876025in}{1.079087in}%
\pgfsys@useobject{currentmarker}{}%
\end{pgfscope}%
\end{pgfscope}%
\begin{pgfscope}%
\pgfsetbuttcap%
\pgfsetroundjoin%
\definecolor{currentfill}{rgb}{0.000000,0.000000,0.000000}%
\pgfsetfillcolor{currentfill}%
\pgfsetlinewidth{0.602250pt}%
\definecolor{currentstroke}{rgb}{0.000000,0.000000,0.000000}%
\pgfsetstrokecolor{currentstroke}%
\pgfsetdash{}{0pt}%
\pgfsys@defobject{currentmarker}{\pgfqpoint{0.000000in}{-0.027778in}}{\pgfqpoint{0.000000in}{0.000000in}}{%
\pgfpathmoveto{\pgfqpoint{0.000000in}{0.000000in}}%
\pgfpathlineto{\pgfqpoint{0.000000in}{-0.027778in}}%
\pgfusepath{stroke,fill}%
}%
\begin{pgfscope}%
\pgfsys@transformshift{1.965769in}{1.079087in}%
\pgfsys@useobject{currentmarker}{}%
\end{pgfscope}%
\end{pgfscope}%
\begin{pgfscope}%
\pgfsetbuttcap%
\pgfsetroundjoin%
\definecolor{currentfill}{rgb}{0.000000,0.000000,0.000000}%
\pgfsetfillcolor{currentfill}%
\pgfsetlinewidth{0.803000pt}%
\definecolor{currentstroke}{rgb}{0.000000,0.000000,0.000000}%
\pgfsetstrokecolor{currentstroke}%
\pgfsetdash{}{0pt}%
\pgfsys@defobject{currentmarker}{\pgfqpoint{-0.048611in}{0.000000in}}{\pgfqpoint{0.000000in}{0.000000in}}{%
\pgfpathmoveto{\pgfqpoint{0.000000in}{0.000000in}}%
\pgfpathlineto{\pgfqpoint{-0.048611in}{0.000000in}}%
\pgfusepath{stroke,fill}%
}%
\begin{pgfscope}%
\pgfsys@transformshift{1.068333in}{0.181651in}%
\pgfsys@useobject{currentmarker}{}%
\end{pgfscope}%
\end{pgfscope}%
\begin{pgfscope}%
\pgftext[x=0.832629in,y=0.128890in,left,base]{\sffamily\fontsize{10.000000}{12.000000}\selectfont -1}%
\end{pgfscope}%
\begin{pgfscope}%
\pgfsetbuttcap%
\pgfsetroundjoin%
\definecolor{currentfill}{rgb}{0.000000,0.000000,0.000000}%
\pgfsetfillcolor{currentfill}%
\pgfsetlinewidth{0.803000pt}%
\definecolor{currentstroke}{rgb}{0.000000,0.000000,0.000000}%
\pgfsetstrokecolor{currentstroke}%
\pgfsetdash{}{0pt}%
\pgfsys@defobject{currentmarker}{\pgfqpoint{-0.048611in}{0.000000in}}{\pgfqpoint{0.000000in}{0.000000in}}{%
\pgfpathmoveto{\pgfqpoint{0.000000in}{0.000000in}}%
\pgfpathlineto{\pgfqpoint{-0.048611in}{0.000000in}}%
\pgfusepath{stroke,fill}%
}%
\begin{pgfscope}%
\pgfsys@transformshift{1.068333in}{0.630369in}%
\pgfsys@useobject{currentmarker}{}%
\end{pgfscope}%
\end{pgfscope}%
\begin{pgfscope}%
\pgftext[x=0.700115in,y=0.577608in,left,base]{\sffamily\fontsize{10.000000}{12.000000}\selectfont -0.5}%
\end{pgfscope}%
\begin{pgfscope}%
\pgfsetbuttcap%
\pgfsetroundjoin%
\definecolor{currentfill}{rgb}{0.000000,0.000000,0.000000}%
\pgfsetfillcolor{currentfill}%
\pgfsetlinewidth{0.803000pt}%
\definecolor{currentstroke}{rgb}{0.000000,0.000000,0.000000}%
\pgfsetstrokecolor{currentstroke}%
\pgfsetdash{}{0pt}%
\pgfsys@defobject{currentmarker}{\pgfqpoint{-0.048611in}{0.000000in}}{\pgfqpoint{0.000000in}{0.000000in}}{%
\pgfpathmoveto{\pgfqpoint{0.000000in}{0.000000in}}%
\pgfpathlineto{\pgfqpoint{-0.048611in}{0.000000in}}%
\pgfusepath{stroke,fill}%
}%
\begin{pgfscope}%
\pgfsys@transformshift{1.068333in}{1.079087in}%
\pgfsys@useobject{currentmarker}{}%
\end{pgfscope}%
\end{pgfscope}%
\begin{pgfscope}%
\pgfsetbuttcap%
\pgfsetroundjoin%
\definecolor{currentfill}{rgb}{0.000000,0.000000,0.000000}%
\pgfsetfillcolor{currentfill}%
\pgfsetlinewidth{0.803000pt}%
\definecolor{currentstroke}{rgb}{0.000000,0.000000,0.000000}%
\pgfsetstrokecolor{currentstroke}%
\pgfsetdash{}{0pt}%
\pgfsys@defobject{currentmarker}{\pgfqpoint{-0.048611in}{0.000000in}}{\pgfqpoint{0.000000in}{0.000000in}}{%
\pgfpathmoveto{\pgfqpoint{0.000000in}{0.000000in}}%
\pgfpathlineto{\pgfqpoint{-0.048611in}{0.000000in}}%
\pgfusepath{stroke,fill}%
}%
\begin{pgfscope}%
\pgfsys@transformshift{1.068333in}{1.527805in}%
\pgfsys@useobject{currentmarker}{}%
\end{pgfscope}%
\end{pgfscope}%
\begin{pgfscope}%
\pgftext[x=0.750231in,y=1.475043in,left,base]{\sffamily\fontsize{10.000000}{12.000000}\selectfont 0.5}%
\end{pgfscope}%
\begin{pgfscope}%
\pgfsetbuttcap%
\pgfsetroundjoin%
\definecolor{currentfill}{rgb}{0.000000,0.000000,0.000000}%
\pgfsetfillcolor{currentfill}%
\pgfsetlinewidth{0.803000pt}%
\definecolor{currentstroke}{rgb}{0.000000,0.000000,0.000000}%
\pgfsetstrokecolor{currentstroke}%
\pgfsetdash{}{0pt}%
\pgfsys@defobject{currentmarker}{\pgfqpoint{-0.048611in}{0.000000in}}{\pgfqpoint{0.000000in}{0.000000in}}{%
\pgfpathmoveto{\pgfqpoint{0.000000in}{0.000000in}}%
\pgfpathlineto{\pgfqpoint{-0.048611in}{0.000000in}}%
\pgfusepath{stroke,fill}%
}%
\begin{pgfscope}%
\pgfsys@transformshift{1.068333in}{1.976522in}%
\pgfsys@useobject{currentmarker}{}%
\end{pgfscope}%
\end{pgfscope}%
\begin{pgfscope}%
\pgftext[x=0.882746in,y=1.923761in,left,base]{\sffamily\fontsize{10.000000}{12.000000}\selectfont 1}%
\end{pgfscope}%
\begin{pgfscope}%
\pgfsetbuttcap%
\pgfsetroundjoin%
\definecolor{currentfill}{rgb}{0.000000,0.000000,0.000000}%
\pgfsetfillcolor{currentfill}%
\pgfsetlinewidth{0.602250pt}%
\definecolor{currentstroke}{rgb}{0.000000,0.000000,0.000000}%
\pgfsetstrokecolor{currentstroke}%
\pgfsetdash{}{0pt}%
\pgfsys@defobject{currentmarker}{\pgfqpoint{-0.027778in}{0.000000in}}{\pgfqpoint{0.000000in}{0.000000in}}{%
\pgfpathmoveto{\pgfqpoint{0.000000in}{0.000000in}}%
\pgfpathlineto{\pgfqpoint{-0.027778in}{0.000000in}}%
\pgfusepath{stroke,fill}%
}%
\begin{pgfscope}%
\pgfsys@transformshift{1.068333in}{0.271395in}%
\pgfsys@useobject{currentmarker}{}%
\end{pgfscope}%
\end{pgfscope}%
\begin{pgfscope}%
\pgfsetbuttcap%
\pgfsetroundjoin%
\definecolor{currentfill}{rgb}{0.000000,0.000000,0.000000}%
\pgfsetfillcolor{currentfill}%
\pgfsetlinewidth{0.602250pt}%
\definecolor{currentstroke}{rgb}{0.000000,0.000000,0.000000}%
\pgfsetstrokecolor{currentstroke}%
\pgfsetdash{}{0pt}%
\pgfsys@defobject{currentmarker}{\pgfqpoint{-0.027778in}{0.000000in}}{\pgfqpoint{0.000000in}{0.000000in}}{%
\pgfpathmoveto{\pgfqpoint{0.000000in}{0.000000in}}%
\pgfpathlineto{\pgfqpoint{-0.027778in}{0.000000in}}%
\pgfusepath{stroke,fill}%
}%
\begin{pgfscope}%
\pgfsys@transformshift{1.068333in}{0.361138in}%
\pgfsys@useobject{currentmarker}{}%
\end{pgfscope}%
\end{pgfscope}%
\begin{pgfscope}%
\pgfsetbuttcap%
\pgfsetroundjoin%
\definecolor{currentfill}{rgb}{0.000000,0.000000,0.000000}%
\pgfsetfillcolor{currentfill}%
\pgfsetlinewidth{0.602250pt}%
\definecolor{currentstroke}{rgb}{0.000000,0.000000,0.000000}%
\pgfsetstrokecolor{currentstroke}%
\pgfsetdash{}{0pt}%
\pgfsys@defobject{currentmarker}{\pgfqpoint{-0.027778in}{0.000000in}}{\pgfqpoint{0.000000in}{0.000000in}}{%
\pgfpathmoveto{\pgfqpoint{0.000000in}{0.000000in}}%
\pgfpathlineto{\pgfqpoint{-0.027778in}{0.000000in}}%
\pgfusepath{stroke,fill}%
}%
\begin{pgfscope}%
\pgfsys@transformshift{1.068333in}{0.450882in}%
\pgfsys@useobject{currentmarker}{}%
\end{pgfscope}%
\end{pgfscope}%
\begin{pgfscope}%
\pgfsetbuttcap%
\pgfsetroundjoin%
\definecolor{currentfill}{rgb}{0.000000,0.000000,0.000000}%
\pgfsetfillcolor{currentfill}%
\pgfsetlinewidth{0.602250pt}%
\definecolor{currentstroke}{rgb}{0.000000,0.000000,0.000000}%
\pgfsetstrokecolor{currentstroke}%
\pgfsetdash{}{0pt}%
\pgfsys@defobject{currentmarker}{\pgfqpoint{-0.027778in}{0.000000in}}{\pgfqpoint{0.000000in}{0.000000in}}{%
\pgfpathmoveto{\pgfqpoint{0.000000in}{0.000000in}}%
\pgfpathlineto{\pgfqpoint{-0.027778in}{0.000000in}}%
\pgfusepath{stroke,fill}%
}%
\begin{pgfscope}%
\pgfsys@transformshift{1.068333in}{0.540625in}%
\pgfsys@useobject{currentmarker}{}%
\end{pgfscope}%
\end{pgfscope}%
\begin{pgfscope}%
\pgfsetbuttcap%
\pgfsetroundjoin%
\definecolor{currentfill}{rgb}{0.000000,0.000000,0.000000}%
\pgfsetfillcolor{currentfill}%
\pgfsetlinewidth{0.602250pt}%
\definecolor{currentstroke}{rgb}{0.000000,0.000000,0.000000}%
\pgfsetstrokecolor{currentstroke}%
\pgfsetdash{}{0pt}%
\pgfsys@defobject{currentmarker}{\pgfqpoint{-0.027778in}{0.000000in}}{\pgfqpoint{0.000000in}{0.000000in}}{%
\pgfpathmoveto{\pgfqpoint{0.000000in}{0.000000in}}%
\pgfpathlineto{\pgfqpoint{-0.027778in}{0.000000in}}%
\pgfusepath{stroke,fill}%
}%
\begin{pgfscope}%
\pgfsys@transformshift{1.068333in}{0.630369in}%
\pgfsys@useobject{currentmarker}{}%
\end{pgfscope}%
\end{pgfscope}%
\begin{pgfscope}%
\pgfsetbuttcap%
\pgfsetroundjoin%
\definecolor{currentfill}{rgb}{0.000000,0.000000,0.000000}%
\pgfsetfillcolor{currentfill}%
\pgfsetlinewidth{0.602250pt}%
\definecolor{currentstroke}{rgb}{0.000000,0.000000,0.000000}%
\pgfsetstrokecolor{currentstroke}%
\pgfsetdash{}{0pt}%
\pgfsys@defobject{currentmarker}{\pgfqpoint{-0.027778in}{0.000000in}}{\pgfqpoint{0.000000in}{0.000000in}}{%
\pgfpathmoveto{\pgfqpoint{0.000000in}{0.000000in}}%
\pgfpathlineto{\pgfqpoint{-0.027778in}{0.000000in}}%
\pgfusepath{stroke,fill}%
}%
\begin{pgfscope}%
\pgfsys@transformshift{1.068333in}{0.720113in}%
\pgfsys@useobject{currentmarker}{}%
\end{pgfscope}%
\end{pgfscope}%
\begin{pgfscope}%
\pgfsetbuttcap%
\pgfsetroundjoin%
\definecolor{currentfill}{rgb}{0.000000,0.000000,0.000000}%
\pgfsetfillcolor{currentfill}%
\pgfsetlinewidth{0.602250pt}%
\definecolor{currentstroke}{rgb}{0.000000,0.000000,0.000000}%
\pgfsetstrokecolor{currentstroke}%
\pgfsetdash{}{0pt}%
\pgfsys@defobject{currentmarker}{\pgfqpoint{-0.027778in}{0.000000in}}{\pgfqpoint{0.000000in}{0.000000in}}{%
\pgfpathmoveto{\pgfqpoint{0.000000in}{0.000000in}}%
\pgfpathlineto{\pgfqpoint{-0.027778in}{0.000000in}}%
\pgfusepath{stroke,fill}%
}%
\begin{pgfscope}%
\pgfsys@transformshift{1.068333in}{0.809856in}%
\pgfsys@useobject{currentmarker}{}%
\end{pgfscope}%
\end{pgfscope}%
\begin{pgfscope}%
\pgfsetbuttcap%
\pgfsetroundjoin%
\definecolor{currentfill}{rgb}{0.000000,0.000000,0.000000}%
\pgfsetfillcolor{currentfill}%
\pgfsetlinewidth{0.602250pt}%
\definecolor{currentstroke}{rgb}{0.000000,0.000000,0.000000}%
\pgfsetstrokecolor{currentstroke}%
\pgfsetdash{}{0pt}%
\pgfsys@defobject{currentmarker}{\pgfqpoint{-0.027778in}{0.000000in}}{\pgfqpoint{0.000000in}{0.000000in}}{%
\pgfpathmoveto{\pgfqpoint{0.000000in}{0.000000in}}%
\pgfpathlineto{\pgfqpoint{-0.027778in}{0.000000in}}%
\pgfusepath{stroke,fill}%
}%
\begin{pgfscope}%
\pgfsys@transformshift{1.068333in}{0.899600in}%
\pgfsys@useobject{currentmarker}{}%
\end{pgfscope}%
\end{pgfscope}%
\begin{pgfscope}%
\pgfsetbuttcap%
\pgfsetroundjoin%
\definecolor{currentfill}{rgb}{0.000000,0.000000,0.000000}%
\pgfsetfillcolor{currentfill}%
\pgfsetlinewidth{0.602250pt}%
\definecolor{currentstroke}{rgb}{0.000000,0.000000,0.000000}%
\pgfsetstrokecolor{currentstroke}%
\pgfsetdash{}{0pt}%
\pgfsys@defobject{currentmarker}{\pgfqpoint{-0.027778in}{0.000000in}}{\pgfqpoint{0.000000in}{0.000000in}}{%
\pgfpathmoveto{\pgfqpoint{0.000000in}{0.000000in}}%
\pgfpathlineto{\pgfqpoint{-0.027778in}{0.000000in}}%
\pgfusepath{stroke,fill}%
}%
\begin{pgfscope}%
\pgfsys@transformshift{1.068333in}{0.989343in}%
\pgfsys@useobject{currentmarker}{}%
\end{pgfscope}%
\end{pgfscope}%
\begin{pgfscope}%
\pgfsetbuttcap%
\pgfsetroundjoin%
\definecolor{currentfill}{rgb}{0.000000,0.000000,0.000000}%
\pgfsetfillcolor{currentfill}%
\pgfsetlinewidth{0.602250pt}%
\definecolor{currentstroke}{rgb}{0.000000,0.000000,0.000000}%
\pgfsetstrokecolor{currentstroke}%
\pgfsetdash{}{0pt}%
\pgfsys@defobject{currentmarker}{\pgfqpoint{-0.027778in}{0.000000in}}{\pgfqpoint{0.000000in}{0.000000in}}{%
\pgfpathmoveto{\pgfqpoint{0.000000in}{0.000000in}}%
\pgfpathlineto{\pgfqpoint{-0.027778in}{0.000000in}}%
\pgfusepath{stroke,fill}%
}%
\begin{pgfscope}%
\pgfsys@transformshift{1.068333in}{1.079087in}%
\pgfsys@useobject{currentmarker}{}%
\end{pgfscope}%
\end{pgfscope}%
\begin{pgfscope}%
\pgfsetbuttcap%
\pgfsetroundjoin%
\definecolor{currentfill}{rgb}{0.000000,0.000000,0.000000}%
\pgfsetfillcolor{currentfill}%
\pgfsetlinewidth{0.602250pt}%
\definecolor{currentstroke}{rgb}{0.000000,0.000000,0.000000}%
\pgfsetstrokecolor{currentstroke}%
\pgfsetdash{}{0pt}%
\pgfsys@defobject{currentmarker}{\pgfqpoint{-0.027778in}{0.000000in}}{\pgfqpoint{0.000000in}{0.000000in}}{%
\pgfpathmoveto{\pgfqpoint{0.000000in}{0.000000in}}%
\pgfpathlineto{\pgfqpoint{-0.027778in}{0.000000in}}%
\pgfusepath{stroke,fill}%
}%
\begin{pgfscope}%
\pgfsys@transformshift{1.068333in}{1.168830in}%
\pgfsys@useobject{currentmarker}{}%
\end{pgfscope}%
\end{pgfscope}%
\begin{pgfscope}%
\pgfsetbuttcap%
\pgfsetroundjoin%
\definecolor{currentfill}{rgb}{0.000000,0.000000,0.000000}%
\pgfsetfillcolor{currentfill}%
\pgfsetlinewidth{0.602250pt}%
\definecolor{currentstroke}{rgb}{0.000000,0.000000,0.000000}%
\pgfsetstrokecolor{currentstroke}%
\pgfsetdash{}{0pt}%
\pgfsys@defobject{currentmarker}{\pgfqpoint{-0.027778in}{0.000000in}}{\pgfqpoint{0.000000in}{0.000000in}}{%
\pgfpathmoveto{\pgfqpoint{0.000000in}{0.000000in}}%
\pgfpathlineto{\pgfqpoint{-0.027778in}{0.000000in}}%
\pgfusepath{stroke,fill}%
}%
\begin{pgfscope}%
\pgfsys@transformshift{1.068333in}{1.258574in}%
\pgfsys@useobject{currentmarker}{}%
\end{pgfscope}%
\end{pgfscope}%
\begin{pgfscope}%
\pgfsetbuttcap%
\pgfsetroundjoin%
\definecolor{currentfill}{rgb}{0.000000,0.000000,0.000000}%
\pgfsetfillcolor{currentfill}%
\pgfsetlinewidth{0.602250pt}%
\definecolor{currentstroke}{rgb}{0.000000,0.000000,0.000000}%
\pgfsetstrokecolor{currentstroke}%
\pgfsetdash{}{0pt}%
\pgfsys@defobject{currentmarker}{\pgfqpoint{-0.027778in}{0.000000in}}{\pgfqpoint{0.000000in}{0.000000in}}{%
\pgfpathmoveto{\pgfqpoint{0.000000in}{0.000000in}}%
\pgfpathlineto{\pgfqpoint{-0.027778in}{0.000000in}}%
\pgfusepath{stroke,fill}%
}%
\begin{pgfscope}%
\pgfsys@transformshift{1.068333in}{1.348318in}%
\pgfsys@useobject{currentmarker}{}%
\end{pgfscope}%
\end{pgfscope}%
\begin{pgfscope}%
\pgfsetbuttcap%
\pgfsetroundjoin%
\definecolor{currentfill}{rgb}{0.000000,0.000000,0.000000}%
\pgfsetfillcolor{currentfill}%
\pgfsetlinewidth{0.602250pt}%
\definecolor{currentstroke}{rgb}{0.000000,0.000000,0.000000}%
\pgfsetstrokecolor{currentstroke}%
\pgfsetdash{}{0pt}%
\pgfsys@defobject{currentmarker}{\pgfqpoint{-0.027778in}{0.000000in}}{\pgfqpoint{0.000000in}{0.000000in}}{%
\pgfpathmoveto{\pgfqpoint{0.000000in}{0.000000in}}%
\pgfpathlineto{\pgfqpoint{-0.027778in}{0.000000in}}%
\pgfusepath{stroke,fill}%
}%
\begin{pgfscope}%
\pgfsys@transformshift{1.068333in}{1.438061in}%
\pgfsys@useobject{currentmarker}{}%
\end{pgfscope}%
\end{pgfscope}%
\begin{pgfscope}%
\pgfsetbuttcap%
\pgfsetroundjoin%
\definecolor{currentfill}{rgb}{0.000000,0.000000,0.000000}%
\pgfsetfillcolor{currentfill}%
\pgfsetlinewidth{0.602250pt}%
\definecolor{currentstroke}{rgb}{0.000000,0.000000,0.000000}%
\pgfsetstrokecolor{currentstroke}%
\pgfsetdash{}{0pt}%
\pgfsys@defobject{currentmarker}{\pgfqpoint{-0.027778in}{0.000000in}}{\pgfqpoint{0.000000in}{0.000000in}}{%
\pgfpathmoveto{\pgfqpoint{0.000000in}{0.000000in}}%
\pgfpathlineto{\pgfqpoint{-0.027778in}{0.000000in}}%
\pgfusepath{stroke,fill}%
}%
\begin{pgfscope}%
\pgfsys@transformshift{1.068333in}{1.527805in}%
\pgfsys@useobject{currentmarker}{}%
\end{pgfscope}%
\end{pgfscope}%
\begin{pgfscope}%
\pgfsetbuttcap%
\pgfsetroundjoin%
\definecolor{currentfill}{rgb}{0.000000,0.000000,0.000000}%
\pgfsetfillcolor{currentfill}%
\pgfsetlinewidth{0.602250pt}%
\definecolor{currentstroke}{rgb}{0.000000,0.000000,0.000000}%
\pgfsetstrokecolor{currentstroke}%
\pgfsetdash{}{0pt}%
\pgfsys@defobject{currentmarker}{\pgfqpoint{-0.027778in}{0.000000in}}{\pgfqpoint{0.000000in}{0.000000in}}{%
\pgfpathmoveto{\pgfqpoint{0.000000in}{0.000000in}}%
\pgfpathlineto{\pgfqpoint{-0.027778in}{0.000000in}}%
\pgfusepath{stroke,fill}%
}%
\begin{pgfscope}%
\pgfsys@transformshift{1.068333in}{1.617548in}%
\pgfsys@useobject{currentmarker}{}%
\end{pgfscope}%
\end{pgfscope}%
\begin{pgfscope}%
\pgfsetbuttcap%
\pgfsetroundjoin%
\definecolor{currentfill}{rgb}{0.000000,0.000000,0.000000}%
\pgfsetfillcolor{currentfill}%
\pgfsetlinewidth{0.602250pt}%
\definecolor{currentstroke}{rgb}{0.000000,0.000000,0.000000}%
\pgfsetstrokecolor{currentstroke}%
\pgfsetdash{}{0pt}%
\pgfsys@defobject{currentmarker}{\pgfqpoint{-0.027778in}{0.000000in}}{\pgfqpoint{0.000000in}{0.000000in}}{%
\pgfpathmoveto{\pgfqpoint{0.000000in}{0.000000in}}%
\pgfpathlineto{\pgfqpoint{-0.027778in}{0.000000in}}%
\pgfusepath{stroke,fill}%
}%
\begin{pgfscope}%
\pgfsys@transformshift{1.068333in}{1.707292in}%
\pgfsys@useobject{currentmarker}{}%
\end{pgfscope}%
\end{pgfscope}%
\begin{pgfscope}%
\pgfsetbuttcap%
\pgfsetroundjoin%
\definecolor{currentfill}{rgb}{0.000000,0.000000,0.000000}%
\pgfsetfillcolor{currentfill}%
\pgfsetlinewidth{0.602250pt}%
\definecolor{currentstroke}{rgb}{0.000000,0.000000,0.000000}%
\pgfsetstrokecolor{currentstroke}%
\pgfsetdash{}{0pt}%
\pgfsys@defobject{currentmarker}{\pgfqpoint{-0.027778in}{0.000000in}}{\pgfqpoint{0.000000in}{0.000000in}}{%
\pgfpathmoveto{\pgfqpoint{0.000000in}{0.000000in}}%
\pgfpathlineto{\pgfqpoint{-0.027778in}{0.000000in}}%
\pgfusepath{stroke,fill}%
}%
\begin{pgfscope}%
\pgfsys@transformshift{1.068333in}{1.797035in}%
\pgfsys@useobject{currentmarker}{}%
\end{pgfscope}%
\end{pgfscope}%
\begin{pgfscope}%
\pgfsetbuttcap%
\pgfsetroundjoin%
\definecolor{currentfill}{rgb}{0.000000,0.000000,0.000000}%
\pgfsetfillcolor{currentfill}%
\pgfsetlinewidth{0.602250pt}%
\definecolor{currentstroke}{rgb}{0.000000,0.000000,0.000000}%
\pgfsetstrokecolor{currentstroke}%
\pgfsetdash{}{0pt}%
\pgfsys@defobject{currentmarker}{\pgfqpoint{-0.027778in}{0.000000in}}{\pgfqpoint{0.000000in}{0.000000in}}{%
\pgfpathmoveto{\pgfqpoint{0.000000in}{0.000000in}}%
\pgfpathlineto{\pgfqpoint{-0.027778in}{0.000000in}}%
\pgfusepath{stroke,fill}%
}%
\begin{pgfscope}%
\pgfsys@transformshift{1.068333in}{1.886779in}%
\pgfsys@useobject{currentmarker}{}%
\end{pgfscope}%
\end{pgfscope}%
\begin{pgfscope}%
\pgfsetbuttcap%
\pgfsetroundjoin%
\definecolor{currentfill}{rgb}{0.000000,0.000000,0.000000}%
\pgfsetfillcolor{currentfill}%
\pgfsetlinewidth{0.602250pt}%
\definecolor{currentstroke}{rgb}{0.000000,0.000000,0.000000}%
\pgfsetstrokecolor{currentstroke}%
\pgfsetdash{}{0pt}%
\pgfsys@defobject{currentmarker}{\pgfqpoint{-0.027778in}{0.000000in}}{\pgfqpoint{0.000000in}{0.000000in}}{%
\pgfpathmoveto{\pgfqpoint{0.000000in}{0.000000in}}%
\pgfpathlineto{\pgfqpoint{-0.027778in}{0.000000in}}%
\pgfusepath{stroke,fill}%
}%
\begin{pgfscope}%
\pgfsys@transformshift{1.068333in}{1.976522in}%
\pgfsys@useobject{currentmarker}{}%
\end{pgfscope}%
\end{pgfscope}%
\begin{pgfscope}%
\pgfpathrectangle{\pgfqpoint{0.135000in}{0.145754in}}{\pgfqpoint{1.866666in}{1.866666in}} %
\pgfusepath{clip}%
\pgfsetbuttcap%
\pgfsetroundjoin%
\pgfsetlinewidth{1.505625pt}%
\definecolor{currentstroke}{rgb}{0.000000,0.000000,1.000000}%
\pgfsetstrokecolor{currentstroke}%
\pgfsetdash{}{0pt}%
\pgfpathmoveto{\pgfqpoint{1.062310in}{0.187674in}}%
\pgfpathlineto{\pgfqpoint{0.176920in}{1.073064in}}%
\pgfpathlineto{\pgfqpoint{0.176920in}{1.085110in}}%
\pgfpathlineto{\pgfqpoint{1.062310in}{1.970499in}}%
\pgfpathlineto{\pgfqpoint{1.074356in}{1.970499in}}%
\pgfpathlineto{\pgfqpoint{1.959746in}{1.085110in}}%
\pgfpathlineto{\pgfqpoint{1.959746in}{1.073064in}}%
\pgfpathlineto{\pgfqpoint{1.074356in}{0.187674in}}%
\pgfpathlineto{\pgfqpoint{1.062310in}{0.187674in}}%
\pgfpathlineto{\pgfqpoint{1.062310in}{0.187674in}}%
\pgfusepath{stroke}%
\end{pgfscope}%
\begin{pgfscope}%
\pgfpathrectangle{\pgfqpoint{0.135000in}{0.145754in}}{\pgfqpoint{1.866666in}{1.866666in}} %
\pgfusepath{clip}%
\pgfsetbuttcap%
\pgfsetroundjoin%
\pgfsetlinewidth{1.505625pt}%
\definecolor{currentstroke}{rgb}{1.000000,1.000000,1.000000}%
\pgfsetstrokecolor{currentstroke}%
\pgfsetdash{}{0pt}%
\pgfpathmoveto{\pgfqpoint{1.062310in}{0.187674in}}%
\pgfpathlineto{\pgfqpoint{0.176920in}{1.073064in}}%
\pgfpathlineto{\pgfqpoint{0.176920in}{1.085110in}}%
\pgfpathlineto{\pgfqpoint{1.062310in}{1.970499in}}%
\pgfpathlineto{\pgfqpoint{1.074356in}{1.970499in}}%
\pgfpathlineto{\pgfqpoint{1.959746in}{1.085110in}}%
\pgfpathlineto{\pgfqpoint{1.959746in}{1.073064in}}%
\pgfpathlineto{\pgfqpoint{1.074356in}{0.187674in}}%
\pgfpathlineto{\pgfqpoint{1.062310in}{0.187674in}}%
\pgfpathlineto{\pgfqpoint{1.062310in}{0.187674in}}%
\pgfusepath{stroke}%
\end{pgfscope}%
\begin{pgfscope}%
\pgfpathrectangle{\pgfqpoint{0.135000in}{0.145754in}}{\pgfqpoint{1.866666in}{1.866666in}} %
\pgfusepath{clip}%
\pgfsetbuttcap%
\pgfsetroundjoin%
\pgfsetlinewidth{1.505625pt}%
\definecolor{currentstroke}{rgb}{1.000000,1.000000,1.000000}%
\pgfsetstrokecolor{currentstroke}%
\pgfsetdash{}{0pt}%
\pgfpathmoveto{\pgfqpoint{1.062310in}{0.187674in}}%
\pgfpathlineto{\pgfqpoint{0.176920in}{1.073064in}}%
\pgfpathlineto{\pgfqpoint{0.176920in}{1.085110in}}%
\pgfpathlineto{\pgfqpoint{1.062310in}{1.970499in}}%
\pgfpathlineto{\pgfqpoint{1.074356in}{1.970499in}}%
\pgfpathlineto{\pgfqpoint{1.959746in}{1.085110in}}%
\pgfpathlineto{\pgfqpoint{1.959746in}{1.073064in}}%
\pgfpathlineto{\pgfqpoint{1.074356in}{0.187674in}}%
\pgfpathlineto{\pgfqpoint{1.062310in}{0.187674in}}%
\pgfpathlineto{\pgfqpoint{1.062310in}{0.187674in}}%
\pgfusepath{stroke}%
\end{pgfscope}%
\begin{pgfscope}%
\pgfpathrectangle{\pgfqpoint{0.135000in}{0.145754in}}{\pgfqpoint{1.866666in}{1.866666in}} %
\pgfusepath{clip}%
\pgfsetbuttcap%
\pgfsetroundjoin%
\pgfsetlinewidth{1.505625pt}%
\definecolor{currentstroke}{rgb}{0.000000,0.000000,1.000000}%
\pgfsetstrokecolor{currentstroke}%
\pgfsetdash{}{0pt}%
\pgfpathmoveto{\pgfqpoint{1.062310in}{0.187674in}}%
\pgfpathlineto{\pgfqpoint{0.176920in}{1.073064in}}%
\pgfpathlineto{\pgfqpoint{0.176920in}{1.085110in}}%
\pgfpathlineto{\pgfqpoint{1.062310in}{1.970499in}}%
\pgfpathlineto{\pgfqpoint{1.074356in}{1.970499in}}%
\pgfpathlineto{\pgfqpoint{1.959746in}{1.085110in}}%
\pgfpathlineto{\pgfqpoint{1.959746in}{1.073064in}}%
\pgfpathlineto{\pgfqpoint{1.074356in}{0.187674in}}%
\pgfpathlineto{\pgfqpoint{1.062310in}{0.187674in}}%
\pgfpathlineto{\pgfqpoint{1.062310in}{0.187674in}}%
\pgfusepath{stroke}%
\end{pgfscope}%
\begin{pgfscope}%
\pgfsetrectcap%
\pgfsetmiterjoin%
\pgfsetlinewidth{0.803000pt}%
\definecolor{currentstroke}{rgb}{0.000000,0.000000,0.000000}%
\pgfsetstrokecolor{currentstroke}%
\pgfsetdash{}{0pt}%
\pgfpathmoveto{\pgfqpoint{1.068333in}{0.145754in}}%
\pgfpathlineto{\pgfqpoint{1.068333in}{2.012420in}}%
\pgfusepath{stroke}%
\end{pgfscope}%
\begin{pgfscope}%
\pgfsetrectcap%
\pgfsetmiterjoin%
\pgfsetlinewidth{0.803000pt}%
\definecolor{currentstroke}{rgb}{0.000000,0.000000,0.000000}%
\pgfsetstrokecolor{currentstroke}%
\pgfsetdash{}{0pt}%
\pgfpathmoveto{\pgfqpoint{0.135000in}{1.079087in}}%
\pgfpathlineto{\pgfqpoint{2.001666in}{1.079087in}}%
\pgfusepath{stroke}%
\end{pgfscope}%
\end{pgfpicture}%
\makeatother%
\endgroup%

            \end{center}
        \item No es una m\'etrica, por ejemplo, $d\left( (1,0), (1,1) \right) = 0$, pero $(1,0) \neq (1,1)$.
        \item Sí que es una m\'etrica
            \begin{center}
                %% Creator: Matplotlib, PGF backend
%%
%% To include the figure in your LaTeX document, write
%%   \input{<filename>.pgf}
%%
%% Make sure the required packages are loaded in your preamble
%%   \usepackage{pgf}
%%
%% Figures using additional raster images can only be included by \input if
%% they are in the same directory as the main LaTeX file. For loading figures
%% from other directories you can use the `import` package
%%   \usepackage{import}
%% and then include the figures with
%%   \import{<path to file>}{<filename>.pgf}
%%
%% Matplotlib used the following preamble
%%   \usepackage{fontspec}
%%   \setmainfont{DejaVu Serif}
%%   \setsansfont{DejaVu Sans}
%%   \setmonofont{DejaVu Sans Mono}
%%
\begingroup%
\makeatletter%
\begin{pgfpicture}%
\pgfpathrectangle{\pgfpointorigin}{\pgfqpoint{2.136666in}{2.147420in}}%
\pgfusepath{use as bounding box, clip}%
\begin{pgfscope}%
\pgfsetbuttcap%
\pgfsetmiterjoin%
\definecolor{currentfill}{rgb}{1.000000,1.000000,1.000000}%
\pgfsetfillcolor{currentfill}%
\pgfsetlinewidth{0.000000pt}%
\definecolor{currentstroke}{rgb}{1.000000,1.000000,1.000000}%
\pgfsetstrokecolor{currentstroke}%
\pgfsetdash{}{0pt}%
\pgfpathmoveto{\pgfqpoint{0.000000in}{0.000000in}}%
\pgfpathlineto{\pgfqpoint{2.136666in}{0.000000in}}%
\pgfpathlineto{\pgfqpoint{2.136666in}{2.147420in}}%
\pgfpathlineto{\pgfqpoint{0.000000in}{2.147420in}}%
\pgfpathclose%
\pgfusepath{fill}%
\end{pgfscope}%
\begin{pgfscope}%
\pgfsetbuttcap%
\pgfsetmiterjoin%
\definecolor{currentfill}{rgb}{1.000000,1.000000,1.000000}%
\pgfsetfillcolor{currentfill}%
\pgfsetlinewidth{0.000000pt}%
\definecolor{currentstroke}{rgb}{0.000000,0.000000,0.000000}%
\pgfsetstrokecolor{currentstroke}%
\pgfsetstrokeopacity{0.000000}%
\pgfsetdash{}{0pt}%
\pgfpathmoveto{\pgfqpoint{0.135000in}{0.145754in}}%
\pgfpathlineto{\pgfqpoint{2.001666in}{0.145754in}}%
\pgfpathlineto{\pgfqpoint{2.001666in}{2.012420in}}%
\pgfpathlineto{\pgfqpoint{0.135000in}{2.012420in}}%
\pgfpathclose%
\pgfusepath{fill}%
\end{pgfscope}%
\begin{pgfscope}%
\pgfpathrectangle{\pgfqpoint{0.135000in}{0.145754in}}{\pgfqpoint{1.866666in}{1.866666in}} %
\pgfusepath{clip}%
\pgfsetbuttcap%
\pgfsetmiterjoin%
\definecolor{currentfill}{rgb}{0.000000,0.000000,1.000000}%
\pgfsetfillcolor{currentfill}%
\pgfsetfillopacity{0.300000}%
\pgfsetlinewidth{0.000000pt}%
\definecolor{currentstroke}{rgb}{0.000000,0.000000,1.000000}%
\pgfsetstrokecolor{currentstroke}%
\pgfsetstrokeopacity{0.300000}%
\pgfsetdash{}{0pt}%
\pgfpathmoveto{\pgfqpoint{0.170897in}{0.181651in}}%
\pgfpathlineto{\pgfqpoint{0.170897in}{1.976522in}}%
\pgfpathlineto{\pgfqpoint{1.965769in}{1.976522in}}%
\pgfpathlineto{\pgfqpoint{1.965769in}{0.181651in}}%
\pgfpathclose%
\pgfusepath{fill}%
\end{pgfscope}%
\begin{pgfscope}%
\pgfsetbuttcap%
\pgfsetroundjoin%
\definecolor{currentfill}{rgb}{0.000000,0.000000,0.000000}%
\pgfsetfillcolor{currentfill}%
\pgfsetlinewidth{0.803000pt}%
\definecolor{currentstroke}{rgb}{0.000000,0.000000,0.000000}%
\pgfsetstrokecolor{currentstroke}%
\pgfsetdash{}{0pt}%
\pgfsys@defobject{currentmarker}{\pgfqpoint{0.000000in}{-0.048611in}}{\pgfqpoint{0.000000in}{0.000000in}}{%
\pgfpathmoveto{\pgfqpoint{0.000000in}{0.000000in}}%
\pgfpathlineto{\pgfqpoint{0.000000in}{-0.048611in}}%
\pgfusepath{stroke,fill}%
}%
\begin{pgfscope}%
\pgfsys@transformshift{0.170897in}{1.079087in}%
\pgfsys@useobject{currentmarker}{}%
\end{pgfscope}%
\end{pgfscope}%
\begin{pgfscope}%
\pgftext[x=0.170897in,y=0.981865in,,top]{\sffamily\fontsize{10.000000}{12.000000}\selectfont -1}%
\end{pgfscope}%
\begin{pgfscope}%
\pgfsetbuttcap%
\pgfsetroundjoin%
\definecolor{currentfill}{rgb}{0.000000,0.000000,0.000000}%
\pgfsetfillcolor{currentfill}%
\pgfsetlinewidth{0.803000pt}%
\definecolor{currentstroke}{rgb}{0.000000,0.000000,0.000000}%
\pgfsetstrokecolor{currentstroke}%
\pgfsetdash{}{0pt}%
\pgfsys@defobject{currentmarker}{\pgfqpoint{0.000000in}{-0.048611in}}{\pgfqpoint{0.000000in}{0.000000in}}{%
\pgfpathmoveto{\pgfqpoint{0.000000in}{0.000000in}}%
\pgfpathlineto{\pgfqpoint{0.000000in}{-0.048611in}}%
\pgfusepath{stroke,fill}%
}%
\begin{pgfscope}%
\pgfsys@transformshift{0.619615in}{1.079087in}%
\pgfsys@useobject{currentmarker}{}%
\end{pgfscope}%
\end{pgfscope}%
\begin{pgfscope}%
\pgftext[x=0.619615in,y=0.981865in,,top]{\sffamily\fontsize{10.000000}{12.000000}\selectfont -0.5}%
\end{pgfscope}%
\begin{pgfscope}%
\pgfsetbuttcap%
\pgfsetroundjoin%
\definecolor{currentfill}{rgb}{0.000000,0.000000,0.000000}%
\pgfsetfillcolor{currentfill}%
\pgfsetlinewidth{0.803000pt}%
\definecolor{currentstroke}{rgb}{0.000000,0.000000,0.000000}%
\pgfsetstrokecolor{currentstroke}%
\pgfsetdash{}{0pt}%
\pgfsys@defobject{currentmarker}{\pgfqpoint{0.000000in}{-0.048611in}}{\pgfqpoint{0.000000in}{0.000000in}}{%
\pgfpathmoveto{\pgfqpoint{0.000000in}{0.000000in}}%
\pgfpathlineto{\pgfqpoint{0.000000in}{-0.048611in}}%
\pgfusepath{stroke,fill}%
}%
\begin{pgfscope}%
\pgfsys@transformshift{1.068333in}{1.079087in}%
\pgfsys@useobject{currentmarker}{}%
\end{pgfscope}%
\end{pgfscope}%
\begin{pgfscope}%
\pgfsetbuttcap%
\pgfsetroundjoin%
\definecolor{currentfill}{rgb}{0.000000,0.000000,0.000000}%
\pgfsetfillcolor{currentfill}%
\pgfsetlinewidth{0.803000pt}%
\definecolor{currentstroke}{rgb}{0.000000,0.000000,0.000000}%
\pgfsetstrokecolor{currentstroke}%
\pgfsetdash{}{0pt}%
\pgfsys@defobject{currentmarker}{\pgfqpoint{0.000000in}{-0.048611in}}{\pgfqpoint{0.000000in}{0.000000in}}{%
\pgfpathmoveto{\pgfqpoint{0.000000in}{0.000000in}}%
\pgfpathlineto{\pgfqpoint{0.000000in}{-0.048611in}}%
\pgfusepath{stroke,fill}%
}%
\begin{pgfscope}%
\pgfsys@transformshift{1.517051in}{1.079087in}%
\pgfsys@useobject{currentmarker}{}%
\end{pgfscope}%
\end{pgfscope}%
\begin{pgfscope}%
\pgftext[x=1.517051in,y=0.981865in,,top]{\sffamily\fontsize{10.000000}{12.000000}\selectfont 0.5}%
\end{pgfscope}%
\begin{pgfscope}%
\pgfsetbuttcap%
\pgfsetroundjoin%
\definecolor{currentfill}{rgb}{0.000000,0.000000,0.000000}%
\pgfsetfillcolor{currentfill}%
\pgfsetlinewidth{0.803000pt}%
\definecolor{currentstroke}{rgb}{0.000000,0.000000,0.000000}%
\pgfsetstrokecolor{currentstroke}%
\pgfsetdash{}{0pt}%
\pgfsys@defobject{currentmarker}{\pgfqpoint{0.000000in}{-0.048611in}}{\pgfqpoint{0.000000in}{0.000000in}}{%
\pgfpathmoveto{\pgfqpoint{0.000000in}{0.000000in}}%
\pgfpathlineto{\pgfqpoint{0.000000in}{-0.048611in}}%
\pgfusepath{stroke,fill}%
}%
\begin{pgfscope}%
\pgfsys@transformshift{1.965769in}{1.079087in}%
\pgfsys@useobject{currentmarker}{}%
\end{pgfscope}%
\end{pgfscope}%
\begin{pgfscope}%
\pgftext[x=1.965769in,y=0.981865in,,top]{\sffamily\fontsize{10.000000}{12.000000}\selectfont 1}%
\end{pgfscope}%
\begin{pgfscope}%
\pgfsetbuttcap%
\pgfsetroundjoin%
\definecolor{currentfill}{rgb}{0.000000,0.000000,0.000000}%
\pgfsetfillcolor{currentfill}%
\pgfsetlinewidth{0.602250pt}%
\definecolor{currentstroke}{rgb}{0.000000,0.000000,0.000000}%
\pgfsetstrokecolor{currentstroke}%
\pgfsetdash{}{0pt}%
\pgfsys@defobject{currentmarker}{\pgfqpoint{0.000000in}{-0.027778in}}{\pgfqpoint{0.000000in}{0.000000in}}{%
\pgfpathmoveto{\pgfqpoint{0.000000in}{0.000000in}}%
\pgfpathlineto{\pgfqpoint{0.000000in}{-0.027778in}}%
\pgfusepath{stroke,fill}%
}%
\begin{pgfscope}%
\pgfsys@transformshift{0.260641in}{1.079087in}%
\pgfsys@useobject{currentmarker}{}%
\end{pgfscope}%
\end{pgfscope}%
\begin{pgfscope}%
\pgfsetbuttcap%
\pgfsetroundjoin%
\definecolor{currentfill}{rgb}{0.000000,0.000000,0.000000}%
\pgfsetfillcolor{currentfill}%
\pgfsetlinewidth{0.602250pt}%
\definecolor{currentstroke}{rgb}{0.000000,0.000000,0.000000}%
\pgfsetstrokecolor{currentstroke}%
\pgfsetdash{}{0pt}%
\pgfsys@defobject{currentmarker}{\pgfqpoint{0.000000in}{-0.027778in}}{\pgfqpoint{0.000000in}{0.000000in}}{%
\pgfpathmoveto{\pgfqpoint{0.000000in}{0.000000in}}%
\pgfpathlineto{\pgfqpoint{0.000000in}{-0.027778in}}%
\pgfusepath{stroke,fill}%
}%
\begin{pgfscope}%
\pgfsys@transformshift{0.350385in}{1.079087in}%
\pgfsys@useobject{currentmarker}{}%
\end{pgfscope}%
\end{pgfscope}%
\begin{pgfscope}%
\pgfsetbuttcap%
\pgfsetroundjoin%
\definecolor{currentfill}{rgb}{0.000000,0.000000,0.000000}%
\pgfsetfillcolor{currentfill}%
\pgfsetlinewidth{0.602250pt}%
\definecolor{currentstroke}{rgb}{0.000000,0.000000,0.000000}%
\pgfsetstrokecolor{currentstroke}%
\pgfsetdash{}{0pt}%
\pgfsys@defobject{currentmarker}{\pgfqpoint{0.000000in}{-0.027778in}}{\pgfqpoint{0.000000in}{0.000000in}}{%
\pgfpathmoveto{\pgfqpoint{0.000000in}{0.000000in}}%
\pgfpathlineto{\pgfqpoint{0.000000in}{-0.027778in}}%
\pgfusepath{stroke,fill}%
}%
\begin{pgfscope}%
\pgfsys@transformshift{0.440128in}{1.079087in}%
\pgfsys@useobject{currentmarker}{}%
\end{pgfscope}%
\end{pgfscope}%
\begin{pgfscope}%
\pgfsetbuttcap%
\pgfsetroundjoin%
\definecolor{currentfill}{rgb}{0.000000,0.000000,0.000000}%
\pgfsetfillcolor{currentfill}%
\pgfsetlinewidth{0.602250pt}%
\definecolor{currentstroke}{rgb}{0.000000,0.000000,0.000000}%
\pgfsetstrokecolor{currentstroke}%
\pgfsetdash{}{0pt}%
\pgfsys@defobject{currentmarker}{\pgfqpoint{0.000000in}{-0.027778in}}{\pgfqpoint{0.000000in}{0.000000in}}{%
\pgfpathmoveto{\pgfqpoint{0.000000in}{0.000000in}}%
\pgfpathlineto{\pgfqpoint{0.000000in}{-0.027778in}}%
\pgfusepath{stroke,fill}%
}%
\begin{pgfscope}%
\pgfsys@transformshift{0.529872in}{1.079087in}%
\pgfsys@useobject{currentmarker}{}%
\end{pgfscope}%
\end{pgfscope}%
\begin{pgfscope}%
\pgfsetbuttcap%
\pgfsetroundjoin%
\definecolor{currentfill}{rgb}{0.000000,0.000000,0.000000}%
\pgfsetfillcolor{currentfill}%
\pgfsetlinewidth{0.602250pt}%
\definecolor{currentstroke}{rgb}{0.000000,0.000000,0.000000}%
\pgfsetstrokecolor{currentstroke}%
\pgfsetdash{}{0pt}%
\pgfsys@defobject{currentmarker}{\pgfqpoint{0.000000in}{-0.027778in}}{\pgfqpoint{0.000000in}{0.000000in}}{%
\pgfpathmoveto{\pgfqpoint{0.000000in}{0.000000in}}%
\pgfpathlineto{\pgfqpoint{0.000000in}{-0.027778in}}%
\pgfusepath{stroke,fill}%
}%
\begin{pgfscope}%
\pgfsys@transformshift{0.619615in}{1.079087in}%
\pgfsys@useobject{currentmarker}{}%
\end{pgfscope}%
\end{pgfscope}%
\begin{pgfscope}%
\pgfsetbuttcap%
\pgfsetroundjoin%
\definecolor{currentfill}{rgb}{0.000000,0.000000,0.000000}%
\pgfsetfillcolor{currentfill}%
\pgfsetlinewidth{0.602250pt}%
\definecolor{currentstroke}{rgb}{0.000000,0.000000,0.000000}%
\pgfsetstrokecolor{currentstroke}%
\pgfsetdash{}{0pt}%
\pgfsys@defobject{currentmarker}{\pgfqpoint{0.000000in}{-0.027778in}}{\pgfqpoint{0.000000in}{0.000000in}}{%
\pgfpathmoveto{\pgfqpoint{0.000000in}{0.000000in}}%
\pgfpathlineto{\pgfqpoint{0.000000in}{-0.027778in}}%
\pgfusepath{stroke,fill}%
}%
\begin{pgfscope}%
\pgfsys@transformshift{0.709359in}{1.079087in}%
\pgfsys@useobject{currentmarker}{}%
\end{pgfscope}%
\end{pgfscope}%
\begin{pgfscope}%
\pgfsetbuttcap%
\pgfsetroundjoin%
\definecolor{currentfill}{rgb}{0.000000,0.000000,0.000000}%
\pgfsetfillcolor{currentfill}%
\pgfsetlinewidth{0.602250pt}%
\definecolor{currentstroke}{rgb}{0.000000,0.000000,0.000000}%
\pgfsetstrokecolor{currentstroke}%
\pgfsetdash{}{0pt}%
\pgfsys@defobject{currentmarker}{\pgfqpoint{0.000000in}{-0.027778in}}{\pgfqpoint{0.000000in}{0.000000in}}{%
\pgfpathmoveto{\pgfqpoint{0.000000in}{0.000000in}}%
\pgfpathlineto{\pgfqpoint{0.000000in}{-0.027778in}}%
\pgfusepath{stroke,fill}%
}%
\begin{pgfscope}%
\pgfsys@transformshift{0.799102in}{1.079087in}%
\pgfsys@useobject{currentmarker}{}%
\end{pgfscope}%
\end{pgfscope}%
\begin{pgfscope}%
\pgfsetbuttcap%
\pgfsetroundjoin%
\definecolor{currentfill}{rgb}{0.000000,0.000000,0.000000}%
\pgfsetfillcolor{currentfill}%
\pgfsetlinewidth{0.602250pt}%
\definecolor{currentstroke}{rgb}{0.000000,0.000000,0.000000}%
\pgfsetstrokecolor{currentstroke}%
\pgfsetdash{}{0pt}%
\pgfsys@defobject{currentmarker}{\pgfqpoint{0.000000in}{-0.027778in}}{\pgfqpoint{0.000000in}{0.000000in}}{%
\pgfpathmoveto{\pgfqpoint{0.000000in}{0.000000in}}%
\pgfpathlineto{\pgfqpoint{0.000000in}{-0.027778in}}%
\pgfusepath{stroke,fill}%
}%
\begin{pgfscope}%
\pgfsys@transformshift{0.888846in}{1.079087in}%
\pgfsys@useobject{currentmarker}{}%
\end{pgfscope}%
\end{pgfscope}%
\begin{pgfscope}%
\pgfsetbuttcap%
\pgfsetroundjoin%
\definecolor{currentfill}{rgb}{0.000000,0.000000,0.000000}%
\pgfsetfillcolor{currentfill}%
\pgfsetlinewidth{0.602250pt}%
\definecolor{currentstroke}{rgb}{0.000000,0.000000,0.000000}%
\pgfsetstrokecolor{currentstroke}%
\pgfsetdash{}{0pt}%
\pgfsys@defobject{currentmarker}{\pgfqpoint{0.000000in}{-0.027778in}}{\pgfqpoint{0.000000in}{0.000000in}}{%
\pgfpathmoveto{\pgfqpoint{0.000000in}{0.000000in}}%
\pgfpathlineto{\pgfqpoint{0.000000in}{-0.027778in}}%
\pgfusepath{stroke,fill}%
}%
\begin{pgfscope}%
\pgfsys@transformshift{0.978589in}{1.079087in}%
\pgfsys@useobject{currentmarker}{}%
\end{pgfscope}%
\end{pgfscope}%
\begin{pgfscope}%
\pgfsetbuttcap%
\pgfsetroundjoin%
\definecolor{currentfill}{rgb}{0.000000,0.000000,0.000000}%
\pgfsetfillcolor{currentfill}%
\pgfsetlinewidth{0.602250pt}%
\definecolor{currentstroke}{rgb}{0.000000,0.000000,0.000000}%
\pgfsetstrokecolor{currentstroke}%
\pgfsetdash{}{0pt}%
\pgfsys@defobject{currentmarker}{\pgfqpoint{0.000000in}{-0.027778in}}{\pgfqpoint{0.000000in}{0.000000in}}{%
\pgfpathmoveto{\pgfqpoint{0.000000in}{0.000000in}}%
\pgfpathlineto{\pgfqpoint{0.000000in}{-0.027778in}}%
\pgfusepath{stroke,fill}%
}%
\begin{pgfscope}%
\pgfsys@transformshift{1.068333in}{1.079087in}%
\pgfsys@useobject{currentmarker}{}%
\end{pgfscope}%
\end{pgfscope}%
\begin{pgfscope}%
\pgfsetbuttcap%
\pgfsetroundjoin%
\definecolor{currentfill}{rgb}{0.000000,0.000000,0.000000}%
\pgfsetfillcolor{currentfill}%
\pgfsetlinewidth{0.602250pt}%
\definecolor{currentstroke}{rgb}{0.000000,0.000000,0.000000}%
\pgfsetstrokecolor{currentstroke}%
\pgfsetdash{}{0pt}%
\pgfsys@defobject{currentmarker}{\pgfqpoint{0.000000in}{-0.027778in}}{\pgfqpoint{0.000000in}{0.000000in}}{%
\pgfpathmoveto{\pgfqpoint{0.000000in}{0.000000in}}%
\pgfpathlineto{\pgfqpoint{0.000000in}{-0.027778in}}%
\pgfusepath{stroke,fill}%
}%
\begin{pgfscope}%
\pgfsys@transformshift{1.158077in}{1.079087in}%
\pgfsys@useobject{currentmarker}{}%
\end{pgfscope}%
\end{pgfscope}%
\begin{pgfscope}%
\pgfsetbuttcap%
\pgfsetroundjoin%
\definecolor{currentfill}{rgb}{0.000000,0.000000,0.000000}%
\pgfsetfillcolor{currentfill}%
\pgfsetlinewidth{0.602250pt}%
\definecolor{currentstroke}{rgb}{0.000000,0.000000,0.000000}%
\pgfsetstrokecolor{currentstroke}%
\pgfsetdash{}{0pt}%
\pgfsys@defobject{currentmarker}{\pgfqpoint{0.000000in}{-0.027778in}}{\pgfqpoint{0.000000in}{0.000000in}}{%
\pgfpathmoveto{\pgfqpoint{0.000000in}{0.000000in}}%
\pgfpathlineto{\pgfqpoint{0.000000in}{-0.027778in}}%
\pgfusepath{stroke,fill}%
}%
\begin{pgfscope}%
\pgfsys@transformshift{1.247820in}{1.079087in}%
\pgfsys@useobject{currentmarker}{}%
\end{pgfscope}%
\end{pgfscope}%
\begin{pgfscope}%
\pgfsetbuttcap%
\pgfsetroundjoin%
\definecolor{currentfill}{rgb}{0.000000,0.000000,0.000000}%
\pgfsetfillcolor{currentfill}%
\pgfsetlinewidth{0.602250pt}%
\definecolor{currentstroke}{rgb}{0.000000,0.000000,0.000000}%
\pgfsetstrokecolor{currentstroke}%
\pgfsetdash{}{0pt}%
\pgfsys@defobject{currentmarker}{\pgfqpoint{0.000000in}{-0.027778in}}{\pgfqpoint{0.000000in}{0.000000in}}{%
\pgfpathmoveto{\pgfqpoint{0.000000in}{0.000000in}}%
\pgfpathlineto{\pgfqpoint{0.000000in}{-0.027778in}}%
\pgfusepath{stroke,fill}%
}%
\begin{pgfscope}%
\pgfsys@transformshift{1.337564in}{1.079087in}%
\pgfsys@useobject{currentmarker}{}%
\end{pgfscope}%
\end{pgfscope}%
\begin{pgfscope}%
\pgfsetbuttcap%
\pgfsetroundjoin%
\definecolor{currentfill}{rgb}{0.000000,0.000000,0.000000}%
\pgfsetfillcolor{currentfill}%
\pgfsetlinewidth{0.602250pt}%
\definecolor{currentstroke}{rgb}{0.000000,0.000000,0.000000}%
\pgfsetstrokecolor{currentstroke}%
\pgfsetdash{}{0pt}%
\pgfsys@defobject{currentmarker}{\pgfqpoint{0.000000in}{-0.027778in}}{\pgfqpoint{0.000000in}{0.000000in}}{%
\pgfpathmoveto{\pgfqpoint{0.000000in}{0.000000in}}%
\pgfpathlineto{\pgfqpoint{0.000000in}{-0.027778in}}%
\pgfusepath{stroke,fill}%
}%
\begin{pgfscope}%
\pgfsys@transformshift{1.427307in}{1.079087in}%
\pgfsys@useobject{currentmarker}{}%
\end{pgfscope}%
\end{pgfscope}%
\begin{pgfscope}%
\pgfsetbuttcap%
\pgfsetroundjoin%
\definecolor{currentfill}{rgb}{0.000000,0.000000,0.000000}%
\pgfsetfillcolor{currentfill}%
\pgfsetlinewidth{0.602250pt}%
\definecolor{currentstroke}{rgb}{0.000000,0.000000,0.000000}%
\pgfsetstrokecolor{currentstroke}%
\pgfsetdash{}{0pt}%
\pgfsys@defobject{currentmarker}{\pgfqpoint{0.000000in}{-0.027778in}}{\pgfqpoint{0.000000in}{0.000000in}}{%
\pgfpathmoveto{\pgfqpoint{0.000000in}{0.000000in}}%
\pgfpathlineto{\pgfqpoint{0.000000in}{-0.027778in}}%
\pgfusepath{stroke,fill}%
}%
\begin{pgfscope}%
\pgfsys@transformshift{1.517051in}{1.079087in}%
\pgfsys@useobject{currentmarker}{}%
\end{pgfscope}%
\end{pgfscope}%
\begin{pgfscope}%
\pgfsetbuttcap%
\pgfsetroundjoin%
\definecolor{currentfill}{rgb}{0.000000,0.000000,0.000000}%
\pgfsetfillcolor{currentfill}%
\pgfsetlinewidth{0.602250pt}%
\definecolor{currentstroke}{rgb}{0.000000,0.000000,0.000000}%
\pgfsetstrokecolor{currentstroke}%
\pgfsetdash{}{0pt}%
\pgfsys@defobject{currentmarker}{\pgfqpoint{0.000000in}{-0.027778in}}{\pgfqpoint{0.000000in}{0.000000in}}{%
\pgfpathmoveto{\pgfqpoint{0.000000in}{0.000000in}}%
\pgfpathlineto{\pgfqpoint{0.000000in}{-0.027778in}}%
\pgfusepath{stroke,fill}%
}%
\begin{pgfscope}%
\pgfsys@transformshift{1.606794in}{1.079087in}%
\pgfsys@useobject{currentmarker}{}%
\end{pgfscope}%
\end{pgfscope}%
\begin{pgfscope}%
\pgfsetbuttcap%
\pgfsetroundjoin%
\definecolor{currentfill}{rgb}{0.000000,0.000000,0.000000}%
\pgfsetfillcolor{currentfill}%
\pgfsetlinewidth{0.602250pt}%
\definecolor{currentstroke}{rgb}{0.000000,0.000000,0.000000}%
\pgfsetstrokecolor{currentstroke}%
\pgfsetdash{}{0pt}%
\pgfsys@defobject{currentmarker}{\pgfqpoint{0.000000in}{-0.027778in}}{\pgfqpoint{0.000000in}{0.000000in}}{%
\pgfpathmoveto{\pgfqpoint{0.000000in}{0.000000in}}%
\pgfpathlineto{\pgfqpoint{0.000000in}{-0.027778in}}%
\pgfusepath{stroke,fill}%
}%
\begin{pgfscope}%
\pgfsys@transformshift{1.696538in}{1.079087in}%
\pgfsys@useobject{currentmarker}{}%
\end{pgfscope}%
\end{pgfscope}%
\begin{pgfscope}%
\pgfsetbuttcap%
\pgfsetroundjoin%
\definecolor{currentfill}{rgb}{0.000000,0.000000,0.000000}%
\pgfsetfillcolor{currentfill}%
\pgfsetlinewidth{0.602250pt}%
\definecolor{currentstroke}{rgb}{0.000000,0.000000,0.000000}%
\pgfsetstrokecolor{currentstroke}%
\pgfsetdash{}{0pt}%
\pgfsys@defobject{currentmarker}{\pgfqpoint{0.000000in}{-0.027778in}}{\pgfqpoint{0.000000in}{0.000000in}}{%
\pgfpathmoveto{\pgfqpoint{0.000000in}{0.000000in}}%
\pgfpathlineto{\pgfqpoint{0.000000in}{-0.027778in}}%
\pgfusepath{stroke,fill}%
}%
\begin{pgfscope}%
\pgfsys@transformshift{1.786282in}{1.079087in}%
\pgfsys@useobject{currentmarker}{}%
\end{pgfscope}%
\end{pgfscope}%
\begin{pgfscope}%
\pgfsetbuttcap%
\pgfsetroundjoin%
\definecolor{currentfill}{rgb}{0.000000,0.000000,0.000000}%
\pgfsetfillcolor{currentfill}%
\pgfsetlinewidth{0.602250pt}%
\definecolor{currentstroke}{rgb}{0.000000,0.000000,0.000000}%
\pgfsetstrokecolor{currentstroke}%
\pgfsetdash{}{0pt}%
\pgfsys@defobject{currentmarker}{\pgfqpoint{0.000000in}{-0.027778in}}{\pgfqpoint{0.000000in}{0.000000in}}{%
\pgfpathmoveto{\pgfqpoint{0.000000in}{0.000000in}}%
\pgfpathlineto{\pgfqpoint{0.000000in}{-0.027778in}}%
\pgfusepath{stroke,fill}%
}%
\begin{pgfscope}%
\pgfsys@transformshift{1.876025in}{1.079087in}%
\pgfsys@useobject{currentmarker}{}%
\end{pgfscope}%
\end{pgfscope}%
\begin{pgfscope}%
\pgfsetbuttcap%
\pgfsetroundjoin%
\definecolor{currentfill}{rgb}{0.000000,0.000000,0.000000}%
\pgfsetfillcolor{currentfill}%
\pgfsetlinewidth{0.602250pt}%
\definecolor{currentstroke}{rgb}{0.000000,0.000000,0.000000}%
\pgfsetstrokecolor{currentstroke}%
\pgfsetdash{}{0pt}%
\pgfsys@defobject{currentmarker}{\pgfqpoint{0.000000in}{-0.027778in}}{\pgfqpoint{0.000000in}{0.000000in}}{%
\pgfpathmoveto{\pgfqpoint{0.000000in}{0.000000in}}%
\pgfpathlineto{\pgfqpoint{0.000000in}{-0.027778in}}%
\pgfusepath{stroke,fill}%
}%
\begin{pgfscope}%
\pgfsys@transformshift{1.965769in}{1.079087in}%
\pgfsys@useobject{currentmarker}{}%
\end{pgfscope}%
\end{pgfscope}%
\begin{pgfscope}%
\pgfsetbuttcap%
\pgfsetroundjoin%
\definecolor{currentfill}{rgb}{0.000000,0.000000,0.000000}%
\pgfsetfillcolor{currentfill}%
\pgfsetlinewidth{0.803000pt}%
\definecolor{currentstroke}{rgb}{0.000000,0.000000,0.000000}%
\pgfsetstrokecolor{currentstroke}%
\pgfsetdash{}{0pt}%
\pgfsys@defobject{currentmarker}{\pgfqpoint{-0.048611in}{0.000000in}}{\pgfqpoint{0.000000in}{0.000000in}}{%
\pgfpathmoveto{\pgfqpoint{0.000000in}{0.000000in}}%
\pgfpathlineto{\pgfqpoint{-0.048611in}{0.000000in}}%
\pgfusepath{stroke,fill}%
}%
\begin{pgfscope}%
\pgfsys@transformshift{1.068333in}{0.181651in}%
\pgfsys@useobject{currentmarker}{}%
\end{pgfscope}%
\end{pgfscope}%
\begin{pgfscope}%
\pgftext[x=0.832629in,y=0.128890in,left,base]{\sffamily\fontsize{10.000000}{12.000000}\selectfont -1}%
\end{pgfscope}%
\begin{pgfscope}%
\pgfsetbuttcap%
\pgfsetroundjoin%
\definecolor{currentfill}{rgb}{0.000000,0.000000,0.000000}%
\pgfsetfillcolor{currentfill}%
\pgfsetlinewidth{0.803000pt}%
\definecolor{currentstroke}{rgb}{0.000000,0.000000,0.000000}%
\pgfsetstrokecolor{currentstroke}%
\pgfsetdash{}{0pt}%
\pgfsys@defobject{currentmarker}{\pgfqpoint{-0.048611in}{0.000000in}}{\pgfqpoint{0.000000in}{0.000000in}}{%
\pgfpathmoveto{\pgfqpoint{0.000000in}{0.000000in}}%
\pgfpathlineto{\pgfqpoint{-0.048611in}{0.000000in}}%
\pgfusepath{stroke,fill}%
}%
\begin{pgfscope}%
\pgfsys@transformshift{1.068333in}{0.630369in}%
\pgfsys@useobject{currentmarker}{}%
\end{pgfscope}%
\end{pgfscope}%
\begin{pgfscope}%
\pgftext[x=0.700115in,y=0.577608in,left,base]{\sffamily\fontsize{10.000000}{12.000000}\selectfont -0.5}%
\end{pgfscope}%
\begin{pgfscope}%
\pgfsetbuttcap%
\pgfsetroundjoin%
\definecolor{currentfill}{rgb}{0.000000,0.000000,0.000000}%
\pgfsetfillcolor{currentfill}%
\pgfsetlinewidth{0.803000pt}%
\definecolor{currentstroke}{rgb}{0.000000,0.000000,0.000000}%
\pgfsetstrokecolor{currentstroke}%
\pgfsetdash{}{0pt}%
\pgfsys@defobject{currentmarker}{\pgfqpoint{-0.048611in}{0.000000in}}{\pgfqpoint{0.000000in}{0.000000in}}{%
\pgfpathmoveto{\pgfqpoint{0.000000in}{0.000000in}}%
\pgfpathlineto{\pgfqpoint{-0.048611in}{0.000000in}}%
\pgfusepath{stroke,fill}%
}%
\begin{pgfscope}%
\pgfsys@transformshift{1.068333in}{1.079087in}%
\pgfsys@useobject{currentmarker}{}%
\end{pgfscope}%
\end{pgfscope}%
\begin{pgfscope}%
\pgfsetbuttcap%
\pgfsetroundjoin%
\definecolor{currentfill}{rgb}{0.000000,0.000000,0.000000}%
\pgfsetfillcolor{currentfill}%
\pgfsetlinewidth{0.803000pt}%
\definecolor{currentstroke}{rgb}{0.000000,0.000000,0.000000}%
\pgfsetstrokecolor{currentstroke}%
\pgfsetdash{}{0pt}%
\pgfsys@defobject{currentmarker}{\pgfqpoint{-0.048611in}{0.000000in}}{\pgfqpoint{0.000000in}{0.000000in}}{%
\pgfpathmoveto{\pgfqpoint{0.000000in}{0.000000in}}%
\pgfpathlineto{\pgfqpoint{-0.048611in}{0.000000in}}%
\pgfusepath{stroke,fill}%
}%
\begin{pgfscope}%
\pgfsys@transformshift{1.068333in}{1.527805in}%
\pgfsys@useobject{currentmarker}{}%
\end{pgfscope}%
\end{pgfscope}%
\begin{pgfscope}%
\pgftext[x=0.750231in,y=1.475043in,left,base]{\sffamily\fontsize{10.000000}{12.000000}\selectfont 0.5}%
\end{pgfscope}%
\begin{pgfscope}%
\pgfsetbuttcap%
\pgfsetroundjoin%
\definecolor{currentfill}{rgb}{0.000000,0.000000,0.000000}%
\pgfsetfillcolor{currentfill}%
\pgfsetlinewidth{0.803000pt}%
\definecolor{currentstroke}{rgb}{0.000000,0.000000,0.000000}%
\pgfsetstrokecolor{currentstroke}%
\pgfsetdash{}{0pt}%
\pgfsys@defobject{currentmarker}{\pgfqpoint{-0.048611in}{0.000000in}}{\pgfqpoint{0.000000in}{0.000000in}}{%
\pgfpathmoveto{\pgfqpoint{0.000000in}{0.000000in}}%
\pgfpathlineto{\pgfqpoint{-0.048611in}{0.000000in}}%
\pgfusepath{stroke,fill}%
}%
\begin{pgfscope}%
\pgfsys@transformshift{1.068333in}{1.976522in}%
\pgfsys@useobject{currentmarker}{}%
\end{pgfscope}%
\end{pgfscope}%
\begin{pgfscope}%
\pgftext[x=0.882746in,y=1.923761in,left,base]{\sffamily\fontsize{10.000000}{12.000000}\selectfont 1}%
\end{pgfscope}%
\begin{pgfscope}%
\pgfsetbuttcap%
\pgfsetroundjoin%
\definecolor{currentfill}{rgb}{0.000000,0.000000,0.000000}%
\pgfsetfillcolor{currentfill}%
\pgfsetlinewidth{0.602250pt}%
\definecolor{currentstroke}{rgb}{0.000000,0.000000,0.000000}%
\pgfsetstrokecolor{currentstroke}%
\pgfsetdash{}{0pt}%
\pgfsys@defobject{currentmarker}{\pgfqpoint{-0.027778in}{0.000000in}}{\pgfqpoint{0.000000in}{0.000000in}}{%
\pgfpathmoveto{\pgfqpoint{0.000000in}{0.000000in}}%
\pgfpathlineto{\pgfqpoint{-0.027778in}{0.000000in}}%
\pgfusepath{stroke,fill}%
}%
\begin{pgfscope}%
\pgfsys@transformshift{1.068333in}{0.271395in}%
\pgfsys@useobject{currentmarker}{}%
\end{pgfscope}%
\end{pgfscope}%
\begin{pgfscope}%
\pgfsetbuttcap%
\pgfsetroundjoin%
\definecolor{currentfill}{rgb}{0.000000,0.000000,0.000000}%
\pgfsetfillcolor{currentfill}%
\pgfsetlinewidth{0.602250pt}%
\definecolor{currentstroke}{rgb}{0.000000,0.000000,0.000000}%
\pgfsetstrokecolor{currentstroke}%
\pgfsetdash{}{0pt}%
\pgfsys@defobject{currentmarker}{\pgfqpoint{-0.027778in}{0.000000in}}{\pgfqpoint{0.000000in}{0.000000in}}{%
\pgfpathmoveto{\pgfqpoint{0.000000in}{0.000000in}}%
\pgfpathlineto{\pgfqpoint{-0.027778in}{0.000000in}}%
\pgfusepath{stroke,fill}%
}%
\begin{pgfscope}%
\pgfsys@transformshift{1.068333in}{0.361138in}%
\pgfsys@useobject{currentmarker}{}%
\end{pgfscope}%
\end{pgfscope}%
\begin{pgfscope}%
\pgfsetbuttcap%
\pgfsetroundjoin%
\definecolor{currentfill}{rgb}{0.000000,0.000000,0.000000}%
\pgfsetfillcolor{currentfill}%
\pgfsetlinewidth{0.602250pt}%
\definecolor{currentstroke}{rgb}{0.000000,0.000000,0.000000}%
\pgfsetstrokecolor{currentstroke}%
\pgfsetdash{}{0pt}%
\pgfsys@defobject{currentmarker}{\pgfqpoint{-0.027778in}{0.000000in}}{\pgfqpoint{0.000000in}{0.000000in}}{%
\pgfpathmoveto{\pgfqpoint{0.000000in}{0.000000in}}%
\pgfpathlineto{\pgfqpoint{-0.027778in}{0.000000in}}%
\pgfusepath{stroke,fill}%
}%
\begin{pgfscope}%
\pgfsys@transformshift{1.068333in}{0.450882in}%
\pgfsys@useobject{currentmarker}{}%
\end{pgfscope}%
\end{pgfscope}%
\begin{pgfscope}%
\pgfsetbuttcap%
\pgfsetroundjoin%
\definecolor{currentfill}{rgb}{0.000000,0.000000,0.000000}%
\pgfsetfillcolor{currentfill}%
\pgfsetlinewidth{0.602250pt}%
\definecolor{currentstroke}{rgb}{0.000000,0.000000,0.000000}%
\pgfsetstrokecolor{currentstroke}%
\pgfsetdash{}{0pt}%
\pgfsys@defobject{currentmarker}{\pgfqpoint{-0.027778in}{0.000000in}}{\pgfqpoint{0.000000in}{0.000000in}}{%
\pgfpathmoveto{\pgfqpoint{0.000000in}{0.000000in}}%
\pgfpathlineto{\pgfqpoint{-0.027778in}{0.000000in}}%
\pgfusepath{stroke,fill}%
}%
\begin{pgfscope}%
\pgfsys@transformshift{1.068333in}{0.540625in}%
\pgfsys@useobject{currentmarker}{}%
\end{pgfscope}%
\end{pgfscope}%
\begin{pgfscope}%
\pgfsetbuttcap%
\pgfsetroundjoin%
\definecolor{currentfill}{rgb}{0.000000,0.000000,0.000000}%
\pgfsetfillcolor{currentfill}%
\pgfsetlinewidth{0.602250pt}%
\definecolor{currentstroke}{rgb}{0.000000,0.000000,0.000000}%
\pgfsetstrokecolor{currentstroke}%
\pgfsetdash{}{0pt}%
\pgfsys@defobject{currentmarker}{\pgfqpoint{-0.027778in}{0.000000in}}{\pgfqpoint{0.000000in}{0.000000in}}{%
\pgfpathmoveto{\pgfqpoint{0.000000in}{0.000000in}}%
\pgfpathlineto{\pgfqpoint{-0.027778in}{0.000000in}}%
\pgfusepath{stroke,fill}%
}%
\begin{pgfscope}%
\pgfsys@transformshift{1.068333in}{0.630369in}%
\pgfsys@useobject{currentmarker}{}%
\end{pgfscope}%
\end{pgfscope}%
\begin{pgfscope}%
\pgfsetbuttcap%
\pgfsetroundjoin%
\definecolor{currentfill}{rgb}{0.000000,0.000000,0.000000}%
\pgfsetfillcolor{currentfill}%
\pgfsetlinewidth{0.602250pt}%
\definecolor{currentstroke}{rgb}{0.000000,0.000000,0.000000}%
\pgfsetstrokecolor{currentstroke}%
\pgfsetdash{}{0pt}%
\pgfsys@defobject{currentmarker}{\pgfqpoint{-0.027778in}{0.000000in}}{\pgfqpoint{0.000000in}{0.000000in}}{%
\pgfpathmoveto{\pgfqpoint{0.000000in}{0.000000in}}%
\pgfpathlineto{\pgfqpoint{-0.027778in}{0.000000in}}%
\pgfusepath{stroke,fill}%
}%
\begin{pgfscope}%
\pgfsys@transformshift{1.068333in}{0.720113in}%
\pgfsys@useobject{currentmarker}{}%
\end{pgfscope}%
\end{pgfscope}%
\begin{pgfscope}%
\pgfsetbuttcap%
\pgfsetroundjoin%
\definecolor{currentfill}{rgb}{0.000000,0.000000,0.000000}%
\pgfsetfillcolor{currentfill}%
\pgfsetlinewidth{0.602250pt}%
\definecolor{currentstroke}{rgb}{0.000000,0.000000,0.000000}%
\pgfsetstrokecolor{currentstroke}%
\pgfsetdash{}{0pt}%
\pgfsys@defobject{currentmarker}{\pgfqpoint{-0.027778in}{0.000000in}}{\pgfqpoint{0.000000in}{0.000000in}}{%
\pgfpathmoveto{\pgfqpoint{0.000000in}{0.000000in}}%
\pgfpathlineto{\pgfqpoint{-0.027778in}{0.000000in}}%
\pgfusepath{stroke,fill}%
}%
\begin{pgfscope}%
\pgfsys@transformshift{1.068333in}{0.809856in}%
\pgfsys@useobject{currentmarker}{}%
\end{pgfscope}%
\end{pgfscope}%
\begin{pgfscope}%
\pgfsetbuttcap%
\pgfsetroundjoin%
\definecolor{currentfill}{rgb}{0.000000,0.000000,0.000000}%
\pgfsetfillcolor{currentfill}%
\pgfsetlinewidth{0.602250pt}%
\definecolor{currentstroke}{rgb}{0.000000,0.000000,0.000000}%
\pgfsetstrokecolor{currentstroke}%
\pgfsetdash{}{0pt}%
\pgfsys@defobject{currentmarker}{\pgfqpoint{-0.027778in}{0.000000in}}{\pgfqpoint{0.000000in}{0.000000in}}{%
\pgfpathmoveto{\pgfqpoint{0.000000in}{0.000000in}}%
\pgfpathlineto{\pgfqpoint{-0.027778in}{0.000000in}}%
\pgfusepath{stroke,fill}%
}%
\begin{pgfscope}%
\pgfsys@transformshift{1.068333in}{0.899600in}%
\pgfsys@useobject{currentmarker}{}%
\end{pgfscope}%
\end{pgfscope}%
\begin{pgfscope}%
\pgfsetbuttcap%
\pgfsetroundjoin%
\definecolor{currentfill}{rgb}{0.000000,0.000000,0.000000}%
\pgfsetfillcolor{currentfill}%
\pgfsetlinewidth{0.602250pt}%
\definecolor{currentstroke}{rgb}{0.000000,0.000000,0.000000}%
\pgfsetstrokecolor{currentstroke}%
\pgfsetdash{}{0pt}%
\pgfsys@defobject{currentmarker}{\pgfqpoint{-0.027778in}{0.000000in}}{\pgfqpoint{0.000000in}{0.000000in}}{%
\pgfpathmoveto{\pgfqpoint{0.000000in}{0.000000in}}%
\pgfpathlineto{\pgfqpoint{-0.027778in}{0.000000in}}%
\pgfusepath{stroke,fill}%
}%
\begin{pgfscope}%
\pgfsys@transformshift{1.068333in}{0.989343in}%
\pgfsys@useobject{currentmarker}{}%
\end{pgfscope}%
\end{pgfscope}%
\begin{pgfscope}%
\pgfsetbuttcap%
\pgfsetroundjoin%
\definecolor{currentfill}{rgb}{0.000000,0.000000,0.000000}%
\pgfsetfillcolor{currentfill}%
\pgfsetlinewidth{0.602250pt}%
\definecolor{currentstroke}{rgb}{0.000000,0.000000,0.000000}%
\pgfsetstrokecolor{currentstroke}%
\pgfsetdash{}{0pt}%
\pgfsys@defobject{currentmarker}{\pgfqpoint{-0.027778in}{0.000000in}}{\pgfqpoint{0.000000in}{0.000000in}}{%
\pgfpathmoveto{\pgfqpoint{0.000000in}{0.000000in}}%
\pgfpathlineto{\pgfqpoint{-0.027778in}{0.000000in}}%
\pgfusepath{stroke,fill}%
}%
\begin{pgfscope}%
\pgfsys@transformshift{1.068333in}{1.079087in}%
\pgfsys@useobject{currentmarker}{}%
\end{pgfscope}%
\end{pgfscope}%
\begin{pgfscope}%
\pgfsetbuttcap%
\pgfsetroundjoin%
\definecolor{currentfill}{rgb}{0.000000,0.000000,0.000000}%
\pgfsetfillcolor{currentfill}%
\pgfsetlinewidth{0.602250pt}%
\definecolor{currentstroke}{rgb}{0.000000,0.000000,0.000000}%
\pgfsetstrokecolor{currentstroke}%
\pgfsetdash{}{0pt}%
\pgfsys@defobject{currentmarker}{\pgfqpoint{-0.027778in}{0.000000in}}{\pgfqpoint{0.000000in}{0.000000in}}{%
\pgfpathmoveto{\pgfqpoint{0.000000in}{0.000000in}}%
\pgfpathlineto{\pgfqpoint{-0.027778in}{0.000000in}}%
\pgfusepath{stroke,fill}%
}%
\begin{pgfscope}%
\pgfsys@transformshift{1.068333in}{1.168830in}%
\pgfsys@useobject{currentmarker}{}%
\end{pgfscope}%
\end{pgfscope}%
\begin{pgfscope}%
\pgfsetbuttcap%
\pgfsetroundjoin%
\definecolor{currentfill}{rgb}{0.000000,0.000000,0.000000}%
\pgfsetfillcolor{currentfill}%
\pgfsetlinewidth{0.602250pt}%
\definecolor{currentstroke}{rgb}{0.000000,0.000000,0.000000}%
\pgfsetstrokecolor{currentstroke}%
\pgfsetdash{}{0pt}%
\pgfsys@defobject{currentmarker}{\pgfqpoint{-0.027778in}{0.000000in}}{\pgfqpoint{0.000000in}{0.000000in}}{%
\pgfpathmoveto{\pgfqpoint{0.000000in}{0.000000in}}%
\pgfpathlineto{\pgfqpoint{-0.027778in}{0.000000in}}%
\pgfusepath{stroke,fill}%
}%
\begin{pgfscope}%
\pgfsys@transformshift{1.068333in}{1.258574in}%
\pgfsys@useobject{currentmarker}{}%
\end{pgfscope}%
\end{pgfscope}%
\begin{pgfscope}%
\pgfsetbuttcap%
\pgfsetroundjoin%
\definecolor{currentfill}{rgb}{0.000000,0.000000,0.000000}%
\pgfsetfillcolor{currentfill}%
\pgfsetlinewidth{0.602250pt}%
\definecolor{currentstroke}{rgb}{0.000000,0.000000,0.000000}%
\pgfsetstrokecolor{currentstroke}%
\pgfsetdash{}{0pt}%
\pgfsys@defobject{currentmarker}{\pgfqpoint{-0.027778in}{0.000000in}}{\pgfqpoint{0.000000in}{0.000000in}}{%
\pgfpathmoveto{\pgfqpoint{0.000000in}{0.000000in}}%
\pgfpathlineto{\pgfqpoint{-0.027778in}{0.000000in}}%
\pgfusepath{stroke,fill}%
}%
\begin{pgfscope}%
\pgfsys@transformshift{1.068333in}{1.348318in}%
\pgfsys@useobject{currentmarker}{}%
\end{pgfscope}%
\end{pgfscope}%
\begin{pgfscope}%
\pgfsetbuttcap%
\pgfsetroundjoin%
\definecolor{currentfill}{rgb}{0.000000,0.000000,0.000000}%
\pgfsetfillcolor{currentfill}%
\pgfsetlinewidth{0.602250pt}%
\definecolor{currentstroke}{rgb}{0.000000,0.000000,0.000000}%
\pgfsetstrokecolor{currentstroke}%
\pgfsetdash{}{0pt}%
\pgfsys@defobject{currentmarker}{\pgfqpoint{-0.027778in}{0.000000in}}{\pgfqpoint{0.000000in}{0.000000in}}{%
\pgfpathmoveto{\pgfqpoint{0.000000in}{0.000000in}}%
\pgfpathlineto{\pgfqpoint{-0.027778in}{0.000000in}}%
\pgfusepath{stroke,fill}%
}%
\begin{pgfscope}%
\pgfsys@transformshift{1.068333in}{1.438061in}%
\pgfsys@useobject{currentmarker}{}%
\end{pgfscope}%
\end{pgfscope}%
\begin{pgfscope}%
\pgfsetbuttcap%
\pgfsetroundjoin%
\definecolor{currentfill}{rgb}{0.000000,0.000000,0.000000}%
\pgfsetfillcolor{currentfill}%
\pgfsetlinewidth{0.602250pt}%
\definecolor{currentstroke}{rgb}{0.000000,0.000000,0.000000}%
\pgfsetstrokecolor{currentstroke}%
\pgfsetdash{}{0pt}%
\pgfsys@defobject{currentmarker}{\pgfqpoint{-0.027778in}{0.000000in}}{\pgfqpoint{0.000000in}{0.000000in}}{%
\pgfpathmoveto{\pgfqpoint{0.000000in}{0.000000in}}%
\pgfpathlineto{\pgfqpoint{-0.027778in}{0.000000in}}%
\pgfusepath{stroke,fill}%
}%
\begin{pgfscope}%
\pgfsys@transformshift{1.068333in}{1.527805in}%
\pgfsys@useobject{currentmarker}{}%
\end{pgfscope}%
\end{pgfscope}%
\begin{pgfscope}%
\pgfsetbuttcap%
\pgfsetroundjoin%
\definecolor{currentfill}{rgb}{0.000000,0.000000,0.000000}%
\pgfsetfillcolor{currentfill}%
\pgfsetlinewidth{0.602250pt}%
\definecolor{currentstroke}{rgb}{0.000000,0.000000,0.000000}%
\pgfsetstrokecolor{currentstroke}%
\pgfsetdash{}{0pt}%
\pgfsys@defobject{currentmarker}{\pgfqpoint{-0.027778in}{0.000000in}}{\pgfqpoint{0.000000in}{0.000000in}}{%
\pgfpathmoveto{\pgfqpoint{0.000000in}{0.000000in}}%
\pgfpathlineto{\pgfqpoint{-0.027778in}{0.000000in}}%
\pgfusepath{stroke,fill}%
}%
\begin{pgfscope}%
\pgfsys@transformshift{1.068333in}{1.617548in}%
\pgfsys@useobject{currentmarker}{}%
\end{pgfscope}%
\end{pgfscope}%
\begin{pgfscope}%
\pgfsetbuttcap%
\pgfsetroundjoin%
\definecolor{currentfill}{rgb}{0.000000,0.000000,0.000000}%
\pgfsetfillcolor{currentfill}%
\pgfsetlinewidth{0.602250pt}%
\definecolor{currentstroke}{rgb}{0.000000,0.000000,0.000000}%
\pgfsetstrokecolor{currentstroke}%
\pgfsetdash{}{0pt}%
\pgfsys@defobject{currentmarker}{\pgfqpoint{-0.027778in}{0.000000in}}{\pgfqpoint{0.000000in}{0.000000in}}{%
\pgfpathmoveto{\pgfqpoint{0.000000in}{0.000000in}}%
\pgfpathlineto{\pgfqpoint{-0.027778in}{0.000000in}}%
\pgfusepath{stroke,fill}%
}%
\begin{pgfscope}%
\pgfsys@transformshift{1.068333in}{1.707292in}%
\pgfsys@useobject{currentmarker}{}%
\end{pgfscope}%
\end{pgfscope}%
\begin{pgfscope}%
\pgfsetbuttcap%
\pgfsetroundjoin%
\definecolor{currentfill}{rgb}{0.000000,0.000000,0.000000}%
\pgfsetfillcolor{currentfill}%
\pgfsetlinewidth{0.602250pt}%
\definecolor{currentstroke}{rgb}{0.000000,0.000000,0.000000}%
\pgfsetstrokecolor{currentstroke}%
\pgfsetdash{}{0pt}%
\pgfsys@defobject{currentmarker}{\pgfqpoint{-0.027778in}{0.000000in}}{\pgfqpoint{0.000000in}{0.000000in}}{%
\pgfpathmoveto{\pgfqpoint{0.000000in}{0.000000in}}%
\pgfpathlineto{\pgfqpoint{-0.027778in}{0.000000in}}%
\pgfusepath{stroke,fill}%
}%
\begin{pgfscope}%
\pgfsys@transformshift{1.068333in}{1.797035in}%
\pgfsys@useobject{currentmarker}{}%
\end{pgfscope}%
\end{pgfscope}%
\begin{pgfscope}%
\pgfsetbuttcap%
\pgfsetroundjoin%
\definecolor{currentfill}{rgb}{0.000000,0.000000,0.000000}%
\pgfsetfillcolor{currentfill}%
\pgfsetlinewidth{0.602250pt}%
\definecolor{currentstroke}{rgb}{0.000000,0.000000,0.000000}%
\pgfsetstrokecolor{currentstroke}%
\pgfsetdash{}{0pt}%
\pgfsys@defobject{currentmarker}{\pgfqpoint{-0.027778in}{0.000000in}}{\pgfqpoint{0.000000in}{0.000000in}}{%
\pgfpathmoveto{\pgfqpoint{0.000000in}{0.000000in}}%
\pgfpathlineto{\pgfqpoint{-0.027778in}{0.000000in}}%
\pgfusepath{stroke,fill}%
}%
\begin{pgfscope}%
\pgfsys@transformshift{1.068333in}{1.886779in}%
\pgfsys@useobject{currentmarker}{}%
\end{pgfscope}%
\end{pgfscope}%
\begin{pgfscope}%
\pgfsetbuttcap%
\pgfsetroundjoin%
\definecolor{currentfill}{rgb}{0.000000,0.000000,0.000000}%
\pgfsetfillcolor{currentfill}%
\pgfsetlinewidth{0.602250pt}%
\definecolor{currentstroke}{rgb}{0.000000,0.000000,0.000000}%
\pgfsetstrokecolor{currentstroke}%
\pgfsetdash{}{0pt}%
\pgfsys@defobject{currentmarker}{\pgfqpoint{-0.027778in}{0.000000in}}{\pgfqpoint{0.000000in}{0.000000in}}{%
\pgfpathmoveto{\pgfqpoint{0.000000in}{0.000000in}}%
\pgfpathlineto{\pgfqpoint{-0.027778in}{0.000000in}}%
\pgfusepath{stroke,fill}%
}%
\begin{pgfscope}%
\pgfsys@transformshift{1.068333in}{1.976522in}%
\pgfsys@useobject{currentmarker}{}%
\end{pgfscope}%
\end{pgfscope}%
\begin{pgfscope}%
\pgfpathrectangle{\pgfqpoint{0.135000in}{0.145754in}}{\pgfqpoint{1.866666in}{1.866666in}} %
\pgfusepath{clip}%
\pgfsetrectcap%
\pgfsetroundjoin%
\pgfsetlinewidth{1.505625pt}%
\definecolor{currentstroke}{rgb}{0.000000,0.000000,1.000000}%
\pgfsetstrokecolor{currentstroke}%
\pgfsetdash{}{0pt}%
\pgfpathmoveto{\pgfqpoint{1.965769in}{1.976522in}}%
\pgfpathlineto{\pgfqpoint{1.965769in}{0.181651in}}%
\pgfpathlineto{\pgfqpoint{0.170897in}{0.181651in}}%
\pgfpathlineto{\pgfqpoint{0.170897in}{1.976522in}}%
\pgfpathlineto{\pgfqpoint{1.965769in}{1.976522in}}%
\pgfusepath{stroke}%
\end{pgfscope}%
\begin{pgfscope}%
\pgfsetrectcap%
\pgfsetmiterjoin%
\pgfsetlinewidth{0.803000pt}%
\definecolor{currentstroke}{rgb}{0.000000,0.000000,0.000000}%
\pgfsetstrokecolor{currentstroke}%
\pgfsetdash{}{0pt}%
\pgfpathmoveto{\pgfqpoint{1.068333in}{0.145754in}}%
\pgfpathlineto{\pgfqpoint{1.068333in}{2.012420in}}%
\pgfusepath{stroke}%
\end{pgfscope}%
\begin{pgfscope}%
\pgfsetrectcap%
\pgfsetmiterjoin%
\pgfsetlinewidth{0.803000pt}%
\definecolor{currentstroke}{rgb}{0.000000,0.000000,0.000000}%
\pgfsetstrokecolor{currentstroke}%
\pgfsetdash{}{0pt}%
\pgfpathmoveto{\pgfqpoint{0.135000in}{1.079087in}}%
\pgfpathlineto{\pgfqpoint{2.001666in}{1.079087in}}%
\pgfusepath{stroke}%
\end{pgfscope}%
\end{pgfpicture}%
\makeatother%
\endgroup%

            \end{center}
        \item No es una m\'etrica, por ejemplo, $d\left( (1,0), (1,1) \right) = 0$, pero $(1,0) \neq (1,1)$.
        \item En general no  es una m\'etrica, si tomamos la distancia euclidiana, $d_{\text{eq}} \left( (1, 0) , (0.5, 0) \right) = 0.5 \implies
            d\left( (1,0), (0.5, 0) \right) = \lfloor0.5 \rfloor = 0$.
        \item En general no es una m\'etrica, por ejemplo si $A$ tiene un vep $v$ de vap $\lambda < 0$, entonces $d(v, 0) = \sqrt{(v - 0) A (v-0)^t}$ no existe,
            ya que el contenido de dentro de la raiz es negativo.
        \item Vemos que cumple que
            \begin{enumerate}[i)]
                \item $d(x, y) \geq 0$, sí, ya que $A$ es definida positiva y por lo tanto, $(x-y)A(x-y)^t \geq 0 \implies \sqrt{(x-y)A(x-y)^t} \geq 0$.
                \item $d(x, y) = 0 \iff \sqrt{(x-y)A(x-y)^t} = 0 \iff (x-y)A(x-y)^t = 0 \stackrel{A \text{ def. pos.}}{\iff} x = y$.
                \item $d(x, y) = \sqrt{(x-y)A(x-y)^t} = \sqrt{(-1)(y-x)A(-1)(y-x)^t} =$ \\ $\sqrt{(y-x)A(y-x)^t} = d(y, x)$
                \item Se tiene que
                    \begin{align*}
                        d^2(x,y) &= (x-y)A(x-y)^t = (x-z+z-y)A(x-z+z-y)^t \\
                        &= (x-z)A(x-z)^t + (x-z)A(z-y)^t + (z-y)A(x-z)^t\, + \\
                        &\qquad\qquad + (z-y)A(z-y)^t \\
                        &\leq (x-z)A(x-z)^t + 2 \abs{(x-z)A(z-y)^t} + (z-y)A(z-y)^t \\
                        &\leq (x-z)A(x-z)^t + 2 \sqrt{(x-z)A(x-z)^t(z-y)A(z-y)^t} \,\,+ \\
                        &\qquad\qquad + (z-y)A(z-y)^t \\
                        &= \left( (x-z)A(x-z)^t + (z-y)A(z-y)^t \right)^2 \\
                        &= \left( d(x,z) + d(z,y) \right)^2.
                    \end{align*}
                    Por lo tanto, $d(x,y) \leq d(x,z) + d(z,y)$.
            \end{enumerate}
            Ponemos un ejemplo, tomando
            \[
                A =
                \begin{pmatrix}
                    1 & 2 \\
                    2 & 5
                \end{pmatrix}
            \]
            \begin{center}
                %% Creator: Matplotlib, PGF backend
%%
%% To include the figure in your LaTeX document, write
%%   \input{<filename>.pgf}
%%
%% Make sure the required packages are loaded in your preamble
%%   \usepackage{pgf}
%%
%% Figures using additional raster images can only be included by \input if
%% they are in the same directory as the main LaTeX file. For loading figures
%% from other directories you can use the `import` package
%%   \usepackage{import}
%% and then include the figures with
%%   \import{<path to file>}{<filename>.pgf}
%%
%% Matplotlib used the following preamble
%%   \usepackage{fontspec}
%%   \setmainfont{DejaVu Serif}
%%   \setsansfont{DejaVu Sans}
%%   \setmonofont{DejaVu Sans Mono}
%%
\begingroup%
\makeatletter%
\begin{pgfpicture}%
\pgfpathrectangle{\pgfpointorigin}{\pgfqpoint{4.900000in}{1.830305in}}%
\pgfusepath{use as bounding box, clip}%
\begin{pgfscope}%
\pgfsetbuttcap%
\pgfsetmiterjoin%
\definecolor{currentfill}{rgb}{1.000000,1.000000,1.000000}%
\pgfsetfillcolor{currentfill}%
\pgfsetlinewidth{0.000000pt}%
\definecolor{currentstroke}{rgb}{1.000000,1.000000,1.000000}%
\pgfsetstrokecolor{currentstroke}%
\pgfsetdash{}{0pt}%
\pgfpathmoveto{\pgfqpoint{0.000000in}{-0.000000in}}%
\pgfpathlineto{\pgfqpoint{4.900000in}{-0.000000in}}%
\pgfpathlineto{\pgfqpoint{4.900000in}{1.830305in}}%
\pgfpathlineto{\pgfqpoint{0.000000in}{1.830305in}}%
\pgfpathclose%
\pgfusepath{fill}%
\end{pgfscope}%
\begin{pgfscope}%
\pgfsetbuttcap%
\pgfsetmiterjoin%
\definecolor{currentfill}{rgb}{1.000000,1.000000,1.000000}%
\pgfsetfillcolor{currentfill}%
\pgfsetlinewidth{0.000000pt}%
\definecolor{currentstroke}{rgb}{0.000000,0.000000,0.000000}%
\pgfsetstrokecolor{currentstroke}%
\pgfsetstrokeopacity{0.000000}%
\pgfsetdash{}{0pt}%
\pgfpathmoveto{\pgfqpoint{0.135000in}{0.151972in}}%
\pgfpathlineto{\pgfqpoint{4.765000in}{0.151972in}}%
\pgfpathlineto{\pgfqpoint{4.765000in}{1.695305in}}%
\pgfpathlineto{\pgfqpoint{0.135000in}{1.695305in}}%
\pgfpathclose%
\pgfusepath{fill}%
\end{pgfscope}%
\begin{pgfscope}%
\pgfpathrectangle{\pgfqpoint{0.135000in}{0.151972in}}{\pgfqpoint{4.630000in}{1.543333in}} %
\pgfusepath{clip}%
\pgfsetbuttcap%
\pgfsetroundjoin%
\definecolor{currentfill}{rgb}{0.000000,0.000000,1.000000}%
\pgfsetfillcolor{currentfill}%
\pgfsetfillopacity{0.300000}%
\pgfsetlinewidth{0.000000pt}%
\definecolor{currentstroke}{rgb}{0.000000,0.000000,0.000000}%
\pgfsetstrokecolor{currentstroke}%
\pgfsetdash{}{0pt}%
\pgfpathmoveto{\pgfqpoint{3.809480in}{0.189659in}}%
\pgfpathlineto{\pgfqpoint{3.839359in}{0.186570in}}%
\pgfpathlineto{\pgfqpoint{3.869237in}{0.184112in}}%
\pgfpathlineto{\pgfqpoint{3.899116in}{0.182418in}}%
\pgfpathlineto{\pgfqpoint{3.928995in}{0.181668in}}%
\pgfpathlineto{\pgfqpoint{3.958873in}{0.182113in}}%
\pgfpathlineto{\pgfqpoint{3.988752in}{0.184110in}}%
\pgfpathlineto{\pgfqpoint{4.018631in}{0.188182in}}%
\pgfpathlineto{\pgfqpoint{4.034320in}{0.191611in}}%
\pgfpathlineto{\pgfqpoint{4.048509in}{0.195478in}}%
\pgfpathlineto{\pgfqpoint{4.064288in}{0.201570in}}%
\pgfpathlineto{\pgfqpoint{4.078388in}{0.208846in}}%
\pgfpathlineto{\pgfqpoint{4.082426in}{0.211530in}}%
\pgfpathlineto{\pgfqpoint{4.093739in}{0.221489in}}%
\pgfpathlineto{\pgfqpoint{4.101361in}{0.231449in}}%
\pgfpathlineto{\pgfqpoint{4.106103in}{0.241409in}}%
\pgfpathlineto{\pgfqpoint{4.108267in}{0.250123in}}%
\pgfpathlineto{\pgfqpoint{4.108533in}{0.251368in}}%
\pgfpathlineto{\pgfqpoint{4.109092in}{0.261328in}}%
\pgfpathlineto{\pgfqpoint{4.108267in}{0.270872in}}%
\pgfpathlineto{\pgfqpoint{4.108230in}{0.271287in}}%
\pgfpathlineto{\pgfqpoint{4.106039in}{0.281247in}}%
\pgfpathlineto{\pgfqpoint{4.102777in}{0.291206in}}%
\pgfpathlineto{\pgfqpoint{4.098596in}{0.301166in}}%
\pgfpathlineto{\pgfqpoint{4.093620in}{0.311126in}}%
\pgfpathlineto{\pgfqpoint{4.087943in}{0.321085in}}%
\pgfpathlineto{\pgfqpoint{4.081644in}{0.331045in}}%
\pgfpathlineto{\pgfqpoint{4.078388in}{0.335681in}}%
\pgfpathlineto{\pgfqpoint{4.074638in}{0.341004in}}%
\pgfpathlineto{\pgfqpoint{4.067009in}{0.350964in}}%
\pgfpathlineto{\pgfqpoint{4.058939in}{0.360923in}}%
\pgfpathlineto{\pgfqpoint{4.050462in}{0.370883in}}%
\pgfpathlineto{\pgfqpoint{4.048509in}{0.373028in}}%
\pgfpathlineto{\pgfqpoint{4.041388in}{0.380842in}}%
\pgfpathlineto{\pgfqpoint{4.031926in}{0.390802in}}%
\pgfpathlineto{\pgfqpoint{4.022156in}{0.400762in}}%
\pgfpathlineto{\pgfqpoint{4.018631in}{0.404191in}}%
\pgfpathlineto{\pgfqpoint{4.011918in}{0.410721in}}%
\pgfpathlineto{\pgfqpoint{4.001332in}{0.420681in}}%
\pgfpathlineto{\pgfqpoint{3.990502in}{0.430640in}}%
\pgfpathlineto{\pgfqpoint{3.988752in}{0.432184in}}%
\pgfpathlineto{\pgfqpoint{3.979225in}{0.440600in}}%
\pgfpathlineto{\pgfqpoint{3.967706in}{0.450559in}}%
\pgfpathlineto{\pgfqpoint{3.958873in}{0.458022in}}%
\pgfpathlineto{\pgfqpoint{3.955923in}{0.460519in}}%
\pgfpathlineto{\pgfqpoint{3.943779in}{0.470478in}}%
\pgfpathlineto{\pgfqpoint{3.931460in}{0.480438in}}%
\pgfpathlineto{\pgfqpoint{3.928995in}{0.482373in}}%
\pgfpathlineto{\pgfqpoint{3.918792in}{0.490398in}}%
\pgfpathlineto{\pgfqpoint{3.905929in}{0.500357in}}%
\pgfpathlineto{\pgfqpoint{3.899116in}{0.505524in}}%
\pgfpathlineto{\pgfqpoint{3.892812in}{0.510317in}}%
\pgfpathlineto{\pgfqpoint{3.879448in}{0.520276in}}%
\pgfpathlineto{\pgfqpoint{3.869237in}{0.527775in}}%
\pgfpathlineto{\pgfqpoint{3.865895in}{0.530236in}}%
\pgfpathlineto{\pgfqpoint{3.852070in}{0.540195in}}%
\pgfpathlineto{\pgfqpoint{3.839359in}{0.549252in}}%
\pgfpathlineto{\pgfqpoint{3.838095in}{0.550155in}}%
\pgfpathlineto{\pgfqpoint{3.823841in}{0.560115in}}%
\pgfpathlineto{\pgfqpoint{3.809480in}{0.570061in}}%
\pgfpathlineto{\pgfqpoint{3.809461in}{0.570074in}}%
\pgfpathlineto{\pgfqpoint{3.794806in}{0.580034in}}%
\pgfpathlineto{\pgfqpoint{3.780030in}{0.589993in}}%
\pgfpathlineto{\pgfqpoint{3.779601in}{0.590276in}}%
\pgfpathlineto{\pgfqpoint{3.765002in}{0.599953in}}%
\pgfpathlineto{\pgfqpoint{3.749851in}{0.609912in}}%
\pgfpathlineto{\pgfqpoint{3.749723in}{0.609995in}}%
\pgfpathlineto{\pgfqpoint{3.734463in}{0.619872in}}%
\pgfpathlineto{\pgfqpoint{3.719844in}{0.629256in}}%
\pgfpathlineto{\pgfqpoint{3.718950in}{0.629831in}}%
\pgfpathlineto{\pgfqpoint{3.703219in}{0.639791in}}%
\pgfpathlineto{\pgfqpoint{3.689965in}{0.648108in}}%
\pgfpathlineto{\pgfqpoint{3.687356in}{0.649751in}}%
\pgfpathlineto{\pgfqpoint{3.671298in}{0.659710in}}%
\pgfpathlineto{\pgfqpoint{3.660086in}{0.666596in}}%
\pgfpathlineto{\pgfqpoint{3.655096in}{0.669670in}}%
\pgfpathlineto{\pgfqpoint{3.638722in}{0.679629in}}%
\pgfpathlineto{\pgfqpoint{3.630208in}{0.684752in}}%
\pgfpathlineto{\pgfqpoint{3.622193in}{0.689589in}}%
\pgfpathlineto{\pgfqpoint{3.605511in}{0.699548in}}%
\pgfpathlineto{\pgfqpoint{3.600329in}{0.702606in}}%
\pgfpathlineto{\pgfqpoint{3.588666in}{0.709508in}}%
\pgfpathlineto{\pgfqpoint{3.571684in}{0.719467in}}%
\pgfpathlineto{\pgfqpoint{3.570450in}{0.720182in}}%
\pgfpathlineto{\pgfqpoint{3.554532in}{0.729427in}}%
\pgfpathlineto{\pgfqpoint{3.540572in}{0.737466in}}%
\pgfpathlineto{\pgfqpoint{3.537246in}{0.739387in}}%
\pgfpathlineto{\pgfqpoint{3.519804in}{0.749346in}}%
\pgfpathlineto{\pgfqpoint{3.510693in}{0.754501in}}%
\pgfpathlineto{\pgfqpoint{3.502220in}{0.759306in}}%
\pgfpathlineto{\pgfqpoint{3.484494in}{0.769265in}}%
\pgfpathlineto{\pgfqpoint{3.480814in}{0.771312in}}%
\pgfpathlineto{\pgfqpoint{3.466619in}{0.779225in}}%
\pgfpathlineto{\pgfqpoint{3.450936in}{0.787896in}}%
\pgfpathlineto{\pgfqpoint{3.448610in}{0.789184in}}%
\pgfpathlineto{\pgfqpoint{3.430454in}{0.799144in}}%
\pgfpathlineto{\pgfqpoint{3.421057in}{0.804255in}}%
\pgfpathlineto{\pgfqpoint{3.412161in}{0.809103in}}%
\pgfpathlineto{\pgfqpoint{3.393731in}{0.819063in}}%
\pgfpathlineto{\pgfqpoint{3.391178in}{0.820430in}}%
\pgfpathlineto{\pgfqpoint{3.375159in}{0.829023in}}%
\pgfpathlineto{\pgfqpoint{3.361300in}{0.836399in}}%
\pgfpathlineto{\pgfqpoint{3.356453in}{0.838982in}}%
\pgfpathlineto{\pgfqpoint{3.337610in}{0.848942in}}%
\pgfpathlineto{\pgfqpoint{3.331421in}{0.852187in}}%
\pgfpathlineto{\pgfqpoint{3.318631in}{0.858901in}}%
\pgfpathlineto{\pgfqpoint{3.301542in}{0.867805in}}%
\pgfpathlineto{\pgfqpoint{3.299517in}{0.868861in}}%
\pgfpathlineto{\pgfqpoint{3.280268in}{0.878820in}}%
\pgfpathlineto{\pgfqpoint{3.271664in}{0.883240in}}%
\pgfpathlineto{\pgfqpoint{3.260885in}{0.888780in}}%
\pgfpathlineto{\pgfqpoint{3.241785in}{0.898526in}}%
\pgfpathlineto{\pgfqpoint{3.241367in}{0.898740in}}%
\pgfpathlineto{\pgfqpoint{3.221715in}{0.908699in}}%
\pgfpathlineto{\pgfqpoint{3.211906in}{0.913636in}}%
\pgfpathlineto{\pgfqpoint{3.201930in}{0.918659in}}%
\pgfpathlineto{\pgfqpoint{3.182028in}{0.928610in}}%
\pgfpathlineto{\pgfqpoint{3.182011in}{0.928618in}}%
\pgfpathlineto{\pgfqpoint{3.161958in}{0.938578in}}%
\pgfpathlineto{\pgfqpoint{3.152149in}{0.943418in}}%
\pgfpathlineto{\pgfqpoint{3.141771in}{0.948537in}}%
\pgfpathlineto{\pgfqpoint{3.122270in}{0.958096in}}%
\pgfpathlineto{\pgfqpoint{3.121451in}{0.958497in}}%
\pgfpathlineto{\pgfqpoint{3.100996in}{0.968456in}}%
\pgfpathlineto{\pgfqpoint{3.092392in}{0.972621in}}%
\pgfpathlineto{\pgfqpoint{3.080406in}{0.978416in}}%
\pgfpathlineto{\pgfqpoint{3.062513in}{0.987017in}}%
\pgfpathlineto{\pgfqpoint{3.059683in}{0.988376in}}%
\pgfpathlineto{\pgfqpoint{3.038823in}{0.998335in}}%
\pgfpathlineto{\pgfqpoint{3.032634in}{1.001274in}}%
\pgfpathlineto{\pgfqpoint{3.017827in}{1.008295in}}%
\pgfpathlineto{\pgfqpoint{3.002756in}{1.015402in}}%
\pgfpathlineto{\pgfqpoint{2.996697in}{1.018254in}}%
\pgfpathlineto{\pgfqpoint{2.975429in}{1.028214in}}%
\pgfpathlineto{\pgfqpoint{2.972877in}{1.029403in}}%
\pgfpathlineto{\pgfqpoint{2.954021in}{1.038173in}}%
\pgfpathlineto{\pgfqpoint{2.942998in}{1.043275in}}%
\pgfpathlineto{\pgfqpoint{2.932475in}{1.048133in}}%
\pgfpathlineto{\pgfqpoint{2.913120in}{1.057027in}}%
\pgfpathlineto{\pgfqpoint{2.910794in}{1.058092in}}%
\pgfpathlineto{\pgfqpoint{2.888967in}{1.068052in}}%
\pgfpathlineto{\pgfqpoint{2.883241in}{1.070654in}}%
\pgfpathlineto{\pgfqpoint{2.866998in}{1.078012in}}%
\pgfpathlineto{\pgfqpoint{2.853362in}{1.084162in}}%
\pgfpathlineto{\pgfqpoint{2.844889in}{1.087971in}}%
\pgfpathlineto{\pgfqpoint{2.823483in}{1.097555in}}%
\pgfpathlineto{\pgfqpoint{2.822641in}{1.097931in}}%
\pgfpathlineto{\pgfqpoint{2.800236in}{1.107890in}}%
\pgfpathlineto{\pgfqpoint{2.793605in}{1.110828in}}%
\pgfpathlineto{\pgfqpoint{2.777686in}{1.117850in}}%
\pgfpathlineto{\pgfqpoint{2.763726in}{1.123987in}}%
\pgfpathlineto{\pgfqpoint{2.754992in}{1.127809in}}%
\pgfpathlineto{\pgfqpoint{2.733847in}{1.137034in}}%
\pgfpathlineto{\pgfqpoint{2.732154in}{1.137769in}}%
\pgfpathlineto{\pgfqpoint{2.709151in}{1.147729in}}%
\pgfpathlineto{\pgfqpoint{2.703969in}{1.149966in}}%
\pgfpathlineto{\pgfqpoint{2.685992in}{1.157688in}}%
\pgfpathlineto{\pgfqpoint{2.674090in}{1.162788in}}%
\pgfpathlineto{\pgfqpoint{2.662682in}{1.167648in}}%
\pgfpathlineto{\pgfqpoint{2.644211in}{1.175499in}}%
\pgfpathlineto{\pgfqpoint{2.639221in}{1.177607in}}%
\pgfpathlineto{\pgfqpoint{2.615600in}{1.187567in}}%
\pgfpathlineto{\pgfqpoint{2.614333in}{1.188100in}}%
\pgfpathlineto{\pgfqpoint{2.591795in}{1.197526in}}%
\pgfpathlineto{\pgfqpoint{2.584454in}{1.200592in}}%
\pgfpathlineto{\pgfqpoint{2.567829in}{1.207486in}}%
\pgfpathlineto{\pgfqpoint{2.554575in}{1.212975in}}%
\pgfpathlineto{\pgfqpoint{2.543701in}{1.217445in}}%
\pgfpathlineto{\pgfqpoint{2.524697in}{1.225250in}}%
\pgfpathlineto{\pgfqpoint{2.519407in}{1.227405in}}%
\pgfpathlineto{\pgfqpoint{2.494947in}{1.237365in}}%
\pgfpathlineto{\pgfqpoint{2.494818in}{1.237417in}}%
\pgfpathlineto{\pgfqpoint{2.470272in}{1.247324in}}%
\pgfpathlineto{\pgfqpoint{2.464939in}{1.249476in}}%
\pgfpathlineto{\pgfqpoint{2.445420in}{1.257284in}}%
\pgfpathlineto{\pgfqpoint{2.435061in}{1.261427in}}%
\pgfpathlineto{\pgfqpoint{2.420387in}{1.267243in}}%
\pgfpathlineto{\pgfqpoint{2.405182in}{1.273271in}}%
\pgfpathlineto{\pgfqpoint{2.395168in}{1.277203in}}%
\pgfpathlineto{\pgfqpoint{2.375303in}{1.285007in}}%
\pgfpathlineto{\pgfqpoint{2.369761in}{1.287162in}}%
\pgfpathlineto{\pgfqpoint{2.345425in}{1.296636in}}%
\pgfpathlineto{\pgfqpoint{2.344161in}{1.297122in}}%
\pgfpathlineto{\pgfqpoint{2.318328in}{1.307081in}}%
\pgfpathlineto{\pgfqpoint{2.315546in}{1.308155in}}%
\pgfpathlineto{\pgfqpoint{2.292271in}{1.317041in}}%
\pgfpathlineto{\pgfqpoint{2.285667in}{1.319566in}}%
\pgfpathlineto{\pgfqpoint{2.266000in}{1.327001in}}%
\pgfpathlineto{\pgfqpoint{2.255789in}{1.330868in}}%
\pgfpathlineto{\pgfqpoint{2.239507in}{1.336960in}}%
\pgfpathlineto{\pgfqpoint{2.225910in}{1.342060in}}%
\pgfpathlineto{\pgfqpoint{2.212786in}{1.346920in}}%
\pgfpathlineto{\pgfqpoint{2.196031in}{1.353141in}}%
\pgfpathlineto{\pgfqpoint{2.185829in}{1.356879in}}%
\pgfpathlineto{\pgfqpoint{2.166153in}{1.364111in}}%
\pgfpathlineto{\pgfqpoint{2.158626in}{1.366839in}}%
\pgfpathlineto{\pgfqpoint{2.136274in}{1.374969in}}%
\pgfpathlineto{\pgfqpoint{2.131169in}{1.376798in}}%
\pgfpathlineto{\pgfqpoint{2.106395in}{1.385712in}}%
\pgfpathlineto{\pgfqpoint{2.103445in}{1.386758in}}%
\pgfpathlineto{\pgfqpoint{2.076517in}{1.396342in}}%
\pgfpathlineto{\pgfqpoint{2.075443in}{1.396718in}}%
\pgfpathlineto{\pgfqpoint{2.047139in}{1.406677in}}%
\pgfpathlineto{\pgfqpoint{2.046638in}{1.406854in}}%
\pgfpathlineto{\pgfqpoint{2.018509in}{1.416637in}}%
\pgfpathlineto{\pgfqpoint{2.016759in}{1.417248in}}%
\pgfpathlineto{\pgfqpoint{1.989562in}{1.426596in}}%
\pgfpathlineto{\pgfqpoint{1.986880in}{1.427523in}}%
\pgfpathlineto{\pgfqpoint{1.960281in}{1.436556in}}%
\pgfpathlineto{\pgfqpoint{1.957002in}{1.437676in}}%
\pgfpathlineto{\pgfqpoint{1.930649in}{1.446515in}}%
\pgfpathlineto{\pgfqpoint{1.927123in}{1.447705in}}%
\pgfpathlineto{\pgfqpoint{1.900646in}{1.456475in}}%
\pgfpathlineto{\pgfqpoint{1.897244in}{1.457609in}}%
\pgfpathlineto{\pgfqpoint{1.870251in}{1.466434in}}%
\pgfpathlineto{\pgfqpoint{1.867366in}{1.467384in}}%
\pgfpathlineto{\pgfqpoint{1.839440in}{1.476394in}}%
\pgfpathlineto{\pgfqpoint{1.837487in}{1.477029in}}%
\pgfpathlineto{\pgfqpoint{1.808189in}{1.486354in}}%
\pgfpathlineto{\pgfqpoint{1.807608in}{1.486540in}}%
\pgfpathlineto{\pgfqpoint{1.777730in}{1.495912in}}%
\pgfpathlineto{\pgfqpoint{1.776420in}{1.496313in}}%
\pgfpathlineto{\pgfqpoint{1.747851in}{1.505142in}}%
\pgfpathlineto{\pgfqpoint{1.744102in}{1.506273in}}%
\pgfpathlineto{\pgfqpoint{1.717972in}{1.514227in}}%
\pgfpathlineto{\pgfqpoint{1.711216in}{1.516232in}}%
\pgfpathlineto{\pgfqpoint{1.688094in}{1.523163in}}%
\pgfpathlineto{\pgfqpoint{1.677721in}{1.526192in}}%
\pgfpathlineto{\pgfqpoint{1.658215in}{1.531945in}}%
\pgfpathlineto{\pgfqpoint{1.643568in}{1.536151in}}%
\pgfpathlineto{\pgfqpoint{1.628336in}{1.540571in}}%
\pgfpathlineto{\pgfqpoint{1.608704in}{1.546111in}}%
\pgfpathlineto{\pgfqpoint{1.598458in}{1.549034in}}%
\pgfpathlineto{\pgfqpoint{1.573070in}{1.556070in}}%
\pgfpathlineto{\pgfqpoint{1.568579in}{1.557329in}}%
\pgfpathlineto{\pgfqpoint{1.538700in}{1.565446in}}%
\pgfpathlineto{\pgfqpoint{1.536473in}{1.566030in}}%
\pgfpathlineto{\pgfqpoint{1.508822in}{1.573368in}}%
\pgfpathlineto{\pgfqpoint{1.498606in}{1.575990in}}%
\pgfpathlineto{\pgfqpoint{1.478943in}{1.581101in}}%
\pgfpathlineto{\pgfqpoint{1.459636in}{1.585949in}}%
\pgfpathlineto{\pgfqpoint{1.449064in}{1.588639in}}%
\pgfpathlineto{\pgfqpoint{1.419452in}{1.595909in}}%
\pgfpathlineto{\pgfqpoint{1.419186in}{1.595975in}}%
\pgfpathlineto{\pgfqpoint{1.389307in}{1.603059in}}%
\pgfpathlineto{\pgfqpoint{1.376957in}{1.605868in}}%
\pgfpathlineto{\pgfqpoint{1.359428in}{1.609917in}}%
\pgfpathlineto{\pgfqpoint{1.332728in}{1.615828in}}%
\pgfpathlineto{\pgfqpoint{1.329550in}{1.616542in}}%
\pgfpathlineto{\pgfqpoint{1.299671in}{1.622870in}}%
\pgfpathlineto{\pgfqpoint{1.285143in}{1.625787in}}%
\pgfpathlineto{\pgfqpoint{1.269792in}{1.628922in}}%
\pgfpathlineto{\pgfqpoint{1.239914in}{1.634678in}}%
\pgfpathlineto{\pgfqpoint{1.233902in}{1.635747in}}%
\pgfpathlineto{\pgfqpoint{1.210035in}{1.640067in}}%
\pgfpathlineto{\pgfqpoint{1.180156in}{1.645131in}}%
\pgfpathlineto{\pgfqpoint{1.176358in}{1.645706in}}%
\pgfpathlineto{\pgfqpoint{1.150277in}{1.649739in}}%
\pgfpathlineto{\pgfqpoint{1.120399in}{1.653938in}}%
\pgfpathlineto{\pgfqpoint{1.106210in}{1.655666in}}%
\pgfpathlineto{\pgfqpoint{1.090520in}{1.657618in}}%
\pgfpathlineto{\pgfqpoint{1.060641in}{1.660707in}}%
\pgfpathlineto{\pgfqpoint{1.030763in}{1.663165in}}%
\pgfpathlineto{\pgfqpoint{1.000884in}{1.664859in}}%
\pgfpathlineto{\pgfqpoint{0.971005in}{1.665609in}}%
\pgfpathlineto{\pgfqpoint{0.941127in}{1.665164in}}%
\pgfpathlineto{\pgfqpoint{0.911248in}{1.663166in}}%
\pgfpathlineto{\pgfqpoint{0.881369in}{1.659095in}}%
\pgfpathlineto{\pgfqpoint{0.865680in}{1.655666in}}%
\pgfpathlineto{\pgfqpoint{0.851491in}{1.651799in}}%
\pgfpathlineto{\pgfqpoint{0.835712in}{1.645706in}}%
\pgfpathlineto{\pgfqpoint{0.821612in}{1.638431in}}%
\pgfpathlineto{\pgfqpoint{0.817574in}{1.635747in}}%
\pgfpathlineto{\pgfqpoint{0.806261in}{1.625787in}}%
\pgfpathlineto{\pgfqpoint{0.798639in}{1.615828in}}%
\pgfpathlineto{\pgfqpoint{0.793897in}{1.605868in}}%
\pgfpathlineto{\pgfqpoint{0.791733in}{1.597154in}}%
\pgfpathlineto{\pgfqpoint{0.791467in}{1.595909in}}%
\pgfpathlineto{\pgfqpoint{0.790908in}{1.585949in}}%
\pgfpathlineto{\pgfqpoint{0.791733in}{1.576404in}}%
\pgfpathlineto{\pgfqpoint{0.791770in}{1.575990in}}%
\pgfpathlineto{\pgfqpoint{0.793961in}{1.566030in}}%
\pgfpathlineto{\pgfqpoint{0.797223in}{1.556070in}}%
\pgfpathlineto{\pgfqpoint{0.801404in}{1.546111in}}%
\pgfpathlineto{\pgfqpoint{0.806380in}{1.536151in}}%
\pgfpathlineto{\pgfqpoint{0.812057in}{1.526192in}}%
\pgfpathlineto{\pgfqpoint{0.818356in}{1.516232in}}%
\pgfpathlineto{\pgfqpoint{0.821612in}{1.511595in}}%
\pgfpathlineto{\pgfqpoint{0.825362in}{1.506273in}}%
\pgfpathlineto{\pgfqpoint{0.832991in}{1.496313in}}%
\pgfpathlineto{\pgfqpoint{0.841061in}{1.486354in}}%
\pgfpathlineto{\pgfqpoint{0.849538in}{1.476394in}}%
\pgfpathlineto{\pgfqpoint{0.851491in}{1.474249in}}%
\pgfpathlineto{\pgfqpoint{0.858612in}{1.466434in}}%
\pgfpathlineto{\pgfqpoint{0.868074in}{1.456475in}}%
\pgfpathlineto{\pgfqpoint{0.877844in}{1.446515in}}%
\pgfpathlineto{\pgfqpoint{0.881369in}{1.443086in}}%
\pgfpathlineto{\pgfqpoint{0.888082in}{1.436556in}}%
\pgfpathlineto{\pgfqpoint{0.898668in}{1.426596in}}%
\pgfpathlineto{\pgfqpoint{0.909498in}{1.416637in}}%
\pgfpathlineto{\pgfqpoint{0.911248in}{1.415092in}}%
\pgfpathlineto{\pgfqpoint{0.920775in}{1.406677in}}%
\pgfpathlineto{\pgfqpoint{0.932294in}{1.396718in}}%
\pgfpathlineto{\pgfqpoint{0.941127in}{1.389255in}}%
\pgfpathlineto{\pgfqpoint{0.944077in}{1.386758in}}%
\pgfpathlineto{\pgfqpoint{0.956221in}{1.376798in}}%
\pgfpathlineto{\pgfqpoint{0.968540in}{1.366839in}}%
\pgfpathlineto{\pgfqpoint{0.971005in}{1.364904in}}%
\pgfpathlineto{\pgfqpoint{0.981208in}{1.356879in}}%
\pgfpathlineto{\pgfqpoint{0.994071in}{1.346920in}}%
\pgfpathlineto{\pgfqpoint{1.000884in}{1.341753in}}%
\pgfpathlineto{\pgfqpoint{1.007188in}{1.336960in}}%
\pgfpathlineto{\pgfqpoint{1.020552in}{1.327001in}}%
\pgfpathlineto{\pgfqpoint{1.030763in}{1.319502in}}%
\pgfpathlineto{\pgfqpoint{1.034105in}{1.317041in}}%
\pgfpathlineto{\pgfqpoint{1.047930in}{1.307081in}}%
\pgfpathlineto{\pgfqpoint{1.060641in}{1.298024in}}%
\pgfpathlineto{\pgfqpoint{1.061905in}{1.297122in}}%
\pgfpathlineto{\pgfqpoint{1.076159in}{1.287162in}}%
\pgfpathlineto{\pgfqpoint{1.090520in}{1.277216in}}%
\pgfpathlineto{\pgfqpoint{1.090539in}{1.277203in}}%
\pgfpathlineto{\pgfqpoint{1.105194in}{1.267243in}}%
\pgfpathlineto{\pgfqpoint{1.119970in}{1.257284in}}%
\pgfpathlineto{\pgfqpoint{1.120399in}{1.257000in}}%
\pgfpathlineto{\pgfqpoint{1.134998in}{1.247324in}}%
\pgfpathlineto{\pgfqpoint{1.150149in}{1.237365in}}%
\pgfpathlineto{\pgfqpoint{1.150277in}{1.237282in}}%
\pgfpathlineto{\pgfqpoint{1.165537in}{1.227405in}}%
\pgfpathlineto{\pgfqpoint{1.180156in}{1.218021in}}%
\pgfpathlineto{\pgfqpoint{1.181050in}{1.217445in}}%
\pgfpathlineto{\pgfqpoint{1.196781in}{1.207486in}}%
\pgfpathlineto{\pgfqpoint{1.210035in}{1.199169in}}%
\pgfpathlineto{\pgfqpoint{1.212644in}{1.197526in}}%
\pgfpathlineto{\pgfqpoint{1.228702in}{1.187567in}}%
\pgfpathlineto{\pgfqpoint{1.239914in}{1.180681in}}%
\pgfpathlineto{\pgfqpoint{1.244904in}{1.177607in}}%
\pgfpathlineto{\pgfqpoint{1.261278in}{1.167648in}}%
\pgfpathlineto{\pgfqpoint{1.269792in}{1.162525in}}%
\pgfpathlineto{\pgfqpoint{1.277807in}{1.157688in}}%
\pgfpathlineto{\pgfqpoint{1.294489in}{1.147729in}}%
\pgfpathlineto{\pgfqpoint{1.299671in}{1.144670in}}%
\pgfpathlineto{\pgfqpoint{1.311334in}{1.137769in}}%
\pgfpathlineto{\pgfqpoint{1.328316in}{1.127809in}}%
\pgfpathlineto{\pgfqpoint{1.329550in}{1.127095in}}%
\pgfpathlineto{\pgfqpoint{1.345468in}{1.117850in}}%
\pgfpathlineto{\pgfqpoint{1.359428in}{1.109810in}}%
\pgfpathlineto{\pgfqpoint{1.362754in}{1.107890in}}%
\pgfpathlineto{\pgfqpoint{1.380196in}{1.097931in}}%
\pgfpathlineto{\pgfqpoint{1.389307in}{1.092776in}}%
\pgfpathlineto{\pgfqpoint{1.397780in}{1.087971in}}%
\pgfpathlineto{\pgfqpoint{1.415506in}{1.078012in}}%
\pgfpathlineto{\pgfqpoint{1.419186in}{1.075965in}}%
\pgfpathlineto{\pgfqpoint{1.433381in}{1.068052in}}%
\pgfpathlineto{\pgfqpoint{1.449064in}{1.059381in}}%
\pgfpathlineto{\pgfqpoint{1.451390in}{1.058092in}}%
\pgfpathlineto{\pgfqpoint{1.469546in}{1.048133in}}%
\pgfpathlineto{\pgfqpoint{1.478943in}{1.043021in}}%
\pgfpathlineto{\pgfqpoint{1.487839in}{1.038173in}}%
\pgfpathlineto{\pgfqpoint{1.506269in}{1.028214in}}%
\pgfpathlineto{\pgfqpoint{1.508822in}{1.026847in}}%
\pgfpathlineto{\pgfqpoint{1.524841in}{1.018254in}}%
\pgfpathlineto{\pgfqpoint{1.538700in}{1.010878in}}%
\pgfpathlineto{\pgfqpoint{1.543547in}{1.008295in}}%
\pgfpathlineto{\pgfqpoint{1.562390in}{0.998335in}}%
\pgfpathlineto{\pgfqpoint{1.568579in}{0.995090in}}%
\pgfpathlineto{\pgfqpoint{1.581369in}{0.988376in}}%
\pgfpathlineto{\pgfqpoint{1.598458in}{0.979472in}}%
\pgfpathlineto{\pgfqpoint{1.600483in}{0.978416in}}%
\pgfpathlineto{\pgfqpoint{1.619732in}{0.968456in}}%
\pgfpathlineto{\pgfqpoint{1.628336in}{0.964037in}}%
\pgfpathlineto{\pgfqpoint{1.639115in}{0.958497in}}%
\pgfpathlineto{\pgfqpoint{1.658215in}{0.948751in}}%
\pgfpathlineto{\pgfqpoint{1.658633in}{0.948537in}}%
\pgfpathlineto{\pgfqpoint{1.678285in}{0.938578in}}%
\pgfpathlineto{\pgfqpoint{1.688094in}{0.933641in}}%
\pgfpathlineto{\pgfqpoint{1.698070in}{0.928618in}}%
\pgfpathlineto{\pgfqpoint{1.717972in}{0.918667in}}%
\pgfpathlineto{\pgfqpoint{1.717989in}{0.918659in}}%
\pgfpathlineto{\pgfqpoint{1.738042in}{0.908699in}}%
\pgfpathlineto{\pgfqpoint{1.747851in}{0.903859in}}%
\pgfpathlineto{\pgfqpoint{1.758229in}{0.898740in}}%
\pgfpathlineto{\pgfqpoint{1.777730in}{0.889181in}}%
\pgfpathlineto{\pgfqpoint{1.778549in}{0.888780in}}%
\pgfpathlineto{\pgfqpoint{1.799004in}{0.878820in}}%
\pgfpathlineto{\pgfqpoint{1.807608in}{0.874656in}}%
\pgfpathlineto{\pgfqpoint{1.819594in}{0.868861in}}%
\pgfpathlineto{\pgfqpoint{1.837487in}{0.860260in}}%
\pgfpathlineto{\pgfqpoint{1.840317in}{0.858901in}}%
\pgfpathlineto{\pgfqpoint{1.861177in}{0.848942in}}%
\pgfpathlineto{\pgfqpoint{1.867366in}{0.846003in}}%
\pgfpathlineto{\pgfqpoint{1.882173in}{0.838982in}}%
\pgfpathlineto{\pgfqpoint{1.897244in}{0.831875in}}%
\pgfpathlineto{\pgfqpoint{1.903303in}{0.829023in}}%
\pgfpathlineto{\pgfqpoint{1.924571in}{0.819063in}}%
\pgfpathlineto{\pgfqpoint{1.927123in}{0.817873in}}%
\pgfpathlineto{\pgfqpoint{1.945979in}{0.809103in}}%
\pgfpathlineto{\pgfqpoint{1.957002in}{0.804002in}}%
\pgfpathlineto{\pgfqpoint{1.967525in}{0.799144in}}%
\pgfpathlineto{\pgfqpoint{1.986880in}{0.790250in}}%
\pgfpathlineto{\pgfqpoint{1.989206in}{0.789184in}}%
\pgfpathlineto{\pgfqpoint{2.011033in}{0.779225in}}%
\pgfpathlineto{\pgfqpoint{2.016759in}{0.776623in}}%
\pgfpathlineto{\pgfqpoint{2.033002in}{0.769265in}}%
\pgfpathlineto{\pgfqpoint{2.046638in}{0.763114in}}%
\pgfpathlineto{\pgfqpoint{2.055111in}{0.759306in}}%
\pgfpathlineto{\pgfqpoint{2.076517in}{0.749722in}}%
\pgfpathlineto{\pgfqpoint{2.077359in}{0.749346in}}%
\pgfpathlineto{\pgfqpoint{2.099764in}{0.739387in}}%
\pgfpathlineto{\pgfqpoint{2.106395in}{0.736449in}}%
\pgfpathlineto{\pgfqpoint{2.122314in}{0.729427in}}%
\pgfpathlineto{\pgfqpoint{2.136274in}{0.723290in}}%
\pgfpathlineto{\pgfqpoint{2.145008in}{0.719467in}}%
\pgfpathlineto{\pgfqpoint{2.166153in}{0.710243in}}%
\pgfpathlineto{\pgfqpoint{2.167846in}{0.709508in}}%
\pgfpathlineto{\pgfqpoint{2.190849in}{0.699548in}}%
\pgfpathlineto{\pgfqpoint{2.196031in}{0.697310in}}%
\pgfpathlineto{\pgfqpoint{2.214008in}{0.689589in}}%
\pgfpathlineto{\pgfqpoint{2.225910in}{0.684489in}}%
\pgfpathlineto{\pgfqpoint{2.237318in}{0.679629in}}%
\pgfpathlineto{\pgfqpoint{2.255789in}{0.671778in}}%
\pgfpathlineto{\pgfqpoint{2.260779in}{0.669670in}}%
\pgfpathlineto{\pgfqpoint{2.284400in}{0.659710in}}%
\pgfpathlineto{\pgfqpoint{2.285667in}{0.659177in}}%
\pgfpathlineto{\pgfqpoint{2.308205in}{0.649751in}}%
\pgfpathlineto{\pgfqpoint{2.315546in}{0.646685in}}%
\pgfpathlineto{\pgfqpoint{2.332171in}{0.639791in}}%
\pgfpathlineto{\pgfqpoint{2.345425in}{0.634302in}}%
\pgfpathlineto{\pgfqpoint{2.356299in}{0.629831in}}%
\pgfpathlineto{\pgfqpoint{2.375303in}{0.622027in}}%
\pgfpathlineto{\pgfqpoint{2.380593in}{0.619872in}}%
\pgfpathlineto{\pgfqpoint{2.405053in}{0.609912in}}%
\pgfpathlineto{\pgfqpoint{2.405182in}{0.609860in}}%
\pgfpathlineto{\pgfqpoint{2.429728in}{0.599953in}}%
\pgfpathlineto{\pgfqpoint{2.435061in}{0.597801in}}%
\pgfpathlineto{\pgfqpoint{2.454580in}{0.589993in}}%
\pgfpathlineto{\pgfqpoint{2.464939in}{0.585849in}}%
\pgfpathlineto{\pgfqpoint{2.479613in}{0.580034in}}%
\pgfpathlineto{\pgfqpoint{2.494818in}{0.574006in}}%
\pgfpathlineto{\pgfqpoint{2.504832in}{0.570074in}}%
\pgfpathlineto{\pgfqpoint{2.524697in}{0.562270in}}%
\pgfpathlineto{\pgfqpoint{2.530239in}{0.560115in}}%
\pgfpathlineto{\pgfqpoint{2.554575in}{0.550641in}}%
\pgfpathlineto{\pgfqpoint{2.555839in}{0.550155in}}%
\pgfpathlineto{\pgfqpoint{2.581672in}{0.540195in}}%
\pgfpathlineto{\pgfqpoint{2.584454in}{0.539122in}}%
\pgfpathlineto{\pgfqpoint{2.607729in}{0.530236in}}%
\pgfpathlineto{\pgfqpoint{2.614333in}{0.527711in}}%
\pgfpathlineto{\pgfqpoint{2.634000in}{0.520276in}}%
\pgfpathlineto{\pgfqpoint{2.644211in}{0.516409in}}%
\pgfpathlineto{\pgfqpoint{2.660493in}{0.510317in}}%
\pgfpathlineto{\pgfqpoint{2.674090in}{0.505217in}}%
\pgfpathlineto{\pgfqpoint{2.687214in}{0.500357in}}%
\pgfpathlineto{\pgfqpoint{2.703969in}{0.494136in}}%
\pgfpathlineto{\pgfqpoint{2.714171in}{0.490398in}}%
\pgfpathlineto{\pgfqpoint{2.733847in}{0.483166in}}%
\pgfpathlineto{\pgfqpoint{2.741374in}{0.480438in}}%
\pgfpathlineto{\pgfqpoint{2.763726in}{0.472308in}}%
\pgfpathlineto{\pgfqpoint{2.768831in}{0.470478in}}%
\pgfpathlineto{\pgfqpoint{2.793605in}{0.461564in}}%
\pgfpathlineto{\pgfqpoint{2.796555in}{0.460519in}}%
\pgfpathlineto{\pgfqpoint{2.823483in}{0.450935in}}%
\pgfpathlineto{\pgfqpoint{2.824557in}{0.450559in}}%
\pgfpathlineto{\pgfqpoint{2.852861in}{0.440600in}}%
\pgfpathlineto{\pgfqpoint{2.853362in}{0.440423in}}%
\pgfpathlineto{\pgfqpoint{2.881491in}{0.430640in}}%
\pgfpathlineto{\pgfqpoint{2.883241in}{0.430029in}}%
\pgfpathlineto{\pgfqpoint{2.910438in}{0.420681in}}%
\pgfpathlineto{\pgfqpoint{2.913120in}{0.419754in}}%
\pgfpathlineto{\pgfqpoint{2.939719in}{0.410721in}}%
\pgfpathlineto{\pgfqpoint{2.942998in}{0.409601in}}%
\pgfpathlineto{\pgfqpoint{2.969351in}{0.400762in}}%
\pgfpathlineto{\pgfqpoint{2.972877in}{0.399572in}}%
\pgfpathlineto{\pgfqpoint{2.999354in}{0.390802in}}%
\pgfpathlineto{\pgfqpoint{3.002756in}{0.389668in}}%
\pgfpathlineto{\pgfqpoint{3.029749in}{0.380842in}}%
\pgfpathlineto{\pgfqpoint{3.032634in}{0.379893in}}%
\pgfpathlineto{\pgfqpoint{3.060560in}{0.370883in}}%
\pgfpathlineto{\pgfqpoint{3.062513in}{0.370248in}}%
\pgfpathlineto{\pgfqpoint{3.091811in}{0.360923in}}%
\pgfpathlineto{\pgfqpoint{3.092392in}{0.360737in}}%
\pgfpathlineto{\pgfqpoint{3.122270in}{0.351365in}}%
\pgfpathlineto{\pgfqpoint{3.123580in}{0.350964in}}%
\pgfpathlineto{\pgfqpoint{3.152149in}{0.342135in}}%
\pgfpathlineto{\pgfqpoint{3.155898in}{0.341004in}}%
\pgfpathlineto{\pgfqpoint{3.182028in}{0.333050in}}%
\pgfpathlineto{\pgfqpoint{3.188784in}{0.331045in}}%
\pgfpathlineto{\pgfqpoint{3.211906in}{0.324114in}}%
\pgfpathlineto{\pgfqpoint{3.222279in}{0.321085in}}%
\pgfpathlineto{\pgfqpoint{3.241785in}{0.315331in}}%
\pgfpathlineto{\pgfqpoint{3.256432in}{0.311126in}}%
\pgfpathlineto{\pgfqpoint{3.271664in}{0.306706in}}%
\pgfpathlineto{\pgfqpoint{3.291296in}{0.301166in}}%
\pgfpathlineto{\pgfqpoint{3.301542in}{0.298243in}}%
\pgfpathlineto{\pgfqpoint{3.326930in}{0.291206in}}%
\pgfpathlineto{\pgfqpoint{3.331421in}{0.289948in}}%
\pgfpathlineto{\pgfqpoint{3.361300in}{0.281831in}}%
\pgfpathlineto{\pgfqpoint{3.363527in}{0.281247in}}%
\pgfpathlineto{\pgfqpoint{3.391178in}{0.273909in}}%
\pgfpathlineto{\pgfqpoint{3.401394in}{0.271287in}}%
\pgfpathlineto{\pgfqpoint{3.421057in}{0.266176in}}%
\pgfpathlineto{\pgfqpoint{3.440364in}{0.261328in}}%
\pgfpathlineto{\pgfqpoint{3.450936in}{0.258638in}}%
\pgfpathlineto{\pgfqpoint{3.480548in}{0.251368in}}%
\pgfpathlineto{\pgfqpoint{3.480814in}{0.251302in}}%
\pgfpathlineto{\pgfqpoint{3.510693in}{0.244218in}}%
\pgfpathlineto{\pgfqpoint{3.523043in}{0.241409in}}%
\pgfpathlineto{\pgfqpoint{3.540572in}{0.237360in}}%
\pgfpathlineto{\pgfqpoint{3.567272in}{0.231449in}}%
\pgfpathlineto{\pgfqpoint{3.570450in}{0.230735in}}%
\pgfpathlineto{\pgfqpoint{3.600329in}{0.224407in}}%
\pgfpathlineto{\pgfqpoint{3.614857in}{0.221489in}}%
\pgfpathlineto{\pgfqpoint{3.630208in}{0.218354in}}%
\pgfpathlineto{\pgfqpoint{3.660086in}{0.212598in}}%
\pgfpathlineto{\pgfqpoint{3.666098in}{0.211530in}}%
\pgfpathlineto{\pgfqpoint{3.689965in}{0.207210in}}%
\pgfpathlineto{\pgfqpoint{3.719844in}{0.202146in}}%
\pgfpathlineto{\pgfqpoint{3.723642in}{0.201570in}}%
\pgfpathlineto{\pgfqpoint{3.749723in}{0.197538in}}%
\pgfpathlineto{\pgfqpoint{3.779601in}{0.193338in}}%
\pgfpathlineto{\pgfqpoint{3.793790in}{0.191611in}}%
\pgfpathclose%
\pgfusepath{fill}%
\end{pgfscope}%
\begin{pgfscope}%
\pgfpathrectangle{\pgfqpoint{0.135000in}{0.151972in}}{\pgfqpoint{4.630000in}{1.543333in}} %
\pgfusepath{clip}%
\pgfsetbuttcap%
\pgfsetroundjoin%
\definecolor{currentfill}{rgb}{0.000000,0.000000,1.000000}%
\pgfsetfillcolor{currentfill}%
\pgfsetlinewidth{0.000000pt}%
\definecolor{currentstroke}{rgb}{0.000000,0.000000,0.000000}%
\pgfsetstrokecolor{currentstroke}%
\pgfsetdash{}{0pt}%
\pgfpathmoveto{\pgfqpoint{3.809480in}{0.189659in}}%
\pgfpathlineto{\pgfqpoint{3.839359in}{0.186570in}}%
\pgfpathlineto{\pgfqpoint{3.869237in}{0.184112in}}%
\pgfpathlineto{\pgfqpoint{3.899116in}{0.182418in}}%
\pgfpathlineto{\pgfqpoint{3.928995in}{0.181668in}}%
\pgfpathlineto{\pgfqpoint{3.958873in}{0.182113in}}%
\pgfpathlineto{\pgfqpoint{3.988752in}{0.184110in}}%
\pgfpathlineto{\pgfqpoint{4.018631in}{0.188182in}}%
\pgfpathlineto{\pgfqpoint{4.034320in}{0.191611in}}%
\pgfpathlineto{\pgfqpoint{4.048509in}{0.195478in}}%
\pgfpathlineto{\pgfqpoint{4.064288in}{0.201570in}}%
\pgfpathlineto{\pgfqpoint{4.078388in}{0.208846in}}%
\pgfpathlineto{\pgfqpoint{4.082426in}{0.211530in}}%
\pgfpathlineto{\pgfqpoint{4.093739in}{0.221489in}}%
\pgfpathlineto{\pgfqpoint{4.101361in}{0.231449in}}%
\pgfpathlineto{\pgfqpoint{4.106103in}{0.241409in}}%
\pgfpathlineto{\pgfqpoint{4.108267in}{0.250123in}}%
\pgfpathlineto{\pgfqpoint{4.108533in}{0.251368in}}%
\pgfpathlineto{\pgfqpoint{4.109092in}{0.261328in}}%
\pgfpathlineto{\pgfqpoint{4.108267in}{0.270872in}}%
\pgfpathlineto{\pgfqpoint{4.108230in}{0.271287in}}%
\pgfpathlineto{\pgfqpoint{4.106039in}{0.281247in}}%
\pgfpathlineto{\pgfqpoint{4.102777in}{0.291206in}}%
\pgfpathlineto{\pgfqpoint{4.098596in}{0.301166in}}%
\pgfpathlineto{\pgfqpoint{4.093620in}{0.311126in}}%
\pgfpathlineto{\pgfqpoint{4.087943in}{0.321085in}}%
\pgfpathlineto{\pgfqpoint{4.081644in}{0.331045in}}%
\pgfpathlineto{\pgfqpoint{4.078388in}{0.335681in}}%
\pgfpathlineto{\pgfqpoint{4.074638in}{0.341004in}}%
\pgfpathlineto{\pgfqpoint{4.067009in}{0.350964in}}%
\pgfpathlineto{\pgfqpoint{4.058939in}{0.360923in}}%
\pgfpathlineto{\pgfqpoint{4.050462in}{0.370883in}}%
\pgfpathlineto{\pgfqpoint{4.048509in}{0.373028in}}%
\pgfpathlineto{\pgfqpoint{4.041388in}{0.380842in}}%
\pgfpathlineto{\pgfqpoint{4.031926in}{0.390802in}}%
\pgfpathlineto{\pgfqpoint{4.022156in}{0.400762in}}%
\pgfpathlineto{\pgfqpoint{4.018631in}{0.404191in}}%
\pgfpathlineto{\pgfqpoint{4.011918in}{0.410721in}}%
\pgfpathlineto{\pgfqpoint{4.001332in}{0.420681in}}%
\pgfpathlineto{\pgfqpoint{3.990502in}{0.430640in}}%
\pgfpathlineto{\pgfqpoint{3.988752in}{0.432184in}}%
\pgfpathlineto{\pgfqpoint{3.979225in}{0.440600in}}%
\pgfpathlineto{\pgfqpoint{3.967706in}{0.450559in}}%
\pgfpathlineto{\pgfqpoint{3.958873in}{0.458022in}}%
\pgfpathlineto{\pgfqpoint{3.955923in}{0.460519in}}%
\pgfpathlineto{\pgfqpoint{3.943779in}{0.470478in}}%
\pgfpathlineto{\pgfqpoint{3.931460in}{0.480438in}}%
\pgfpathlineto{\pgfqpoint{3.928995in}{0.482373in}}%
\pgfpathlineto{\pgfqpoint{3.918792in}{0.490398in}}%
\pgfpathlineto{\pgfqpoint{3.905929in}{0.500357in}}%
\pgfpathlineto{\pgfqpoint{3.899116in}{0.505524in}}%
\pgfpathlineto{\pgfqpoint{3.892812in}{0.510317in}}%
\pgfpathlineto{\pgfqpoint{3.879448in}{0.520276in}}%
\pgfpathlineto{\pgfqpoint{3.869237in}{0.527775in}}%
\pgfpathlineto{\pgfqpoint{3.865895in}{0.530236in}}%
\pgfpathlineto{\pgfqpoint{3.852070in}{0.540195in}}%
\pgfpathlineto{\pgfqpoint{3.839359in}{0.549252in}}%
\pgfpathlineto{\pgfqpoint{3.838095in}{0.550155in}}%
\pgfpathlineto{\pgfqpoint{3.823841in}{0.560115in}}%
\pgfpathlineto{\pgfqpoint{3.809480in}{0.570061in}}%
\pgfpathlineto{\pgfqpoint{3.809461in}{0.570074in}}%
\pgfpathlineto{\pgfqpoint{3.794806in}{0.580034in}}%
\pgfpathlineto{\pgfqpoint{3.780030in}{0.589993in}}%
\pgfpathlineto{\pgfqpoint{3.779601in}{0.590276in}}%
\pgfpathlineto{\pgfqpoint{3.765002in}{0.599953in}}%
\pgfpathlineto{\pgfqpoint{3.749851in}{0.609912in}}%
\pgfpathlineto{\pgfqpoint{3.749723in}{0.609995in}}%
\pgfpathlineto{\pgfqpoint{3.734463in}{0.619872in}}%
\pgfpathlineto{\pgfqpoint{3.719844in}{0.629256in}}%
\pgfpathlineto{\pgfqpoint{3.718950in}{0.629831in}}%
\pgfpathlineto{\pgfqpoint{3.703219in}{0.639791in}}%
\pgfpathlineto{\pgfqpoint{3.689965in}{0.648108in}}%
\pgfpathlineto{\pgfqpoint{3.687356in}{0.649751in}}%
\pgfpathlineto{\pgfqpoint{3.671298in}{0.659710in}}%
\pgfpathlineto{\pgfqpoint{3.660086in}{0.666596in}}%
\pgfpathlineto{\pgfqpoint{3.655096in}{0.669670in}}%
\pgfpathlineto{\pgfqpoint{3.638722in}{0.679629in}}%
\pgfpathlineto{\pgfqpoint{3.630208in}{0.684752in}}%
\pgfpathlineto{\pgfqpoint{3.622193in}{0.689589in}}%
\pgfpathlineto{\pgfqpoint{3.605511in}{0.699548in}}%
\pgfpathlineto{\pgfqpoint{3.600329in}{0.702606in}}%
\pgfpathlineto{\pgfqpoint{3.588666in}{0.709508in}}%
\pgfpathlineto{\pgfqpoint{3.571684in}{0.719467in}}%
\pgfpathlineto{\pgfqpoint{3.570450in}{0.720182in}}%
\pgfpathlineto{\pgfqpoint{3.554532in}{0.729427in}}%
\pgfpathlineto{\pgfqpoint{3.540572in}{0.737466in}}%
\pgfpathlineto{\pgfqpoint{3.537246in}{0.739387in}}%
\pgfpathlineto{\pgfqpoint{3.519804in}{0.749346in}}%
\pgfpathlineto{\pgfqpoint{3.510693in}{0.754501in}}%
\pgfpathlineto{\pgfqpoint{3.502220in}{0.759306in}}%
\pgfpathlineto{\pgfqpoint{3.484494in}{0.769265in}}%
\pgfpathlineto{\pgfqpoint{3.480814in}{0.771312in}}%
\pgfpathlineto{\pgfqpoint{3.466619in}{0.779225in}}%
\pgfpathlineto{\pgfqpoint{3.450936in}{0.787896in}}%
\pgfpathlineto{\pgfqpoint{3.448610in}{0.789184in}}%
\pgfpathlineto{\pgfqpoint{3.430454in}{0.799144in}}%
\pgfpathlineto{\pgfqpoint{3.421057in}{0.804255in}}%
\pgfpathlineto{\pgfqpoint{3.412161in}{0.809103in}}%
\pgfpathlineto{\pgfqpoint{3.393731in}{0.819063in}}%
\pgfpathlineto{\pgfqpoint{3.391178in}{0.820430in}}%
\pgfpathlineto{\pgfqpoint{3.375159in}{0.829023in}}%
\pgfpathlineto{\pgfqpoint{3.361300in}{0.836399in}}%
\pgfpathlineto{\pgfqpoint{3.356453in}{0.838982in}}%
\pgfpathlineto{\pgfqpoint{3.337610in}{0.848942in}}%
\pgfpathlineto{\pgfqpoint{3.331421in}{0.852187in}}%
\pgfpathlineto{\pgfqpoint{3.318631in}{0.858901in}}%
\pgfpathlineto{\pgfqpoint{3.301542in}{0.867805in}}%
\pgfpathlineto{\pgfqpoint{3.299517in}{0.868861in}}%
\pgfpathlineto{\pgfqpoint{3.280268in}{0.878820in}}%
\pgfpathlineto{\pgfqpoint{3.271664in}{0.883240in}}%
\pgfpathlineto{\pgfqpoint{3.260885in}{0.888780in}}%
\pgfpathlineto{\pgfqpoint{3.241785in}{0.898526in}}%
\pgfpathlineto{\pgfqpoint{3.241367in}{0.898740in}}%
\pgfpathlineto{\pgfqpoint{3.221715in}{0.908699in}}%
\pgfpathlineto{\pgfqpoint{3.211906in}{0.913636in}}%
\pgfpathlineto{\pgfqpoint{3.201930in}{0.918659in}}%
\pgfpathlineto{\pgfqpoint{3.182028in}{0.928610in}}%
\pgfpathlineto{\pgfqpoint{3.182011in}{0.928618in}}%
\pgfpathlineto{\pgfqpoint{3.161958in}{0.938578in}}%
\pgfpathlineto{\pgfqpoint{3.152149in}{0.943418in}}%
\pgfpathlineto{\pgfqpoint{3.141771in}{0.948537in}}%
\pgfpathlineto{\pgfqpoint{3.122270in}{0.958096in}}%
\pgfpathlineto{\pgfqpoint{3.121451in}{0.958497in}}%
\pgfpathlineto{\pgfqpoint{3.100996in}{0.968456in}}%
\pgfpathlineto{\pgfqpoint{3.092392in}{0.972621in}}%
\pgfpathlineto{\pgfqpoint{3.080406in}{0.978416in}}%
\pgfpathlineto{\pgfqpoint{3.062513in}{0.987017in}}%
\pgfpathlineto{\pgfqpoint{3.059683in}{0.988376in}}%
\pgfpathlineto{\pgfqpoint{3.038823in}{0.998335in}}%
\pgfpathlineto{\pgfqpoint{3.032634in}{1.001274in}}%
\pgfpathlineto{\pgfqpoint{3.017827in}{1.008295in}}%
\pgfpathlineto{\pgfqpoint{3.002756in}{1.015402in}}%
\pgfpathlineto{\pgfqpoint{2.996697in}{1.018254in}}%
\pgfpathlineto{\pgfqpoint{2.975429in}{1.028214in}}%
\pgfpathlineto{\pgfqpoint{2.972877in}{1.029403in}}%
\pgfpathlineto{\pgfqpoint{2.954021in}{1.038173in}}%
\pgfpathlineto{\pgfqpoint{2.942998in}{1.043275in}}%
\pgfpathlineto{\pgfqpoint{2.932475in}{1.048133in}}%
\pgfpathlineto{\pgfqpoint{2.913120in}{1.057027in}}%
\pgfpathlineto{\pgfqpoint{2.910794in}{1.058092in}}%
\pgfpathlineto{\pgfqpoint{2.888967in}{1.068052in}}%
\pgfpathlineto{\pgfqpoint{2.883241in}{1.070654in}}%
\pgfpathlineto{\pgfqpoint{2.866998in}{1.078012in}}%
\pgfpathlineto{\pgfqpoint{2.853362in}{1.084162in}}%
\pgfpathlineto{\pgfqpoint{2.844889in}{1.087971in}}%
\pgfpathlineto{\pgfqpoint{2.823483in}{1.097555in}}%
\pgfpathlineto{\pgfqpoint{2.822641in}{1.097931in}}%
\pgfpathlineto{\pgfqpoint{2.800236in}{1.107890in}}%
\pgfpathlineto{\pgfqpoint{2.793605in}{1.110828in}}%
\pgfpathlineto{\pgfqpoint{2.777686in}{1.117850in}}%
\pgfpathlineto{\pgfqpoint{2.763726in}{1.123987in}}%
\pgfpathlineto{\pgfqpoint{2.754992in}{1.127809in}}%
\pgfpathlineto{\pgfqpoint{2.733847in}{1.137034in}}%
\pgfpathlineto{\pgfqpoint{2.732154in}{1.137769in}}%
\pgfpathlineto{\pgfqpoint{2.709151in}{1.147729in}}%
\pgfpathlineto{\pgfqpoint{2.703969in}{1.149966in}}%
\pgfpathlineto{\pgfqpoint{2.685992in}{1.157688in}}%
\pgfpathlineto{\pgfqpoint{2.674090in}{1.162788in}}%
\pgfpathlineto{\pgfqpoint{2.662682in}{1.167648in}}%
\pgfpathlineto{\pgfqpoint{2.644211in}{1.175499in}}%
\pgfpathlineto{\pgfqpoint{2.639221in}{1.177607in}}%
\pgfpathlineto{\pgfqpoint{2.615600in}{1.187567in}}%
\pgfpathlineto{\pgfqpoint{2.614333in}{1.188100in}}%
\pgfpathlineto{\pgfqpoint{2.591795in}{1.197526in}}%
\pgfpathlineto{\pgfqpoint{2.584454in}{1.200592in}}%
\pgfpathlineto{\pgfqpoint{2.567829in}{1.207486in}}%
\pgfpathlineto{\pgfqpoint{2.554575in}{1.212975in}}%
\pgfpathlineto{\pgfqpoint{2.543701in}{1.217445in}}%
\pgfpathlineto{\pgfqpoint{2.524697in}{1.225250in}}%
\pgfpathlineto{\pgfqpoint{2.519407in}{1.227405in}}%
\pgfpathlineto{\pgfqpoint{2.494947in}{1.237365in}}%
\pgfpathlineto{\pgfqpoint{2.494818in}{1.237417in}}%
\pgfpathlineto{\pgfqpoint{2.470272in}{1.247324in}}%
\pgfpathlineto{\pgfqpoint{2.464939in}{1.249476in}}%
\pgfpathlineto{\pgfqpoint{2.445420in}{1.257284in}}%
\pgfpathlineto{\pgfqpoint{2.435061in}{1.261427in}}%
\pgfpathlineto{\pgfqpoint{2.420387in}{1.267243in}}%
\pgfpathlineto{\pgfqpoint{2.405182in}{1.273271in}}%
\pgfpathlineto{\pgfqpoint{2.395168in}{1.277203in}}%
\pgfpathlineto{\pgfqpoint{2.375303in}{1.285007in}}%
\pgfpathlineto{\pgfqpoint{2.369761in}{1.287162in}}%
\pgfpathlineto{\pgfqpoint{2.345425in}{1.296636in}}%
\pgfpathlineto{\pgfqpoint{2.344161in}{1.297122in}}%
\pgfpathlineto{\pgfqpoint{2.318328in}{1.307081in}}%
\pgfpathlineto{\pgfqpoint{2.315546in}{1.308155in}}%
\pgfpathlineto{\pgfqpoint{2.292271in}{1.317041in}}%
\pgfpathlineto{\pgfqpoint{2.285667in}{1.319566in}}%
\pgfpathlineto{\pgfqpoint{2.266000in}{1.327001in}}%
\pgfpathlineto{\pgfqpoint{2.255789in}{1.330868in}}%
\pgfpathlineto{\pgfqpoint{2.239507in}{1.336960in}}%
\pgfpathlineto{\pgfqpoint{2.225910in}{1.342060in}}%
\pgfpathlineto{\pgfqpoint{2.212786in}{1.346920in}}%
\pgfpathlineto{\pgfqpoint{2.196031in}{1.353141in}}%
\pgfpathlineto{\pgfqpoint{2.185829in}{1.356879in}}%
\pgfpathlineto{\pgfqpoint{2.166153in}{1.364111in}}%
\pgfpathlineto{\pgfqpoint{2.158626in}{1.366839in}}%
\pgfpathlineto{\pgfqpoint{2.136274in}{1.374969in}}%
\pgfpathlineto{\pgfqpoint{2.131169in}{1.376798in}}%
\pgfpathlineto{\pgfqpoint{2.106395in}{1.385712in}}%
\pgfpathlineto{\pgfqpoint{2.103445in}{1.386758in}}%
\pgfpathlineto{\pgfqpoint{2.076517in}{1.396342in}}%
\pgfpathlineto{\pgfqpoint{2.075443in}{1.396718in}}%
\pgfpathlineto{\pgfqpoint{2.047139in}{1.406677in}}%
\pgfpathlineto{\pgfqpoint{2.046638in}{1.406854in}}%
\pgfpathlineto{\pgfqpoint{2.018509in}{1.416637in}}%
\pgfpathlineto{\pgfqpoint{2.016759in}{1.417248in}}%
\pgfpathlineto{\pgfqpoint{1.989562in}{1.426596in}}%
\pgfpathlineto{\pgfqpoint{1.986880in}{1.427523in}}%
\pgfpathlineto{\pgfqpoint{1.960281in}{1.436556in}}%
\pgfpathlineto{\pgfqpoint{1.957002in}{1.437676in}}%
\pgfpathlineto{\pgfqpoint{1.930649in}{1.446515in}}%
\pgfpathlineto{\pgfqpoint{1.927123in}{1.447705in}}%
\pgfpathlineto{\pgfqpoint{1.900646in}{1.456475in}}%
\pgfpathlineto{\pgfqpoint{1.897244in}{1.457609in}}%
\pgfpathlineto{\pgfqpoint{1.870251in}{1.466434in}}%
\pgfpathlineto{\pgfqpoint{1.867366in}{1.467384in}}%
\pgfpathlineto{\pgfqpoint{1.839440in}{1.476394in}}%
\pgfpathlineto{\pgfqpoint{1.837487in}{1.477029in}}%
\pgfpathlineto{\pgfqpoint{1.808189in}{1.486354in}}%
\pgfpathlineto{\pgfqpoint{1.807608in}{1.486540in}}%
\pgfpathlineto{\pgfqpoint{1.777730in}{1.495912in}}%
\pgfpathlineto{\pgfqpoint{1.776420in}{1.496313in}}%
\pgfpathlineto{\pgfqpoint{1.747851in}{1.505142in}}%
\pgfpathlineto{\pgfqpoint{1.744102in}{1.506273in}}%
\pgfpathlineto{\pgfqpoint{1.717972in}{1.514227in}}%
\pgfpathlineto{\pgfqpoint{1.711216in}{1.516232in}}%
\pgfpathlineto{\pgfqpoint{1.688094in}{1.523163in}}%
\pgfpathlineto{\pgfqpoint{1.677721in}{1.526192in}}%
\pgfpathlineto{\pgfqpoint{1.658215in}{1.531945in}}%
\pgfpathlineto{\pgfqpoint{1.643568in}{1.536151in}}%
\pgfpathlineto{\pgfqpoint{1.628336in}{1.540571in}}%
\pgfpathlineto{\pgfqpoint{1.608704in}{1.546111in}}%
\pgfpathlineto{\pgfqpoint{1.598458in}{1.549034in}}%
\pgfpathlineto{\pgfqpoint{1.573070in}{1.556070in}}%
\pgfpathlineto{\pgfqpoint{1.568579in}{1.557329in}}%
\pgfpathlineto{\pgfqpoint{1.538700in}{1.565446in}}%
\pgfpathlineto{\pgfqpoint{1.536473in}{1.566030in}}%
\pgfpathlineto{\pgfqpoint{1.508822in}{1.573368in}}%
\pgfpathlineto{\pgfqpoint{1.498606in}{1.575990in}}%
\pgfpathlineto{\pgfqpoint{1.478943in}{1.581101in}}%
\pgfpathlineto{\pgfqpoint{1.459636in}{1.585949in}}%
\pgfpathlineto{\pgfqpoint{1.449064in}{1.588639in}}%
\pgfpathlineto{\pgfqpoint{1.419452in}{1.595909in}}%
\pgfpathlineto{\pgfqpoint{1.419186in}{1.595975in}}%
\pgfpathlineto{\pgfqpoint{1.389307in}{1.603059in}}%
\pgfpathlineto{\pgfqpoint{1.376957in}{1.605868in}}%
\pgfpathlineto{\pgfqpoint{1.359428in}{1.609917in}}%
\pgfpathlineto{\pgfqpoint{1.332728in}{1.615828in}}%
\pgfpathlineto{\pgfqpoint{1.329550in}{1.616542in}}%
\pgfpathlineto{\pgfqpoint{1.299671in}{1.622870in}}%
\pgfpathlineto{\pgfqpoint{1.285143in}{1.625787in}}%
\pgfpathlineto{\pgfqpoint{1.269792in}{1.628922in}}%
\pgfpathlineto{\pgfqpoint{1.239914in}{1.634678in}}%
\pgfpathlineto{\pgfqpoint{1.233902in}{1.635747in}}%
\pgfpathlineto{\pgfqpoint{1.210035in}{1.640067in}}%
\pgfpathlineto{\pgfqpoint{1.180156in}{1.645131in}}%
\pgfpathlineto{\pgfqpoint{1.176358in}{1.645706in}}%
\pgfpathlineto{\pgfqpoint{1.150277in}{1.649739in}}%
\pgfpathlineto{\pgfqpoint{1.120399in}{1.653938in}}%
\pgfpathlineto{\pgfqpoint{1.106210in}{1.655666in}}%
\pgfpathlineto{\pgfqpoint{1.090520in}{1.657618in}}%
\pgfpathlineto{\pgfqpoint{1.060641in}{1.660707in}}%
\pgfpathlineto{\pgfqpoint{1.030763in}{1.663165in}}%
\pgfpathlineto{\pgfqpoint{1.000884in}{1.664859in}}%
\pgfpathlineto{\pgfqpoint{0.971005in}{1.665609in}}%
\pgfpathlineto{\pgfqpoint{0.941127in}{1.665164in}}%
\pgfpathlineto{\pgfqpoint{0.911248in}{1.663166in}}%
\pgfpathlineto{\pgfqpoint{0.881369in}{1.659095in}}%
\pgfpathlineto{\pgfqpoint{0.865680in}{1.655666in}}%
\pgfpathlineto{\pgfqpoint{0.851491in}{1.651799in}}%
\pgfpathlineto{\pgfqpoint{0.835712in}{1.645706in}}%
\pgfpathlineto{\pgfqpoint{0.821612in}{1.638431in}}%
\pgfpathlineto{\pgfqpoint{0.817574in}{1.635747in}}%
\pgfpathlineto{\pgfqpoint{0.806261in}{1.625787in}}%
\pgfpathlineto{\pgfqpoint{0.798639in}{1.615828in}}%
\pgfpathlineto{\pgfqpoint{0.793897in}{1.605868in}}%
\pgfpathlineto{\pgfqpoint{0.791733in}{1.597154in}}%
\pgfpathlineto{\pgfqpoint{0.791467in}{1.595909in}}%
\pgfpathlineto{\pgfqpoint{0.790908in}{1.585949in}}%
\pgfpathlineto{\pgfqpoint{0.791733in}{1.576404in}}%
\pgfpathlineto{\pgfqpoint{0.791770in}{1.575990in}}%
\pgfpathlineto{\pgfqpoint{0.793961in}{1.566030in}}%
\pgfpathlineto{\pgfqpoint{0.797223in}{1.556070in}}%
\pgfpathlineto{\pgfqpoint{0.801404in}{1.546111in}}%
\pgfpathlineto{\pgfqpoint{0.806380in}{1.536151in}}%
\pgfpathlineto{\pgfqpoint{0.812057in}{1.526192in}}%
\pgfpathlineto{\pgfqpoint{0.818356in}{1.516232in}}%
\pgfpathlineto{\pgfqpoint{0.821612in}{1.511595in}}%
\pgfpathlineto{\pgfqpoint{0.825362in}{1.506273in}}%
\pgfpathlineto{\pgfqpoint{0.832991in}{1.496313in}}%
\pgfpathlineto{\pgfqpoint{0.841061in}{1.486354in}}%
\pgfpathlineto{\pgfqpoint{0.849538in}{1.476394in}}%
\pgfpathlineto{\pgfqpoint{0.851491in}{1.474249in}}%
\pgfpathlineto{\pgfqpoint{0.858612in}{1.466434in}}%
\pgfpathlineto{\pgfqpoint{0.868074in}{1.456475in}}%
\pgfpathlineto{\pgfqpoint{0.877844in}{1.446515in}}%
\pgfpathlineto{\pgfqpoint{0.881369in}{1.443086in}}%
\pgfpathlineto{\pgfqpoint{0.888082in}{1.436556in}}%
\pgfpathlineto{\pgfqpoint{0.898668in}{1.426596in}}%
\pgfpathlineto{\pgfqpoint{0.909498in}{1.416637in}}%
\pgfpathlineto{\pgfqpoint{0.911248in}{1.415092in}}%
\pgfpathlineto{\pgfqpoint{0.920775in}{1.406677in}}%
\pgfpathlineto{\pgfqpoint{0.932294in}{1.396718in}}%
\pgfpathlineto{\pgfqpoint{0.941127in}{1.389255in}}%
\pgfpathlineto{\pgfqpoint{0.944077in}{1.386758in}}%
\pgfpathlineto{\pgfqpoint{0.956221in}{1.376798in}}%
\pgfpathlineto{\pgfqpoint{0.968540in}{1.366839in}}%
\pgfpathlineto{\pgfqpoint{0.971005in}{1.364904in}}%
\pgfpathlineto{\pgfqpoint{0.981208in}{1.356879in}}%
\pgfpathlineto{\pgfqpoint{0.994071in}{1.346920in}}%
\pgfpathlineto{\pgfqpoint{1.000884in}{1.341753in}}%
\pgfpathlineto{\pgfqpoint{1.007188in}{1.336960in}}%
\pgfpathlineto{\pgfqpoint{1.020552in}{1.327001in}}%
\pgfpathlineto{\pgfqpoint{1.030763in}{1.319502in}}%
\pgfpathlineto{\pgfqpoint{1.034105in}{1.317041in}}%
\pgfpathlineto{\pgfqpoint{1.047930in}{1.307081in}}%
\pgfpathlineto{\pgfqpoint{1.060641in}{1.298024in}}%
\pgfpathlineto{\pgfqpoint{1.061905in}{1.297122in}}%
\pgfpathlineto{\pgfqpoint{1.076159in}{1.287162in}}%
\pgfpathlineto{\pgfqpoint{1.090520in}{1.277216in}}%
\pgfpathlineto{\pgfqpoint{1.090539in}{1.277203in}}%
\pgfpathlineto{\pgfqpoint{1.105194in}{1.267243in}}%
\pgfpathlineto{\pgfqpoint{1.119970in}{1.257284in}}%
\pgfpathlineto{\pgfqpoint{1.120399in}{1.257000in}}%
\pgfpathlineto{\pgfqpoint{1.134998in}{1.247324in}}%
\pgfpathlineto{\pgfqpoint{1.150149in}{1.237365in}}%
\pgfpathlineto{\pgfqpoint{1.150277in}{1.237282in}}%
\pgfpathlineto{\pgfqpoint{1.165537in}{1.227405in}}%
\pgfpathlineto{\pgfqpoint{1.180156in}{1.218021in}}%
\pgfpathlineto{\pgfqpoint{1.181050in}{1.217445in}}%
\pgfpathlineto{\pgfqpoint{1.196781in}{1.207486in}}%
\pgfpathlineto{\pgfqpoint{1.210035in}{1.199169in}}%
\pgfpathlineto{\pgfqpoint{1.212644in}{1.197526in}}%
\pgfpathlineto{\pgfqpoint{1.228702in}{1.187567in}}%
\pgfpathlineto{\pgfqpoint{1.239914in}{1.180681in}}%
\pgfpathlineto{\pgfqpoint{1.244904in}{1.177607in}}%
\pgfpathlineto{\pgfqpoint{1.261278in}{1.167648in}}%
\pgfpathlineto{\pgfqpoint{1.269792in}{1.162525in}}%
\pgfpathlineto{\pgfqpoint{1.277807in}{1.157688in}}%
\pgfpathlineto{\pgfqpoint{1.294489in}{1.147729in}}%
\pgfpathlineto{\pgfqpoint{1.299671in}{1.144670in}}%
\pgfpathlineto{\pgfqpoint{1.311334in}{1.137769in}}%
\pgfpathlineto{\pgfqpoint{1.328316in}{1.127809in}}%
\pgfpathlineto{\pgfqpoint{1.329550in}{1.127095in}}%
\pgfpathlineto{\pgfqpoint{1.345468in}{1.117850in}}%
\pgfpathlineto{\pgfqpoint{1.359428in}{1.109810in}}%
\pgfpathlineto{\pgfqpoint{1.362754in}{1.107890in}}%
\pgfpathlineto{\pgfqpoint{1.380196in}{1.097931in}}%
\pgfpathlineto{\pgfqpoint{1.389307in}{1.092776in}}%
\pgfpathlineto{\pgfqpoint{1.397780in}{1.087971in}}%
\pgfpathlineto{\pgfqpoint{1.415506in}{1.078012in}}%
\pgfpathlineto{\pgfqpoint{1.419186in}{1.075965in}}%
\pgfpathlineto{\pgfqpoint{1.433381in}{1.068052in}}%
\pgfpathlineto{\pgfqpoint{1.449064in}{1.059381in}}%
\pgfpathlineto{\pgfqpoint{1.451390in}{1.058092in}}%
\pgfpathlineto{\pgfqpoint{1.469546in}{1.048133in}}%
\pgfpathlineto{\pgfqpoint{1.478943in}{1.043021in}}%
\pgfpathlineto{\pgfqpoint{1.487839in}{1.038173in}}%
\pgfpathlineto{\pgfqpoint{1.506269in}{1.028214in}}%
\pgfpathlineto{\pgfqpoint{1.508822in}{1.026847in}}%
\pgfpathlineto{\pgfqpoint{1.524841in}{1.018254in}}%
\pgfpathlineto{\pgfqpoint{1.538700in}{1.010878in}}%
\pgfpathlineto{\pgfqpoint{1.543547in}{1.008295in}}%
\pgfpathlineto{\pgfqpoint{1.562390in}{0.998335in}}%
\pgfpathlineto{\pgfqpoint{1.568579in}{0.995090in}}%
\pgfpathlineto{\pgfqpoint{1.581369in}{0.988376in}}%
\pgfpathlineto{\pgfqpoint{1.598458in}{0.979472in}}%
\pgfpathlineto{\pgfqpoint{1.600483in}{0.978416in}}%
\pgfpathlineto{\pgfqpoint{1.619732in}{0.968456in}}%
\pgfpathlineto{\pgfqpoint{1.628336in}{0.964037in}}%
\pgfpathlineto{\pgfqpoint{1.639115in}{0.958497in}}%
\pgfpathlineto{\pgfqpoint{1.658215in}{0.948751in}}%
\pgfpathlineto{\pgfqpoint{1.658633in}{0.948537in}}%
\pgfpathlineto{\pgfqpoint{1.678285in}{0.938578in}}%
\pgfpathlineto{\pgfqpoint{1.688094in}{0.933641in}}%
\pgfpathlineto{\pgfqpoint{1.698070in}{0.928618in}}%
\pgfpathlineto{\pgfqpoint{1.717972in}{0.918667in}}%
\pgfpathlineto{\pgfqpoint{1.717989in}{0.918659in}}%
\pgfpathlineto{\pgfqpoint{1.738042in}{0.908699in}}%
\pgfpathlineto{\pgfqpoint{1.747851in}{0.903859in}}%
\pgfpathlineto{\pgfqpoint{1.758229in}{0.898740in}}%
\pgfpathlineto{\pgfqpoint{1.777730in}{0.889181in}}%
\pgfpathlineto{\pgfqpoint{1.778549in}{0.888780in}}%
\pgfpathlineto{\pgfqpoint{1.799004in}{0.878820in}}%
\pgfpathlineto{\pgfqpoint{1.807608in}{0.874656in}}%
\pgfpathlineto{\pgfqpoint{1.819594in}{0.868861in}}%
\pgfpathlineto{\pgfqpoint{1.837487in}{0.860260in}}%
\pgfpathlineto{\pgfqpoint{1.840317in}{0.858901in}}%
\pgfpathlineto{\pgfqpoint{1.861177in}{0.848942in}}%
\pgfpathlineto{\pgfqpoint{1.867366in}{0.846003in}}%
\pgfpathlineto{\pgfqpoint{1.882173in}{0.838982in}}%
\pgfpathlineto{\pgfqpoint{1.897244in}{0.831875in}}%
\pgfpathlineto{\pgfqpoint{1.903303in}{0.829023in}}%
\pgfpathlineto{\pgfqpoint{1.924571in}{0.819063in}}%
\pgfpathlineto{\pgfqpoint{1.927123in}{0.817873in}}%
\pgfpathlineto{\pgfqpoint{1.945979in}{0.809103in}}%
\pgfpathlineto{\pgfqpoint{1.957002in}{0.804002in}}%
\pgfpathlineto{\pgfqpoint{1.967525in}{0.799144in}}%
\pgfpathlineto{\pgfqpoint{1.986880in}{0.790250in}}%
\pgfpathlineto{\pgfqpoint{1.989206in}{0.789184in}}%
\pgfpathlineto{\pgfqpoint{2.011033in}{0.779225in}}%
\pgfpathlineto{\pgfqpoint{2.016759in}{0.776623in}}%
\pgfpathlineto{\pgfqpoint{2.033002in}{0.769265in}}%
\pgfpathlineto{\pgfqpoint{2.046638in}{0.763114in}}%
\pgfpathlineto{\pgfqpoint{2.055111in}{0.759306in}}%
\pgfpathlineto{\pgfqpoint{2.076517in}{0.749722in}}%
\pgfpathlineto{\pgfqpoint{2.077359in}{0.749346in}}%
\pgfpathlineto{\pgfqpoint{2.099764in}{0.739387in}}%
\pgfpathlineto{\pgfqpoint{2.106395in}{0.736449in}}%
\pgfpathlineto{\pgfqpoint{2.122314in}{0.729427in}}%
\pgfpathlineto{\pgfqpoint{2.136274in}{0.723290in}}%
\pgfpathlineto{\pgfqpoint{2.145008in}{0.719467in}}%
\pgfpathlineto{\pgfqpoint{2.166153in}{0.710243in}}%
\pgfpathlineto{\pgfqpoint{2.167846in}{0.709508in}}%
\pgfpathlineto{\pgfqpoint{2.190849in}{0.699548in}}%
\pgfpathlineto{\pgfqpoint{2.196031in}{0.697310in}}%
\pgfpathlineto{\pgfqpoint{2.214008in}{0.689589in}}%
\pgfpathlineto{\pgfqpoint{2.225910in}{0.684489in}}%
\pgfpathlineto{\pgfqpoint{2.237318in}{0.679629in}}%
\pgfpathlineto{\pgfqpoint{2.255789in}{0.671778in}}%
\pgfpathlineto{\pgfqpoint{2.260779in}{0.669670in}}%
\pgfpathlineto{\pgfqpoint{2.284400in}{0.659710in}}%
\pgfpathlineto{\pgfqpoint{2.285667in}{0.659177in}}%
\pgfpathlineto{\pgfqpoint{2.308205in}{0.649751in}}%
\pgfpathlineto{\pgfqpoint{2.315546in}{0.646685in}}%
\pgfpathlineto{\pgfqpoint{2.332171in}{0.639791in}}%
\pgfpathlineto{\pgfqpoint{2.345425in}{0.634302in}}%
\pgfpathlineto{\pgfqpoint{2.356299in}{0.629831in}}%
\pgfpathlineto{\pgfqpoint{2.375303in}{0.622027in}}%
\pgfpathlineto{\pgfqpoint{2.380593in}{0.619872in}}%
\pgfpathlineto{\pgfqpoint{2.405053in}{0.609912in}}%
\pgfpathlineto{\pgfqpoint{2.405182in}{0.609860in}}%
\pgfpathlineto{\pgfqpoint{2.429728in}{0.599953in}}%
\pgfpathlineto{\pgfqpoint{2.435061in}{0.597801in}}%
\pgfpathlineto{\pgfqpoint{2.454580in}{0.589993in}}%
\pgfpathlineto{\pgfqpoint{2.464939in}{0.585849in}}%
\pgfpathlineto{\pgfqpoint{2.479613in}{0.580034in}}%
\pgfpathlineto{\pgfqpoint{2.494818in}{0.574006in}}%
\pgfpathlineto{\pgfqpoint{2.504832in}{0.570074in}}%
\pgfpathlineto{\pgfqpoint{2.524697in}{0.562270in}}%
\pgfpathlineto{\pgfqpoint{2.530239in}{0.560115in}}%
\pgfpathlineto{\pgfqpoint{2.554575in}{0.550641in}}%
\pgfpathlineto{\pgfqpoint{2.555839in}{0.550155in}}%
\pgfpathlineto{\pgfqpoint{2.581672in}{0.540195in}}%
\pgfpathlineto{\pgfqpoint{2.584454in}{0.539122in}}%
\pgfpathlineto{\pgfqpoint{2.607729in}{0.530236in}}%
\pgfpathlineto{\pgfqpoint{2.614333in}{0.527711in}}%
\pgfpathlineto{\pgfqpoint{2.634000in}{0.520276in}}%
\pgfpathlineto{\pgfqpoint{2.644211in}{0.516409in}}%
\pgfpathlineto{\pgfqpoint{2.660493in}{0.510317in}}%
\pgfpathlineto{\pgfqpoint{2.674090in}{0.505217in}}%
\pgfpathlineto{\pgfqpoint{2.687214in}{0.500357in}}%
\pgfpathlineto{\pgfqpoint{2.703969in}{0.494136in}}%
\pgfpathlineto{\pgfqpoint{2.714171in}{0.490398in}}%
\pgfpathlineto{\pgfqpoint{2.733847in}{0.483166in}}%
\pgfpathlineto{\pgfqpoint{2.741374in}{0.480438in}}%
\pgfpathlineto{\pgfqpoint{2.763726in}{0.472308in}}%
\pgfpathlineto{\pgfqpoint{2.768831in}{0.470478in}}%
\pgfpathlineto{\pgfqpoint{2.793605in}{0.461564in}}%
\pgfpathlineto{\pgfqpoint{2.796555in}{0.460519in}}%
\pgfpathlineto{\pgfqpoint{2.823483in}{0.450935in}}%
\pgfpathlineto{\pgfqpoint{2.824557in}{0.450559in}}%
\pgfpathlineto{\pgfqpoint{2.852861in}{0.440600in}}%
\pgfpathlineto{\pgfqpoint{2.853362in}{0.440423in}}%
\pgfpathlineto{\pgfqpoint{2.881491in}{0.430640in}}%
\pgfpathlineto{\pgfqpoint{2.883241in}{0.430029in}}%
\pgfpathlineto{\pgfqpoint{2.910438in}{0.420681in}}%
\pgfpathlineto{\pgfqpoint{2.913120in}{0.419754in}}%
\pgfpathlineto{\pgfqpoint{2.939719in}{0.410721in}}%
\pgfpathlineto{\pgfqpoint{2.942998in}{0.409601in}}%
\pgfpathlineto{\pgfqpoint{2.969351in}{0.400762in}}%
\pgfpathlineto{\pgfqpoint{2.972877in}{0.399572in}}%
\pgfpathlineto{\pgfqpoint{2.999354in}{0.390802in}}%
\pgfpathlineto{\pgfqpoint{3.002756in}{0.389668in}}%
\pgfpathlineto{\pgfqpoint{3.029749in}{0.380842in}}%
\pgfpathlineto{\pgfqpoint{3.032634in}{0.379893in}}%
\pgfpathlineto{\pgfqpoint{3.060560in}{0.370883in}}%
\pgfpathlineto{\pgfqpoint{3.062513in}{0.370248in}}%
\pgfpathlineto{\pgfqpoint{3.091811in}{0.360923in}}%
\pgfpathlineto{\pgfqpoint{3.092392in}{0.360737in}}%
\pgfpathlineto{\pgfqpoint{3.122270in}{0.351365in}}%
\pgfpathlineto{\pgfqpoint{3.123580in}{0.350964in}}%
\pgfpathlineto{\pgfqpoint{3.152149in}{0.342135in}}%
\pgfpathlineto{\pgfqpoint{3.155898in}{0.341004in}}%
\pgfpathlineto{\pgfqpoint{3.182028in}{0.333050in}}%
\pgfpathlineto{\pgfqpoint{3.188784in}{0.331045in}}%
\pgfpathlineto{\pgfqpoint{3.211906in}{0.324114in}}%
\pgfpathlineto{\pgfqpoint{3.222279in}{0.321085in}}%
\pgfpathlineto{\pgfqpoint{3.241785in}{0.315331in}}%
\pgfpathlineto{\pgfqpoint{3.256432in}{0.311126in}}%
\pgfpathlineto{\pgfqpoint{3.271664in}{0.306706in}}%
\pgfpathlineto{\pgfqpoint{3.291296in}{0.301166in}}%
\pgfpathlineto{\pgfqpoint{3.301542in}{0.298243in}}%
\pgfpathlineto{\pgfqpoint{3.326930in}{0.291206in}}%
\pgfpathlineto{\pgfqpoint{3.331421in}{0.289948in}}%
\pgfpathlineto{\pgfqpoint{3.361300in}{0.281831in}}%
\pgfpathlineto{\pgfqpoint{3.363527in}{0.281247in}}%
\pgfpathlineto{\pgfqpoint{3.391178in}{0.273909in}}%
\pgfpathlineto{\pgfqpoint{3.401394in}{0.271287in}}%
\pgfpathlineto{\pgfqpoint{3.421057in}{0.266176in}}%
\pgfpathlineto{\pgfqpoint{3.440364in}{0.261328in}}%
\pgfpathlineto{\pgfqpoint{3.450936in}{0.258638in}}%
\pgfpathlineto{\pgfqpoint{3.480548in}{0.251368in}}%
\pgfpathlineto{\pgfqpoint{3.480814in}{0.251302in}}%
\pgfpathlineto{\pgfqpoint{3.510693in}{0.244218in}}%
\pgfpathlineto{\pgfqpoint{3.523043in}{0.241409in}}%
\pgfpathlineto{\pgfqpoint{3.540572in}{0.237360in}}%
\pgfpathlineto{\pgfqpoint{3.567272in}{0.231449in}}%
\pgfpathlineto{\pgfqpoint{3.570450in}{0.230735in}}%
\pgfpathlineto{\pgfqpoint{3.600329in}{0.224407in}}%
\pgfpathlineto{\pgfqpoint{3.614857in}{0.221489in}}%
\pgfpathlineto{\pgfqpoint{3.630208in}{0.218354in}}%
\pgfpathlineto{\pgfqpoint{3.660086in}{0.212598in}}%
\pgfpathlineto{\pgfqpoint{3.666098in}{0.211530in}}%
\pgfpathlineto{\pgfqpoint{3.689965in}{0.207210in}}%
\pgfpathlineto{\pgfqpoint{3.719844in}{0.202146in}}%
\pgfpathlineto{\pgfqpoint{3.723642in}{0.201570in}}%
\pgfpathlineto{\pgfqpoint{3.749723in}{0.197538in}}%
\pgfpathlineto{\pgfqpoint{3.779601in}{0.193338in}}%
\pgfpathlineto{\pgfqpoint{3.793790in}{0.191611in}}%
\pgfpathclose%
\pgfpathmoveto{\pgfqpoint{3.793790in}{0.191611in}}%
\pgfpathlineto{\pgfqpoint{3.779601in}{0.193338in}}%
\pgfpathlineto{\pgfqpoint{3.749723in}{0.197538in}}%
\pgfpathlineto{\pgfqpoint{3.723642in}{0.201570in}}%
\pgfpathlineto{\pgfqpoint{3.719844in}{0.202146in}}%
\pgfpathlineto{\pgfqpoint{3.689965in}{0.207210in}}%
\pgfpathlineto{\pgfqpoint{3.666098in}{0.211530in}}%
\pgfpathlineto{\pgfqpoint{3.660086in}{0.212598in}}%
\pgfpathlineto{\pgfqpoint{3.630208in}{0.218354in}}%
\pgfpathlineto{\pgfqpoint{3.614857in}{0.221489in}}%
\pgfpathlineto{\pgfqpoint{3.600329in}{0.224407in}}%
\pgfpathlineto{\pgfqpoint{3.570450in}{0.230735in}}%
\pgfpathlineto{\pgfqpoint{3.567272in}{0.231449in}}%
\pgfpathlineto{\pgfqpoint{3.540572in}{0.237360in}}%
\pgfpathlineto{\pgfqpoint{3.523043in}{0.241409in}}%
\pgfpathlineto{\pgfqpoint{3.510693in}{0.244218in}}%
\pgfpathlineto{\pgfqpoint{3.480814in}{0.251302in}}%
\pgfpathlineto{\pgfqpoint{3.480548in}{0.251368in}}%
\pgfpathlineto{\pgfqpoint{3.450936in}{0.258638in}}%
\pgfpathlineto{\pgfqpoint{3.440364in}{0.261328in}}%
\pgfpathlineto{\pgfqpoint{3.421057in}{0.266176in}}%
\pgfpathlineto{\pgfqpoint{3.401394in}{0.271287in}}%
\pgfpathlineto{\pgfqpoint{3.391178in}{0.273909in}}%
\pgfpathlineto{\pgfqpoint{3.363527in}{0.281247in}}%
\pgfpathlineto{\pgfqpoint{3.361300in}{0.281831in}}%
\pgfpathlineto{\pgfqpoint{3.331421in}{0.289948in}}%
\pgfpathlineto{\pgfqpoint{3.326930in}{0.291206in}}%
\pgfpathlineto{\pgfqpoint{3.301542in}{0.298243in}}%
\pgfpathlineto{\pgfqpoint{3.291296in}{0.301166in}}%
\pgfpathlineto{\pgfqpoint{3.271664in}{0.306706in}}%
\pgfpathlineto{\pgfqpoint{3.256432in}{0.311126in}}%
\pgfpathlineto{\pgfqpoint{3.241785in}{0.315331in}}%
\pgfpathlineto{\pgfqpoint{3.222279in}{0.321085in}}%
\pgfpathlineto{\pgfqpoint{3.211906in}{0.324114in}}%
\pgfpathlineto{\pgfqpoint{3.188784in}{0.331045in}}%
\pgfpathlineto{\pgfqpoint{3.182028in}{0.333050in}}%
\pgfpathlineto{\pgfqpoint{3.155898in}{0.341004in}}%
\pgfpathlineto{\pgfqpoint{3.152149in}{0.342135in}}%
\pgfpathlineto{\pgfqpoint{3.123580in}{0.350964in}}%
\pgfpathlineto{\pgfqpoint{3.122270in}{0.351365in}}%
\pgfpathlineto{\pgfqpoint{3.092392in}{0.360737in}}%
\pgfpathlineto{\pgfqpoint{3.091811in}{0.360923in}}%
\pgfpathlineto{\pgfqpoint{3.062513in}{0.370248in}}%
\pgfpathlineto{\pgfqpoint{3.060560in}{0.370883in}}%
\pgfpathlineto{\pgfqpoint{3.032634in}{0.379893in}}%
\pgfpathlineto{\pgfqpoint{3.029749in}{0.380842in}}%
\pgfpathlineto{\pgfqpoint{3.002756in}{0.389668in}}%
\pgfpathlineto{\pgfqpoint{2.999354in}{0.390802in}}%
\pgfpathlineto{\pgfqpoint{2.972877in}{0.399572in}}%
\pgfpathlineto{\pgfqpoint{2.969351in}{0.400762in}}%
\pgfpathlineto{\pgfqpoint{2.942998in}{0.409601in}}%
\pgfpathlineto{\pgfqpoint{2.939719in}{0.410721in}}%
\pgfpathlineto{\pgfqpoint{2.913120in}{0.419754in}}%
\pgfpathlineto{\pgfqpoint{2.910438in}{0.420681in}}%
\pgfpathlineto{\pgfqpoint{2.883241in}{0.430029in}}%
\pgfpathlineto{\pgfqpoint{2.881491in}{0.430640in}}%
\pgfpathlineto{\pgfqpoint{2.853362in}{0.440423in}}%
\pgfpathlineto{\pgfqpoint{2.852861in}{0.440600in}}%
\pgfpathlineto{\pgfqpoint{2.824557in}{0.450559in}}%
\pgfpathlineto{\pgfqpoint{2.823483in}{0.450935in}}%
\pgfpathlineto{\pgfqpoint{2.796555in}{0.460519in}}%
\pgfpathlineto{\pgfqpoint{2.793605in}{0.461564in}}%
\pgfpathlineto{\pgfqpoint{2.768831in}{0.470478in}}%
\pgfpathlineto{\pgfqpoint{2.763726in}{0.472308in}}%
\pgfpathlineto{\pgfqpoint{2.741374in}{0.480438in}}%
\pgfpathlineto{\pgfqpoint{2.733847in}{0.483166in}}%
\pgfpathlineto{\pgfqpoint{2.714171in}{0.490398in}}%
\pgfpathlineto{\pgfqpoint{2.703969in}{0.494136in}}%
\pgfpathlineto{\pgfqpoint{2.687214in}{0.500357in}}%
\pgfpathlineto{\pgfqpoint{2.674090in}{0.505217in}}%
\pgfpathlineto{\pgfqpoint{2.660493in}{0.510317in}}%
\pgfpathlineto{\pgfqpoint{2.644211in}{0.516409in}}%
\pgfpathlineto{\pgfqpoint{2.634000in}{0.520276in}}%
\pgfpathlineto{\pgfqpoint{2.614333in}{0.527711in}}%
\pgfpathlineto{\pgfqpoint{2.607729in}{0.530236in}}%
\pgfpathlineto{\pgfqpoint{2.584454in}{0.539122in}}%
\pgfpathlineto{\pgfqpoint{2.581672in}{0.540195in}}%
\pgfpathlineto{\pgfqpoint{2.555839in}{0.550155in}}%
\pgfpathlineto{\pgfqpoint{2.554575in}{0.550641in}}%
\pgfpathlineto{\pgfqpoint{2.530239in}{0.560115in}}%
\pgfpathlineto{\pgfqpoint{2.524697in}{0.562270in}}%
\pgfpathlineto{\pgfqpoint{2.504832in}{0.570074in}}%
\pgfpathlineto{\pgfqpoint{2.494818in}{0.574006in}}%
\pgfpathlineto{\pgfqpoint{2.479613in}{0.580034in}}%
\pgfpathlineto{\pgfqpoint{2.464939in}{0.585849in}}%
\pgfpathlineto{\pgfqpoint{2.454580in}{0.589993in}}%
\pgfpathlineto{\pgfqpoint{2.435061in}{0.597801in}}%
\pgfpathlineto{\pgfqpoint{2.429728in}{0.599953in}}%
\pgfpathlineto{\pgfqpoint{2.405182in}{0.609860in}}%
\pgfpathlineto{\pgfqpoint{2.405053in}{0.609912in}}%
\pgfpathlineto{\pgfqpoint{2.380593in}{0.619872in}}%
\pgfpathlineto{\pgfqpoint{2.375303in}{0.622027in}}%
\pgfpathlineto{\pgfqpoint{2.356299in}{0.629831in}}%
\pgfpathlineto{\pgfqpoint{2.345425in}{0.634302in}}%
\pgfpathlineto{\pgfqpoint{2.332171in}{0.639791in}}%
\pgfpathlineto{\pgfqpoint{2.315546in}{0.646685in}}%
\pgfpathlineto{\pgfqpoint{2.308205in}{0.649751in}}%
\pgfpathlineto{\pgfqpoint{2.285667in}{0.659177in}}%
\pgfpathlineto{\pgfqpoint{2.284400in}{0.659710in}}%
\pgfpathlineto{\pgfqpoint{2.260779in}{0.669670in}}%
\pgfpathlineto{\pgfqpoint{2.255789in}{0.671778in}}%
\pgfpathlineto{\pgfqpoint{2.237318in}{0.679629in}}%
\pgfpathlineto{\pgfqpoint{2.225910in}{0.684489in}}%
\pgfpathlineto{\pgfqpoint{2.214008in}{0.689589in}}%
\pgfpathlineto{\pgfqpoint{2.196031in}{0.697310in}}%
\pgfpathlineto{\pgfqpoint{2.190849in}{0.699548in}}%
\pgfpathlineto{\pgfqpoint{2.167846in}{0.709508in}}%
\pgfpathlineto{\pgfqpoint{2.166153in}{0.710243in}}%
\pgfpathlineto{\pgfqpoint{2.145008in}{0.719467in}}%
\pgfpathlineto{\pgfqpoint{2.136274in}{0.723290in}}%
\pgfpathlineto{\pgfqpoint{2.122314in}{0.729427in}}%
\pgfpathlineto{\pgfqpoint{2.106395in}{0.736449in}}%
\pgfpathlineto{\pgfqpoint{2.099764in}{0.739387in}}%
\pgfpathlineto{\pgfqpoint{2.077359in}{0.749346in}}%
\pgfpathlineto{\pgfqpoint{2.076517in}{0.749722in}}%
\pgfpathlineto{\pgfqpoint{2.055111in}{0.759306in}}%
\pgfpathlineto{\pgfqpoint{2.046638in}{0.763114in}}%
\pgfpathlineto{\pgfqpoint{2.033002in}{0.769265in}}%
\pgfpathlineto{\pgfqpoint{2.016759in}{0.776623in}}%
\pgfpathlineto{\pgfqpoint{2.011033in}{0.779225in}}%
\pgfpathlineto{\pgfqpoint{1.989206in}{0.789184in}}%
\pgfpathlineto{\pgfqpoint{1.986880in}{0.790250in}}%
\pgfpathlineto{\pgfqpoint{1.967525in}{0.799144in}}%
\pgfpathlineto{\pgfqpoint{1.957002in}{0.804002in}}%
\pgfpathlineto{\pgfqpoint{1.945979in}{0.809103in}}%
\pgfpathlineto{\pgfqpoint{1.927123in}{0.817873in}}%
\pgfpathlineto{\pgfqpoint{1.924571in}{0.819063in}}%
\pgfpathlineto{\pgfqpoint{1.903303in}{0.829023in}}%
\pgfpathlineto{\pgfqpoint{1.897244in}{0.831875in}}%
\pgfpathlineto{\pgfqpoint{1.882173in}{0.838982in}}%
\pgfpathlineto{\pgfqpoint{1.867366in}{0.846003in}}%
\pgfpathlineto{\pgfqpoint{1.861177in}{0.848942in}}%
\pgfpathlineto{\pgfqpoint{1.840317in}{0.858901in}}%
\pgfpathlineto{\pgfqpoint{1.837487in}{0.860260in}}%
\pgfpathlineto{\pgfqpoint{1.819594in}{0.868861in}}%
\pgfpathlineto{\pgfqpoint{1.807608in}{0.874656in}}%
\pgfpathlineto{\pgfqpoint{1.799004in}{0.878820in}}%
\pgfpathlineto{\pgfqpoint{1.778549in}{0.888780in}}%
\pgfpathlineto{\pgfqpoint{1.777730in}{0.889181in}}%
\pgfpathlineto{\pgfqpoint{1.758229in}{0.898740in}}%
\pgfpathlineto{\pgfqpoint{1.747851in}{0.903859in}}%
\pgfpathlineto{\pgfqpoint{1.738042in}{0.908699in}}%
\pgfpathlineto{\pgfqpoint{1.717989in}{0.918659in}}%
\pgfpathlineto{\pgfqpoint{1.717972in}{0.918667in}}%
\pgfpathlineto{\pgfqpoint{1.698070in}{0.928618in}}%
\pgfpathlineto{\pgfqpoint{1.688094in}{0.933641in}}%
\pgfpathlineto{\pgfqpoint{1.678285in}{0.938578in}}%
\pgfpathlineto{\pgfqpoint{1.658633in}{0.948537in}}%
\pgfpathlineto{\pgfqpoint{1.658215in}{0.948751in}}%
\pgfpathlineto{\pgfqpoint{1.639115in}{0.958497in}}%
\pgfpathlineto{\pgfqpoint{1.628336in}{0.964037in}}%
\pgfpathlineto{\pgfqpoint{1.619732in}{0.968456in}}%
\pgfpathlineto{\pgfqpoint{1.600483in}{0.978416in}}%
\pgfpathlineto{\pgfqpoint{1.598458in}{0.979472in}}%
\pgfpathlineto{\pgfqpoint{1.581369in}{0.988376in}}%
\pgfpathlineto{\pgfqpoint{1.568579in}{0.995090in}}%
\pgfpathlineto{\pgfqpoint{1.562390in}{0.998335in}}%
\pgfpathlineto{\pgfqpoint{1.543547in}{1.008295in}}%
\pgfpathlineto{\pgfqpoint{1.538700in}{1.010878in}}%
\pgfpathlineto{\pgfqpoint{1.524841in}{1.018254in}}%
\pgfpathlineto{\pgfqpoint{1.508822in}{1.026847in}}%
\pgfpathlineto{\pgfqpoint{1.506269in}{1.028214in}}%
\pgfpathlineto{\pgfqpoint{1.487839in}{1.038173in}}%
\pgfpathlineto{\pgfqpoint{1.478943in}{1.043021in}}%
\pgfpathlineto{\pgfqpoint{1.469546in}{1.048133in}}%
\pgfpathlineto{\pgfqpoint{1.451390in}{1.058092in}}%
\pgfpathlineto{\pgfqpoint{1.449064in}{1.059381in}}%
\pgfpathlineto{\pgfqpoint{1.433381in}{1.068052in}}%
\pgfpathlineto{\pgfqpoint{1.419186in}{1.075965in}}%
\pgfpathlineto{\pgfqpoint{1.415506in}{1.078012in}}%
\pgfpathlineto{\pgfqpoint{1.397780in}{1.087971in}}%
\pgfpathlineto{\pgfqpoint{1.389307in}{1.092776in}}%
\pgfpathlineto{\pgfqpoint{1.380196in}{1.097931in}}%
\pgfpathlineto{\pgfqpoint{1.362754in}{1.107890in}}%
\pgfpathlineto{\pgfqpoint{1.359428in}{1.109810in}}%
\pgfpathlineto{\pgfqpoint{1.345468in}{1.117850in}}%
\pgfpathlineto{\pgfqpoint{1.329550in}{1.127095in}}%
\pgfpathlineto{\pgfqpoint{1.328316in}{1.127809in}}%
\pgfpathlineto{\pgfqpoint{1.311334in}{1.137769in}}%
\pgfpathlineto{\pgfqpoint{1.299671in}{1.144670in}}%
\pgfpathlineto{\pgfqpoint{1.294489in}{1.147729in}}%
\pgfpathlineto{\pgfqpoint{1.277807in}{1.157688in}}%
\pgfpathlineto{\pgfqpoint{1.269792in}{1.162525in}}%
\pgfpathlineto{\pgfqpoint{1.261278in}{1.167648in}}%
\pgfpathlineto{\pgfqpoint{1.244904in}{1.177607in}}%
\pgfpathlineto{\pgfqpoint{1.239914in}{1.180681in}}%
\pgfpathlineto{\pgfqpoint{1.228702in}{1.187567in}}%
\pgfpathlineto{\pgfqpoint{1.212644in}{1.197526in}}%
\pgfpathlineto{\pgfqpoint{1.210035in}{1.199169in}}%
\pgfpathlineto{\pgfqpoint{1.196781in}{1.207486in}}%
\pgfpathlineto{\pgfqpoint{1.181050in}{1.217445in}}%
\pgfpathlineto{\pgfqpoint{1.180156in}{1.218021in}}%
\pgfpathlineto{\pgfqpoint{1.165537in}{1.227405in}}%
\pgfpathlineto{\pgfqpoint{1.150277in}{1.237282in}}%
\pgfpathlineto{\pgfqpoint{1.150149in}{1.237365in}}%
\pgfpathlineto{\pgfqpoint{1.134998in}{1.247324in}}%
\pgfpathlineto{\pgfqpoint{1.120399in}{1.257000in}}%
\pgfpathlineto{\pgfqpoint{1.119970in}{1.257284in}}%
\pgfpathlineto{\pgfqpoint{1.105194in}{1.267243in}}%
\pgfpathlineto{\pgfqpoint{1.090539in}{1.277203in}}%
\pgfpathlineto{\pgfqpoint{1.090520in}{1.277216in}}%
\pgfpathlineto{\pgfqpoint{1.076159in}{1.287162in}}%
\pgfpathlineto{\pgfqpoint{1.061905in}{1.297122in}}%
\pgfpathlineto{\pgfqpoint{1.060641in}{1.298024in}}%
\pgfpathlineto{\pgfqpoint{1.047930in}{1.307081in}}%
\pgfpathlineto{\pgfqpoint{1.034105in}{1.317041in}}%
\pgfpathlineto{\pgfqpoint{1.030763in}{1.319502in}}%
\pgfpathlineto{\pgfqpoint{1.020552in}{1.327001in}}%
\pgfpathlineto{\pgfqpoint{1.007188in}{1.336960in}}%
\pgfpathlineto{\pgfqpoint{1.000884in}{1.341753in}}%
\pgfpathlineto{\pgfqpoint{0.994071in}{1.346920in}}%
\pgfpathlineto{\pgfqpoint{0.981208in}{1.356879in}}%
\pgfpathlineto{\pgfqpoint{0.971005in}{1.364904in}}%
\pgfpathlineto{\pgfqpoint{0.968540in}{1.366839in}}%
\pgfpathlineto{\pgfqpoint{0.956221in}{1.376798in}}%
\pgfpathlineto{\pgfqpoint{0.944077in}{1.386758in}}%
\pgfpathlineto{\pgfqpoint{0.941127in}{1.389255in}}%
\pgfpathlineto{\pgfqpoint{0.932294in}{1.396718in}}%
\pgfpathlineto{\pgfqpoint{0.920775in}{1.406677in}}%
\pgfpathlineto{\pgfqpoint{0.911248in}{1.415092in}}%
\pgfpathlineto{\pgfqpoint{0.909498in}{1.416637in}}%
\pgfpathlineto{\pgfqpoint{0.898668in}{1.426596in}}%
\pgfpathlineto{\pgfqpoint{0.888082in}{1.436556in}}%
\pgfpathlineto{\pgfqpoint{0.881369in}{1.443086in}}%
\pgfpathlineto{\pgfqpoint{0.877844in}{1.446515in}}%
\pgfpathlineto{\pgfqpoint{0.868074in}{1.456475in}}%
\pgfpathlineto{\pgfqpoint{0.858612in}{1.466434in}}%
\pgfpathlineto{\pgfqpoint{0.851491in}{1.474249in}}%
\pgfpathlineto{\pgfqpoint{0.849538in}{1.476394in}}%
\pgfpathlineto{\pgfqpoint{0.841061in}{1.486354in}}%
\pgfpathlineto{\pgfqpoint{0.832991in}{1.496313in}}%
\pgfpathlineto{\pgfqpoint{0.825362in}{1.506273in}}%
\pgfpathlineto{\pgfqpoint{0.821612in}{1.511595in}}%
\pgfpathlineto{\pgfqpoint{0.818356in}{1.516232in}}%
\pgfpathlineto{\pgfqpoint{0.812057in}{1.526192in}}%
\pgfpathlineto{\pgfqpoint{0.806380in}{1.536151in}}%
\pgfpathlineto{\pgfqpoint{0.801404in}{1.546111in}}%
\pgfpathlineto{\pgfqpoint{0.797223in}{1.556070in}}%
\pgfpathlineto{\pgfqpoint{0.793961in}{1.566030in}}%
\pgfpathlineto{\pgfqpoint{0.791770in}{1.575990in}}%
\pgfpathlineto{\pgfqpoint{0.791733in}{1.576404in}}%
\pgfpathlineto{\pgfqpoint{0.790908in}{1.585949in}}%
\pgfpathlineto{\pgfqpoint{0.791467in}{1.595909in}}%
\pgfpathlineto{\pgfqpoint{0.791733in}{1.597154in}}%
\pgfpathlineto{\pgfqpoint{0.793897in}{1.605868in}}%
\pgfpathlineto{\pgfqpoint{0.798639in}{1.615828in}}%
\pgfpathlineto{\pgfqpoint{0.806261in}{1.625787in}}%
\pgfpathlineto{\pgfqpoint{0.817574in}{1.635747in}}%
\pgfpathlineto{\pgfqpoint{0.821612in}{1.638431in}}%
\pgfpathlineto{\pgfqpoint{0.835712in}{1.645706in}}%
\pgfpathlineto{\pgfqpoint{0.851491in}{1.651799in}}%
\pgfpathlineto{\pgfqpoint{0.865680in}{1.655666in}}%
\pgfpathlineto{\pgfqpoint{0.881369in}{1.659095in}}%
\pgfpathlineto{\pgfqpoint{0.911248in}{1.663166in}}%
\pgfpathlineto{\pgfqpoint{0.941127in}{1.665164in}}%
\pgfpathlineto{\pgfqpoint{0.971005in}{1.665609in}}%
\pgfpathlineto{\pgfqpoint{1.000884in}{1.664859in}}%
\pgfpathlineto{\pgfqpoint{1.030763in}{1.663165in}}%
\pgfpathlineto{\pgfqpoint{1.060641in}{1.660707in}}%
\pgfpathlineto{\pgfqpoint{1.090520in}{1.657618in}}%
\pgfpathlineto{\pgfqpoint{1.106210in}{1.655666in}}%
\pgfpathlineto{\pgfqpoint{1.120399in}{1.653938in}}%
\pgfpathlineto{\pgfqpoint{1.150277in}{1.649739in}}%
\pgfpathlineto{\pgfqpoint{1.176358in}{1.645706in}}%
\pgfpathlineto{\pgfqpoint{1.180156in}{1.645131in}}%
\pgfpathlineto{\pgfqpoint{1.210035in}{1.640067in}}%
\pgfpathlineto{\pgfqpoint{1.233902in}{1.635747in}}%
\pgfpathlineto{\pgfqpoint{1.239914in}{1.634678in}}%
\pgfpathlineto{\pgfqpoint{1.269792in}{1.628922in}}%
\pgfpathlineto{\pgfqpoint{1.285143in}{1.625787in}}%
\pgfpathlineto{\pgfqpoint{1.299671in}{1.622870in}}%
\pgfpathlineto{\pgfqpoint{1.329550in}{1.616542in}}%
\pgfpathlineto{\pgfqpoint{1.332728in}{1.615828in}}%
\pgfpathlineto{\pgfqpoint{1.359428in}{1.609917in}}%
\pgfpathlineto{\pgfqpoint{1.376957in}{1.605868in}}%
\pgfpathlineto{\pgfqpoint{1.389307in}{1.603059in}}%
\pgfpathlineto{\pgfqpoint{1.419186in}{1.595975in}}%
\pgfpathlineto{\pgfqpoint{1.419452in}{1.595909in}}%
\pgfpathlineto{\pgfqpoint{1.449064in}{1.588639in}}%
\pgfpathlineto{\pgfqpoint{1.459636in}{1.585949in}}%
\pgfpathlineto{\pgfqpoint{1.478943in}{1.581101in}}%
\pgfpathlineto{\pgfqpoint{1.498606in}{1.575990in}}%
\pgfpathlineto{\pgfqpoint{1.508822in}{1.573368in}}%
\pgfpathlineto{\pgfqpoint{1.536473in}{1.566030in}}%
\pgfpathlineto{\pgfqpoint{1.538700in}{1.565446in}}%
\pgfpathlineto{\pgfqpoint{1.568579in}{1.557329in}}%
\pgfpathlineto{\pgfqpoint{1.573070in}{1.556070in}}%
\pgfpathlineto{\pgfqpoint{1.598458in}{1.549034in}}%
\pgfpathlineto{\pgfqpoint{1.608704in}{1.546111in}}%
\pgfpathlineto{\pgfqpoint{1.628336in}{1.540571in}}%
\pgfpathlineto{\pgfqpoint{1.643568in}{1.536151in}}%
\pgfpathlineto{\pgfqpoint{1.658215in}{1.531945in}}%
\pgfpathlineto{\pgfqpoint{1.677721in}{1.526192in}}%
\pgfpathlineto{\pgfqpoint{1.688094in}{1.523163in}}%
\pgfpathlineto{\pgfqpoint{1.711216in}{1.516232in}}%
\pgfpathlineto{\pgfqpoint{1.717972in}{1.514227in}}%
\pgfpathlineto{\pgfqpoint{1.744102in}{1.506273in}}%
\pgfpathlineto{\pgfqpoint{1.747851in}{1.505142in}}%
\pgfpathlineto{\pgfqpoint{1.776420in}{1.496313in}}%
\pgfpathlineto{\pgfqpoint{1.777730in}{1.495912in}}%
\pgfpathlineto{\pgfqpoint{1.807608in}{1.486540in}}%
\pgfpathlineto{\pgfqpoint{1.808189in}{1.486354in}}%
\pgfpathlineto{\pgfqpoint{1.837487in}{1.477029in}}%
\pgfpathlineto{\pgfqpoint{1.839440in}{1.476394in}}%
\pgfpathlineto{\pgfqpoint{1.867366in}{1.467384in}}%
\pgfpathlineto{\pgfqpoint{1.870251in}{1.466434in}}%
\pgfpathlineto{\pgfqpoint{1.897244in}{1.457609in}}%
\pgfpathlineto{\pgfqpoint{1.900646in}{1.456475in}}%
\pgfpathlineto{\pgfqpoint{1.927123in}{1.447705in}}%
\pgfpathlineto{\pgfqpoint{1.930649in}{1.446515in}}%
\pgfpathlineto{\pgfqpoint{1.957002in}{1.437676in}}%
\pgfpathlineto{\pgfqpoint{1.960281in}{1.436556in}}%
\pgfpathlineto{\pgfqpoint{1.986880in}{1.427523in}}%
\pgfpathlineto{\pgfqpoint{1.989562in}{1.426596in}}%
\pgfpathlineto{\pgfqpoint{2.016759in}{1.417248in}}%
\pgfpathlineto{\pgfqpoint{2.018509in}{1.416637in}}%
\pgfpathlineto{\pgfqpoint{2.046638in}{1.406854in}}%
\pgfpathlineto{\pgfqpoint{2.047139in}{1.406677in}}%
\pgfpathlineto{\pgfqpoint{2.075443in}{1.396718in}}%
\pgfpathlineto{\pgfqpoint{2.076517in}{1.396342in}}%
\pgfpathlineto{\pgfqpoint{2.103445in}{1.386758in}}%
\pgfpathlineto{\pgfqpoint{2.106395in}{1.385712in}}%
\pgfpathlineto{\pgfqpoint{2.131169in}{1.376798in}}%
\pgfpathlineto{\pgfqpoint{2.136274in}{1.374969in}}%
\pgfpathlineto{\pgfqpoint{2.158626in}{1.366839in}}%
\pgfpathlineto{\pgfqpoint{2.166153in}{1.364111in}}%
\pgfpathlineto{\pgfqpoint{2.185829in}{1.356879in}}%
\pgfpathlineto{\pgfqpoint{2.196031in}{1.353141in}}%
\pgfpathlineto{\pgfqpoint{2.212786in}{1.346920in}}%
\pgfpathlineto{\pgfqpoint{2.225910in}{1.342060in}}%
\pgfpathlineto{\pgfqpoint{2.239507in}{1.336960in}}%
\pgfpathlineto{\pgfqpoint{2.255789in}{1.330868in}}%
\pgfpathlineto{\pgfqpoint{2.266000in}{1.327001in}}%
\pgfpathlineto{\pgfqpoint{2.285667in}{1.319566in}}%
\pgfpathlineto{\pgfqpoint{2.292271in}{1.317041in}}%
\pgfpathlineto{\pgfqpoint{2.315546in}{1.308155in}}%
\pgfpathlineto{\pgfqpoint{2.318328in}{1.307081in}}%
\pgfpathlineto{\pgfqpoint{2.344161in}{1.297122in}}%
\pgfpathlineto{\pgfqpoint{2.345425in}{1.296636in}}%
\pgfpathlineto{\pgfqpoint{2.369761in}{1.287162in}}%
\pgfpathlineto{\pgfqpoint{2.375303in}{1.285007in}}%
\pgfpathlineto{\pgfqpoint{2.395168in}{1.277203in}}%
\pgfpathlineto{\pgfqpoint{2.405182in}{1.273271in}}%
\pgfpathlineto{\pgfqpoint{2.420387in}{1.267243in}}%
\pgfpathlineto{\pgfqpoint{2.435061in}{1.261427in}}%
\pgfpathlineto{\pgfqpoint{2.445420in}{1.257284in}}%
\pgfpathlineto{\pgfqpoint{2.464939in}{1.249476in}}%
\pgfpathlineto{\pgfqpoint{2.470272in}{1.247324in}}%
\pgfpathlineto{\pgfqpoint{2.494818in}{1.237417in}}%
\pgfpathlineto{\pgfqpoint{2.494947in}{1.237365in}}%
\pgfpathlineto{\pgfqpoint{2.519407in}{1.227405in}}%
\pgfpathlineto{\pgfqpoint{2.524697in}{1.225250in}}%
\pgfpathlineto{\pgfqpoint{2.543701in}{1.217445in}}%
\pgfpathlineto{\pgfqpoint{2.554575in}{1.212975in}}%
\pgfpathlineto{\pgfqpoint{2.567829in}{1.207486in}}%
\pgfpathlineto{\pgfqpoint{2.584454in}{1.200592in}}%
\pgfpathlineto{\pgfqpoint{2.591795in}{1.197526in}}%
\pgfpathlineto{\pgfqpoint{2.614333in}{1.188100in}}%
\pgfpathlineto{\pgfqpoint{2.615600in}{1.187567in}}%
\pgfpathlineto{\pgfqpoint{2.639221in}{1.177607in}}%
\pgfpathlineto{\pgfqpoint{2.644211in}{1.175499in}}%
\pgfpathlineto{\pgfqpoint{2.662682in}{1.167648in}}%
\pgfpathlineto{\pgfqpoint{2.674090in}{1.162788in}}%
\pgfpathlineto{\pgfqpoint{2.685992in}{1.157688in}}%
\pgfpathlineto{\pgfqpoint{2.703969in}{1.149966in}}%
\pgfpathlineto{\pgfqpoint{2.709151in}{1.147729in}}%
\pgfpathlineto{\pgfqpoint{2.732154in}{1.137769in}}%
\pgfpathlineto{\pgfqpoint{2.733847in}{1.137034in}}%
\pgfpathlineto{\pgfqpoint{2.754992in}{1.127809in}}%
\pgfpathlineto{\pgfqpoint{2.763726in}{1.123987in}}%
\pgfpathlineto{\pgfqpoint{2.777686in}{1.117850in}}%
\pgfpathlineto{\pgfqpoint{2.793605in}{1.110828in}}%
\pgfpathlineto{\pgfqpoint{2.800236in}{1.107890in}}%
\pgfpathlineto{\pgfqpoint{2.822641in}{1.097931in}}%
\pgfpathlineto{\pgfqpoint{2.823483in}{1.097555in}}%
\pgfpathlineto{\pgfqpoint{2.844889in}{1.087971in}}%
\pgfpathlineto{\pgfqpoint{2.853362in}{1.084162in}}%
\pgfpathlineto{\pgfqpoint{2.866998in}{1.078012in}}%
\pgfpathlineto{\pgfqpoint{2.883241in}{1.070654in}}%
\pgfpathlineto{\pgfqpoint{2.888967in}{1.068052in}}%
\pgfpathlineto{\pgfqpoint{2.910794in}{1.058092in}}%
\pgfpathlineto{\pgfqpoint{2.913120in}{1.057027in}}%
\pgfpathlineto{\pgfqpoint{2.932475in}{1.048133in}}%
\pgfpathlineto{\pgfqpoint{2.942998in}{1.043275in}}%
\pgfpathlineto{\pgfqpoint{2.954021in}{1.038173in}}%
\pgfpathlineto{\pgfqpoint{2.972877in}{1.029403in}}%
\pgfpathlineto{\pgfqpoint{2.975429in}{1.028214in}}%
\pgfpathlineto{\pgfqpoint{2.996697in}{1.018254in}}%
\pgfpathlineto{\pgfqpoint{3.002756in}{1.015402in}}%
\pgfpathlineto{\pgfqpoint{3.017827in}{1.008295in}}%
\pgfpathlineto{\pgfqpoint{3.032634in}{1.001274in}}%
\pgfpathlineto{\pgfqpoint{3.038823in}{0.998335in}}%
\pgfpathlineto{\pgfqpoint{3.059683in}{0.988376in}}%
\pgfpathlineto{\pgfqpoint{3.062513in}{0.987017in}}%
\pgfpathlineto{\pgfqpoint{3.080406in}{0.978416in}}%
\pgfpathlineto{\pgfqpoint{3.092392in}{0.972621in}}%
\pgfpathlineto{\pgfqpoint{3.100996in}{0.968456in}}%
\pgfpathlineto{\pgfqpoint{3.121451in}{0.958497in}}%
\pgfpathlineto{\pgfqpoint{3.122270in}{0.958096in}}%
\pgfpathlineto{\pgfqpoint{3.141771in}{0.948537in}}%
\pgfpathlineto{\pgfqpoint{3.152149in}{0.943418in}}%
\pgfpathlineto{\pgfqpoint{3.161958in}{0.938578in}}%
\pgfpathlineto{\pgfqpoint{3.182011in}{0.928618in}}%
\pgfpathlineto{\pgfqpoint{3.182028in}{0.928610in}}%
\pgfpathlineto{\pgfqpoint{3.201930in}{0.918659in}}%
\pgfpathlineto{\pgfqpoint{3.211906in}{0.913636in}}%
\pgfpathlineto{\pgfqpoint{3.221715in}{0.908699in}}%
\pgfpathlineto{\pgfqpoint{3.241367in}{0.898740in}}%
\pgfpathlineto{\pgfqpoint{3.241785in}{0.898526in}}%
\pgfpathlineto{\pgfqpoint{3.260885in}{0.888780in}}%
\pgfpathlineto{\pgfqpoint{3.271664in}{0.883240in}}%
\pgfpathlineto{\pgfqpoint{3.280268in}{0.878820in}}%
\pgfpathlineto{\pgfqpoint{3.299517in}{0.868861in}}%
\pgfpathlineto{\pgfqpoint{3.301542in}{0.867805in}}%
\pgfpathlineto{\pgfqpoint{3.318631in}{0.858901in}}%
\pgfpathlineto{\pgfqpoint{3.331421in}{0.852187in}}%
\pgfpathlineto{\pgfqpoint{3.337610in}{0.848942in}}%
\pgfpathlineto{\pgfqpoint{3.356453in}{0.838982in}}%
\pgfpathlineto{\pgfqpoint{3.361300in}{0.836399in}}%
\pgfpathlineto{\pgfqpoint{3.375159in}{0.829023in}}%
\pgfpathlineto{\pgfqpoint{3.391178in}{0.820430in}}%
\pgfpathlineto{\pgfqpoint{3.393731in}{0.819063in}}%
\pgfpathlineto{\pgfqpoint{3.412161in}{0.809103in}}%
\pgfpathlineto{\pgfqpoint{3.421057in}{0.804255in}}%
\pgfpathlineto{\pgfqpoint{3.430454in}{0.799144in}}%
\pgfpathlineto{\pgfqpoint{3.448610in}{0.789184in}}%
\pgfpathlineto{\pgfqpoint{3.450936in}{0.787896in}}%
\pgfpathlineto{\pgfqpoint{3.466619in}{0.779225in}}%
\pgfpathlineto{\pgfqpoint{3.480814in}{0.771312in}}%
\pgfpathlineto{\pgfqpoint{3.484494in}{0.769265in}}%
\pgfpathlineto{\pgfqpoint{3.502220in}{0.759306in}}%
\pgfpathlineto{\pgfqpoint{3.510693in}{0.754501in}}%
\pgfpathlineto{\pgfqpoint{3.519804in}{0.749346in}}%
\pgfpathlineto{\pgfqpoint{3.537246in}{0.739387in}}%
\pgfpathlineto{\pgfqpoint{3.540572in}{0.737466in}}%
\pgfpathlineto{\pgfqpoint{3.554532in}{0.729427in}}%
\pgfpathlineto{\pgfqpoint{3.570450in}{0.720182in}}%
\pgfpathlineto{\pgfqpoint{3.571684in}{0.719467in}}%
\pgfpathlineto{\pgfqpoint{3.588666in}{0.709508in}}%
\pgfpathlineto{\pgfqpoint{3.600329in}{0.702606in}}%
\pgfpathlineto{\pgfqpoint{3.605511in}{0.699548in}}%
\pgfpathlineto{\pgfqpoint{3.622193in}{0.689589in}}%
\pgfpathlineto{\pgfqpoint{3.630208in}{0.684752in}}%
\pgfpathlineto{\pgfqpoint{3.638722in}{0.679629in}}%
\pgfpathlineto{\pgfqpoint{3.655096in}{0.669670in}}%
\pgfpathlineto{\pgfqpoint{3.660086in}{0.666596in}}%
\pgfpathlineto{\pgfqpoint{3.671298in}{0.659710in}}%
\pgfpathlineto{\pgfqpoint{3.687356in}{0.649751in}}%
\pgfpathlineto{\pgfqpoint{3.689965in}{0.648108in}}%
\pgfpathlineto{\pgfqpoint{3.703219in}{0.639791in}}%
\pgfpathlineto{\pgfqpoint{3.718950in}{0.629831in}}%
\pgfpathlineto{\pgfqpoint{3.719844in}{0.629256in}}%
\pgfpathlineto{\pgfqpoint{3.734463in}{0.619872in}}%
\pgfpathlineto{\pgfqpoint{3.749723in}{0.609995in}}%
\pgfpathlineto{\pgfqpoint{3.749851in}{0.609912in}}%
\pgfpathlineto{\pgfqpoint{3.765002in}{0.599953in}}%
\pgfpathlineto{\pgfqpoint{3.779601in}{0.590276in}}%
\pgfpathlineto{\pgfqpoint{3.780030in}{0.589993in}}%
\pgfpathlineto{\pgfqpoint{3.794806in}{0.580034in}}%
\pgfpathlineto{\pgfqpoint{3.809461in}{0.570074in}}%
\pgfpathlineto{\pgfqpoint{3.809480in}{0.570061in}}%
\pgfpathlineto{\pgfqpoint{3.823841in}{0.560115in}}%
\pgfpathlineto{\pgfqpoint{3.838095in}{0.550155in}}%
\pgfpathlineto{\pgfqpoint{3.839359in}{0.549252in}}%
\pgfpathlineto{\pgfqpoint{3.852070in}{0.540195in}}%
\pgfpathlineto{\pgfqpoint{3.865895in}{0.530236in}}%
\pgfpathlineto{\pgfqpoint{3.869237in}{0.527775in}}%
\pgfpathlineto{\pgfqpoint{3.879448in}{0.520276in}}%
\pgfpathlineto{\pgfqpoint{3.892812in}{0.510317in}}%
\pgfpathlineto{\pgfqpoint{3.899116in}{0.505524in}}%
\pgfpathlineto{\pgfqpoint{3.905929in}{0.500357in}}%
\pgfpathlineto{\pgfqpoint{3.918792in}{0.490398in}}%
\pgfpathlineto{\pgfqpoint{3.928995in}{0.482373in}}%
\pgfpathlineto{\pgfqpoint{3.931460in}{0.480438in}}%
\pgfpathlineto{\pgfqpoint{3.943779in}{0.470478in}}%
\pgfpathlineto{\pgfqpoint{3.955923in}{0.460519in}}%
\pgfpathlineto{\pgfqpoint{3.958873in}{0.458022in}}%
\pgfpathlineto{\pgfqpoint{3.967706in}{0.450559in}}%
\pgfpathlineto{\pgfqpoint{3.979225in}{0.440600in}}%
\pgfpathlineto{\pgfqpoint{3.988752in}{0.432184in}}%
\pgfpathlineto{\pgfqpoint{3.990502in}{0.430640in}}%
\pgfpathlineto{\pgfqpoint{4.001332in}{0.420681in}}%
\pgfpathlineto{\pgfqpoint{4.011918in}{0.410721in}}%
\pgfpathlineto{\pgfqpoint{4.018631in}{0.404191in}}%
\pgfpathlineto{\pgfqpoint{4.022156in}{0.400762in}}%
\pgfpathlineto{\pgfqpoint{4.031926in}{0.390802in}}%
\pgfpathlineto{\pgfqpoint{4.041388in}{0.380842in}}%
\pgfpathlineto{\pgfqpoint{4.048509in}{0.373028in}}%
\pgfpathlineto{\pgfqpoint{4.050462in}{0.370883in}}%
\pgfpathlineto{\pgfqpoint{4.058939in}{0.360923in}}%
\pgfpathlineto{\pgfqpoint{4.067009in}{0.350964in}}%
\pgfpathlineto{\pgfqpoint{4.074638in}{0.341004in}}%
\pgfpathlineto{\pgfqpoint{4.078388in}{0.335681in}}%
\pgfpathlineto{\pgfqpoint{4.081644in}{0.331045in}}%
\pgfpathlineto{\pgfqpoint{4.087943in}{0.321085in}}%
\pgfpathlineto{\pgfqpoint{4.093620in}{0.311126in}}%
\pgfpathlineto{\pgfqpoint{4.098596in}{0.301166in}}%
\pgfpathlineto{\pgfqpoint{4.102777in}{0.291206in}}%
\pgfpathlineto{\pgfqpoint{4.106039in}{0.281247in}}%
\pgfpathlineto{\pgfqpoint{4.108230in}{0.271287in}}%
\pgfpathlineto{\pgfqpoint{4.108267in}{0.270872in}}%
\pgfpathlineto{\pgfqpoint{4.109092in}{0.261328in}}%
\pgfpathlineto{\pgfqpoint{4.108533in}{0.251368in}}%
\pgfpathlineto{\pgfqpoint{4.108267in}{0.250123in}}%
\pgfpathlineto{\pgfqpoint{4.106103in}{0.241409in}}%
\pgfpathlineto{\pgfqpoint{4.101361in}{0.231449in}}%
\pgfpathlineto{\pgfqpoint{4.093739in}{0.221489in}}%
\pgfpathlineto{\pgfqpoint{4.082426in}{0.211530in}}%
\pgfpathlineto{\pgfqpoint{4.078388in}{0.208846in}}%
\pgfpathlineto{\pgfqpoint{4.064288in}{0.201570in}}%
\pgfpathlineto{\pgfqpoint{4.048509in}{0.195478in}}%
\pgfpathlineto{\pgfqpoint{4.034320in}{0.191611in}}%
\pgfpathlineto{\pgfqpoint{4.018631in}{0.188182in}}%
\pgfpathlineto{\pgfqpoint{3.988752in}{0.184110in}}%
\pgfpathlineto{\pgfqpoint{3.958873in}{0.182113in}}%
\pgfpathlineto{\pgfqpoint{3.928995in}{0.181668in}}%
\pgfpathlineto{\pgfqpoint{3.899116in}{0.182418in}}%
\pgfpathlineto{\pgfqpoint{3.869237in}{0.184112in}}%
\pgfpathlineto{\pgfqpoint{3.839359in}{0.186570in}}%
\pgfpathlineto{\pgfqpoint{3.809480in}{0.189659in}}%
\pgfpathclose%
\pgfusepath{fill}%
\end{pgfscope}%
\begin{pgfscope}%
\pgfpathrectangle{\pgfqpoint{0.135000in}{0.151972in}}{\pgfqpoint{4.630000in}{1.543333in}} %
\pgfusepath{clip}%
\pgfsetbuttcap%
\pgfsetroundjoin%
\definecolor{currentfill}{rgb}{1.000000,1.000000,1.000000}%
\pgfsetfillcolor{currentfill}%
\pgfsetlinewidth{0.000000pt}%
\definecolor{currentstroke}{rgb}{0.000000,0.000000,0.000000}%
\pgfsetstrokecolor{currentstroke}%
\pgfsetdash{}{0pt}%
\pgfpathmoveto{\pgfqpoint{3.809480in}{0.189659in}}%
\pgfpathlineto{\pgfqpoint{3.839359in}{0.186570in}}%
\pgfpathlineto{\pgfqpoint{3.869237in}{0.184112in}}%
\pgfpathlineto{\pgfqpoint{3.899116in}{0.182418in}}%
\pgfpathlineto{\pgfqpoint{3.928995in}{0.181668in}}%
\pgfpathlineto{\pgfqpoint{3.958873in}{0.182113in}}%
\pgfpathlineto{\pgfqpoint{3.988752in}{0.184110in}}%
\pgfpathlineto{\pgfqpoint{4.018631in}{0.188182in}}%
\pgfpathlineto{\pgfqpoint{4.034320in}{0.191611in}}%
\pgfpathlineto{\pgfqpoint{4.048509in}{0.195478in}}%
\pgfpathlineto{\pgfqpoint{4.064288in}{0.201570in}}%
\pgfpathlineto{\pgfqpoint{4.078388in}{0.208846in}}%
\pgfpathlineto{\pgfqpoint{4.082426in}{0.211530in}}%
\pgfpathlineto{\pgfqpoint{4.093739in}{0.221489in}}%
\pgfpathlineto{\pgfqpoint{4.101361in}{0.231449in}}%
\pgfpathlineto{\pgfqpoint{4.106103in}{0.241409in}}%
\pgfpathlineto{\pgfqpoint{4.108267in}{0.250123in}}%
\pgfpathlineto{\pgfqpoint{4.108533in}{0.251368in}}%
\pgfpathlineto{\pgfqpoint{4.109092in}{0.261328in}}%
\pgfpathlineto{\pgfqpoint{4.108267in}{0.270872in}}%
\pgfpathlineto{\pgfqpoint{4.108230in}{0.271287in}}%
\pgfpathlineto{\pgfqpoint{4.106039in}{0.281247in}}%
\pgfpathlineto{\pgfqpoint{4.102777in}{0.291206in}}%
\pgfpathlineto{\pgfqpoint{4.098596in}{0.301166in}}%
\pgfpathlineto{\pgfqpoint{4.093620in}{0.311126in}}%
\pgfpathlineto{\pgfqpoint{4.087943in}{0.321085in}}%
\pgfpathlineto{\pgfqpoint{4.081644in}{0.331045in}}%
\pgfpathlineto{\pgfqpoint{4.078388in}{0.335681in}}%
\pgfpathlineto{\pgfqpoint{4.074638in}{0.341004in}}%
\pgfpathlineto{\pgfqpoint{4.067009in}{0.350964in}}%
\pgfpathlineto{\pgfqpoint{4.058939in}{0.360923in}}%
\pgfpathlineto{\pgfqpoint{4.050462in}{0.370883in}}%
\pgfpathlineto{\pgfqpoint{4.048509in}{0.373028in}}%
\pgfpathlineto{\pgfqpoint{4.041388in}{0.380842in}}%
\pgfpathlineto{\pgfqpoint{4.031926in}{0.390802in}}%
\pgfpathlineto{\pgfqpoint{4.022156in}{0.400762in}}%
\pgfpathlineto{\pgfqpoint{4.018631in}{0.404191in}}%
\pgfpathlineto{\pgfqpoint{4.011918in}{0.410721in}}%
\pgfpathlineto{\pgfqpoint{4.001332in}{0.420681in}}%
\pgfpathlineto{\pgfqpoint{3.990502in}{0.430640in}}%
\pgfpathlineto{\pgfqpoint{3.988752in}{0.432184in}}%
\pgfpathlineto{\pgfqpoint{3.979225in}{0.440600in}}%
\pgfpathlineto{\pgfqpoint{3.967706in}{0.450559in}}%
\pgfpathlineto{\pgfqpoint{3.958873in}{0.458022in}}%
\pgfpathlineto{\pgfqpoint{3.955923in}{0.460519in}}%
\pgfpathlineto{\pgfqpoint{3.943779in}{0.470478in}}%
\pgfpathlineto{\pgfqpoint{3.931460in}{0.480438in}}%
\pgfpathlineto{\pgfqpoint{3.928995in}{0.482373in}}%
\pgfpathlineto{\pgfqpoint{3.918792in}{0.490398in}}%
\pgfpathlineto{\pgfqpoint{3.905929in}{0.500357in}}%
\pgfpathlineto{\pgfqpoint{3.899116in}{0.505524in}}%
\pgfpathlineto{\pgfqpoint{3.892812in}{0.510317in}}%
\pgfpathlineto{\pgfqpoint{3.879448in}{0.520276in}}%
\pgfpathlineto{\pgfqpoint{3.869237in}{0.527775in}}%
\pgfpathlineto{\pgfqpoint{3.865895in}{0.530236in}}%
\pgfpathlineto{\pgfqpoint{3.852070in}{0.540195in}}%
\pgfpathlineto{\pgfqpoint{3.839359in}{0.549252in}}%
\pgfpathlineto{\pgfqpoint{3.838095in}{0.550155in}}%
\pgfpathlineto{\pgfqpoint{3.823841in}{0.560115in}}%
\pgfpathlineto{\pgfqpoint{3.809480in}{0.570061in}}%
\pgfpathlineto{\pgfqpoint{3.809461in}{0.570074in}}%
\pgfpathlineto{\pgfqpoint{3.794806in}{0.580034in}}%
\pgfpathlineto{\pgfqpoint{3.780030in}{0.589993in}}%
\pgfpathlineto{\pgfqpoint{3.779601in}{0.590276in}}%
\pgfpathlineto{\pgfqpoint{3.765002in}{0.599953in}}%
\pgfpathlineto{\pgfqpoint{3.749851in}{0.609912in}}%
\pgfpathlineto{\pgfqpoint{3.749723in}{0.609995in}}%
\pgfpathlineto{\pgfqpoint{3.734463in}{0.619872in}}%
\pgfpathlineto{\pgfqpoint{3.719844in}{0.629256in}}%
\pgfpathlineto{\pgfqpoint{3.718950in}{0.629831in}}%
\pgfpathlineto{\pgfqpoint{3.703219in}{0.639791in}}%
\pgfpathlineto{\pgfqpoint{3.689965in}{0.648108in}}%
\pgfpathlineto{\pgfqpoint{3.687356in}{0.649751in}}%
\pgfpathlineto{\pgfqpoint{3.671298in}{0.659710in}}%
\pgfpathlineto{\pgfqpoint{3.660086in}{0.666596in}}%
\pgfpathlineto{\pgfqpoint{3.655096in}{0.669670in}}%
\pgfpathlineto{\pgfqpoint{3.638722in}{0.679629in}}%
\pgfpathlineto{\pgfqpoint{3.630208in}{0.684752in}}%
\pgfpathlineto{\pgfqpoint{3.622193in}{0.689589in}}%
\pgfpathlineto{\pgfqpoint{3.605511in}{0.699548in}}%
\pgfpathlineto{\pgfqpoint{3.600329in}{0.702606in}}%
\pgfpathlineto{\pgfqpoint{3.588666in}{0.709508in}}%
\pgfpathlineto{\pgfqpoint{3.571684in}{0.719467in}}%
\pgfpathlineto{\pgfqpoint{3.570450in}{0.720182in}}%
\pgfpathlineto{\pgfqpoint{3.554532in}{0.729427in}}%
\pgfpathlineto{\pgfqpoint{3.540572in}{0.737466in}}%
\pgfpathlineto{\pgfqpoint{3.537246in}{0.739387in}}%
\pgfpathlineto{\pgfqpoint{3.519804in}{0.749346in}}%
\pgfpathlineto{\pgfqpoint{3.510693in}{0.754501in}}%
\pgfpathlineto{\pgfqpoint{3.502220in}{0.759306in}}%
\pgfpathlineto{\pgfqpoint{3.484494in}{0.769265in}}%
\pgfpathlineto{\pgfqpoint{3.480814in}{0.771312in}}%
\pgfpathlineto{\pgfqpoint{3.466619in}{0.779225in}}%
\pgfpathlineto{\pgfqpoint{3.450936in}{0.787896in}}%
\pgfpathlineto{\pgfqpoint{3.448610in}{0.789184in}}%
\pgfpathlineto{\pgfqpoint{3.430454in}{0.799144in}}%
\pgfpathlineto{\pgfqpoint{3.421057in}{0.804255in}}%
\pgfpathlineto{\pgfqpoint{3.412161in}{0.809103in}}%
\pgfpathlineto{\pgfqpoint{3.393731in}{0.819063in}}%
\pgfpathlineto{\pgfqpoint{3.391178in}{0.820430in}}%
\pgfpathlineto{\pgfqpoint{3.375159in}{0.829023in}}%
\pgfpathlineto{\pgfqpoint{3.361300in}{0.836399in}}%
\pgfpathlineto{\pgfqpoint{3.356453in}{0.838982in}}%
\pgfpathlineto{\pgfqpoint{3.337610in}{0.848942in}}%
\pgfpathlineto{\pgfqpoint{3.331421in}{0.852187in}}%
\pgfpathlineto{\pgfqpoint{3.318631in}{0.858901in}}%
\pgfpathlineto{\pgfqpoint{3.301542in}{0.867805in}}%
\pgfpathlineto{\pgfqpoint{3.299517in}{0.868861in}}%
\pgfpathlineto{\pgfqpoint{3.280268in}{0.878820in}}%
\pgfpathlineto{\pgfqpoint{3.271664in}{0.883240in}}%
\pgfpathlineto{\pgfqpoint{3.260885in}{0.888780in}}%
\pgfpathlineto{\pgfqpoint{3.241785in}{0.898526in}}%
\pgfpathlineto{\pgfqpoint{3.241367in}{0.898740in}}%
\pgfpathlineto{\pgfqpoint{3.221715in}{0.908699in}}%
\pgfpathlineto{\pgfqpoint{3.211906in}{0.913636in}}%
\pgfpathlineto{\pgfqpoint{3.201930in}{0.918659in}}%
\pgfpathlineto{\pgfqpoint{3.182028in}{0.928610in}}%
\pgfpathlineto{\pgfqpoint{3.182011in}{0.928618in}}%
\pgfpathlineto{\pgfqpoint{3.161958in}{0.938578in}}%
\pgfpathlineto{\pgfqpoint{3.152149in}{0.943418in}}%
\pgfpathlineto{\pgfqpoint{3.141771in}{0.948537in}}%
\pgfpathlineto{\pgfqpoint{3.122270in}{0.958096in}}%
\pgfpathlineto{\pgfqpoint{3.121451in}{0.958497in}}%
\pgfpathlineto{\pgfqpoint{3.100996in}{0.968456in}}%
\pgfpathlineto{\pgfqpoint{3.092392in}{0.972621in}}%
\pgfpathlineto{\pgfqpoint{3.080406in}{0.978416in}}%
\pgfpathlineto{\pgfqpoint{3.062513in}{0.987017in}}%
\pgfpathlineto{\pgfqpoint{3.059683in}{0.988376in}}%
\pgfpathlineto{\pgfqpoint{3.038823in}{0.998335in}}%
\pgfpathlineto{\pgfqpoint{3.032634in}{1.001274in}}%
\pgfpathlineto{\pgfqpoint{3.017827in}{1.008295in}}%
\pgfpathlineto{\pgfqpoint{3.002756in}{1.015402in}}%
\pgfpathlineto{\pgfqpoint{2.996697in}{1.018254in}}%
\pgfpathlineto{\pgfqpoint{2.975429in}{1.028214in}}%
\pgfpathlineto{\pgfqpoint{2.972877in}{1.029403in}}%
\pgfpathlineto{\pgfqpoint{2.954021in}{1.038173in}}%
\pgfpathlineto{\pgfqpoint{2.942998in}{1.043275in}}%
\pgfpathlineto{\pgfqpoint{2.932475in}{1.048133in}}%
\pgfpathlineto{\pgfqpoint{2.913120in}{1.057027in}}%
\pgfpathlineto{\pgfqpoint{2.910794in}{1.058092in}}%
\pgfpathlineto{\pgfqpoint{2.888967in}{1.068052in}}%
\pgfpathlineto{\pgfqpoint{2.883241in}{1.070654in}}%
\pgfpathlineto{\pgfqpoint{2.866998in}{1.078012in}}%
\pgfpathlineto{\pgfqpoint{2.853362in}{1.084162in}}%
\pgfpathlineto{\pgfqpoint{2.844889in}{1.087971in}}%
\pgfpathlineto{\pgfqpoint{2.823483in}{1.097555in}}%
\pgfpathlineto{\pgfqpoint{2.822641in}{1.097931in}}%
\pgfpathlineto{\pgfqpoint{2.800236in}{1.107890in}}%
\pgfpathlineto{\pgfqpoint{2.793605in}{1.110828in}}%
\pgfpathlineto{\pgfqpoint{2.777686in}{1.117850in}}%
\pgfpathlineto{\pgfqpoint{2.763726in}{1.123987in}}%
\pgfpathlineto{\pgfqpoint{2.754992in}{1.127809in}}%
\pgfpathlineto{\pgfqpoint{2.733847in}{1.137034in}}%
\pgfpathlineto{\pgfqpoint{2.732154in}{1.137769in}}%
\pgfpathlineto{\pgfqpoint{2.709151in}{1.147729in}}%
\pgfpathlineto{\pgfqpoint{2.703969in}{1.149966in}}%
\pgfpathlineto{\pgfqpoint{2.685992in}{1.157688in}}%
\pgfpathlineto{\pgfqpoint{2.674090in}{1.162788in}}%
\pgfpathlineto{\pgfqpoint{2.662682in}{1.167648in}}%
\pgfpathlineto{\pgfqpoint{2.644211in}{1.175499in}}%
\pgfpathlineto{\pgfqpoint{2.639221in}{1.177607in}}%
\pgfpathlineto{\pgfqpoint{2.615600in}{1.187567in}}%
\pgfpathlineto{\pgfqpoint{2.614333in}{1.188100in}}%
\pgfpathlineto{\pgfqpoint{2.591795in}{1.197526in}}%
\pgfpathlineto{\pgfqpoint{2.584454in}{1.200592in}}%
\pgfpathlineto{\pgfqpoint{2.567829in}{1.207486in}}%
\pgfpathlineto{\pgfqpoint{2.554575in}{1.212975in}}%
\pgfpathlineto{\pgfqpoint{2.543701in}{1.217445in}}%
\pgfpathlineto{\pgfqpoint{2.524697in}{1.225250in}}%
\pgfpathlineto{\pgfqpoint{2.519407in}{1.227405in}}%
\pgfpathlineto{\pgfqpoint{2.494947in}{1.237365in}}%
\pgfpathlineto{\pgfqpoint{2.494818in}{1.237417in}}%
\pgfpathlineto{\pgfqpoint{2.470272in}{1.247324in}}%
\pgfpathlineto{\pgfqpoint{2.464939in}{1.249476in}}%
\pgfpathlineto{\pgfqpoint{2.445420in}{1.257284in}}%
\pgfpathlineto{\pgfqpoint{2.435061in}{1.261427in}}%
\pgfpathlineto{\pgfqpoint{2.420387in}{1.267243in}}%
\pgfpathlineto{\pgfqpoint{2.405182in}{1.273271in}}%
\pgfpathlineto{\pgfqpoint{2.395168in}{1.277203in}}%
\pgfpathlineto{\pgfqpoint{2.375303in}{1.285007in}}%
\pgfpathlineto{\pgfqpoint{2.369761in}{1.287162in}}%
\pgfpathlineto{\pgfqpoint{2.345425in}{1.296636in}}%
\pgfpathlineto{\pgfqpoint{2.344161in}{1.297122in}}%
\pgfpathlineto{\pgfqpoint{2.318328in}{1.307081in}}%
\pgfpathlineto{\pgfqpoint{2.315546in}{1.308155in}}%
\pgfpathlineto{\pgfqpoint{2.292271in}{1.317041in}}%
\pgfpathlineto{\pgfqpoint{2.285667in}{1.319566in}}%
\pgfpathlineto{\pgfqpoint{2.266000in}{1.327001in}}%
\pgfpathlineto{\pgfqpoint{2.255789in}{1.330868in}}%
\pgfpathlineto{\pgfqpoint{2.239507in}{1.336960in}}%
\pgfpathlineto{\pgfqpoint{2.225910in}{1.342060in}}%
\pgfpathlineto{\pgfqpoint{2.212786in}{1.346920in}}%
\pgfpathlineto{\pgfqpoint{2.196031in}{1.353141in}}%
\pgfpathlineto{\pgfqpoint{2.185829in}{1.356879in}}%
\pgfpathlineto{\pgfqpoint{2.166153in}{1.364111in}}%
\pgfpathlineto{\pgfqpoint{2.158626in}{1.366839in}}%
\pgfpathlineto{\pgfqpoint{2.136274in}{1.374969in}}%
\pgfpathlineto{\pgfqpoint{2.131169in}{1.376798in}}%
\pgfpathlineto{\pgfqpoint{2.106395in}{1.385712in}}%
\pgfpathlineto{\pgfqpoint{2.103445in}{1.386758in}}%
\pgfpathlineto{\pgfqpoint{2.076517in}{1.396342in}}%
\pgfpathlineto{\pgfqpoint{2.075443in}{1.396718in}}%
\pgfpathlineto{\pgfqpoint{2.047139in}{1.406677in}}%
\pgfpathlineto{\pgfqpoint{2.046638in}{1.406854in}}%
\pgfpathlineto{\pgfqpoint{2.018509in}{1.416637in}}%
\pgfpathlineto{\pgfqpoint{2.016759in}{1.417248in}}%
\pgfpathlineto{\pgfqpoint{1.989562in}{1.426596in}}%
\pgfpathlineto{\pgfqpoint{1.986880in}{1.427523in}}%
\pgfpathlineto{\pgfqpoint{1.960281in}{1.436556in}}%
\pgfpathlineto{\pgfqpoint{1.957002in}{1.437676in}}%
\pgfpathlineto{\pgfqpoint{1.930649in}{1.446515in}}%
\pgfpathlineto{\pgfqpoint{1.927123in}{1.447705in}}%
\pgfpathlineto{\pgfqpoint{1.900646in}{1.456475in}}%
\pgfpathlineto{\pgfqpoint{1.897244in}{1.457609in}}%
\pgfpathlineto{\pgfqpoint{1.870251in}{1.466434in}}%
\pgfpathlineto{\pgfqpoint{1.867366in}{1.467384in}}%
\pgfpathlineto{\pgfqpoint{1.839440in}{1.476394in}}%
\pgfpathlineto{\pgfqpoint{1.837487in}{1.477029in}}%
\pgfpathlineto{\pgfqpoint{1.808189in}{1.486354in}}%
\pgfpathlineto{\pgfqpoint{1.807608in}{1.486540in}}%
\pgfpathlineto{\pgfqpoint{1.777730in}{1.495912in}}%
\pgfpathlineto{\pgfqpoint{1.776420in}{1.496313in}}%
\pgfpathlineto{\pgfqpoint{1.747851in}{1.505142in}}%
\pgfpathlineto{\pgfqpoint{1.744102in}{1.506273in}}%
\pgfpathlineto{\pgfqpoint{1.717972in}{1.514227in}}%
\pgfpathlineto{\pgfqpoint{1.711216in}{1.516232in}}%
\pgfpathlineto{\pgfqpoint{1.688094in}{1.523163in}}%
\pgfpathlineto{\pgfqpoint{1.677721in}{1.526192in}}%
\pgfpathlineto{\pgfqpoint{1.658215in}{1.531945in}}%
\pgfpathlineto{\pgfqpoint{1.643568in}{1.536151in}}%
\pgfpathlineto{\pgfqpoint{1.628336in}{1.540571in}}%
\pgfpathlineto{\pgfqpoint{1.608704in}{1.546111in}}%
\pgfpathlineto{\pgfqpoint{1.598458in}{1.549034in}}%
\pgfpathlineto{\pgfqpoint{1.573070in}{1.556070in}}%
\pgfpathlineto{\pgfqpoint{1.568579in}{1.557329in}}%
\pgfpathlineto{\pgfqpoint{1.538700in}{1.565446in}}%
\pgfpathlineto{\pgfqpoint{1.536473in}{1.566030in}}%
\pgfpathlineto{\pgfqpoint{1.508822in}{1.573368in}}%
\pgfpathlineto{\pgfqpoint{1.498606in}{1.575990in}}%
\pgfpathlineto{\pgfqpoint{1.478943in}{1.581101in}}%
\pgfpathlineto{\pgfqpoint{1.459636in}{1.585949in}}%
\pgfpathlineto{\pgfqpoint{1.449064in}{1.588639in}}%
\pgfpathlineto{\pgfqpoint{1.419452in}{1.595909in}}%
\pgfpathlineto{\pgfqpoint{1.419186in}{1.595975in}}%
\pgfpathlineto{\pgfqpoint{1.389307in}{1.603059in}}%
\pgfpathlineto{\pgfqpoint{1.376957in}{1.605868in}}%
\pgfpathlineto{\pgfqpoint{1.359428in}{1.609917in}}%
\pgfpathlineto{\pgfqpoint{1.332728in}{1.615828in}}%
\pgfpathlineto{\pgfqpoint{1.329550in}{1.616542in}}%
\pgfpathlineto{\pgfqpoint{1.299671in}{1.622870in}}%
\pgfpathlineto{\pgfqpoint{1.285143in}{1.625787in}}%
\pgfpathlineto{\pgfqpoint{1.269792in}{1.628922in}}%
\pgfpathlineto{\pgfqpoint{1.239914in}{1.634678in}}%
\pgfpathlineto{\pgfqpoint{1.233902in}{1.635747in}}%
\pgfpathlineto{\pgfqpoint{1.210035in}{1.640067in}}%
\pgfpathlineto{\pgfqpoint{1.180156in}{1.645131in}}%
\pgfpathlineto{\pgfqpoint{1.176358in}{1.645706in}}%
\pgfpathlineto{\pgfqpoint{1.150277in}{1.649739in}}%
\pgfpathlineto{\pgfqpoint{1.120399in}{1.653938in}}%
\pgfpathlineto{\pgfqpoint{1.106210in}{1.655666in}}%
\pgfpathlineto{\pgfqpoint{1.090520in}{1.657618in}}%
\pgfpathlineto{\pgfqpoint{1.060641in}{1.660707in}}%
\pgfpathlineto{\pgfqpoint{1.030763in}{1.663165in}}%
\pgfpathlineto{\pgfqpoint{1.000884in}{1.664859in}}%
\pgfpathlineto{\pgfqpoint{0.971005in}{1.665609in}}%
\pgfpathlineto{\pgfqpoint{0.941127in}{1.665164in}}%
\pgfpathlineto{\pgfqpoint{0.911248in}{1.663166in}}%
\pgfpathlineto{\pgfqpoint{0.881369in}{1.659095in}}%
\pgfpathlineto{\pgfqpoint{0.865680in}{1.655666in}}%
\pgfpathlineto{\pgfqpoint{0.851491in}{1.651799in}}%
\pgfpathlineto{\pgfqpoint{0.835712in}{1.645706in}}%
\pgfpathlineto{\pgfqpoint{0.821612in}{1.638431in}}%
\pgfpathlineto{\pgfqpoint{0.817574in}{1.635747in}}%
\pgfpathlineto{\pgfqpoint{0.806261in}{1.625787in}}%
\pgfpathlineto{\pgfqpoint{0.798639in}{1.615828in}}%
\pgfpathlineto{\pgfqpoint{0.793897in}{1.605868in}}%
\pgfpathlineto{\pgfqpoint{0.791733in}{1.597154in}}%
\pgfpathlineto{\pgfqpoint{0.791467in}{1.595909in}}%
\pgfpathlineto{\pgfqpoint{0.790908in}{1.585949in}}%
\pgfpathlineto{\pgfqpoint{0.791733in}{1.576404in}}%
\pgfpathlineto{\pgfqpoint{0.791770in}{1.575990in}}%
\pgfpathlineto{\pgfqpoint{0.793961in}{1.566030in}}%
\pgfpathlineto{\pgfqpoint{0.797223in}{1.556070in}}%
\pgfpathlineto{\pgfqpoint{0.801404in}{1.546111in}}%
\pgfpathlineto{\pgfqpoint{0.806380in}{1.536151in}}%
\pgfpathlineto{\pgfqpoint{0.812057in}{1.526192in}}%
\pgfpathlineto{\pgfqpoint{0.818356in}{1.516232in}}%
\pgfpathlineto{\pgfqpoint{0.821612in}{1.511595in}}%
\pgfpathlineto{\pgfqpoint{0.825362in}{1.506273in}}%
\pgfpathlineto{\pgfqpoint{0.832991in}{1.496313in}}%
\pgfpathlineto{\pgfqpoint{0.841061in}{1.486354in}}%
\pgfpathlineto{\pgfqpoint{0.849538in}{1.476394in}}%
\pgfpathlineto{\pgfqpoint{0.851491in}{1.474249in}}%
\pgfpathlineto{\pgfqpoint{0.858612in}{1.466434in}}%
\pgfpathlineto{\pgfqpoint{0.868074in}{1.456475in}}%
\pgfpathlineto{\pgfqpoint{0.877844in}{1.446515in}}%
\pgfpathlineto{\pgfqpoint{0.881369in}{1.443086in}}%
\pgfpathlineto{\pgfqpoint{0.888082in}{1.436556in}}%
\pgfpathlineto{\pgfqpoint{0.898668in}{1.426596in}}%
\pgfpathlineto{\pgfqpoint{0.909498in}{1.416637in}}%
\pgfpathlineto{\pgfqpoint{0.911248in}{1.415092in}}%
\pgfpathlineto{\pgfqpoint{0.920775in}{1.406677in}}%
\pgfpathlineto{\pgfqpoint{0.932294in}{1.396718in}}%
\pgfpathlineto{\pgfqpoint{0.941127in}{1.389255in}}%
\pgfpathlineto{\pgfqpoint{0.944077in}{1.386758in}}%
\pgfpathlineto{\pgfqpoint{0.956221in}{1.376798in}}%
\pgfpathlineto{\pgfqpoint{0.968540in}{1.366839in}}%
\pgfpathlineto{\pgfqpoint{0.971005in}{1.364904in}}%
\pgfpathlineto{\pgfqpoint{0.981208in}{1.356879in}}%
\pgfpathlineto{\pgfqpoint{0.994071in}{1.346920in}}%
\pgfpathlineto{\pgfqpoint{1.000884in}{1.341753in}}%
\pgfpathlineto{\pgfqpoint{1.007188in}{1.336960in}}%
\pgfpathlineto{\pgfqpoint{1.020552in}{1.327001in}}%
\pgfpathlineto{\pgfqpoint{1.030763in}{1.319502in}}%
\pgfpathlineto{\pgfqpoint{1.034105in}{1.317041in}}%
\pgfpathlineto{\pgfqpoint{1.047930in}{1.307081in}}%
\pgfpathlineto{\pgfqpoint{1.060641in}{1.298024in}}%
\pgfpathlineto{\pgfqpoint{1.061905in}{1.297122in}}%
\pgfpathlineto{\pgfqpoint{1.076159in}{1.287162in}}%
\pgfpathlineto{\pgfqpoint{1.090520in}{1.277216in}}%
\pgfpathlineto{\pgfqpoint{1.090539in}{1.277203in}}%
\pgfpathlineto{\pgfqpoint{1.105194in}{1.267243in}}%
\pgfpathlineto{\pgfqpoint{1.119970in}{1.257284in}}%
\pgfpathlineto{\pgfqpoint{1.120399in}{1.257000in}}%
\pgfpathlineto{\pgfqpoint{1.134998in}{1.247324in}}%
\pgfpathlineto{\pgfqpoint{1.150149in}{1.237365in}}%
\pgfpathlineto{\pgfqpoint{1.150277in}{1.237282in}}%
\pgfpathlineto{\pgfqpoint{1.165537in}{1.227405in}}%
\pgfpathlineto{\pgfqpoint{1.180156in}{1.218021in}}%
\pgfpathlineto{\pgfqpoint{1.181050in}{1.217445in}}%
\pgfpathlineto{\pgfqpoint{1.196781in}{1.207486in}}%
\pgfpathlineto{\pgfqpoint{1.210035in}{1.199169in}}%
\pgfpathlineto{\pgfqpoint{1.212644in}{1.197526in}}%
\pgfpathlineto{\pgfqpoint{1.228702in}{1.187567in}}%
\pgfpathlineto{\pgfqpoint{1.239914in}{1.180681in}}%
\pgfpathlineto{\pgfqpoint{1.244904in}{1.177607in}}%
\pgfpathlineto{\pgfqpoint{1.261278in}{1.167648in}}%
\pgfpathlineto{\pgfqpoint{1.269792in}{1.162525in}}%
\pgfpathlineto{\pgfqpoint{1.277807in}{1.157688in}}%
\pgfpathlineto{\pgfqpoint{1.294489in}{1.147729in}}%
\pgfpathlineto{\pgfqpoint{1.299671in}{1.144670in}}%
\pgfpathlineto{\pgfqpoint{1.311334in}{1.137769in}}%
\pgfpathlineto{\pgfqpoint{1.328316in}{1.127809in}}%
\pgfpathlineto{\pgfqpoint{1.329550in}{1.127095in}}%
\pgfpathlineto{\pgfqpoint{1.345468in}{1.117850in}}%
\pgfpathlineto{\pgfqpoint{1.359428in}{1.109810in}}%
\pgfpathlineto{\pgfqpoint{1.362754in}{1.107890in}}%
\pgfpathlineto{\pgfqpoint{1.380196in}{1.097931in}}%
\pgfpathlineto{\pgfqpoint{1.389307in}{1.092776in}}%
\pgfpathlineto{\pgfqpoint{1.397780in}{1.087971in}}%
\pgfpathlineto{\pgfqpoint{1.415506in}{1.078012in}}%
\pgfpathlineto{\pgfqpoint{1.419186in}{1.075965in}}%
\pgfpathlineto{\pgfqpoint{1.433381in}{1.068052in}}%
\pgfpathlineto{\pgfqpoint{1.449064in}{1.059381in}}%
\pgfpathlineto{\pgfqpoint{1.451390in}{1.058092in}}%
\pgfpathlineto{\pgfqpoint{1.469546in}{1.048133in}}%
\pgfpathlineto{\pgfqpoint{1.478943in}{1.043021in}}%
\pgfpathlineto{\pgfqpoint{1.487839in}{1.038173in}}%
\pgfpathlineto{\pgfqpoint{1.506269in}{1.028214in}}%
\pgfpathlineto{\pgfqpoint{1.508822in}{1.026847in}}%
\pgfpathlineto{\pgfqpoint{1.524841in}{1.018254in}}%
\pgfpathlineto{\pgfqpoint{1.538700in}{1.010878in}}%
\pgfpathlineto{\pgfqpoint{1.543547in}{1.008295in}}%
\pgfpathlineto{\pgfqpoint{1.562390in}{0.998335in}}%
\pgfpathlineto{\pgfqpoint{1.568579in}{0.995090in}}%
\pgfpathlineto{\pgfqpoint{1.581369in}{0.988376in}}%
\pgfpathlineto{\pgfqpoint{1.598458in}{0.979472in}}%
\pgfpathlineto{\pgfqpoint{1.600483in}{0.978416in}}%
\pgfpathlineto{\pgfqpoint{1.619732in}{0.968456in}}%
\pgfpathlineto{\pgfqpoint{1.628336in}{0.964037in}}%
\pgfpathlineto{\pgfqpoint{1.639115in}{0.958497in}}%
\pgfpathlineto{\pgfqpoint{1.658215in}{0.948751in}}%
\pgfpathlineto{\pgfqpoint{1.658633in}{0.948537in}}%
\pgfpathlineto{\pgfqpoint{1.678285in}{0.938578in}}%
\pgfpathlineto{\pgfqpoint{1.688094in}{0.933641in}}%
\pgfpathlineto{\pgfqpoint{1.698070in}{0.928618in}}%
\pgfpathlineto{\pgfqpoint{1.717972in}{0.918667in}}%
\pgfpathlineto{\pgfqpoint{1.717989in}{0.918659in}}%
\pgfpathlineto{\pgfqpoint{1.738042in}{0.908699in}}%
\pgfpathlineto{\pgfqpoint{1.747851in}{0.903859in}}%
\pgfpathlineto{\pgfqpoint{1.758229in}{0.898740in}}%
\pgfpathlineto{\pgfqpoint{1.777730in}{0.889181in}}%
\pgfpathlineto{\pgfqpoint{1.778549in}{0.888780in}}%
\pgfpathlineto{\pgfqpoint{1.799004in}{0.878820in}}%
\pgfpathlineto{\pgfqpoint{1.807608in}{0.874656in}}%
\pgfpathlineto{\pgfqpoint{1.819594in}{0.868861in}}%
\pgfpathlineto{\pgfqpoint{1.837487in}{0.860260in}}%
\pgfpathlineto{\pgfqpoint{1.840317in}{0.858901in}}%
\pgfpathlineto{\pgfqpoint{1.861177in}{0.848942in}}%
\pgfpathlineto{\pgfqpoint{1.867366in}{0.846003in}}%
\pgfpathlineto{\pgfqpoint{1.882173in}{0.838982in}}%
\pgfpathlineto{\pgfqpoint{1.897244in}{0.831875in}}%
\pgfpathlineto{\pgfqpoint{1.903303in}{0.829023in}}%
\pgfpathlineto{\pgfqpoint{1.924571in}{0.819063in}}%
\pgfpathlineto{\pgfqpoint{1.927123in}{0.817873in}}%
\pgfpathlineto{\pgfqpoint{1.945979in}{0.809103in}}%
\pgfpathlineto{\pgfqpoint{1.957002in}{0.804002in}}%
\pgfpathlineto{\pgfqpoint{1.967525in}{0.799144in}}%
\pgfpathlineto{\pgfqpoint{1.986880in}{0.790250in}}%
\pgfpathlineto{\pgfqpoint{1.989206in}{0.789184in}}%
\pgfpathlineto{\pgfqpoint{2.011033in}{0.779225in}}%
\pgfpathlineto{\pgfqpoint{2.016759in}{0.776623in}}%
\pgfpathlineto{\pgfqpoint{2.033002in}{0.769265in}}%
\pgfpathlineto{\pgfqpoint{2.046638in}{0.763114in}}%
\pgfpathlineto{\pgfqpoint{2.055111in}{0.759306in}}%
\pgfpathlineto{\pgfqpoint{2.076517in}{0.749722in}}%
\pgfpathlineto{\pgfqpoint{2.077359in}{0.749346in}}%
\pgfpathlineto{\pgfqpoint{2.099764in}{0.739387in}}%
\pgfpathlineto{\pgfqpoint{2.106395in}{0.736449in}}%
\pgfpathlineto{\pgfqpoint{2.122314in}{0.729427in}}%
\pgfpathlineto{\pgfqpoint{2.136274in}{0.723290in}}%
\pgfpathlineto{\pgfqpoint{2.145008in}{0.719467in}}%
\pgfpathlineto{\pgfqpoint{2.166153in}{0.710243in}}%
\pgfpathlineto{\pgfqpoint{2.167846in}{0.709508in}}%
\pgfpathlineto{\pgfqpoint{2.190849in}{0.699548in}}%
\pgfpathlineto{\pgfqpoint{2.196031in}{0.697310in}}%
\pgfpathlineto{\pgfqpoint{2.214008in}{0.689589in}}%
\pgfpathlineto{\pgfqpoint{2.225910in}{0.684489in}}%
\pgfpathlineto{\pgfqpoint{2.237318in}{0.679629in}}%
\pgfpathlineto{\pgfqpoint{2.255789in}{0.671778in}}%
\pgfpathlineto{\pgfqpoint{2.260779in}{0.669670in}}%
\pgfpathlineto{\pgfqpoint{2.284400in}{0.659710in}}%
\pgfpathlineto{\pgfqpoint{2.285667in}{0.659177in}}%
\pgfpathlineto{\pgfqpoint{2.308205in}{0.649751in}}%
\pgfpathlineto{\pgfqpoint{2.315546in}{0.646685in}}%
\pgfpathlineto{\pgfqpoint{2.332171in}{0.639791in}}%
\pgfpathlineto{\pgfqpoint{2.345425in}{0.634302in}}%
\pgfpathlineto{\pgfqpoint{2.356299in}{0.629831in}}%
\pgfpathlineto{\pgfqpoint{2.375303in}{0.622027in}}%
\pgfpathlineto{\pgfqpoint{2.380593in}{0.619872in}}%
\pgfpathlineto{\pgfqpoint{2.405053in}{0.609912in}}%
\pgfpathlineto{\pgfqpoint{2.405182in}{0.609860in}}%
\pgfpathlineto{\pgfqpoint{2.429728in}{0.599953in}}%
\pgfpathlineto{\pgfqpoint{2.435061in}{0.597801in}}%
\pgfpathlineto{\pgfqpoint{2.454580in}{0.589993in}}%
\pgfpathlineto{\pgfqpoint{2.464939in}{0.585849in}}%
\pgfpathlineto{\pgfqpoint{2.479613in}{0.580034in}}%
\pgfpathlineto{\pgfqpoint{2.494818in}{0.574006in}}%
\pgfpathlineto{\pgfqpoint{2.504832in}{0.570074in}}%
\pgfpathlineto{\pgfqpoint{2.524697in}{0.562270in}}%
\pgfpathlineto{\pgfqpoint{2.530239in}{0.560115in}}%
\pgfpathlineto{\pgfqpoint{2.554575in}{0.550641in}}%
\pgfpathlineto{\pgfqpoint{2.555839in}{0.550155in}}%
\pgfpathlineto{\pgfqpoint{2.581672in}{0.540195in}}%
\pgfpathlineto{\pgfqpoint{2.584454in}{0.539122in}}%
\pgfpathlineto{\pgfqpoint{2.607729in}{0.530236in}}%
\pgfpathlineto{\pgfqpoint{2.614333in}{0.527711in}}%
\pgfpathlineto{\pgfqpoint{2.634000in}{0.520276in}}%
\pgfpathlineto{\pgfqpoint{2.644211in}{0.516409in}}%
\pgfpathlineto{\pgfqpoint{2.660493in}{0.510317in}}%
\pgfpathlineto{\pgfqpoint{2.674090in}{0.505217in}}%
\pgfpathlineto{\pgfqpoint{2.687214in}{0.500357in}}%
\pgfpathlineto{\pgfqpoint{2.703969in}{0.494136in}}%
\pgfpathlineto{\pgfqpoint{2.714171in}{0.490398in}}%
\pgfpathlineto{\pgfqpoint{2.733847in}{0.483166in}}%
\pgfpathlineto{\pgfqpoint{2.741374in}{0.480438in}}%
\pgfpathlineto{\pgfqpoint{2.763726in}{0.472308in}}%
\pgfpathlineto{\pgfqpoint{2.768831in}{0.470478in}}%
\pgfpathlineto{\pgfqpoint{2.793605in}{0.461564in}}%
\pgfpathlineto{\pgfqpoint{2.796555in}{0.460519in}}%
\pgfpathlineto{\pgfqpoint{2.823483in}{0.450935in}}%
\pgfpathlineto{\pgfqpoint{2.824557in}{0.450559in}}%
\pgfpathlineto{\pgfqpoint{2.852861in}{0.440600in}}%
\pgfpathlineto{\pgfqpoint{2.853362in}{0.440423in}}%
\pgfpathlineto{\pgfqpoint{2.881491in}{0.430640in}}%
\pgfpathlineto{\pgfqpoint{2.883241in}{0.430029in}}%
\pgfpathlineto{\pgfqpoint{2.910438in}{0.420681in}}%
\pgfpathlineto{\pgfqpoint{2.913120in}{0.419754in}}%
\pgfpathlineto{\pgfqpoint{2.939719in}{0.410721in}}%
\pgfpathlineto{\pgfqpoint{2.942998in}{0.409601in}}%
\pgfpathlineto{\pgfqpoint{2.969351in}{0.400762in}}%
\pgfpathlineto{\pgfqpoint{2.972877in}{0.399572in}}%
\pgfpathlineto{\pgfqpoint{2.999354in}{0.390802in}}%
\pgfpathlineto{\pgfqpoint{3.002756in}{0.389668in}}%
\pgfpathlineto{\pgfqpoint{3.029749in}{0.380842in}}%
\pgfpathlineto{\pgfqpoint{3.032634in}{0.379893in}}%
\pgfpathlineto{\pgfqpoint{3.060560in}{0.370883in}}%
\pgfpathlineto{\pgfqpoint{3.062513in}{0.370248in}}%
\pgfpathlineto{\pgfqpoint{3.091811in}{0.360923in}}%
\pgfpathlineto{\pgfqpoint{3.092392in}{0.360737in}}%
\pgfpathlineto{\pgfqpoint{3.122270in}{0.351365in}}%
\pgfpathlineto{\pgfqpoint{3.123580in}{0.350964in}}%
\pgfpathlineto{\pgfqpoint{3.152149in}{0.342135in}}%
\pgfpathlineto{\pgfqpoint{3.155898in}{0.341004in}}%
\pgfpathlineto{\pgfqpoint{3.182028in}{0.333050in}}%
\pgfpathlineto{\pgfqpoint{3.188784in}{0.331045in}}%
\pgfpathlineto{\pgfqpoint{3.211906in}{0.324114in}}%
\pgfpathlineto{\pgfqpoint{3.222279in}{0.321085in}}%
\pgfpathlineto{\pgfqpoint{3.241785in}{0.315331in}}%
\pgfpathlineto{\pgfqpoint{3.256432in}{0.311126in}}%
\pgfpathlineto{\pgfqpoint{3.271664in}{0.306706in}}%
\pgfpathlineto{\pgfqpoint{3.291296in}{0.301166in}}%
\pgfpathlineto{\pgfqpoint{3.301542in}{0.298243in}}%
\pgfpathlineto{\pgfqpoint{3.326930in}{0.291206in}}%
\pgfpathlineto{\pgfqpoint{3.331421in}{0.289948in}}%
\pgfpathlineto{\pgfqpoint{3.361300in}{0.281831in}}%
\pgfpathlineto{\pgfqpoint{3.363527in}{0.281247in}}%
\pgfpathlineto{\pgfqpoint{3.391178in}{0.273909in}}%
\pgfpathlineto{\pgfqpoint{3.401394in}{0.271287in}}%
\pgfpathlineto{\pgfqpoint{3.421057in}{0.266176in}}%
\pgfpathlineto{\pgfqpoint{3.440364in}{0.261328in}}%
\pgfpathlineto{\pgfqpoint{3.450936in}{0.258638in}}%
\pgfpathlineto{\pgfqpoint{3.480548in}{0.251368in}}%
\pgfpathlineto{\pgfqpoint{3.480814in}{0.251302in}}%
\pgfpathlineto{\pgfqpoint{3.510693in}{0.244218in}}%
\pgfpathlineto{\pgfqpoint{3.523043in}{0.241409in}}%
\pgfpathlineto{\pgfqpoint{3.540572in}{0.237360in}}%
\pgfpathlineto{\pgfqpoint{3.567272in}{0.231449in}}%
\pgfpathlineto{\pgfqpoint{3.570450in}{0.230735in}}%
\pgfpathlineto{\pgfqpoint{3.600329in}{0.224407in}}%
\pgfpathlineto{\pgfqpoint{3.614857in}{0.221489in}}%
\pgfpathlineto{\pgfqpoint{3.630208in}{0.218354in}}%
\pgfpathlineto{\pgfqpoint{3.660086in}{0.212598in}}%
\pgfpathlineto{\pgfqpoint{3.666098in}{0.211530in}}%
\pgfpathlineto{\pgfqpoint{3.689965in}{0.207210in}}%
\pgfpathlineto{\pgfqpoint{3.719844in}{0.202146in}}%
\pgfpathlineto{\pgfqpoint{3.723642in}{0.201570in}}%
\pgfpathlineto{\pgfqpoint{3.749723in}{0.197538in}}%
\pgfpathlineto{\pgfqpoint{3.779601in}{0.193338in}}%
\pgfpathlineto{\pgfqpoint{3.793790in}{0.191611in}}%
\pgfpathclose%
\pgfpathmoveto{\pgfqpoint{3.793790in}{0.191611in}}%
\pgfpathlineto{\pgfqpoint{3.779601in}{0.193338in}}%
\pgfpathlineto{\pgfqpoint{3.749723in}{0.197538in}}%
\pgfpathlineto{\pgfqpoint{3.723642in}{0.201570in}}%
\pgfpathlineto{\pgfqpoint{3.719844in}{0.202146in}}%
\pgfpathlineto{\pgfqpoint{3.689965in}{0.207210in}}%
\pgfpathlineto{\pgfqpoint{3.666098in}{0.211530in}}%
\pgfpathlineto{\pgfqpoint{3.660086in}{0.212598in}}%
\pgfpathlineto{\pgfqpoint{3.630208in}{0.218354in}}%
\pgfpathlineto{\pgfqpoint{3.614857in}{0.221489in}}%
\pgfpathlineto{\pgfqpoint{3.600329in}{0.224407in}}%
\pgfpathlineto{\pgfqpoint{3.570450in}{0.230735in}}%
\pgfpathlineto{\pgfqpoint{3.567272in}{0.231449in}}%
\pgfpathlineto{\pgfqpoint{3.540572in}{0.237360in}}%
\pgfpathlineto{\pgfqpoint{3.523043in}{0.241409in}}%
\pgfpathlineto{\pgfqpoint{3.510693in}{0.244218in}}%
\pgfpathlineto{\pgfqpoint{3.480814in}{0.251302in}}%
\pgfpathlineto{\pgfqpoint{3.480548in}{0.251368in}}%
\pgfpathlineto{\pgfqpoint{3.450936in}{0.258638in}}%
\pgfpathlineto{\pgfqpoint{3.440364in}{0.261328in}}%
\pgfpathlineto{\pgfqpoint{3.421057in}{0.266176in}}%
\pgfpathlineto{\pgfqpoint{3.401394in}{0.271287in}}%
\pgfpathlineto{\pgfqpoint{3.391178in}{0.273909in}}%
\pgfpathlineto{\pgfqpoint{3.363527in}{0.281247in}}%
\pgfpathlineto{\pgfqpoint{3.361300in}{0.281831in}}%
\pgfpathlineto{\pgfqpoint{3.331421in}{0.289948in}}%
\pgfpathlineto{\pgfqpoint{3.326930in}{0.291206in}}%
\pgfpathlineto{\pgfqpoint{3.301542in}{0.298243in}}%
\pgfpathlineto{\pgfqpoint{3.291296in}{0.301166in}}%
\pgfpathlineto{\pgfqpoint{3.271664in}{0.306706in}}%
\pgfpathlineto{\pgfqpoint{3.256432in}{0.311126in}}%
\pgfpathlineto{\pgfqpoint{3.241785in}{0.315331in}}%
\pgfpathlineto{\pgfqpoint{3.222279in}{0.321085in}}%
\pgfpathlineto{\pgfqpoint{3.211906in}{0.324114in}}%
\pgfpathlineto{\pgfqpoint{3.188784in}{0.331045in}}%
\pgfpathlineto{\pgfqpoint{3.182028in}{0.333050in}}%
\pgfpathlineto{\pgfqpoint{3.155898in}{0.341004in}}%
\pgfpathlineto{\pgfqpoint{3.152149in}{0.342135in}}%
\pgfpathlineto{\pgfqpoint{3.123580in}{0.350964in}}%
\pgfpathlineto{\pgfqpoint{3.122270in}{0.351365in}}%
\pgfpathlineto{\pgfqpoint{3.092392in}{0.360737in}}%
\pgfpathlineto{\pgfqpoint{3.091811in}{0.360923in}}%
\pgfpathlineto{\pgfqpoint{3.062513in}{0.370248in}}%
\pgfpathlineto{\pgfqpoint{3.060560in}{0.370883in}}%
\pgfpathlineto{\pgfqpoint{3.032634in}{0.379893in}}%
\pgfpathlineto{\pgfqpoint{3.029749in}{0.380842in}}%
\pgfpathlineto{\pgfqpoint{3.002756in}{0.389668in}}%
\pgfpathlineto{\pgfqpoint{2.999354in}{0.390802in}}%
\pgfpathlineto{\pgfqpoint{2.972877in}{0.399572in}}%
\pgfpathlineto{\pgfqpoint{2.969351in}{0.400762in}}%
\pgfpathlineto{\pgfqpoint{2.942998in}{0.409601in}}%
\pgfpathlineto{\pgfqpoint{2.939719in}{0.410721in}}%
\pgfpathlineto{\pgfqpoint{2.913120in}{0.419754in}}%
\pgfpathlineto{\pgfqpoint{2.910438in}{0.420681in}}%
\pgfpathlineto{\pgfqpoint{2.883241in}{0.430029in}}%
\pgfpathlineto{\pgfqpoint{2.881491in}{0.430640in}}%
\pgfpathlineto{\pgfqpoint{2.853362in}{0.440423in}}%
\pgfpathlineto{\pgfqpoint{2.852861in}{0.440600in}}%
\pgfpathlineto{\pgfqpoint{2.824557in}{0.450559in}}%
\pgfpathlineto{\pgfqpoint{2.823483in}{0.450935in}}%
\pgfpathlineto{\pgfqpoint{2.796555in}{0.460519in}}%
\pgfpathlineto{\pgfqpoint{2.793605in}{0.461564in}}%
\pgfpathlineto{\pgfqpoint{2.768831in}{0.470478in}}%
\pgfpathlineto{\pgfqpoint{2.763726in}{0.472308in}}%
\pgfpathlineto{\pgfqpoint{2.741374in}{0.480438in}}%
\pgfpathlineto{\pgfqpoint{2.733847in}{0.483166in}}%
\pgfpathlineto{\pgfqpoint{2.714171in}{0.490398in}}%
\pgfpathlineto{\pgfqpoint{2.703969in}{0.494136in}}%
\pgfpathlineto{\pgfqpoint{2.687214in}{0.500357in}}%
\pgfpathlineto{\pgfqpoint{2.674090in}{0.505217in}}%
\pgfpathlineto{\pgfqpoint{2.660493in}{0.510317in}}%
\pgfpathlineto{\pgfqpoint{2.644211in}{0.516409in}}%
\pgfpathlineto{\pgfqpoint{2.634000in}{0.520276in}}%
\pgfpathlineto{\pgfqpoint{2.614333in}{0.527711in}}%
\pgfpathlineto{\pgfqpoint{2.607729in}{0.530236in}}%
\pgfpathlineto{\pgfqpoint{2.584454in}{0.539122in}}%
\pgfpathlineto{\pgfqpoint{2.581672in}{0.540195in}}%
\pgfpathlineto{\pgfqpoint{2.555839in}{0.550155in}}%
\pgfpathlineto{\pgfqpoint{2.554575in}{0.550641in}}%
\pgfpathlineto{\pgfqpoint{2.530239in}{0.560115in}}%
\pgfpathlineto{\pgfqpoint{2.524697in}{0.562270in}}%
\pgfpathlineto{\pgfqpoint{2.504832in}{0.570074in}}%
\pgfpathlineto{\pgfqpoint{2.494818in}{0.574006in}}%
\pgfpathlineto{\pgfqpoint{2.479613in}{0.580034in}}%
\pgfpathlineto{\pgfqpoint{2.464939in}{0.585849in}}%
\pgfpathlineto{\pgfqpoint{2.454580in}{0.589993in}}%
\pgfpathlineto{\pgfqpoint{2.435061in}{0.597801in}}%
\pgfpathlineto{\pgfqpoint{2.429728in}{0.599953in}}%
\pgfpathlineto{\pgfqpoint{2.405182in}{0.609860in}}%
\pgfpathlineto{\pgfqpoint{2.405053in}{0.609912in}}%
\pgfpathlineto{\pgfqpoint{2.380593in}{0.619872in}}%
\pgfpathlineto{\pgfqpoint{2.375303in}{0.622027in}}%
\pgfpathlineto{\pgfqpoint{2.356299in}{0.629831in}}%
\pgfpathlineto{\pgfqpoint{2.345425in}{0.634302in}}%
\pgfpathlineto{\pgfqpoint{2.332171in}{0.639791in}}%
\pgfpathlineto{\pgfqpoint{2.315546in}{0.646685in}}%
\pgfpathlineto{\pgfqpoint{2.308205in}{0.649751in}}%
\pgfpathlineto{\pgfqpoint{2.285667in}{0.659177in}}%
\pgfpathlineto{\pgfqpoint{2.284400in}{0.659710in}}%
\pgfpathlineto{\pgfqpoint{2.260779in}{0.669670in}}%
\pgfpathlineto{\pgfqpoint{2.255789in}{0.671778in}}%
\pgfpathlineto{\pgfqpoint{2.237318in}{0.679629in}}%
\pgfpathlineto{\pgfqpoint{2.225910in}{0.684489in}}%
\pgfpathlineto{\pgfqpoint{2.214008in}{0.689589in}}%
\pgfpathlineto{\pgfqpoint{2.196031in}{0.697310in}}%
\pgfpathlineto{\pgfqpoint{2.190849in}{0.699548in}}%
\pgfpathlineto{\pgfqpoint{2.167846in}{0.709508in}}%
\pgfpathlineto{\pgfqpoint{2.166153in}{0.710243in}}%
\pgfpathlineto{\pgfqpoint{2.145008in}{0.719467in}}%
\pgfpathlineto{\pgfqpoint{2.136274in}{0.723290in}}%
\pgfpathlineto{\pgfqpoint{2.122314in}{0.729427in}}%
\pgfpathlineto{\pgfqpoint{2.106395in}{0.736449in}}%
\pgfpathlineto{\pgfqpoint{2.099764in}{0.739387in}}%
\pgfpathlineto{\pgfqpoint{2.077359in}{0.749346in}}%
\pgfpathlineto{\pgfqpoint{2.076517in}{0.749722in}}%
\pgfpathlineto{\pgfqpoint{2.055111in}{0.759306in}}%
\pgfpathlineto{\pgfqpoint{2.046638in}{0.763114in}}%
\pgfpathlineto{\pgfqpoint{2.033002in}{0.769265in}}%
\pgfpathlineto{\pgfqpoint{2.016759in}{0.776623in}}%
\pgfpathlineto{\pgfqpoint{2.011033in}{0.779225in}}%
\pgfpathlineto{\pgfqpoint{1.989206in}{0.789184in}}%
\pgfpathlineto{\pgfqpoint{1.986880in}{0.790250in}}%
\pgfpathlineto{\pgfqpoint{1.967525in}{0.799144in}}%
\pgfpathlineto{\pgfqpoint{1.957002in}{0.804002in}}%
\pgfpathlineto{\pgfqpoint{1.945979in}{0.809103in}}%
\pgfpathlineto{\pgfqpoint{1.927123in}{0.817873in}}%
\pgfpathlineto{\pgfqpoint{1.924571in}{0.819063in}}%
\pgfpathlineto{\pgfqpoint{1.903303in}{0.829023in}}%
\pgfpathlineto{\pgfqpoint{1.897244in}{0.831875in}}%
\pgfpathlineto{\pgfqpoint{1.882173in}{0.838982in}}%
\pgfpathlineto{\pgfqpoint{1.867366in}{0.846003in}}%
\pgfpathlineto{\pgfqpoint{1.861177in}{0.848942in}}%
\pgfpathlineto{\pgfqpoint{1.840317in}{0.858901in}}%
\pgfpathlineto{\pgfqpoint{1.837487in}{0.860260in}}%
\pgfpathlineto{\pgfqpoint{1.819594in}{0.868861in}}%
\pgfpathlineto{\pgfqpoint{1.807608in}{0.874656in}}%
\pgfpathlineto{\pgfqpoint{1.799004in}{0.878820in}}%
\pgfpathlineto{\pgfqpoint{1.778549in}{0.888780in}}%
\pgfpathlineto{\pgfqpoint{1.777730in}{0.889181in}}%
\pgfpathlineto{\pgfqpoint{1.758229in}{0.898740in}}%
\pgfpathlineto{\pgfqpoint{1.747851in}{0.903859in}}%
\pgfpathlineto{\pgfqpoint{1.738042in}{0.908699in}}%
\pgfpathlineto{\pgfqpoint{1.717989in}{0.918659in}}%
\pgfpathlineto{\pgfqpoint{1.717972in}{0.918667in}}%
\pgfpathlineto{\pgfqpoint{1.698070in}{0.928618in}}%
\pgfpathlineto{\pgfqpoint{1.688094in}{0.933641in}}%
\pgfpathlineto{\pgfqpoint{1.678285in}{0.938578in}}%
\pgfpathlineto{\pgfqpoint{1.658633in}{0.948537in}}%
\pgfpathlineto{\pgfqpoint{1.658215in}{0.948751in}}%
\pgfpathlineto{\pgfqpoint{1.639115in}{0.958497in}}%
\pgfpathlineto{\pgfqpoint{1.628336in}{0.964037in}}%
\pgfpathlineto{\pgfqpoint{1.619732in}{0.968456in}}%
\pgfpathlineto{\pgfqpoint{1.600483in}{0.978416in}}%
\pgfpathlineto{\pgfqpoint{1.598458in}{0.979472in}}%
\pgfpathlineto{\pgfqpoint{1.581369in}{0.988376in}}%
\pgfpathlineto{\pgfqpoint{1.568579in}{0.995090in}}%
\pgfpathlineto{\pgfqpoint{1.562390in}{0.998335in}}%
\pgfpathlineto{\pgfqpoint{1.543547in}{1.008295in}}%
\pgfpathlineto{\pgfqpoint{1.538700in}{1.010878in}}%
\pgfpathlineto{\pgfqpoint{1.524841in}{1.018254in}}%
\pgfpathlineto{\pgfqpoint{1.508822in}{1.026847in}}%
\pgfpathlineto{\pgfqpoint{1.506269in}{1.028214in}}%
\pgfpathlineto{\pgfqpoint{1.487839in}{1.038173in}}%
\pgfpathlineto{\pgfqpoint{1.478943in}{1.043021in}}%
\pgfpathlineto{\pgfqpoint{1.469546in}{1.048133in}}%
\pgfpathlineto{\pgfqpoint{1.451390in}{1.058092in}}%
\pgfpathlineto{\pgfqpoint{1.449064in}{1.059381in}}%
\pgfpathlineto{\pgfqpoint{1.433381in}{1.068052in}}%
\pgfpathlineto{\pgfqpoint{1.419186in}{1.075965in}}%
\pgfpathlineto{\pgfqpoint{1.415506in}{1.078012in}}%
\pgfpathlineto{\pgfqpoint{1.397780in}{1.087971in}}%
\pgfpathlineto{\pgfqpoint{1.389307in}{1.092776in}}%
\pgfpathlineto{\pgfqpoint{1.380196in}{1.097931in}}%
\pgfpathlineto{\pgfqpoint{1.362754in}{1.107890in}}%
\pgfpathlineto{\pgfqpoint{1.359428in}{1.109810in}}%
\pgfpathlineto{\pgfqpoint{1.345468in}{1.117850in}}%
\pgfpathlineto{\pgfqpoint{1.329550in}{1.127095in}}%
\pgfpathlineto{\pgfqpoint{1.328316in}{1.127809in}}%
\pgfpathlineto{\pgfqpoint{1.311334in}{1.137769in}}%
\pgfpathlineto{\pgfqpoint{1.299671in}{1.144670in}}%
\pgfpathlineto{\pgfqpoint{1.294489in}{1.147729in}}%
\pgfpathlineto{\pgfqpoint{1.277807in}{1.157688in}}%
\pgfpathlineto{\pgfqpoint{1.269792in}{1.162525in}}%
\pgfpathlineto{\pgfqpoint{1.261278in}{1.167648in}}%
\pgfpathlineto{\pgfqpoint{1.244904in}{1.177607in}}%
\pgfpathlineto{\pgfqpoint{1.239914in}{1.180681in}}%
\pgfpathlineto{\pgfqpoint{1.228702in}{1.187567in}}%
\pgfpathlineto{\pgfqpoint{1.212644in}{1.197526in}}%
\pgfpathlineto{\pgfqpoint{1.210035in}{1.199169in}}%
\pgfpathlineto{\pgfqpoint{1.196781in}{1.207486in}}%
\pgfpathlineto{\pgfqpoint{1.181050in}{1.217445in}}%
\pgfpathlineto{\pgfqpoint{1.180156in}{1.218021in}}%
\pgfpathlineto{\pgfqpoint{1.165537in}{1.227405in}}%
\pgfpathlineto{\pgfqpoint{1.150277in}{1.237282in}}%
\pgfpathlineto{\pgfqpoint{1.150149in}{1.237365in}}%
\pgfpathlineto{\pgfqpoint{1.134998in}{1.247324in}}%
\pgfpathlineto{\pgfqpoint{1.120399in}{1.257000in}}%
\pgfpathlineto{\pgfqpoint{1.119970in}{1.257284in}}%
\pgfpathlineto{\pgfqpoint{1.105194in}{1.267243in}}%
\pgfpathlineto{\pgfqpoint{1.090539in}{1.277203in}}%
\pgfpathlineto{\pgfqpoint{1.090520in}{1.277216in}}%
\pgfpathlineto{\pgfqpoint{1.076159in}{1.287162in}}%
\pgfpathlineto{\pgfqpoint{1.061905in}{1.297122in}}%
\pgfpathlineto{\pgfqpoint{1.060641in}{1.298024in}}%
\pgfpathlineto{\pgfqpoint{1.047930in}{1.307081in}}%
\pgfpathlineto{\pgfqpoint{1.034105in}{1.317041in}}%
\pgfpathlineto{\pgfqpoint{1.030763in}{1.319502in}}%
\pgfpathlineto{\pgfqpoint{1.020552in}{1.327001in}}%
\pgfpathlineto{\pgfqpoint{1.007188in}{1.336960in}}%
\pgfpathlineto{\pgfqpoint{1.000884in}{1.341753in}}%
\pgfpathlineto{\pgfqpoint{0.994071in}{1.346920in}}%
\pgfpathlineto{\pgfqpoint{0.981208in}{1.356879in}}%
\pgfpathlineto{\pgfqpoint{0.971005in}{1.364904in}}%
\pgfpathlineto{\pgfqpoint{0.968540in}{1.366839in}}%
\pgfpathlineto{\pgfqpoint{0.956221in}{1.376798in}}%
\pgfpathlineto{\pgfqpoint{0.944077in}{1.386758in}}%
\pgfpathlineto{\pgfqpoint{0.941127in}{1.389255in}}%
\pgfpathlineto{\pgfqpoint{0.932294in}{1.396718in}}%
\pgfpathlineto{\pgfqpoint{0.920775in}{1.406677in}}%
\pgfpathlineto{\pgfqpoint{0.911248in}{1.415092in}}%
\pgfpathlineto{\pgfqpoint{0.909498in}{1.416637in}}%
\pgfpathlineto{\pgfqpoint{0.898668in}{1.426596in}}%
\pgfpathlineto{\pgfqpoint{0.888082in}{1.436556in}}%
\pgfpathlineto{\pgfqpoint{0.881369in}{1.443086in}}%
\pgfpathlineto{\pgfqpoint{0.877844in}{1.446515in}}%
\pgfpathlineto{\pgfqpoint{0.868074in}{1.456475in}}%
\pgfpathlineto{\pgfqpoint{0.858612in}{1.466434in}}%
\pgfpathlineto{\pgfqpoint{0.851491in}{1.474249in}}%
\pgfpathlineto{\pgfqpoint{0.849538in}{1.476394in}}%
\pgfpathlineto{\pgfqpoint{0.841061in}{1.486354in}}%
\pgfpathlineto{\pgfqpoint{0.832991in}{1.496313in}}%
\pgfpathlineto{\pgfqpoint{0.825362in}{1.506273in}}%
\pgfpathlineto{\pgfqpoint{0.821612in}{1.511595in}}%
\pgfpathlineto{\pgfqpoint{0.818356in}{1.516232in}}%
\pgfpathlineto{\pgfqpoint{0.812057in}{1.526192in}}%
\pgfpathlineto{\pgfqpoint{0.806380in}{1.536151in}}%
\pgfpathlineto{\pgfqpoint{0.801404in}{1.546111in}}%
\pgfpathlineto{\pgfqpoint{0.797223in}{1.556070in}}%
\pgfpathlineto{\pgfqpoint{0.793961in}{1.566030in}}%
\pgfpathlineto{\pgfqpoint{0.791770in}{1.575990in}}%
\pgfpathlineto{\pgfqpoint{0.791733in}{1.576404in}}%
\pgfpathlineto{\pgfqpoint{0.790908in}{1.585949in}}%
\pgfpathlineto{\pgfqpoint{0.791467in}{1.595909in}}%
\pgfpathlineto{\pgfqpoint{0.791733in}{1.597154in}}%
\pgfpathlineto{\pgfqpoint{0.793897in}{1.605868in}}%
\pgfpathlineto{\pgfqpoint{0.798639in}{1.615828in}}%
\pgfpathlineto{\pgfqpoint{0.806261in}{1.625787in}}%
\pgfpathlineto{\pgfqpoint{0.817574in}{1.635747in}}%
\pgfpathlineto{\pgfqpoint{0.821612in}{1.638431in}}%
\pgfpathlineto{\pgfqpoint{0.835712in}{1.645706in}}%
\pgfpathlineto{\pgfqpoint{0.851491in}{1.651799in}}%
\pgfpathlineto{\pgfqpoint{0.865680in}{1.655666in}}%
\pgfpathlineto{\pgfqpoint{0.881369in}{1.659095in}}%
\pgfpathlineto{\pgfqpoint{0.911248in}{1.663166in}}%
\pgfpathlineto{\pgfqpoint{0.941127in}{1.665164in}}%
\pgfpathlineto{\pgfqpoint{0.971005in}{1.665609in}}%
\pgfpathlineto{\pgfqpoint{1.000884in}{1.664859in}}%
\pgfpathlineto{\pgfqpoint{1.030763in}{1.663165in}}%
\pgfpathlineto{\pgfqpoint{1.060641in}{1.660707in}}%
\pgfpathlineto{\pgfqpoint{1.090520in}{1.657618in}}%
\pgfpathlineto{\pgfqpoint{1.106210in}{1.655666in}}%
\pgfpathlineto{\pgfqpoint{1.120399in}{1.653938in}}%
\pgfpathlineto{\pgfqpoint{1.150277in}{1.649739in}}%
\pgfpathlineto{\pgfqpoint{1.176358in}{1.645706in}}%
\pgfpathlineto{\pgfqpoint{1.180156in}{1.645131in}}%
\pgfpathlineto{\pgfqpoint{1.210035in}{1.640067in}}%
\pgfpathlineto{\pgfqpoint{1.233902in}{1.635747in}}%
\pgfpathlineto{\pgfqpoint{1.239914in}{1.634678in}}%
\pgfpathlineto{\pgfqpoint{1.269792in}{1.628922in}}%
\pgfpathlineto{\pgfqpoint{1.285143in}{1.625787in}}%
\pgfpathlineto{\pgfqpoint{1.299671in}{1.622870in}}%
\pgfpathlineto{\pgfqpoint{1.329550in}{1.616542in}}%
\pgfpathlineto{\pgfqpoint{1.332728in}{1.615828in}}%
\pgfpathlineto{\pgfqpoint{1.359428in}{1.609917in}}%
\pgfpathlineto{\pgfqpoint{1.376957in}{1.605868in}}%
\pgfpathlineto{\pgfqpoint{1.389307in}{1.603059in}}%
\pgfpathlineto{\pgfqpoint{1.419186in}{1.595975in}}%
\pgfpathlineto{\pgfqpoint{1.419452in}{1.595909in}}%
\pgfpathlineto{\pgfqpoint{1.449064in}{1.588639in}}%
\pgfpathlineto{\pgfqpoint{1.459636in}{1.585949in}}%
\pgfpathlineto{\pgfqpoint{1.478943in}{1.581101in}}%
\pgfpathlineto{\pgfqpoint{1.498606in}{1.575990in}}%
\pgfpathlineto{\pgfqpoint{1.508822in}{1.573368in}}%
\pgfpathlineto{\pgfqpoint{1.536473in}{1.566030in}}%
\pgfpathlineto{\pgfqpoint{1.538700in}{1.565446in}}%
\pgfpathlineto{\pgfqpoint{1.568579in}{1.557329in}}%
\pgfpathlineto{\pgfqpoint{1.573070in}{1.556070in}}%
\pgfpathlineto{\pgfqpoint{1.598458in}{1.549034in}}%
\pgfpathlineto{\pgfqpoint{1.608704in}{1.546111in}}%
\pgfpathlineto{\pgfqpoint{1.628336in}{1.540571in}}%
\pgfpathlineto{\pgfqpoint{1.643568in}{1.536151in}}%
\pgfpathlineto{\pgfqpoint{1.658215in}{1.531945in}}%
\pgfpathlineto{\pgfqpoint{1.677721in}{1.526192in}}%
\pgfpathlineto{\pgfqpoint{1.688094in}{1.523163in}}%
\pgfpathlineto{\pgfqpoint{1.711216in}{1.516232in}}%
\pgfpathlineto{\pgfqpoint{1.717972in}{1.514227in}}%
\pgfpathlineto{\pgfqpoint{1.744102in}{1.506273in}}%
\pgfpathlineto{\pgfqpoint{1.747851in}{1.505142in}}%
\pgfpathlineto{\pgfqpoint{1.776420in}{1.496313in}}%
\pgfpathlineto{\pgfqpoint{1.777730in}{1.495912in}}%
\pgfpathlineto{\pgfqpoint{1.807608in}{1.486540in}}%
\pgfpathlineto{\pgfqpoint{1.808189in}{1.486354in}}%
\pgfpathlineto{\pgfqpoint{1.837487in}{1.477029in}}%
\pgfpathlineto{\pgfqpoint{1.839440in}{1.476394in}}%
\pgfpathlineto{\pgfqpoint{1.867366in}{1.467384in}}%
\pgfpathlineto{\pgfqpoint{1.870251in}{1.466434in}}%
\pgfpathlineto{\pgfqpoint{1.897244in}{1.457609in}}%
\pgfpathlineto{\pgfqpoint{1.900646in}{1.456475in}}%
\pgfpathlineto{\pgfqpoint{1.927123in}{1.447705in}}%
\pgfpathlineto{\pgfqpoint{1.930649in}{1.446515in}}%
\pgfpathlineto{\pgfqpoint{1.957002in}{1.437676in}}%
\pgfpathlineto{\pgfqpoint{1.960281in}{1.436556in}}%
\pgfpathlineto{\pgfqpoint{1.986880in}{1.427523in}}%
\pgfpathlineto{\pgfqpoint{1.989562in}{1.426596in}}%
\pgfpathlineto{\pgfqpoint{2.016759in}{1.417248in}}%
\pgfpathlineto{\pgfqpoint{2.018509in}{1.416637in}}%
\pgfpathlineto{\pgfqpoint{2.046638in}{1.406854in}}%
\pgfpathlineto{\pgfqpoint{2.047139in}{1.406677in}}%
\pgfpathlineto{\pgfqpoint{2.075443in}{1.396718in}}%
\pgfpathlineto{\pgfqpoint{2.076517in}{1.396342in}}%
\pgfpathlineto{\pgfqpoint{2.103445in}{1.386758in}}%
\pgfpathlineto{\pgfqpoint{2.106395in}{1.385712in}}%
\pgfpathlineto{\pgfqpoint{2.131169in}{1.376798in}}%
\pgfpathlineto{\pgfqpoint{2.136274in}{1.374969in}}%
\pgfpathlineto{\pgfqpoint{2.158626in}{1.366839in}}%
\pgfpathlineto{\pgfqpoint{2.166153in}{1.364111in}}%
\pgfpathlineto{\pgfqpoint{2.185829in}{1.356879in}}%
\pgfpathlineto{\pgfqpoint{2.196031in}{1.353141in}}%
\pgfpathlineto{\pgfqpoint{2.212786in}{1.346920in}}%
\pgfpathlineto{\pgfqpoint{2.225910in}{1.342060in}}%
\pgfpathlineto{\pgfqpoint{2.239507in}{1.336960in}}%
\pgfpathlineto{\pgfqpoint{2.255789in}{1.330868in}}%
\pgfpathlineto{\pgfqpoint{2.266000in}{1.327001in}}%
\pgfpathlineto{\pgfqpoint{2.285667in}{1.319566in}}%
\pgfpathlineto{\pgfqpoint{2.292271in}{1.317041in}}%
\pgfpathlineto{\pgfqpoint{2.315546in}{1.308155in}}%
\pgfpathlineto{\pgfqpoint{2.318328in}{1.307081in}}%
\pgfpathlineto{\pgfqpoint{2.344161in}{1.297122in}}%
\pgfpathlineto{\pgfqpoint{2.345425in}{1.296636in}}%
\pgfpathlineto{\pgfqpoint{2.369761in}{1.287162in}}%
\pgfpathlineto{\pgfqpoint{2.375303in}{1.285007in}}%
\pgfpathlineto{\pgfqpoint{2.395168in}{1.277203in}}%
\pgfpathlineto{\pgfqpoint{2.405182in}{1.273271in}}%
\pgfpathlineto{\pgfqpoint{2.420387in}{1.267243in}}%
\pgfpathlineto{\pgfqpoint{2.435061in}{1.261427in}}%
\pgfpathlineto{\pgfqpoint{2.445420in}{1.257284in}}%
\pgfpathlineto{\pgfqpoint{2.464939in}{1.249476in}}%
\pgfpathlineto{\pgfqpoint{2.470272in}{1.247324in}}%
\pgfpathlineto{\pgfqpoint{2.494818in}{1.237417in}}%
\pgfpathlineto{\pgfqpoint{2.494947in}{1.237365in}}%
\pgfpathlineto{\pgfqpoint{2.519407in}{1.227405in}}%
\pgfpathlineto{\pgfqpoint{2.524697in}{1.225250in}}%
\pgfpathlineto{\pgfqpoint{2.543701in}{1.217445in}}%
\pgfpathlineto{\pgfqpoint{2.554575in}{1.212975in}}%
\pgfpathlineto{\pgfqpoint{2.567829in}{1.207486in}}%
\pgfpathlineto{\pgfqpoint{2.584454in}{1.200592in}}%
\pgfpathlineto{\pgfqpoint{2.591795in}{1.197526in}}%
\pgfpathlineto{\pgfqpoint{2.614333in}{1.188100in}}%
\pgfpathlineto{\pgfqpoint{2.615600in}{1.187567in}}%
\pgfpathlineto{\pgfqpoint{2.639221in}{1.177607in}}%
\pgfpathlineto{\pgfqpoint{2.644211in}{1.175499in}}%
\pgfpathlineto{\pgfqpoint{2.662682in}{1.167648in}}%
\pgfpathlineto{\pgfqpoint{2.674090in}{1.162788in}}%
\pgfpathlineto{\pgfqpoint{2.685992in}{1.157688in}}%
\pgfpathlineto{\pgfqpoint{2.703969in}{1.149966in}}%
\pgfpathlineto{\pgfqpoint{2.709151in}{1.147729in}}%
\pgfpathlineto{\pgfqpoint{2.732154in}{1.137769in}}%
\pgfpathlineto{\pgfqpoint{2.733847in}{1.137034in}}%
\pgfpathlineto{\pgfqpoint{2.754992in}{1.127809in}}%
\pgfpathlineto{\pgfqpoint{2.763726in}{1.123987in}}%
\pgfpathlineto{\pgfqpoint{2.777686in}{1.117850in}}%
\pgfpathlineto{\pgfqpoint{2.793605in}{1.110828in}}%
\pgfpathlineto{\pgfqpoint{2.800236in}{1.107890in}}%
\pgfpathlineto{\pgfqpoint{2.822641in}{1.097931in}}%
\pgfpathlineto{\pgfqpoint{2.823483in}{1.097555in}}%
\pgfpathlineto{\pgfqpoint{2.844889in}{1.087971in}}%
\pgfpathlineto{\pgfqpoint{2.853362in}{1.084162in}}%
\pgfpathlineto{\pgfqpoint{2.866998in}{1.078012in}}%
\pgfpathlineto{\pgfqpoint{2.883241in}{1.070654in}}%
\pgfpathlineto{\pgfqpoint{2.888967in}{1.068052in}}%
\pgfpathlineto{\pgfqpoint{2.910794in}{1.058092in}}%
\pgfpathlineto{\pgfqpoint{2.913120in}{1.057027in}}%
\pgfpathlineto{\pgfqpoint{2.932475in}{1.048133in}}%
\pgfpathlineto{\pgfqpoint{2.942998in}{1.043275in}}%
\pgfpathlineto{\pgfqpoint{2.954021in}{1.038173in}}%
\pgfpathlineto{\pgfqpoint{2.972877in}{1.029403in}}%
\pgfpathlineto{\pgfqpoint{2.975429in}{1.028214in}}%
\pgfpathlineto{\pgfqpoint{2.996697in}{1.018254in}}%
\pgfpathlineto{\pgfqpoint{3.002756in}{1.015402in}}%
\pgfpathlineto{\pgfqpoint{3.017827in}{1.008295in}}%
\pgfpathlineto{\pgfqpoint{3.032634in}{1.001274in}}%
\pgfpathlineto{\pgfqpoint{3.038823in}{0.998335in}}%
\pgfpathlineto{\pgfqpoint{3.059683in}{0.988376in}}%
\pgfpathlineto{\pgfqpoint{3.062513in}{0.987017in}}%
\pgfpathlineto{\pgfqpoint{3.080406in}{0.978416in}}%
\pgfpathlineto{\pgfqpoint{3.092392in}{0.972621in}}%
\pgfpathlineto{\pgfqpoint{3.100996in}{0.968456in}}%
\pgfpathlineto{\pgfqpoint{3.121451in}{0.958497in}}%
\pgfpathlineto{\pgfqpoint{3.122270in}{0.958096in}}%
\pgfpathlineto{\pgfqpoint{3.141771in}{0.948537in}}%
\pgfpathlineto{\pgfqpoint{3.152149in}{0.943418in}}%
\pgfpathlineto{\pgfqpoint{3.161958in}{0.938578in}}%
\pgfpathlineto{\pgfqpoint{3.182011in}{0.928618in}}%
\pgfpathlineto{\pgfqpoint{3.182028in}{0.928610in}}%
\pgfpathlineto{\pgfqpoint{3.201930in}{0.918659in}}%
\pgfpathlineto{\pgfqpoint{3.211906in}{0.913636in}}%
\pgfpathlineto{\pgfqpoint{3.221715in}{0.908699in}}%
\pgfpathlineto{\pgfqpoint{3.241367in}{0.898740in}}%
\pgfpathlineto{\pgfqpoint{3.241785in}{0.898526in}}%
\pgfpathlineto{\pgfqpoint{3.260885in}{0.888780in}}%
\pgfpathlineto{\pgfqpoint{3.271664in}{0.883240in}}%
\pgfpathlineto{\pgfqpoint{3.280268in}{0.878820in}}%
\pgfpathlineto{\pgfqpoint{3.299517in}{0.868861in}}%
\pgfpathlineto{\pgfqpoint{3.301542in}{0.867805in}}%
\pgfpathlineto{\pgfqpoint{3.318631in}{0.858901in}}%
\pgfpathlineto{\pgfqpoint{3.331421in}{0.852187in}}%
\pgfpathlineto{\pgfqpoint{3.337610in}{0.848942in}}%
\pgfpathlineto{\pgfqpoint{3.356453in}{0.838982in}}%
\pgfpathlineto{\pgfqpoint{3.361300in}{0.836399in}}%
\pgfpathlineto{\pgfqpoint{3.375159in}{0.829023in}}%
\pgfpathlineto{\pgfqpoint{3.391178in}{0.820430in}}%
\pgfpathlineto{\pgfqpoint{3.393731in}{0.819063in}}%
\pgfpathlineto{\pgfqpoint{3.412161in}{0.809103in}}%
\pgfpathlineto{\pgfqpoint{3.421057in}{0.804255in}}%
\pgfpathlineto{\pgfqpoint{3.430454in}{0.799144in}}%
\pgfpathlineto{\pgfqpoint{3.448610in}{0.789184in}}%
\pgfpathlineto{\pgfqpoint{3.450936in}{0.787896in}}%
\pgfpathlineto{\pgfqpoint{3.466619in}{0.779225in}}%
\pgfpathlineto{\pgfqpoint{3.480814in}{0.771312in}}%
\pgfpathlineto{\pgfqpoint{3.484494in}{0.769265in}}%
\pgfpathlineto{\pgfqpoint{3.502220in}{0.759306in}}%
\pgfpathlineto{\pgfqpoint{3.510693in}{0.754501in}}%
\pgfpathlineto{\pgfqpoint{3.519804in}{0.749346in}}%
\pgfpathlineto{\pgfqpoint{3.537246in}{0.739387in}}%
\pgfpathlineto{\pgfqpoint{3.540572in}{0.737466in}}%
\pgfpathlineto{\pgfqpoint{3.554532in}{0.729427in}}%
\pgfpathlineto{\pgfqpoint{3.570450in}{0.720182in}}%
\pgfpathlineto{\pgfqpoint{3.571684in}{0.719467in}}%
\pgfpathlineto{\pgfqpoint{3.588666in}{0.709508in}}%
\pgfpathlineto{\pgfqpoint{3.600329in}{0.702606in}}%
\pgfpathlineto{\pgfqpoint{3.605511in}{0.699548in}}%
\pgfpathlineto{\pgfqpoint{3.622193in}{0.689589in}}%
\pgfpathlineto{\pgfqpoint{3.630208in}{0.684752in}}%
\pgfpathlineto{\pgfqpoint{3.638722in}{0.679629in}}%
\pgfpathlineto{\pgfqpoint{3.655096in}{0.669670in}}%
\pgfpathlineto{\pgfqpoint{3.660086in}{0.666596in}}%
\pgfpathlineto{\pgfqpoint{3.671298in}{0.659710in}}%
\pgfpathlineto{\pgfqpoint{3.687356in}{0.649751in}}%
\pgfpathlineto{\pgfqpoint{3.689965in}{0.648108in}}%
\pgfpathlineto{\pgfqpoint{3.703219in}{0.639791in}}%
\pgfpathlineto{\pgfqpoint{3.718950in}{0.629831in}}%
\pgfpathlineto{\pgfqpoint{3.719844in}{0.629256in}}%
\pgfpathlineto{\pgfqpoint{3.734463in}{0.619872in}}%
\pgfpathlineto{\pgfqpoint{3.749723in}{0.609995in}}%
\pgfpathlineto{\pgfqpoint{3.749851in}{0.609912in}}%
\pgfpathlineto{\pgfqpoint{3.765002in}{0.599953in}}%
\pgfpathlineto{\pgfqpoint{3.779601in}{0.590276in}}%
\pgfpathlineto{\pgfqpoint{3.780030in}{0.589993in}}%
\pgfpathlineto{\pgfqpoint{3.794806in}{0.580034in}}%
\pgfpathlineto{\pgfqpoint{3.809461in}{0.570074in}}%
\pgfpathlineto{\pgfqpoint{3.809480in}{0.570061in}}%
\pgfpathlineto{\pgfqpoint{3.823841in}{0.560115in}}%
\pgfpathlineto{\pgfqpoint{3.838095in}{0.550155in}}%
\pgfpathlineto{\pgfqpoint{3.839359in}{0.549252in}}%
\pgfpathlineto{\pgfqpoint{3.852070in}{0.540195in}}%
\pgfpathlineto{\pgfqpoint{3.865895in}{0.530236in}}%
\pgfpathlineto{\pgfqpoint{3.869237in}{0.527775in}}%
\pgfpathlineto{\pgfqpoint{3.879448in}{0.520276in}}%
\pgfpathlineto{\pgfqpoint{3.892812in}{0.510317in}}%
\pgfpathlineto{\pgfqpoint{3.899116in}{0.505524in}}%
\pgfpathlineto{\pgfqpoint{3.905929in}{0.500357in}}%
\pgfpathlineto{\pgfqpoint{3.918792in}{0.490398in}}%
\pgfpathlineto{\pgfqpoint{3.928995in}{0.482373in}}%
\pgfpathlineto{\pgfqpoint{3.931460in}{0.480438in}}%
\pgfpathlineto{\pgfqpoint{3.943779in}{0.470478in}}%
\pgfpathlineto{\pgfqpoint{3.955923in}{0.460519in}}%
\pgfpathlineto{\pgfqpoint{3.958873in}{0.458022in}}%
\pgfpathlineto{\pgfqpoint{3.967706in}{0.450559in}}%
\pgfpathlineto{\pgfqpoint{3.979225in}{0.440600in}}%
\pgfpathlineto{\pgfqpoint{3.988752in}{0.432184in}}%
\pgfpathlineto{\pgfqpoint{3.990502in}{0.430640in}}%
\pgfpathlineto{\pgfqpoint{4.001332in}{0.420681in}}%
\pgfpathlineto{\pgfqpoint{4.011918in}{0.410721in}}%
\pgfpathlineto{\pgfqpoint{4.018631in}{0.404191in}}%
\pgfpathlineto{\pgfqpoint{4.022156in}{0.400762in}}%
\pgfpathlineto{\pgfqpoint{4.031926in}{0.390802in}}%
\pgfpathlineto{\pgfqpoint{4.041388in}{0.380842in}}%
\pgfpathlineto{\pgfqpoint{4.048509in}{0.373028in}}%
\pgfpathlineto{\pgfqpoint{4.050462in}{0.370883in}}%
\pgfpathlineto{\pgfqpoint{4.058939in}{0.360923in}}%
\pgfpathlineto{\pgfqpoint{4.067009in}{0.350964in}}%
\pgfpathlineto{\pgfqpoint{4.074638in}{0.341004in}}%
\pgfpathlineto{\pgfqpoint{4.078388in}{0.335681in}}%
\pgfpathlineto{\pgfqpoint{4.081644in}{0.331045in}}%
\pgfpathlineto{\pgfqpoint{4.087943in}{0.321085in}}%
\pgfpathlineto{\pgfqpoint{4.093620in}{0.311126in}}%
\pgfpathlineto{\pgfqpoint{4.098596in}{0.301166in}}%
\pgfpathlineto{\pgfqpoint{4.102777in}{0.291206in}}%
\pgfpathlineto{\pgfqpoint{4.106039in}{0.281247in}}%
\pgfpathlineto{\pgfqpoint{4.108230in}{0.271287in}}%
\pgfpathlineto{\pgfqpoint{4.108267in}{0.270872in}}%
\pgfpathlineto{\pgfqpoint{4.109092in}{0.261328in}}%
\pgfpathlineto{\pgfqpoint{4.108533in}{0.251368in}}%
\pgfpathlineto{\pgfqpoint{4.108267in}{0.250123in}}%
\pgfpathlineto{\pgfqpoint{4.106103in}{0.241409in}}%
\pgfpathlineto{\pgfqpoint{4.101361in}{0.231449in}}%
\pgfpathlineto{\pgfqpoint{4.093739in}{0.221489in}}%
\pgfpathlineto{\pgfqpoint{4.082426in}{0.211530in}}%
\pgfpathlineto{\pgfqpoint{4.078388in}{0.208846in}}%
\pgfpathlineto{\pgfqpoint{4.064288in}{0.201570in}}%
\pgfpathlineto{\pgfqpoint{4.048509in}{0.195478in}}%
\pgfpathlineto{\pgfqpoint{4.034320in}{0.191611in}}%
\pgfpathlineto{\pgfqpoint{4.018631in}{0.188182in}}%
\pgfpathlineto{\pgfqpoint{3.988752in}{0.184110in}}%
\pgfpathlineto{\pgfqpoint{3.958873in}{0.182113in}}%
\pgfpathlineto{\pgfqpoint{3.928995in}{0.181668in}}%
\pgfpathlineto{\pgfqpoint{3.899116in}{0.182418in}}%
\pgfpathlineto{\pgfqpoint{3.869237in}{0.184112in}}%
\pgfpathlineto{\pgfqpoint{3.839359in}{0.186570in}}%
\pgfpathlineto{\pgfqpoint{3.809480in}{0.189659in}}%
\pgfpathclose%
\pgfusepath{fill}%
\end{pgfscope}%
\begin{pgfscope}%
\pgfpathrectangle{\pgfqpoint{0.135000in}{0.151972in}}{\pgfqpoint{4.630000in}{1.543333in}} %
\pgfusepath{clip}%
\pgfsetbuttcap%
\pgfsetroundjoin%
\pgfsetlinewidth{0.000000pt}%
\definecolor{currentstroke}{rgb}{0.000000,0.000000,0.000000}%
\pgfsetstrokecolor{currentstroke}%
\pgfsetdash{}{0pt}%
\pgfpathmoveto{\pgfqpoint{0.253917in}{0.181651in}}%
\pgfpathlineto{\pgfqpoint{0.283796in}{0.181651in}}%
\pgfpathlineto{\pgfqpoint{0.313674in}{0.181651in}}%
\pgfpathlineto{\pgfqpoint{0.343553in}{0.181651in}}%
\pgfpathlineto{\pgfqpoint{0.373432in}{0.181651in}}%
\pgfpathlineto{\pgfqpoint{0.403311in}{0.181651in}}%
\pgfpathlineto{\pgfqpoint{0.433189in}{0.181651in}}%
\pgfpathlineto{\pgfqpoint{0.463068in}{0.181651in}}%
\pgfpathlineto{\pgfqpoint{0.492947in}{0.181651in}}%
\pgfpathlineto{\pgfqpoint{0.522825in}{0.181651in}}%
\pgfpathlineto{\pgfqpoint{0.552704in}{0.181651in}}%
\pgfpathlineto{\pgfqpoint{0.582583in}{0.181651in}}%
\pgfpathlineto{\pgfqpoint{0.612461in}{0.181651in}}%
\pgfpathlineto{\pgfqpoint{0.642340in}{0.181651in}}%
\pgfpathlineto{\pgfqpoint{0.672219in}{0.181651in}}%
\pgfpathlineto{\pgfqpoint{0.702097in}{0.181651in}}%
\pgfpathlineto{\pgfqpoint{0.731976in}{0.181651in}}%
\pgfpathlineto{\pgfqpoint{0.761855in}{0.181651in}}%
\pgfpathlineto{\pgfqpoint{0.791733in}{0.181651in}}%
\pgfpathlineto{\pgfqpoint{0.821612in}{0.181651in}}%
\pgfpathlineto{\pgfqpoint{0.851491in}{0.181651in}}%
\pgfpathlineto{\pgfqpoint{0.881369in}{0.181651in}}%
\pgfpathlineto{\pgfqpoint{0.911248in}{0.181651in}}%
\pgfpathlineto{\pgfqpoint{0.941127in}{0.181651in}}%
\pgfpathlineto{\pgfqpoint{0.971005in}{0.181651in}}%
\pgfpathlineto{\pgfqpoint{1.000884in}{0.181651in}}%
\pgfpathlineto{\pgfqpoint{1.030763in}{0.181651in}}%
\pgfpathlineto{\pgfqpoint{1.060641in}{0.181651in}}%
\pgfpathlineto{\pgfqpoint{1.090520in}{0.181651in}}%
\pgfpathlineto{\pgfqpoint{1.120399in}{0.181651in}}%
\pgfpathlineto{\pgfqpoint{1.150277in}{0.181651in}}%
\pgfpathlineto{\pgfqpoint{1.180156in}{0.181651in}}%
\pgfpathlineto{\pgfqpoint{1.210035in}{0.181651in}}%
\pgfpathlineto{\pgfqpoint{1.239914in}{0.181651in}}%
\pgfpathlineto{\pgfqpoint{1.269792in}{0.181651in}}%
\pgfpathlineto{\pgfqpoint{1.299671in}{0.181651in}}%
\pgfpathlineto{\pgfqpoint{1.329550in}{0.181651in}}%
\pgfpathlineto{\pgfqpoint{1.359428in}{0.181651in}}%
\pgfpathlineto{\pgfqpoint{1.389307in}{0.181651in}}%
\pgfpathlineto{\pgfqpoint{1.419186in}{0.181651in}}%
\pgfpathlineto{\pgfqpoint{1.449064in}{0.181651in}}%
\pgfpathlineto{\pgfqpoint{1.478943in}{0.181651in}}%
\pgfpathlineto{\pgfqpoint{1.508822in}{0.181651in}}%
\pgfpathlineto{\pgfqpoint{1.538700in}{0.181651in}}%
\pgfpathlineto{\pgfqpoint{1.568579in}{0.181651in}}%
\pgfpathlineto{\pgfqpoint{1.598458in}{0.181651in}}%
\pgfpathlineto{\pgfqpoint{1.628336in}{0.181651in}}%
\pgfpathlineto{\pgfqpoint{1.658215in}{0.181651in}}%
\pgfpathlineto{\pgfqpoint{1.688094in}{0.181651in}}%
\pgfpathlineto{\pgfqpoint{1.717972in}{0.181651in}}%
\pgfpathlineto{\pgfqpoint{1.747851in}{0.181651in}}%
\pgfpathlineto{\pgfqpoint{1.777730in}{0.181651in}}%
\pgfpathlineto{\pgfqpoint{1.807608in}{0.181651in}}%
\pgfpathlineto{\pgfqpoint{1.837487in}{0.181651in}}%
\pgfpathlineto{\pgfqpoint{1.867366in}{0.181651in}}%
\pgfpathlineto{\pgfqpoint{1.897244in}{0.181651in}}%
\pgfpathlineto{\pgfqpoint{1.927123in}{0.181651in}}%
\pgfpathlineto{\pgfqpoint{1.957002in}{0.181651in}}%
\pgfpathlineto{\pgfqpoint{1.986880in}{0.181651in}}%
\pgfpathlineto{\pgfqpoint{2.016759in}{0.181651in}}%
\pgfpathlineto{\pgfqpoint{2.046638in}{0.181651in}}%
\pgfpathlineto{\pgfqpoint{2.076517in}{0.181651in}}%
\pgfpathlineto{\pgfqpoint{2.106395in}{0.181651in}}%
\pgfpathlineto{\pgfqpoint{2.136274in}{0.181651in}}%
\pgfpathlineto{\pgfqpoint{2.166153in}{0.181651in}}%
\pgfpathlineto{\pgfqpoint{2.196031in}{0.181651in}}%
\pgfpathlineto{\pgfqpoint{2.225910in}{0.181651in}}%
\pgfpathlineto{\pgfqpoint{2.255789in}{0.181651in}}%
\pgfpathlineto{\pgfqpoint{2.285667in}{0.181651in}}%
\pgfpathlineto{\pgfqpoint{2.315546in}{0.181651in}}%
\pgfpathlineto{\pgfqpoint{2.345425in}{0.181651in}}%
\pgfpathlineto{\pgfqpoint{2.375303in}{0.181651in}}%
\pgfpathlineto{\pgfqpoint{2.405182in}{0.181651in}}%
\pgfpathlineto{\pgfqpoint{2.435061in}{0.181651in}}%
\pgfpathlineto{\pgfqpoint{2.464939in}{0.181651in}}%
\pgfpathlineto{\pgfqpoint{2.494818in}{0.181651in}}%
\pgfpathlineto{\pgfqpoint{2.524697in}{0.181651in}}%
\pgfpathlineto{\pgfqpoint{2.554575in}{0.181651in}}%
\pgfpathlineto{\pgfqpoint{2.584454in}{0.181651in}}%
\pgfpathlineto{\pgfqpoint{2.614333in}{0.181651in}}%
\pgfpathlineto{\pgfqpoint{2.644211in}{0.181651in}}%
\pgfpathlineto{\pgfqpoint{2.674090in}{0.181651in}}%
\pgfpathlineto{\pgfqpoint{2.703969in}{0.181651in}}%
\pgfpathlineto{\pgfqpoint{2.733847in}{0.181651in}}%
\pgfpathlineto{\pgfqpoint{2.763726in}{0.181651in}}%
\pgfpathlineto{\pgfqpoint{2.793605in}{0.181651in}}%
\pgfpathlineto{\pgfqpoint{2.823483in}{0.181651in}}%
\pgfpathlineto{\pgfqpoint{2.853362in}{0.181651in}}%
\pgfpathlineto{\pgfqpoint{2.883241in}{0.181651in}}%
\pgfpathlineto{\pgfqpoint{2.913120in}{0.181651in}}%
\pgfpathlineto{\pgfqpoint{2.942998in}{0.181651in}}%
\pgfpathlineto{\pgfqpoint{2.972877in}{0.181651in}}%
\pgfpathlineto{\pgfqpoint{3.002756in}{0.181651in}}%
\pgfpathlineto{\pgfqpoint{3.032634in}{0.181651in}}%
\pgfpathlineto{\pgfqpoint{3.062513in}{0.181651in}}%
\pgfpathlineto{\pgfqpoint{3.092392in}{0.181651in}}%
\pgfpathlineto{\pgfqpoint{3.122270in}{0.181651in}}%
\pgfpathlineto{\pgfqpoint{3.152149in}{0.181651in}}%
\pgfpathlineto{\pgfqpoint{3.182028in}{0.181651in}}%
\pgfpathlineto{\pgfqpoint{3.211906in}{0.181651in}}%
\pgfpathlineto{\pgfqpoint{3.241785in}{0.181651in}}%
\pgfpathlineto{\pgfqpoint{3.271664in}{0.181651in}}%
\pgfpathlineto{\pgfqpoint{3.301542in}{0.181651in}}%
\pgfpathlineto{\pgfqpoint{3.331421in}{0.181651in}}%
\pgfpathlineto{\pgfqpoint{3.361300in}{0.181651in}}%
\pgfpathlineto{\pgfqpoint{3.391178in}{0.181651in}}%
\pgfpathlineto{\pgfqpoint{3.421057in}{0.181651in}}%
\pgfpathlineto{\pgfqpoint{3.450936in}{0.181651in}}%
\pgfpathlineto{\pgfqpoint{3.480814in}{0.181651in}}%
\pgfpathlineto{\pgfqpoint{3.510693in}{0.181651in}}%
\pgfpathlineto{\pgfqpoint{3.540572in}{0.181651in}}%
\pgfpathlineto{\pgfqpoint{3.570450in}{0.181651in}}%
\pgfpathlineto{\pgfqpoint{3.600329in}{0.181651in}}%
\pgfpathlineto{\pgfqpoint{3.630208in}{0.181651in}}%
\pgfpathlineto{\pgfqpoint{3.660086in}{0.181651in}}%
\pgfpathlineto{\pgfqpoint{3.689965in}{0.181651in}}%
\pgfpathlineto{\pgfqpoint{3.719844in}{0.181651in}}%
\pgfpathlineto{\pgfqpoint{3.749723in}{0.181651in}}%
\pgfpathlineto{\pgfqpoint{3.779601in}{0.181651in}}%
\pgfpathlineto{\pgfqpoint{3.809480in}{0.181651in}}%
\pgfpathlineto{\pgfqpoint{3.839359in}{0.181651in}}%
\pgfpathlineto{\pgfqpoint{3.869237in}{0.181651in}}%
\pgfpathlineto{\pgfqpoint{3.899116in}{0.181651in}}%
\pgfpathlineto{\pgfqpoint{3.928995in}{0.181651in}}%
\pgfpathlineto{\pgfqpoint{3.958873in}{0.181651in}}%
\pgfpathlineto{\pgfqpoint{3.988752in}{0.181651in}}%
\pgfpathlineto{\pgfqpoint{4.018631in}{0.181651in}}%
\pgfpathlineto{\pgfqpoint{4.048509in}{0.181651in}}%
\pgfpathlineto{\pgfqpoint{4.078388in}{0.181651in}}%
\pgfpathlineto{\pgfqpoint{4.108267in}{0.181651in}}%
\pgfpathlineto{\pgfqpoint{4.138145in}{0.181651in}}%
\pgfpathlineto{\pgfqpoint{4.168024in}{0.181651in}}%
\pgfpathlineto{\pgfqpoint{4.197903in}{0.181651in}}%
\pgfpathlineto{\pgfqpoint{4.227781in}{0.181651in}}%
\pgfpathlineto{\pgfqpoint{4.257660in}{0.181651in}}%
\pgfpathlineto{\pgfqpoint{4.287539in}{0.181651in}}%
\pgfpathlineto{\pgfqpoint{4.317417in}{0.181651in}}%
\pgfpathlineto{\pgfqpoint{4.347296in}{0.181651in}}%
\pgfpathlineto{\pgfqpoint{4.377175in}{0.181651in}}%
\pgfpathlineto{\pgfqpoint{4.407053in}{0.181651in}}%
\pgfpathlineto{\pgfqpoint{4.436932in}{0.181651in}}%
\pgfpathlineto{\pgfqpoint{4.466811in}{0.181651in}}%
\pgfpathlineto{\pgfqpoint{4.496689in}{0.181651in}}%
\pgfpathlineto{\pgfqpoint{4.526568in}{0.181651in}}%
\pgfpathlineto{\pgfqpoint{4.556447in}{0.181651in}}%
\pgfpathlineto{\pgfqpoint{4.586326in}{0.181651in}}%
\pgfpathlineto{\pgfqpoint{4.616204in}{0.181651in}}%
\pgfpathlineto{\pgfqpoint{4.646083in}{0.181651in}}%
\pgfpathlineto{\pgfqpoint{4.675962in}{0.181651in}}%
\pgfpathlineto{\pgfqpoint{4.675962in}{0.191611in}}%
\pgfpathlineto{\pgfqpoint{4.675962in}{0.201570in}}%
\pgfpathlineto{\pgfqpoint{4.675962in}{0.211530in}}%
\pgfpathlineto{\pgfqpoint{4.675962in}{0.221489in}}%
\pgfpathlineto{\pgfqpoint{4.675962in}{0.231449in}}%
\pgfpathlineto{\pgfqpoint{4.675962in}{0.241409in}}%
\pgfpathlineto{\pgfqpoint{4.675962in}{0.251368in}}%
\pgfpathlineto{\pgfqpoint{4.675962in}{0.261328in}}%
\pgfpathlineto{\pgfqpoint{4.675962in}{0.271287in}}%
\pgfpathlineto{\pgfqpoint{4.675962in}{0.281247in}}%
\pgfpathlineto{\pgfqpoint{4.675962in}{0.291206in}}%
\pgfpathlineto{\pgfqpoint{4.675962in}{0.301166in}}%
\pgfpathlineto{\pgfqpoint{4.675962in}{0.311126in}}%
\pgfpathlineto{\pgfqpoint{4.675962in}{0.321085in}}%
\pgfpathlineto{\pgfqpoint{4.675962in}{0.331045in}}%
\pgfpathlineto{\pgfqpoint{4.675962in}{0.341004in}}%
\pgfpathlineto{\pgfqpoint{4.675962in}{0.350964in}}%
\pgfpathlineto{\pgfqpoint{4.675962in}{0.360923in}}%
\pgfpathlineto{\pgfqpoint{4.675962in}{0.370883in}}%
\pgfpathlineto{\pgfqpoint{4.675962in}{0.380842in}}%
\pgfpathlineto{\pgfqpoint{4.675962in}{0.390802in}}%
\pgfpathlineto{\pgfqpoint{4.675962in}{0.400762in}}%
\pgfpathlineto{\pgfqpoint{4.675962in}{0.410721in}}%
\pgfpathlineto{\pgfqpoint{4.675962in}{0.420681in}}%
\pgfpathlineto{\pgfqpoint{4.675962in}{0.430640in}}%
\pgfpathlineto{\pgfqpoint{4.675962in}{0.440600in}}%
\pgfpathlineto{\pgfqpoint{4.675962in}{0.450559in}}%
\pgfpathlineto{\pgfqpoint{4.675962in}{0.460519in}}%
\pgfpathlineto{\pgfqpoint{4.675962in}{0.470478in}}%
\pgfpathlineto{\pgfqpoint{4.675962in}{0.480438in}}%
\pgfpathlineto{\pgfqpoint{4.675962in}{0.490398in}}%
\pgfpathlineto{\pgfqpoint{4.675962in}{0.500357in}}%
\pgfpathlineto{\pgfqpoint{4.675962in}{0.510317in}}%
\pgfpathlineto{\pgfqpoint{4.675962in}{0.520276in}}%
\pgfpathlineto{\pgfqpoint{4.675962in}{0.530236in}}%
\pgfpathlineto{\pgfqpoint{4.675962in}{0.540195in}}%
\pgfpathlineto{\pgfqpoint{4.675962in}{0.550155in}}%
\pgfpathlineto{\pgfqpoint{4.675962in}{0.560115in}}%
\pgfpathlineto{\pgfqpoint{4.675962in}{0.570074in}}%
\pgfpathlineto{\pgfqpoint{4.675962in}{0.580034in}}%
\pgfpathlineto{\pgfqpoint{4.675962in}{0.589993in}}%
\pgfpathlineto{\pgfqpoint{4.675962in}{0.599953in}}%
\pgfpathlineto{\pgfqpoint{4.675962in}{0.609912in}}%
\pgfpathlineto{\pgfqpoint{4.675962in}{0.619872in}}%
\pgfpathlineto{\pgfqpoint{4.675962in}{0.629831in}}%
\pgfpathlineto{\pgfqpoint{4.675962in}{0.639791in}}%
\pgfpathlineto{\pgfqpoint{4.675962in}{0.649751in}}%
\pgfpathlineto{\pgfqpoint{4.675962in}{0.659710in}}%
\pgfpathlineto{\pgfqpoint{4.675962in}{0.669670in}}%
\pgfpathlineto{\pgfqpoint{4.675962in}{0.679629in}}%
\pgfpathlineto{\pgfqpoint{4.675962in}{0.689589in}}%
\pgfpathlineto{\pgfqpoint{4.675962in}{0.699548in}}%
\pgfpathlineto{\pgfqpoint{4.675962in}{0.709508in}}%
\pgfpathlineto{\pgfqpoint{4.675962in}{0.719467in}}%
\pgfpathlineto{\pgfqpoint{4.675962in}{0.729427in}}%
\pgfpathlineto{\pgfqpoint{4.675962in}{0.739387in}}%
\pgfpathlineto{\pgfqpoint{4.675962in}{0.749346in}}%
\pgfpathlineto{\pgfqpoint{4.675962in}{0.759306in}}%
\pgfpathlineto{\pgfqpoint{4.675962in}{0.769265in}}%
\pgfpathlineto{\pgfqpoint{4.675962in}{0.779225in}}%
\pgfpathlineto{\pgfqpoint{4.675962in}{0.789184in}}%
\pgfpathlineto{\pgfqpoint{4.675962in}{0.799144in}}%
\pgfpathlineto{\pgfqpoint{4.675962in}{0.809103in}}%
\pgfpathlineto{\pgfqpoint{4.675962in}{0.819063in}}%
\pgfpathlineto{\pgfqpoint{4.675962in}{0.829023in}}%
\pgfpathlineto{\pgfqpoint{4.675962in}{0.838982in}}%
\pgfpathlineto{\pgfqpoint{4.675962in}{0.848942in}}%
\pgfpathlineto{\pgfqpoint{4.675962in}{0.858901in}}%
\pgfpathlineto{\pgfqpoint{4.675962in}{0.868861in}}%
\pgfpathlineto{\pgfqpoint{4.675962in}{0.878820in}}%
\pgfpathlineto{\pgfqpoint{4.675962in}{0.888780in}}%
\pgfpathlineto{\pgfqpoint{4.675962in}{0.898740in}}%
\pgfpathlineto{\pgfqpoint{4.675962in}{0.908699in}}%
\pgfpathlineto{\pgfqpoint{4.675962in}{0.918659in}}%
\pgfpathlineto{\pgfqpoint{4.675962in}{0.928618in}}%
\pgfpathlineto{\pgfqpoint{4.675962in}{0.938578in}}%
\pgfpathlineto{\pgfqpoint{4.675962in}{0.948537in}}%
\pgfpathlineto{\pgfqpoint{4.675962in}{0.958497in}}%
\pgfpathlineto{\pgfqpoint{4.675962in}{0.968456in}}%
\pgfpathlineto{\pgfqpoint{4.675962in}{0.978416in}}%
\pgfpathlineto{\pgfqpoint{4.675962in}{0.988376in}}%
\pgfpathlineto{\pgfqpoint{4.675962in}{0.998335in}}%
\pgfpathlineto{\pgfqpoint{4.675962in}{1.008295in}}%
\pgfpathlineto{\pgfqpoint{4.675962in}{1.018254in}}%
\pgfpathlineto{\pgfqpoint{4.675962in}{1.028214in}}%
\pgfpathlineto{\pgfqpoint{4.675962in}{1.038173in}}%
\pgfpathlineto{\pgfqpoint{4.675962in}{1.048133in}}%
\pgfpathlineto{\pgfqpoint{4.675962in}{1.058092in}}%
\pgfpathlineto{\pgfqpoint{4.675962in}{1.068052in}}%
\pgfpathlineto{\pgfqpoint{4.675962in}{1.078012in}}%
\pgfpathlineto{\pgfqpoint{4.675962in}{1.087971in}}%
\pgfpathlineto{\pgfqpoint{4.675962in}{1.097931in}}%
\pgfpathlineto{\pgfqpoint{4.675962in}{1.107890in}}%
\pgfpathlineto{\pgfqpoint{4.675962in}{1.117850in}}%
\pgfpathlineto{\pgfqpoint{4.675962in}{1.127809in}}%
\pgfpathlineto{\pgfqpoint{4.675962in}{1.137769in}}%
\pgfpathlineto{\pgfqpoint{4.675962in}{1.147729in}}%
\pgfpathlineto{\pgfqpoint{4.675962in}{1.157688in}}%
\pgfpathlineto{\pgfqpoint{4.675962in}{1.167648in}}%
\pgfpathlineto{\pgfqpoint{4.675962in}{1.177607in}}%
\pgfpathlineto{\pgfqpoint{4.675962in}{1.187567in}}%
\pgfpathlineto{\pgfqpoint{4.675962in}{1.197526in}}%
\pgfpathlineto{\pgfqpoint{4.675962in}{1.207486in}}%
\pgfpathlineto{\pgfqpoint{4.675962in}{1.217445in}}%
\pgfpathlineto{\pgfqpoint{4.675962in}{1.227405in}}%
\pgfpathlineto{\pgfqpoint{4.675962in}{1.237365in}}%
\pgfpathlineto{\pgfqpoint{4.675962in}{1.247324in}}%
\pgfpathlineto{\pgfqpoint{4.675962in}{1.257284in}}%
\pgfpathlineto{\pgfqpoint{4.675962in}{1.267243in}}%
\pgfpathlineto{\pgfqpoint{4.675962in}{1.277203in}}%
\pgfpathlineto{\pgfqpoint{4.675962in}{1.287162in}}%
\pgfpathlineto{\pgfqpoint{4.675962in}{1.297122in}}%
\pgfpathlineto{\pgfqpoint{4.675962in}{1.307081in}}%
\pgfpathlineto{\pgfqpoint{4.675962in}{1.317041in}}%
\pgfpathlineto{\pgfqpoint{4.675962in}{1.327001in}}%
\pgfpathlineto{\pgfqpoint{4.675962in}{1.336960in}}%
\pgfpathlineto{\pgfqpoint{4.675962in}{1.346920in}}%
\pgfpathlineto{\pgfqpoint{4.675962in}{1.356879in}}%
\pgfpathlineto{\pgfqpoint{4.675962in}{1.366839in}}%
\pgfpathlineto{\pgfqpoint{4.675962in}{1.376798in}}%
\pgfpathlineto{\pgfqpoint{4.675962in}{1.386758in}}%
\pgfpathlineto{\pgfqpoint{4.675962in}{1.396718in}}%
\pgfpathlineto{\pgfqpoint{4.675962in}{1.406677in}}%
\pgfpathlineto{\pgfqpoint{4.675962in}{1.416637in}}%
\pgfpathlineto{\pgfqpoint{4.675962in}{1.426596in}}%
\pgfpathlineto{\pgfqpoint{4.675962in}{1.436556in}}%
\pgfpathlineto{\pgfqpoint{4.675962in}{1.446515in}}%
\pgfpathlineto{\pgfqpoint{4.675962in}{1.456475in}}%
\pgfpathlineto{\pgfqpoint{4.675962in}{1.466434in}}%
\pgfpathlineto{\pgfqpoint{4.675962in}{1.476394in}}%
\pgfpathlineto{\pgfqpoint{4.675962in}{1.486354in}}%
\pgfpathlineto{\pgfqpoint{4.675962in}{1.496313in}}%
\pgfpathlineto{\pgfqpoint{4.675962in}{1.506273in}}%
\pgfpathlineto{\pgfqpoint{4.675962in}{1.516232in}}%
\pgfpathlineto{\pgfqpoint{4.675962in}{1.526192in}}%
\pgfpathlineto{\pgfqpoint{4.675962in}{1.536151in}}%
\pgfpathlineto{\pgfqpoint{4.675962in}{1.546111in}}%
\pgfpathlineto{\pgfqpoint{4.675962in}{1.556070in}}%
\pgfpathlineto{\pgfqpoint{4.675962in}{1.566030in}}%
\pgfpathlineto{\pgfqpoint{4.675962in}{1.575990in}}%
\pgfpathlineto{\pgfqpoint{4.675962in}{1.585949in}}%
\pgfpathlineto{\pgfqpoint{4.675962in}{1.595909in}}%
\pgfpathlineto{\pgfqpoint{4.675962in}{1.605868in}}%
\pgfpathlineto{\pgfqpoint{4.675962in}{1.615828in}}%
\pgfpathlineto{\pgfqpoint{4.675962in}{1.625787in}}%
\pgfpathlineto{\pgfqpoint{4.675962in}{1.635747in}}%
\pgfpathlineto{\pgfqpoint{4.675962in}{1.645706in}}%
\pgfpathlineto{\pgfqpoint{4.675962in}{1.655666in}}%
\pgfpathlineto{\pgfqpoint{4.675962in}{1.665626in}}%
\pgfpathlineto{\pgfqpoint{4.646083in}{1.665626in}}%
\pgfpathlineto{\pgfqpoint{4.616204in}{1.665626in}}%
\pgfpathlineto{\pgfqpoint{4.586326in}{1.665626in}}%
\pgfpathlineto{\pgfqpoint{4.556447in}{1.665626in}}%
\pgfpathlineto{\pgfqpoint{4.526568in}{1.665626in}}%
\pgfpathlineto{\pgfqpoint{4.496689in}{1.665626in}}%
\pgfpathlineto{\pgfqpoint{4.466811in}{1.665626in}}%
\pgfpathlineto{\pgfqpoint{4.436932in}{1.665626in}}%
\pgfpathlineto{\pgfqpoint{4.407053in}{1.665626in}}%
\pgfpathlineto{\pgfqpoint{4.377175in}{1.665626in}}%
\pgfpathlineto{\pgfqpoint{4.347296in}{1.665626in}}%
\pgfpathlineto{\pgfqpoint{4.317417in}{1.665626in}}%
\pgfpathlineto{\pgfqpoint{4.287539in}{1.665626in}}%
\pgfpathlineto{\pgfqpoint{4.257660in}{1.665626in}}%
\pgfpathlineto{\pgfqpoint{4.227781in}{1.665626in}}%
\pgfpathlineto{\pgfqpoint{4.197903in}{1.665626in}}%
\pgfpathlineto{\pgfqpoint{4.168024in}{1.665626in}}%
\pgfpathlineto{\pgfqpoint{4.138145in}{1.665626in}}%
\pgfpathlineto{\pgfqpoint{4.108267in}{1.665626in}}%
\pgfpathlineto{\pgfqpoint{4.078388in}{1.665626in}}%
\pgfpathlineto{\pgfqpoint{4.048509in}{1.665626in}}%
\pgfpathlineto{\pgfqpoint{4.018631in}{1.665626in}}%
\pgfpathlineto{\pgfqpoint{3.988752in}{1.665626in}}%
\pgfpathlineto{\pgfqpoint{3.958873in}{1.665626in}}%
\pgfpathlineto{\pgfqpoint{3.928995in}{1.665626in}}%
\pgfpathlineto{\pgfqpoint{3.899116in}{1.665626in}}%
\pgfpathlineto{\pgfqpoint{3.869237in}{1.665626in}}%
\pgfpathlineto{\pgfqpoint{3.839359in}{1.665626in}}%
\pgfpathlineto{\pgfqpoint{3.809480in}{1.665626in}}%
\pgfpathlineto{\pgfqpoint{3.779601in}{1.665626in}}%
\pgfpathlineto{\pgfqpoint{3.749723in}{1.665626in}}%
\pgfpathlineto{\pgfqpoint{3.719844in}{1.665626in}}%
\pgfpathlineto{\pgfqpoint{3.689965in}{1.665626in}}%
\pgfpathlineto{\pgfqpoint{3.660086in}{1.665626in}}%
\pgfpathlineto{\pgfqpoint{3.630208in}{1.665626in}}%
\pgfpathlineto{\pgfqpoint{3.600329in}{1.665626in}}%
\pgfpathlineto{\pgfqpoint{3.570450in}{1.665626in}}%
\pgfpathlineto{\pgfqpoint{3.540572in}{1.665626in}}%
\pgfpathlineto{\pgfqpoint{3.510693in}{1.665626in}}%
\pgfpathlineto{\pgfqpoint{3.480814in}{1.665626in}}%
\pgfpathlineto{\pgfqpoint{3.450936in}{1.665626in}}%
\pgfpathlineto{\pgfqpoint{3.421057in}{1.665626in}}%
\pgfpathlineto{\pgfqpoint{3.391178in}{1.665626in}}%
\pgfpathlineto{\pgfqpoint{3.361300in}{1.665626in}}%
\pgfpathlineto{\pgfqpoint{3.331421in}{1.665626in}}%
\pgfpathlineto{\pgfqpoint{3.301542in}{1.665626in}}%
\pgfpathlineto{\pgfqpoint{3.271664in}{1.665626in}}%
\pgfpathlineto{\pgfqpoint{3.241785in}{1.665626in}}%
\pgfpathlineto{\pgfqpoint{3.211906in}{1.665626in}}%
\pgfpathlineto{\pgfqpoint{3.182028in}{1.665626in}}%
\pgfpathlineto{\pgfqpoint{3.152149in}{1.665626in}}%
\pgfpathlineto{\pgfqpoint{3.122270in}{1.665626in}}%
\pgfpathlineto{\pgfqpoint{3.092392in}{1.665626in}}%
\pgfpathlineto{\pgfqpoint{3.062513in}{1.665626in}}%
\pgfpathlineto{\pgfqpoint{3.032634in}{1.665626in}}%
\pgfpathlineto{\pgfqpoint{3.002756in}{1.665626in}}%
\pgfpathlineto{\pgfqpoint{2.972877in}{1.665626in}}%
\pgfpathlineto{\pgfqpoint{2.942998in}{1.665626in}}%
\pgfpathlineto{\pgfqpoint{2.913120in}{1.665626in}}%
\pgfpathlineto{\pgfqpoint{2.883241in}{1.665626in}}%
\pgfpathlineto{\pgfqpoint{2.853362in}{1.665626in}}%
\pgfpathlineto{\pgfqpoint{2.823483in}{1.665626in}}%
\pgfpathlineto{\pgfqpoint{2.793605in}{1.665626in}}%
\pgfpathlineto{\pgfqpoint{2.763726in}{1.665626in}}%
\pgfpathlineto{\pgfqpoint{2.733847in}{1.665626in}}%
\pgfpathlineto{\pgfqpoint{2.703969in}{1.665626in}}%
\pgfpathlineto{\pgfqpoint{2.674090in}{1.665626in}}%
\pgfpathlineto{\pgfqpoint{2.644211in}{1.665626in}}%
\pgfpathlineto{\pgfqpoint{2.614333in}{1.665626in}}%
\pgfpathlineto{\pgfqpoint{2.584454in}{1.665626in}}%
\pgfpathlineto{\pgfqpoint{2.554575in}{1.665626in}}%
\pgfpathlineto{\pgfqpoint{2.524697in}{1.665626in}}%
\pgfpathlineto{\pgfqpoint{2.494818in}{1.665626in}}%
\pgfpathlineto{\pgfqpoint{2.464939in}{1.665626in}}%
\pgfpathlineto{\pgfqpoint{2.435061in}{1.665626in}}%
\pgfpathlineto{\pgfqpoint{2.405182in}{1.665626in}}%
\pgfpathlineto{\pgfqpoint{2.375303in}{1.665626in}}%
\pgfpathlineto{\pgfqpoint{2.345425in}{1.665626in}}%
\pgfpathlineto{\pgfqpoint{2.315546in}{1.665626in}}%
\pgfpathlineto{\pgfqpoint{2.285667in}{1.665626in}}%
\pgfpathlineto{\pgfqpoint{2.255789in}{1.665626in}}%
\pgfpathlineto{\pgfqpoint{2.225910in}{1.665626in}}%
\pgfpathlineto{\pgfqpoint{2.196031in}{1.665626in}}%
\pgfpathlineto{\pgfqpoint{2.166153in}{1.665626in}}%
\pgfpathlineto{\pgfqpoint{2.136274in}{1.665626in}}%
\pgfpathlineto{\pgfqpoint{2.106395in}{1.665626in}}%
\pgfpathlineto{\pgfqpoint{2.076517in}{1.665626in}}%
\pgfpathlineto{\pgfqpoint{2.046638in}{1.665626in}}%
\pgfpathlineto{\pgfqpoint{2.016759in}{1.665626in}}%
\pgfpathlineto{\pgfqpoint{1.986880in}{1.665626in}}%
\pgfpathlineto{\pgfqpoint{1.957002in}{1.665626in}}%
\pgfpathlineto{\pgfqpoint{1.927123in}{1.665626in}}%
\pgfpathlineto{\pgfqpoint{1.897244in}{1.665626in}}%
\pgfpathlineto{\pgfqpoint{1.867366in}{1.665626in}}%
\pgfpathlineto{\pgfqpoint{1.837487in}{1.665626in}}%
\pgfpathlineto{\pgfqpoint{1.807608in}{1.665626in}}%
\pgfpathlineto{\pgfqpoint{1.777730in}{1.665626in}}%
\pgfpathlineto{\pgfqpoint{1.747851in}{1.665626in}}%
\pgfpathlineto{\pgfqpoint{1.717972in}{1.665626in}}%
\pgfpathlineto{\pgfqpoint{1.688094in}{1.665626in}}%
\pgfpathlineto{\pgfqpoint{1.658215in}{1.665626in}}%
\pgfpathlineto{\pgfqpoint{1.628336in}{1.665626in}}%
\pgfpathlineto{\pgfqpoint{1.598458in}{1.665626in}}%
\pgfpathlineto{\pgfqpoint{1.568579in}{1.665626in}}%
\pgfpathlineto{\pgfqpoint{1.538700in}{1.665626in}}%
\pgfpathlineto{\pgfqpoint{1.508822in}{1.665626in}}%
\pgfpathlineto{\pgfqpoint{1.478943in}{1.665626in}}%
\pgfpathlineto{\pgfqpoint{1.449064in}{1.665626in}}%
\pgfpathlineto{\pgfqpoint{1.419186in}{1.665626in}}%
\pgfpathlineto{\pgfqpoint{1.389307in}{1.665626in}}%
\pgfpathlineto{\pgfqpoint{1.359428in}{1.665626in}}%
\pgfpathlineto{\pgfqpoint{1.329550in}{1.665626in}}%
\pgfpathlineto{\pgfqpoint{1.299671in}{1.665626in}}%
\pgfpathlineto{\pgfqpoint{1.269792in}{1.665626in}}%
\pgfpathlineto{\pgfqpoint{1.239914in}{1.665626in}}%
\pgfpathlineto{\pgfqpoint{1.210035in}{1.665626in}}%
\pgfpathlineto{\pgfqpoint{1.180156in}{1.665626in}}%
\pgfpathlineto{\pgfqpoint{1.150277in}{1.665626in}}%
\pgfpathlineto{\pgfqpoint{1.120399in}{1.665626in}}%
\pgfpathlineto{\pgfqpoint{1.090520in}{1.665626in}}%
\pgfpathlineto{\pgfqpoint{1.060641in}{1.665626in}}%
\pgfpathlineto{\pgfqpoint{1.030763in}{1.665626in}}%
\pgfpathlineto{\pgfqpoint{1.000884in}{1.665626in}}%
\pgfpathlineto{\pgfqpoint{0.971005in}{1.665626in}}%
\pgfpathlineto{\pgfqpoint{0.941127in}{1.665626in}}%
\pgfpathlineto{\pgfqpoint{0.911248in}{1.665626in}}%
\pgfpathlineto{\pgfqpoint{0.881369in}{1.665626in}}%
\pgfpathlineto{\pgfqpoint{0.851491in}{1.665626in}}%
\pgfpathlineto{\pgfqpoint{0.821612in}{1.665626in}}%
\pgfpathlineto{\pgfqpoint{0.791733in}{1.665626in}}%
\pgfpathlineto{\pgfqpoint{0.761855in}{1.665626in}}%
\pgfpathlineto{\pgfqpoint{0.731976in}{1.665626in}}%
\pgfpathlineto{\pgfqpoint{0.702097in}{1.665626in}}%
\pgfpathlineto{\pgfqpoint{0.672219in}{1.665626in}}%
\pgfpathlineto{\pgfqpoint{0.642340in}{1.665626in}}%
\pgfpathlineto{\pgfqpoint{0.612461in}{1.665626in}}%
\pgfpathlineto{\pgfqpoint{0.582583in}{1.665626in}}%
\pgfpathlineto{\pgfqpoint{0.552704in}{1.665626in}}%
\pgfpathlineto{\pgfqpoint{0.522825in}{1.665626in}}%
\pgfpathlineto{\pgfqpoint{0.492947in}{1.665626in}}%
\pgfpathlineto{\pgfqpoint{0.463068in}{1.665626in}}%
\pgfpathlineto{\pgfqpoint{0.433189in}{1.665626in}}%
\pgfpathlineto{\pgfqpoint{0.403311in}{1.665626in}}%
\pgfpathlineto{\pgfqpoint{0.373432in}{1.665626in}}%
\pgfpathlineto{\pgfqpoint{0.343553in}{1.665626in}}%
\pgfpathlineto{\pgfqpoint{0.313674in}{1.665626in}}%
\pgfpathlineto{\pgfqpoint{0.283796in}{1.665626in}}%
\pgfpathlineto{\pgfqpoint{0.253917in}{1.665626in}}%
\pgfpathlineto{\pgfqpoint{0.224038in}{1.665626in}}%
\pgfpathlineto{\pgfqpoint{0.224038in}{1.655666in}}%
\pgfpathlineto{\pgfqpoint{0.224038in}{1.645706in}}%
\pgfpathlineto{\pgfqpoint{0.224038in}{1.635747in}}%
\pgfpathlineto{\pgfqpoint{0.224038in}{1.625787in}}%
\pgfpathlineto{\pgfqpoint{0.224038in}{1.615828in}}%
\pgfpathlineto{\pgfqpoint{0.224038in}{1.605868in}}%
\pgfpathlineto{\pgfqpoint{0.224038in}{1.595909in}}%
\pgfpathlineto{\pgfqpoint{0.224038in}{1.585949in}}%
\pgfpathlineto{\pgfqpoint{0.224038in}{1.575990in}}%
\pgfpathlineto{\pgfqpoint{0.224038in}{1.566030in}}%
\pgfpathlineto{\pgfqpoint{0.224038in}{1.556070in}}%
\pgfpathlineto{\pgfqpoint{0.224038in}{1.546111in}}%
\pgfpathlineto{\pgfqpoint{0.224038in}{1.536151in}}%
\pgfpathlineto{\pgfqpoint{0.224038in}{1.526192in}}%
\pgfpathlineto{\pgfqpoint{0.224038in}{1.516232in}}%
\pgfpathlineto{\pgfqpoint{0.224038in}{1.506273in}}%
\pgfpathlineto{\pgfqpoint{0.224038in}{1.496313in}}%
\pgfpathlineto{\pgfqpoint{0.224038in}{1.486354in}}%
\pgfpathlineto{\pgfqpoint{0.224038in}{1.476394in}}%
\pgfpathlineto{\pgfqpoint{0.224038in}{1.466434in}}%
\pgfpathlineto{\pgfqpoint{0.224038in}{1.456475in}}%
\pgfpathlineto{\pgfqpoint{0.224038in}{1.446515in}}%
\pgfpathlineto{\pgfqpoint{0.224038in}{1.436556in}}%
\pgfpathlineto{\pgfqpoint{0.224038in}{1.426596in}}%
\pgfpathlineto{\pgfqpoint{0.224038in}{1.416637in}}%
\pgfpathlineto{\pgfqpoint{0.224038in}{1.406677in}}%
\pgfpathlineto{\pgfqpoint{0.224038in}{1.396718in}}%
\pgfpathlineto{\pgfqpoint{0.224038in}{1.386758in}}%
\pgfpathlineto{\pgfqpoint{0.224038in}{1.376798in}}%
\pgfpathlineto{\pgfqpoint{0.224038in}{1.366839in}}%
\pgfpathlineto{\pgfqpoint{0.224038in}{1.356879in}}%
\pgfpathlineto{\pgfqpoint{0.224038in}{1.346920in}}%
\pgfpathlineto{\pgfqpoint{0.224038in}{1.336960in}}%
\pgfpathlineto{\pgfqpoint{0.224038in}{1.327001in}}%
\pgfpathlineto{\pgfqpoint{0.224038in}{1.317041in}}%
\pgfpathlineto{\pgfqpoint{0.224038in}{1.307081in}}%
\pgfpathlineto{\pgfqpoint{0.224038in}{1.297122in}}%
\pgfpathlineto{\pgfqpoint{0.224038in}{1.287162in}}%
\pgfpathlineto{\pgfqpoint{0.224038in}{1.277203in}}%
\pgfpathlineto{\pgfqpoint{0.224038in}{1.267243in}}%
\pgfpathlineto{\pgfqpoint{0.224038in}{1.257284in}}%
\pgfpathlineto{\pgfqpoint{0.224038in}{1.247324in}}%
\pgfpathlineto{\pgfqpoint{0.224038in}{1.237365in}}%
\pgfpathlineto{\pgfqpoint{0.224038in}{1.227405in}}%
\pgfpathlineto{\pgfqpoint{0.224038in}{1.217445in}}%
\pgfpathlineto{\pgfqpoint{0.224038in}{1.207486in}}%
\pgfpathlineto{\pgfqpoint{0.224038in}{1.197526in}}%
\pgfpathlineto{\pgfqpoint{0.224038in}{1.187567in}}%
\pgfpathlineto{\pgfqpoint{0.224038in}{1.177607in}}%
\pgfpathlineto{\pgfqpoint{0.224038in}{1.167648in}}%
\pgfpathlineto{\pgfqpoint{0.224038in}{1.157688in}}%
\pgfpathlineto{\pgfqpoint{0.224038in}{1.147729in}}%
\pgfpathlineto{\pgfqpoint{0.224038in}{1.137769in}}%
\pgfpathlineto{\pgfqpoint{0.224038in}{1.127809in}}%
\pgfpathlineto{\pgfqpoint{0.224038in}{1.117850in}}%
\pgfpathlineto{\pgfqpoint{0.224038in}{1.107890in}}%
\pgfpathlineto{\pgfqpoint{0.224038in}{1.097931in}}%
\pgfpathlineto{\pgfqpoint{0.224038in}{1.087971in}}%
\pgfpathlineto{\pgfqpoint{0.224038in}{1.078012in}}%
\pgfpathlineto{\pgfqpoint{0.224038in}{1.068052in}}%
\pgfpathlineto{\pgfqpoint{0.224038in}{1.058092in}}%
\pgfpathlineto{\pgfqpoint{0.224038in}{1.048133in}}%
\pgfpathlineto{\pgfqpoint{0.224038in}{1.038173in}}%
\pgfpathlineto{\pgfqpoint{0.224038in}{1.028214in}}%
\pgfpathlineto{\pgfqpoint{0.224038in}{1.018254in}}%
\pgfpathlineto{\pgfqpoint{0.224038in}{1.008295in}}%
\pgfpathlineto{\pgfqpoint{0.224038in}{0.998335in}}%
\pgfpathlineto{\pgfqpoint{0.224038in}{0.988376in}}%
\pgfpathlineto{\pgfqpoint{0.224038in}{0.978416in}}%
\pgfpathlineto{\pgfqpoint{0.224038in}{0.968456in}}%
\pgfpathlineto{\pgfqpoint{0.224038in}{0.958497in}}%
\pgfpathlineto{\pgfqpoint{0.224038in}{0.948537in}}%
\pgfpathlineto{\pgfqpoint{0.224038in}{0.938578in}}%
\pgfpathlineto{\pgfqpoint{0.224038in}{0.928618in}}%
\pgfpathlineto{\pgfqpoint{0.224038in}{0.918659in}}%
\pgfpathlineto{\pgfqpoint{0.224038in}{0.908699in}}%
\pgfpathlineto{\pgfqpoint{0.224038in}{0.898740in}}%
\pgfpathlineto{\pgfqpoint{0.224038in}{0.888780in}}%
\pgfpathlineto{\pgfqpoint{0.224038in}{0.878820in}}%
\pgfpathlineto{\pgfqpoint{0.224038in}{0.868861in}}%
\pgfpathlineto{\pgfqpoint{0.224038in}{0.858901in}}%
\pgfpathlineto{\pgfqpoint{0.224038in}{0.848942in}}%
\pgfpathlineto{\pgfqpoint{0.224038in}{0.838982in}}%
\pgfpathlineto{\pgfqpoint{0.224038in}{0.829023in}}%
\pgfpathlineto{\pgfqpoint{0.224038in}{0.819063in}}%
\pgfpathlineto{\pgfqpoint{0.224038in}{0.809103in}}%
\pgfpathlineto{\pgfqpoint{0.224038in}{0.799144in}}%
\pgfpathlineto{\pgfqpoint{0.224038in}{0.789184in}}%
\pgfpathlineto{\pgfqpoint{0.224038in}{0.779225in}}%
\pgfpathlineto{\pgfqpoint{0.224038in}{0.769265in}}%
\pgfpathlineto{\pgfqpoint{0.224038in}{0.759306in}}%
\pgfpathlineto{\pgfqpoint{0.224038in}{0.749346in}}%
\pgfpathlineto{\pgfqpoint{0.224038in}{0.739387in}}%
\pgfpathlineto{\pgfqpoint{0.224038in}{0.729427in}}%
\pgfpathlineto{\pgfqpoint{0.224038in}{0.719467in}}%
\pgfpathlineto{\pgfqpoint{0.224038in}{0.709508in}}%
\pgfpathlineto{\pgfqpoint{0.224038in}{0.699548in}}%
\pgfpathlineto{\pgfqpoint{0.224038in}{0.689589in}}%
\pgfpathlineto{\pgfqpoint{0.224038in}{0.679629in}}%
\pgfpathlineto{\pgfqpoint{0.224038in}{0.669670in}}%
\pgfpathlineto{\pgfqpoint{0.224038in}{0.659710in}}%
\pgfpathlineto{\pgfqpoint{0.224038in}{0.649751in}}%
\pgfpathlineto{\pgfqpoint{0.224038in}{0.639791in}}%
\pgfpathlineto{\pgfqpoint{0.224038in}{0.629831in}}%
\pgfpathlineto{\pgfqpoint{0.224038in}{0.619872in}}%
\pgfpathlineto{\pgfqpoint{0.224038in}{0.609912in}}%
\pgfpathlineto{\pgfqpoint{0.224038in}{0.599953in}}%
\pgfpathlineto{\pgfqpoint{0.224038in}{0.589993in}}%
\pgfpathlineto{\pgfqpoint{0.224038in}{0.580034in}}%
\pgfpathlineto{\pgfqpoint{0.224038in}{0.570074in}}%
\pgfpathlineto{\pgfqpoint{0.224038in}{0.560115in}}%
\pgfpathlineto{\pgfqpoint{0.224038in}{0.550155in}}%
\pgfpathlineto{\pgfqpoint{0.224038in}{0.540195in}}%
\pgfpathlineto{\pgfqpoint{0.224038in}{0.530236in}}%
\pgfpathlineto{\pgfqpoint{0.224038in}{0.520276in}}%
\pgfpathlineto{\pgfqpoint{0.224038in}{0.510317in}}%
\pgfpathlineto{\pgfqpoint{0.224038in}{0.500357in}}%
\pgfpathlineto{\pgfqpoint{0.224038in}{0.490398in}}%
\pgfpathlineto{\pgfqpoint{0.224038in}{0.480438in}}%
\pgfpathlineto{\pgfqpoint{0.224038in}{0.470478in}}%
\pgfpathlineto{\pgfqpoint{0.224038in}{0.460519in}}%
\pgfpathlineto{\pgfqpoint{0.224038in}{0.450559in}}%
\pgfpathlineto{\pgfqpoint{0.224038in}{0.440600in}}%
\pgfpathlineto{\pgfqpoint{0.224038in}{0.430640in}}%
\pgfpathlineto{\pgfqpoint{0.224038in}{0.420681in}}%
\pgfpathlineto{\pgfqpoint{0.224038in}{0.410721in}}%
\pgfpathlineto{\pgfqpoint{0.224038in}{0.400762in}}%
\pgfpathlineto{\pgfqpoint{0.224038in}{0.390802in}}%
\pgfpathlineto{\pgfqpoint{0.224038in}{0.380842in}}%
\pgfpathlineto{\pgfqpoint{0.224038in}{0.370883in}}%
\pgfpathlineto{\pgfqpoint{0.224038in}{0.360923in}}%
\pgfpathlineto{\pgfqpoint{0.224038in}{0.350964in}}%
\pgfpathlineto{\pgfqpoint{0.224038in}{0.341004in}}%
\pgfpathlineto{\pgfqpoint{0.224038in}{0.331045in}}%
\pgfpathlineto{\pgfqpoint{0.224038in}{0.321085in}}%
\pgfpathlineto{\pgfqpoint{0.224038in}{0.311126in}}%
\pgfpathlineto{\pgfqpoint{0.224038in}{0.301166in}}%
\pgfpathlineto{\pgfqpoint{0.224038in}{0.291206in}}%
\pgfpathlineto{\pgfqpoint{0.224038in}{0.281247in}}%
\pgfpathlineto{\pgfqpoint{0.224038in}{0.271287in}}%
\pgfpathlineto{\pgfqpoint{0.224038in}{0.261328in}}%
\pgfpathlineto{\pgfqpoint{0.224038in}{0.251368in}}%
\pgfpathlineto{\pgfqpoint{0.224038in}{0.241409in}}%
\pgfpathlineto{\pgfqpoint{0.224038in}{0.231449in}}%
\pgfpathlineto{\pgfqpoint{0.224038in}{0.221489in}}%
\pgfpathlineto{\pgfqpoint{0.224038in}{0.211530in}}%
\pgfpathlineto{\pgfqpoint{0.224038in}{0.201570in}}%
\pgfpathlineto{\pgfqpoint{0.224038in}{0.191611in}}%
\pgfpathlineto{\pgfqpoint{0.224038in}{0.181651in}}%
\pgfpathclose%
\pgfpathmoveto{\pgfqpoint{3.793790in}{0.191611in}}%
\pgfpathlineto{\pgfqpoint{3.779601in}{0.193338in}}%
\pgfpathlineto{\pgfqpoint{3.749723in}{0.197538in}}%
\pgfpathlineto{\pgfqpoint{3.723642in}{0.201570in}}%
\pgfpathlineto{\pgfqpoint{3.719844in}{0.202146in}}%
\pgfpathlineto{\pgfqpoint{3.689965in}{0.207210in}}%
\pgfpathlineto{\pgfqpoint{3.666098in}{0.211530in}}%
\pgfpathlineto{\pgfqpoint{3.660086in}{0.212598in}}%
\pgfpathlineto{\pgfqpoint{3.630208in}{0.218354in}}%
\pgfpathlineto{\pgfqpoint{3.614857in}{0.221489in}}%
\pgfpathlineto{\pgfqpoint{3.600329in}{0.224407in}}%
\pgfpathlineto{\pgfqpoint{3.570450in}{0.230735in}}%
\pgfpathlineto{\pgfqpoint{3.567272in}{0.231449in}}%
\pgfpathlineto{\pgfqpoint{3.540572in}{0.237360in}}%
\pgfpathlineto{\pgfqpoint{3.523043in}{0.241409in}}%
\pgfpathlineto{\pgfqpoint{3.510693in}{0.244218in}}%
\pgfpathlineto{\pgfqpoint{3.480814in}{0.251302in}}%
\pgfpathlineto{\pgfqpoint{3.480548in}{0.251368in}}%
\pgfpathlineto{\pgfqpoint{3.450936in}{0.258638in}}%
\pgfpathlineto{\pgfqpoint{3.440364in}{0.261328in}}%
\pgfpathlineto{\pgfqpoint{3.421057in}{0.266176in}}%
\pgfpathlineto{\pgfqpoint{3.401394in}{0.271287in}}%
\pgfpathlineto{\pgfqpoint{3.391178in}{0.273909in}}%
\pgfpathlineto{\pgfqpoint{3.363527in}{0.281247in}}%
\pgfpathlineto{\pgfqpoint{3.361300in}{0.281831in}}%
\pgfpathlineto{\pgfqpoint{3.331421in}{0.289948in}}%
\pgfpathlineto{\pgfqpoint{3.326930in}{0.291206in}}%
\pgfpathlineto{\pgfqpoint{3.301542in}{0.298243in}}%
\pgfpathlineto{\pgfqpoint{3.291296in}{0.301166in}}%
\pgfpathlineto{\pgfqpoint{3.271664in}{0.306706in}}%
\pgfpathlineto{\pgfqpoint{3.256432in}{0.311126in}}%
\pgfpathlineto{\pgfqpoint{3.241785in}{0.315331in}}%
\pgfpathlineto{\pgfqpoint{3.222279in}{0.321085in}}%
\pgfpathlineto{\pgfqpoint{3.211906in}{0.324114in}}%
\pgfpathlineto{\pgfqpoint{3.188784in}{0.331045in}}%
\pgfpathlineto{\pgfqpoint{3.182028in}{0.333050in}}%
\pgfpathlineto{\pgfqpoint{3.155898in}{0.341004in}}%
\pgfpathlineto{\pgfqpoint{3.152149in}{0.342135in}}%
\pgfpathlineto{\pgfqpoint{3.123580in}{0.350964in}}%
\pgfpathlineto{\pgfqpoint{3.122270in}{0.351365in}}%
\pgfpathlineto{\pgfqpoint{3.092392in}{0.360737in}}%
\pgfpathlineto{\pgfqpoint{3.091811in}{0.360923in}}%
\pgfpathlineto{\pgfqpoint{3.062513in}{0.370248in}}%
\pgfpathlineto{\pgfqpoint{3.060560in}{0.370883in}}%
\pgfpathlineto{\pgfqpoint{3.032634in}{0.379893in}}%
\pgfpathlineto{\pgfqpoint{3.029749in}{0.380842in}}%
\pgfpathlineto{\pgfqpoint{3.002756in}{0.389668in}}%
\pgfpathlineto{\pgfqpoint{2.999354in}{0.390802in}}%
\pgfpathlineto{\pgfqpoint{2.972877in}{0.399572in}}%
\pgfpathlineto{\pgfqpoint{2.969351in}{0.400762in}}%
\pgfpathlineto{\pgfqpoint{2.942998in}{0.409601in}}%
\pgfpathlineto{\pgfqpoint{2.939719in}{0.410721in}}%
\pgfpathlineto{\pgfqpoint{2.913120in}{0.419754in}}%
\pgfpathlineto{\pgfqpoint{2.910438in}{0.420681in}}%
\pgfpathlineto{\pgfqpoint{2.883241in}{0.430029in}}%
\pgfpathlineto{\pgfqpoint{2.881491in}{0.430640in}}%
\pgfpathlineto{\pgfqpoint{2.853362in}{0.440423in}}%
\pgfpathlineto{\pgfqpoint{2.852861in}{0.440600in}}%
\pgfpathlineto{\pgfqpoint{2.824557in}{0.450559in}}%
\pgfpathlineto{\pgfqpoint{2.823483in}{0.450935in}}%
\pgfpathlineto{\pgfqpoint{2.796555in}{0.460519in}}%
\pgfpathlineto{\pgfqpoint{2.793605in}{0.461564in}}%
\pgfpathlineto{\pgfqpoint{2.768831in}{0.470478in}}%
\pgfpathlineto{\pgfqpoint{2.763726in}{0.472308in}}%
\pgfpathlineto{\pgfqpoint{2.741374in}{0.480438in}}%
\pgfpathlineto{\pgfqpoint{2.733847in}{0.483166in}}%
\pgfpathlineto{\pgfqpoint{2.714171in}{0.490398in}}%
\pgfpathlineto{\pgfqpoint{2.703969in}{0.494136in}}%
\pgfpathlineto{\pgfqpoint{2.687214in}{0.500357in}}%
\pgfpathlineto{\pgfqpoint{2.674090in}{0.505217in}}%
\pgfpathlineto{\pgfqpoint{2.660493in}{0.510317in}}%
\pgfpathlineto{\pgfqpoint{2.644211in}{0.516409in}}%
\pgfpathlineto{\pgfqpoint{2.634000in}{0.520276in}}%
\pgfpathlineto{\pgfqpoint{2.614333in}{0.527711in}}%
\pgfpathlineto{\pgfqpoint{2.607729in}{0.530236in}}%
\pgfpathlineto{\pgfqpoint{2.584454in}{0.539122in}}%
\pgfpathlineto{\pgfqpoint{2.581672in}{0.540195in}}%
\pgfpathlineto{\pgfqpoint{2.555839in}{0.550155in}}%
\pgfpathlineto{\pgfqpoint{2.554575in}{0.550641in}}%
\pgfpathlineto{\pgfqpoint{2.530239in}{0.560115in}}%
\pgfpathlineto{\pgfqpoint{2.524697in}{0.562270in}}%
\pgfpathlineto{\pgfqpoint{2.504832in}{0.570074in}}%
\pgfpathlineto{\pgfqpoint{2.494818in}{0.574006in}}%
\pgfpathlineto{\pgfqpoint{2.479613in}{0.580034in}}%
\pgfpathlineto{\pgfqpoint{2.464939in}{0.585849in}}%
\pgfpathlineto{\pgfqpoint{2.454580in}{0.589993in}}%
\pgfpathlineto{\pgfqpoint{2.435061in}{0.597801in}}%
\pgfpathlineto{\pgfqpoint{2.429728in}{0.599953in}}%
\pgfpathlineto{\pgfqpoint{2.405182in}{0.609860in}}%
\pgfpathlineto{\pgfqpoint{2.405053in}{0.609912in}}%
\pgfpathlineto{\pgfqpoint{2.380593in}{0.619872in}}%
\pgfpathlineto{\pgfqpoint{2.375303in}{0.622027in}}%
\pgfpathlineto{\pgfqpoint{2.356299in}{0.629831in}}%
\pgfpathlineto{\pgfqpoint{2.345425in}{0.634302in}}%
\pgfpathlineto{\pgfqpoint{2.332171in}{0.639791in}}%
\pgfpathlineto{\pgfqpoint{2.315546in}{0.646685in}}%
\pgfpathlineto{\pgfqpoint{2.308205in}{0.649751in}}%
\pgfpathlineto{\pgfqpoint{2.285667in}{0.659177in}}%
\pgfpathlineto{\pgfqpoint{2.284400in}{0.659710in}}%
\pgfpathlineto{\pgfqpoint{2.260779in}{0.669670in}}%
\pgfpathlineto{\pgfqpoint{2.255789in}{0.671778in}}%
\pgfpathlineto{\pgfqpoint{2.237318in}{0.679629in}}%
\pgfpathlineto{\pgfqpoint{2.225910in}{0.684489in}}%
\pgfpathlineto{\pgfqpoint{2.214008in}{0.689589in}}%
\pgfpathlineto{\pgfqpoint{2.196031in}{0.697310in}}%
\pgfpathlineto{\pgfqpoint{2.190849in}{0.699548in}}%
\pgfpathlineto{\pgfqpoint{2.167846in}{0.709508in}}%
\pgfpathlineto{\pgfqpoint{2.166153in}{0.710243in}}%
\pgfpathlineto{\pgfqpoint{2.145008in}{0.719467in}}%
\pgfpathlineto{\pgfqpoint{2.136274in}{0.723290in}}%
\pgfpathlineto{\pgfqpoint{2.122314in}{0.729427in}}%
\pgfpathlineto{\pgfqpoint{2.106395in}{0.736449in}}%
\pgfpathlineto{\pgfqpoint{2.099764in}{0.739387in}}%
\pgfpathlineto{\pgfqpoint{2.077359in}{0.749346in}}%
\pgfpathlineto{\pgfqpoint{2.076517in}{0.749722in}}%
\pgfpathlineto{\pgfqpoint{2.055111in}{0.759306in}}%
\pgfpathlineto{\pgfqpoint{2.046638in}{0.763114in}}%
\pgfpathlineto{\pgfqpoint{2.033002in}{0.769265in}}%
\pgfpathlineto{\pgfqpoint{2.016759in}{0.776623in}}%
\pgfpathlineto{\pgfqpoint{2.011033in}{0.779225in}}%
\pgfpathlineto{\pgfqpoint{1.989206in}{0.789184in}}%
\pgfpathlineto{\pgfqpoint{1.986880in}{0.790250in}}%
\pgfpathlineto{\pgfqpoint{1.967525in}{0.799144in}}%
\pgfpathlineto{\pgfqpoint{1.957002in}{0.804002in}}%
\pgfpathlineto{\pgfqpoint{1.945979in}{0.809103in}}%
\pgfpathlineto{\pgfqpoint{1.927123in}{0.817873in}}%
\pgfpathlineto{\pgfqpoint{1.924571in}{0.819063in}}%
\pgfpathlineto{\pgfqpoint{1.903303in}{0.829023in}}%
\pgfpathlineto{\pgfqpoint{1.897244in}{0.831875in}}%
\pgfpathlineto{\pgfqpoint{1.882173in}{0.838982in}}%
\pgfpathlineto{\pgfqpoint{1.867366in}{0.846003in}}%
\pgfpathlineto{\pgfqpoint{1.861177in}{0.848942in}}%
\pgfpathlineto{\pgfqpoint{1.840317in}{0.858901in}}%
\pgfpathlineto{\pgfqpoint{1.837487in}{0.860260in}}%
\pgfpathlineto{\pgfqpoint{1.819594in}{0.868861in}}%
\pgfpathlineto{\pgfqpoint{1.807608in}{0.874656in}}%
\pgfpathlineto{\pgfqpoint{1.799004in}{0.878820in}}%
\pgfpathlineto{\pgfqpoint{1.778549in}{0.888780in}}%
\pgfpathlineto{\pgfqpoint{1.777730in}{0.889181in}}%
\pgfpathlineto{\pgfqpoint{1.758229in}{0.898740in}}%
\pgfpathlineto{\pgfqpoint{1.747851in}{0.903859in}}%
\pgfpathlineto{\pgfqpoint{1.738042in}{0.908699in}}%
\pgfpathlineto{\pgfqpoint{1.717989in}{0.918659in}}%
\pgfpathlineto{\pgfqpoint{1.717972in}{0.918667in}}%
\pgfpathlineto{\pgfqpoint{1.698070in}{0.928618in}}%
\pgfpathlineto{\pgfqpoint{1.688094in}{0.933641in}}%
\pgfpathlineto{\pgfqpoint{1.678285in}{0.938578in}}%
\pgfpathlineto{\pgfqpoint{1.658633in}{0.948537in}}%
\pgfpathlineto{\pgfqpoint{1.658215in}{0.948751in}}%
\pgfpathlineto{\pgfqpoint{1.639115in}{0.958497in}}%
\pgfpathlineto{\pgfqpoint{1.628336in}{0.964037in}}%
\pgfpathlineto{\pgfqpoint{1.619732in}{0.968456in}}%
\pgfpathlineto{\pgfqpoint{1.600483in}{0.978416in}}%
\pgfpathlineto{\pgfqpoint{1.598458in}{0.979472in}}%
\pgfpathlineto{\pgfqpoint{1.581369in}{0.988376in}}%
\pgfpathlineto{\pgfqpoint{1.568579in}{0.995090in}}%
\pgfpathlineto{\pgfqpoint{1.562390in}{0.998335in}}%
\pgfpathlineto{\pgfqpoint{1.543547in}{1.008295in}}%
\pgfpathlineto{\pgfqpoint{1.538700in}{1.010878in}}%
\pgfpathlineto{\pgfqpoint{1.524841in}{1.018254in}}%
\pgfpathlineto{\pgfqpoint{1.508822in}{1.026847in}}%
\pgfpathlineto{\pgfqpoint{1.506269in}{1.028214in}}%
\pgfpathlineto{\pgfqpoint{1.487839in}{1.038173in}}%
\pgfpathlineto{\pgfqpoint{1.478943in}{1.043021in}}%
\pgfpathlineto{\pgfqpoint{1.469546in}{1.048133in}}%
\pgfpathlineto{\pgfqpoint{1.451390in}{1.058092in}}%
\pgfpathlineto{\pgfqpoint{1.449064in}{1.059381in}}%
\pgfpathlineto{\pgfqpoint{1.433381in}{1.068052in}}%
\pgfpathlineto{\pgfqpoint{1.419186in}{1.075965in}}%
\pgfpathlineto{\pgfqpoint{1.415506in}{1.078012in}}%
\pgfpathlineto{\pgfqpoint{1.397780in}{1.087971in}}%
\pgfpathlineto{\pgfqpoint{1.389307in}{1.092776in}}%
\pgfpathlineto{\pgfqpoint{1.380196in}{1.097931in}}%
\pgfpathlineto{\pgfqpoint{1.362754in}{1.107890in}}%
\pgfpathlineto{\pgfqpoint{1.359428in}{1.109810in}}%
\pgfpathlineto{\pgfqpoint{1.345468in}{1.117850in}}%
\pgfpathlineto{\pgfqpoint{1.329550in}{1.127095in}}%
\pgfpathlineto{\pgfqpoint{1.328316in}{1.127809in}}%
\pgfpathlineto{\pgfqpoint{1.311334in}{1.137769in}}%
\pgfpathlineto{\pgfqpoint{1.299671in}{1.144670in}}%
\pgfpathlineto{\pgfqpoint{1.294489in}{1.147729in}}%
\pgfpathlineto{\pgfqpoint{1.277807in}{1.157688in}}%
\pgfpathlineto{\pgfqpoint{1.269792in}{1.162525in}}%
\pgfpathlineto{\pgfqpoint{1.261278in}{1.167648in}}%
\pgfpathlineto{\pgfqpoint{1.244904in}{1.177607in}}%
\pgfpathlineto{\pgfqpoint{1.239914in}{1.180681in}}%
\pgfpathlineto{\pgfqpoint{1.228702in}{1.187567in}}%
\pgfpathlineto{\pgfqpoint{1.212644in}{1.197526in}}%
\pgfpathlineto{\pgfqpoint{1.210035in}{1.199169in}}%
\pgfpathlineto{\pgfqpoint{1.196781in}{1.207486in}}%
\pgfpathlineto{\pgfqpoint{1.181050in}{1.217445in}}%
\pgfpathlineto{\pgfqpoint{1.180156in}{1.218021in}}%
\pgfpathlineto{\pgfqpoint{1.165537in}{1.227405in}}%
\pgfpathlineto{\pgfqpoint{1.150277in}{1.237282in}}%
\pgfpathlineto{\pgfqpoint{1.150149in}{1.237365in}}%
\pgfpathlineto{\pgfqpoint{1.134998in}{1.247324in}}%
\pgfpathlineto{\pgfqpoint{1.120399in}{1.257000in}}%
\pgfpathlineto{\pgfqpoint{1.119970in}{1.257284in}}%
\pgfpathlineto{\pgfqpoint{1.105194in}{1.267243in}}%
\pgfpathlineto{\pgfqpoint{1.090539in}{1.277203in}}%
\pgfpathlineto{\pgfqpoint{1.090520in}{1.277216in}}%
\pgfpathlineto{\pgfqpoint{1.076159in}{1.287162in}}%
\pgfpathlineto{\pgfqpoint{1.061905in}{1.297122in}}%
\pgfpathlineto{\pgfqpoint{1.060641in}{1.298024in}}%
\pgfpathlineto{\pgfqpoint{1.047930in}{1.307081in}}%
\pgfpathlineto{\pgfqpoint{1.034105in}{1.317041in}}%
\pgfpathlineto{\pgfqpoint{1.030763in}{1.319502in}}%
\pgfpathlineto{\pgfqpoint{1.020552in}{1.327001in}}%
\pgfpathlineto{\pgfqpoint{1.007188in}{1.336960in}}%
\pgfpathlineto{\pgfqpoint{1.000884in}{1.341753in}}%
\pgfpathlineto{\pgfqpoint{0.994071in}{1.346920in}}%
\pgfpathlineto{\pgfqpoint{0.981208in}{1.356879in}}%
\pgfpathlineto{\pgfqpoint{0.971005in}{1.364904in}}%
\pgfpathlineto{\pgfqpoint{0.968540in}{1.366839in}}%
\pgfpathlineto{\pgfqpoint{0.956221in}{1.376798in}}%
\pgfpathlineto{\pgfqpoint{0.944077in}{1.386758in}}%
\pgfpathlineto{\pgfqpoint{0.941127in}{1.389255in}}%
\pgfpathlineto{\pgfqpoint{0.932294in}{1.396718in}}%
\pgfpathlineto{\pgfqpoint{0.920775in}{1.406677in}}%
\pgfpathlineto{\pgfqpoint{0.911248in}{1.415092in}}%
\pgfpathlineto{\pgfqpoint{0.909498in}{1.416637in}}%
\pgfpathlineto{\pgfqpoint{0.898668in}{1.426596in}}%
\pgfpathlineto{\pgfqpoint{0.888082in}{1.436556in}}%
\pgfpathlineto{\pgfqpoint{0.881369in}{1.443086in}}%
\pgfpathlineto{\pgfqpoint{0.877844in}{1.446515in}}%
\pgfpathlineto{\pgfqpoint{0.868074in}{1.456475in}}%
\pgfpathlineto{\pgfqpoint{0.858612in}{1.466434in}}%
\pgfpathlineto{\pgfqpoint{0.851491in}{1.474249in}}%
\pgfpathlineto{\pgfqpoint{0.849538in}{1.476394in}}%
\pgfpathlineto{\pgfqpoint{0.841061in}{1.486354in}}%
\pgfpathlineto{\pgfqpoint{0.832991in}{1.496313in}}%
\pgfpathlineto{\pgfqpoint{0.825362in}{1.506273in}}%
\pgfpathlineto{\pgfqpoint{0.821612in}{1.511595in}}%
\pgfpathlineto{\pgfqpoint{0.818356in}{1.516232in}}%
\pgfpathlineto{\pgfqpoint{0.812057in}{1.526192in}}%
\pgfpathlineto{\pgfqpoint{0.806380in}{1.536151in}}%
\pgfpathlineto{\pgfqpoint{0.801404in}{1.546111in}}%
\pgfpathlineto{\pgfqpoint{0.797223in}{1.556070in}}%
\pgfpathlineto{\pgfqpoint{0.793961in}{1.566030in}}%
\pgfpathlineto{\pgfqpoint{0.791770in}{1.575990in}}%
\pgfpathlineto{\pgfqpoint{0.791733in}{1.576404in}}%
\pgfpathlineto{\pgfqpoint{0.790908in}{1.585949in}}%
\pgfpathlineto{\pgfqpoint{0.791467in}{1.595909in}}%
\pgfpathlineto{\pgfqpoint{0.791733in}{1.597154in}}%
\pgfpathlineto{\pgfqpoint{0.793897in}{1.605868in}}%
\pgfpathlineto{\pgfqpoint{0.798639in}{1.615828in}}%
\pgfpathlineto{\pgfqpoint{0.806261in}{1.625787in}}%
\pgfpathlineto{\pgfqpoint{0.817574in}{1.635747in}}%
\pgfpathlineto{\pgfqpoint{0.821612in}{1.638431in}}%
\pgfpathlineto{\pgfqpoint{0.835712in}{1.645706in}}%
\pgfpathlineto{\pgfqpoint{0.851491in}{1.651799in}}%
\pgfpathlineto{\pgfqpoint{0.865680in}{1.655666in}}%
\pgfpathlineto{\pgfqpoint{0.881369in}{1.659095in}}%
\pgfpathlineto{\pgfqpoint{0.911248in}{1.663166in}}%
\pgfpathlineto{\pgfqpoint{0.941127in}{1.665164in}}%
\pgfpathlineto{\pgfqpoint{0.971005in}{1.665609in}}%
\pgfpathlineto{\pgfqpoint{1.000884in}{1.664859in}}%
\pgfpathlineto{\pgfqpoint{1.030763in}{1.663165in}}%
\pgfpathlineto{\pgfqpoint{1.060641in}{1.660707in}}%
\pgfpathlineto{\pgfqpoint{1.090520in}{1.657618in}}%
\pgfpathlineto{\pgfqpoint{1.106210in}{1.655666in}}%
\pgfpathlineto{\pgfqpoint{1.120399in}{1.653938in}}%
\pgfpathlineto{\pgfqpoint{1.150277in}{1.649739in}}%
\pgfpathlineto{\pgfqpoint{1.176358in}{1.645706in}}%
\pgfpathlineto{\pgfqpoint{1.180156in}{1.645131in}}%
\pgfpathlineto{\pgfqpoint{1.210035in}{1.640067in}}%
\pgfpathlineto{\pgfqpoint{1.233902in}{1.635747in}}%
\pgfpathlineto{\pgfqpoint{1.239914in}{1.634678in}}%
\pgfpathlineto{\pgfqpoint{1.269792in}{1.628922in}}%
\pgfpathlineto{\pgfqpoint{1.285143in}{1.625787in}}%
\pgfpathlineto{\pgfqpoint{1.299671in}{1.622870in}}%
\pgfpathlineto{\pgfqpoint{1.329550in}{1.616542in}}%
\pgfpathlineto{\pgfqpoint{1.332728in}{1.615828in}}%
\pgfpathlineto{\pgfqpoint{1.359428in}{1.609917in}}%
\pgfpathlineto{\pgfqpoint{1.376957in}{1.605868in}}%
\pgfpathlineto{\pgfqpoint{1.389307in}{1.603059in}}%
\pgfpathlineto{\pgfqpoint{1.419186in}{1.595975in}}%
\pgfpathlineto{\pgfqpoint{1.419452in}{1.595909in}}%
\pgfpathlineto{\pgfqpoint{1.449064in}{1.588639in}}%
\pgfpathlineto{\pgfqpoint{1.459636in}{1.585949in}}%
\pgfpathlineto{\pgfqpoint{1.478943in}{1.581101in}}%
\pgfpathlineto{\pgfqpoint{1.498606in}{1.575990in}}%
\pgfpathlineto{\pgfqpoint{1.508822in}{1.573368in}}%
\pgfpathlineto{\pgfqpoint{1.536473in}{1.566030in}}%
\pgfpathlineto{\pgfqpoint{1.538700in}{1.565446in}}%
\pgfpathlineto{\pgfqpoint{1.568579in}{1.557329in}}%
\pgfpathlineto{\pgfqpoint{1.573070in}{1.556070in}}%
\pgfpathlineto{\pgfqpoint{1.598458in}{1.549034in}}%
\pgfpathlineto{\pgfqpoint{1.608704in}{1.546111in}}%
\pgfpathlineto{\pgfqpoint{1.628336in}{1.540571in}}%
\pgfpathlineto{\pgfqpoint{1.643568in}{1.536151in}}%
\pgfpathlineto{\pgfqpoint{1.658215in}{1.531945in}}%
\pgfpathlineto{\pgfqpoint{1.677721in}{1.526192in}}%
\pgfpathlineto{\pgfqpoint{1.688094in}{1.523163in}}%
\pgfpathlineto{\pgfqpoint{1.711216in}{1.516232in}}%
\pgfpathlineto{\pgfqpoint{1.717972in}{1.514227in}}%
\pgfpathlineto{\pgfqpoint{1.744102in}{1.506273in}}%
\pgfpathlineto{\pgfqpoint{1.747851in}{1.505142in}}%
\pgfpathlineto{\pgfqpoint{1.776420in}{1.496313in}}%
\pgfpathlineto{\pgfqpoint{1.777730in}{1.495912in}}%
\pgfpathlineto{\pgfqpoint{1.807608in}{1.486540in}}%
\pgfpathlineto{\pgfqpoint{1.808189in}{1.486354in}}%
\pgfpathlineto{\pgfqpoint{1.837487in}{1.477029in}}%
\pgfpathlineto{\pgfqpoint{1.839440in}{1.476394in}}%
\pgfpathlineto{\pgfqpoint{1.867366in}{1.467384in}}%
\pgfpathlineto{\pgfqpoint{1.870251in}{1.466434in}}%
\pgfpathlineto{\pgfqpoint{1.897244in}{1.457609in}}%
\pgfpathlineto{\pgfqpoint{1.900646in}{1.456475in}}%
\pgfpathlineto{\pgfqpoint{1.927123in}{1.447705in}}%
\pgfpathlineto{\pgfqpoint{1.930649in}{1.446515in}}%
\pgfpathlineto{\pgfqpoint{1.957002in}{1.437676in}}%
\pgfpathlineto{\pgfqpoint{1.960281in}{1.436556in}}%
\pgfpathlineto{\pgfqpoint{1.986880in}{1.427523in}}%
\pgfpathlineto{\pgfqpoint{1.989562in}{1.426596in}}%
\pgfpathlineto{\pgfqpoint{2.016759in}{1.417248in}}%
\pgfpathlineto{\pgfqpoint{2.018509in}{1.416637in}}%
\pgfpathlineto{\pgfqpoint{2.046638in}{1.406854in}}%
\pgfpathlineto{\pgfqpoint{2.047139in}{1.406677in}}%
\pgfpathlineto{\pgfqpoint{2.075443in}{1.396718in}}%
\pgfpathlineto{\pgfqpoint{2.076517in}{1.396342in}}%
\pgfpathlineto{\pgfqpoint{2.103445in}{1.386758in}}%
\pgfpathlineto{\pgfqpoint{2.106395in}{1.385712in}}%
\pgfpathlineto{\pgfqpoint{2.131169in}{1.376798in}}%
\pgfpathlineto{\pgfqpoint{2.136274in}{1.374969in}}%
\pgfpathlineto{\pgfqpoint{2.158626in}{1.366839in}}%
\pgfpathlineto{\pgfqpoint{2.166153in}{1.364111in}}%
\pgfpathlineto{\pgfqpoint{2.185829in}{1.356879in}}%
\pgfpathlineto{\pgfqpoint{2.196031in}{1.353141in}}%
\pgfpathlineto{\pgfqpoint{2.212786in}{1.346920in}}%
\pgfpathlineto{\pgfqpoint{2.225910in}{1.342060in}}%
\pgfpathlineto{\pgfqpoint{2.239507in}{1.336960in}}%
\pgfpathlineto{\pgfqpoint{2.255789in}{1.330868in}}%
\pgfpathlineto{\pgfqpoint{2.266000in}{1.327001in}}%
\pgfpathlineto{\pgfqpoint{2.285667in}{1.319566in}}%
\pgfpathlineto{\pgfqpoint{2.292271in}{1.317041in}}%
\pgfpathlineto{\pgfqpoint{2.315546in}{1.308155in}}%
\pgfpathlineto{\pgfqpoint{2.318328in}{1.307081in}}%
\pgfpathlineto{\pgfqpoint{2.344161in}{1.297122in}}%
\pgfpathlineto{\pgfqpoint{2.345425in}{1.296636in}}%
\pgfpathlineto{\pgfqpoint{2.369761in}{1.287162in}}%
\pgfpathlineto{\pgfqpoint{2.375303in}{1.285007in}}%
\pgfpathlineto{\pgfqpoint{2.395168in}{1.277203in}}%
\pgfpathlineto{\pgfqpoint{2.405182in}{1.273271in}}%
\pgfpathlineto{\pgfqpoint{2.420387in}{1.267243in}}%
\pgfpathlineto{\pgfqpoint{2.435061in}{1.261427in}}%
\pgfpathlineto{\pgfqpoint{2.445420in}{1.257284in}}%
\pgfpathlineto{\pgfqpoint{2.464939in}{1.249476in}}%
\pgfpathlineto{\pgfqpoint{2.470272in}{1.247324in}}%
\pgfpathlineto{\pgfqpoint{2.494818in}{1.237417in}}%
\pgfpathlineto{\pgfqpoint{2.494947in}{1.237365in}}%
\pgfpathlineto{\pgfqpoint{2.519407in}{1.227405in}}%
\pgfpathlineto{\pgfqpoint{2.524697in}{1.225250in}}%
\pgfpathlineto{\pgfqpoint{2.543701in}{1.217445in}}%
\pgfpathlineto{\pgfqpoint{2.554575in}{1.212975in}}%
\pgfpathlineto{\pgfqpoint{2.567829in}{1.207486in}}%
\pgfpathlineto{\pgfqpoint{2.584454in}{1.200592in}}%
\pgfpathlineto{\pgfqpoint{2.591795in}{1.197526in}}%
\pgfpathlineto{\pgfqpoint{2.614333in}{1.188100in}}%
\pgfpathlineto{\pgfqpoint{2.615600in}{1.187567in}}%
\pgfpathlineto{\pgfqpoint{2.639221in}{1.177607in}}%
\pgfpathlineto{\pgfqpoint{2.644211in}{1.175499in}}%
\pgfpathlineto{\pgfqpoint{2.662682in}{1.167648in}}%
\pgfpathlineto{\pgfqpoint{2.674090in}{1.162788in}}%
\pgfpathlineto{\pgfqpoint{2.685992in}{1.157688in}}%
\pgfpathlineto{\pgfqpoint{2.703969in}{1.149966in}}%
\pgfpathlineto{\pgfqpoint{2.709151in}{1.147729in}}%
\pgfpathlineto{\pgfqpoint{2.732154in}{1.137769in}}%
\pgfpathlineto{\pgfqpoint{2.733847in}{1.137034in}}%
\pgfpathlineto{\pgfqpoint{2.754992in}{1.127809in}}%
\pgfpathlineto{\pgfqpoint{2.763726in}{1.123987in}}%
\pgfpathlineto{\pgfqpoint{2.777686in}{1.117850in}}%
\pgfpathlineto{\pgfqpoint{2.793605in}{1.110828in}}%
\pgfpathlineto{\pgfqpoint{2.800236in}{1.107890in}}%
\pgfpathlineto{\pgfqpoint{2.822641in}{1.097931in}}%
\pgfpathlineto{\pgfqpoint{2.823483in}{1.097555in}}%
\pgfpathlineto{\pgfqpoint{2.844889in}{1.087971in}}%
\pgfpathlineto{\pgfqpoint{2.853362in}{1.084162in}}%
\pgfpathlineto{\pgfqpoint{2.866998in}{1.078012in}}%
\pgfpathlineto{\pgfqpoint{2.883241in}{1.070654in}}%
\pgfpathlineto{\pgfqpoint{2.888967in}{1.068052in}}%
\pgfpathlineto{\pgfqpoint{2.910794in}{1.058092in}}%
\pgfpathlineto{\pgfqpoint{2.913120in}{1.057027in}}%
\pgfpathlineto{\pgfqpoint{2.932475in}{1.048133in}}%
\pgfpathlineto{\pgfqpoint{2.942998in}{1.043275in}}%
\pgfpathlineto{\pgfqpoint{2.954021in}{1.038173in}}%
\pgfpathlineto{\pgfqpoint{2.972877in}{1.029403in}}%
\pgfpathlineto{\pgfqpoint{2.975429in}{1.028214in}}%
\pgfpathlineto{\pgfqpoint{2.996697in}{1.018254in}}%
\pgfpathlineto{\pgfqpoint{3.002756in}{1.015402in}}%
\pgfpathlineto{\pgfqpoint{3.017827in}{1.008295in}}%
\pgfpathlineto{\pgfqpoint{3.032634in}{1.001274in}}%
\pgfpathlineto{\pgfqpoint{3.038823in}{0.998335in}}%
\pgfpathlineto{\pgfqpoint{3.059683in}{0.988376in}}%
\pgfpathlineto{\pgfqpoint{3.062513in}{0.987017in}}%
\pgfpathlineto{\pgfqpoint{3.080406in}{0.978416in}}%
\pgfpathlineto{\pgfqpoint{3.092392in}{0.972621in}}%
\pgfpathlineto{\pgfqpoint{3.100996in}{0.968456in}}%
\pgfpathlineto{\pgfqpoint{3.121451in}{0.958497in}}%
\pgfpathlineto{\pgfqpoint{3.122270in}{0.958096in}}%
\pgfpathlineto{\pgfqpoint{3.141771in}{0.948537in}}%
\pgfpathlineto{\pgfqpoint{3.152149in}{0.943418in}}%
\pgfpathlineto{\pgfqpoint{3.161958in}{0.938578in}}%
\pgfpathlineto{\pgfqpoint{3.182011in}{0.928618in}}%
\pgfpathlineto{\pgfqpoint{3.182028in}{0.928610in}}%
\pgfpathlineto{\pgfqpoint{3.201930in}{0.918659in}}%
\pgfpathlineto{\pgfqpoint{3.211906in}{0.913636in}}%
\pgfpathlineto{\pgfqpoint{3.221715in}{0.908699in}}%
\pgfpathlineto{\pgfqpoint{3.241367in}{0.898740in}}%
\pgfpathlineto{\pgfqpoint{3.241785in}{0.898526in}}%
\pgfpathlineto{\pgfqpoint{3.260885in}{0.888780in}}%
\pgfpathlineto{\pgfqpoint{3.271664in}{0.883240in}}%
\pgfpathlineto{\pgfqpoint{3.280268in}{0.878820in}}%
\pgfpathlineto{\pgfqpoint{3.299517in}{0.868861in}}%
\pgfpathlineto{\pgfqpoint{3.301542in}{0.867805in}}%
\pgfpathlineto{\pgfqpoint{3.318631in}{0.858901in}}%
\pgfpathlineto{\pgfqpoint{3.331421in}{0.852187in}}%
\pgfpathlineto{\pgfqpoint{3.337610in}{0.848942in}}%
\pgfpathlineto{\pgfqpoint{3.356453in}{0.838982in}}%
\pgfpathlineto{\pgfqpoint{3.361300in}{0.836399in}}%
\pgfpathlineto{\pgfqpoint{3.375159in}{0.829023in}}%
\pgfpathlineto{\pgfqpoint{3.391178in}{0.820430in}}%
\pgfpathlineto{\pgfqpoint{3.393731in}{0.819063in}}%
\pgfpathlineto{\pgfqpoint{3.412161in}{0.809103in}}%
\pgfpathlineto{\pgfqpoint{3.421057in}{0.804255in}}%
\pgfpathlineto{\pgfqpoint{3.430454in}{0.799144in}}%
\pgfpathlineto{\pgfqpoint{3.448610in}{0.789184in}}%
\pgfpathlineto{\pgfqpoint{3.450936in}{0.787896in}}%
\pgfpathlineto{\pgfqpoint{3.466619in}{0.779225in}}%
\pgfpathlineto{\pgfqpoint{3.480814in}{0.771312in}}%
\pgfpathlineto{\pgfqpoint{3.484494in}{0.769265in}}%
\pgfpathlineto{\pgfqpoint{3.502220in}{0.759306in}}%
\pgfpathlineto{\pgfqpoint{3.510693in}{0.754501in}}%
\pgfpathlineto{\pgfqpoint{3.519804in}{0.749346in}}%
\pgfpathlineto{\pgfqpoint{3.537246in}{0.739387in}}%
\pgfpathlineto{\pgfqpoint{3.540572in}{0.737466in}}%
\pgfpathlineto{\pgfqpoint{3.554532in}{0.729427in}}%
\pgfpathlineto{\pgfqpoint{3.570450in}{0.720182in}}%
\pgfpathlineto{\pgfqpoint{3.571684in}{0.719467in}}%
\pgfpathlineto{\pgfqpoint{3.588666in}{0.709508in}}%
\pgfpathlineto{\pgfqpoint{3.600329in}{0.702606in}}%
\pgfpathlineto{\pgfqpoint{3.605511in}{0.699548in}}%
\pgfpathlineto{\pgfqpoint{3.622193in}{0.689589in}}%
\pgfpathlineto{\pgfqpoint{3.630208in}{0.684752in}}%
\pgfpathlineto{\pgfqpoint{3.638722in}{0.679629in}}%
\pgfpathlineto{\pgfqpoint{3.655096in}{0.669670in}}%
\pgfpathlineto{\pgfqpoint{3.660086in}{0.666596in}}%
\pgfpathlineto{\pgfqpoint{3.671298in}{0.659710in}}%
\pgfpathlineto{\pgfqpoint{3.687356in}{0.649751in}}%
\pgfpathlineto{\pgfqpoint{3.689965in}{0.648108in}}%
\pgfpathlineto{\pgfqpoint{3.703219in}{0.639791in}}%
\pgfpathlineto{\pgfqpoint{3.718950in}{0.629831in}}%
\pgfpathlineto{\pgfqpoint{3.719844in}{0.629256in}}%
\pgfpathlineto{\pgfqpoint{3.734463in}{0.619872in}}%
\pgfpathlineto{\pgfqpoint{3.749723in}{0.609995in}}%
\pgfpathlineto{\pgfqpoint{3.749851in}{0.609912in}}%
\pgfpathlineto{\pgfqpoint{3.765002in}{0.599953in}}%
\pgfpathlineto{\pgfqpoint{3.779601in}{0.590276in}}%
\pgfpathlineto{\pgfqpoint{3.780030in}{0.589993in}}%
\pgfpathlineto{\pgfqpoint{3.794806in}{0.580034in}}%
\pgfpathlineto{\pgfqpoint{3.809461in}{0.570074in}}%
\pgfpathlineto{\pgfqpoint{3.809480in}{0.570061in}}%
\pgfpathlineto{\pgfqpoint{3.823841in}{0.560115in}}%
\pgfpathlineto{\pgfqpoint{3.838095in}{0.550155in}}%
\pgfpathlineto{\pgfqpoint{3.839359in}{0.549252in}}%
\pgfpathlineto{\pgfqpoint{3.852070in}{0.540195in}}%
\pgfpathlineto{\pgfqpoint{3.865895in}{0.530236in}}%
\pgfpathlineto{\pgfqpoint{3.869237in}{0.527775in}}%
\pgfpathlineto{\pgfqpoint{3.879448in}{0.520276in}}%
\pgfpathlineto{\pgfqpoint{3.892812in}{0.510317in}}%
\pgfpathlineto{\pgfqpoint{3.899116in}{0.505524in}}%
\pgfpathlineto{\pgfqpoint{3.905929in}{0.500357in}}%
\pgfpathlineto{\pgfqpoint{3.918792in}{0.490398in}}%
\pgfpathlineto{\pgfqpoint{3.928995in}{0.482373in}}%
\pgfpathlineto{\pgfqpoint{3.931460in}{0.480438in}}%
\pgfpathlineto{\pgfqpoint{3.943779in}{0.470478in}}%
\pgfpathlineto{\pgfqpoint{3.955923in}{0.460519in}}%
\pgfpathlineto{\pgfqpoint{3.958873in}{0.458022in}}%
\pgfpathlineto{\pgfqpoint{3.967706in}{0.450559in}}%
\pgfpathlineto{\pgfqpoint{3.979225in}{0.440600in}}%
\pgfpathlineto{\pgfqpoint{3.988752in}{0.432184in}}%
\pgfpathlineto{\pgfqpoint{3.990502in}{0.430640in}}%
\pgfpathlineto{\pgfqpoint{4.001332in}{0.420681in}}%
\pgfpathlineto{\pgfqpoint{4.011918in}{0.410721in}}%
\pgfpathlineto{\pgfqpoint{4.018631in}{0.404191in}}%
\pgfpathlineto{\pgfqpoint{4.022156in}{0.400762in}}%
\pgfpathlineto{\pgfqpoint{4.031926in}{0.390802in}}%
\pgfpathlineto{\pgfqpoint{4.041388in}{0.380842in}}%
\pgfpathlineto{\pgfqpoint{4.048509in}{0.373028in}}%
\pgfpathlineto{\pgfqpoint{4.050462in}{0.370883in}}%
\pgfpathlineto{\pgfqpoint{4.058939in}{0.360923in}}%
\pgfpathlineto{\pgfqpoint{4.067009in}{0.350964in}}%
\pgfpathlineto{\pgfqpoint{4.074638in}{0.341004in}}%
\pgfpathlineto{\pgfqpoint{4.078388in}{0.335681in}}%
\pgfpathlineto{\pgfqpoint{4.081644in}{0.331045in}}%
\pgfpathlineto{\pgfqpoint{4.087943in}{0.321085in}}%
\pgfpathlineto{\pgfqpoint{4.093620in}{0.311126in}}%
\pgfpathlineto{\pgfqpoint{4.098596in}{0.301166in}}%
\pgfpathlineto{\pgfqpoint{4.102777in}{0.291206in}}%
\pgfpathlineto{\pgfqpoint{4.106039in}{0.281247in}}%
\pgfpathlineto{\pgfqpoint{4.108230in}{0.271287in}}%
\pgfpathlineto{\pgfqpoint{4.108267in}{0.270872in}}%
\pgfpathlineto{\pgfqpoint{4.109092in}{0.261328in}}%
\pgfpathlineto{\pgfqpoint{4.108533in}{0.251368in}}%
\pgfpathlineto{\pgfqpoint{4.108267in}{0.250123in}}%
\pgfpathlineto{\pgfqpoint{4.106103in}{0.241409in}}%
\pgfpathlineto{\pgfqpoint{4.101361in}{0.231449in}}%
\pgfpathlineto{\pgfqpoint{4.093739in}{0.221489in}}%
\pgfpathlineto{\pgfqpoint{4.082426in}{0.211530in}}%
\pgfpathlineto{\pgfqpoint{4.078388in}{0.208846in}}%
\pgfpathlineto{\pgfqpoint{4.064288in}{0.201570in}}%
\pgfpathlineto{\pgfqpoint{4.048509in}{0.195478in}}%
\pgfpathlineto{\pgfqpoint{4.034320in}{0.191611in}}%
\pgfpathlineto{\pgfqpoint{4.018631in}{0.188182in}}%
\pgfpathlineto{\pgfqpoint{3.988752in}{0.184110in}}%
\pgfpathlineto{\pgfqpoint{3.958873in}{0.182113in}}%
\pgfpathlineto{\pgfqpoint{3.928995in}{0.181668in}}%
\pgfpathlineto{\pgfqpoint{3.899116in}{0.182418in}}%
\pgfpathlineto{\pgfqpoint{3.869237in}{0.184112in}}%
\pgfpathlineto{\pgfqpoint{3.839359in}{0.186570in}}%
\pgfpathlineto{\pgfqpoint{3.809480in}{0.189659in}}%
\pgfpathclose%
\pgfusepath{}%
\end{pgfscope}%
\begin{pgfscope}%
\pgfsetbuttcap%
\pgfsetroundjoin%
\definecolor{currentfill}{rgb}{0.000000,0.000000,0.000000}%
\pgfsetfillcolor{currentfill}%
\pgfsetlinewidth{0.803000pt}%
\definecolor{currentstroke}{rgb}{0.000000,0.000000,0.000000}%
\pgfsetstrokecolor{currentstroke}%
\pgfsetdash{}{0pt}%
\pgfsys@defobject{currentmarker}{\pgfqpoint{0.000000in}{-0.048611in}}{\pgfqpoint{0.000000in}{0.000000in}}{%
\pgfpathmoveto{\pgfqpoint{0.000000in}{0.000000in}}%
\pgfpathlineto{\pgfqpoint{0.000000in}{-0.048611in}}%
\pgfusepath{stroke,fill}%
}%
\begin{pgfscope}%
\pgfsys@transformshift{0.224038in}{0.923638in}%
\pgfsys@useobject{currentmarker}{}%
\end{pgfscope}%
\end{pgfscope}%
\begin{pgfscope}%
\pgftext[x=0.224038in,y=0.826416in,,top]{\sffamily\fontsize{10.000000}{12.000000}\selectfont -3}%
\end{pgfscope}%
\begin{pgfscope}%
\pgfsetbuttcap%
\pgfsetroundjoin%
\definecolor{currentfill}{rgb}{0.000000,0.000000,0.000000}%
\pgfsetfillcolor{currentfill}%
\pgfsetlinewidth{0.803000pt}%
\definecolor{currentstroke}{rgb}{0.000000,0.000000,0.000000}%
\pgfsetstrokecolor{currentstroke}%
\pgfsetdash{}{0pt}%
\pgfsys@defobject{currentmarker}{\pgfqpoint{0.000000in}{-0.048611in}}{\pgfqpoint{0.000000in}{0.000000in}}{%
\pgfpathmoveto{\pgfqpoint{0.000000in}{0.000000in}}%
\pgfpathlineto{\pgfqpoint{0.000000in}{-0.048611in}}%
\pgfusepath{stroke,fill}%
}%
\begin{pgfscope}%
\pgfsys@transformshift{0.966026in}{0.923638in}%
\pgfsys@useobject{currentmarker}{}%
\end{pgfscope}%
\end{pgfscope}%
\begin{pgfscope}%
\pgftext[x=0.966026in,y=0.826416in,,top]{\sffamily\fontsize{10.000000}{12.000000}\selectfont -2}%
\end{pgfscope}%
\begin{pgfscope}%
\pgfsetbuttcap%
\pgfsetroundjoin%
\definecolor{currentfill}{rgb}{0.000000,0.000000,0.000000}%
\pgfsetfillcolor{currentfill}%
\pgfsetlinewidth{0.803000pt}%
\definecolor{currentstroke}{rgb}{0.000000,0.000000,0.000000}%
\pgfsetstrokecolor{currentstroke}%
\pgfsetdash{}{0pt}%
\pgfsys@defobject{currentmarker}{\pgfqpoint{0.000000in}{-0.048611in}}{\pgfqpoint{0.000000in}{0.000000in}}{%
\pgfpathmoveto{\pgfqpoint{0.000000in}{0.000000in}}%
\pgfpathlineto{\pgfqpoint{0.000000in}{-0.048611in}}%
\pgfusepath{stroke,fill}%
}%
\begin{pgfscope}%
\pgfsys@transformshift{1.708013in}{0.923638in}%
\pgfsys@useobject{currentmarker}{}%
\end{pgfscope}%
\end{pgfscope}%
\begin{pgfscope}%
\pgftext[x=1.708013in,y=0.826416in,,top]{\sffamily\fontsize{10.000000}{12.000000}\selectfont -1}%
\end{pgfscope}%
\begin{pgfscope}%
\pgfsetbuttcap%
\pgfsetroundjoin%
\definecolor{currentfill}{rgb}{0.000000,0.000000,0.000000}%
\pgfsetfillcolor{currentfill}%
\pgfsetlinewidth{0.803000pt}%
\definecolor{currentstroke}{rgb}{0.000000,0.000000,0.000000}%
\pgfsetstrokecolor{currentstroke}%
\pgfsetdash{}{0pt}%
\pgfsys@defobject{currentmarker}{\pgfqpoint{0.000000in}{-0.048611in}}{\pgfqpoint{0.000000in}{0.000000in}}{%
\pgfpathmoveto{\pgfqpoint{0.000000in}{0.000000in}}%
\pgfpathlineto{\pgfqpoint{0.000000in}{-0.048611in}}%
\pgfusepath{stroke,fill}%
}%
\begin{pgfscope}%
\pgfsys@transformshift{2.450000in}{0.923638in}%
\pgfsys@useobject{currentmarker}{}%
\end{pgfscope}%
\end{pgfscope}%
\begin{pgfscope}%
\pgfsetbuttcap%
\pgfsetroundjoin%
\definecolor{currentfill}{rgb}{0.000000,0.000000,0.000000}%
\pgfsetfillcolor{currentfill}%
\pgfsetlinewidth{0.803000pt}%
\definecolor{currentstroke}{rgb}{0.000000,0.000000,0.000000}%
\pgfsetstrokecolor{currentstroke}%
\pgfsetdash{}{0pt}%
\pgfsys@defobject{currentmarker}{\pgfqpoint{0.000000in}{-0.048611in}}{\pgfqpoint{0.000000in}{0.000000in}}{%
\pgfpathmoveto{\pgfqpoint{0.000000in}{0.000000in}}%
\pgfpathlineto{\pgfqpoint{0.000000in}{-0.048611in}}%
\pgfusepath{stroke,fill}%
}%
\begin{pgfscope}%
\pgfsys@transformshift{3.191987in}{0.923638in}%
\pgfsys@useobject{currentmarker}{}%
\end{pgfscope}%
\end{pgfscope}%
\begin{pgfscope}%
\pgftext[x=3.191987in,y=0.826416in,,top]{\sffamily\fontsize{10.000000}{12.000000}\selectfont 1}%
\end{pgfscope}%
\begin{pgfscope}%
\pgfsetbuttcap%
\pgfsetroundjoin%
\definecolor{currentfill}{rgb}{0.000000,0.000000,0.000000}%
\pgfsetfillcolor{currentfill}%
\pgfsetlinewidth{0.803000pt}%
\definecolor{currentstroke}{rgb}{0.000000,0.000000,0.000000}%
\pgfsetstrokecolor{currentstroke}%
\pgfsetdash{}{0pt}%
\pgfsys@defobject{currentmarker}{\pgfqpoint{0.000000in}{-0.048611in}}{\pgfqpoint{0.000000in}{0.000000in}}{%
\pgfpathmoveto{\pgfqpoint{0.000000in}{0.000000in}}%
\pgfpathlineto{\pgfqpoint{0.000000in}{-0.048611in}}%
\pgfusepath{stroke,fill}%
}%
\begin{pgfscope}%
\pgfsys@transformshift{3.933974in}{0.923638in}%
\pgfsys@useobject{currentmarker}{}%
\end{pgfscope}%
\end{pgfscope}%
\begin{pgfscope}%
\pgftext[x=3.933974in,y=0.826416in,,top]{\sffamily\fontsize{10.000000}{12.000000}\selectfont 2}%
\end{pgfscope}%
\begin{pgfscope}%
\pgfsetbuttcap%
\pgfsetroundjoin%
\definecolor{currentfill}{rgb}{0.000000,0.000000,0.000000}%
\pgfsetfillcolor{currentfill}%
\pgfsetlinewidth{0.803000pt}%
\definecolor{currentstroke}{rgb}{0.000000,0.000000,0.000000}%
\pgfsetstrokecolor{currentstroke}%
\pgfsetdash{}{0pt}%
\pgfsys@defobject{currentmarker}{\pgfqpoint{0.000000in}{-0.048611in}}{\pgfqpoint{0.000000in}{0.000000in}}{%
\pgfpathmoveto{\pgfqpoint{0.000000in}{0.000000in}}%
\pgfpathlineto{\pgfqpoint{0.000000in}{-0.048611in}}%
\pgfusepath{stroke,fill}%
}%
\begin{pgfscope}%
\pgfsys@transformshift{4.675962in}{0.923638in}%
\pgfsys@useobject{currentmarker}{}%
\end{pgfscope}%
\end{pgfscope}%
\begin{pgfscope}%
\pgftext[x=4.675962in,y=0.826416in,,top]{\sffamily\fontsize{10.000000}{12.000000}\selectfont 3}%
\end{pgfscope}%
\begin{pgfscope}%
\pgfsetbuttcap%
\pgfsetroundjoin%
\definecolor{currentfill}{rgb}{0.000000,0.000000,0.000000}%
\pgfsetfillcolor{currentfill}%
\pgfsetlinewidth{0.602250pt}%
\definecolor{currentstroke}{rgb}{0.000000,0.000000,0.000000}%
\pgfsetstrokecolor{currentstroke}%
\pgfsetdash{}{0pt}%
\pgfsys@defobject{currentmarker}{\pgfqpoint{0.000000in}{-0.027778in}}{\pgfqpoint{0.000000in}{0.000000in}}{%
\pgfpathmoveto{\pgfqpoint{0.000000in}{0.000000in}}%
\pgfpathlineto{\pgfqpoint{0.000000in}{-0.027778in}}%
\pgfusepath{stroke,fill}%
}%
\begin{pgfscope}%
\pgfsys@transformshift{0.372436in}{0.923638in}%
\pgfsys@useobject{currentmarker}{}%
\end{pgfscope}%
\end{pgfscope}%
\begin{pgfscope}%
\pgfsetbuttcap%
\pgfsetroundjoin%
\definecolor{currentfill}{rgb}{0.000000,0.000000,0.000000}%
\pgfsetfillcolor{currentfill}%
\pgfsetlinewidth{0.602250pt}%
\definecolor{currentstroke}{rgb}{0.000000,0.000000,0.000000}%
\pgfsetstrokecolor{currentstroke}%
\pgfsetdash{}{0pt}%
\pgfsys@defobject{currentmarker}{\pgfqpoint{0.000000in}{-0.027778in}}{\pgfqpoint{0.000000in}{0.000000in}}{%
\pgfpathmoveto{\pgfqpoint{0.000000in}{0.000000in}}%
\pgfpathlineto{\pgfqpoint{0.000000in}{-0.027778in}}%
\pgfusepath{stroke,fill}%
}%
\begin{pgfscope}%
\pgfsys@transformshift{0.520833in}{0.923638in}%
\pgfsys@useobject{currentmarker}{}%
\end{pgfscope}%
\end{pgfscope}%
\begin{pgfscope}%
\pgfsetbuttcap%
\pgfsetroundjoin%
\definecolor{currentfill}{rgb}{0.000000,0.000000,0.000000}%
\pgfsetfillcolor{currentfill}%
\pgfsetlinewidth{0.602250pt}%
\definecolor{currentstroke}{rgb}{0.000000,0.000000,0.000000}%
\pgfsetstrokecolor{currentstroke}%
\pgfsetdash{}{0pt}%
\pgfsys@defobject{currentmarker}{\pgfqpoint{0.000000in}{-0.027778in}}{\pgfqpoint{0.000000in}{0.000000in}}{%
\pgfpathmoveto{\pgfqpoint{0.000000in}{0.000000in}}%
\pgfpathlineto{\pgfqpoint{0.000000in}{-0.027778in}}%
\pgfusepath{stroke,fill}%
}%
\begin{pgfscope}%
\pgfsys@transformshift{0.669231in}{0.923638in}%
\pgfsys@useobject{currentmarker}{}%
\end{pgfscope}%
\end{pgfscope}%
\begin{pgfscope}%
\pgfsetbuttcap%
\pgfsetroundjoin%
\definecolor{currentfill}{rgb}{0.000000,0.000000,0.000000}%
\pgfsetfillcolor{currentfill}%
\pgfsetlinewidth{0.602250pt}%
\definecolor{currentstroke}{rgb}{0.000000,0.000000,0.000000}%
\pgfsetstrokecolor{currentstroke}%
\pgfsetdash{}{0pt}%
\pgfsys@defobject{currentmarker}{\pgfqpoint{0.000000in}{-0.027778in}}{\pgfqpoint{0.000000in}{0.000000in}}{%
\pgfpathmoveto{\pgfqpoint{0.000000in}{0.000000in}}%
\pgfpathlineto{\pgfqpoint{0.000000in}{-0.027778in}}%
\pgfusepath{stroke,fill}%
}%
\begin{pgfscope}%
\pgfsys@transformshift{0.817628in}{0.923638in}%
\pgfsys@useobject{currentmarker}{}%
\end{pgfscope}%
\end{pgfscope}%
\begin{pgfscope}%
\pgfsetbuttcap%
\pgfsetroundjoin%
\definecolor{currentfill}{rgb}{0.000000,0.000000,0.000000}%
\pgfsetfillcolor{currentfill}%
\pgfsetlinewidth{0.602250pt}%
\definecolor{currentstroke}{rgb}{0.000000,0.000000,0.000000}%
\pgfsetstrokecolor{currentstroke}%
\pgfsetdash{}{0pt}%
\pgfsys@defobject{currentmarker}{\pgfqpoint{0.000000in}{-0.027778in}}{\pgfqpoint{0.000000in}{0.000000in}}{%
\pgfpathmoveto{\pgfqpoint{0.000000in}{0.000000in}}%
\pgfpathlineto{\pgfqpoint{0.000000in}{-0.027778in}}%
\pgfusepath{stroke,fill}%
}%
\begin{pgfscope}%
\pgfsys@transformshift{1.114423in}{0.923638in}%
\pgfsys@useobject{currentmarker}{}%
\end{pgfscope}%
\end{pgfscope}%
\begin{pgfscope}%
\pgfsetbuttcap%
\pgfsetroundjoin%
\definecolor{currentfill}{rgb}{0.000000,0.000000,0.000000}%
\pgfsetfillcolor{currentfill}%
\pgfsetlinewidth{0.602250pt}%
\definecolor{currentstroke}{rgb}{0.000000,0.000000,0.000000}%
\pgfsetstrokecolor{currentstroke}%
\pgfsetdash{}{0pt}%
\pgfsys@defobject{currentmarker}{\pgfqpoint{0.000000in}{-0.027778in}}{\pgfqpoint{0.000000in}{0.000000in}}{%
\pgfpathmoveto{\pgfqpoint{0.000000in}{0.000000in}}%
\pgfpathlineto{\pgfqpoint{0.000000in}{-0.027778in}}%
\pgfusepath{stroke,fill}%
}%
\begin{pgfscope}%
\pgfsys@transformshift{1.262821in}{0.923638in}%
\pgfsys@useobject{currentmarker}{}%
\end{pgfscope}%
\end{pgfscope}%
\begin{pgfscope}%
\pgfsetbuttcap%
\pgfsetroundjoin%
\definecolor{currentfill}{rgb}{0.000000,0.000000,0.000000}%
\pgfsetfillcolor{currentfill}%
\pgfsetlinewidth{0.602250pt}%
\definecolor{currentstroke}{rgb}{0.000000,0.000000,0.000000}%
\pgfsetstrokecolor{currentstroke}%
\pgfsetdash{}{0pt}%
\pgfsys@defobject{currentmarker}{\pgfqpoint{0.000000in}{-0.027778in}}{\pgfqpoint{0.000000in}{0.000000in}}{%
\pgfpathmoveto{\pgfqpoint{0.000000in}{0.000000in}}%
\pgfpathlineto{\pgfqpoint{0.000000in}{-0.027778in}}%
\pgfusepath{stroke,fill}%
}%
\begin{pgfscope}%
\pgfsys@transformshift{1.411218in}{0.923638in}%
\pgfsys@useobject{currentmarker}{}%
\end{pgfscope}%
\end{pgfscope}%
\begin{pgfscope}%
\pgfsetbuttcap%
\pgfsetroundjoin%
\definecolor{currentfill}{rgb}{0.000000,0.000000,0.000000}%
\pgfsetfillcolor{currentfill}%
\pgfsetlinewidth{0.602250pt}%
\definecolor{currentstroke}{rgb}{0.000000,0.000000,0.000000}%
\pgfsetstrokecolor{currentstroke}%
\pgfsetdash{}{0pt}%
\pgfsys@defobject{currentmarker}{\pgfqpoint{0.000000in}{-0.027778in}}{\pgfqpoint{0.000000in}{0.000000in}}{%
\pgfpathmoveto{\pgfqpoint{0.000000in}{0.000000in}}%
\pgfpathlineto{\pgfqpoint{0.000000in}{-0.027778in}}%
\pgfusepath{stroke,fill}%
}%
\begin{pgfscope}%
\pgfsys@transformshift{1.559615in}{0.923638in}%
\pgfsys@useobject{currentmarker}{}%
\end{pgfscope}%
\end{pgfscope}%
\begin{pgfscope}%
\pgfsetbuttcap%
\pgfsetroundjoin%
\definecolor{currentfill}{rgb}{0.000000,0.000000,0.000000}%
\pgfsetfillcolor{currentfill}%
\pgfsetlinewidth{0.602250pt}%
\definecolor{currentstroke}{rgb}{0.000000,0.000000,0.000000}%
\pgfsetstrokecolor{currentstroke}%
\pgfsetdash{}{0pt}%
\pgfsys@defobject{currentmarker}{\pgfqpoint{0.000000in}{-0.027778in}}{\pgfqpoint{0.000000in}{0.000000in}}{%
\pgfpathmoveto{\pgfqpoint{0.000000in}{0.000000in}}%
\pgfpathlineto{\pgfqpoint{0.000000in}{-0.027778in}}%
\pgfusepath{stroke,fill}%
}%
\begin{pgfscope}%
\pgfsys@transformshift{1.708013in}{0.923638in}%
\pgfsys@useobject{currentmarker}{}%
\end{pgfscope}%
\end{pgfscope}%
\begin{pgfscope}%
\pgfsetbuttcap%
\pgfsetroundjoin%
\definecolor{currentfill}{rgb}{0.000000,0.000000,0.000000}%
\pgfsetfillcolor{currentfill}%
\pgfsetlinewidth{0.602250pt}%
\definecolor{currentstroke}{rgb}{0.000000,0.000000,0.000000}%
\pgfsetstrokecolor{currentstroke}%
\pgfsetdash{}{0pt}%
\pgfsys@defobject{currentmarker}{\pgfqpoint{0.000000in}{-0.027778in}}{\pgfqpoint{0.000000in}{0.000000in}}{%
\pgfpathmoveto{\pgfqpoint{0.000000in}{0.000000in}}%
\pgfpathlineto{\pgfqpoint{0.000000in}{-0.027778in}}%
\pgfusepath{stroke,fill}%
}%
\begin{pgfscope}%
\pgfsys@transformshift{1.856410in}{0.923638in}%
\pgfsys@useobject{currentmarker}{}%
\end{pgfscope}%
\end{pgfscope}%
\begin{pgfscope}%
\pgfsetbuttcap%
\pgfsetroundjoin%
\definecolor{currentfill}{rgb}{0.000000,0.000000,0.000000}%
\pgfsetfillcolor{currentfill}%
\pgfsetlinewidth{0.602250pt}%
\definecolor{currentstroke}{rgb}{0.000000,0.000000,0.000000}%
\pgfsetstrokecolor{currentstroke}%
\pgfsetdash{}{0pt}%
\pgfsys@defobject{currentmarker}{\pgfqpoint{0.000000in}{-0.027778in}}{\pgfqpoint{0.000000in}{0.000000in}}{%
\pgfpathmoveto{\pgfqpoint{0.000000in}{0.000000in}}%
\pgfpathlineto{\pgfqpoint{0.000000in}{-0.027778in}}%
\pgfusepath{stroke,fill}%
}%
\begin{pgfscope}%
\pgfsys@transformshift{2.004808in}{0.923638in}%
\pgfsys@useobject{currentmarker}{}%
\end{pgfscope}%
\end{pgfscope}%
\begin{pgfscope}%
\pgfsetbuttcap%
\pgfsetroundjoin%
\definecolor{currentfill}{rgb}{0.000000,0.000000,0.000000}%
\pgfsetfillcolor{currentfill}%
\pgfsetlinewidth{0.602250pt}%
\definecolor{currentstroke}{rgb}{0.000000,0.000000,0.000000}%
\pgfsetstrokecolor{currentstroke}%
\pgfsetdash{}{0pt}%
\pgfsys@defobject{currentmarker}{\pgfqpoint{0.000000in}{-0.027778in}}{\pgfqpoint{0.000000in}{0.000000in}}{%
\pgfpathmoveto{\pgfqpoint{0.000000in}{0.000000in}}%
\pgfpathlineto{\pgfqpoint{0.000000in}{-0.027778in}}%
\pgfusepath{stroke,fill}%
}%
\begin{pgfscope}%
\pgfsys@transformshift{2.153205in}{0.923638in}%
\pgfsys@useobject{currentmarker}{}%
\end{pgfscope}%
\end{pgfscope}%
\begin{pgfscope}%
\pgfsetbuttcap%
\pgfsetroundjoin%
\definecolor{currentfill}{rgb}{0.000000,0.000000,0.000000}%
\pgfsetfillcolor{currentfill}%
\pgfsetlinewidth{0.602250pt}%
\definecolor{currentstroke}{rgb}{0.000000,0.000000,0.000000}%
\pgfsetstrokecolor{currentstroke}%
\pgfsetdash{}{0pt}%
\pgfsys@defobject{currentmarker}{\pgfqpoint{0.000000in}{-0.027778in}}{\pgfqpoint{0.000000in}{0.000000in}}{%
\pgfpathmoveto{\pgfqpoint{0.000000in}{0.000000in}}%
\pgfpathlineto{\pgfqpoint{0.000000in}{-0.027778in}}%
\pgfusepath{stroke,fill}%
}%
\begin{pgfscope}%
\pgfsys@transformshift{2.301603in}{0.923638in}%
\pgfsys@useobject{currentmarker}{}%
\end{pgfscope}%
\end{pgfscope}%
\begin{pgfscope}%
\pgfsetbuttcap%
\pgfsetroundjoin%
\definecolor{currentfill}{rgb}{0.000000,0.000000,0.000000}%
\pgfsetfillcolor{currentfill}%
\pgfsetlinewidth{0.602250pt}%
\definecolor{currentstroke}{rgb}{0.000000,0.000000,0.000000}%
\pgfsetstrokecolor{currentstroke}%
\pgfsetdash{}{0pt}%
\pgfsys@defobject{currentmarker}{\pgfqpoint{0.000000in}{-0.027778in}}{\pgfqpoint{0.000000in}{0.000000in}}{%
\pgfpathmoveto{\pgfqpoint{0.000000in}{0.000000in}}%
\pgfpathlineto{\pgfqpoint{0.000000in}{-0.027778in}}%
\pgfusepath{stroke,fill}%
}%
\begin{pgfscope}%
\pgfsys@transformshift{2.450000in}{0.923638in}%
\pgfsys@useobject{currentmarker}{}%
\end{pgfscope}%
\end{pgfscope}%
\begin{pgfscope}%
\pgfsetbuttcap%
\pgfsetroundjoin%
\definecolor{currentfill}{rgb}{0.000000,0.000000,0.000000}%
\pgfsetfillcolor{currentfill}%
\pgfsetlinewidth{0.602250pt}%
\definecolor{currentstroke}{rgb}{0.000000,0.000000,0.000000}%
\pgfsetstrokecolor{currentstroke}%
\pgfsetdash{}{0pt}%
\pgfsys@defobject{currentmarker}{\pgfqpoint{0.000000in}{-0.027778in}}{\pgfqpoint{0.000000in}{0.000000in}}{%
\pgfpathmoveto{\pgfqpoint{0.000000in}{0.000000in}}%
\pgfpathlineto{\pgfqpoint{0.000000in}{-0.027778in}}%
\pgfusepath{stroke,fill}%
}%
\begin{pgfscope}%
\pgfsys@transformshift{2.598397in}{0.923638in}%
\pgfsys@useobject{currentmarker}{}%
\end{pgfscope}%
\end{pgfscope}%
\begin{pgfscope}%
\pgfsetbuttcap%
\pgfsetroundjoin%
\definecolor{currentfill}{rgb}{0.000000,0.000000,0.000000}%
\pgfsetfillcolor{currentfill}%
\pgfsetlinewidth{0.602250pt}%
\definecolor{currentstroke}{rgb}{0.000000,0.000000,0.000000}%
\pgfsetstrokecolor{currentstroke}%
\pgfsetdash{}{0pt}%
\pgfsys@defobject{currentmarker}{\pgfqpoint{0.000000in}{-0.027778in}}{\pgfqpoint{0.000000in}{0.000000in}}{%
\pgfpathmoveto{\pgfqpoint{0.000000in}{0.000000in}}%
\pgfpathlineto{\pgfqpoint{0.000000in}{-0.027778in}}%
\pgfusepath{stroke,fill}%
}%
\begin{pgfscope}%
\pgfsys@transformshift{2.746795in}{0.923638in}%
\pgfsys@useobject{currentmarker}{}%
\end{pgfscope}%
\end{pgfscope}%
\begin{pgfscope}%
\pgfsetbuttcap%
\pgfsetroundjoin%
\definecolor{currentfill}{rgb}{0.000000,0.000000,0.000000}%
\pgfsetfillcolor{currentfill}%
\pgfsetlinewidth{0.602250pt}%
\definecolor{currentstroke}{rgb}{0.000000,0.000000,0.000000}%
\pgfsetstrokecolor{currentstroke}%
\pgfsetdash{}{0pt}%
\pgfsys@defobject{currentmarker}{\pgfqpoint{0.000000in}{-0.027778in}}{\pgfqpoint{0.000000in}{0.000000in}}{%
\pgfpathmoveto{\pgfqpoint{0.000000in}{0.000000in}}%
\pgfpathlineto{\pgfqpoint{0.000000in}{-0.027778in}}%
\pgfusepath{stroke,fill}%
}%
\begin{pgfscope}%
\pgfsys@transformshift{2.895192in}{0.923638in}%
\pgfsys@useobject{currentmarker}{}%
\end{pgfscope}%
\end{pgfscope}%
\begin{pgfscope}%
\pgfsetbuttcap%
\pgfsetroundjoin%
\definecolor{currentfill}{rgb}{0.000000,0.000000,0.000000}%
\pgfsetfillcolor{currentfill}%
\pgfsetlinewidth{0.602250pt}%
\definecolor{currentstroke}{rgb}{0.000000,0.000000,0.000000}%
\pgfsetstrokecolor{currentstroke}%
\pgfsetdash{}{0pt}%
\pgfsys@defobject{currentmarker}{\pgfqpoint{0.000000in}{-0.027778in}}{\pgfqpoint{0.000000in}{0.000000in}}{%
\pgfpathmoveto{\pgfqpoint{0.000000in}{0.000000in}}%
\pgfpathlineto{\pgfqpoint{0.000000in}{-0.027778in}}%
\pgfusepath{stroke,fill}%
}%
\begin{pgfscope}%
\pgfsys@transformshift{3.043590in}{0.923638in}%
\pgfsys@useobject{currentmarker}{}%
\end{pgfscope}%
\end{pgfscope}%
\begin{pgfscope}%
\pgfsetbuttcap%
\pgfsetroundjoin%
\definecolor{currentfill}{rgb}{0.000000,0.000000,0.000000}%
\pgfsetfillcolor{currentfill}%
\pgfsetlinewidth{0.602250pt}%
\definecolor{currentstroke}{rgb}{0.000000,0.000000,0.000000}%
\pgfsetstrokecolor{currentstroke}%
\pgfsetdash{}{0pt}%
\pgfsys@defobject{currentmarker}{\pgfqpoint{0.000000in}{-0.027778in}}{\pgfqpoint{0.000000in}{0.000000in}}{%
\pgfpathmoveto{\pgfqpoint{0.000000in}{0.000000in}}%
\pgfpathlineto{\pgfqpoint{0.000000in}{-0.027778in}}%
\pgfusepath{stroke,fill}%
}%
\begin{pgfscope}%
\pgfsys@transformshift{3.191987in}{0.923638in}%
\pgfsys@useobject{currentmarker}{}%
\end{pgfscope}%
\end{pgfscope}%
\begin{pgfscope}%
\pgfsetbuttcap%
\pgfsetroundjoin%
\definecolor{currentfill}{rgb}{0.000000,0.000000,0.000000}%
\pgfsetfillcolor{currentfill}%
\pgfsetlinewidth{0.602250pt}%
\definecolor{currentstroke}{rgb}{0.000000,0.000000,0.000000}%
\pgfsetstrokecolor{currentstroke}%
\pgfsetdash{}{0pt}%
\pgfsys@defobject{currentmarker}{\pgfqpoint{0.000000in}{-0.027778in}}{\pgfqpoint{0.000000in}{0.000000in}}{%
\pgfpathmoveto{\pgfqpoint{0.000000in}{0.000000in}}%
\pgfpathlineto{\pgfqpoint{0.000000in}{-0.027778in}}%
\pgfusepath{stroke,fill}%
}%
\begin{pgfscope}%
\pgfsys@transformshift{3.340385in}{0.923638in}%
\pgfsys@useobject{currentmarker}{}%
\end{pgfscope}%
\end{pgfscope}%
\begin{pgfscope}%
\pgfsetbuttcap%
\pgfsetroundjoin%
\definecolor{currentfill}{rgb}{0.000000,0.000000,0.000000}%
\pgfsetfillcolor{currentfill}%
\pgfsetlinewidth{0.602250pt}%
\definecolor{currentstroke}{rgb}{0.000000,0.000000,0.000000}%
\pgfsetstrokecolor{currentstroke}%
\pgfsetdash{}{0pt}%
\pgfsys@defobject{currentmarker}{\pgfqpoint{0.000000in}{-0.027778in}}{\pgfqpoint{0.000000in}{0.000000in}}{%
\pgfpathmoveto{\pgfqpoint{0.000000in}{0.000000in}}%
\pgfpathlineto{\pgfqpoint{0.000000in}{-0.027778in}}%
\pgfusepath{stroke,fill}%
}%
\begin{pgfscope}%
\pgfsys@transformshift{3.488782in}{0.923638in}%
\pgfsys@useobject{currentmarker}{}%
\end{pgfscope}%
\end{pgfscope}%
\begin{pgfscope}%
\pgfsetbuttcap%
\pgfsetroundjoin%
\definecolor{currentfill}{rgb}{0.000000,0.000000,0.000000}%
\pgfsetfillcolor{currentfill}%
\pgfsetlinewidth{0.602250pt}%
\definecolor{currentstroke}{rgb}{0.000000,0.000000,0.000000}%
\pgfsetstrokecolor{currentstroke}%
\pgfsetdash{}{0pt}%
\pgfsys@defobject{currentmarker}{\pgfqpoint{0.000000in}{-0.027778in}}{\pgfqpoint{0.000000in}{0.000000in}}{%
\pgfpathmoveto{\pgfqpoint{0.000000in}{0.000000in}}%
\pgfpathlineto{\pgfqpoint{0.000000in}{-0.027778in}}%
\pgfusepath{stroke,fill}%
}%
\begin{pgfscope}%
\pgfsys@transformshift{3.637179in}{0.923638in}%
\pgfsys@useobject{currentmarker}{}%
\end{pgfscope}%
\end{pgfscope}%
\begin{pgfscope}%
\pgfsetbuttcap%
\pgfsetroundjoin%
\definecolor{currentfill}{rgb}{0.000000,0.000000,0.000000}%
\pgfsetfillcolor{currentfill}%
\pgfsetlinewidth{0.602250pt}%
\definecolor{currentstroke}{rgb}{0.000000,0.000000,0.000000}%
\pgfsetstrokecolor{currentstroke}%
\pgfsetdash{}{0pt}%
\pgfsys@defobject{currentmarker}{\pgfqpoint{0.000000in}{-0.027778in}}{\pgfqpoint{0.000000in}{0.000000in}}{%
\pgfpathmoveto{\pgfqpoint{0.000000in}{0.000000in}}%
\pgfpathlineto{\pgfqpoint{0.000000in}{-0.027778in}}%
\pgfusepath{stroke,fill}%
}%
\begin{pgfscope}%
\pgfsys@transformshift{3.785577in}{0.923638in}%
\pgfsys@useobject{currentmarker}{}%
\end{pgfscope}%
\end{pgfscope}%
\begin{pgfscope}%
\pgfsetbuttcap%
\pgfsetroundjoin%
\definecolor{currentfill}{rgb}{0.000000,0.000000,0.000000}%
\pgfsetfillcolor{currentfill}%
\pgfsetlinewidth{0.602250pt}%
\definecolor{currentstroke}{rgb}{0.000000,0.000000,0.000000}%
\pgfsetstrokecolor{currentstroke}%
\pgfsetdash{}{0pt}%
\pgfsys@defobject{currentmarker}{\pgfqpoint{0.000000in}{-0.027778in}}{\pgfqpoint{0.000000in}{0.000000in}}{%
\pgfpathmoveto{\pgfqpoint{0.000000in}{0.000000in}}%
\pgfpathlineto{\pgfqpoint{0.000000in}{-0.027778in}}%
\pgfusepath{stroke,fill}%
}%
\begin{pgfscope}%
\pgfsys@transformshift{3.933974in}{0.923638in}%
\pgfsys@useobject{currentmarker}{}%
\end{pgfscope}%
\end{pgfscope}%
\begin{pgfscope}%
\pgfsetbuttcap%
\pgfsetroundjoin%
\definecolor{currentfill}{rgb}{0.000000,0.000000,0.000000}%
\pgfsetfillcolor{currentfill}%
\pgfsetlinewidth{0.602250pt}%
\definecolor{currentstroke}{rgb}{0.000000,0.000000,0.000000}%
\pgfsetstrokecolor{currentstroke}%
\pgfsetdash{}{0pt}%
\pgfsys@defobject{currentmarker}{\pgfqpoint{0.000000in}{-0.027778in}}{\pgfqpoint{0.000000in}{0.000000in}}{%
\pgfpathmoveto{\pgfqpoint{0.000000in}{0.000000in}}%
\pgfpathlineto{\pgfqpoint{0.000000in}{-0.027778in}}%
\pgfusepath{stroke,fill}%
}%
\begin{pgfscope}%
\pgfsys@transformshift{4.082372in}{0.923638in}%
\pgfsys@useobject{currentmarker}{}%
\end{pgfscope}%
\end{pgfscope}%
\begin{pgfscope}%
\pgfsetbuttcap%
\pgfsetroundjoin%
\definecolor{currentfill}{rgb}{0.000000,0.000000,0.000000}%
\pgfsetfillcolor{currentfill}%
\pgfsetlinewidth{0.602250pt}%
\definecolor{currentstroke}{rgb}{0.000000,0.000000,0.000000}%
\pgfsetstrokecolor{currentstroke}%
\pgfsetdash{}{0pt}%
\pgfsys@defobject{currentmarker}{\pgfqpoint{0.000000in}{-0.027778in}}{\pgfqpoint{0.000000in}{0.000000in}}{%
\pgfpathmoveto{\pgfqpoint{0.000000in}{0.000000in}}%
\pgfpathlineto{\pgfqpoint{0.000000in}{-0.027778in}}%
\pgfusepath{stroke,fill}%
}%
\begin{pgfscope}%
\pgfsys@transformshift{4.230769in}{0.923638in}%
\pgfsys@useobject{currentmarker}{}%
\end{pgfscope}%
\end{pgfscope}%
\begin{pgfscope}%
\pgfsetbuttcap%
\pgfsetroundjoin%
\definecolor{currentfill}{rgb}{0.000000,0.000000,0.000000}%
\pgfsetfillcolor{currentfill}%
\pgfsetlinewidth{0.602250pt}%
\definecolor{currentstroke}{rgb}{0.000000,0.000000,0.000000}%
\pgfsetstrokecolor{currentstroke}%
\pgfsetdash{}{0pt}%
\pgfsys@defobject{currentmarker}{\pgfqpoint{0.000000in}{-0.027778in}}{\pgfqpoint{0.000000in}{0.000000in}}{%
\pgfpathmoveto{\pgfqpoint{0.000000in}{0.000000in}}%
\pgfpathlineto{\pgfqpoint{0.000000in}{-0.027778in}}%
\pgfusepath{stroke,fill}%
}%
\begin{pgfscope}%
\pgfsys@transformshift{4.379167in}{0.923638in}%
\pgfsys@useobject{currentmarker}{}%
\end{pgfscope}%
\end{pgfscope}%
\begin{pgfscope}%
\pgfsetbuttcap%
\pgfsetroundjoin%
\definecolor{currentfill}{rgb}{0.000000,0.000000,0.000000}%
\pgfsetfillcolor{currentfill}%
\pgfsetlinewidth{0.602250pt}%
\definecolor{currentstroke}{rgb}{0.000000,0.000000,0.000000}%
\pgfsetstrokecolor{currentstroke}%
\pgfsetdash{}{0pt}%
\pgfsys@defobject{currentmarker}{\pgfqpoint{0.000000in}{-0.027778in}}{\pgfqpoint{0.000000in}{0.000000in}}{%
\pgfpathmoveto{\pgfqpoint{0.000000in}{0.000000in}}%
\pgfpathlineto{\pgfqpoint{0.000000in}{-0.027778in}}%
\pgfusepath{stroke,fill}%
}%
\begin{pgfscope}%
\pgfsys@transformshift{4.527564in}{0.923638in}%
\pgfsys@useobject{currentmarker}{}%
\end{pgfscope}%
\end{pgfscope}%
\begin{pgfscope}%
\pgfsetbuttcap%
\pgfsetroundjoin%
\definecolor{currentfill}{rgb}{0.000000,0.000000,0.000000}%
\pgfsetfillcolor{currentfill}%
\pgfsetlinewidth{0.602250pt}%
\definecolor{currentstroke}{rgb}{0.000000,0.000000,0.000000}%
\pgfsetstrokecolor{currentstroke}%
\pgfsetdash{}{0pt}%
\pgfsys@defobject{currentmarker}{\pgfqpoint{0.000000in}{-0.027778in}}{\pgfqpoint{0.000000in}{0.000000in}}{%
\pgfpathmoveto{\pgfqpoint{0.000000in}{0.000000in}}%
\pgfpathlineto{\pgfqpoint{0.000000in}{-0.027778in}}%
\pgfusepath{stroke,fill}%
}%
\begin{pgfscope}%
\pgfsys@transformshift{4.675962in}{0.923638in}%
\pgfsys@useobject{currentmarker}{}%
\end{pgfscope}%
\end{pgfscope}%
\begin{pgfscope}%
\pgfsetbuttcap%
\pgfsetroundjoin%
\definecolor{currentfill}{rgb}{0.000000,0.000000,0.000000}%
\pgfsetfillcolor{currentfill}%
\pgfsetlinewidth{0.803000pt}%
\definecolor{currentstroke}{rgb}{0.000000,0.000000,0.000000}%
\pgfsetstrokecolor{currentstroke}%
\pgfsetdash{}{0pt}%
\pgfsys@defobject{currentmarker}{\pgfqpoint{-0.048611in}{0.000000in}}{\pgfqpoint{0.000000in}{0.000000in}}{%
\pgfpathmoveto{\pgfqpoint{0.000000in}{0.000000in}}%
\pgfpathlineto{\pgfqpoint{-0.048611in}{0.000000in}}%
\pgfusepath{stroke,fill}%
}%
\begin{pgfscope}%
\pgfsys@transformshift{2.450000in}{0.181651in}%
\pgfsys@useobject{currentmarker}{}%
\end{pgfscope}%
\end{pgfscope}%
\begin{pgfscope}%
\pgftext[x=2.214296in,y=0.128890in,left,base]{\sffamily\fontsize{10.000000}{12.000000}\selectfont -1}%
\end{pgfscope}%
\begin{pgfscope}%
\pgfsetbuttcap%
\pgfsetroundjoin%
\definecolor{currentfill}{rgb}{0.000000,0.000000,0.000000}%
\pgfsetfillcolor{currentfill}%
\pgfsetlinewidth{0.803000pt}%
\definecolor{currentstroke}{rgb}{0.000000,0.000000,0.000000}%
\pgfsetstrokecolor{currentstroke}%
\pgfsetdash{}{0pt}%
\pgfsys@defobject{currentmarker}{\pgfqpoint{-0.048611in}{0.000000in}}{\pgfqpoint{0.000000in}{0.000000in}}{%
\pgfpathmoveto{\pgfqpoint{0.000000in}{0.000000in}}%
\pgfpathlineto{\pgfqpoint{-0.048611in}{0.000000in}}%
\pgfusepath{stroke,fill}%
}%
\begin{pgfscope}%
\pgfsys@transformshift{2.450000in}{0.552645in}%
\pgfsys@useobject{currentmarker}{}%
\end{pgfscope}%
\end{pgfscope}%
\begin{pgfscope}%
\pgftext[x=2.081782in,y=0.499883in,left,base]{\sffamily\fontsize{10.000000}{12.000000}\selectfont -0.5}%
\end{pgfscope}%
\begin{pgfscope}%
\pgfsetbuttcap%
\pgfsetroundjoin%
\definecolor{currentfill}{rgb}{0.000000,0.000000,0.000000}%
\pgfsetfillcolor{currentfill}%
\pgfsetlinewidth{0.803000pt}%
\definecolor{currentstroke}{rgb}{0.000000,0.000000,0.000000}%
\pgfsetstrokecolor{currentstroke}%
\pgfsetdash{}{0pt}%
\pgfsys@defobject{currentmarker}{\pgfqpoint{-0.048611in}{0.000000in}}{\pgfqpoint{0.000000in}{0.000000in}}{%
\pgfpathmoveto{\pgfqpoint{0.000000in}{0.000000in}}%
\pgfpathlineto{\pgfqpoint{-0.048611in}{0.000000in}}%
\pgfusepath{stroke,fill}%
}%
\begin{pgfscope}%
\pgfsys@transformshift{2.450000in}{0.923638in}%
\pgfsys@useobject{currentmarker}{}%
\end{pgfscope}%
\end{pgfscope}%
\begin{pgfscope}%
\pgfsetbuttcap%
\pgfsetroundjoin%
\definecolor{currentfill}{rgb}{0.000000,0.000000,0.000000}%
\pgfsetfillcolor{currentfill}%
\pgfsetlinewidth{0.803000pt}%
\definecolor{currentstroke}{rgb}{0.000000,0.000000,0.000000}%
\pgfsetstrokecolor{currentstroke}%
\pgfsetdash{}{0pt}%
\pgfsys@defobject{currentmarker}{\pgfqpoint{-0.048611in}{0.000000in}}{\pgfqpoint{0.000000in}{0.000000in}}{%
\pgfpathmoveto{\pgfqpoint{0.000000in}{0.000000in}}%
\pgfpathlineto{\pgfqpoint{-0.048611in}{0.000000in}}%
\pgfusepath{stroke,fill}%
}%
\begin{pgfscope}%
\pgfsys@transformshift{2.450000in}{1.294632in}%
\pgfsys@useobject{currentmarker}{}%
\end{pgfscope}%
\end{pgfscope}%
\begin{pgfscope}%
\pgftext[x=2.131898in,y=1.241870in,left,base]{\sffamily\fontsize{10.000000}{12.000000}\selectfont 0.5}%
\end{pgfscope}%
\begin{pgfscope}%
\pgfsetbuttcap%
\pgfsetroundjoin%
\definecolor{currentfill}{rgb}{0.000000,0.000000,0.000000}%
\pgfsetfillcolor{currentfill}%
\pgfsetlinewidth{0.803000pt}%
\definecolor{currentstroke}{rgb}{0.000000,0.000000,0.000000}%
\pgfsetstrokecolor{currentstroke}%
\pgfsetdash{}{0pt}%
\pgfsys@defobject{currentmarker}{\pgfqpoint{-0.048611in}{0.000000in}}{\pgfqpoint{0.000000in}{0.000000in}}{%
\pgfpathmoveto{\pgfqpoint{0.000000in}{0.000000in}}%
\pgfpathlineto{\pgfqpoint{-0.048611in}{0.000000in}}%
\pgfusepath{stroke,fill}%
}%
\begin{pgfscope}%
\pgfsys@transformshift{2.450000in}{1.665626in}%
\pgfsys@useobject{currentmarker}{}%
\end{pgfscope}%
\end{pgfscope}%
\begin{pgfscope}%
\pgftext[x=2.264413in,y=1.612864in,left,base]{\sffamily\fontsize{10.000000}{12.000000}\selectfont 1}%
\end{pgfscope}%
\begin{pgfscope}%
\pgfsetbuttcap%
\pgfsetroundjoin%
\definecolor{currentfill}{rgb}{0.000000,0.000000,0.000000}%
\pgfsetfillcolor{currentfill}%
\pgfsetlinewidth{0.602250pt}%
\definecolor{currentstroke}{rgb}{0.000000,0.000000,0.000000}%
\pgfsetstrokecolor{currentstroke}%
\pgfsetdash{}{0pt}%
\pgfsys@defobject{currentmarker}{\pgfqpoint{-0.027778in}{0.000000in}}{\pgfqpoint{0.000000in}{0.000000in}}{%
\pgfpathmoveto{\pgfqpoint{0.000000in}{0.000000in}}%
\pgfpathlineto{\pgfqpoint{-0.027778in}{0.000000in}}%
\pgfusepath{stroke,fill}%
}%
\begin{pgfscope}%
\pgfsys@transformshift{2.450000in}{0.255850in}%
\pgfsys@useobject{currentmarker}{}%
\end{pgfscope}%
\end{pgfscope}%
\begin{pgfscope}%
\pgfsetbuttcap%
\pgfsetroundjoin%
\definecolor{currentfill}{rgb}{0.000000,0.000000,0.000000}%
\pgfsetfillcolor{currentfill}%
\pgfsetlinewidth{0.602250pt}%
\definecolor{currentstroke}{rgb}{0.000000,0.000000,0.000000}%
\pgfsetstrokecolor{currentstroke}%
\pgfsetdash{}{0pt}%
\pgfsys@defobject{currentmarker}{\pgfqpoint{-0.027778in}{0.000000in}}{\pgfqpoint{0.000000in}{0.000000in}}{%
\pgfpathmoveto{\pgfqpoint{0.000000in}{0.000000in}}%
\pgfpathlineto{\pgfqpoint{-0.027778in}{0.000000in}}%
\pgfusepath{stroke,fill}%
}%
\begin{pgfscope}%
\pgfsys@transformshift{2.450000in}{0.330049in}%
\pgfsys@useobject{currentmarker}{}%
\end{pgfscope}%
\end{pgfscope}%
\begin{pgfscope}%
\pgfsetbuttcap%
\pgfsetroundjoin%
\definecolor{currentfill}{rgb}{0.000000,0.000000,0.000000}%
\pgfsetfillcolor{currentfill}%
\pgfsetlinewidth{0.602250pt}%
\definecolor{currentstroke}{rgb}{0.000000,0.000000,0.000000}%
\pgfsetstrokecolor{currentstroke}%
\pgfsetdash{}{0pt}%
\pgfsys@defobject{currentmarker}{\pgfqpoint{-0.027778in}{0.000000in}}{\pgfqpoint{0.000000in}{0.000000in}}{%
\pgfpathmoveto{\pgfqpoint{0.000000in}{0.000000in}}%
\pgfpathlineto{\pgfqpoint{-0.027778in}{0.000000in}}%
\pgfusepath{stroke,fill}%
}%
\begin{pgfscope}%
\pgfsys@transformshift{2.450000in}{0.404247in}%
\pgfsys@useobject{currentmarker}{}%
\end{pgfscope}%
\end{pgfscope}%
\begin{pgfscope}%
\pgfsetbuttcap%
\pgfsetroundjoin%
\definecolor{currentfill}{rgb}{0.000000,0.000000,0.000000}%
\pgfsetfillcolor{currentfill}%
\pgfsetlinewidth{0.602250pt}%
\definecolor{currentstroke}{rgb}{0.000000,0.000000,0.000000}%
\pgfsetstrokecolor{currentstroke}%
\pgfsetdash{}{0pt}%
\pgfsys@defobject{currentmarker}{\pgfqpoint{-0.027778in}{0.000000in}}{\pgfqpoint{0.000000in}{0.000000in}}{%
\pgfpathmoveto{\pgfqpoint{0.000000in}{0.000000in}}%
\pgfpathlineto{\pgfqpoint{-0.027778in}{0.000000in}}%
\pgfusepath{stroke,fill}%
}%
\begin{pgfscope}%
\pgfsys@transformshift{2.450000in}{0.478446in}%
\pgfsys@useobject{currentmarker}{}%
\end{pgfscope}%
\end{pgfscope}%
\begin{pgfscope}%
\pgfsetbuttcap%
\pgfsetroundjoin%
\definecolor{currentfill}{rgb}{0.000000,0.000000,0.000000}%
\pgfsetfillcolor{currentfill}%
\pgfsetlinewidth{0.602250pt}%
\definecolor{currentstroke}{rgb}{0.000000,0.000000,0.000000}%
\pgfsetstrokecolor{currentstroke}%
\pgfsetdash{}{0pt}%
\pgfsys@defobject{currentmarker}{\pgfqpoint{-0.027778in}{0.000000in}}{\pgfqpoint{0.000000in}{0.000000in}}{%
\pgfpathmoveto{\pgfqpoint{0.000000in}{0.000000in}}%
\pgfpathlineto{\pgfqpoint{-0.027778in}{0.000000in}}%
\pgfusepath{stroke,fill}%
}%
\begin{pgfscope}%
\pgfsys@transformshift{2.450000in}{0.552645in}%
\pgfsys@useobject{currentmarker}{}%
\end{pgfscope}%
\end{pgfscope}%
\begin{pgfscope}%
\pgfsetbuttcap%
\pgfsetroundjoin%
\definecolor{currentfill}{rgb}{0.000000,0.000000,0.000000}%
\pgfsetfillcolor{currentfill}%
\pgfsetlinewidth{0.602250pt}%
\definecolor{currentstroke}{rgb}{0.000000,0.000000,0.000000}%
\pgfsetstrokecolor{currentstroke}%
\pgfsetdash{}{0pt}%
\pgfsys@defobject{currentmarker}{\pgfqpoint{-0.027778in}{0.000000in}}{\pgfqpoint{0.000000in}{0.000000in}}{%
\pgfpathmoveto{\pgfqpoint{0.000000in}{0.000000in}}%
\pgfpathlineto{\pgfqpoint{-0.027778in}{0.000000in}}%
\pgfusepath{stroke,fill}%
}%
\begin{pgfscope}%
\pgfsys@transformshift{2.450000in}{0.626844in}%
\pgfsys@useobject{currentmarker}{}%
\end{pgfscope}%
\end{pgfscope}%
\begin{pgfscope}%
\pgfsetbuttcap%
\pgfsetroundjoin%
\definecolor{currentfill}{rgb}{0.000000,0.000000,0.000000}%
\pgfsetfillcolor{currentfill}%
\pgfsetlinewidth{0.602250pt}%
\definecolor{currentstroke}{rgb}{0.000000,0.000000,0.000000}%
\pgfsetstrokecolor{currentstroke}%
\pgfsetdash{}{0pt}%
\pgfsys@defobject{currentmarker}{\pgfqpoint{-0.027778in}{0.000000in}}{\pgfqpoint{0.000000in}{0.000000in}}{%
\pgfpathmoveto{\pgfqpoint{0.000000in}{0.000000in}}%
\pgfpathlineto{\pgfqpoint{-0.027778in}{0.000000in}}%
\pgfusepath{stroke,fill}%
}%
\begin{pgfscope}%
\pgfsys@transformshift{2.450000in}{0.701042in}%
\pgfsys@useobject{currentmarker}{}%
\end{pgfscope}%
\end{pgfscope}%
\begin{pgfscope}%
\pgfsetbuttcap%
\pgfsetroundjoin%
\definecolor{currentfill}{rgb}{0.000000,0.000000,0.000000}%
\pgfsetfillcolor{currentfill}%
\pgfsetlinewidth{0.602250pt}%
\definecolor{currentstroke}{rgb}{0.000000,0.000000,0.000000}%
\pgfsetstrokecolor{currentstroke}%
\pgfsetdash{}{0pt}%
\pgfsys@defobject{currentmarker}{\pgfqpoint{-0.027778in}{0.000000in}}{\pgfqpoint{0.000000in}{0.000000in}}{%
\pgfpathmoveto{\pgfqpoint{0.000000in}{0.000000in}}%
\pgfpathlineto{\pgfqpoint{-0.027778in}{0.000000in}}%
\pgfusepath{stroke,fill}%
}%
\begin{pgfscope}%
\pgfsys@transformshift{2.450000in}{0.775241in}%
\pgfsys@useobject{currentmarker}{}%
\end{pgfscope}%
\end{pgfscope}%
\begin{pgfscope}%
\pgfsetbuttcap%
\pgfsetroundjoin%
\definecolor{currentfill}{rgb}{0.000000,0.000000,0.000000}%
\pgfsetfillcolor{currentfill}%
\pgfsetlinewidth{0.602250pt}%
\definecolor{currentstroke}{rgb}{0.000000,0.000000,0.000000}%
\pgfsetstrokecolor{currentstroke}%
\pgfsetdash{}{0pt}%
\pgfsys@defobject{currentmarker}{\pgfqpoint{-0.027778in}{0.000000in}}{\pgfqpoint{0.000000in}{0.000000in}}{%
\pgfpathmoveto{\pgfqpoint{0.000000in}{0.000000in}}%
\pgfpathlineto{\pgfqpoint{-0.027778in}{0.000000in}}%
\pgfusepath{stroke,fill}%
}%
\begin{pgfscope}%
\pgfsys@transformshift{2.450000in}{0.849440in}%
\pgfsys@useobject{currentmarker}{}%
\end{pgfscope}%
\end{pgfscope}%
\begin{pgfscope}%
\pgfsetbuttcap%
\pgfsetroundjoin%
\definecolor{currentfill}{rgb}{0.000000,0.000000,0.000000}%
\pgfsetfillcolor{currentfill}%
\pgfsetlinewidth{0.602250pt}%
\definecolor{currentstroke}{rgb}{0.000000,0.000000,0.000000}%
\pgfsetstrokecolor{currentstroke}%
\pgfsetdash{}{0pt}%
\pgfsys@defobject{currentmarker}{\pgfqpoint{-0.027778in}{0.000000in}}{\pgfqpoint{0.000000in}{0.000000in}}{%
\pgfpathmoveto{\pgfqpoint{0.000000in}{0.000000in}}%
\pgfpathlineto{\pgfqpoint{-0.027778in}{0.000000in}}%
\pgfusepath{stroke,fill}%
}%
\begin{pgfscope}%
\pgfsys@transformshift{2.450000in}{0.923638in}%
\pgfsys@useobject{currentmarker}{}%
\end{pgfscope}%
\end{pgfscope}%
\begin{pgfscope}%
\pgfsetbuttcap%
\pgfsetroundjoin%
\definecolor{currentfill}{rgb}{0.000000,0.000000,0.000000}%
\pgfsetfillcolor{currentfill}%
\pgfsetlinewidth{0.602250pt}%
\definecolor{currentstroke}{rgb}{0.000000,0.000000,0.000000}%
\pgfsetstrokecolor{currentstroke}%
\pgfsetdash{}{0pt}%
\pgfsys@defobject{currentmarker}{\pgfqpoint{-0.027778in}{0.000000in}}{\pgfqpoint{0.000000in}{0.000000in}}{%
\pgfpathmoveto{\pgfqpoint{0.000000in}{0.000000in}}%
\pgfpathlineto{\pgfqpoint{-0.027778in}{0.000000in}}%
\pgfusepath{stroke,fill}%
}%
\begin{pgfscope}%
\pgfsys@transformshift{2.450000in}{0.997837in}%
\pgfsys@useobject{currentmarker}{}%
\end{pgfscope}%
\end{pgfscope}%
\begin{pgfscope}%
\pgfsetbuttcap%
\pgfsetroundjoin%
\definecolor{currentfill}{rgb}{0.000000,0.000000,0.000000}%
\pgfsetfillcolor{currentfill}%
\pgfsetlinewidth{0.602250pt}%
\definecolor{currentstroke}{rgb}{0.000000,0.000000,0.000000}%
\pgfsetstrokecolor{currentstroke}%
\pgfsetdash{}{0pt}%
\pgfsys@defobject{currentmarker}{\pgfqpoint{-0.027778in}{0.000000in}}{\pgfqpoint{0.000000in}{0.000000in}}{%
\pgfpathmoveto{\pgfqpoint{0.000000in}{0.000000in}}%
\pgfpathlineto{\pgfqpoint{-0.027778in}{0.000000in}}%
\pgfusepath{stroke,fill}%
}%
\begin{pgfscope}%
\pgfsys@transformshift{2.450000in}{1.072036in}%
\pgfsys@useobject{currentmarker}{}%
\end{pgfscope}%
\end{pgfscope}%
\begin{pgfscope}%
\pgfsetbuttcap%
\pgfsetroundjoin%
\definecolor{currentfill}{rgb}{0.000000,0.000000,0.000000}%
\pgfsetfillcolor{currentfill}%
\pgfsetlinewidth{0.602250pt}%
\definecolor{currentstroke}{rgb}{0.000000,0.000000,0.000000}%
\pgfsetstrokecolor{currentstroke}%
\pgfsetdash{}{0pt}%
\pgfsys@defobject{currentmarker}{\pgfqpoint{-0.027778in}{0.000000in}}{\pgfqpoint{0.000000in}{0.000000in}}{%
\pgfpathmoveto{\pgfqpoint{0.000000in}{0.000000in}}%
\pgfpathlineto{\pgfqpoint{-0.027778in}{0.000000in}}%
\pgfusepath{stroke,fill}%
}%
\begin{pgfscope}%
\pgfsys@transformshift{2.450000in}{1.146235in}%
\pgfsys@useobject{currentmarker}{}%
\end{pgfscope}%
\end{pgfscope}%
\begin{pgfscope}%
\pgfsetbuttcap%
\pgfsetroundjoin%
\definecolor{currentfill}{rgb}{0.000000,0.000000,0.000000}%
\pgfsetfillcolor{currentfill}%
\pgfsetlinewidth{0.602250pt}%
\definecolor{currentstroke}{rgb}{0.000000,0.000000,0.000000}%
\pgfsetstrokecolor{currentstroke}%
\pgfsetdash{}{0pt}%
\pgfsys@defobject{currentmarker}{\pgfqpoint{-0.027778in}{0.000000in}}{\pgfqpoint{0.000000in}{0.000000in}}{%
\pgfpathmoveto{\pgfqpoint{0.000000in}{0.000000in}}%
\pgfpathlineto{\pgfqpoint{-0.027778in}{0.000000in}}%
\pgfusepath{stroke,fill}%
}%
\begin{pgfscope}%
\pgfsys@transformshift{2.450000in}{1.220433in}%
\pgfsys@useobject{currentmarker}{}%
\end{pgfscope}%
\end{pgfscope}%
\begin{pgfscope}%
\pgfsetbuttcap%
\pgfsetroundjoin%
\definecolor{currentfill}{rgb}{0.000000,0.000000,0.000000}%
\pgfsetfillcolor{currentfill}%
\pgfsetlinewidth{0.602250pt}%
\definecolor{currentstroke}{rgb}{0.000000,0.000000,0.000000}%
\pgfsetstrokecolor{currentstroke}%
\pgfsetdash{}{0pt}%
\pgfsys@defobject{currentmarker}{\pgfqpoint{-0.027778in}{0.000000in}}{\pgfqpoint{0.000000in}{0.000000in}}{%
\pgfpathmoveto{\pgfqpoint{0.000000in}{0.000000in}}%
\pgfpathlineto{\pgfqpoint{-0.027778in}{0.000000in}}%
\pgfusepath{stroke,fill}%
}%
\begin{pgfscope}%
\pgfsys@transformshift{2.450000in}{1.294632in}%
\pgfsys@useobject{currentmarker}{}%
\end{pgfscope}%
\end{pgfscope}%
\begin{pgfscope}%
\pgfsetbuttcap%
\pgfsetroundjoin%
\definecolor{currentfill}{rgb}{0.000000,0.000000,0.000000}%
\pgfsetfillcolor{currentfill}%
\pgfsetlinewidth{0.602250pt}%
\definecolor{currentstroke}{rgb}{0.000000,0.000000,0.000000}%
\pgfsetstrokecolor{currentstroke}%
\pgfsetdash{}{0pt}%
\pgfsys@defobject{currentmarker}{\pgfqpoint{-0.027778in}{0.000000in}}{\pgfqpoint{0.000000in}{0.000000in}}{%
\pgfpathmoveto{\pgfqpoint{0.000000in}{0.000000in}}%
\pgfpathlineto{\pgfqpoint{-0.027778in}{0.000000in}}%
\pgfusepath{stroke,fill}%
}%
\begin{pgfscope}%
\pgfsys@transformshift{2.450000in}{1.368831in}%
\pgfsys@useobject{currentmarker}{}%
\end{pgfscope}%
\end{pgfscope}%
\begin{pgfscope}%
\pgfsetbuttcap%
\pgfsetroundjoin%
\definecolor{currentfill}{rgb}{0.000000,0.000000,0.000000}%
\pgfsetfillcolor{currentfill}%
\pgfsetlinewidth{0.602250pt}%
\definecolor{currentstroke}{rgb}{0.000000,0.000000,0.000000}%
\pgfsetstrokecolor{currentstroke}%
\pgfsetdash{}{0pt}%
\pgfsys@defobject{currentmarker}{\pgfqpoint{-0.027778in}{0.000000in}}{\pgfqpoint{0.000000in}{0.000000in}}{%
\pgfpathmoveto{\pgfqpoint{0.000000in}{0.000000in}}%
\pgfpathlineto{\pgfqpoint{-0.027778in}{0.000000in}}%
\pgfusepath{stroke,fill}%
}%
\begin{pgfscope}%
\pgfsys@transformshift{2.450000in}{1.443029in}%
\pgfsys@useobject{currentmarker}{}%
\end{pgfscope}%
\end{pgfscope}%
\begin{pgfscope}%
\pgfsetbuttcap%
\pgfsetroundjoin%
\definecolor{currentfill}{rgb}{0.000000,0.000000,0.000000}%
\pgfsetfillcolor{currentfill}%
\pgfsetlinewidth{0.602250pt}%
\definecolor{currentstroke}{rgb}{0.000000,0.000000,0.000000}%
\pgfsetstrokecolor{currentstroke}%
\pgfsetdash{}{0pt}%
\pgfsys@defobject{currentmarker}{\pgfqpoint{-0.027778in}{0.000000in}}{\pgfqpoint{0.000000in}{0.000000in}}{%
\pgfpathmoveto{\pgfqpoint{0.000000in}{0.000000in}}%
\pgfpathlineto{\pgfqpoint{-0.027778in}{0.000000in}}%
\pgfusepath{stroke,fill}%
}%
\begin{pgfscope}%
\pgfsys@transformshift{2.450000in}{1.517228in}%
\pgfsys@useobject{currentmarker}{}%
\end{pgfscope}%
\end{pgfscope}%
\begin{pgfscope}%
\pgfsetbuttcap%
\pgfsetroundjoin%
\definecolor{currentfill}{rgb}{0.000000,0.000000,0.000000}%
\pgfsetfillcolor{currentfill}%
\pgfsetlinewidth{0.602250pt}%
\definecolor{currentstroke}{rgb}{0.000000,0.000000,0.000000}%
\pgfsetstrokecolor{currentstroke}%
\pgfsetdash{}{0pt}%
\pgfsys@defobject{currentmarker}{\pgfqpoint{-0.027778in}{0.000000in}}{\pgfqpoint{0.000000in}{0.000000in}}{%
\pgfpathmoveto{\pgfqpoint{0.000000in}{0.000000in}}%
\pgfpathlineto{\pgfqpoint{-0.027778in}{0.000000in}}%
\pgfusepath{stroke,fill}%
}%
\begin{pgfscope}%
\pgfsys@transformshift{2.450000in}{1.591427in}%
\pgfsys@useobject{currentmarker}{}%
\end{pgfscope}%
\end{pgfscope}%
\begin{pgfscope}%
\pgfsetbuttcap%
\pgfsetroundjoin%
\definecolor{currentfill}{rgb}{0.000000,0.000000,0.000000}%
\pgfsetfillcolor{currentfill}%
\pgfsetlinewidth{0.602250pt}%
\definecolor{currentstroke}{rgb}{0.000000,0.000000,0.000000}%
\pgfsetstrokecolor{currentstroke}%
\pgfsetdash{}{0pt}%
\pgfsys@defobject{currentmarker}{\pgfqpoint{-0.027778in}{0.000000in}}{\pgfqpoint{0.000000in}{0.000000in}}{%
\pgfpathmoveto{\pgfqpoint{0.000000in}{0.000000in}}%
\pgfpathlineto{\pgfqpoint{-0.027778in}{0.000000in}}%
\pgfusepath{stroke,fill}%
}%
\begin{pgfscope}%
\pgfsys@transformshift{2.450000in}{1.665626in}%
\pgfsys@useobject{currentmarker}{}%
\end{pgfscope}%
\end{pgfscope}%
\begin{pgfscope}%
\pgfpathrectangle{\pgfqpoint{0.135000in}{0.151972in}}{\pgfqpoint{4.630000in}{1.543333in}} %
\pgfusepath{clip}%
\pgfsetbuttcap%
\pgfsetroundjoin%
\pgfsetlinewidth{1.505625pt}%
\definecolor{currentstroke}{rgb}{0.000000,0.000000,1.000000}%
\pgfsetstrokecolor{currentstroke}%
\pgfsetdash{}{0pt}%
\pgfpathmoveto{\pgfqpoint{3.809480in}{0.189659in}}%
\pgfpathlineto{\pgfqpoint{3.749723in}{0.197538in}}%
\pgfpathlineto{\pgfqpoint{3.666098in}{0.211530in}}%
\pgfpathlineto{\pgfqpoint{3.614857in}{0.221489in}}%
\pgfpathlineto{\pgfqpoint{3.540572in}{0.237360in}}%
\pgfpathlineto{\pgfqpoint{3.440364in}{0.261328in}}%
\pgfpathlineto{\pgfqpoint{3.361300in}{0.281831in}}%
\pgfpathlineto{\pgfqpoint{3.241785in}{0.315331in}}%
\pgfpathlineto{\pgfqpoint{3.122270in}{0.351365in}}%
\pgfpathlineto{\pgfqpoint{2.999354in}{0.390802in}}%
\pgfpathlineto{\pgfqpoint{2.881491in}{0.430640in}}%
\pgfpathlineto{\pgfqpoint{2.741374in}{0.480438in}}%
\pgfpathlineto{\pgfqpoint{2.634000in}{0.520276in}}%
\pgfpathlineto{\pgfqpoint{2.494818in}{0.574006in}}%
\pgfpathlineto{\pgfqpoint{2.345425in}{0.634302in}}%
\pgfpathlineto{\pgfqpoint{2.196031in}{0.697310in}}%
\pgfpathlineto{\pgfqpoint{2.055111in}{0.759306in}}%
\pgfpathlineto{\pgfqpoint{1.924571in}{0.819063in}}%
\pgfpathlineto{\pgfqpoint{1.799004in}{0.878820in}}%
\pgfpathlineto{\pgfqpoint{1.678285in}{0.938578in}}%
\pgfpathlineto{\pgfqpoint{1.562390in}{0.998335in}}%
\pgfpathlineto{\pgfqpoint{1.449064in}{1.059381in}}%
\pgfpathlineto{\pgfqpoint{1.345468in}{1.117850in}}%
\pgfpathlineto{\pgfqpoint{1.261278in}{1.167648in}}%
\pgfpathlineto{\pgfqpoint{1.180156in}{1.218021in}}%
\pgfpathlineto{\pgfqpoint{1.105194in}{1.267243in}}%
\pgfpathlineto{\pgfqpoint{1.047930in}{1.307081in}}%
\pgfpathlineto{\pgfqpoint{0.994071in}{1.346920in}}%
\pgfpathlineto{\pgfqpoint{0.941127in}{1.389255in}}%
\pgfpathlineto{\pgfqpoint{0.909498in}{1.416637in}}%
\pgfpathlineto{\pgfqpoint{0.877844in}{1.446515in}}%
\pgfpathlineto{\pgfqpoint{0.849538in}{1.476394in}}%
\pgfpathlineto{\pgfqpoint{0.825362in}{1.506273in}}%
\pgfpathlineto{\pgfqpoint{0.812057in}{1.526192in}}%
\pgfpathlineto{\pgfqpoint{0.801404in}{1.546111in}}%
\pgfpathlineto{\pgfqpoint{0.793961in}{1.566030in}}%
\pgfpathlineto{\pgfqpoint{0.791733in}{1.576404in}}%
\pgfpathlineto{\pgfqpoint{0.790908in}{1.585949in}}%
\pgfpathlineto{\pgfqpoint{0.791733in}{1.597154in}}%
\pgfpathlineto{\pgfqpoint{0.793897in}{1.605868in}}%
\pgfpathlineto{\pgfqpoint{0.798639in}{1.615828in}}%
\pgfpathlineto{\pgfqpoint{0.806261in}{1.625787in}}%
\pgfpathlineto{\pgfqpoint{0.821612in}{1.638431in}}%
\pgfpathlineto{\pgfqpoint{0.835712in}{1.645706in}}%
\pgfpathlineto{\pgfqpoint{0.851491in}{1.651799in}}%
\pgfpathlineto{\pgfqpoint{0.881369in}{1.659095in}}%
\pgfpathlineto{\pgfqpoint{0.911248in}{1.663166in}}%
\pgfpathlineto{\pgfqpoint{0.941127in}{1.665164in}}%
\pgfpathlineto{\pgfqpoint{0.971005in}{1.665609in}}%
\pgfpathlineto{\pgfqpoint{1.030763in}{1.663165in}}%
\pgfpathlineto{\pgfqpoint{1.090520in}{1.657618in}}%
\pgfpathlineto{\pgfqpoint{1.150277in}{1.649739in}}%
\pgfpathlineto{\pgfqpoint{1.233902in}{1.635747in}}%
\pgfpathlineto{\pgfqpoint{1.285143in}{1.625787in}}%
\pgfpathlineto{\pgfqpoint{1.359428in}{1.609917in}}%
\pgfpathlineto{\pgfqpoint{1.459636in}{1.585949in}}%
\pgfpathlineto{\pgfqpoint{1.538700in}{1.565446in}}%
\pgfpathlineto{\pgfqpoint{1.658215in}{1.531945in}}%
\pgfpathlineto{\pgfqpoint{1.777730in}{1.495912in}}%
\pgfpathlineto{\pgfqpoint{1.900646in}{1.456475in}}%
\pgfpathlineto{\pgfqpoint{2.018509in}{1.416637in}}%
\pgfpathlineto{\pgfqpoint{2.158626in}{1.366839in}}%
\pgfpathlineto{\pgfqpoint{2.266000in}{1.327001in}}%
\pgfpathlineto{\pgfqpoint{2.405182in}{1.273271in}}%
\pgfpathlineto{\pgfqpoint{2.554575in}{1.212975in}}%
\pgfpathlineto{\pgfqpoint{2.703969in}{1.149966in}}%
\pgfpathlineto{\pgfqpoint{2.844889in}{1.087971in}}%
\pgfpathlineto{\pgfqpoint{2.975429in}{1.028214in}}%
\pgfpathlineto{\pgfqpoint{3.100996in}{0.968456in}}%
\pgfpathlineto{\pgfqpoint{3.221715in}{0.908699in}}%
\pgfpathlineto{\pgfqpoint{3.337610in}{0.848942in}}%
\pgfpathlineto{\pgfqpoint{3.450936in}{0.787896in}}%
\pgfpathlineto{\pgfqpoint{3.554532in}{0.729427in}}%
\pgfpathlineto{\pgfqpoint{3.638722in}{0.679629in}}%
\pgfpathlineto{\pgfqpoint{3.719844in}{0.629256in}}%
\pgfpathlineto{\pgfqpoint{3.794806in}{0.580034in}}%
\pgfpathlineto{\pgfqpoint{3.852070in}{0.540195in}}%
\pgfpathlineto{\pgfqpoint{3.905929in}{0.500357in}}%
\pgfpathlineto{\pgfqpoint{3.958873in}{0.458022in}}%
\pgfpathlineto{\pgfqpoint{3.990502in}{0.430640in}}%
\pgfpathlineto{\pgfqpoint{4.022156in}{0.400762in}}%
\pgfpathlineto{\pgfqpoint{4.050462in}{0.370883in}}%
\pgfpathlineto{\pgfqpoint{4.074638in}{0.341004in}}%
\pgfpathlineto{\pgfqpoint{4.087943in}{0.321085in}}%
\pgfpathlineto{\pgfqpoint{4.098596in}{0.301166in}}%
\pgfpathlineto{\pgfqpoint{4.106039in}{0.281247in}}%
\pgfpathlineto{\pgfqpoint{4.108267in}{0.270872in}}%
\pgfpathlineto{\pgfqpoint{4.109092in}{0.261328in}}%
\pgfpathlineto{\pgfqpoint{4.108267in}{0.250123in}}%
\pgfpathlineto{\pgfqpoint{4.106103in}{0.241409in}}%
\pgfpathlineto{\pgfqpoint{4.101361in}{0.231449in}}%
\pgfpathlineto{\pgfqpoint{4.093739in}{0.221489in}}%
\pgfpathlineto{\pgfqpoint{4.078388in}{0.208846in}}%
\pgfpathlineto{\pgfqpoint{4.064288in}{0.201570in}}%
\pgfpathlineto{\pgfqpoint{4.048509in}{0.195478in}}%
\pgfpathlineto{\pgfqpoint{4.018631in}{0.188182in}}%
\pgfpathlineto{\pgfqpoint{3.988752in}{0.184110in}}%
\pgfpathlineto{\pgfqpoint{3.958873in}{0.182113in}}%
\pgfpathlineto{\pgfqpoint{3.928995in}{0.181668in}}%
\pgfpathlineto{\pgfqpoint{3.869237in}{0.184112in}}%
\pgfpathlineto{\pgfqpoint{3.809480in}{0.189659in}}%
\pgfpathlineto{\pgfqpoint{3.809480in}{0.189659in}}%
\pgfusepath{stroke}%
\end{pgfscope}%
\begin{pgfscope}%
\pgfpathrectangle{\pgfqpoint{0.135000in}{0.151972in}}{\pgfqpoint{4.630000in}{1.543333in}} %
\pgfusepath{clip}%
\pgfsetbuttcap%
\pgfsetroundjoin%
\pgfsetlinewidth{1.505625pt}%
\definecolor{currentstroke}{rgb}{1.000000,1.000000,1.000000}%
\pgfsetstrokecolor{currentstroke}%
\pgfsetdash{}{0pt}%
\pgfpathmoveto{\pgfqpoint{3.809480in}{0.189659in}}%
\pgfpathlineto{\pgfqpoint{3.749723in}{0.197538in}}%
\pgfpathlineto{\pgfqpoint{3.666098in}{0.211530in}}%
\pgfpathlineto{\pgfqpoint{3.614857in}{0.221489in}}%
\pgfpathlineto{\pgfqpoint{3.540572in}{0.237360in}}%
\pgfpathlineto{\pgfqpoint{3.440364in}{0.261328in}}%
\pgfpathlineto{\pgfqpoint{3.361300in}{0.281831in}}%
\pgfpathlineto{\pgfqpoint{3.241785in}{0.315331in}}%
\pgfpathlineto{\pgfqpoint{3.122270in}{0.351365in}}%
\pgfpathlineto{\pgfqpoint{2.999354in}{0.390802in}}%
\pgfpathlineto{\pgfqpoint{2.881491in}{0.430640in}}%
\pgfpathlineto{\pgfqpoint{2.741374in}{0.480438in}}%
\pgfpathlineto{\pgfqpoint{2.634000in}{0.520276in}}%
\pgfpathlineto{\pgfqpoint{2.494818in}{0.574006in}}%
\pgfpathlineto{\pgfqpoint{2.345425in}{0.634302in}}%
\pgfpathlineto{\pgfqpoint{2.196031in}{0.697310in}}%
\pgfpathlineto{\pgfqpoint{2.055111in}{0.759306in}}%
\pgfpathlineto{\pgfqpoint{1.924571in}{0.819063in}}%
\pgfpathlineto{\pgfqpoint{1.799004in}{0.878820in}}%
\pgfpathlineto{\pgfqpoint{1.678285in}{0.938578in}}%
\pgfpathlineto{\pgfqpoint{1.562390in}{0.998335in}}%
\pgfpathlineto{\pgfqpoint{1.449064in}{1.059381in}}%
\pgfpathlineto{\pgfqpoint{1.345468in}{1.117850in}}%
\pgfpathlineto{\pgfqpoint{1.261278in}{1.167648in}}%
\pgfpathlineto{\pgfqpoint{1.180156in}{1.218021in}}%
\pgfpathlineto{\pgfqpoint{1.105194in}{1.267243in}}%
\pgfpathlineto{\pgfqpoint{1.047930in}{1.307081in}}%
\pgfpathlineto{\pgfqpoint{0.994071in}{1.346920in}}%
\pgfpathlineto{\pgfqpoint{0.941127in}{1.389255in}}%
\pgfpathlineto{\pgfqpoint{0.909498in}{1.416637in}}%
\pgfpathlineto{\pgfqpoint{0.877844in}{1.446515in}}%
\pgfpathlineto{\pgfqpoint{0.849538in}{1.476394in}}%
\pgfpathlineto{\pgfqpoint{0.825362in}{1.506273in}}%
\pgfpathlineto{\pgfqpoint{0.812057in}{1.526192in}}%
\pgfpathlineto{\pgfqpoint{0.801404in}{1.546111in}}%
\pgfpathlineto{\pgfqpoint{0.793961in}{1.566030in}}%
\pgfpathlineto{\pgfqpoint{0.791733in}{1.576404in}}%
\pgfpathlineto{\pgfqpoint{0.790908in}{1.585949in}}%
\pgfpathlineto{\pgfqpoint{0.791733in}{1.597154in}}%
\pgfpathlineto{\pgfqpoint{0.793897in}{1.605868in}}%
\pgfpathlineto{\pgfqpoint{0.798639in}{1.615828in}}%
\pgfpathlineto{\pgfqpoint{0.806261in}{1.625787in}}%
\pgfpathlineto{\pgfqpoint{0.821612in}{1.638431in}}%
\pgfpathlineto{\pgfqpoint{0.835712in}{1.645706in}}%
\pgfpathlineto{\pgfqpoint{0.851491in}{1.651799in}}%
\pgfpathlineto{\pgfqpoint{0.881369in}{1.659095in}}%
\pgfpathlineto{\pgfqpoint{0.911248in}{1.663166in}}%
\pgfpathlineto{\pgfqpoint{0.941127in}{1.665164in}}%
\pgfpathlineto{\pgfqpoint{0.971005in}{1.665609in}}%
\pgfpathlineto{\pgfqpoint{1.030763in}{1.663165in}}%
\pgfpathlineto{\pgfqpoint{1.090520in}{1.657618in}}%
\pgfpathlineto{\pgfqpoint{1.150277in}{1.649739in}}%
\pgfpathlineto{\pgfqpoint{1.233902in}{1.635747in}}%
\pgfpathlineto{\pgfqpoint{1.285143in}{1.625787in}}%
\pgfpathlineto{\pgfqpoint{1.359428in}{1.609917in}}%
\pgfpathlineto{\pgfqpoint{1.459636in}{1.585949in}}%
\pgfpathlineto{\pgfqpoint{1.538700in}{1.565446in}}%
\pgfpathlineto{\pgfqpoint{1.658215in}{1.531945in}}%
\pgfpathlineto{\pgfqpoint{1.777730in}{1.495912in}}%
\pgfpathlineto{\pgfqpoint{1.900646in}{1.456475in}}%
\pgfpathlineto{\pgfqpoint{2.018509in}{1.416637in}}%
\pgfpathlineto{\pgfqpoint{2.158626in}{1.366839in}}%
\pgfpathlineto{\pgfqpoint{2.266000in}{1.327001in}}%
\pgfpathlineto{\pgfqpoint{2.405182in}{1.273271in}}%
\pgfpathlineto{\pgfqpoint{2.554575in}{1.212975in}}%
\pgfpathlineto{\pgfqpoint{2.703969in}{1.149966in}}%
\pgfpathlineto{\pgfqpoint{2.844889in}{1.087971in}}%
\pgfpathlineto{\pgfqpoint{2.975429in}{1.028214in}}%
\pgfpathlineto{\pgfqpoint{3.100996in}{0.968456in}}%
\pgfpathlineto{\pgfqpoint{3.221715in}{0.908699in}}%
\pgfpathlineto{\pgfqpoint{3.337610in}{0.848942in}}%
\pgfpathlineto{\pgfqpoint{3.450936in}{0.787896in}}%
\pgfpathlineto{\pgfqpoint{3.554532in}{0.729427in}}%
\pgfpathlineto{\pgfqpoint{3.638722in}{0.679629in}}%
\pgfpathlineto{\pgfqpoint{3.719844in}{0.629256in}}%
\pgfpathlineto{\pgfqpoint{3.794806in}{0.580034in}}%
\pgfpathlineto{\pgfqpoint{3.852070in}{0.540195in}}%
\pgfpathlineto{\pgfqpoint{3.905929in}{0.500357in}}%
\pgfpathlineto{\pgfqpoint{3.958873in}{0.458022in}}%
\pgfpathlineto{\pgfqpoint{3.990502in}{0.430640in}}%
\pgfpathlineto{\pgfqpoint{4.022156in}{0.400762in}}%
\pgfpathlineto{\pgfqpoint{4.050462in}{0.370883in}}%
\pgfpathlineto{\pgfqpoint{4.074638in}{0.341004in}}%
\pgfpathlineto{\pgfqpoint{4.087943in}{0.321085in}}%
\pgfpathlineto{\pgfqpoint{4.098596in}{0.301166in}}%
\pgfpathlineto{\pgfqpoint{4.106039in}{0.281247in}}%
\pgfpathlineto{\pgfqpoint{4.108267in}{0.270872in}}%
\pgfpathlineto{\pgfqpoint{4.109092in}{0.261328in}}%
\pgfpathlineto{\pgfqpoint{4.108267in}{0.250123in}}%
\pgfpathlineto{\pgfqpoint{4.106103in}{0.241409in}}%
\pgfpathlineto{\pgfqpoint{4.101361in}{0.231449in}}%
\pgfpathlineto{\pgfqpoint{4.093739in}{0.221489in}}%
\pgfpathlineto{\pgfqpoint{4.078388in}{0.208846in}}%
\pgfpathlineto{\pgfqpoint{4.064288in}{0.201570in}}%
\pgfpathlineto{\pgfqpoint{4.048509in}{0.195478in}}%
\pgfpathlineto{\pgfqpoint{4.018631in}{0.188182in}}%
\pgfpathlineto{\pgfqpoint{3.988752in}{0.184110in}}%
\pgfpathlineto{\pgfqpoint{3.958873in}{0.182113in}}%
\pgfpathlineto{\pgfqpoint{3.928995in}{0.181668in}}%
\pgfpathlineto{\pgfqpoint{3.869237in}{0.184112in}}%
\pgfpathlineto{\pgfqpoint{3.809480in}{0.189659in}}%
\pgfpathlineto{\pgfqpoint{3.809480in}{0.189659in}}%
\pgfusepath{stroke}%
\end{pgfscope}%
\begin{pgfscope}%
\pgfpathrectangle{\pgfqpoint{0.135000in}{0.151972in}}{\pgfqpoint{4.630000in}{1.543333in}} %
\pgfusepath{clip}%
\pgfsetbuttcap%
\pgfsetroundjoin%
\pgfsetlinewidth{1.505625pt}%
\definecolor{currentstroke}{rgb}{1.000000,1.000000,1.000000}%
\pgfsetstrokecolor{currentstroke}%
\pgfsetdash{}{0pt}%
\pgfpathmoveto{\pgfqpoint{3.809480in}{0.189659in}}%
\pgfpathlineto{\pgfqpoint{3.749723in}{0.197538in}}%
\pgfpathlineto{\pgfqpoint{3.666098in}{0.211530in}}%
\pgfpathlineto{\pgfqpoint{3.614857in}{0.221489in}}%
\pgfpathlineto{\pgfqpoint{3.540572in}{0.237360in}}%
\pgfpathlineto{\pgfqpoint{3.440364in}{0.261328in}}%
\pgfpathlineto{\pgfqpoint{3.361300in}{0.281831in}}%
\pgfpathlineto{\pgfqpoint{3.241785in}{0.315331in}}%
\pgfpathlineto{\pgfqpoint{3.122270in}{0.351365in}}%
\pgfpathlineto{\pgfqpoint{2.999354in}{0.390802in}}%
\pgfpathlineto{\pgfqpoint{2.881491in}{0.430640in}}%
\pgfpathlineto{\pgfqpoint{2.741374in}{0.480438in}}%
\pgfpathlineto{\pgfqpoint{2.634000in}{0.520276in}}%
\pgfpathlineto{\pgfqpoint{2.494818in}{0.574006in}}%
\pgfpathlineto{\pgfqpoint{2.345425in}{0.634302in}}%
\pgfpathlineto{\pgfqpoint{2.196031in}{0.697310in}}%
\pgfpathlineto{\pgfqpoint{2.055111in}{0.759306in}}%
\pgfpathlineto{\pgfqpoint{1.924571in}{0.819063in}}%
\pgfpathlineto{\pgfqpoint{1.799004in}{0.878820in}}%
\pgfpathlineto{\pgfqpoint{1.678285in}{0.938578in}}%
\pgfpathlineto{\pgfqpoint{1.562390in}{0.998335in}}%
\pgfpathlineto{\pgfqpoint{1.449064in}{1.059381in}}%
\pgfpathlineto{\pgfqpoint{1.345468in}{1.117850in}}%
\pgfpathlineto{\pgfqpoint{1.261278in}{1.167648in}}%
\pgfpathlineto{\pgfqpoint{1.180156in}{1.218021in}}%
\pgfpathlineto{\pgfqpoint{1.105194in}{1.267243in}}%
\pgfpathlineto{\pgfqpoint{1.047930in}{1.307081in}}%
\pgfpathlineto{\pgfqpoint{0.994071in}{1.346920in}}%
\pgfpathlineto{\pgfqpoint{0.941127in}{1.389255in}}%
\pgfpathlineto{\pgfqpoint{0.909498in}{1.416637in}}%
\pgfpathlineto{\pgfqpoint{0.877844in}{1.446515in}}%
\pgfpathlineto{\pgfqpoint{0.849538in}{1.476394in}}%
\pgfpathlineto{\pgfqpoint{0.825362in}{1.506273in}}%
\pgfpathlineto{\pgfqpoint{0.812057in}{1.526192in}}%
\pgfpathlineto{\pgfqpoint{0.801404in}{1.546111in}}%
\pgfpathlineto{\pgfqpoint{0.793961in}{1.566030in}}%
\pgfpathlineto{\pgfqpoint{0.791733in}{1.576404in}}%
\pgfpathlineto{\pgfqpoint{0.790908in}{1.585949in}}%
\pgfpathlineto{\pgfqpoint{0.791733in}{1.597154in}}%
\pgfpathlineto{\pgfqpoint{0.793897in}{1.605868in}}%
\pgfpathlineto{\pgfqpoint{0.798639in}{1.615828in}}%
\pgfpathlineto{\pgfqpoint{0.806261in}{1.625787in}}%
\pgfpathlineto{\pgfqpoint{0.821612in}{1.638431in}}%
\pgfpathlineto{\pgfqpoint{0.835712in}{1.645706in}}%
\pgfpathlineto{\pgfqpoint{0.851491in}{1.651799in}}%
\pgfpathlineto{\pgfqpoint{0.881369in}{1.659095in}}%
\pgfpathlineto{\pgfqpoint{0.911248in}{1.663166in}}%
\pgfpathlineto{\pgfqpoint{0.941127in}{1.665164in}}%
\pgfpathlineto{\pgfqpoint{0.971005in}{1.665609in}}%
\pgfpathlineto{\pgfqpoint{1.030763in}{1.663165in}}%
\pgfpathlineto{\pgfqpoint{1.090520in}{1.657618in}}%
\pgfpathlineto{\pgfqpoint{1.150277in}{1.649739in}}%
\pgfpathlineto{\pgfqpoint{1.233902in}{1.635747in}}%
\pgfpathlineto{\pgfqpoint{1.285143in}{1.625787in}}%
\pgfpathlineto{\pgfqpoint{1.359428in}{1.609917in}}%
\pgfpathlineto{\pgfqpoint{1.459636in}{1.585949in}}%
\pgfpathlineto{\pgfqpoint{1.538700in}{1.565446in}}%
\pgfpathlineto{\pgfqpoint{1.658215in}{1.531945in}}%
\pgfpathlineto{\pgfqpoint{1.777730in}{1.495912in}}%
\pgfpathlineto{\pgfqpoint{1.900646in}{1.456475in}}%
\pgfpathlineto{\pgfqpoint{2.018509in}{1.416637in}}%
\pgfpathlineto{\pgfqpoint{2.158626in}{1.366839in}}%
\pgfpathlineto{\pgfqpoint{2.266000in}{1.327001in}}%
\pgfpathlineto{\pgfqpoint{2.405182in}{1.273271in}}%
\pgfpathlineto{\pgfqpoint{2.554575in}{1.212975in}}%
\pgfpathlineto{\pgfqpoint{2.703969in}{1.149966in}}%
\pgfpathlineto{\pgfqpoint{2.844889in}{1.087971in}}%
\pgfpathlineto{\pgfqpoint{2.975429in}{1.028214in}}%
\pgfpathlineto{\pgfqpoint{3.100996in}{0.968456in}}%
\pgfpathlineto{\pgfqpoint{3.221715in}{0.908699in}}%
\pgfpathlineto{\pgfqpoint{3.337610in}{0.848942in}}%
\pgfpathlineto{\pgfqpoint{3.450936in}{0.787896in}}%
\pgfpathlineto{\pgfqpoint{3.554532in}{0.729427in}}%
\pgfpathlineto{\pgfqpoint{3.638722in}{0.679629in}}%
\pgfpathlineto{\pgfqpoint{3.719844in}{0.629256in}}%
\pgfpathlineto{\pgfqpoint{3.794806in}{0.580034in}}%
\pgfpathlineto{\pgfqpoint{3.852070in}{0.540195in}}%
\pgfpathlineto{\pgfqpoint{3.905929in}{0.500357in}}%
\pgfpathlineto{\pgfqpoint{3.958873in}{0.458022in}}%
\pgfpathlineto{\pgfqpoint{3.990502in}{0.430640in}}%
\pgfpathlineto{\pgfqpoint{4.022156in}{0.400762in}}%
\pgfpathlineto{\pgfqpoint{4.050462in}{0.370883in}}%
\pgfpathlineto{\pgfqpoint{4.074638in}{0.341004in}}%
\pgfpathlineto{\pgfqpoint{4.087943in}{0.321085in}}%
\pgfpathlineto{\pgfqpoint{4.098596in}{0.301166in}}%
\pgfpathlineto{\pgfqpoint{4.106039in}{0.281247in}}%
\pgfpathlineto{\pgfqpoint{4.108267in}{0.270872in}}%
\pgfpathlineto{\pgfqpoint{4.109092in}{0.261328in}}%
\pgfpathlineto{\pgfqpoint{4.108267in}{0.250123in}}%
\pgfpathlineto{\pgfqpoint{4.106103in}{0.241409in}}%
\pgfpathlineto{\pgfqpoint{4.101361in}{0.231449in}}%
\pgfpathlineto{\pgfqpoint{4.093739in}{0.221489in}}%
\pgfpathlineto{\pgfqpoint{4.078388in}{0.208846in}}%
\pgfpathlineto{\pgfqpoint{4.064288in}{0.201570in}}%
\pgfpathlineto{\pgfqpoint{4.048509in}{0.195478in}}%
\pgfpathlineto{\pgfqpoint{4.018631in}{0.188182in}}%
\pgfpathlineto{\pgfqpoint{3.988752in}{0.184110in}}%
\pgfpathlineto{\pgfqpoint{3.958873in}{0.182113in}}%
\pgfpathlineto{\pgfqpoint{3.928995in}{0.181668in}}%
\pgfpathlineto{\pgfqpoint{3.869237in}{0.184112in}}%
\pgfpathlineto{\pgfqpoint{3.809480in}{0.189659in}}%
\pgfpathlineto{\pgfqpoint{3.809480in}{0.189659in}}%
\pgfusepath{stroke}%
\end{pgfscope}%
\begin{pgfscope}%
\pgfpathrectangle{\pgfqpoint{0.135000in}{0.151972in}}{\pgfqpoint{4.630000in}{1.543333in}} %
\pgfusepath{clip}%
\pgfsetbuttcap%
\pgfsetroundjoin%
\pgfsetlinewidth{1.505625pt}%
\definecolor{currentstroke}{rgb}{0.000000,0.000000,1.000000}%
\pgfsetstrokecolor{currentstroke}%
\pgfsetdash{}{0pt}%
\pgfpathmoveto{\pgfqpoint{3.809480in}{0.189659in}}%
\pgfpathlineto{\pgfqpoint{3.749723in}{0.197538in}}%
\pgfpathlineto{\pgfqpoint{3.666098in}{0.211530in}}%
\pgfpathlineto{\pgfqpoint{3.614857in}{0.221489in}}%
\pgfpathlineto{\pgfqpoint{3.540572in}{0.237360in}}%
\pgfpathlineto{\pgfqpoint{3.440364in}{0.261328in}}%
\pgfpathlineto{\pgfqpoint{3.361300in}{0.281831in}}%
\pgfpathlineto{\pgfqpoint{3.241785in}{0.315331in}}%
\pgfpathlineto{\pgfqpoint{3.122270in}{0.351365in}}%
\pgfpathlineto{\pgfqpoint{2.999354in}{0.390802in}}%
\pgfpathlineto{\pgfqpoint{2.881491in}{0.430640in}}%
\pgfpathlineto{\pgfqpoint{2.741374in}{0.480438in}}%
\pgfpathlineto{\pgfqpoint{2.634000in}{0.520276in}}%
\pgfpathlineto{\pgfqpoint{2.494818in}{0.574006in}}%
\pgfpathlineto{\pgfqpoint{2.345425in}{0.634302in}}%
\pgfpathlineto{\pgfqpoint{2.196031in}{0.697310in}}%
\pgfpathlineto{\pgfqpoint{2.055111in}{0.759306in}}%
\pgfpathlineto{\pgfqpoint{1.924571in}{0.819063in}}%
\pgfpathlineto{\pgfqpoint{1.799004in}{0.878820in}}%
\pgfpathlineto{\pgfqpoint{1.678285in}{0.938578in}}%
\pgfpathlineto{\pgfqpoint{1.562390in}{0.998335in}}%
\pgfpathlineto{\pgfqpoint{1.449064in}{1.059381in}}%
\pgfpathlineto{\pgfqpoint{1.345468in}{1.117850in}}%
\pgfpathlineto{\pgfqpoint{1.261278in}{1.167648in}}%
\pgfpathlineto{\pgfqpoint{1.180156in}{1.218021in}}%
\pgfpathlineto{\pgfqpoint{1.105194in}{1.267243in}}%
\pgfpathlineto{\pgfqpoint{1.047930in}{1.307081in}}%
\pgfpathlineto{\pgfqpoint{0.994071in}{1.346920in}}%
\pgfpathlineto{\pgfqpoint{0.941127in}{1.389255in}}%
\pgfpathlineto{\pgfqpoint{0.909498in}{1.416637in}}%
\pgfpathlineto{\pgfqpoint{0.877844in}{1.446515in}}%
\pgfpathlineto{\pgfqpoint{0.849538in}{1.476394in}}%
\pgfpathlineto{\pgfqpoint{0.825362in}{1.506273in}}%
\pgfpathlineto{\pgfqpoint{0.812057in}{1.526192in}}%
\pgfpathlineto{\pgfqpoint{0.801404in}{1.546111in}}%
\pgfpathlineto{\pgfqpoint{0.793961in}{1.566030in}}%
\pgfpathlineto{\pgfqpoint{0.791733in}{1.576404in}}%
\pgfpathlineto{\pgfqpoint{0.790908in}{1.585949in}}%
\pgfpathlineto{\pgfqpoint{0.791733in}{1.597154in}}%
\pgfpathlineto{\pgfqpoint{0.793897in}{1.605868in}}%
\pgfpathlineto{\pgfqpoint{0.798639in}{1.615828in}}%
\pgfpathlineto{\pgfqpoint{0.806261in}{1.625787in}}%
\pgfpathlineto{\pgfqpoint{0.821612in}{1.638431in}}%
\pgfpathlineto{\pgfqpoint{0.835712in}{1.645706in}}%
\pgfpathlineto{\pgfqpoint{0.851491in}{1.651799in}}%
\pgfpathlineto{\pgfqpoint{0.881369in}{1.659095in}}%
\pgfpathlineto{\pgfqpoint{0.911248in}{1.663166in}}%
\pgfpathlineto{\pgfqpoint{0.941127in}{1.665164in}}%
\pgfpathlineto{\pgfqpoint{0.971005in}{1.665609in}}%
\pgfpathlineto{\pgfqpoint{1.030763in}{1.663165in}}%
\pgfpathlineto{\pgfqpoint{1.090520in}{1.657618in}}%
\pgfpathlineto{\pgfqpoint{1.150277in}{1.649739in}}%
\pgfpathlineto{\pgfqpoint{1.233902in}{1.635747in}}%
\pgfpathlineto{\pgfqpoint{1.285143in}{1.625787in}}%
\pgfpathlineto{\pgfqpoint{1.359428in}{1.609917in}}%
\pgfpathlineto{\pgfqpoint{1.459636in}{1.585949in}}%
\pgfpathlineto{\pgfqpoint{1.538700in}{1.565446in}}%
\pgfpathlineto{\pgfqpoint{1.658215in}{1.531945in}}%
\pgfpathlineto{\pgfqpoint{1.777730in}{1.495912in}}%
\pgfpathlineto{\pgfqpoint{1.900646in}{1.456475in}}%
\pgfpathlineto{\pgfqpoint{2.018509in}{1.416637in}}%
\pgfpathlineto{\pgfqpoint{2.158626in}{1.366839in}}%
\pgfpathlineto{\pgfqpoint{2.266000in}{1.327001in}}%
\pgfpathlineto{\pgfqpoint{2.405182in}{1.273271in}}%
\pgfpathlineto{\pgfqpoint{2.554575in}{1.212975in}}%
\pgfpathlineto{\pgfqpoint{2.703969in}{1.149966in}}%
\pgfpathlineto{\pgfqpoint{2.844889in}{1.087971in}}%
\pgfpathlineto{\pgfqpoint{2.975429in}{1.028214in}}%
\pgfpathlineto{\pgfqpoint{3.100996in}{0.968456in}}%
\pgfpathlineto{\pgfqpoint{3.221715in}{0.908699in}}%
\pgfpathlineto{\pgfqpoint{3.337610in}{0.848942in}}%
\pgfpathlineto{\pgfqpoint{3.450936in}{0.787896in}}%
\pgfpathlineto{\pgfqpoint{3.554532in}{0.729427in}}%
\pgfpathlineto{\pgfqpoint{3.638722in}{0.679629in}}%
\pgfpathlineto{\pgfqpoint{3.719844in}{0.629256in}}%
\pgfpathlineto{\pgfqpoint{3.794806in}{0.580034in}}%
\pgfpathlineto{\pgfqpoint{3.852070in}{0.540195in}}%
\pgfpathlineto{\pgfqpoint{3.905929in}{0.500357in}}%
\pgfpathlineto{\pgfqpoint{3.958873in}{0.458022in}}%
\pgfpathlineto{\pgfqpoint{3.990502in}{0.430640in}}%
\pgfpathlineto{\pgfqpoint{4.022156in}{0.400762in}}%
\pgfpathlineto{\pgfqpoint{4.050462in}{0.370883in}}%
\pgfpathlineto{\pgfqpoint{4.074638in}{0.341004in}}%
\pgfpathlineto{\pgfqpoint{4.087943in}{0.321085in}}%
\pgfpathlineto{\pgfqpoint{4.098596in}{0.301166in}}%
\pgfpathlineto{\pgfqpoint{4.106039in}{0.281247in}}%
\pgfpathlineto{\pgfqpoint{4.108267in}{0.270872in}}%
\pgfpathlineto{\pgfqpoint{4.109092in}{0.261328in}}%
\pgfpathlineto{\pgfqpoint{4.108267in}{0.250123in}}%
\pgfpathlineto{\pgfqpoint{4.106103in}{0.241409in}}%
\pgfpathlineto{\pgfqpoint{4.101361in}{0.231449in}}%
\pgfpathlineto{\pgfqpoint{4.093739in}{0.221489in}}%
\pgfpathlineto{\pgfqpoint{4.078388in}{0.208846in}}%
\pgfpathlineto{\pgfqpoint{4.064288in}{0.201570in}}%
\pgfpathlineto{\pgfqpoint{4.048509in}{0.195478in}}%
\pgfpathlineto{\pgfqpoint{4.018631in}{0.188182in}}%
\pgfpathlineto{\pgfqpoint{3.988752in}{0.184110in}}%
\pgfpathlineto{\pgfqpoint{3.958873in}{0.182113in}}%
\pgfpathlineto{\pgfqpoint{3.928995in}{0.181668in}}%
\pgfpathlineto{\pgfqpoint{3.869237in}{0.184112in}}%
\pgfpathlineto{\pgfqpoint{3.809480in}{0.189659in}}%
\pgfpathlineto{\pgfqpoint{3.809480in}{0.189659in}}%
\pgfusepath{stroke}%
\end{pgfscope}%
\begin{pgfscope}%
\pgfsetrectcap%
\pgfsetmiterjoin%
\pgfsetlinewidth{0.803000pt}%
\definecolor{currentstroke}{rgb}{0.000000,0.000000,0.000000}%
\pgfsetstrokecolor{currentstroke}%
\pgfsetdash{}{0pt}%
\pgfpathmoveto{\pgfqpoint{2.450000in}{0.151972in}}%
\pgfpathlineto{\pgfqpoint{2.450000in}{1.695305in}}%
\pgfusepath{stroke}%
\end{pgfscope}%
\begin{pgfscope}%
\pgfsetrectcap%
\pgfsetmiterjoin%
\pgfsetlinewidth{0.803000pt}%
\definecolor{currentstroke}{rgb}{0.000000,0.000000,0.000000}%
\pgfsetstrokecolor{currentstroke}%
\pgfsetdash{}{0pt}%
\pgfpathmoveto{\pgfqpoint{0.135000in}{0.923638in}}%
\pgfpathlineto{\pgfqpoint{4.765000in}{0.923638in}}%
\pgfusepath{stroke}%
\end{pgfscope}%
\end{pgfpicture}%
\makeatother%
\endgroup%

            \end{center}
    \end{enumerate}
\end{ej}

\begin{ej}
    \begin{enumerate}[(a)]
        \item Comprobaremos las propiedades para $\bar{d}$:
            \begin{enumerate}[i)]
                \item $\bar{d}(x,y) \geq 0$ trivial por la definición.
                \item $\bar{d}(x,y) = 0 \iff \bar{d}(x,y) = d(x,y) = 0 \iff x = y$.
                \item $\bar{d}(x,y) = \bar{d}(y,x)$ trivial por la definición.
                \item Si $d(x,y) < 1$ entonces $\bar{d}(x,y) = d(x,y) \leq d(x,z) + d(z,y) \leq \bar{d}(x,z) + \bar{d}(z,y)$. Si $d(x,y) \geq 1$,
                    entonces $\bar{d}(x,y) = 1$, y $\bar{d}(x,z) + \bar{d}(z,y)$ es mayor o igual que 1, ya que, si $d(x,z) \geq 1$ o
                    $d(z,y) \geq 1$ ya se cumple la desigualdad. En otro caso, se tiene que $\bar{d}(x,y) = 1 \leq d(x,y) \leq d(x,z) + d(z,y) =
                    \bar{d}(x,z) + \bar{d}(z,y)$.
            \end{enumerate}
        \item Comprobamos las propiedades
            \begin{enumerate}[i)]
                \item $\bar{d}(x,y) \geq 0$ división de dos números positivos.
                \item $\bar{d}(x,y) = 0 \iff d(x,y) = 0 \iff x = y$.
                \item $\bar{d}(x,y) = \bar{d}(y,x)$ trivial por la definición.
                \item 
                    \begin{align*}
                        \bar{d}(x,y) &= \frac{d(x,y)}{1 + d(x,y)} \leq \frac{d(x,z)}{1 + d(x,z)} + \frac{d(z,y)}{1 + d(z,y)} =
                        \bar{d}(x,z) + \bar{d}(z,y) \iff \\
                        &\iff \frac{d(x,y)}{1+d(x,y)} \leq \frac{d(x,z) + d(z,y) + 2d(x,z)d(z,y)}{1 + d(x,z) + d(z,y) + d(x,z)d(z,y)} \\
                        &\iff d(x,y) + \cancel{d(x,y)d(x,z)} + \cancel{d(x,y)d(z,y)} + \cancel{d(x,y)d(x,z)d(z,y)} \leq \\
                        &\qquad\qquad\leq d(x,z) + d(z,y) + 2d(x,z)d(z,y) + \cancel{d(x,y)d(x,z)} \,\,+ \\
                            &\qquad\qquad\qquad\qquad +\cancel{d(x,y)d(z,y)} + \cancel{2}d(x,y)d(x,z)d(z,y) \\
                        &\iff d(x,y) \leq d(x,z) + d(z,y) + 2d(x,z)d(z,y) + d(x,y)d(x,z)d(z,y).
                    \end{align*}
                    Lo cual es cierto ya que $d$ es una distancia.
            \end{enumerate}
        \item De nuevo, comprobamos
            \begin{enumerate}[i)]
                \item $\bar{d}(x,y) \geq 0 \iff d(x,y) \geq \frac{0}{\alpha}$ OK.
                \item $\bar{d}(x,y) = 0 \iff d(x,y) = 0 \iff x = y$.
                \item $\bar{d}(x,y) = \bar{d}(y,x)$ trivial.
                \item $\bar{d}(x,y) = \alpha d(x,y) \leq \alpha \left( d(x,z) + d(z,y) \right) = \alpha d(x,z) + \alpha d(z,y) = \bar{d}(x,z) + \bar{d}(z,y)$.
            \end{enumerate}
        \item Comprobamos
            \begin{enumerate}[i)]
                \item $\bar{d}(x,y) \geq 0$ inmediato.
                \item $\bar{d}(x,y) = 0 \iff d(x,y) = 0 \iff x = y$.
                \item $\bar{d}(x,y) = \bar{d}(y,x)$ trivial.
                \item $\bar{d}(x,y) = \left( d(x,y) \right)^c \leq \left( d(x,z) + d(z,y) \right)^c \leq d(x,z)^c + d(z,y)^c$, ya que la función $f(x) = x^c$ es
                    cóncava.
            \end{enumerate}
    \end{enumerate}
\end{ej}

\begin{ej}
    \begin{enumerate}[(a)]
        \item \label{item:conc_a} Como $f$ es cóncava, cumple la desigualdad $\alpha f(z) + (1-\alpha)f(t) \leq f\left( \alpha z + (1-\alpha)t \right)$, tomamos ahora
            $z = 0$ y $t=x+y$, entonces, substituyendo para unos valores concretos de $\alpha$:
            \begin{gather*}
                \begin{rcases}
                    \alpha = \frac{x}{x+y} \implies \frac{x}{x+y}f(0) + \frac{y}{x+y}f(x+y) \leq f\left( \frac{x}{x+y} 0 + \frac{y}{x+y}(x+y)\right) =f(y)\\
                    \alpha = \frac{y}{x+y} \implies \frac{y}{x+y}f(0) + \frac{x}{x+y}f(x+y) \leq f\left( \frac{y}{x+y} 0 + \frac{x}{x+y}(x+y)\right) =f(x)
                \end{rcases}
                \implies\\ \implies
                f(0) + f(x+y) \leq f(x) + f(y) \stackrel{f(0) \geq 0}{\implies} f(x+y) \leq f(x) + f(y)
            \end{gather*}
        \item Comporbamos las propiedades
            \begin{enumerate}[i)]
                \item $f$ creciente $\implies f(x) \geq f(0) = 0 \implies (f \circ d)(x,y) \geq 0$.
                \item $(f \circ d)(x,y) = 0 \iff d(x,y) = 0 \iff x = y$.
                \item $(f \circ d)(x,y) = f\left( d(x,y) \right) = f\left( d(y,x) \right) = (f \circ d)(y,x)$.
                \item $(f \circ d)(x,y) = f\left( d(x,y) \right) \leq f\left( d(x,z) + d(z,y) \right) \stackrel{\ref{item:conc_a}}{\leq}f\left( d(x,z) \right) +
                    f\left( d(z,y) \right) = (f \circ d)(x,z) + (f \circ d)(z,y)$.
            \end{enumerate}
    \end{enumerate}
\end{ej}

\begin{ej}
    Veremos que $(a) \implies (b)$. Sea $U \subset X$ un abierto de $(X, d_2)$, entonces $\Id^{-1}(U) = U$, que es un abierto en $(X,d_1)$ y análogamente con $\Id^{-1}$.
    
    Veremos ahora que $(b) \implies (c)$. Sea $B_1(x, r)$ una bola abierta según la m\'etrica $d_1$ de centro $x$, entonces, por $(b)$, $\Id^{-1}\left( B_1(x,r) \right)$
    es un abierto, y $x \in \Id^{-1}\left( B_1(x,r) \right)$. Por lo tanto, $\exists s \tq B_2(x,s) \subset \Id^{-1}\left( B_1(x,r) \right) = B_1(x,r)$. (la otra
    implicación es análoga).

    Por último, comprobaremos que $(c) \implies (a)$. Sea $U$ un abierto en $(X,d_1)$, entonces $\forall x \in U \; \exists r \tq B_1(x,r) \subset U$, pero por $(c)$,
    $\exists s \tq x \in B_2(x,s) \subset B_1(x,r) \subset U \implies x$ es un punto interior en $(X, d_2)$, por lo tanto, $U$ es abierto en $(X, d_2)$ y la otra
    implicación se demuestra de forma análoga.
\end{ej}

\begin{ej}
    Sean $d_1, d_2$ dos m\'etricas fuertemente equivalentes. Sean $r \in \real$ y $x \in X$, entonces, $\exists s = \frac{r}{M} \in \real \tq B_2(x, s) \subset B_1(x, r)$.
    De nuevo, la otra implicación es análoga.

    Las m\'etricas $d_1(x,y) = \abs{x-y}$ y $d_2(x,y) = \abs{\frac{1}{x} - \frac{1}{y}}$ son equivalentes, pero no son fuertemente equivalentes.
\end{ej}

\begin{ej}
    \begin{enumerate}[(a)]
        \item Comprobemos que $D_1$ satisface las propiedades de las métricas. Por ser $d_1$ y $d_2$ métricas,
        \begin{enumerate}[i)]
            \item 
            \[
                D_1 \left( \left( x_1, x_2 \right) , \left( y_1 , y_2 \right) \right) = d_1 \left( x_1, y_1\right) + d_2 \left( x_2 , y_2 \right) \geq 0.
            \]
            \item 
            \begin{gather*}
                D_1 \left( \left( x_1, x_2 \right) , \left( y_1 , y_2 \right) \right) = d_1 \left( x_1, y_1\right) + d_2 \left( x_2 , y_2 \right) = 0 \iff \\
                \iff \left\{ \begin{array}{c}
                    d_1 \left( x_1, y_1 \right) = 0 \\
                    d_2 \left( x_2, y_2 \right) = 0
                \end{array} \right\} \iff \left\{ \begin{array}{c}
                    x_1=y_1 \\
                    x_2=y_2
                \end{array} \right\} \iff \left( x_1, y_1 \right) = \left( x_2, y_2 \right).
            \end{gather*}
            \item 
            \begin{gather*}
                D_1 \left( \left( x_1, x_2 \right) , \left( y_1 , y_2 \right) \right) = d_1 \left( x_1, y_1 \right) + d_2 \left( x_2 , y_2 \right) = d_1 \left( y_1, x_1 \right) + d_2 \left( y_2 , x_2 \right)= \\
                =D_1 \left( \left( y_1, y_2 \right) , \left( x_1 , x_2 \right) \right) .
            \end{gather*}
            \item 
            \begin{gather*}
                D_1 \left( \left( x_1, x_2 \right) , \left( y_1 , y_2 \right) \right) = d_1 \left( x_1, y_1 \right) + d_2 \left( x_2 , y_2 \right) \leq d_1 \left( x_1, z_1 \right) + d_1 \left( z_1, y_1 \right) + \\
                + d_2 \left( x_2, z_2 \right) + d_2 \left( z_2, y_2 \right) = D_1 \left( \left( x_1, x_2 \right) , \left( z_1 , z_2 \right) \right) + D_1 \left( \left( z_1, z_2 \right) , \left( y_1 , y_2 \right) \right).
            \end{gather*}
        \end{enumerate}
        \item Comprobemos que $D_2$ satisface las propiedades de las métricas. Por ser $d_1$ y $d_2$ métricas,
        \begin{enumerate}[i)]
            \item
            \[
                D_2 \left( \left( x_1, x_2 \right) , \left( y_1 , y_2 \right) \right) = \sqrt{ d_1 \left( x_1, y_1\right) ^2 + d_2 \left( x_2 , y_2 \right)^2 } \geq 0.
            \]
            \item 
            \begin{gather*}
                D_2 \left( \left( x_1, x_2 \right) , \left( y_1 , y_2 \right) \right) = \sqrt{ d_1 \left( x_1, y_1\right) ^2 + d_2 \left( x_2 , y_2 \right)^2 } = 0 \iff \\
                \iff \left\{ \begin{array}{c}
                    d_1 \left( x_1, y_1 \right) = 0 \\
                    d_2 \left( x_2, y_2 \right) = 0
                \end{array} \right\} \iff  \left\{ \begin{array}{c}
                    x_1=y_1 \\
                    x_2=y_2
                \end{array} \right\} \iff \left( x_1, y_1 \right) = \left( x_2, y_2 \right).
            \end{gather*}
            \item 
            \begin{gather*}
                D_2 \left( \left( x_1, x_2 \right) , \left( y_1 , y_2 \right) \right) = \sqrt{ d_1 \left( x_1, y_1\right) ^2 + d_2 \left( x_2 , y_2 \right)^2 } = \\ 
                = \sqrt{ d_1 \left( y_1, x_1\right) ^2 + d_2 \left( y_2 , x_2 \right)^2 } = D_2 \left( \left( y_1, y_2 \right) , \left( x_1 , x_2 \right) \right) .
            \end{gather*}
            \item Primero, demostraremos un resultado auxiliar. Para cualesquiera reales positivos $a, b, c, d$ se tiene que
            \begin{align} \nonumber
                0 \leq \left( ad -cb \right) ^2 &\implies 2abcd \leq a^2d^2+c^2b^2 \\  \nonumber
                &\implies \left( ab+cd \right) ^2 = a^2b^2 + 2abcd + c^2d^2 \\ \nonumber
                    &\qquad\qquad\qquad\quad \leq a^2b^2+a^2d^2+c^2b^2+c^2d^2 \\ \nonumber
                    &\qquad\qquad\qquad\quad = \left( a^2+c^2 \right) \left( b^2+d^2 \right) \\ \nonumber
                &\implies \left( a+b \right) ^2 + \left( c+d\right) ^2 = a^2+b^2+c^2+d^2 +2ab +2cd \\ \nonumber
                &\qquad \qquad \leq a^2+b^2+c^2+d^2+2\sqrt{\left( a^2+c^2\right) \left( b^2+d^2\right)} \\ \nonumber
                &\implies\left( \sqrt{\left( a+b \right) ^2 + \left( c+d \right) ^2} \right) ^2 \leq \left( \sqrt{a^2+c^2} + \sqrt{b^2+d^2} \right) ^2 \\
                \label{item:11.a.lema} &\implies \sqrt{\left( a+b \right) ^2 + \left( c+d \right) ^2} \leq \sqrt{a^2+c^2} + \sqrt{b^2+d^2}.
            \end{align}
            Entonces,
            \begin{align*}
                D_2 \left( \left( x_1, x_2 \right) , \left( y_1 , y_2 \right) \right) &= \sqrt{ d_1 \left( x_1, y_1\right) ^2 + d_2 \left( x_2 , y_2 \right)^2 } \\
                &\leq \sqrt{ \left( d_1 \left( x_1, z_1\right) + d_1\left( z_1, y_1 \right) \right) ^2 + \left( d_2 \left( x_2, z_2 \right) + d_2\left( z_2, y_2 \right) \right) ^2 }\\
                & \stackrel{\eqref{item:11.a.lema}}{\leq} \sqrt{d_1\left( x_1, z_1 \right) ^2+d_2 \left( x_2, z_2 \right) ^2} + \sqrt{d_1\left( z_1, y_1 \right) ^2+d_2 \left( z_2, y_2 \right) ^2}\\
                &= D_2 \left( \left( x_1, x_2 \right) , \left( z_1 , z_2 \right) \right) + D_2 \left( \left( z_1, z_2 \right) , \left( y_1 , y_2 \right) \right).
            \end{align*}            
        \end{enumerate}
        \item Comprobemos que $D_{\infty}$ satisface las propiedades de las métricas. Por ser $d_1$ y $d_2$ métricas,
        \begin{enumerate}[i)]
            \item
            \[
                D_{\infty} \left( \left( x_1, x_2 \right) , \left( y_1 , y_2 \right) \right) = \max \left\{ d_1 \left( x_1, y_1\right) , d_2 \left( x_2 , y_2 \right) \right\} \geq 0.
            \]
            \item 
            \begin{gather*}
                D_{\infty} \left( \left( x_1, x_2 \right) , \left( y_1 , y_2 \right) \right) = \max \left\{ d_1 \left( x_1, y_1\right) , d_2 \left( x_2 , y_2 \right) \right\} = 0  \iff \\
                \iff \left\{ \begin{array}{c}
                    d_1 \left( x_1, y_1 \right) = 0 \\
                    d_2 \left( x_2, y_2 \right) = 0
                \end{array} \right\} \iff  \left\{ \begin{array}{c}
                    x_1=y_1 \\
                    x_2=y_2
                \end{array} \right\} \iff \left( x_1, y_1 \right) = \left( x_2, y_2 \right).
            \end{gather*}
            \item 
            \begin{gather*}
                D_{\infty} \left( \left( x_1, x_2 \right) , \left( y_1 , y_2 \right) \right) = \max \left\{ d_1 \left( x_1, y_1\right)  , d_2 \left( x_2 , y_2 \right) \right\} = \\
                = \max \left\{ d_1 \left( y_1, x_1\right) , d_2 \left( y_2 , x_2 \right) \right\} =D_2 \left( \left( y_1, y_2 \right) , \left( x_1 , x_2 \right) \right) .
            \end{gather*}
            \item 
            \begin{align*}
                D_{\infty} \left( \left( x_1, x_2 \right) , \left( y_1 , y_2 \right) \right) &= \max \left\{ d_1 \left( x_1, y_1\right) , d_2 \left( x_2 , y_2 \right) \right\} \\
                &\leq \max \left\{ d_1 \left( x_1, z_1 \right) +d_1 \left( z_1, y_1 \right) , d_1 \left( x_2, z_2 \right) +d_2 \left( z_2, y_2 \right)  \right\} \\
                &\leq \max \left\{ d_1 \left( x_1, z_1\right) , d_2 \left( x_2 , z_2 \right) \right\} + \\
                &\qquad\qquad + \max \left\{ d_1 \left( z_1, y_1\right) , d_2 \left( z_2 , y_2 \right) \right\} \\
                &= D_{\infty} \left( \left( x_1, x_2 \right) , \left( z_1 , z_2 \right) \right) + D_{\infty} \left( \left( z_1, z_2 \right) , \left( y_1 , y_2 \right) \right).
            \end{align*}
        \end{enumerate}
    \end{enumerate}
    Veamos ahora que $D_1$ y $D_2$ son métricas fuertemente equivalentes. Observamos que
    \[
        D_1 \left( \left( x_1, x_2 \right) , \left( y_1 , y_2 \right) \right)^2-D_2 \left( \left( x_1, x_2 \right) , \left( y_1 , y_2 \right) \right)^2=2d_1\left( x_1, y_1 \right) d_2 \left( x_2, y_2 \right).
    \]
    Entonces,
    \[
        2d_1\left( x_1, y_1 \right) d_2 \left( x_2, y_2 \right) \geq 0 \implies D_2 \left( \left( x_1, x_2 \right) , \left( y_1 , y_2 \right) \right) \leq D_1 \left( \left( x_1, x_2 \right) , \left( y_1 , y_2 \right) \right).
    \]
    Y también
    \begin{gather*}
        2d_1\left( x_1, y_1 \right) d_2 \left( x_2, y_2 \right) \leq d_1\left( x_1, y_1 \right) ^2 + d_2 \left( x_2, y_2 \right) ^2 = D_2 \left( \left( x_1, x_2 \right) , \left( y_1 , y_2 \right) \right)^2 \implies \\
        \implies D_1 \left( \left( x_1, x_2 \right) , \left( y_1 , y_2 \right) \right) \leq \sqrt{2} D_2 \left( \left( x_1, x_2 \right) , \left( y_1 , y_2 \right) \right).
    \end{gather*}
    Por lo tanto,
    \[
        D_2 \left( \left( x_1, x_2 \right) , \left( y_1 , y_2 \right) \right) \leq D_1 \left( \left( x_1, x_2 \right) , \left( y_1 , y_2 \right) \right) \leq \sqrt{2} D_2 \left( \left( x_1, x_2 \right) , \left( y_1 , y_2 \right) \right),
    \]
    con lo que $D_1$ y $D_2$ son métricas fuertemente equivalentes. Veamos ahora que $D_1$ y $D_{\infty}$ también lo son. Es trivial comprobar que
    \[
        D_{\infty} \left( \left( x_1, x_2 \right) , \left( y_1 , y_2 \right) \right) \leq D_1 \left( \left( x_1, x_2 \right) , \left( y_1 , y_2 \right) \right) \leq 2 D_{\infty} \left( \left( x_1, x_2 \right) , \left( y_1 , y_2 \right) \right).
    \]
    Finalmente, como que la equivalencia fuerte es una propiedad transitiva, $D_2$ y $D_{\infty}$ son métricas fuertemente equivalentes.
    
    Para acabar, mostraremos la generalización de estos resultados. Sean $\left( X_i, d_i \right)$ espacios métricos para todo $i \in \left\{1, \dots , n \right\}$, y sea $X=X_1 \times \cdots \times X_n$. Definimos en $X$ las aplicaciones
    \begin{enumerate}[(a)]
        \item $D_1\left( \left( x_1, y_1 \right), \dots , \left( x_n, y_n \right) \right) = d_1\left( x_1, y_1 \right) + \cdots + d_n \left( x_n, y_n \right).$
        \item $D_2\left( \left( x_1, y_1 \right), \dots , \left( x_n, y_n \right) \right) = \sqrt{d_1\left( x_1, y_1 \right)^2 + \cdots + d_n \left( x_n, y_n \right)^2}.$
        \item $D_{\infty}\left( \left( x_1, y_1 \right), \dots , \left( x_n, y_n \right) \right) = \max \left\{d_1\left( x_1, y_1 \right), \dots, d_n \left( x_n, y_n \right) \right\}.$
    \end{enumerate}
    A través de un razonamiento inductivo inmediato a partir de lo anteriormente visto, obtenemos que $D_1$, $D_2$ y $D_{\infty}$ son métricas en $X$ y fuertemente equivalentes entre ellas.
\end{ej}

\begin{ej}
    Sea $x= \lp x_1, x_2 \rp \in X \times X$ y sean $\varepsilon > 0, \, \delta=\varepsilon$. Entonces, $\forall y = \lp y_1, y_2 \rp \in X \times X$ con $D_1 \lp x, y \rp < \delta$,
    \begin{align*}
        \left| d\lp x_1, x_2 \rp - d\lp y_1, y_2 \rp \right| &\leq \left| d\lp x_1, x_2 \rp -d\lp x_2, y_2 \rp + d \lp x_2, y_1 \rp - d\lp y_1, y_2 \rp \right| \\
        &\leq \left| d\lp x_1, x_2 \rp -d\lp x_2, y_2 \rp \right| + \left| d \lp x_2, y_1 \rp - d\lp y_1, y_2 \rp \right| \\
        &\leq d\lp x_1, y_1 \rp + d\lp x_2, y_2 \rp = D_1 \lp x, y \rp \\
        &< \delta = \varepsilon.
    \end{align*}
    Así pues, la aplicación $d \colon X \times X \to \real$ es continua tomando en $X \times X$ la métrica $D_1$. Por ser $D_1$, $D_2$ y $D_{\infty}$ métricas fuertemente equivalentes, $d$ también es continua si tomamos en $X \times X$ las métricas $D_2$ o $D_{\infty}$.
\end{ej}
