\chapter{Espacios topológicos y aplicaciones continuas}
%\section{Distancias}

\begin{eje}
    \begin{enumerate}[(a)]
        \item[]
        \item Comprobamos que
            \begin{enumerate}[i)]
                \item El intervalo $\lp a, a \rp = \emptyset \in \T$ y $\lp -\infty, +\infty \rp = X \in \T$.
                \item Sea $U = \bigcup\limits_{i \in I} U_i$ la unión de un numero arbitrario de abiertos, entonces, $\forall x \in U, \exists i \in I \tq x \in U_i \subseteq U$ y por tanto $x$ es un punto interior y $U$ es un abierto.
                \item Sea $U = \bigcap\limits_{i = 1}^n U_i$ la intersección de un numero finito de abiertos, entonces, $\forall x \in U, \forall i \in \left\{ 1, \dots, n \right\}, \exists a_i, b_i \tq x \in \left(a_i, b_i\right) \subseteq U_i$. Sean
                    \begin{gather*}
                        a = \max_{i \in \left\{ 1, \dots, n \right\}} \left\{a_i\right\}, \\
                        b = \min_{i \in \left\{ 1, \dots, n \right\}} \left\{b_i\right\},
                    \end{gather*}
                entonces $x \in \lp a, b \rp \subseteq \bigcap\limits_{i = 1}^n U_i$ y por tanto $x$ es un punto interior y $U$ es un abierto.
            \end{enumerate}
        \item Sea $\B = \left\{ \lp a, b \rp \colon a, b \in X \cup \left\{ \pm \infty \right\} \right\}$, comprobamos que
            \begin{enumerate}[i)]
                \item $\forall x \in X, x \in \lp -\infty, +\infty \rp \in \B$.
                \item $\forall a, b, c, d \in X \cup \left\{ \pm \infty \right\}$, si $\lp a, b \rp \cap \lp c, d \rp \neq \emptyset$, entonces,
                    \begin{gather*}
                        \alpha = \max \left\{ a, c \right\}, \\
                        \beta = \min \left\{ b, d \right\}, \\
                        \lp a, b \rp \cap \lp c, d \rp = \lp \alpha, \beta \rp.
                    \end{gather*}
                    y por tanto $\forall x \in \lp a, b \rp \cap \lp c, d \rp, x \in \lp \alpha, \beta \rp \subseteq \lp a, b \rp \cap \lp c, d \rp$, y $\lp \alpha, \beta \rp \in \B$.
            \end{enumerate}
        \item Si vemos que $\forall x \in X, \left\{ x \right\}$ es un abierto, ya abremos acabado, ya que todo conjunto de $\Pa \lp X \rp$ contiene únicamente puntos de $X$ y por lo tanto sera la unión de abiertos. En $\z$, $\forall x \in \z$, tenemos que $\lc x \rc = \lp n-1, n+1 \rp$ y por tanto ya estamos. En $\n$, $\forall x \in \n \setminus \lc 1 \rc$, tenemos que $\lc x \rc = \lp n-1, n+1 \rp$ y $\lc 1 \rc = \lp -\infty, 2 \rp$ y por tanto ya estamos, suponiendo que $0 \notin \n$.
        \item Sea $X$ un espacio topológico con la topología del orden, entonces $\forall x, y \in X, x < y$,
            \begin{itemize}
                \item Si $\exists z$ tal que $x < z < y$, entonces $x \in \lp -\infty, z \rp, y \in \lp z, +\infty \rp, \lp -\infty, z \rp \cap \lp z, +\infty \rp = \emptyset$.
                \item Si $\nexists z$ tal que $x < z < y$, entonces $x \in \lp -\infty, y \rp, y \in \lp x, +\infty \rp, \lp -\infty, y \rp \cap \lp x, +\infty \rp = \emptyset$.
            \end{itemize}
        \item Aquest dibuix està en contrucció. % TODO dibuix!!
        \item Tenemos que ver que $\T_{\text{ord}} \subset \T_\leq$, es decir, $\T_{\text{ord}} \subseteq \T_\leq$ y $\T_\leq \neq \T_{\text{ord}}$.
            \begin{itemize}
                \item Veamos que $\T_{\text{ord}} \subseteq \T_\leq$. Sea $U \subseteq \T_{\text{ord}},\, \forall x \equiv \lp x_1, x_2 \rp \in U, \, \exists r \in \real^+ \tq B_r \lp x \rp \subseteq U$, y por tanto, $A = \lp \lp x_1 - r, x_2 \rp, \lp x_1 + r, x_2 \rp \rp \subseteq B_r \lp x \rp, A \in \T_\leq$. Así pues, todos los puntos de $U$ son interiores en la topología del orden y por tanto $U \in \T_\leq$.
                \item Veamos que $\T_\leq \neq \T_{\text{ord}}$. Sean $x = \lp 0,0 \rp, y = \lp 0, 1 \rp$, entonces $\lp x, y \rp \in \T_\leq, \lp x, y \rp \notin \T_{\text{ord}}$. Por tanto $\T_\leq \neq \T_{\text{ord}}$.
            \end{itemize}
        \item Son las topologías discretas.
    \end{enumerate}
\end{eje}
