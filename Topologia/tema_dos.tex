\chapter{Espacios topológicos y aplicaciones continuas}
\section{Distancias}

\begin{eje}
    \begin{enumerate}[(a)]
        \item[]
        \item Comprobamos que
            \begin{enumerate}[i)]
                \item El intervalo $\lp a, a \rp = \emptyset \in \mathcal{T}$ y $\lp -\infty, +\infty \rp = X \in \mathcal{T}$.
                \item Sea $U = \bigcup_{i \in I} U_i$ la unión de un numero arbitrario de abiertos, entonces, $\forall x \in U, \exists i \in I \tq x \in U_i \subseteq U$ y por tanto $x$ es un punto interior y $U$ es un abierto.
                \item Sea $U = \bigcap_{i = 1}^n U_i$ la intersección de un numero finito de abiertos, entonces, $\forall x \in U, \forall i \in \left\{ 1, \dots, n \right\}, \exists a_i, b_i \tq x \in \left(a_i, b_i\right) \subseteq U_i$. Sean
                    \begin{gather*}
                        a = \max_{i \in \left\{ 1, \dots, n \right\}} \left\{a_i\right\}, \\
                        b = \min_{i \in \left\{ 1, \dots, n \right\}} \left\{b_i\right\},
                    \end{gather*}
                entonces $x \in \lp a, b \rp \subseteq \bigcap_{i = 1}^n U_i$ y por tanto $x$ es un punto interior y $U$ es un abierto.
            \end{enumerate}
        \item Hay que ver que $\forall U \in \T, \forall x \in U, \exists B \in \mathcal{B} \tq x \in B \subseteq U$.
    \end{enumerate}
\end{eje}
