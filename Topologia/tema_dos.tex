\chapter{Espacios topológicos y aplicaciones continuas}

\begin{eje}
    \begin{enumerate}[(a)]
        \item[]
        \item Comprobamos que
            \begin{enumerate}[i)]
                \item El intervalo $\lp a, a \rp = \emptyset \in \T$ y $\lp -\infty, +\infty \rp = X \in \T$.
                \item Sea $U = \bigcup\limits_{i \in I} U_i$ la unión de un numero arbitrario de abiertos, entonces, $\forall x \in U, \exists i \in I \tq x \in U_i \subseteq U$ y por tanto $x$ es un punto interior y $U$ es un abierto.
                \item Sea $U = \bigcap\limits_{i = 1}^n U_i$ la intersección de un numero finito de abiertos, entonces, $\forall x \in U, \forall i \in \left\{ 1, \dots, n \right\}, \exists a_i, b_i \tq x \in \left(a_i, b_i\right) \subseteq U_i$, porque $x \in U_i$ y $U_i$ abierto. Sean
                    \begin{gather*}
                        a = \max_{i \in \left\{ 1, \dots, n \right\}} \left\{a_i\right\}, \\
                        b = \min_{i \in \left\{ 1, \dots, n \right\}} \left\{b_i\right\},
                    \end{gather*}
                entonces $x \in \lp a, b \rp \subseteq \bigcap\limits_{i = 1}^n U_i$ y por tanto $x$ es un punto interior y $U$ es un abierto.
            \end{enumerate}
        \item Sea $\B = \left\{ \lp a, b \rp \colon a, b \in X \cup \left\{ \pm \infty \right\} \right\}$, comprobamos que
            \begin{enumerate}[i)]
                \item $\forall x \in X, x \in \lp -\infty, +\infty \rp \in \B$.
                \item $\forall a, b, c, d \in X \cup \left\{ \pm \infty \right\}$, si $\lp a, b \rp \cap \lp c, d \rp \neq \emptyset$, entonces,
                    \begin{gather*}
                        \alpha = \max \left\{ a, c \right\}, \\
                        \beta = \min \left\{ b, d \right\}, \\
                        \lp a, b \rp \cap \lp c, d \rp = \lp \alpha, \beta \rp.
                    \end{gather*}
                    y por tanto $\forall x \in \lp a, b \rp \cap \lp c, d \rp, x \in \lp \alpha, \beta \rp \subseteq \lp a, b \rp \cap \lp c, d \rp$, y $\lp \alpha, \beta \rp \in \B$.
            \end{enumerate}
        \item Si vemos que $\forall x \in X, \left\{ x \right\}$ es un abierto, ya abremos acabado, ya que todo conjunto de $\Pa \lp X \rp$ contiene únicamente puntos de $X$ y por lo tanto sera la unión de abiertos. En $\z$, $\forall x \in \z$, tenemos que $\lc x \rc = \lp n-1, n+1 \rp$ y por tanto ya estamos. En $\n$, $\forall x \in \n \setminus \lc 1 \rc$, tenemos que $\lc x \rc = \lp n-1, n+1 \rp$ y $\lc 1 \rc = \lp -\infty, 2 \rp$ y por tanto ya estamos, suponiendo que $0 \notin \n$.
        \item Sea $X$ un espacio topológico con la topología del orden, entonces $\forall x, y \in X, x < y$,
            \begin{itemize}
                \item Si $\exists z$ tal que $x < z < y$, entonces $x \in \lp -\infty, z \rp, y \in \lp z, +\infty \rp, \lp -\infty, z \rp \cap \lp z, +\infty \rp = \emptyset$.
                \item Si $\nexists z$ tal que $x < z < y$, entonces $x \in \lp -\infty, y \rp, y \in \lp x, +\infty \rp, \lp -\infty, y \rp \cap \lp x, +\infty \rp = \emptyset$.
            \end{itemize}
        \item Aquest dibuix està en contrucció. % TODO dibuix!!
        \item Tenemos que ver que $\T_{\text{ord}} \subset \T_\leq$, es decir, $\T_{\text{ord}} \subseteq \T_\leq$ y $\T_\leq \neq \T_{\text{ord}}$.
            \begin{itemize}
                \item Veamos que $\T_{\text{ord}} \subseteq \T_\leq$. Sea $U \subseteq \T_{\text{ord}},\, \forall x \equiv \lp x_1, x_2 \rp \in U, \, \exists r \in \real^+ \tq B_r \lp x \rp \subseteq U$, y por tanto, $A = \lp \lp x_1 - r, x_2 \rp, \lp x_1 + r, x_2 \rp \rp \subseteq B_r \lp x \rp, A \in \T_\leq$. Así pues, todos los puntos de $U$ son interiores en la topología del orden y por tanto $U \in \T_\leq$.
                \item Veamos que $\T_\leq \neq \T_{\text{ord}}$. Sean $x = \lp 0,0 \rp, y = \lp 0, 1 \rp$, entonces $\lp x, y \rp \in \T_\leq, \lp x, y \rp \notin \T_{\text{ord}}$. Por tanto $\T_\leq \neq \T_{\text{ord}}$.
            \end{itemize}
        \item Son las topologías discretas.
    \end{enumerate}
\end{eje}
\begin{eje}
    \begin{enumerate}[(a)]
        \item[]
        \item Comprobamos que
            \begin{enumerate}[i)]
                \item $\emptyset = \left[ x, x \rp \in \T_\ell$ y $X = \bigcup\limits_{x \in X} \left[ x, \infty \rp \in \T_\ell$.
                \item Sea $U = \bigcup\limits_{i \in I} U_i$ la unión de un numero arbitrario de abiertos, entonces, $\forall x \in U, \exists i \in I \tq x \in U_i \subseteq U$ y por tanto $x$ es un punto interior y $U$ es un abierto.
                \item Sea $U = \bigcap\limits_{i = 1}^n U_i$ la intersección de un numero finito de abiertos, entonces, $\forall x \in U, \forall i \in \left\{ 1, \dots, n \right\}, \exists a_i, b_i \tq x \in \left[a_i, b_i\right) \subseteq U_i$, porque $x \in U_i$ y $U_i$ abierto. Sean
                    \begin{gather*}
                        a = \max_{i \in \left\{ 1, \dots, n \right\}} \left\{a_i\right\}, \\
                        b = \min_{i \in \left\{ 1, \dots, n \right\}} \left\{b_i\right\},
                    \end{gather*}
                entonces $x \in \left[ a, b \rp \subseteq \bigcap\limits_{i = 1}^n U_i$ y por tanto $x$ es un punto interior y $U$ es un abierto.
            \end{enumerate}
        \item Sea $\B = \left\{ \left[ a, b \rp \colon a \in X, b \in X \cup \left\{ \infty \right\} \right\}$, comprobamos que
            \begin{enumerate}[i)]
                \item $\forall x \in X, x \in \left[ x, \infty \rp \in \B$.
                \item $\forall a, b, c, d \in X \cup \left\{ \infty \right\}$, si $\left[ a, b \rp \cap \left[ c, d \rp \neq \emptyset$, entonces,
                    \begin{gather*}
                        \alpha = \max \left\{ a, c \right\}, \\
                        \beta = \min \left\{ b, d \right\}, \\
                        \left[ a, b \rp \cap \left[ c, d \rp = \left[ \alpha, \beta \rp.
                    \end{gather*}
                    y por tanto $\forall x \in \left[ a, b \rp \cap \left[ c, d \rp, x \in \left[ \alpha, \beta \rp \subseteq \left[ a, b \rp \cap \left[ c, d \rp$, y $\left[ \alpha, \beta \rp \in \B$.
            \end{enumerate}
        \item \item[] % OJO amb aqueta cutrada
            \begin{center}
                \begin{tabular}{|l||c|c|c|c|c|c|} \hline
                    & $\lp a, b \rp$ & $\left[ a, b \rp$ & $\lp a, b \right]$ & $\left[ a,b \right]$ & $\lc 0 \rc \cup \lc \sfrac{1}{n} \rc_{n\geq 1}$ & $\lc 0 \rc \cup \lc \sfrac{-1}{n} \rc_{n\geq 1}$ \\ \hline \hline
                    Adherencia & $\left[a,b\rp$ & $\left[ a, b \rp$ & $\left[ a, b \right]$ & $\left[ a, b \right]$ & $\lc 0 \rc \cup \lc \sfrac{1}{n} \rc_{n\geq 1}$ & $\lc 0 \rc \cup \lc \sfrac{-1}{n} \rc_{n\geq 1}$ \\ \hline
                    Interior & $\lp a, b \rp$ & $\left[ a, b \rp$ & $\lp a, b \rp$ & $\left[ a, b \rp$ & $\emptyset$ & $\emptyset$ \\ \hline
                    Frontera & $\lc a \rc$ & $\emptyset$ & $\lc a, b \rc$ & $\lc b \rc$ & $\lc 0 \rc \cup \lc \sfrac{1}{n} \rc_{n\geq 1}$ & $\lc 0 \rc \cup \lc \sfrac{-1}{n} \rc_{n\geq 1}$ \\ \hline
                    Acumulación & $\left[a,b\rp$ & $\left[ a, b \rp$ & $\left[ a, b \rp$ & $\left[ a, b \right)$ & $\lc 0 \rc$ & $\emptyset$ \\ \hline
                    Puntos aislados & $\emptyset$ & $\emptyset$ & $\lc b \rc$ & $\lc b \rc$ & $\lc \sfrac{1}{n} \rc_{n\geq 1}$ & $\lc 0 \rc \cup \lc \sfrac{-1}{n} \rc_{n\geq 1}$ \\ \hline
                \end{tabular}
            \end{center}
    \end{enumerate}
\end{eje}

%TODO aixo sha dacabar (ferran)
\begin{eje}
    \begin{enumerate}[(a)]
        \item[]
        \item Comprobamos que
            \begin{enumerate}[i)]
                \item $\emptyset = \left[ x, x \rp \in \T_\ell$
                \item Sea $U = \bigcup\limits_{i \in I} U_i$ la unión de un numero arbitrario de abiertos, entonces, $\forall x \in U, \exists i \in I \tq x \in U_i \subseteq U$ y por tanto $x$ es un punto interior y $U$ es un abierto.
                \item Sea $U = \bigcap\limits_{i = 1}^n U_i$ la intersección de un numero finito de abiertos, entonces, $\forall x \in U, \forall i \in \left\{ 1, \dots, n \right\}, \exists a_i, b_i \tq x \in \left(a_i, b_i\right) \subseteq U_i$. Sean
                    \begin{gather*}
                        a = \max_{i \in \left\{ 1, \dots, n \right\}} \left\{a_i\right\}, \\
                        b = \min_{i \in \left\{ 1, \dots, n \right\}} \left\{b_i\right\},
                    \end{gather*}
                entonces $x \in \lp a, b \rp \subseteq \bigcap\limits_{i = 1}^n U_i$ y por tanto $x$ es un punto interior y $U$ es un abierto.
            \end{enumerate}
    \end{enumerate}
\end{eje}

%TODO 2.4 (dale ernesto)
\begin{eje}
 Este ejercicio aún no está resuelto.
\end{eje}

\begin{eje}
    Como viene siendo costumbre, para demostrar que $\T$ es una topología debemos
    ver que cumple las propiedades de esta:
    \begin{enumerate}[i)]
        \item El vacío es un abierto de $\real$ que no contiene 0. 
        \item $\real = \lp\real \setminus \lc 0 \rc \rp \cup \lc 0^-, 0^+ \rc \in \T$.
        \item $\bigcup\limits_{i \in I} U_i = \bigcup\limits_{i \in I} U'_i 
            \setminus \lc 0 \rc
            \cup S_i = \lp \bigcup\limits_{i \in I} U'_i \rp
            \setminus \lc 0 \rc \cup \bigcup\limits_{i \in I} S_i = V \setminus \lc 0 \rc 
            \cup S \in \T$. Los $S_i$ son vacíos si $0 \notin U_i$ y en caso contrario
            $S_i \subseteq \lc 0^-, 0^+ \rc$. Se ha usado que 
            la unión de abiertos en $\real$ es un abierto.
        \item $\bigcap\limits_{i = 1}^n U_i =  \bigcap\limits_{i = 1}^n U'_i 
            \setminus \lc 0 \rc
            \cup S_i = \lp \bigcap\limits_{i = 1}^n U'_i \rp
            \setminus \lc 0 \rc \cup \bigcap\limits_{i = 1}^n S_i = V \setminus \lc 0 \rc 
            \cup S \in \T$. 
    \end{enumerate}
   
    $\T$ no es Hausdorff puesto que cualquier abierto que contenga $0^+$ debe contener
    un intervalo alrededor del $0$ en $\real$ y lo mismo es cierto para $0^-$, con lo
    que cualquier pareja de abiertos que los contenga tendrá intersección no nula.
    $\T$ no es la topología m\'etrica de una distancia dado que cualquier topología
    m\'etrica es Hausdorff: Dados dos puntos $x, y$ distintos, la bola
    abierta con centro $x$ y radio $\frac{d \lp x, y \rp}{2}$ y la bola con centro
    $y$ y de mismo radio tienen intersección nula (es consecuencia directa de la
    desigualdad triangular), pero contienen a sus respectivos centros.
\end{eje}

%TODO 2.6 no es de ningu fes aquest exercici tan maco de PUTAAAAA SIDAAAAAA
\begin{eje}
 Este ejercicio aún no está resuelto
\end{eje}

\begin{eje}
 Sea $\T$ una topología y $X$ un espacio topológico. Por definición de $\psi$ se tiene que $\forall A\subseteq X,\, \psi\lp A\rp =\overline{A}$ es un cerrado de $\T$. Por lo tanto, podemos definir un conjunto de abiertos de $\T$ como
 \[
  \B = \lc B\subseteq X\, | \,\exists A \subseteq X \tq B=X\setminus \psi\lp A\rp \rc
 \]
 que son abiertos por ser el complementario de un cerrado ($\overline{A}$ es el cerrado más pequeño que contiene a $A$).
 
 Ahora demostraremos que $\B$ es una base de la topología $\T$. 

 Primero veamos que
 \begin{gather*}
  X\setminus \psi\lp \varnothing\rp = X\setminus\varnothing = X \in \B\\
  X\setminus \psi\lp X\rp =X\setminus X = \varnothing \in \B.
 \end{gather*}
 
Entonces tenemos lo siguiente:
\begin{enumerate}[i)]
 \item $\forall x\in X,\, \exists A\subseteq \tq \overline{A}\cap\lc x\rc = \varnothing \implies x\in B = X\setminus \psi\lp A\rp$.
 En concreto podemos coger $A=\varnothing$.
 \item Sean $B_1,B_2 \in \B$ y sea $x\in X \tq x\in B_1\cap B_2$. Entonces
 \begin{gather*}
  x\in B_1\cap B_2 = \lp X\setminus \psi\lp A_1\rp \rp \cap \lp X\setminus\psi\lp A_2\rp \rp = X\setminus \lp\psi\lp A_1\rp\cup\psi\lp A_2\rp\rp = \\
  = X\setminus\psi\lp A_1 \cup A_2\rp = X\setminus\psi\lp\psi\lp A_1 \cup A_2\rp \rp,
 \end{gather*}
 y como $A=\psi\lp A_1\cup A_2\rp \subseteq X$, $B=X\setminus\psi\lp A\rp \in \B$, tenemos que $\exists B \in \B \tq x\in B \subseteq B_1 \cap B_2$.
\end{enumerate}
Por lo tanto, $\B$ es la base de una topología (de $\T$), lo que nos dice que como $\B$ son los mínimos abiertos que debe contener una topología que cumpla que $\psi\lp A\rp = \overline{A}$ y una base define una única topología, existe una única topología $\T$ que lo cumple. 
\end{eje}

