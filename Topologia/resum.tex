\section{Espais topològics 2AN i espais separables}

\begin{defi}
    Un espai topològic $X$ es diu que satisfà el segon axioma de numerabilitat (2AN) si té una base de cardinal numerable.
\end{defi}

\begin{example}
    Amb la topologia usual, $\real$ t\'e una base numerable: la fam\'ilia de tots els intervals oberts $(a, b)$ amb $a, b \in \q$. An\`alogament, també $\real^n$ té una base numerable:
    \[B=\left\{(a_1,b_1)\times\dots\times(a_n,b_n)\,|\,a_i, b_i\in \q, 1\leq i\leq n\right\}. \]
\end{example}

\begin{defi}
    Es diu que un conjunt $A$ és dens en $X$ si $\overline{A} = X$. Un espai s'anomena separable si té un subconjunt numerable dens.
\end{defi}

\begin{example}
    Tenim que $\overline{\q} = \real$. Com que $\q$ és numerable, $\real$ és separable. Més generalment, $\real^n$ és separable perquè admet $\q^n$ com a subconjunt dens.
\end{example}

\begin{prop}
    Un espai mètric $X$ és 2AN si i només si és separable.
\end{prop}
\begin{proof}
    Sigui $\B$ una base numerable de $X$. Per a cada $B \in \B$, prenem un punt $x_B \in B$ qualsevol. Llavors, com és fàcil de comprovar, el conjunt $\left\{x_B\right\}_{B \in \B}$ és dens.
    
    \quad
    
    Sigui $A$ un subconjunt numerable dens de l'espai. Llavors
    \[\B = \left\{B_{\sfrac{1}{n}} (a) \, | \, a\in A, \, n\in \n\right\}\]
    és una base numerable de l'espai. És numerable perquè és reunió numerable de conjunts numerables. A més, és una base ja que sigui $x$ un punt qualsevol i $U$ un entorn obert seu. Com que $x$ és interior a $U$, $\exists n>0$ tal que $B_{\sfrac{1}{2n}}(x)\cap A\subseteq U\cap A$. Provem que $x \in B_{\sfrac{1}{2n}}(a)\subseteq U$. Per la desigualtat triangular, sigui $y \in B_{\sfrac{1}{2n}}(a)$, es té
    \[d(x, y) \leq d(x, a) + d(a, y) < \frac{1}{2n} + \frac{1}{2n} = \frac{1}{n}\]
    i, per tant, $y\in B_{\sfrac{1}{n}}(x) \subseteq U$.
\end{proof}

\section{Topologia quocient}
\begin{obs}
    Siguin $X$ un conjunt i $\sim$ una relació d'equivalència, llavors $X/\sim$ \'es el conjunt de classes d'equivalència. Si $f\colon X\rightarrow Y$ exhaustiva, definim $x\sim x' \iff f(x) = f(x')$ i tenim que
    \begin{align*}
        \overline{f} \colon X/\sim &\rightarrow Y \\
        \left[ x \right] &\mapsto f(x)
    \end{align*}
    \'es bijectiva.
\end{obs}
\begin{defi}
    Sigui $X$ un espai topològic. Siguin $Y\in X$ i $\pi \colon X \rightarrow Y$ exhaustiva. Diem que $Y$ t\'e la topologia quocient per $\pi$ si els oberts de $Y$ són
    \[ \T_Y = \left\{ V\subseteq Y\,|\,\pi^{-1}(V) \in \T_X \right\}. \]
\end{defi}
\begin{prop}
    Propietats:
    \begin{enumerate}
        \item La topologia quocient de $Y$ per $\pi$ \'es la m\'es fina que fa $\pi$ contínua.
        \item Sigui $g\circ\pi \colon X \stackrel{\pi}{\rightarrow} Y \stackrel{g}{\rightarrow} Z$. $g$ \'es contínua $\iff g\circ\pi$ \'es contínua.
    \end{enumerate}
\end{prop}
\begin{proof}
    Vejem-ho.
    \begin{enumerate}
        \item Immediat per la definició de topologia quocient.
        \item La implicació cap a la dreta: $g \circ \pi$ \'es contínua perquè \'es composició de contínues.

            La implicació cap a l'esquerra: Sigui $W \subset Z$ obert, $g^{-1}(W) \subset Y$ \'es obert ja que
            \[\pi^{-1}\left( g^{-1}\left( W \right) \right) = \left( g\circ \pi \right)^{-1}\left( W \right) \implies \text{obert.}\]
    \end{enumerate}
\end{proof}
\begin{prop}[Tallar i enganxar]
    Siguin $f\colon X\rightarrow Y$ un homeomorfisme i $\sim_X$, $\sim_Y$ relacions d'equivalència a $X$ i a $Y$ tals que $x \sim_X x' \iff f\left( x \right) \sim_Y f\left( x' \right)$, llavors $f$ indueix un homeomorfisme $\overline{f} \colon X/\sim \rightarrow Y/\sim$.
\end{prop}
%Falta la demo (pg 64) i exemples
