\chapter{Sistemas de ecuaciones no lineales}

\noindent Dada $\vec{f} \colon \real^n \to \real^n$, queremos encontrar $\vec{\alpha} \in \real^n$ tal que $\vec{f}\left( \vec{\alpha} \right) = \vec{0}$.

\section{M\'etodo de Newton-Raphson}

\noindent Dada una aproximaci\'on $\vec{x}^k \approx \vec{\alpha}$, consideramos l'aproximaci\'on lineal de $\vec{f}\left( \vec{x} \right)$ en $\vec{x}^k$
\begin{align*}
    \vec{f}\left( \vec{x} \right) &\approx \vec{f}\left( \vec{x}^k \right) + \frac{\partial \vec{f}}{\partial \vec{x}}\left( \vec{x}^{k+1}-\vec{x}^k \right) + \cancel{\O \left( \|x-x^k\|^2 \right)} = 0, \\
    \vec{x}^{k+1} &= \vec{x}^k - \left[ \frac{\partial\vec{f}}{\partial\vec{x}}\left( x^k \right) \right]^{-1} \vec{f}\left( \vec{x}^k \right),\, \text{donde } J = \frac{\partial\vec{f}}{\partial \vec{x}} \text{ es la jacobiana.}
\end{align*}

\noindent Se puede ver que tiene convergencia cuadr\'atica (de orden 2) si $\det\left( J(\alpha) \right) \neq 0$.

\begin{obs}
    No calcularemos $J^{-1}$ (porqu\'e es aproximadamente el coste de resolver $n$ sistemas lineales). Calculamos:
    \begin{enumerate}[i)]
        \item $J\left( \vec{x}^k \right)\Delta \vec{x}^k = -\vec{f}\left( \vec{x}^k \right) \longrightarrow \Delta \vec{x}^k$,
        \item $\vec{x}^{k+1} = \vec{x}^k + \Delta\vec{x}^k$.
    \end{enumerate}
    En Matlab: $x = x-(J\setminus f)$ resuelve el sistema.
\end{obs}

\noindent ¿Qu\'e hacemos si no podemos calcular $J = \frac{\partial \vec{f}}{\partial \vec{x}}$?
\emph{Aproximaci\'on de las derivadas}:
\begin{align*}
    \frac{\partial \vec{f}}{\partial x_i}\left( \vec{x}^k \right) &\approx \frac{\vec{f}\left( \vec{x}^k + h\vec{e}_i \right) - \vec{f}\left( \vec{x}^k \right)}{h} \text{ con } h \text{ pequeña,} \\
    J &= \frac{\partial \vec{f}}{\partial \vec{x}} = \left[ \frac{\partial \vec{f}}{\partial x_1}, \dots, \frac{\partial \vec{f}}{\partial x_n} \right].
\end{align*}

\begin{obs}
    Si Newton converge pero no con convergencia cuadr\'atica, hay que comprobar que $J = \frac{\partial\vec{f}}{\partial\vec{x}}$ en la implementaci\'on.
\end{obs}

\section{M\'etodos casi-Newton}

\noindent Si no podemos calcular $J = \frac{\partial \vec{f}}{\partial \vec{x}}$, también podemos usar métodos casi-Newton.

\begin{itemize}
    \item $\vec{x}^{k+1} = \vec{x}^k - \left[ S^k \right]^{-1} \vec{f}\left( \vec{x}^k \right)$, donde $S^k \approx J\left( \vec{x}^k \right)$.
    \item $S^k$ se calcula a partir de $S^{k-1}$, con la \emph{ecuaci\'on secante} $\lp S^k\Delta\vec{x}^k = \Delta \vec{f}^k\rp$:
        \[
            \begin{cases}
                \Delta\vec{x}^k = \vec{x}^k - \vec{x}^{k-1}, \\
                \Delta\vec{f}^k = \vec{f}\left( \vec{x}^k \right) - \vec{f}\left( \vec{x}^{k-1} \right).
            \end{cases}
        \]
\end{itemize}

\begin{obs}
    En una dimensi\'on, es el m\'etodo de la secante. $S^k\left( x^k - x^{k-1} \right) = f^k - f^{k-1}$, por lo tanto
    \[
        S^k = \frac{f^k - f^{k-1}}{x^k - x^{k-1}}.
    \]
    En $n$ dimensiones, la ecuaci\'on secante no determina de manera \'unica $S^k$. Se añaden otras condiciones que generan muchos m\'etodos $QN$.
\end{obs}

\subsection{M\'etodo de Broyden}

\noindent El m\'etodo de Broyden es un m\'etodo $QN$ que impone
\begin{enumerate}[i)]
    \item\label{item:broyden1} $S^k\Delta\vec{x}^k = \Delta f^k$ (ecuaci\'on secante),
    \item\label{item:broyden2} $\left( S^k - S^{k-1} \right)\vec{\omega} = \vec{0}, \, \forall \vec{\omega} \bot \Delta \vec{x}^k$.
\end{enumerate}

\noindent Podemos conseguir \ref{item:broyden2} con $S^k = S^{k-1} + \vec{u}\left( \Delta\vec{x}^k \right)^T$.
Ahora imponemos \ref{item:broyden1}
\[
    \Delta f^k = S^k\Delta \vec{x}^k = \left( S^{k-1}+\vec{u}\left( \Delta\vec{x}^k \right)^T \right)\Delta\vec{x}^k = -\vec{f}^{k-1} + \vec{u}\left\| \Delta \vec{x}^k\right\|^2,
\]

\noindent de donde aislamos

\[
    \vec{u}\left\|\Delta\vec{x}^k\right\|^2 = \Delta \vec{f}^k + \vec{f}^{k-1} = \vec{f}^k \implies \vec{u} = \frac{\vec{f}^k}{\left\|\Delta\vec{x}^k\right\|^2}.
\]

\noindent En resumen,
\begin{itemize}
    \item Dado $\vec{x}^0\in\real^n,\, S^0 \in \real^{n\times n},\, (S_0 = \Id$ ó $S_0 = J\lp \vec{x}^0\rp$ aproximación lineal),
    \item Calcular $\vec{f}^0 = \vec{f}\left( \vec{x}^0 \right)$,
    \item Para $k = 0, 1, \dots$,
        \begin{itemize}
            \item Resolver $S^k \Delta \vec{x}^{k+1} = -f^k \longrightarrow \Delta\vec{x}^{k+1}$,
            \item $\vec{x}^{k+1} = x^k + \Delta x^{k+1}$,
            \item Calcular $\vec{f}^{k+1} = \vec{f}\left( \vec{x}^{k+1} \right)$,
            \item $S^{k+1} = S^k + \vec{f}^{k+1}\left( \frac{\Delta\vec{x}^k}{\left\|\Delta \vec{x}^k \right\|^2} \right)^T$.
        \end{itemize}
\end{itemize}
