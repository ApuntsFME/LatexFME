\chapter{Aproximación funcional}

\noindent Tenemos los siguientes datos:
\begin{itemize}
    \item $f\left( x \right)$ analíticamente o evaluable,
    \item $\left\{ f_i \right\}^n_{i=0}$ valor de $f$ en $n+1$ puntos $\left\{ x_i \right\}^n_{i=0}$,
    \item $\left\{ f_i \right\}^n_{i=0},\left\{ f'_i \right\}^n_{i=0},\dots$,
    \item Espai d'aproximació $V$ (espai vectorial).
\end{itemize}

\noindent El objetivo es determinar $p\in V\tq p\left( x \right) \approx f\left( x \right)$ ($V$ es el tipo de aproximación y $\approx$ el criterio de aproximación). Tanto si los datos son discretos como si son continuos, podemos decidir diferentes criterios y tipos de aproximación. %TODO dibujo

\quad

\noindent Tipos de aproximación:
\begin{itemize}
    \item Polinómica: $f\left( x \right)\approx p_n\left( x \right) \in \Pa^n,\, n$ fixat.
    \item Trogonométrica: $f\left( x \right)\approx \sum\limits^n_{k=1} a_k \sin\left( nkl \right) + b_k \cos\left( nkl \right)$.
    \item Splines: Polinomios a trozos.
\end{itemize}
Criterios de aproximación:
\begin{itemize}
    \item Interpolación: $p\left( x_i \right) = f\left( x_i \right), \, i=0,\dots,n$.
    \item Mínimos cuadrados: $p = \arg\min\limits_{q\in V} \|f-q\|^2$, por ejemplo, $\|f-p\|^2 = \sum\limits^n_{i=0} \left( f_i -p\left( x_i \right) \right)^2$.
    \item Min-max.
\end{itemize}

\section{Interpolación polinómica}

\begin{teo*}
    Dados $n+1$ puntos $\left\{ x_i \right\}^n_{i=0}$ diferents i $n+1$ valors $\left\{ f_i \right\}^n_{i=0}$, existeix un únic polinomi $p_n\left( x \right)$ de grau $\leq n$ tal que $p_n\left( x_i \right) = f_i,\, i=0,\dots,n$.
\end{teo*}

\begin{proof}
    Consideramos $p_n\left( x \right) = a_0+a_1x+ \dots+a_n x^n$ (en la base natural) e imponemos $f_i = p_n\left( x_i \right) = a_0 + a_1 x_i + \dots + a_n x_i^n,\, i = 0,\dots,n$.
    \[
        \underbrace{
        \begin{pmatrix}
            1 & x_0 & \dots & x_0^n \\
            \vdots & \vdots & & \vdots \\
            1 & x_n & \dots & x_n^n
        \end{pmatrix}}_A
        \begin{pmatrix}
            a_0 \\
            \vdots \\
            a_n
        \end{pmatrix}
        =
        \begin{pmatrix}
            f_0 \\
            \vdots \\
            f_n
        \end{pmatrix}.
    \]
    $A$ es una matriz de Vandermonde, $\det\left( A \right) = \prod\limits_{1\leq i < j\leq n} \left( x_i-x_j \right) \neq 0$ si $\left\{ x_i \right\}$ son no repetidos y por tanto existe una única solución al sistema.
\end{proof}

\begin{obs}
    En la práctica, el determinante no sirve para saber si una matriz es singular. Hay que mirar el rango, el numero de condición ($\kappa_2$), etc.
\end{obs}

\begin{example}
    Si tenemos $\det\left( A \right) = 0,1$, entonces $\det\left( 0,1\cdot A \right) = 10^{-(n+1)}$.
\end{example}

\begin{obs}
    Las matrices de Vandermonde están muy mal condicionadas
    \begin{center}
        \begin{tabular}{|c|c|}\hline
            $n$ & $\kappa_2\left( A \right)$ \\ \hline \hline
            $2$ & $13$ \\ \hline
            $3$ & $154$ \\ \hline
            $5$ & $5,7\cdot 10^4$ \\ \hline
            $10$ & $4,5 \cdot 10^{12}$ \\ \hline 
        \end{tabular}
    \end{center}
    $\kappa_2\left( A \right) = \|A\|\cdot \|A^{-1}\|$, si $A$ es simétrica,
    \[
        \kappa_2\left( A \right) = \frac{\abs{\lambda_{\text{máx}}}}{\abs{\lambda_{\text{mín}}}}.
    \]
    Si $\kappa_2\left( A \right) \gg 1 \implies$ mal condicionada.
\end{obs}

\noindent Mal condicionamiento:
\begin{enumerate}[1)]
    \item Los errores de redondeo se propagan mucho. Los métodos iterativos convergen muy lentamente.
    \item Aunque hiciéramos aritmética exacta:
        \[
            \|\vec{r}_x\| \leq \underbrace{\kappa_2 \left( A \right)}_{\mathclap{10^{12}\cdot \frac{1}{2}10^{-16} = \frac{1}{2}10^{-4} \,\rightarrow\, 3\text{ cifras significativas correctas}}} \|\vec{r}_b\|\text{ donde }\vec{x}\text{ es solución de }A\vec{x} = \vec{b}
        \]
\end{enumerate}

\noindent Así pues, intentaremos no trabajar con matrices de Vandermonde. Escogeremos una base $\left\{ L_j\left( x \right) \right\}^n_{j=0}$ (base de Lagranje) tal que, $\forall \left\{ f_i \right\}^n_{i=0}$,
\[
    f_i = f\left( x_i \right) \approx p_n\left( x_i \right) = \sum^n_{j=0} f_jL_j\left( x_i \right).
\]
Es decir, tenemos que imponer
\[
    L_j\left( x_i \right) = \delta_{ij} =
    \begin{cases}
        1 & i=j \\
        0 & i\neq j
    \end{cases}
\]

\begin{defi}[interpolaci\'on de Lagrange]
    \[
        f\left( x \right) \approx p_n\left( x \right) = \sum_{j=0}^{n} f_jL_j\left( x \right), \text{ donde } L_j = \frac{\prod\limits_{k\neq j}\left( x-x_k \right)}{\prod\limits_{k\neq j}\left( x_j - x_k \right)} \in \Pa^n.
    \]
\end{defi}

\begin{example} %TODO dibuixos
    $f\left( x \right)\approx p_2\left( x \right) = f_0L_0\left( x \right) + f_1L_1\left( x \right) + f_2L_2\left( x \right)$
    \begin{align*}
        L_0\left( x \right) &= \frac{\left( x-0,5 \right)\left( x-1 \right)}{\left( 0-0,5 \right)\left( 0-1 \right)} = 2\left( x-0,5 \right)\left( x-1 \right),\\
        L_1\left( x \right) &= \frac{\left( x-0 \right)\left( x-1 \right)}{\left( 0,5-0 \right)\left( 0,5-1 \right)} = \left( -4 \right)x\left( x-1 \right),\\
        L_2\left( x \right) &= 2x\left( x-0,5 \right).
    \end{align*}
\end{example}

\begin{teo}[Teorema del residuo de Lagrange]
    Dada $f\in \C^{n+1}$ i $\left\{ x_i \right\}_{i=0}^n$, el polinomio interpolador $p_n\left( x \right)\in \Pa^n$ tal que $p_n\left( x_i \right) = f\left( x_i \right), i=0,\dots,n$. Tiene error
    \[
        f\left( x \right) = p_n\left( x \right) + \frac{f^{\left( n+1 \right)}\left( \mu\left( x \right) \right)}{\left( n+1 \right)!}L\left( x \right), \text{ donde }
        \begin{cases}
            L\left( x \right) = \left( x-x_0 \right) \dots \left( x-x_n \right), \\
            \mu\left( x \right) \in \left[ x_0, x_n; x \right].
        \end{cases}
    \]
\end{teo}

\begin{proof}
    Observamos que es cierto cuando $x=x_i$ (uno de los puntos base):
    \[
        \underbrace{f\left( x_i \right) - p_n\left( x_i \right)}_{=0} - \frac{f^{n+1}\left( \mu \right)}{\left( n+1 \right)!}\underbrace{L\left( x_i \right)}_{=0} = 0-0 = 0.
    \]
    Consideramos ahora $\hat{x} \neq x_i, \forall i$ (fijado) y $g\left( x \right) = f\left( x \right) - p_n\left( x \right) - k\left( \hat{x} \right)L\left( x \right)$, donde escogemos $k$ tal que $g\left( \hat{x} \right) = 0\, \left( k\left( \hat{x} \right) = \frac{f\left( \hat{x} \right) - p\left( \hat{x} \right)}{L\left( \hat{x} \right)} \right)$.
    
    \noindent Entonces $g\left( x \right)$ tiene $n+2$ ceros diferentes:
    \[
        \begin{cases}
            \hat{x} \text{ es un cero},\\
            \left\{ x_i \right\}_{i=0}^n \text{ son }n\text{ ceros}.
        \end{cases}
    \]
    Por tanto $g'\left( x \right)$ tiene almenos $n+1$ ceros (por el teorema de Rolle), $g''\left( x \right)$ tiene almenos $n$ ceros, \dots, $g^{\left( n+1 \right)} \left( x \right)$ tiene almenos $1$ cero. Llamemosle $\mu\left( \hat{x} \right) = \mu$,
    \[
        g^{\left( n+1 \right)}\left( \mu \right) = f^{\left( n+1 \right)}\left( \mu \right) - k\left( \hat{x} \right)\left( n+1 \right)! = 0.
    \]
    Sabiendo que $k\left( \hat{x} \right) = \frac{f\left( \hat{x} \right) - p\left( \hat{x} \right)}{L\left( \hat{x} \right)}$, tenemos que
    \begin{align*}
        f^{\left( n+1 \right)}\left( \mu \right) - \frac{f\left( \hat{x} \right) - p\left( \hat{x} \right)}{L\left( \hat{x} \right)}\left( n+1 \right)! = 0 &\implies \dots \implies\\
        &\implies f\left( \hat{x} \right) = p_n\left( \hat{x} \right) + \frac{f^{\left( n+1 \right)}\left( \mu\left( \hat{x} \right) \right)}{\left( n+1 \right)!}L\left( \hat{x} \right),\quad \forall \hat{x}.
    \end{align*}
\end{proof}

\begin{obs}[La paradoja de Runge]
    Uno de los problemas que la aproximaci\'on funcional presenta es la ``paradoja de Runge''. Sea $f\left( x \right) = \frac{1}{1+25x^2}$, la interpolaci\'on con $n+1$ puntos equiespaciados empeora cuando $n$ crece. Si $n$ es muy grando (hay muchos datos), es necesario hacer un cambio de criterio (m\'inimos cuadrados) o cambiar de tipo (Splines).
\end{obs}

\section{Interpolaci\'on de Splines}

\begin{prop}
    Dados unos puntos base $\left\{ x_i \right\}_{i=0}^n$, un grado $m$ y una continuidad $\C^q$, consideramos
    \begin{align*}
        E_{S} &= \setb{S\left( x \right) \in \C^q\left( \left[ x_0, x_n \right] \right) \tq S_{|[ x_i, x_{i+1} ]_{i=0,\dots,n}} \in \Pa^m}, \\
        S_i\left( x \right) &= S\left( x \right)_{|[ x_i, x_{i+1} ]}.
    \end{align*}
    Entonces $E_S$ es un espacio vectorial (tiene una base, etc.).
\end{prop}

\subsection{Splines lineals ($m=1$) $\C^0$}

$S_i\left( x \right) = a_i\left( x-x_i \right) + b_i$, ¿cuánto vale $\dim\left( E_S \right)$?
\begin{center}
    \begin{tabular}{crcl}
        & $\#$coeficientes &=& $2n$ (2 coeficientes por $n$ intervalos)\\
        $-$ & $\#$condiciones &=& $n-1$ (continuidad de $S$ en cada punto interior)\\\hline
        &$\dim\left( E_S \right)$ &=& $n+1$ (parámetros libres).
    \end{tabular}
\end{center}

Los $n+1$ parámetros libres están determinados unequívocamente por los $n+1$ valores de la función.

Cálculo del Spline: 
\[
    S_i\left( x \right) = a_i\left( x-x_i \right) + b_i
\]
donde $a_i = \frac{f_{i+1}-f_i}{x_{i+1}-x_i}$, $b_i = f_i$.

De todas las bases de Splines lineales $\C^0$, una interesante es
\[
    \{ \underbrace{\Phi_i}_{\in E_S} \}_{i=0}^n \tq S\left( x \right) = \sum_{i=0}^n f_i\Phi_i\left( x \right).
\]
Las funciones de la base deberán cumplir
\[
    \Phi_i\left( x_j \right) = \delta_{ij}
\]
ya que
\[
    f_j = S\left( x_j \right) = \sum_{i=0}^n f_i\Phi_i\left( x_j \right),\quad \forall \left\{ f_i \right\}_{i=0}^n
\]

\begin{obs}
    Podem hacer servir la base para hacer un ajuste por mínimos cuadrados.
\end{obs}

\subsection{Splines cuadráticos ($m=2$) $\C^1$}

$S_i\left( x \right) = a_i\left( x-x_i \right)^2 + b_i \left( x-x_i \right) + c_i$, ¿cuánto vale $\dim\left( E_S \right)$?

\begin{center}
    \begin{tabular}{crcl}
        & $\#$coeficientes &=& $3n$\\
        $-$ & $\#$condiciones &=& $2\left( n-1\right)$\\\hline
        &$\dim\left( E_S \right)$ &=& $n+2$.
    \end{tabular}
\end{center}

Los $n+2$ parámetros quedan determinados por los $n+1$ valores de $f$ y $S'\left( x_0 \right) = f'_0$.

\subsection{Splines cúbicos ($m=3$) $\C^1$}

$S_i\left( x \right) = a_i\left( x-x_i \right)^3 + b_i \left( x-x_i \right)^2 + c_i\left( x-x_i \right) + d_i$, ¿cuánto vale $\dim\left( E_S \right)$?

\begin{center}
    \begin{tabular}{crcl}
        & $\#$coeficientes &=& $4n$\\
        $-$ & $\#$condiciones &=& $2\left( n-1\right)$\\\hline
        &$\dim\left( E_S \right)$ &=& $2n+2$.
    \end{tabular}
\end{center}

Para determinar un Spline necesitamos $2(n+1)$ datos: $\left\{ f_i \right\}_{i=0}^n$ y $\left\{ f'_i \right\}_{i=0}^n$.

\begin{obs}
    Si solo tenemos $\left\{ f_i \right\}_{i=0}^n$, podemos calcular aproximaciones
    \[
        f'_i = \frac{f_{i+1}-f_{i-1}}{2h}\approx f'\left( x_i \right)
    \]
    si los puntos son equiespaciados. Si los puntos no son equiespaciados hay que deducir una forma diferente.

    Cálculo de una Spline:
    \begin{align*}
        f_i &= S_i\left( x_i \right) = d_i, \\
        f_{i+1} &= S_i\left( x_{i+1} \right) = a_i h_i^3 + b_ih_i^2 + c_ih_i + d_i, \\
        f'_i &= S'_i\left( x_i \right) = c_i, \\
        f'_{i+1} &= S'_i\left( x_{i+1} \right) = 3a_ih_i^2 + 2b_ih_i + c_i,
    \end{align*}
    con $h_i = x_{i+1} - x_i$.

    Si aislamos, obtenemos
    \begin{align*}
        a_i &= \frac{h_i\left( f'_i + f'_{i+1} \right) - 2+i}{h_i^3}, \\
        b_i &= \frac{3t_i - h_i\left( 2f'_i + f'_{i+1} \right)}{h_i^2}, \\
        c_i &= f'_i, \\
        d_i &= f_i,
    \end{align*}
    con $h_i = x_{i+1} - x_i$ y $t_i = f_{i+1} - f_i$.
\end{obs}

La base que nos interesa para los Splines cúbicos $\C^1$ es la base 
\[
    \left\{ \Phi_i \right\}_{i=0}^n \cup \left\{ \tilde{\Phi}_i \right\}_{i=0}^n \tq S\left( x \right) = \sum_{i=0}^n f_i\Phi_i\left( x \right) + \sum_{i=0}^n f'_i\tilde{\Phi}_i\left( x \right).
\]

Tenemos que
\begin{gather*}
    \begin{rcases}
        f_j = S\left( x_j \right) = \sum\limits_{i=0}^n f_i\Phi_i\left( x_j \right) + \sum\limits_{i=0}^n f'_i\tilde{\Phi}_i\left( x_j \right) \\
        f'_j = S'\left( x_j \right) = \sum\limits_{i=0}^n f_i\Phi'_i\left( x_j \right) + \sum\limits_{i=0}^n f'_i\tilde{\Phi}'_i\left( x_j \right)
    \end{rcases}
    \implies \\
    \implies
    \begin{cases}
        \Phi_i \left( x_j \right) = \delta_{ij},\, \tilde{\Phi}_i\left( x_j \right) = 0\quad \forall i,j\\
        \Phi_i \left( x_j \right) = 0\quad \forall i,j,\, \tilde{\Phi}_i\left( x_j \right) = \delta_{ij}.
    \end{cases}
\end{gather*}
Por lo tanto, las funciones de la base deberán cumplir
\begin{align*}
    \Phi_i &= \left\{ f_i = 1, f_j=0\quad \forall j\neq i, f'_j = 0 \quad\forall j \right\}, \\
    \tilde{\Phi}_i &= \left\{ f_i = 0\quad \forall i, f'_i = 1, f'_j = 0 \quad\forall j\neq i \right\}.
\end{align*}

\begin{obs}
    Es una base local, lo cual es bueno para mínimos cuadrados, EDPs, una matriz con muchos valores nulos, etc. ya que hay una dependencia local de $S\left( x \right)$ con los datos.
\end{obs}

\subsection{Splines cúbicos ($m=3$) $\C^2$}

$S_i\left( x \right) = a_i\left( x-x_i \right)^3 + b_i\left( x-x_i \right)^2 + c_i\left( x-x_i \right) + d_i$, ¿cuánto vale $\dim\left( E_S \right)$?

\begin{center}
    \begin{tabular}{crcl}
        & $\#$coeficientes &=& $4n$\\
        $-$ & $\#$condiciones &=& $3\left( n-1\right)$ (continuidad de $S$, $S'$ y $S''$ en $\left\{ x_i \right\}_{i=1}^{n-1}$)\\\hline
        &$\dim\left( E_S \right)$ &=& $n+3$.
    \end{tabular}
\end{center}

Tenemos $n+1$ valores de la función, y hay que añadir $2$ condiciones adicionales. Estas pueden ser:

\begin{itemize}
    \item \emph{Spline natural} (el más ``suave'' posible): $f''_0 = 0$, $f''_n = 0$.
    \item $S'\left( x_0 \right) = f'_0$, $S'\left( x_n \right) = f'_n$.
\end{itemize}
