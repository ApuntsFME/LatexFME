\chapter{Aproximación funcional}

\noindent Tenemos los siguientes datos:
\begin{itemize}
    \item $f\left( x \right)$ analíticamente o evaluable,
    \item $\left\{ f_i \right\}^n_{i=0}$ valor de $f$ en $n+1$ puntos $\left\{ x_i \right\}^n_{i=0}$,
    \item $\left\{ f_i \right\}^n_{i=0},\left\{ f'_i \right\}^n_{i=0},\dots$,
    \item Espai d'aproximació $V$ (espai vectorial).
\end{itemize}

\noindent El objetivo es determinar $p\in V\tq p\left( x \right) \approx f\left( x \right)$ ($V$ es el tipo de aproximación y $\approx$ el criterio de aproximación). Tanto si los datos son discretos como si son continuos, podemos decidir diferentes criterios y tipos de aproximación. %TODO dibujo

\quad

\noindent Tipos de aproximación:
\begin{itemize}
    \item Polinómica: $f\left( x \right)\approx p_n\left( x \right) \in \Pa^n,\, n$ fixat.
    \item Trogonométrica: $f\left( x \right)\approx \sum\limits^n_{k=1} a_k \sin\left( nkl \right) + b_k \cos\left( nkl \right)$.
    \item Splines: Polinomios a trozos.
\end{itemize}
Criterios de aproximación:
\begin{itemize}
    \item Interpolación: $p\left( x_i \right) = f\left( x_i \right), \, i=0,\dots,n$.
    \item Mínimos cuadrados: $p = \arg\min\limits_{q\in V} \|f-q\|^2$, por ejemplo, $\|f-p\|^2 = \sum\limits^n_{i=0} \left( f_i -p\left( x_i \right) \right)^2$.
    \item Min-max.
\end{itemize}

\section{Interpolación polinómica}

\begin{teo*}
    Dados $n+1$ puntos $\left\{ x_i \right\}^n_{i=0}$ diferents i $n+1$ valors $\left\{ f_i \right\}^n_{i=0}$, existeix un únic polinomi $p_n\left( x \right)$ de grau $\leq n$ tal que $p_n\left( x_i \right) = f_i,\, i=0,\dots,n$.
\end{teo*}

\begin{proof}
    Consideramos $p_n\left( x \right) = a_0+a_1x+ \dots+a_n x^n$ (en la base natural) e imponemos $f_i = p_n\left( x_i \right) = a_0 + a_1 x_i + \dots + a_n x_i^n,\, i = 0,\dots,n$.
    \[
        \underbrace{
        \begin{pmatrix}
            1 & x_0 & \dots & x_0^n \\
            \vdots & \vdots & & \vdots \\
            1 & x_n & \dots & x_n^n
        \end{pmatrix}}_A
        \begin{pmatrix}
            a_0 \\
            \vdots \\
            a_n
        \end{pmatrix}
        =
        \begin{pmatrix}
            f_0 \\
            \vdots \\
            f_n
        \end{pmatrix}.
    \]
    $A$ es una matriz de Vandermonde, $\det\left( A \right) = \prod\limits_{1\leq i < j\leq n} \left( x_i-x_j \right) \neq 0$ si $\left\{ x_i \right\}$ son no repetidos y por tanto existe una única solución al sistema.
\end{proof}

\begin{obs}
    En la práctica, el determinante no sirve para saber si una matriz es singular. Hay que mirar el rango, el numero de condición ($\kappa_2$), etc.
\end{obs}

\begin{example}
    Si tenemos $\det\left( A \right) = 0,1$, entonces $\det\left( 0,1\cdot A \right) = 10^{-(n+1)}$.
\end{example}

\begin{obs}
    Las matrices de Vandermonde están muy mal condicionadas
    \begin{center}
        \begin{tabular}{|c|c|}\hline
            $n$ & $\kappa_2\left( A \right)$ \\ \hline \hline
            $2$ & $13$ \\ \hline
            $3$ & $154$ \\ \hline
            $5$ & $5,7\cdot 10^4$ \\ \hline
            $10$ & $4,5 \cdot 10^{12}$ \\ \hline 
        \end{tabular}
    \end{center}
    $\kappa_2\left( A \right) = \|A\|\cdot \|A^{-1}\|$, si $A$ es simétrica,
    \[
        \kappa_2\left( A \right) = \frac{\abs{\lambda_{\text{máx}}}}{\abs{\lambda_{\text{mín}}}}.
    \]
    Si $\kappa_2\left( A \right) \gg 1 \implies$ mal condicionada.
\end{obs}

\noindent Mal condicionamiento:
\begin{enumerate}[1)]
    \item Los errores de redondeo se propagan mucho. Los métodos iterativos convergen muy lentamente.
    \item Aunque hiciéramos aritmética exacta:
        \[
            \|\vec{r}_x\| \leq \underbrace{\kappa_2 \left( A \right)}_{\mathclap{10^{12}\cdot \frac{1}{2}10^{-16} = \frac{1}{2}10^{-4} \,\rightarrow\, 3\text{ cifras significativas correctas}}} \|\vec{r}_b\|\text{ donde }\vec{x}\text{ es solución de }A\vec{x} = \vec{b}
        \]
\end{enumerate}

Así pues, intentaremos no trabajar con matrices de Vandermonde. Escogeremos una base $\left\{ L_j\left( x \right) \right\}^n_{j=0}$ (base de Lagranje) tal que, $\forall \left\{ f_i \right\}^n_{i=0}$,
\[
    f_i = f\left( x_i \right) \approx p_n\left( x_i \right) = \sum^n_{j=0} f_jL_j\left( x_i \right).
\]
Es decir, tenemos que imponer
\[
    L_j\left( x_i \right) = \delta_{ij} =
    \begin{cases}
        1 & i=j \\
        0 & i\neq j
    \end{cases}
\]
