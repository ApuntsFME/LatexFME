\section{Electrostàtica} \emph{Constants}: $k=\frac{1}{4\pi\varepsilon_0}$, $\varepsilon_0=8,854\cdot 10^{-12}$. \\ \emph{Camp e.}: $\vec{E} = k\frac{q_\text{A}}{\|\vec{r}_{\text{AB}}\|^3}\vec{r}_{\text{AB}}$, $\vec{E}(\vec{r}) = \int_\nu \frac{k\rho(\vec{r'})}{\|\vec{r}-\vec{r'}\|^3}(\vec{r}-\vec{r'}) \dif V'$. \\ \emph{Força (Coulomb)}: $\vec{F}_{\text{AB}} = k\frac{q_{\text{A}}q_{\text{B}}}{\|\vec{r}_{\text{AB}}\|^3}\vec{r}_{\text{AB}}=q_\text{B}\vec{E_\text{A}}$. \\ \emph{E. potencial}: $U = k\frac{q_1q_2}{\|\vec{r}\|}$. \\
\emph{Potencial}: $V(\vec{r}) = \int_\nu \frac{k\rho(\vec{r'})}{\|\vec{r}-\vec{r'}\|}\dif V'$; $V=-\int_{\infty}^r \|\vec{E}\| \dif r$, $V_{\infty}=0$, $V_A -V_B = \int_A^B \|\vec{E}\| \dif r$. \\
\ci $\vec{E}=-\nabla V$. \\
\ci $\Delta V = \frac{\Delta U}{q_0}$. \\
\emph{L. Gauss}: $\oint_S \vec{E}\cdot \vec{\dif S} = \frac{1}{\varepsilon_0} \int_\nu \rho \dif V = \frac{Q_\text{int}}{\varepsilon_0}$.

\subsection{Dipols}
\emph{Moment dipolar}: $\vec{P} = 2aq\vec{u}$. \\
\emph{Potencial}: $V(\vec{r}) = k\frac{\cos \theta}{r^2}\cdot \vec{P} \approx \frac{k\vec{P}\cdot\vec{r}}{\vec{r^3}} (a \ll r)$. \\
\emph{Camp e.}: $\vec{E} = \frac{3k\cdot (\vec{P}\cdot\vec{r})\vec{r}}{r^5} - \frac{k\vec{P}}{r^3}$. \\
\emph{Força}: $\vec{F} = \vec{\nabla}(\vec{P}\cdot \vec{E}) = q\vec{E}$. \\
\emph{Moment}: $\vec{M} = \vec{P}\times \vec{E}$.

\subsection{Condensadors}
\emph{Capacitat}: $C = \frac{Q}{|V_1-V_2|}$. \\
\emph{Intensitat}: $I=C\frac{\dif V}{\dif t} = \frac{\dif Q}{\dif t}$.

\section{Electrocinètica}
\emph{L. Ohm}: $V=IR$. \\
\emph{Conductors}: $R = \frac{l}{S\gamma} = \frac{rl}{S}$. \\
\emph{Conductivitat-Resistivitat}: $\gamma = \frac{1}{r}$. \\
\emph{Potència}: $P=\frac{E}{t}=VI$. \\
\emph{Treball}: $W=\int_{r_1}^{r_2} F(\vec{r}) \cdot \dif \vec{r}$.

\section{Magnetostàtica}
\emph{Camp m.}: $\vec{B} = \frac{\mu_0}{4\pi}\frac{q\cdot \vec{v}\times (\vec{r} - \vec{r'})}{\|\vec{r} -\vec{r'}\|^3}$, $\mu_0=4\pi \cdot 10^{-7}$. \\
\emph{Força}: $\vec{F} = q \cdot\vec{v}\times\vec{B} = \vec{Il} \times \vec{B}$. \\
\emph{L. Ampère}: $\oint_C \vec{B}\cdot \vec{\dif l} = I_{\text{int}} \mu_0$.

\section{Maxwell}
\emph{Flux}: $\Phi = \int_S \vec{B}\cdot \vec{\dif S}$. \\
\emph{L. Faraday}: $\varepsilon = -\frac{\dif \Phi_{\text{m}}}{\dif t}=\oint_{\partial S} \vec{E} \cdot \vec{\dif l} = -\frac{\dif}{\dif t}\int_S \vec{B} \cdot \vec{\dif S}$.

\addtocounter{section}{1}
\noindent\makebox[\linewidth]{\rule{\linewidth}{0.5pt}}

\subsection{Integrals}
\ci $\int \frac{1}{\sqrt{a+x^2}}\dif x = \log (\sqrt{a+x^2} + x)$. \\
\ci $\int \frac{x}{\sqrt{a+x^2}}\dif x = \sqrt{a+x^2}$. \\
\ci $\int \frac{1}{\sqrt{a+x^2}^3}\dif x = \frac{x}{a\sqrt{a+x^2}}$. \\
\ci $\int \frac{x}{\sqrt{a+x^2}^3}\dif x = -\frac{1}{\sqrt{a+x^2}}$.

\subsection{Canvis de variables}
\emph{Polars a $\real^2$}: $\int_U f( x,y ) \dif x\dif y = \int_V f( r\cos\varphi, r\sin\varphi )r \dif r\dif \varphi$. \\
\emph{Cilíndriques a $\real^3$}: $\int_U f( x,y,z ) \dif x\dif y\dif z = \int_V f( \rho\cos\varphi, \rho\sin\varphi, z )\rho \dif \rho\dif \varphi\dif z$. \\
\emph{Esfèriques a $\real^3$}: $\int_U f( x,y,z ) \dif x\dif y\dif z = \int_V f ( r\cos\varphi\sin\theta, r\sin\varphi\sin\theta, r\cos\theta )$ $r^2\sin\theta \dif r\dif \varphi\dif \theta$.

\subsection{EDOs}
\ci $\frac{1}{v}\frac{\dif v}{\dif t} = \frac{\dif\, (\ln v)}{\dif t}$. \\
\emph{Fórmules}: $a\dot{v}+bv+c=0$; $\frac{a}{b}\dot{v}+v+\frac{c}{b}=0$; ($w=v+\frac{c}{b}$), $\frac{a}{b}\dot{w}+w=0$; $\frac{a}{b}x+1=0$; $x=\frac{-b}{a}$; $w=k e^{\frac{-b}{a}t}=v+\frac{c}{b}$; $v=ke^{\frac{-b}{a}t}-\frac{c}{b}$.

\subsection{Altres}
\emph{Esfera}: $S=4\pi r^2$, $V=\frac{4}{3}\pi r^3$.
