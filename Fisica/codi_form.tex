\newif\ifland
\landtrue
%\landfalse              % Comenta aquesta linia per obtenir el formulari, descomenta-la per veure totes les formules en gran

\ifland
\documentclass[10pt]{article}
\else
\documentclass[12pt]{article}
\fi

\usepackage[utf8]{inputenc}
\usepackage[T1]{fontenc}

\ifland
\usepackage{fp}
\def\rate{1}

\FPeval{\he}{297 * \rate}
\FPeval{\wi}{210 * \rate}

\usepackage[landscape,paperheight=\he mm,paperwidth=\wi mm,margin=0.2in,top=0.2in,bottom=0.2in]{geometry}

\else
\usepackage[margin=1in]{geometry}
\fi

\usepackage[pdftex]{hyperref}
\usepackage{amsmath,amsthm,amssymb,graphicx,mathtools,tikz,hyperref,enumerate}
\usepackage{mdframed,cleveref,cancel,stackengine,mathrsfs,thmtools}
\usepackage{xfrac,stmaryrd,commath,units,titlesec,multicol}
\usepackage[catalan]{babel}

\ifland
\titlespacing{\section}{0pt}{5pt}{0pt}
\titlespacing{\subsection}{0pt}{5pt}{0pt}
\titlespacing{\subsubsection}{0pt}{5pt}{0pt}
\setlength{\parindent}{0pt}
\pagenumbering{gobble}
\titleformat*{\section}{\large\bfseries}
\titleformat*{\subsection}{\bfseries}
\setlength{\columnseprule}{0.5pt}
\fi

\newcommand{\bimplies}{\fbx{$\implies$}}
\newcommand{\bimpliedby}{\fbx{$\impliedby$}}

\newcommand{\n}{\mathbb{N}}
\newcommand{\z}{\mathbb{Z}}
\newcommand{\q}{\mathbb{Q}}
\newcommand{\cx}{\mathbb{C}}
\newcommand{\real}{\mathbb{R}}
\newcommand{\E}{\mathbb{E}}
\newcommand{\F}{\mathbb{F}}
\newcommand{\A}{\mathbb{A}}
\newcommand{\R}{\mathcal{R}}
\newcommand{\C}{\mathscr{C}}
\newcommand{\Pa}{\mathcal{P}}
\newcommand{\Es}{\mathcal{E}}
\newcommand{\V}{\mathcal{V}}
\newcommand{\bb}[1]{\mathbb{#1}}
\let\u\relax
\newcommand{\u}[1]{\underline{#1}}
\newcommand{\pdv}[3][]{\frac{\partial^{#1} #2}{\partial #3^{#1}}}
\newcommand{\dv}[3][]{\frac{\dif^{#1} #2}{\dif #3^{#1}}}
\let\k\relax
\newcommand{\k}{\Bbbk}
\newcommand{\ita}[1]{\textit{#1}}
\newcommand\inv[1]{#1^{-1}}
\newcommand\setb[1]{\left\{#1\right\}}
\newcommand{\vbrack}[1]{\langle #1\rangle}
\newcommand{\determinant}[1]{\begin{vmatrix}#1\end{vmatrix}}
\newcommand{\Po}{\mathbb{P}}
\newcommand{\lp}{\left(}
\newcommand{\rp}{\right)}
\newcommand{\ci}{\textbullet\;}
\DeclareMathOperator{\fr}{Fr}
\DeclareMathOperator{\Id}{Id}
\DeclareMathOperator{\ext}{Ext}
\DeclareMathOperator{\inte}{Int}
\DeclareMathOperator{\rie}{Rie}
\DeclareMathOperator{\rg}{rg}
\DeclareMathOperator{\gr}{gr}
\DeclareMathOperator{\nuc}{Nuc}
\DeclareMathOperator{\car}{car}
\DeclareMathOperator{\im}{Im}
\DeclareMathOperator{\tr}{tr}
\DeclareMathOperator{\vol}{vol}
\DeclareMathOperator{\grad}{grad}
\DeclareMathOperator{\rot}{rot}
\DeclareMathOperator{\diver}{div}
\DeclareMathOperator{\sinc}{sinc}
\DeclareMathOperator{\graf}{graf}
\DeclareMathOperator{\tq}{\;t.q.\;}
\DeclareMathOperator{\disc}{disc}
\let\emptyset\varnothing
\renewcommand{\thesubsubsection}{\Alph{subsubsection}}
\setcounter{secnumdepth}{4}

\hypersetup{
    colorlinks,
    linkcolor=blue
}
\def\upint{\mathchoice%
    {\mkern13mu\overline{\vphantom{\intop}\mkern7mu}\mkern-20mu}%
    {\mkern7mu\overline{\vphantom{\intop}\mkern7mu}\mkern-14mu}%
    {\mkern7mu\overline{\vphantom{\intop}\mkern7mu}\mkern-14mu}%
    {\mkern7mu\overline{\vphantom{\intop}\mkern7mu}\mkern-14mu}%
  \int}
\def\lowint{\mkern3mu\underline{\vphantom{\intop}\mkern7mu}\mkern-10mu\int}

\setlength\parindent{0pt}

\begin{document}
\ifland
\raggedright
\begin{multicols}{3}
\fi

% AQUI COMENCEN LES FORMULES
\section{Cinem\`atica}

\subsection{Descripci\'o del moviment}
\emph{Arc d'una corba}: $s(t) = \int^t_{t_0} \|\vec{r}'(\tau)\| \dif\tau$. \\
\ci $\frac{\dif s}{\dif t} = \|\vec{v}(t)\|$. \\
\emph{Vec. unitari tg.}: $\frac{\dif \vec{r}}{\dif s} = \frac{\vec{r}'(t)}{\|\vec{r}'(t)\|} = \frac{\vec{v}(t)}{\|\vec{v}(t)\|} = \vec{t}$. \\
\emph{Vec. unitari normal}: $\vec{n} = \frac{\sfrac{\dif \vec{t}}{\dif s}}{\|\sfrac{\dif \vec{t}}{\dif s}\|}$. \\
\emph{Binormal}: $\vec{b} = \vec{t} \times \vec{n}$. \\
\emph{Curvatura}: $k = \|\frac{\dif \vec{t}}{\dif s}\|$. \\
\emph{Radi de curvatura}: $\rho = \frac{1}{k}$. \\
\emph{Centre de curvatura}: $P = \vec{r} + \rho\vec{n}$.\\
\emph{Velocitat}: $\vec{v} = v\vec{t}$. \\
\emph{Acceleraci\'o}: $\vec{a} = \frac{\dif v}{\dif t}\vec{t} + \frac{v^2}{\rho}\vec{n}$.

\subsection{Moviment circular}
\emph{Posici\'o}: $\vec{r}(t) = R(\cos \theta, \sin \theta, 0)$. \\
\emph{Celeritat}: $v = R\dot\theta = R\omega$. \\
\emph{Acceleraci\'o}: $\vec{a} = (R\alpha)\vec{t} + (R\omega^2)\vec{n}$.

\subsection{S\`olid r\'igid}
\emph{Centre instantani de rotaci\'o}: $\vec{r}_{\text{CIR}} = \vec{r}_p + \frac{\vec{\omega} \times \vec{v}_p}{\omega^2}$.

\section{Din\`amica}

\subsection{$\real$}
\emph{F. gravitat\`oria}: $\vec{F}_{ab} = -G \frac{m_a m_b}{\|\vec{r}_{ab}\|^3} \vec{r}_{ab}$. \\
\emph{F. el\`astica}: $\vec{F}_{\text{e}}(x) = -kx$. \\
\emph{Mov. osc. harm.}: $x(t) = A\sin (\omega t + \varphi_0)$, $\omega = \sqrt{\sfrac{k}{m}}$, $T = 2\pi\sqrt{\sfrac{m}{k}}$. \\
\emph{F. fregament visc\'os}: $\vec{F}_{\text{f}} = b\vec{v}$. \\
\emph{E. cin\`etica}: $E_{\text{c}} = \frac{1}{2}mv^2$. \\
\emph{E. potencial}: $U(x) \tq \frac{\dif U}{\dif x} = -F(x)$, $E_{\text{p}} = -\int_{x_0}^x F(z) \dif z$. \\
\emph{E. cin\`etica}: $E_{\text{mec}} = E_{\text{c}} + E_{\text{p}}$. \\
\emph{E. pot. el\`astica}: $\frac{1}{2}kx^2$.

\subsection{$\real^3$}
\emph{Treball (J)}: $W_{1\rightarrow 2} = \int_{x_1}^{x_2} \vec{F} \cdot\dif \vec{l}$. \\
\ci $\dif W = \vec{F} \cdot\dif \vec{l}$. \\
\emph{Pot\`encia (W)}: $P = \frac{\dif W}{\dif t} = \frac{\vec{F} \cdot\dif \vec{r}}{\dif t} = \vec{F}\cdot\vec{v}$. \\
\ci $E_{\text{c}} = \frac{1}{2}m\vec{v}^2$, $\frac{\dif E_{\text{c}}}{\dif t} = m\vec{v}\cdot\vec{a} = \vec{F}\cdot\vec{v}$. \\
\emph{Si F conservativa}: $\vec{F}(\vec{r}) = -\vec{\nabla}U(\vec{r})$. \\
\ci $E_{\text{c}}(\vec{r}_2) - E_{\text{c}}(\vec{r}_1) = W_{1\rightarrow 2}$. \\
\ci $\Delta E_{\text{mec}} = W_{\text{n.c.}}$.

\section{Din\`amica de sistemes puntuals}

\subsection{Moment lineal}
\emph{Moment lineal}: $\vec{P} = m\vec{v}$. \\
\emph{Impuls mec.}: $\vec{I} = \int_{t_1}^{t_2} F \dif t = \vec{P}(t_2) - \vec{P}(t_1) = \Delta\vec{P}$. \\
\emph{Moment d'una força respecte $O$}: $\vec{M}_O = \vec{r}\times\vec{F}$, $\vec{M}_A = \vec{AO}\times\vec{F} + \vec{M}_O$.

\subsection{Moment angular}
\emph{Moment angular}: $\vec{L}_O = \vec{r}\times m\vec{v} = \vec{r}\times\vec{P}$, $\vec{L}_A = \vec{L}_O + \vec{AO}\times\vec{P}$. \\
\ci $\vec{M}_O = \frac{\dif \vec{L}_O}{\dif t}$. \\
\emph{Si $A$ en moviment}: $\frac{\dif \vec{L}_A}{\dif t} = \vec{M}_A - \vec{v}_A\times\vec{P}$.

\subsection{Sistema de part\'icules}
\emph{Centre de massa}: $\vec{r}_{\text{CM}} = \frac{\sum m_i \vec{r}_i}{\sum m_i}$. \\
\emph{Moment lineal}: $\vec{P} = \sum \vec{P}_i = M\vec{v}_{\text{CM}}$. \\
\emph{1a llei conservaci\'o}: $\frac{\dif \vec{P}}{\dif t} = \vec{F}^{\text{ext}}$. \\
\emph{Moment angular}: $\vec{L}_O = \sum \vec{L}_{O_i} = \sum \vec{r}_i \times m_i\vec{v}_i$. \\
\ci $\frac{\dif \vec{L}_O}{\dif t} = \sum \vec{r}_i\times\vec{F} = \vec{M}_O^{\text{ext}}$. \\
\emph{A punt m\`obil}: $\vec{L}_A = \vec{L}_O + M(\vec{r}_A - \vec{r}_{\text{CM}}) \times \vec{v}_A - \vec{r}_A\times\vec{P}$. \\
\emph{$A =$ CM}: $\vec{L}_{\text{CM}} = \vec{L}_O - \vec{r}_{\text{CM}}\times\vec{P}$, $\vec{L}_O = \vec{L}_{\text{CM}} + \vec{r}_{\text{CM}}\times\vec{P}$. \\
\ci $\frac{\dif \vec{L}_A}{\dif t} = \vec{M}_A^{\text{ext}} + M(\vec{r}_A - \vec{r}_{\text{CM}})\times \vec{a}_A$. \\
\ci $\frac{\dif \vec{L}_{\text{CM}}}{\dif t} = \vec{M}_{\text{CM}}^{\text{ext}}$. \\
\emph{E. cin\`etica}: $E_{\text{c}} = \frac{1}{2} \sum m_i (\vec{v}_i - \vec{v}_{\text{CM}})^2 + M\vec{v}_{\text{CM}}^2$.

\section{Canvis de sistema de referència}
\ci $\lp \frac{\dif \vec{u}}{\dif t} \rp_S=\lp \frac{\dif \vec{u}}{\dif t} \rp_{S^{\, \prime}}+\vec{\omega}\times \vec{u}$. \\
\ci $\vec{r}_P = \vec{r}_O + \vec{r}_{P}^{\, \prime}$. \\
\ci $\vec{v}_P = \vec{v}_O + \vec{v}_{P}^{\, \prime} + \vec{\omega}\times\vec{r}_{P}^{\, \prime}$. \\
\ci $\vec{a}_P = \vec{a}_O + \vec{a}_{P}^{\, \prime} + \underbrace{2\vec{\omega}\times\vec{v}_{P}^{\, \prime}}_{\text{acc. Coriolis}} + \underbrace{\vec{\omega}\times \lp \vec{\omega}\times \vec{r}_{P}^{\, \prime} \rp}_{\text{acc. centrípeta}} + \vec{\alpha}\times\vec{r}_{P}^{\, \prime}$. \\
\ci $m\vec{a}_{P}^{\, \prime} = \underbrace{m\vec{a}_P}_{\text{f. real}} - [ \underbrace{m\vec{a}_O}_{\text{f. translació}} + \underbrace{m2\vec{\omega}\times\vec{v}_{P}^{\, \prime}}_{\text{f. Coriolis}} + \underbrace{m\vec{\omega}\times \lp \vec{\omega}\times \vec{r}_{P}^{\, \prime} \rp}_{\text{f. centrífuga}} + \underbrace{m\vec{\alpha}\times\vec{r}_{P}^{\, \prime}}_{\text{f. Euler}} ]$.


\section{Gravitació}
\emph{Camp}: $\vec{g} = -G \frac{M}{\|\vec{r}\|^3} \vec{r}$. \\
\emph{Força}: $\vec{F}_{ab} = -G \frac{m_a m_b}{\|\vec{r}_{ab}\|^3} \vec{r}_{ab}=m\vec{g}$. \\
\emph{Potencial}: $V = -G \frac{M}{\|\vec{r}\|}. \qquad V_a-V_b = \int_a^b \vec{g} \cdot \vec{\dif r}$. \\
\emph{Energia potencial}: $U = -G \frac{Mm}{\|\vec{r}\|}=mV. \quad U_a-U_b = \int_a^b \vec{F} \cdot \vec{\dif r}$.

\subsection{Lleis de Kepler}
\emph{1a}: Òrbites el·líptiques. \\
\emph{2a}: El radi vector escombra àrees iguals en temps iguals. \\
\emph{3a}: $T^2=\frac{4\pi^2}{GM}R^3$ ($R$ radi mitjà).


\ifland
\noindent\makebox[\linewidth]{\rule{\linewidth}{0.5pt}}
\vspace{3pt}
\raggedleft
{\large Nom: \underline{\hspace{5cm}}}


\end{multicols}
\vspace*{\fill}
\begin{center}
\textbf{\huge Aquest formulari no \'es complet.} \\
\textbf{\huge Estem treballant per tal d'acabar-lo a temps.}
\end{center}
\vspace*{\fill}
\fi

\end{document}
