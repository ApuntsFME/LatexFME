\documentclass[12pt,a4paper]{article}

\usepackage[utf8]{inputenc}
\usepackage[T1]{fontenc}

\usepackage[landscape,margin=0.2in,top=0.2in,bottom=0.2in]{geometry}

\usepackage[pdftex]{hyperref}
\usepackage{amsmath,amsthm,amssymb,graphicx,mathtools,tikz,hyperref,enumerate}
\usepackage{mdframed,cleveref,cancel,stackengine,mathrsfs,thmtools,pdfpages}
\usepackage{xfrac,stmaryrd,commath,units,titlesec,multicol,esint,scrextend}
\usepackage[catalan]{babel}

\titlespacing{\section}{0pt}{5pt}{0pt}
\titlespacing{\subsection}{0pt}{5pt}{0pt}
\titlespacing{\subsubsection}{0pt}{5pt}{0pt}
\setlength{\parindent}{0pt}
\pagenumbering{gobble}
\titleformat*{\section}{\large\bfseries}
\titleformat*{\subsection}{\bfseries}
\setlength{\columnseprule}{0.5pt}

\newcommand{\bimplies}{\fbx{$\implies$}}
\newcommand{\bimpliedby}{\fbx{$\impliedby$}}

\newcommand{\n}{\mathbb{N}}
\newcommand{\z}{\mathbb{Z}}
\newcommand{\q}{\mathbb{Q}}
\newcommand{\cx}{\mathbb{C}}
\newcommand{\real}{\mathbb{R}}
\newcommand{\E}{\mathbb{E}}
\newcommand{\F}{\mathbb{F}}
\newcommand{\A}{\mathbb{A}}
\newcommand{\R}{\mathcal{R}}
\newcommand{\C}{\mathscr{C}}
\newcommand{\Pa}{\mathcal{P}}
\newcommand{\Es}{\mathcal{E}}
\newcommand{\V}{\mathcal{V}}
\newcommand{\bb}[1]{\mathbb{#1}}
\let\u\relax
\newcommand{\u}[1]{\underline{#1}}
\newcommand{\pdv}[3][]{\frac{\partial^{#1} #2}{\partial #3^{#1}}}
\newcommand{\dv}[3][]{\frac{\dif^{#1} #2}{\dif #3^{#1}}}
\let\k\relax
\newcommand{\k}{\Bbbk}
\newcommand{\ita}[1]{\textit{#1}}
\newcommand\inv[1]{#1^{-1}}
\newcommand\setb[1]{\left\{#1\right\}}
\newcommand{\vbrack}[1]{\langle #1\rangle}
\newcommand{\determinant}[1]{\begin{vmatrix}#1\end{vmatrix}}
\newcommand{\Po}{\mathbb{P}}
\newcommand{\lp}{\left(}
\newcommand{\rp}{\right)}
\newcommand{\ci}{\textbullet\;}
\DeclareMathOperator{\fr}{Fr}
\DeclareMathOperator{\Id}{Id}
\DeclareMathOperator{\ext}{Ext}
\DeclareMathOperator{\inte}{Int}
\DeclareMathOperator{\rie}{Rie}
\DeclareMathOperator{\rg}{rg}
\DeclareMathOperator{\gr}{gr}
\DeclareMathOperator{\nuc}{Nuc}
\DeclareMathOperator{\car}{car}
\DeclareMathOperator{\im}{Im}
\DeclareMathOperator{\tr}{tr}
\DeclareMathOperator{\vol}{vol}
\DeclareMathOperator{\grad}{grad}
\DeclareMathOperator{\rot}{rot}
\DeclareMathOperator{\diver}{div}
\DeclareMathOperator{\sinc}{sinc}
\DeclareMathOperator{\graf}{graf}
\DeclareMathOperator{\tq}{\;t.q.\;}
\DeclareMathOperator{\disc}{disc}
\let\emptyset\varnothing
\renewcommand{\thesubsubsection}{\Alph{subsubsection}}
\setcounter{secnumdepth}{4}

\hypersetup{
    colorlinks,
    linkcolor=blue
}
\def\upint{\mathchoice%
    {\mkern13mu\overline{\vphantom{\intop}\mkern7mu}\mkern-20mu}%
    {\mkern7mu\overline{\vphantom{\intop}\mkern7mu}\mkern-14mu}%
    {\mkern7mu\overline{\vphantom{\intop}\mkern7mu}\mkern-14mu}%
    {\mkern7mu\overline{\vphantom{\intop}\mkern7mu}\mkern-14mu}%
  \int}
\def\lowint{\mkern3mu\underline{\vphantom{\intop}\mkern7mu}\mkern-10mu\int}

\setlength\parindent{0pt}

\begin{document}

\vspace*{\fill}
\begin{center}
     \Huge {\bf El document consta de 4 pàgines}
\end{center}

\quad

{\Large 
    \begin{addmargin}[10em]{0em}
        \begin{itemize}
            \item Fórmules 1a part (mecànica): pàgina 2.
            \item Fórmules 2a part (electromagnetisme): pàgina 3.
            \item Fórmules 1a part + 2a part: pàgina 4.
        \end{itemize}
    \end{addmargin}
    
    \quad
    
    \begin{addmargin}[8em]{8em}
    Els qui recupereu el parcial imprimiu les pàgines 2 i 3, els que no, podeu portar només la 3 (la pàgina 4 és pels qui no recupereu el parcial, però voleu les fórmules igualment).
    
    \end{addmargin}
}
\vspace*{\fill}

\newpage

\raggedright
\begin{multicols}{4}
{\fontsize{12}{12}\selectfont
\input{parcial1}}
\newpage
\end{multicols}

\begin{multicols}{3}
\section{Electrostàtica} 
\emph{Constants}: $k=\frac{1}{4\pi\varepsilon_0}$, $\varepsilon_0=8,854\cdot 10^{-12}$. \\ 
\emph{Camp e.}: $\vec{E} = k\frac{q_\text{A}}{\|\vec{r}_{\text{AB}}\|^3}\vec{r}_{\text{AB}}$, $\vec{E}(\vec{r}) = \int_\nu \frac{k\rho(\vec{r'})}{\|\vec{r}-\vec{r'}\|^3}(\vec{r}-\vec{r'}) \dif V'$. \\ 
\emph{Força (Coulomb)}: $\vec{F}_{\text{AB}} = k\frac{q_{\text{A}}q_{\text{B}}}{\|\vec{r}_{\text{AB}}\|^3}\vec{r}_{\text{AB}}=q_\text{B}\vec{E_\text{A}}$. \\ 
\emph{E. potencial}: $U = k\frac{q_1q_2}{\|\vec{r}\|}$, $U = \frac{1}{2}\int_\nu \rho(\vec{r})V(\vec{r})\dif V$. \\
\emph{Potencial}: $V = k \frac{q}{r}$, $V(\vec{r}) = \int_\nu \frac{k\rho(\vec{r'})}{\|\vec{r}-\vec{r'}\|}\dif V'$; $V=-\int_{\infty}^r \vec{E} \dif \vec{r}$, $V_{\infty}=0$, $V_A -V_B = \int_A^B \vec{E} \dif \vec{r}$. \\
\ci $\vec{E}=-\nabla V$. \\
\ci $\Delta V = \frac{\Delta U}{q_0}$. \\
\emph{Camp conservatiu}: $\oint_C \vec{E} \cdot \vec{\dif l} = 0$. \\
\emph{L. Gauss}: $\oint_S \vec{E}\cdot \vec{\dif S} = \frac{1}{\varepsilon_0} \int_\nu \rho \dif V = \frac{Q_\text{int}}{\varepsilon_0}$.\\
\emph{Discont. superf.}: $\boldsymbol{n} \cdot (\vec{E}_+ - \vec{E}_-) = \frac{\sigma}{\varepsilon_0}$.

\subsection{Dipols}
\emph{Moment dipolar}: $\vec{p} = 2aq\vec{u}$. \\
\emph{Potencial}: $V(\vec{r}) = k\frac{\cos \theta}{r^2}\cdot \vec{p} \approx \frac{k\vec{p}\cdot\vec{r}}{\vec{r^3}} (a \ll r)$. \\
\emph{Camp e.}: $\vec{E} = \frac{3k\cdot (\vec{p}\cdot\vec{r})\vec{r}}{r^5} - \frac{k\vec{p}}{r^3}$. \\
\emph{Força}: $\vec{F} = \vec{\nabla}(\vec{p}\cdot \vec{E})$. \\
\emph{Moment}: $\vec{M} = \vec{p}\times \vec{E}$.



\subsection{Condensadors}
\emph{Capacitat}: $C = \frac{Q}{|V_1-V_2|}$. \\
\emph{Intensitat}: $I=C\frac{\dif V}{\dif t} = \frac{\dif Q}{\dif t}$.\\
\emph{Energia}: $U = \frac{1}{2}C (V_1-V_2)^2$.\\
\emph{Càrrega RC}: $V(t) = \varepsilon (1 -  e^{-\frac{t}{RC}})$, 
                    $I = \frac{\varepsilon}{R} e^{\frac{-t}{RC}}$.

\section{Electrocinètica}
\emph{Conserv. càrrega}: $\frac{\dif}{\dif t} \int_\nu \rho \dif V + 
                          \oint_{\partial V} \vec{j} \cdot \vec{\dif S} = 0$,
                          $\frac{\partial\rho}{\partial t} + \nabla \vec{j} = 0 $.\\
\emph{Intensitat} $I = \int_S \vec{j} \cdot \vec{\dif S}$\\
\emph{L. Ohm}: $V=IR$, $\vec{j} = \gamma \vec{E}$\\
\emph{Conductors}: $R = \int_a^b \frac{\dif l}{S \gamma} =  \frac{l}{S\gamma} = \frac{rl}{S}$. \\
\emph{Conductivitat-Resistivitat}: $\gamma = \frac{1}{r}$. \\
\emph{Potència}: $P=\frac{E}{t}=VI$, $P = \int_{V} \vec{j} \cdot \vec{E} \dif V$. \\
\emph{Treball}: $W=\int_{r_1}^{r_2} F(\vec{r}) \cdot \dif \vec{r}$.

\section{Magnetostàtica}
\emph{Camp m.}: $\vec{B} = \frac{\mu_0}{4\pi}\frac{q\cdot \vec{v}\times (\vec{r} - \vec{r'})}{\|\vec{r} -\vec{r'}\|^3}$, $\mu_0=4\pi \cdot 10^{-7}$. \\
\emph{Camp m. vol.}: $\vec{B}(\vec{r}) = \frac{\mu_0}{4\pi}\int_V
                     \frac{ \vec{j}(\vec{r}') \times (\vec{r} - \vec{r'})}
                     {\|\vec{r} -\vec{r'}\|^3} \dif V$.\\
\emph{Camp m. fil.}: $\vec{B}(\vec{r}) = \frac{\mu_0 I}{4\pi}\int_C
                     \frac{ \vec{\dif l} \times (\vec{r} - \vec{r'})}
                     {\|\vec{r} -\vec{r'}\|^3}$.\\
\emph{F. Lorentz}: $\vec{F} = q \cdot\vec{v}\times\vec{B} = \vec{Il} \times \vec{B}$. \\

\emph{F. sobre corrent}: $\vec{F} = \int_V \vec{j} \times \vec{B} \dif V$.\\
\emph{Moment sobre corrent}: $\vec{N_0} = \int_V \vec{r} \times (\vec{j} \times
                            \vec{B}) \dif V$.\\
\emph{Camp Solenoidal}: $\oint_S \vec{B} \cdot \vec{\dif S} = 0$, $\nabla \cdot \vec{B} = 0$.\\
\emph{L. Ampère}: $\oint_C \vec{B}\cdot \vec{\dif l} = I_{\text{int}} \mu_0$, $\nabla \times \vec{B} = \mu_0 \vec{j}$.

\section{Inductància}
\emph{Flux}: $\Phi = \int_S \vec{B}\cdot \vec{\dif S}$. \\
\emph{L. Faraday}: $\varepsilon = -\frac{\dif \Phi_{\text{m}}}{\dif t} = \oint_{\partial S} \vec{E} \cdot \vec{\dif l} = -\frac{\dif}{\dif t}\int_S \vec{B} \cdot \vec{\dif S}$.

\subsection{Bobines}
\emph{Camp m.}: $B = \mu_0 n I$, $n = N/l$.\\
\emph{Autoinductància}: $L = \frac{\phi}{I} = \mu_0 n^2 V$.\\
\emph{Potencial}: $\varepsilon = L \frac{\dif I}{\dif t}$.\\
\emph{Càrrega RL}: $I(t) = \frac{\varepsilon}{R} \lp 1 - e^{-\frac{Rt}{L}}\rp$.\\
\emph{Energia RL}: $U = \frac{1}{2}LI^2$.

\section{Maxwell}
\emph{Gauss}: $\nabla \cdot \vec{E} = \frac{\rho}{\varepsilon_0} $, $\oint_{\partial V} \vec{E}\cdot \vec{\dif S} = \frac{1}{\varepsilon_0} \int_\nu \rho \dif V$.\\
\emph{Faraday}: $\nabla \times \vec{E} +  \frac{\partial{\vec{B}}}{\partial t} = \vec{0}$,
                $\oint_{\partial S} \vec{E} \cdot \vec{\dif l} + 
                \frac{\dif}{\dif t} \int_S \vec{B} \cdot \vec{\dif S} = 0$.\\
\emph{Gauss m.}: $\nabla \cdot \vec{B} = 0$, $\oint_{\partial V} \vec{B} \cdot
                \vec{\dif S} = 0$.\\
\emph{Ampère-Maxwell}: $\nabla \times \vec{B} = \mu_0 \vec{j} + \varepsilon_0\mu_0
                    \frac{\partial{\vec{E}}}{\partial t}$, $\oint_{\partial{S}}
                    \vec{B} \cdot \vec{\dif l} = \mu_0 
                    \int_S \vec{j} \cdot \vec{\dif S} + \varepsilon_0 \mu_0
                    \frac{\dif}{\dif t} \int_S \vec{E} \cdot \vec{\dif S}$.

\addtocounter{section}{1}
\noindent\makebox[\linewidth]{\rule{\linewidth}{0.5pt}}

\subsection{Integrals}
\ci $\int \frac{1}{\sqrt{a+x^2}}\dif x = \log (\sqrt{a+x^2} + x)$. \\
\ci $\int \frac{x}{\sqrt{a+x^2}}\dif x = \sqrt{a+x^2}$. \\
\ci $\int \frac{1}{\sqrt{a+x^2}^3}\dif x = \frac{x}{a\sqrt{a+x^2}}$. \\
\ci $\int \frac{x}{\sqrt{a+x^2}^3}\dif x = -\frac{1}{\sqrt{a+x^2}}$.\\
\ci $\int \sec \dif x = \ln \lp |\tan + \sec| \rp$.

\subsection{Canvis de variables}
\emph{Polars a $\real^2$}: $\int_U f( x,y ) \dif x\dif y = \int_V f( r\cos\varphi, r\sin\varphi )r \dif r\dif \varphi$. \\
\emph{Cilíndriques a $\real^3$}: $\int_U f( x,y,z ) \dif x\dif y\dif z = \int_V f( \rho\cos\varphi, \rho\sin\varphi, z )\rho \dif \rho\dif \varphi\dif z$. \\
\emph{Esfèriques a $\real^3$}: $\int_U f( x,y,z ) \dif x\dif y\dif z = \int_V f ( r\cos\varphi\sin\theta, r\sin\varphi\sin\theta, r\cos\theta )$ $r^2\sin\theta \dif r\dif \varphi\dif \theta$.

\subsection{EDOs}
\ci $\frac{1}{v}\frac{\dif v}{\dif t} = \frac{\dif\, (\ln v)}{\dif t}$. \\
\emph{Fórmules}: $a\dot{v}+bv+c=0$; $\frac{a}{b}\dot{v}+v+\frac{c}{b}=0$; ($w=v+\frac{c}{b}$), $\frac{a}{b}\dot{w}+w=0$; $\frac{a}{b}x+1=0$; $x=\frac{-b}{a}$; $w=k e^{\frac{-b}{a}t}=v+\frac{c}{b}$; $v=ke^{\frac{-b}{a}t}-\frac{c}{b}$.

\subsection{Altres}
\emph{Esfera}: $S=4\pi r^2$, $V=\frac{4}{3}\pi r^3$.


\vspace{15pt}
\raggedleft
{\Large Nom: \underline{\hspace{6cm}}}
\end{multicols}

\includepdf[pages=1]{main_aux.pdf}

\end{document}
