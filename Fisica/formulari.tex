\documentclass[12pt,a4paper]{article}

\usepackage[utf8]{inputenc}
\usepackage[T1]{fontenc}

\usepackage[landscape,margin=0.2in,top=0.2in,bottom=0.2in]{geometry}

\usepackage[pdftex]{hyperref}
\usepackage{amsmath,amsthm,amssymb,graphicx,mathtools,tikz,hyperref,enumerate}
\usepackage{mdframed,cleveref,cancel,stackengine,mathrsfs,thmtools,pdfpages}
\usepackage{xfrac,stmaryrd,commath,units,titlesec,multicol,esint,scrextend,datetime}
\usepackage[catalan]{babel}

\titlespacing{\section}{0pt}{5pt}{0pt}
\titlespacing{\subsection}{0pt}{5pt}{0pt}
\titlespacing{\subsubsection}{0pt}{5pt}{0pt}
\setlength{\parindent}{0pt}
\pagenumbering{gobble}
\titleformat*{\section}{\large\bfseries}
\titleformat*{\subsection}{\bfseries}
\setlength{\columnseprule}{0.5pt}

\newcommand{\bimplies}{\fbx{$\implies$}}
\newcommand{\bimpliedby}{\fbx{$\impliedby$}}

\newcommand{\n}{\mathbb{N}}
\newcommand{\z}{\mathbb{Z}}
\newcommand{\q}{\mathbb{Q}}
\newcommand{\cx}{\mathbb{C}}
\newcommand{\real}{\mathbb{R}}
\newcommand{\E}{\mathbb{E}}
\newcommand{\F}{\mathbb{F}}
\newcommand{\A}{\mathbb{A}}
\newcommand{\R}{\mathcal{R}}
\newcommand{\C}{\mathscr{C}}
\newcommand{\Pa}{\mathcal{P}}
\newcommand{\Es}{\mathcal{E}}
\newcommand{\V}{\mathcal{V}}
\newcommand{\bb}[1]{\mathbb{#1}}
\let\u\relax
\newcommand{\u}[1]{\underline{#1}}
\newcommand{\pdv}[3][]{\frac{\partial^{#1} #2}{\partial #3^{#1}}}
\newcommand{\dv}[3][]{\frac{\dif^{#1} #2}{\dif #3^{#1}}}
\let\k\relax
\newcommand{\k}{\Bbbk}
\newcommand{\ita}[1]{\textit{#1}}
\newcommand\inv[1]{#1^{-1}}
\newcommand\setb[1]{\left\{#1\right\}}
\newcommand{\vbrack}[1]{\langle #1\rangle}
\newcommand{\determinant}[1]{\begin{vmatrix}#1\end{vmatrix}}
\newcommand{\Po}{\mathbb{P}}
\newcommand{\lp}{\left(}
\newcommand{\rp}{\right)}
\newcommand{\ci}{\textbullet\;}
\DeclareMathOperator{\fr}{Fr}
\DeclareMathOperator{\Id}{Id}
\DeclareMathOperator{\ext}{Ext}
\DeclareMathOperator{\inte}{Int}
\DeclareMathOperator{\rie}{Rie}
\DeclareMathOperator{\rg}{rg}
\DeclareMathOperator{\gr}{gr}
\DeclareMathOperator{\nuc}{Nuc}
\DeclareMathOperator{\car}{car}
\DeclareMathOperator{\im}{Im}
\DeclareMathOperator{\tr}{tr}
\DeclareMathOperator{\vol}{vol}
\DeclareMathOperator{\grad}{grad}
\DeclareMathOperator{\rot}{rot}
\DeclareMathOperator{\diver}{div}
\DeclareMathOperator{\sinc}{sinc}
\DeclareMathOperator{\graf}{graf}
\DeclareMathOperator{\tq}{\;t.q.\;}
\DeclareMathOperator{\disc}{disc}
\let\emptyset\varnothing
\renewcommand{\thesubsubsection}{\Alph{subsubsection}}
\setcounter{secnumdepth}{4}

\hypersetup{
    colorlinks,
    linkcolor=blue
}
\def\upint{\mathchoice%
    {\mkern13mu\overline{\vphantom{\intop}\mkern7mu}\mkern-20mu}%
    {\mkern7mu\overline{\vphantom{\intop}\mkern7mu}\mkern-14mu}%
    {\mkern7mu\overline{\vphantom{\intop}\mkern7mu}\mkern-14mu}%
    {\mkern7mu\overline{\vphantom{\intop}\mkern7mu}\mkern-14mu}%
  \int}
\def\lowint{\mkern3mu\underline{\vphantom{\intop}\mkern7mu}\mkern-10mu\int}

\setlength\parindent{0pt}

\begin{document}

\vspace*{\fill}
\begin{center}
     \Huge {\bf El document consta de 4 pàgines}
\end{center}

\quad

{\Large 
    \begin{addmargin}[10em]{0em}
        \begin{itemize}
            \item Fórmules 1a part (mecànica): pàgina 2.
            \item Fórmules 2a part (electromagnetisme): pàgina 3.
            \item Fórmules 1a part + 2a part: pàgina 4.
        \end{itemize}
    \end{addmargin}
    
    \quad
    
    \begin{addmargin}[8em]{8em}
    Els qui recupereu el parcial imprimiu les pàgines 2 i 3, els que no, podeu portar només la 3 (la pàgina 4 és pels qui no recupereu el parcial, però voleu algunes fórmules del parcial igualment).
    
    \quad
    
    Assegura't que tens l'última versió el dia de l'examen!
    
    \quad
    
    \begin{flushright}
        {\bf Última modificació: \currenttime, \today}
    \end{flushright}
    
    \end{addmargin}
}
\vspace*{\fill}

\newpage

\raggedright
\begin{multicols}{4}
{\fontsize{12}{12}\selectfont
%<*retallat>
\section{Cinem\`atica}

\subsection{Descripci\'o del moviment}
\emph{Arc d'una corba}: $s(t) = \int^t_{t_0} \|\vec{r}'(\tau)\| \dif\tau$. \\
\ci $\frac{\dif s}{\dif t} = \|\vec{v}(t)\|$. \\
\emph{Vec. unitari tg.}: $\frac{\dif \vec{r}}{\dif s} = \frac{\vec{r}'(t)}{\|\vec{r}'(t)\|} = \frac{\vec{v}(t)}{\|\vec{v}(t)\|} = \vec{t}$. \\
\emph{Vec. unitari normal}: $\vec{n} = \frac{\sfrac{\dif \vec{t}}{\dif s}}{\|\sfrac{\dif \vec{t}}{\dif s}\|}$. \\
\emph{Binormal}: $\vec{b} = \vec{t} \times \vec{n}$. \\
\emph{Curvatura}: $\kappa = \|\frac{\dif \vec{t}}{\dif s}\|$. \\
\emph{Radi de curvatura}: $\rho = \frac{1}{\kappa}$. \\
\emph{Centre de curvatura}: $P = \vec{r} + \rho\vec{n}$.\\
\emph{Velocitat}: $\vec{v} = v\vec{t}$. \\
\emph{Acceleraci\'o}: $\vec{a} = \frac{\dif v}{\dif t}\vec{t} + \frac{v^2}{\rho}\vec{n}$.

\subsection{Moviment circular}
\emph{Posici\'o}: $\vec{r}(t) = R(\cos \theta, \sin \theta, 0)$. \\
\emph{Celeritat}: $v = R\dot\theta = R\omega$. \\
\emph{Acceleraci\'o}: $\vec{a} = (R\alpha)\vec{t} + (R\omega^2)\vec{n}$.

\subsection{S\`olid r\'igid}
\emph{Moviment fixant origen $P$ del sòlid}: $\vec{v} = \vec{\omega} \times \vec{r}$,\\
$\vec{a} =  \vec{\alpha} \times \vec{r} - \omega^2 \vec{r}$.\\
\emph{Centre instantani de rotaci\'o}: $\vec{r}_{\text{CIR}} = \vec{r}_p + \frac{\vec{\omega} \times \vec{v}_p}{\omega^2}$.


\section{Din\`amica}

\subsection{$\real$}
\emph{F. gravitat\`oria}: $\vec{F}_{ab} = -G \frac{m_a m_b}{\|\vec{r}_{ab}\|^3} \vec{r}_{ab}$. \\
\emph{F. el\`astica}: $\vec{F}_{\text{e}}(x) = -kx$. \\
\emph{Mov. osc. harm.}: $x(t) = A\sin (\omega t + \varphi_0)$, $\omega = \sqrt{\sfrac{k}{m}}$, $T = 2\pi\sqrt{\sfrac{m}{k}}$. \\
\emph{F. fregament estàtica}: $|\vec{F}_\mu| \leq \mu |\vec{N}|$\\
\emph{F. fregament dinàmica}: $|\vec{F}_\mu| = \mu' |\vec{N}|$\\
\emph{F. fregament visc\'os}: $\vec{F}_{\text{f}} = b\vec{v}$. \\
\emph{E. cin\`etica}: $E_{\text{c}} = \frac{1}{2}mv^2$. \\
\emph{E. potencial}: $U(x) \tq \frac{\dif U}{\dif x} = -F(x)$, $E_{\text{p}} = -\int_{x_0}^x F(z) \dif z$. \\
\emph{E. mecànica}: $E_{\text{mec}} = E_{\text{c}} + E_{\text{p}}$. \\
\emph{E. pot. el\`astica}: $\frac{1}{2}kx^2$.

\subsection{$\real^3$}
\emph{Treball (J)}: $W_{1\rightarrow 2} = \int_{x_1}^{x_2} \vec{F} \cdot\dif \vec{l}$. \\
\ci $\dif W = \vec{F} \cdot\dif \vec{l}$. \\
\emph{Pot\`encia (W)}: $P = \frac{\dif W}{\dif t} = \frac{\vec{F} \cdot\dif \vec{r}}{\dif t} = \vec{F}\cdot\vec{v}$. \\
\ci $E_{\text{c}} = \frac{1}{2}m\vec{v}^2$, $\frac{\dif E_{\text{c}}}{\dif t} = m\vec{v}\cdot\vec{a} = \vec{F}\cdot\vec{v}$. \\
\emph{Si F conservativa}: $\vec{F}(\vec{r}) = -\vec{\nabla}U(\vec{r})$. \\
\ci $E_{\text{c}}(\vec{r}_2) - E_{\text{c}}(\vec{r}_1) = W_{1\rightarrow 2}$. \\
\ci $\Delta E_{\text{mec}} = W_{\text{n.c.}}$.

\section{D. de sistemes puntuals}

\subsection{Moment lineal}
\emph{Moment lineal}: $\vec{P} = m\vec{v}$. \\
\emph{Impuls mec.}: $\vec{I} = \int_{t_1}^{t_2} F \dif t = \vec{P}(t_2) - \vec{P}(t_1) = \Delta\vec{P}$. \\
\emph{Moment d'una força respecte $O$}: $\vec{M}_O = \vec{r}\times\vec{F}$, $\vec{M}_A = \vec{AO}\times\vec{F} + \vec{M}_O$.

\subsection{Moment angular}
\emph{Moment angular}: $\vec{L}_O = \vec{r}\times m\vec{v} = \vec{r}\times\vec{P}$, $\vec{L}_A = \vec{L}_O + \vec{AO}\times\vec{P}$. \\
\ci $\vec{M}_O = \frac{\dif \vec{L}_O}{\dif t}$. \\
\emph{Si $A$ en moviment}: $\frac{\dif \vec{L}_A}{\dif t} = \vec{M}_A - \vec{v}_A\times\vec{P}$.\\
\emph{Impuls angular}: $\vec{J} = \int_{t_1}^{t_2} \vec{M}_O\dif t$.

\subsection{Sòlid rígid}
\emph{Moment d'inèrcia}: $I = \sum m_i d_i^2$. \\
\ci $\vec{L_e} = I_e\vec{\omega},\, \frac{\dif L_e}{\dif t} = I_e\alpha = M_e$. \\
\ci $I_{\text{CM}} = \int r^2 \dif m$. \\
\emph{Teo. Steiner}: $I_O = I_{\text{CM}} + Md^2$. \\
\emph{Eix fix}: $E_\text{c} = \frac{1}{2} I_e \omega^2 $\\
\emph{Eix mòbil}: $E_{\text{c}} = \frac{1}{2} I_\text{CM}\omega^2 + \frac{1}{2}  M\vec{v}_{\text{CM}}^2$.\\
\emph{E. potencial}: $U = U_{\text{CM}}$.\\
\emph{Gravetat}: $\vec{M}_\text{grav} = \vec{r}_\text{CM} \times \lp m \vec{g} \rp$.

\subsubsection*{Moments d'inèrcia}
    Respecte al CM:\\
    \emph{Barra}: $\frac{1}{12}mL^2$. \\
    \emph{Pla (eix perpendicular)}: $\frac{1}{12}m(h^2+w^2)$. \\
    \emph{Pla (eix paral·lel)}: $\frac{1}{12}mL^2$. \\
    \emph{Corona circular/Tub}: $\frac{1}{2}m(R^2+r^2)$. \\
    %\begin{center}
        \begin{tabular}{|c|c|c|} 
            \hline
            & Sòlid & Buit \\
            \hline
            \emph{Disc/Cil.} & $\sfrac{1}{2} \,\, mr^2$ & $mr^2$ \\ 
            \emph{Esfera} & $\sfrac{2}{5}\,\, mr^2$ & $\sfrac{2}{3}\,\, mr^2$ \\ 
            \emph{Cub} & $\sfrac{1}{6}\,\, ms^2$ & \\ 
            \emph{Con} & $\sfrac{3}{10}\,\, mr^2$ & $\sfrac{1}{2}\,\, mr^2$\\
            \hline
        \end{tabular}
    %\end{center}
  
\subsection{Sistema de part\'icules}
\emph{Centre de massa}: $\vec{r}_{\text{CM}} = \frac{\sum m_i \vec{r}_i}{\sum m_i}$. \\
\emph{Moment lineal}: $\vec{P} = \sum \vec{P}_i = M\vec{v}_{\text{CM}}$. \\
\emph{1a llei conservaci\'o}: $\frac{\dif \vec{P}}{\dif t} = \vec{F}^{\text{ext}} = 
  M \vec{a}_\text{CM}$. \\
\emph{Moment angular}: $\vec{L}_O = \sum \vec{L}_{O_i} = \sum \vec{r}_i \times m_i\vec{v}_i$. \\
\ci $\frac{\dif \vec{L}_O}{\dif t} = \sum \vec{r}_i\times\vec{F} = \vec{M}_O^{\text{ext}}$. \\
\emph{A punt m\`obil}: $\vec{L}_A = \vec{L}_O + M(\vec{r}_A - \vec{r}_{\text{CM}}) \times \vec{v}_A - \vec{r}_A\times\vec{P}$. \\
\ci $\frac{\dif \vec{L}_A}{\dif t} = \vec{M}_A^{\text{ext}} + M(\vec{r}_A - \vec{r}_{\text{CM}})\times \vec{a}_A$. \\
\emph{$A =$ CM o punt fix}: $\vec{L}_{\text{A}} = \vec{L}_O - \vec{r}_{\text{A}}\times\vec{P}$, $\vec{L}_O = \vec{L}_{\text{A}} + \vec{r}_{\text{A}}\times\vec{P}$. \\
\ci $\frac{\dif \vec{L}_{\text{A}}}{\dif t} = \vec{M}_{\text{A}}^{\text{ext}}$. \\
\emph{Energia cinètica}: $E_{\text{c}} = \frac{1}{2} \sum m_i (\vec{v}_i - \vec{v}_{\text{CM}})^2 + \frac{1}{2}  M\vec{v}_{\text{CM}}^2$.\\
\emph{$W_{F_{\text{int}}} = 0$}: $\frac{\dif E}{\dif t}  =\sum \vec{F}^{\text{ext}}_\text{nc}\cdot\vec{v}_i$.

\section{Percussions i xocs}
\emph{Canvi m. lineal percussió}: $\Delta \vec{P} = \vec{I}$.\\
\emph{Canvi m. angular percussió}: Si $A$ és un punt fix o el CM, $\vec{J}_A = \Delta \vec{L}_A$.\\
\emph{Per un sòlid rígid}: $I_A \Delta \omega = J_A = \vec{r}\times \vec{I}$.\\
\emph{Coeficient de percussió (xoc)}: $e = - \frac{v_{1\text{f}} - v_{2\text{f}}}{v_{1\text{i}} - v_{2\text{i}}}=-\frac{v_{\text{rel,f}}}{v_{\text{rel,i}}}$.\\
\emph{Energia xoc partícules}: $\Delta E_\text{c} = -\frac{1}{2} \mu \lp 1 - e^2 \rp v_\text{rel,i}^2$, $\mu=\frac{m_1m_2}{m_1+m_2}$.


\section{Canvis de s. de ref.}
\ci $\lp \frac{\dif \vec{u}}{\dif t} \rp_S=\lp \frac{\dif \vec{u}}{\dif t} \rp_{S^{\, \prime}}+\vec{\omega}\times \vec{u}$. \\
\ci $\vec{r}_P = \vec{r}_{O'} + \vec{r}_{P}^{\, \prime}$. \\
\ci $\vec{v}_P = \vec{v}_{O'} + \vec{v}_{P}^{\, \prime} + \vec{\omega}\times\vec{r}_{P}^{\, \prime}$. \\
\ci $\vec{a}_P = \vec{a}_{O'} + \vec{a}_{P}^{\, \prime} + \underbrace{2\vec{\omega}\times\vec{v}_{P}^{\, \prime}}_{\text{acc. Coriolis}} + \underbrace{\vec{\omega}\times \lp \vec{\omega}\times \vec{r}_{P}^{\, \prime} \rp}_{\text{acc. centrípeta}} + \vec{\alpha}\times\vec{r}_{P}^{\, \prime}$. \\
\ci $m\vec{a}_{P}^{\, \prime} = \underbrace{m\vec{a}_P}_{\text{f. real}} - [\underbrace{m\vec{a}_{O'}}_{\text{f. translació}} + \underbrace{m2\vec{\omega}\times\vec{v}_{P}^{\, \prime}}_{\text{f. Coriolis}} +$ \\
\qquad $+\underbrace{m\vec{\omega}\times \lp \vec{\omega}\times \vec{r}_{P}^{\, \prime} \rp}_{\text{f. centrífuga}} + \underbrace{m\vec{\alpha}\times\vec{r}_{P}^{\, \prime}}_{\text{f. Euler}} ]$.
%</retallat>

\section{Gravitació}
\emph{Camp}: $\vec{g} = -G \frac{M}{\|\vec{r}\|^3} \vec{r}$. \\
\emph{Força}: $\vec{F}_{ab} = -G \frac{m_a m_b}{\|\vec{r}_{ab}\|^3} \vec{r}_{ab}=m\vec{g}$. \\
\emph{Potencial}: $V = -G \frac{M}{\|\vec{r}\|}$,  $V_a-V_b = \int_a^b \vec{g} \cdot \vec{\dif r}$. \\
\emph{Energia potencial}: $U = -G \frac{Mm}{\|\vec{r}\|}=mV. \quad U_a-U_b = \int_a^b \vec{F} \cdot \vec{\dif r}$.\\
\emph{Potencial efectiu}: $U_\text{ef}(r) = \frac{L^2}{2mr^2} - \frac{k}{r}$.\\
\emph{Energia mecànica}: $E_{\text{mec}} = \frac{1}{2} mr'^2 + U_\text{ef}(r)$.\\
\emph{Tma. de Gauss}: $\oiint\limits_{\dif V} g \cdot \dif{\vec{S}} = -4\pi GM_{int} $.
\subsection{Lleis de Kepler}
\emph{2a}: El radi vector escombra àrees iguals en temps iguals. \\
\emph{3a}: $T^2=\frac{4\pi^2}{GM}R^3$ ($R$ radi mitjà).

\subsection{Problema de Kepler}
\emph{Excentricitat}: $e = \sqrt{1 + \frac{2EL^2}{mk^2}}$.\\
\emph{Posició}: $\frac{1}{r} = \frac{mk}{L^2} \lp 1 + e \cos \theta \rp$.

\subsection{Problema dels dos cossos}
\emph{Substitucions a Kepler}: $m = \mu = \frac{m_1m_2}{m_1+m_2},\, k = Gm_1m_2, \, L = L_\text{CM}$.

\noindent\makebox[\linewidth]{\rule{\linewidth}{0.5pt}}
{\fontsize{7}{12}\selectfont $x(t) \sim \theta(t)$, $v \sim \omega$, $a \sim \alpha$, $m \sim I$, $F = ma \sim M = I\alpha$, $W = \int F \dif x \sim W = \int M \dif \theta$, $E_c = \frac{1}{2}mv^2 \sim E_c = \frac{1}{2} I\omega^2$, $U(x) = -\int_{x_0}^x F(x) \dif x \sim U(\theta) = -\int_{\theta_0}^\theta M \dif \theta$, $\vec{P} = mv \sim \vec{L} = I\omega$. }}
\newpage
\end{multicols}

\begin{multicols}{3}
\section{Electrostàtica} 
\emph{Constants}: $k=\frac{1}{4\pi\varepsilon_0}$, $\varepsilon_0=8,854\cdot 10^{-12}$. \\ 
\emph{Camp e.}: $\vec{E} = k\frac{q_\text{A}}{\|\vec{r}_{\text{AB}}\|^3}\vec{r}_{\text{AB}}$, $\vec{E}(\vec{r}) = \int_\nu \frac{k\rho(\vec{r'})}{\|\vec{r}-\vec{r'}\|^3}(\vec{r}-\vec{r'}) \dif V'$. \\ 
\emph{Força (Coulomb)}: $\vec{F}_{\text{AB}} = k\frac{q_{\text{A}}q_{\text{B}}}{\|\vec{r}_{\text{AB}}\|^3}\vec{r}_{\text{AB}}=q_\text{B}\vec{E_\text{A}}$. \\ 
\emph{E. potencial}: $U = k\frac{q_1q_2}{\|\vec{r}\|},\, U = \frac{1}{2}\int_\nu \rho(\vec{r})V(\vec{r})\dif V$. \\
\emph{Potencial}: $V = k \frac{q}{r},\, V(\vec{r}) = \int_\nu \frac{k\rho(\vec{r'})}{\|\vec{r}-\vec{r'}\|}\dif V'$; $V=-\int_{\infty}^r \vec{E} \dif \vec{r}$, $V_{\infty}=0$, $V_A -V_B = \int_A^B \vec{E} \dif \vec{r}$. \\
\ci $\vec{E}=-\nabla V$. \\
\ci $\Delta V = \frac{\Delta U}{q_0}$. \\
\emph{Camp conservatiu}: $\oint_C \vec{E} \cdot \vec{\dif l} = 0$. \\
\emph{L. Gauss}: $\oint_S \vec{E}\cdot \vec{\dif S} = \frac{1}{\varepsilon_0} \int_\nu \rho \dif V = \frac{Q_\text{int}}{\varepsilon_0}$.\\
\emph{Discont. superf.}: $\boldsymbol{n} \cdot (\vec{E}_+ - \vec{E}_-) = \frac{\sigma}{\varepsilon_0}$.

\subsection{Dipols}
\emph{Moment dipolar}: $\vec{p} = 2aq\vec{u}$. \\
\emph{Potencial}: $V(\vec{r}) = k\frac{\cos \theta}{r^2}\cdot \vec{p} \approx \frac{k\vec{p}\cdot\vec{r}}{\vec{r^3}} (a \ll r)$. \\
\emph{Camp e.}: $\vec{E} = \frac{3k\cdot (\vec{p}\cdot\vec{r})\vec{r}}{r^5} - \frac{k\vec{p}}{r^3}$. \\
\emph{Força}: $\vec{F} = \vec{\nabla}(\vec{p}\cdot \vec{E})$. \\
\emph{Moment}: $\vec{M} = \vec{p}\times \vec{E}$.



\subsection{Condensadors}
\emph{Capacitat}: $C = \frac{Q}{|V_1-V_2|}$. \\
\emph{Intensitat}: $I=C\frac{\dif V}{\dif t} = \frac{\dif Q}{\dif t}$.\\
\emph{Energia}: $U = \frac{1}{2}C (V_1-V_2)^2$.\\
\emph{Càrrega RC}: $V(t) = \varepsilon (1 -  e^{-\frac{t}{RC}})$, 
                    $I = \frac{\varepsilon}{R} e^{\frac{-t}{RC}}$.

\section{Electrocinètica}
\emph{Conserv. càrrega}: $\frac{\dif}{\dif t} \int_\nu \rho \dif V + 
                          \oint_{\partial V} \vec{j} \cdot \vec{\dif S} = 0$,
                          $\frac{\partial\rho}{\partial t} + \nabla \vec{j} = 0 $.\\
\emph{Intensitat} $I = \int_S \vec{j} \cdot \vec{\dif S}$\\
\emph{L. Ohm}: $V=IR$, $\vec{j} = \gamma \vec{E}$\\
\emph{Conductors}: $R = \frac{l}{S\gamma} = \frac{rl}{S}$. \\
\emph{Conductivitat-Resistivitat}: $\gamma = \frac{1}{r}$. \\
\emph{Potència}: $P=\frac{E}{t}=VI$, $P = \int_{V} \vec{j} \cdot \vec{E} \dif V$. \\
\emph{Treball}: $W=\int_{r_1}^{r_2} F(\vec{r}) \cdot \dif \vec{r}$.

\section{Magnetostàtica}
\emph{Camp m.}: $\vec{B} = \frac{\mu_0}{4\pi}\frac{q\cdot \vec{v}\times (\vec{r} - \vec{r'})}{\|\vec{r} -\vec{r'}\|^3}$, $\mu_0=4\pi \cdot 10^{-7}$. \\
\emph{Camp m. vol.}: $\vec{B}(\vec{r}) = \frac{\mu_0}{4\pi}\int_V
                     \frac{ \vec{j}(\vec{r}') \times (\vec{r} - \vec{r'})}
                     {\|\vec{r} -\vec{r'}\|^3} \dif V$.\\
\emph{Camp m. fil.}: $\vec{B}(\vec{r}) = \frac{\mu_0 I}{4\pi}\int_C
                     \frac{ \vec{\dif l} \times (\vec{r} - \vec{r'})}
                     {\|\vec{r} -\vec{r'}\|^3}$.\\
\emph{F. Lorentz}: $\vec{F} = q \cdot\vec{v}\times\vec{B} = \vec{Il} \times \vec{B}$. \\

\emph{F. sobre corrent}: $\vec{F} = \int_V \vec{j} \times \vec{B} \dif V$.\\
\emph{Moment sobre corrent}: $\vec{N_0} = \int_V \vec{r} \times (\vec{j} \times
                            \vec{B}) \dif V$.\\
\emph{Camp Solenoidal}: $\oint_S \vec{B} \cdot \vec{\dif S} = 0$, $\nabla \cdot \vec{B} = 0$.\\
\emph{L. Ampère}: $\oint_C \vec{B}\cdot \vec{\dif l} = I_{\text{int}} \mu_0, \, \nabla \times \vec{B} = \mu_0 \vec{j}$.

\section{Inductància}
\emph{Flux}: $\Phi = \int_S \vec{B}\cdot \vec{\dif S}$. \\
\emph{L. Faraday}: $\varepsilon = -\frac{\dif \Phi_{\text{m}}}{\dif t} = \oint_{\partial S} \vec{E} \cdot \vec{\dif l} = -\frac{\dif}{\dif t}\int_S \vec{B} \cdot \vec{\dif S}$.

\subsection{Bobines}
\emph{Camp m.}: $B = \mu_0 n I$, $n = N/l$.\\
\emph{Autoinductància}: $L = \frac{\phi}{I} = \mu_0 n^2 V$.\\
\emph{Potencial}: $\varepsilon = L \frac{\dif I}{\dif t}$.\\
\emph{Càrrega RL}: $I(t) = \frac{\varepsilon}{R} \lp 1 - e^{-\frac{Rt}{L}}\rp$.\\
\emph{Energia RL}: $U = \frac{1}{2}LI^2$.

\section{Maxwell}
\emph{Gauss}: $\nabla \cdot \vec{E} = \frac{\rho}{\varepsilon_0} $, $\oint_{\partial V} \vec{E}\cdot \vec{\dif S} = \frac{1}{\varepsilon_0} \int_\nu \rho \dif V$.\\
\emph{Faraday}: $\nabla \times \vec{E} +  \frac{\partial{\vec{B}}}{\partial t} = \vec{0}$,
                $\oint_{\partial S} \vec{E} \cdot \vec{\dif l} + 
                \frac{\dif}{\dif t} \int_S \vec{B} \cdot \vec{\dif S} = 0$.\\
\emph{Gauss m.}: $\nabla \cdot \vec{B} = 0$, $\oint_{\partial V} \vec{B} \cdot
                \vec{\dif S} = 0$.\\
\emph{Ampère-Maxwell}: $\nabla \times \vec{B} = \mu_0 \vec{j} + \varepsilon_0\mu_0
                    \frac{\partial{\vec{E}}}{\partial t}$, $\oint_{\partial{S}}
                    \vec{B} \cdot \vec{\dif l} = \mu_0 
                    \int_S \vec{j} \cdot \vec{\dif S} + \varepsilon_0 \mu_0
                    \frac{\dif}{\dif t} \int_S \vec{E} \cdot \vec{\dif S}$.

\addtocounter{section}{1}
\noindent\makebox[\linewidth]{\rule{\linewidth}{0.5pt}}

\subsection{Integrals}
\ci $\int \frac{1}{\sqrt{a+x^2}}\dif x = \log (\sqrt{a+x^2} + x)$. \\
\ci $\int \frac{x}{\sqrt{a+x^2}}\dif x = \sqrt{a+x^2}$. \\
\ci $\int \frac{1}{\sqrt{a+x^2}^3}\dif x = \frac{x}{a\sqrt{a+x^2}}$. \\
\ci $\int \frac{x}{\sqrt{a+x^2}^3}\dif x = -\frac{1}{\sqrt{a+x^2}}$.\\
\ci $\int sec \dif x = ln \lp |tan + sec| \rp$.

\subsection{Canvis de variables}
\emph{Polars a $\real^2$}: $\int_U f( x,y ) \dif x\dif y = \int_V f( r\cos\varphi, r\sin\varphi )r \dif r\dif \varphi$. \\
\emph{Cilíndriques a $\real^3$}: $\int_U f( x,y,z ) \dif x\dif y\dif z = \int_V f( \rho\cos\varphi, \rho\sin\varphi, z )\rho \dif \rho\dif \varphi\dif z$. \\
\emph{Esfèriques a $\real^3$}: $\int_U f( x,y,z ) \dif x\dif y\dif z = \int_V f ( r\cos\varphi\sin\theta, r\sin\varphi\sin\theta, r\cos\theta )$ $r^2\sin\theta \dif r\dif \varphi\dif \theta$.

\subsection{EDOs}
\ci $\frac{1}{v}\frac{\dif v}{\dif t} = \frac{\dif\, (\ln v)}{\dif t}$. \\
\emph{Fórmules}: $a\dot{v}+bv+c=0$; $\frac{a}{b}\dot{v}+v+\frac{c}{b}=0$; ($w=v+\frac{c}{b}$), $\frac{a}{b}\dot{w}+w=0$; $\frac{a}{b}x+1=0$; $x=\frac{-b}{a}$; $w=k e^{\frac{-b}{a}t}=v+\frac{c}{b}$; $v=ke^{\frac{-b}{a}t}-\frac{c}{b}$.

\subsection{Altres}
\emph{Esfera}: $S=4\pi r^2$, $V=\frac{4}{3}\pi r^3$.


\vspace{15pt}
\raggedleft
{\Large Nom: \underline{\hspace{6cm}}}
\end{multicols}

\includepdf[pages=1]{main_aux.pdf}

\end{document}
