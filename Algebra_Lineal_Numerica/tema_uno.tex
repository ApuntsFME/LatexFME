\chapter{Errors}

\section{Errors numèrics}
	
\begin{example}\label{ej:1_errores}
   Considerem el següent sistema d'equacions
    \[
        \begin{pmatrix}
            1 & 2\\
            0,4999 & 1,001\\
        \end{pmatrix}
        \begin{pmatrix}
        x\\
        y\\
        \end{pmatrix} =
        \begin{pmatrix}
        3\\
        1,5\\
        \end{pmatrix}
    \]
    amb solució $x = 1, y = 1$. Si canviem $0.499 \rightarrow 0.500$ obenim el sistema
    \[
        \begin{pmatrix}
            1 & 2\\
            0,5000 & 1,001\\
        \end{pmatrix}
        \begin{pmatrix}
        x\\
        y\\
        \end{pmatrix} =
        \begin{pmatrix}
        3\\
        1,5\\
        \end{pmatrix}.
    \]
    La solució ara és $x = 3, y = 0$. Un canvi de  $10^{-3}$ en un dels paràmetres ha provocat un canvi de l'ordre de $10^1$ al resultat. Aquest tipus d'error rep el nom d'error de dada.
\end{example}

\begin{example}\label{ej:2_errores}
    Considerem el càlcul de les arrels del polinomi $p(x)=x^2 - 18x +1$ fent servir quatre xifres decimals en el càlcul de l'arrel quadrada  %TODO aqui hi havia alguna cosa com aixo $\sqrt{\phantom{x}}$
    \[
            x = \dfrac{18 \pm \sqrt{18^2 - 4 \cdot 1 \cdot 1}}{2 \cdot 1} = 9 \pm \sqrt{80} \nonumber.
    \]
    Si considerem l'aproximació $\sqrt{80} \simeq 8.9442$, obtenim les solucions
    \begin{gather*}
            x_{+}=9+8.9442=17.9442 \\
            x_{-}=9-8.9442=0.0558.
    \end{gather*}
    Hem obtingut la primera solució amb sis xifres significatives, però la segona només amb tres. Aquest fet és una pèrdua important d'informació deguda a l'error comès en el truncament als quatre dígits.
\end{example}

\begin{example}\label{ej:3_errores}
    Considerem el càlcul de $I_n \coloneqq \int_{0}^{1}x^ne^{x-1}dx$, per a $n = 0,1,2,...$.
    Per fer-ho, obtindrem una formula recurrent per les $I_n$:
    \[
            I_n = \int_{0}^{1}x^ne^{x-1}\dif x = [x^ne^{x-1}]_{0}^{1} - \int_{0}^{1}nx^{x-1}e^{x-1}\dif x = 1 - nI_{n-1} \nonumber
    \]
    amb valor inicial $I_0 = \int_{0}^{1}x^0e^{x-1}\dif x = [e^{x-1}]_{0}^{1} = 1 - e^{-1} \simeq 0.63212055$.
    
    Així calculem $I_0 = 0.63212055,\, I_1 = 0.367899,\, \dots,\, I_9 = -0.068480$. Veiem que el resultat de $I_9$ és absurd ja que $x^ne^{x-1}$ és no negatiu per $x \in [0,1]$ i per tant la seva integral a l'interval també ho és. Tot i que la recurrència és correcte, l'algorisme és numèricament dolent ja que l'error és prou considerable.
\end{example}

\begin{example}\label{ej:4_errores}
    Considerem el càlcul de la derivada d'una funció $f$ al punt $x_0$
    \[
            f'\lp x_0\rp  = \limvar{h}{0} \dfrac{f\lp x_0 + h\rp  - f\lp x_0\rp }{h} \nonumber.
    \]
    Com un límit és impossible de calcular numèricament, estem obligats a aproximar-lo de la forma
    \[
            f'\lp x_0\rp  \simeq \dfrac{f\lp x_0 + h\rp  - f\lp x_0\rp }{h} = \frac{\Delta f\lp x_0\rp }{h} \nonumber.
    \]
    Per $h$ suficientment petita, la teoria sorbre límits ens diu que a mesura que $h$ s'apropa a $0$, el valor de l'aproximació s'aproparà cada cop més al valor real. Desafortunadament la teoria torna a fallar a la pràctica: si calculem el cocient $\frac{\Delta f\lp x_0\rp }{h}$ per $h$ successivament prou petita, el valor obtingut del límit millora al principi però a partir d'un cert punt s'allunya del valor real.
\end{example}
    
\begin{defi}[error]\index{error!d'arrodoniment((tlab))}\index{error!de truncat((tlab))}
    Existeixen diferents tipus d'errors segons la seva procedència:
    \begin{enumerate}
            \item Error d'\emph{arrodoniment}: $\varepsilon_R$
            \begin{itemize}
                    \item Errors de dada (exemple \ref{ej:1_errores}).
                    \item Error d'operació (exemple\ref{ej:2_errores}).
            \end{itemize}
            \item Error de \emph{truncat}: $\varepsilon_T$
            \begin{itemize}
                    \item Error en l'algorisme al pas de contínua a la discretització (exemples \ref{ej:3_errores} y \ref{ej:4_errores}).
            \end{itemize}
    \end{enumerate}
\end{defi}

\begin{defi}[error!total]
    Definim l'error total $\varepsilon \coloneqq$ $\varepsilon_R + \varepsilon_T$.
\end{defi}

\begin{obs}
  En general, l'error d'arrodoniment serà alt amb pocs passos i baix amb molts passos, mentre que l'error de truncat es comportarà a la inversa. Degut a que l'eerror total és la seva suma, no existeix una solució trivial pel número de pasos òptims que hauriem de fer servir. 
\end{obs}

\section{Representació digital d'un número real}
    
\begin{prop}
    Tot $x \in \R$ admet una representació única en base $b \in \n$, $b>1$ amb dígits $\lp  a_j\rp  _{j \in \z}$ tal que 
    \[
            x = \cdots + a_kb^k + \cdots a_0b^0 + a_{-1}b^{-1} + \cdots \nonumber.
    \]
\end{prop}

\begin{obs}
  En general, la seqüència de xifres $a_j$ tindrà un nombre infinit de valors diferent de zero. Al ser la capacitat dels ordinadors finita, és impossible representar tots els reals amb precisió exacta, el qual generarà errors.
\end{obs}

\begin{defi}[representació!normalitzada]\index{representació!única((tlab))}
    La representació normalitzada o única de $x \in \real$ en base $b$ amb dígits $\lp  d_n\rp $, i exponent $n \in \z^{+}$  $e \in \z$ és
    \[
      x = \pm 0,d_1d_2... \cdot b^e \nonumber,
    \]
    amb la restricció que $d_1 \neq 0$.
\end{defi}

\begin{example}
  Representació normalitzada de $172.456$ en base $10$ és $\SI{0.172456e3}{}$.
\end{example}

\begin{defi}[representació!en coma flotant]
	La representació en coma flotant amb $n$ dígits d'un nombre real $x$ amb representació normalitzada $x = \pm 0d_1..d_n... \cdot b^e$ és:
	\begin{itemize}
		\item  Per truncat:
		\[
			\fl_T\lp x\rp  = \pm 0.d_1..d_n \cdot b^e \nonumber
		\]
		\item Per arrodoniment:
			\[
			\fl\lp x\rp  = \left.
			\begin{cases}
			\pm 0.d_1..d_n \cdot b^e & 0 \leq d_n < \frac{b}{2} \\
			\pm \lp 0.d_1..d_n + b^{-n}\rp  \cdot b^e & \frac{b}{2} \leq x < b
			\end{cases}
			\right.
			\]
	\end{itemize}
\end{defi}

\begin{obs}
  Suposem que volem guardar $\fl\lp x\rp $ amb la següent estructura en base $b = 2$:
  \[
	  \underbrace{[*]}_{s_1}
	  \hspace{0.1cm}
	  \underbrace{[{}*|*|*|*|*|*|*{}]}_{d_1,...,d_n}
	  \hspace{0.1cm}
	  \underbrace{[*]}_{s_2}
	  \hspace{0.1cm}
	  \underbrace{[{}*|*|*|*|*{}]}_{e_1,..,e_k}
	  \nonumber
  \]

  On $s_1$ és el signe de $x$, $d_1,...,d_n$ són les xifres de $\fl\lp x\rp $, $s_2$ és el signe de l'exponent i $e_1,...,e_k$ els dígits de l'exponent de $\fl\lp x\rp $. Veiem que aquest emagatzament es pot fer més eficient.
  
  Una primera millora consistiria en donar-se compte que, al ser en base dos i $d_1 \neq 0$, necessàriament $d_1=1$. Així doncs, no tenim perquè emmagatzemar-lo. Això ens dona una xifra extra de precisió amb la mateixa memòria.
  
  Una segona millora seria canviar la forma en que emmagatzemem l'exponent $e$. Definim $e' = e + n^{\underline{\text{o}}}$, amb un $n^{\underline{\text{o}}}$ fix, definit com el mínim natural tal que $e'>0,\, \forall e$. Observem que la definició de $e'$ estableix una bijecció, així que amb l'emmagatzament de $e'$ obtenim $e$. A més, com també és positiu podem prescindir de guardar el signe.
  El valor de $n^{\underline{\text{o}}}$ per el tipus \emph{float} ($32$ bits, $23$ per la mantissa, $8$ per l'exponent) és $127$, y $1023$ per al tipus \emph{double} ($64$ bits, $52$ per la mantissa, $11$ per l'exponent).
\end{obs}
 

\begin{defi}[èpsilon màquina]
	El $\varepsilon$-màquina d'un ordinador és el mínim $\varepsilon > 0$ tal que $\fl\lp 1 + \varepsilon\rp  > 1$.
\end{defi}
        
\section{Errors en la representació de nombres}

\begin{defi}[error!absolut]\index{error!relatiu((tlab))}
    Sigui $x \in \R$ un nombre real i $\overline{x}$ una aproximació de $x$
    \begin{itemize}
	\item Definim l'error absolut dut a terme en aproximar $x$ amb $\overline{x}$ com $e_a\lp \overline{x}\rp  = \overline{x} - x$.
	\item Definim l'error relatiu dut a terme en aproximar $x$ amb $\overline{x}$ com
	$e_r\lp \overline{x}\rp  = \frac{e_a\lp \overline{x}\rp }{x} \simeq \frac{e_a\lp \overline{x}\rp }{\overline{x}}$.
    \end{itemize}
\end{defi}

\begin{obs}
   Com els errors no solen ser coneguts, treballarem amb cotes superiors d'aquestos:
   \begin{align*}
	\overline{x} &= x + e_a\lp \overline{x}\rp  & \abs{e_a\lp \overline{x}\rp } \leq \varepsilon_a\lp \overline{x}\rp  \text{ cota del error absoluto} \\
	\overline{x} &= x\lp 1 + \delta\rp  & \abs{e_r\lp \overline{x}\rp }=\abs{\delta} \leq \varepsilon_r\lp \overline{x}\rp  \text{ cota del error relativo}.
    \end{align*}
\end{obs}

\begin{obs}
    $\overline{x} = x\lp 1 + \delta\rp  = x + \underbrace{\delta x}_{e_a\lp \overline{x}\rp }$
\end{obs}
  
\begin{prop} 
   Sigui $x = \pm 0.d_1..d_nd_{n+1}... \cdot b^e$. Aleshores, les cotes de l'error absolut i el relatiu al fer el punt flotant per arrodoniment són
  \begin{align*}
    e_a\lp \fl_R\lp x\rp \rp  &=  \frac{1}{2}b^{-n+e} \\
    e_r\lp \fl_R\lp x\rp \rp  &=  \frac{1}{2}b^{-n+1}.
  \end{align*}
\end{prop}
\begin{proof}
   La representació de $x$ en coma flotant amb $n$ xifres per truncat és $\fl\lp x\rp  = \pm 0.d_1..d_n$. Aleshores per definició tenim
    \begin{align*}
            |e_a\lp \fl_T\lp x\rp \rp | &= |\fl\lp x\rp  - x| = d_{n+1}... \cdot b^{-\lp n+1\rp } \cdot b^e \leq b \cdot b^{-\lp n+1\rp } \cdot b^e &= b^{-n+e} = \epsilon_a\lp \fl_T\lp x\rp \rp  \\
            |e_r\lp \fl_T\lp x\rp \rp | &= |\frac{e_a\lp \overline{x}\rp }{x}| \leq \frac{b^{-n+e}}{b^{-1+e}} &= b^{-n+1} = \epsilon_r\lp \fl_T\lp x\rp \rp.
    \end{align*}
    Amb un argument similar a l'anterior veiem que les cotes per l'error absolut i relatiu al fer el punt flotant per arrodoniment són
    \begin{align*}
            \abs{e_a\lp \fl_R\lp x\rp \rp } &=  \frac{1}{2}b^{-n+e} = \varepsilon_a\lp \fl_R\lp x\rp \rp  \\
            \abs{e_r\lp \fl_R\lp x\rp \rp } &=  \frac{1}{2}b^{-n+1} = \varepsilon_r\lp \fl_R\lp x\rp \rp.
    \end{align*}
\end{proof}
 
\begin{defi}[xifres!decimals correctes]
    Considerant $b = 10$ definim
    \begin{itemize}
	\item Si $e_a\lp \overline{x}\rp  \leq \frac{1}{2}10^{-k}$, direm que $\overline{x}$ té $k$ xifres decimals correctes.
	\item Si $e_r\lp \overline{x}\rp  \leq \frac{1}{2}10^{-k+1}$, direm que $\overline{x}$ té $k$ xifres significatives correctes.
    \end{itemize}
\end{defi}

\begin{obs}
  A la definició anterior apareixen els factors $\frac{1}{2}$ ja que per defecte considerem el coma flotant per arrodoniment al ser l'error menor. En caas d'utilitar el coma flotant per truncat, les definicions són les mateixes sense el factor $\frac{1}{2}$.
\end{obs}

\section{Propagació d'errors}

\begin{prop}
  Com a propietats fonamentals de la propagació d'errores tenim
  \begin{enumerate}[i)]
    \item $\varepsilon_a\lp \overline{x} \pm \overline{y}\rp  = \varepsilon_a\lp \overline{x}\rp  + \varepsilon_a\lp \overline{y}\rp$
    \item $\varepsilon_r\lp \overline{x}\cdot\overline{y}\rp  \lessapprox \varepsilon_r\lp \overline{x}\rp  + \varepsilon_r\lp \overline{y}\rp$.
  \end{enumerate}
\end{prop}
\begin{proof}
  \begin{enumerate}[i)]
   \item []
   \item Directe a partir de la definició.
   \item Tenim $e_r\lp \overline{x}\cdot\overline{y}\rp  = xy\lp 1+\delta_x\rp \lp 1+\delta_y\rp  = xy\lp 1 + \delta_x + \delta_y + \delta_x\delta_y\rp  \simeq xy\lp 1 + \delta_x + \delta_y\rp$ y si $\delta_x,\delta_y << 1$, aleshores $\delta_x\delta_y << \delta_x + \delta_y$ i ens diu que $\varepsilon_r\lp \overline{x}\cdot\overline{y}\rp  \lessapprox \varepsilon_r\lp \overline{x}\rp  + \varepsilon_r\lp \overline{y}\rp$.
  \end{enumerate}
\end{proof}

\begin{obs}
    Cuidado al fer la resta de variables properes ja que es generen errors relatius grans.
    \[
	e_r\lp \ap{x}-\ap{y}\rp  = \dfrac{e_a\lp \ap{x} - \ap{y}\rp }{\ap{x}-\ap{y}} = \dfrac{\ap{x}}{\ap{x}-\ap{y}}e_r\lp \ap{x}\rp  - \dfrac{\ap{y}}{\ap{x}-\ap{y}}e_r\lp \ap{y}\rp.
    \]
    Si ara prenem valor absolut,
    \[
	\abs{e_r\lp \ap{x}-\ap{y}\rp} \leq \abs{\dfrac{\ap{x}}{\ap{x}-\ap{y}}}\cdot\abs{e_r\lp \ap{x}\rp} + \abs{\dfrac{\ap{y}}{\ap{x}-\ap{y}}}\cdot\abs{e_r\lp \ap{y}\rp}.
    \]
    Vegem que si $\ap{x}$, $\ap{y}$ són properes, $\ap{x}-\ap{y}$ serà molt petit i al dividir la cota de l'error augmentarà molt.
\end{obs}

\begin{example}
	Calcular les arrels del polinomi  $x^2 - 18x + 1$, amb el càlcul de l'arrel per tuncat a les cuatre xifres de l'exemple \ref{ej:2_errores}.
	\begin{align*}
		x_+ &= 9 + \sqrt{80} = 17.94442 &&\text{ 6 xifres significatives} \\
		x_- &= 9 - \sqrt{80} = \enspace 0.0558 &&\text{ 3 xifres significatives}.
	\end{align*}
	Analitzem aquest resultat a partir de la cota de l'error relatiu en el càlcul de $x_-$. Sigui $x = 9$, $y = \sqrt{80}$, $\ap{x}=9$, $\ap{y}=8.4492$. Aleshores,
	\[
		e_r\lp \ap{x}-\ap{y}\rp  \simeq \abs{\frac{\ap{y}}{\ap{x}-\ap{y}}}\cdot\abs{e_r\lp \ap{y}\rp} = \dfrac{8.9442}{0.0588}\cdot \underbrace{10^{-5+1}}_{\text{5 díg sig}} \simeq \scin{0.16e-1} < 1^{-2+1}
	\]
	És a dir, em puc creure només $\approx 2$ xifres significatives.
\end{example}

\begin{example}
  Considerem el càlcul de $I_n = \int_{0}^{1}x^ne^{x-1}dx$,$n = 0,1,2,...$ de l'exemple  \ref{ej:3_errores}, i, per fer-ho, l'ús de la recurrència $I_n = 1 - nI_{n-1}$, $I_0 = 1 - \frac{1}{e} \approx 0,63212055$. Per $I_9$ obteniem un valor negatiu. Si analitzem aquest problema a partir de l'error absolut tenim que $e_a\lp I_n\rp  \lessapprox e_a\lp 1\rp  + ne_a\lp I_{n-1}\rp $, és a dir $e_a\lp I_n\rp  \simeq n!\,e_a\lp I_0\rp $. L'error absolut augmenta factorialment amb el nombre de pasos i per tant aques procediment no és bo numèricament.
\end{example}

\begin{prop}\label{prop:1_error_funcion}
  Sigui $x \in \real$, $\overline{x}$ una aproximació de $x$ i $f:\: \real \to \real$, $f\in \C^1$, una funció. Aleshores,
  \begin{enumerate}[i)]
   \item $\abs{e_a\lp f\lp\overline{x}\rp\rp} \simeq \abs{f'\lp \overline{x}\rp}\abs{e_a\lp x\rp}$
   \item $\abs{e_r\lp f\lp\overline{x}\rp\rp} \simeq \abs{\frac{\overline{x}f'\lp\overline{x}\rp}{f\lp\overline{x}\rp}}\abs{e_r\lp x\rp}$.
  \end{enumerate}
\end{prop}
\begin{proof}
  \begin{enumerate}
   \item []
   \item 
      $e_a\lp f\lp \overline{x}\rp\rp  = e_a\lp f\lp x + e_a\lp x\rp \rp\rp \simeq e_a\lp f\lp x\rp  + f'\lp x\rp e_a\lp x\rp\rp = \abs{f'\lp \overline{x}\rp}\abs{e_a\lp x\rp}$.
   \item 
      $e_r\lp f\lp \overline{x}\rp \rp  = \frac{e_a\lp f\lp \overline{x}\rp \rp }{f\lp \overline{x}\rp } \simeq \frac{f'\lp \overline{x}\rp e_a\lp x\rp }{f\lp \overline{x}\rp } = \frac{\overline{x}f'\lp \overline{x}\rp }{f\lp \overline{x}\rp }e_r\lp x\rp$.
  \end{enumerate}
\end{proof}

\begin{defi}[factor de propagació]
   Sigui $x \in \real$, $\overline{x}$ una aproximació de $x$ i $f:\: \real \to \real$, $f\in \C^1$, una funció. Aleshores definim el factor de propagació de l'error (f.d.p) per $f$ com a $\alpha = \abs{\frac{\overline{x}f'\lp\overline{x}\rp}{f\lp\overline{x}\rp}}$.
\end{defi}

\begin{example}[Seguretat de l'operació \emph{arrel quadrada}]
      Considerem el càlcul de $f\lp x\rp  = \sqrt{x}$. Tenim que en aquest cas f.d.p = $\frac{x\frac{1}{2\sqrt{x}}}{\sqrt{x}} = \frac{1}{2}$. Per tant,  $\abs{e_r\lp \sqrt{\overline{x}}\rp} \simeq \frac{1}{2}\abs{e_r\lp \overline{x}\rp}$. El que ens diu que fer l'arrel quadrada no és una operació que augmenti l'error: és una operació segura.
\end{example}

\begin{example}[Seguretat de l'operació $\frac{1}{1-x^2}$]
      Considerem el càlcul de $f\lp x\rp  = \frac{1}{1-x^2}$. Aleshores, f.d.p = $\frac{x\frac{4x}{\lp 1-x^2\rp ^2}}{\frac{1}{1-x^2}} = \frac{4x^2}{1-x^2} = 4\lp 1 - \frac{1}{1-x^2}\rp $. Vegem que l'f.d.p serà molt gran quan $x \simeq 1$, i en aquest cas, calcular $f$ d'aquesta manera seria una operació insegura. En comptes de fer el càlcul a partir de la divisió, podem expandir per Taylor i obtenir $f\lp x\rp  \equiv 1 + x^2 + x^4 + x^6 + ...$ per finalment fer el càlcul.
\end{example}

\begin{example}[Efecte del canvi de coeficients d'un sistema lineal en la solució]
      
    Considerem el sistema d'equacions:
    \[
	\begin{pmatrix}
	    a & b\\
	    c & d\\
	\end{pmatrix}
	\begin{pmatrix}
	x\\
	y\\
	\end{pmatrix} =
	\begin{pmatrix}
	3\\
	1,5\\
	\end{pmatrix}.
    \]
    Amb solució $x = \frac{3d-1.5b}{ad-bc}$,$y = \frac{1.5a-3c}{ad-bc}$. Suposem que sabem $a = 1,b = 2,d = 1.001$ de manera exacta, però només tenim una aproximació de $c$ donada per $\overline{c} = 0.499$, amb error absolut $e_r\lp c\rp  \simeq \scin{1e-3}$. Al resoldre el sistema, calculem que $\overline{x} = \frac{3d-1.5b}{ad-b\overline{c}} \cong f\lp \overline{c}\rp $. L'error absolut comès en el càlcul de $\overline{x}$ ve donat per
    \[
	    \abs{e_a\lp \overline{x}\rp} = \abs{e_a\lp f\lp \overline{c}\rp \rp} \simeq \abs{f'\lp \overline{c}\rp}\abs{e_a\lp \overline{c}\rp}.
    \]
    On
    \[
	    \abs{f'\lp \overline{c}\rp}= \abs{\frac{\lp 3\cdot 1.001-1.5\cdot 2\rp \cdot 2}{\lp 1\cdot 1.001 -2\cdot\overline{c}\rp ^2}}_{\overline{c}=0.499} \simeq \scin{6e3}.
    \]
    I per tant, $\abs{e_a\lp x\rp}\simeq 6$, el que ens diu que per aquests valors la resolució del sistema és una operació insegura.
\end{example}

\begin{obs}
      Suposem que volem calcular $f\lp x\rp  \equiv f\lp x_1,...,x_n\rp $, però només tenim $\overline{x} = \lp \overline{x_1},...,\overline{x_n}\rp $. Aleshores l'error comès, suposant el càlcul de $f$ perfecte, ve donat per
      \[
	      f\lp \overline{x}\rp  = f\lp x + e_a\lp x\rp \rp  = f\lp x\rp  + \nabla f\lp x\rp^Te_a\lp x\rp  + \theta_2\lp e_a\lp x\rp \rp.
      \]
      Aleshores,
      \[
	      e_a\lp f\lp \overline{x}\rp \rp  \simeq \nabla f\lp x\rp ^Te_a\lp x\rp  = \sum_{k=1}^{n}\pdv{f}{x_k} \lp \overline{x}\rp e_a\lp \overline{x}_k\rp.
      \]
\end{obs}

\begin{prop}
  Sigui $x \in \real$, $\overline{x}$ una aproximació de $x$ i $f:\: \real \to \real$, $f\in \C^1$, una funció. Sigui $\delta_f$ una cota de l'error de la funció tal que $\abs{f(x)-\overline{f}(x)}<\delta_f$, $\forall x \in \real$. Aleshores tenim la següent expresió per a l'error comès:
  \[
      e_a\lp \overline{f}\lp \overline{x}\rp \rp  \simeq \abs{\overline{f}'\lp \overline{x}\rp }\abs{e_a\lp x\rp} + \delta_f\abs{\overline{f}\lp \overline{x}\rp}.
  \]
\end{prop}
\begin{proof}
  Igual que la demostració de la proposició \ref{prop:1_error_funcion}.
\end{proof}

\section{Càlcul d'arrels d'un polinomi}

\begin{prop}
  Sigui $p(x)=\sum\limits_{i=0}^{n} a_ix^i$ un polinomi real d'arrels reals $\alpha_1,\,\dots,\,\alpha_n$. Sigui $1\leq k\leq n$ tal que $a_k$ és l'únic coeficient amb error i del qual només en coneixem una aproximació $\overline{a}_k$. Aleshores,
  \begin{enumerate}[i)]
    \item $\abs{e_a\lp \alpha_l\lp \ap{a}_k\rp \rp} \simeq \abs{\frac{\alpha_l^k}{ a_n\prod_{j=1,j\neq l}^{n}\lp \alpha_l-\alpha_j\rp }}\abs{e_a\lp \ap{a}_k\rp}$.
    \item $e_r\lp \alpha_l\lp \ap{a}_k\rp \rp \simeq \left|\frac{a_k\alpha_l^{k-1}}{a_n\prod_{j=1,j\neq l}^{n}\lp \alpha_l-\alpha_j\rp }\right| \abs{e_r\lp \ap{a}_k\rp}$.
  \end{enumerate}
\end{prop}
\begin{proof}
  Per Cardano-Vieta, sabem que $\forall l,\, \alpha_l\lp\overline{a}_n,\,\dots\,\overline{a}_0\rp$. Degut a que només hi ha error al coeficient $a_k$ i per la fórmula de propagació d'errors tenim que
  \[
    e_a\lp \alpha_l\lp \ap{a}_k\rp \rp  \simeq \pdv{\alpha_l}{a_k}\lp a_k\rp e_a\lp \ap{a}_k\rp .
  \]
  Fent servir derivació implícita respecte a $a_k$ sobre la següent igualtat
  \[
     a_n\alpha_l^n + \cdots a_k\alpha_l^k + \cdots a_0 = p\lp \alpha_l\rp  = p\lp \alpha_l\lp a_k\rp \rp 
  \]
  tenim que
  \[
    \alpha_l^k = p'\lp \alpha_l\lp a_k\rp \rp \pdv{\alpha_l}{a_k}\lp a_k\rp  \implies \pdv{\alpha_l}{a_k}\lp a_k\rp  = \dfrac{\alpha_l^k}{p'\lp \alpha_l\lp a_k\rp \rp}.
  \]
  El numerador de l'expresió obtinguda és fàcil de calcular. Per calcular-lo farem servir la representació d'un polinomi com a producte de factors simples
  \[
      p\lp x\rp = a_n\lp x-\alpha_l\rp \prod_{j=1,j\neq l}^{n}\lp x-\alpha_j\rp.
  \]
  Derivant obtenim
  \[
    p'\lp x\rp  = a_n\prod_{j=1,j\neq l}^{n}\lp x-\alpha_j\rp  + a_n\lp x-\alpha_l\rp \frac{\partial}{\partial x}\lp \prod_{j=1,j\neq l}^{n}\lp x-\alpha_j\rp\rp,
  \]
  el que ens diu que
  \[
    p'\lp \alpha_l\rp = a_n\prod_{j=1,j\neq l}^{n}\lp \alpha_l-\alpha_j\rp.
  \]
  Finalment obtenim el resultat desitjat per l'error absolut del càlcul de $\alpha_l$ degut a l'error en $\ap{a}_k$
  \[
    \abs{e_a\lp \alpha_l\lp \ap{a}_k\rp \rp} \simeq \abs{\frac{\alpha_l^k}{ a_n\prod_{j=1,j\neq l}^{n}\lp \alpha_l-\alpha_j\rp }}\abs{e_a\lp \ap{a}_k\rp}.
  \]
  Pel segon apartat, simplement fent servir la definició d'error relatiu 
  \begin{align*}
    e_r\lp \alpha_l\lp \ap{a}_k\rp \rp & = \left|\frac{e_a\lp \alpha_l\lp \ap{a}_k\rp \rp }{\alpha_l\lp \ap{a}_k\rp }\right| \simeq \left|\frac{\alpha_l^{k-1}}{ a_n\prod_{j=1,j\neq l}^{n}\lp \alpha_l-\alpha_j\rp }\right|\left|\frac{e_a\lp \ap{a}_k\rp }{a_k}\right||a_k|=\\
           & = \left|\frac{a_k\alpha_l^{k-1}}{a_n\prod_{j=1,j\neq l}^{n}\lp \alpha_l-\alpha_j\rp }\right| |e_r\lp \ap{a}_k\rp |
  \end{align*}
\end{proof}

\begin{example}[Exemple de Wilkinson]
  Si considerem $n=20$, $\alpha_1 = 1,\,\cdots,\,\alpha_{20} = 20$, $k = 19 \leftrightarrow a_{19} = -210$ i $l = 16 \leftrightarrow \alpha_l = 16$, calculant l'error relatiu tenim
  \[
     e_r\lp \alpha_l\lp \ap{a}_k\rp \rp  = \left|\frac{\lp -210\rp \cdot\lp 16^{19-1}\rp }{1\cdot\lp 15!4!\lp -1\rp ^4\rp }\right| |e_r\lp \ap{a}_{19}\rp | \approx \scin{0.31e11}e_r\lp \ap{a}_{19}\rp.
  \]
  Si volem calcular $\alpha_l\lp \ap{a}_k\rp $ (ex. $\alpha_{16}\lp \ap{a}_{19}\rp $) amb $t$ xifres singificatives, les xifres correctes que necessitariem de $a_{19}$ serien
  \[
    \frac{1}{2}\cdot10^{1-t} \sim \scin{0.31e11}\frac{1}{2}\cdot10^{1-s},
  \]
  i per tant
  \[
    1 - t = 12 - s \iff s \sim 11 + t.
  \]
  És a dir, que si volem un valor per a l'arrel amb $t$ xifres correctes, necessitem saber $11+t$ xifres del coeficient $a_{19}$ (suposant que coneixem els altres de forma exacta).
\end{example}
\begin{example}[La no asociativitat de la suma numèrica]
  Sigui $\delta$ cota de l'error relatiu en l'emmagatzament ($\delta \sim \frac{1}{2}10^{1-t} \equiv t$ xifres significatives). Volem analitzar les diferencies entre $\sum_{n=1}^{N}\frac{1}{n^2}$ i $\sum_{n=N}^{1}\frac{1}{n^2}$. Calculem primer l'error comès al sumar una seqüència general de $n$ termes.
  \begin{table}[H]
	  \centering
	  \label{no-asociativa}
	  \begin{tabular}{lll}
		  Valor real & Aproximació & Cota de l'error absolut \\
		  $s_1 = x_ 1$ & $\ap{s}_1 = \ap{x}_1$ & $\varepsilon\lp \ap{s}_1\rp  = \varepsilon_a\lp \ap{x}_1\rp  = \ap{x}_1\delta$ \\
		  $s_2 = s_1 + x_ 2$ & $\ap{s}_2 = \ap{s}_1 + \ap{x}_2$  &   $\varepsilon_a\lp \ap{s}_2\rp  \simeq \lp 2x_1 + 2x_2\rp \delta$  \\
		  $s_3 = s_2 + x_3$ & $\ap{s}_3 = \ap{s}_2 + \ap{x}_3$   &   $\varepsilon_a\lp \ap{s}_3\rp  \simeq \lp 3x_1 + 3x_2 + 2x_3\rp \delta$\\
		  $s_4 = s_3 + x_4$ & $\ap{s}_4 = \ap{s}_3 + \ap{x}_4$   &   $\varepsilon_a\lp \ap{s_4}\rp  \simeq \lp 4x_1 + 4x_2 + 3x_3 + 2x_4\rp \delta$ \\
		  $s_N = s_{N-1} + x_N$ & $\ap{s}_N = \ap{s}_{N-1} + \ap{x}_N$ & $\varepsilon_a\lp \ap{s}_N\rp  \simeq \lp N\lp x_1 +x_2\rp + \cdots + 2x_N\rp \delta$
	  \end{tabular}
  \end{table}

  Derivació de les cotes (casos $s_1$,$s_2$; la resta s'obtenen anàlogament):

  \begin{enumerate}[1)]
	  \item $\varepsilon_a\lp \ap{s}_2\rp  = \varepsilon_a\lp \ap{s}_1\rp  + \varepsilon_a\lp x_2\rp  + \lp \ap{s}_1 + \ap{x}_2\rp \delta = x_1\delta + x_2\delta + \lp \ap{x}_1 + \ap{x}_2\rp \delta \simeq \lp 2x_1 + 2x_2\rp \delta $
	  \item $\varepsilon_a\lp \ap{s}_3\rp  = \varepsilon_a\lp \ap{s}_2\rp  + \varepsilon_a\lp x_3\rp  + \lp \ap{s}_2 + \ap{x}_3\rp \delta \simeq \lp 2x_1 + 2x_2\rp \delta + x_3\delta + \lp x_1 + x_2 + x_3\rp \delta = \lp 3x_1 + 3x_2 + 2x_3\rp \delta $
  \end{enumerate}
  
  Per minimitzar l'error voldríem que $x_1 \leq x_ 2 \leq x_3 \leq \cdots \leq x_N$. En el nostre cas tenim $x_j = \frac{1}{j^2}$. Aquest resultat ens indica que la suma inversa ($\sum_{j=N}^{1}\frac{1}{j^2}$)  donaria un millor resultat.
\end{example}