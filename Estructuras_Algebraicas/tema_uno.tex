\chapter{Grupos}

\begin{defi}[grupo]
    Un grupo $G$ es un conjunto, no vacío, donde hay definida una operación interna
    \[
        \begin{aligned}
            G \times G &\mapsto G \\
            (a, b) &\to a * b
        \end{aligned}
    \]
    que cumple
    \begin{enumerate}
        \item $(a*b)*c = a*(b*c)$
        \item $\exists$ elemento neutro, e.d., $\exists e \forall a \in G \tq a*e = e*a = a$
        \item $\forall a \in G, \, \exists \tilde{a} \in G \tq a * \tilde{a} = \tilde{a}*a = e$
    \end{enumerate}
\end{defi}

\begin{defi}[grupo!conmutativo]
    Diremos que $G$ es un grupo conmutativo, si es un grupo y además verifica la propiedad conmutativa:
    \[
        a*b = b*a \quad \forall a, b \in G
    \]
\end{defi}

\begin{obs}
    Notación: existen varias notaciones para referirnos a esta operación:
    \begin{tabular}{cccc}
        Notación & Operación & Elemento neutro & Inverso \\
        \hline \\
        Aditiva & $+$ & 0 & $-a$ \\
        Multiplicativa & $\cdot$ & 1 & $a^{-1}$ 
    \end{tabular}
\end{obs}

\begin{defi}[subgrupo]
    Sea $G$ un grupo, diremos que un subgrupo de $G$ es un subconjunto $H \subseteq G$ tal que
    \begin{itemize}
        \item $H \neq \emptyset$
        \item $a, b \in H \implies a*b \in H$ (operación cerrada)
        \item $\forall a \in H, a^{-1} \in H$
    \end{itemize}
\end{defi}

\begin{obs}
    Los subgrupos son los grupos (con operación restringida) incluidos en el grupo.
\end{obs}
\begin{proof}
    Sea $H \subset G$ un subgrupo de $G$. Queremos ver que $H$ es un grupo:
    Tenemos una operación:
    \[
        \begin{aligned}
            H \times H &\mapsto H \\
            (x, y) &\to x*y \in H
        \end{aligned}
    \]
    Tiene la propiedad asociativa ya que todos los elementos de $G$ la cumplen, por lo tanto también los de $H$.
    Existe elemento neutro ya que $\exists a \in H$ y $\exists a^{-1} \in H$, entonces $a*a^{-1}=e \in H$.
    Y la última propiedad está impuesta.
\end{proof}

\begin{example}
    \begin{itemize}
        \item
            \[
                \text{Sea $G$ un grupo, los subgrupos impropios son}
                \begin{cases}
                    \setb{1} \text{ trivial} \\
                    G
                \end{cases}
            \]
        \item $\z \subset \q \subset \real \subset \cx$ con la suma son grupos y subgrupos
        \item TODO $\z$ modulo n
        \item Si $G$ y $H$ son dos grupos
            \[
                 G \times H = \setb{(x,y) \vert x \in G, \, y \in H}   
            \]
            es un grupo, con $(a, b) \cdot (c, d) = (ac, bd)$.
        \item $S_n$ es el grupo simétrico de $n$ elementos (permutaciones de $n$ elementos)
        \item Grupo diedial. $D_{2n} =$ conjuntos de las isometrías del plano que dejan invariante $P_n$.
            Donde $P_n$ es un polígono regular de $n$ lados (raices $n$-esimas de 1)

            \[
                D_{2 \cdot 4} = \setb{id, r, r^2, r^3, s, rs, r^2s, r^3s}
            \]
            Con $r$ la rotación horaria de 90º y $s$ la simetría respecto al eje $x$.
    \end{itemize}
\end{example}


