\section{Definicions d'espais i subespais vectorials}
\begin{defi}[Espai vectorial]
	Un espai vectorial $V$ sobre un cos $\mathbb{K}$ (també anomenat un $\K$-espai vectorial) equipat d'un producte (o operació) intern
	\begin{equation*}
	\begin{aligned}
		(+)\colon\quad	&V\times V &&\rightarrow&& V\\
						&(v,w) &&\mapsto&& v+w \,,
	\end{aligned}
	\end{equation*}
	anomenat suma, i d'un producte extern
	\begin{equation*}
	\begin{aligned}
		(\cdot)\colon\quad	&\mathbb{K}\times V &&\rightarrow&& V\\
							&(\lambda,v) &&\mapsto&& \lambda\cdot v\,,
	\end{aligned}
	\end{equation*} 
	anomenat multiplicació escalar, és un conjunt d'elements anomenats vectors que compleix les següents propietats:
	\begin{enumerate}[i)]
		\item \textit{Tancat per la suma.} $\forall v,w \in V\qcolon v+w\in V$
		\item \textit{Tancat per la multiplicació escalar.} $\forall \lambda \in \mathbb{K}\; \forall v \in V\qcolon \lambda\cdot v\in V$
	\end{enumerate}
\end{defi}

\begin{defi}
	Sigui $E$ un $\K$-espai vectorial. Es diu que un subconjunt $V\subseteq E$ és un subespai vectorial d'$E$ si (i només si) $V$ també és un espai vectorial per sí mateix (és a dir, està tancat per la suma, per la multiplicació escalar, etc.).
\end{defi}

\begin{prop}
	content...
\end{prop}