\definecolor{ForestGreen}{rgb}{0,0.6,0}

\newcommand{\bl}{\color{blue}}
\newcommand{\rd}{\color{red}}
\newcommand{\gre}{\color{ForestGreen}}

\newcommand{\ucarr}{\mathbin{\text{\rotatebox[origin=c]{270}{$\curvearrowleft$}}}}
\newcommand{\dcarr}{\mathbin{\text{\rotatebox[origin=c]{270}{$\curvearrowright$}}}}

\begin{defi*}[Determinant]
	Sigui $M \in \mathcal{M}_n(\mathbb{R})$ una matriu $n\times n$. Anomenarem, si $n>1$, ``determinant de $M$'' (abreviat $\; \det M$) a l'expressió següent: \[\det M \defeq \sum_{j=1}^{n} m_{1,j}\cdot(-1)^{1+j}\det M_{1,j}  \,,\] on $M_{i,j}$ denota la submatriu que s'obté treient la fila $i$ i la columna $j$ de $M$. En el cas $n=1$, el determinant serà igual a l'únic coeficient de la matriu.
\end{defi*}

\vspace{1cm}

\begin{teo}[Teorema de Laplace]\label{teo:laplace}
	L'expansió del determinant per la primera fila és equivalent a a l'expansió del determinant per qualsevol fila o columna.

\begin{proof}
	La demostració del teorema parteix de la demostració de cada un dels següents lemmes.
\begin{lema}\label{lema:columna}
		L'expansió del determinant per la primera fila és equivalent a l'expansió del determinant per la primera columna: \[\det M = \sum_{j=1}^{n} m_{1,j}\cdot(-1)^{1+j}\det M_{1,j} = \sum_{i=1}^{n} m_{i,1}\cdot(-1)^{i+1}\det M_{i,1} \]
		%
		\begin{adjustwidth}{0.8cm}{}
		\begin{proof}
		Sigui $A \in \mathcal{M}_n(\mathbb{R})$. Per $n = 2$ és cert:
		\begin{equation*}
		\begin{gathered}		
		\det A = 
		\sum_{j=1}^{2} a_{1j}\cdot (-1)^{1+j}\det A_{1j} = a_{11}\begin{vmatrix}a_{22}\end{vmatrix} - a_{12}\begin{vmatrix}a_{21}\end{vmatrix} =\\
%
		=a_{11}a_{22} - a_{12}a_{21} =\\
%
		 =a_{11}\begin{vmatrix}a_{22}\end{vmatrix} - a_{21}\begin{vmatrix}a_{12}\end{vmatrix} = \sum_{i=1}^{2} a_{i1}\cdot (-1)^{i+1}\det A_{i1}\,.
		\end{gathered}
		\end{equation*}
		
		Suposem que és cert per $n-1$, per un valor arbitrari de $n>2$ (hipòtesi d'inducció). El determinant d'$A\in\mathcal{M}_n(\mathbb{R})$ vindrà donat per
		\begin{equation}\label{eq:det1}
			 \det A = \sum_{j=1}^{n} a_{1,j}\cdot(-1)^{1+j}\det A_{1,j}  \,.
		\end{equation} 
		El primer terme d'aquesta expressió serà $a_{1,1}\cdot C_{1,1}$, on $C_{1,1} =(-1)^{1+1}\cdot\det A_{1,1} = \det A_{1,1}$. Aquest coincideix exactament amb el primer terme de l'expansió per primera columna. 
		
		Analitzem la resta de termes ($j>1$). Sigui $\tilde{A} \defeq A_{1,j}$ i siguin $\tilde{a}_{\ell,m}$ els seus elements. La matriu $\tilde{A}$ és d'ordre $n-1$; per tant, per hipòtesi d'inducció, el seu determinant es pot calcular expandint per la primera columna: 
		\begin{equation}\label{eq:subdet}
			\det \tilde{A} = \sum_{\ell=1}^{n-1}\tilde{a}_{\ell,1}\cdot (-1)^{\ell+1}\det \tilde{A}_{\ell,1}\,. 
		\end{equation} 
		Com que aquesta matriu s'obté eliminant la primera fila i la $j$-èsima columna (on, recordi's, $j>1$), es té l'equivalència $\forall \ell \le n-1 : {\tilde{a}_{\ell,1} = a_{\ell+1,1}}\,$. De manera similar, $\tilde{A}_{\ell,1} = (A_{1,j})_{\ell,1}= A_{(1,\ell+1),(j,1)}$, on la notació $M_{(a,b),(\alpha,\beta)}$ denota la submatriu que s'obté eliminant les files $a$, $b$ i les columnes $\alpha$, $\beta$ de la matriu $M$. Visualitzant-ho:
		\begin{equation*}
			\begin{split}
			A =&
			\begin{pmatrix}
				a_{1,1}&	a_{1,2}&	\cdots&		a_{1,n}\\
				a_{2,1}&	a_{2,2}&	\cdots&		a_{2,n}\\
				\vdots&		\vdots&		\ddots&		\vdots\\
				a_{n,1}&	a_{n,2}&	\cdots&		a_{n,n}
			\end{pmatrix}\\
%
			A_{1,j} =&\
			\begin{blockarray}{rcccccc}
			\begin{block}{*{7}{>{\scriptstyle\color[gray]{0.6}}c<{}}}
				1&	\cdots&	 j-1&	j&	\cdots&	n-1 &\\
			\end{block}
			\begin{block}{(cccccc)*{1}{>{\scriptstyle\color[gray]{0.6}}l<{}}}
				\mathbf{a_{2,1}}&\cdots&a_{2,j-1}&a_{2,j+1}&\cdots&		a_{2,n}& 1\\
				\vdots&	   \ddots&	\vdots&	\vdots&	\ddots&		\vdots& \vdots\\
				\mathbf{a_{n,1}}&   \cdots&	a_{n,j-1}&	a_{n,j+1}&	\cdots&		a_{n,n}&	n-1\\
			\end{block}
			\end{blockarray}\\
%
			\tilde{A}_{\ell,1} =&
			\begin{pmatrix}
				a_{2,2}&\cdots&a_{2,j-1}&a_{2,j+1}&\cdots&		a_{2,n}\\
				\vdots&	\ddots&	   \vdots&	\vdots&	\ddots&	\vdots\\
				a_{\ell,2}&\cdots&a_{\ell,j-1}&a_{\ell,j+1}&\cdots& a_{\ell,n}\\
				a_{\ell+2,2}&\cdots&a_{\ell+2,j-1}&a_{\ell+2,j+1}&\cdots&	a_{\ell+2,n}\\
				\vdots&	\ddots&	\vdots&	\vdots&	\ddots&		\vdots\\
				a_{n,2}&   \cdots&	a_{n,j-1}&	a_{n,j+1}&\cdots&a_{n,n}\\
			\end{pmatrix} = A_{(1,\ell+1),(j,1)}
			\,.
			\end{split}
		\end{equation*}
		Per tant podem escriure \eqref{eq:subdet}, fent el canvi d'índex $k\defeq \ell+1$, com 
		\begin{equation}\label{eq:jthsubdet}
			\det A_{1,j} = \sum_{k=2}^{n} a_{k,1}\cdot(-1)^{k}\det A_{(1,k),(j,1)}\,.
		\end{equation}
		
		A continuació, separant el primer terme de \eqref{eq:det1}, i substituint $\det A_{1,j}$ per l'expressió \eqref{eq:jthsubdet} a la resta de termes, obtenim
		\begin{equation}\label{eq:parrafada}
		\begin{split}
			\det A ={}& a_{1,1}\cdot A_{1,1} \\
			&+\sum_{j=2}^{n} \left(a_{1,j} (-1)^{1+j}  \sum_{k=2}^{n} a_{k,1} (-1)^{k} \det A_{(1,k),(j,1)}\right)\,.
		\end{split}
		\end{equation}
		Ara manipulem algebraicament el sumatori  ($\sum_{j=2}^n [\cdots]$) per tal de ``girar'' els sumatoris:
		\begin{multline}\label{eq:sumofjth}
			\sum_{j=2}^{n} \left(a_{1,j} (-1)^{1+j}  \sum_{k=2}^{n} a_{k,1} (-1)^{k} \det A_{(1,k),(j,1)}\right)=\\
			=\sum_{j=2}^{n} \left(  \sum_{k=2}^{n} a_{1,j}\cdot a_{k,1} (-1)^{1+j+k} \det A_{(1,k),(j,1)}\right)=\\
			=\sum_{k=2}^{n} \left(  \sum_{j=2}^{n} a_{1,j}\cdot a_{k,1} (-1)^{1+j+k} \det A_{(1,k),(j,1)}\right)=\\
			=\sum_{k=2}^{n} \left(a_{k,1} (-1)^{k+1}  \sum_{j=2}^{n} a_{1,j} (-1)^{j} \det A_{(1,k),(j,1)}\right)\,.
		\end{multline}

		Ara, fent el mateix raonament que ens ha fet arribar a \eqref{eq:jthsubdet} però en direcció inversa, notem que el sumatori $\sum_{j=2}^{n}[\cdots]$ és l'expansió per primera fila del determinant de la matriu $A_{k,1}$. Per tant, l'expressió \eqref{eq:sumofjth} queda així:
		\begin{equation}\label{eq:simplesumofjth}
			\sum_{k=2}^{n} a_{k,1} (-1)^{k+1} \cdot A_{k,1} \,.
		\end{equation}
		Finalment, tornant a substituir \eqref{eq:simplesumofjth} en l'equació \eqref{eq:parrafada}, obtenim
		\begin{equation}
		\begin{split}
		\det A = a_{1,1}\cdot A_{1,1}
		+\sum_{k=2}^{n} a_{k,1}\cdot (-1)^{k+1} A_{k,1}\,.
		\end{split}
		\end{equation}
		Noteu que ara podem tornar a introduir el primer terme ($a_{1,1}\cdot A_{1,1}$, que és igual a $a_{1,1}\cdot(-1)^{1+1} A_{1,1}$) al sumatori:
		\begin{equation}
		\begin{split}
		\det A = \sum_{i=1}^{n} a_{i,1}\cdot (-1)^{i+1} A_{k,1}\,.
		\end{split}
		\end{equation}
		Aquesta darrera expressió és l'expansió per primera columna del determinant d'$A$.
		\end{proof}
		\end{adjustwidth}
	\end{lema}
%
\vspace{5mm}
%
\begin{lema}\label{lema:intercanvi}
		Sigui $A \in \mathcal{M}_n(\mathbb{R})$. El determinant d'una matriu $B$ que s'obté intercanviant dues files adjacents de la matriu $A$ és $\;-\det A$.
		
		\begin{adjustwidth}{0.8cm}{}
		\begin{proof}
			Per $n=2$, és cert: \[
			\begin{vmatrix}
			a & b \\
			c & d
			\end{vmatrix} = ad - cb = -(cb - ad) = -
			\begin{vmatrix}
			c & d \\
			a & b
			\end{vmatrix}\,. \]
			Suposem que és cert per $n - 1$, per un valor arbitrari de $n > 2$ (hipòtesi d'inducció). Sigui $A$ una matriu $n\times n$ i $B$ la matriu obtinguda intercanviant les files $r$ i $r+1$ d'$A$. 
			\[A =
			\begin{pmatrix}
			a_{1,1} &	\cdots &	a_{1,n} \\
			\vdots &	\ddots &	\vdots \\
			a_{r,1} &	\cdots &	a_{r,n} \\
			a_{r+1,1} &	\cdots &	a_{r+1,n} \\
			\vdots &	\ddots &	\vdots \\
			a_{n,1} &	\cdots &	a_{n,n} \\
			\end{pmatrix}\,, \quad
			%
			B =
			\begin{pmatrix}
			a_{1,1} &	\cdots &	a_{1,n} \\
			\vdots &	\ddots &	\vdots \\
			a_{r+1,1} &	\cdots &	a_{r+1,n} \\
			a_{r,1} &	\cdots &	a_{r,n} \\
			\vdots &	\ddots &	\vdots \\
			a_{n,1} &	\cdots &	a_{n,n} \\
			\end{pmatrix}\,.
			\]
			
			A partir del lema \ref{lema:columna}, podem calcular el determinant de $B$ expandint per la primera columna: 
			\begin{equation}\label{eq:det2}
				\det B = \sum_{i=1}^{n} b_{i,1}\cdot(-1)^{i+1}\det B_{i,1}\,.
			\end{equation}
			Per tots els índex $i$ tals que $i\ne r \land i\ne r+1$, els coeficients $b_{i,1}$ i $a_{i,1}$ són equivalents; d'altra banda, 
			\begin{equation}\label{eq:submat}
			\forall i \notin \{r,r+1\} :\quad B_{i,1} = (A_{i,1})^{f_r\rightleftarrows f_{r+1}}\,,
			\end{equation}
			on el superíndex $f_r\rightleftarrows f_{r+1}$ denota que s'intercanvien de posició les files $r$ i $r+1$. Intuïtivament, això és el mateix que constatar que intercanviar primer les files (d'on s'obté la matriu $B$) i després obtenir la submatriu (ergo, $B_{i,1}$) és equivalent a obtenir en primer lloc la submatriu (ergo, $A_{i,1}$) i intercanviar les files després (d'on s'obté $(A_{i,1})^{f_r\rightleftarrows f_{r+1}}$).
			
			La submatriu $B_{i,1}$ és d'ordre $n-1$; llavors, utilitzant l'hipòtesi d'inducció en el fet \eqref{eq:submat}, es té que $\;\det B_{i,1} = - \det A_{i,1}\,$. Per tant, utilitzant aquestes equivalències en l'eq. \eqref{eq:det2},
			\begin{equation}\label{eq:dev1}
			\begin{split}
			\det B =& 	-\left(\sum_{i\notin \{r, r+1\}} a_{i,1}\cdot(-1)^{i+1}\det A_{i,1}\right)\\
					&	+ b_{r,1}\cdot(-1)^{r+1}\det B_{r,1}\\
					&	+ b_{r+1,1}\cdot(-1)^{(r+1)+1}\det B_{r+1,1}\,
			\end{split}
			\end{equation}
			(s'han separat els termes $i=r$ i $i=r+1$ de la suma, ja que per aquests no valen les equivalències anteriors).
			En el cas on $i = r$, es té que $b_{r,1} = a_{r+1,1}$---ja que $B$ s'ha obtingut intercanviant les files $r$ i $r+1$ en $A$. Per la mateixa raó, $B_{r,1} = A_{r+1, 1}\,$, ja eliminar la fila $r$ en $B$ és equivalent a eliminar la fila $r+1$ en $A$ (aquí ``equivalent'' vol dir que s'obté la mateixa submatriu):
%
			\begin{alignat*}{2}
			B_{\color[gray]{0.6}r,1} = &
			\begin{pmatrix}
			\color[gray]{0.6}
			a_{1,1} &	a_{1,2} &	\cdots &	a_{1,n} \\
			\color[gray]{0.6}
			\vdots &	\vdots &	\ddots &	\vdots \\
			\color[gray]{0.6}
			a_{r-1,1} &	a_{r-1,2} &	\cdots &	a_{r-1,n} \\
			\color[gray]{0.6}
			a_{r+1,1} &	\color[gray]{0.6}a_{r+1,2} &\color[gray]{0.6}	\cdots &	\color[gray]{0.6}a_{r+1,n}  \\
			\color[gray]{0.6}
			a_{r,1} &	a_{r,1} &	\cdots &	a_{r,n} \\
			\color[gray]{0.6}
			a_{r+2,1} &	a_{r+2,2} &	\cdots &	a_{r+2,n} \\
			\color[gray]{0.6}
			\vdots &	\vdots &	\ddots &	\vdots \\
			\color[gray]{0.6}
			a_{n,1} &	a_{n,2} &	\cdots &	a_{n,n} \\
			\end{pmatrix}
%
			\\ \\
%
			A_{\color[gray]{0.6}r+1,1} = &
			\begin{pmatrix}
			\color[gray]{0.6}
			a_{1,1} &	a_{1,2} &	\cdots &	a_{1,n} \\
			\color[gray]{0.6}
			\vdots &	\vdots &	\ddots &	\vdots \\
			\color[gray]{0.6}
			a_{r-1,1} &	a_{r-1,2} &	\cdots &	a_{r-1,n} \\
			\color[gray]{0.6}
			a_{r,1} &	a_{r,1} &	\cdots &	a_{r,n} \\
			%
			\color[gray]{0.6}	a_{r+1,1} &	\color[gray]{0.6}a_{r+1,2} &\color[gray]{0.6}	\cdots &	\color[gray]{0.6}a_{r+1,n}  \\
			%
			\color[gray]{0.6}
			a_{r+2,1} &	a_{r+2,2} &	\cdots &	a_{r+2,n} \\
			\color[gray]{0.6}
			\vdots &	\vdots &	\ddots &	\vdots \\
			\color[gray]{0.6}
			a_{n,1} &	a_{n,2} &	\cdots &	a_{n,n} \\
			\end{pmatrix}\,.
			\end{alignat*}
			De manera similar, $b_{r+1, 1} = a_{r, 1}\;$ i $\;B_{r+1,1} = A_{r,1}$.
			
			Per tant, els últims dos termes de l'expressió \eqref{eq:development} es poden reescriure com
			\begin{equation*}\label{eq:penult}
			\begin{gathered}
			a_{r+1,1}\cdot(-1)^{r+1}\det A_{r+1,1}\\
			a_{r,1}\cdot(-1)^{(r+1)+1}\det A_{r,1}\,.
			\end{gathered}
			\end{equation*}
			Observem que ara en cada terme tot està ``en funció de'' $r$ i $r+1$, respectivament, exceptuant l'exponent del $-1$. Utilitzant la identitat $\forall a\in\mathbb{Z} : (-1)^a = -(-1)^{a\pm 1}$, podem tornar a reescriure com
			\begin{equation*}\label{eq:last}
			\begin{gathered}
			-a_{r+1,1}\cdot(-1)^{(r+1)+1}\det A_{r+1,1}\\
			-a_{r,1}\cdot(-1)^{r+1}\det A_{r,1}\,.
			\end{gathered}
			\end{equation*}
			
			Finalment, a partir d'això podem simplificar l'expressió \eqref{eq:dev1} reintroduint aquests termes al sumatori:
			\begin{equation}\label{eq:development}
			\begin{split}
			\det B =& 	-\left(\sum_{i\notin \{r, r+1\}} a_{i,1}\cdot(-1)^{i+1}\det A_{i,1}\right)\\
			&	-a_{r+1,1}\cdot(-1)^{(r+1)+1}\det A_{r+1,1}\\
			&	-a_{r,1}\cdot(-1)^{r+1}\det A_{r,1} = \\
			=&-\left(\sum_{i=1}^{n} a_{i,1}\cdot(-1)^{i+1}\det A_{i,1}\right)\,.
			\end{split}
			\end{equation}
			L'expressió entre parèntesis és el determinant de la matriu $A$ (expansió per primera columna). Per tant, $\det B = -\det A$.
		\end{proof}
		\end{adjustwidth}
	\end{lema}

\vspace{5mm}

És conseqüència directa d'aquest últim lema (\ref{lema:intercanvi}) que el determinant de la matriu obtinguda intercanviant dues files $r$ i $s$ qualssevol (no necessàriament adjacents) d'una matriu $A$ és $\ -\det A$, ja que intercanviar la posició de dues files és equivalent a fer un nombre imparell d'intercanvis entre files adjacents. Això es justifica seguidament.

Primer cal fer $s-r$ (suposant que $s>r$) intercanvis per posar la fila $r$ a la posició de $s$. Ara la fila $s$ es troba a la posició $s-1$, per tant només cal fer $(s-1)-r = s-r-1$ intercanvis de files adjacents. Llavors el nombre total d'intercanvis és $(s-r)+(s-r-1) = 2(s-r) - 1$, que és imparell.

\[
	\begin{blockarray}{cccc}
	\begin{block}{(ccc)*{1}{>{\color{red}}l<{}}}
	a_{1,1}&	\cdots&		a_{1,n}& 		\\
	\vdots&		\ddots&		\vdots& 		\\
	a_{r-1,1}&	\cdots&		a_{r-1,n}&    	\\
	\rd a_{r,1}&\rd\cdots&	\rd a_{r,n}&    \dcarr \\
	a_{r+1,1}&	\cdots&		a_{r+1,n}&    	\dcarr\\
	\vdots&		\ddots&		\vdots& 		\vdots\\
	a_{s-1,1}&	\cdots&		a_{s-1,n}&    	\dcarr \\
	\gre a_{s,1}&\gre\cdots&	\gre a_{s,n}&    \\
	a_{s+1,1}&	\cdots&		a_{s+1,n}&    	\\
	\vdots&		\ddots&		\vdots& 		\\
	a_{n,1}&	\cdots&		a_{n,n}& 		\\	
	\end{block}
	\end{blockarray}
	%
	\quad
	%
	\begin{blockarray}{cccc}
	\begin{block}{(ccc)*{1}{>{\color{ForestGreen}}l<{}}}
	a_{1,1}&	\cdots&		a_{1,n}& 		\\
	\vdots&		\ddots&		\vdots& 		\\
	a_{r-1,1}&	\cdots&		a_{r-1,n}&    	\\
	a_{r+1,1}&	\cdots&		a_{r+1,n}&    	\ucarr\\
	\vdots&		\ddots&		\vdots& 		\vdots\\
	a_{s-1,1}&	\cdots&		a_{s-1,n}&    	\ucarr\\
	\gre a_{s,1}&\gre\cdots&	\gre a_{s,n}&    \ucarr\\
	\rd a_{r,1}&\rd\cdots&	\rd a_{r,n}&    \\
	a_{s+1,1}&	\cdots&		a_{s+1,n}&    	\\
	\vdots&		\ddots&		\vdots& 		\\
	a_{n,1}&	\cdots&		a_{n,n}& 		\\	
	\end{block}
	\end{blockarray}
%
	\quad
%	
	\begin{pmatrix}
	a_{1,1}&	\cdots&		a_{1,n}\\
	\vdots&		\ddots&		\vdots\\
	a_{r-1,1}&	\cdots&		a_{r-1,n}\\
	\gre a_{s,1}&\gre\cdots&	\gre a_{s,n}\\
	a_{r+1,1}&	\cdots&		a_{r+1,n}\\
	\vdots&		\ddots&		\vdots\\
	a_{s-1,1}&	\cdots&		a_{s-1,n}\\
	\rd a_{r,1}&\rd\cdots&	\rd a_{r,n}\\
	a_{s+1,1}&	\cdots&		a_{s+1,n}\\
	\vdots&		\ddots&		\vdots\\
	a_{n,1}&	\cdots&		a_{n,n}\\
	\end{pmatrix}
\]

\begin{lema}\label{lema:iesimafila}
Sigui $M\in\mathcal{M}_n(\mathbb{R})$. L'expansió del determinant de $M$ per la primera fila és equivalent a l'expansió per la $i$-èsima fila:
\begin{multline*}
	\forall i \in \{1,2,\ldots, n\}:\\
	\det M = \sum_{j=1}^{n} m_{1,j}\cdot(-1)^{1+j}\det M_{1,j} = \sum_{j=1}^{n} m_{i,j}\cdot(-1)^{i+j}\det M_{i,j}\,.
\end{multline*}
	\begin{adjustwidth}{0.8cm}{}
	\begin{proof}
		Sigui $A\in\mathcal{M}_n(\mathbb{R})$ i sigui $B$ la matriu obtinguda movent la fila $i$ d'$A$ a la primera fila mitjançant $i-1$ intercanvis consecutius de files adjacents:
		\[
		A =\
		\begin{blockarray}{cccc}
		\begin{block}{(ccc)*{1}{>{\color{blue}}l<{}}}
		a_{1,1}&	\cdots&		a_{1,n}& 		\\
		a_{2,1}&	\cdots&		a_{2,n}& 		\ucarr\\
		\vdots&		\ddots&		\vdots& 		\vdots\\
		a_{i-1,1}&	\cdots&		a_{i-1,n}&    	\ucarr \\
		\bl a_{i,1}&\bl\cdots&	\bl a_{i,n}&    \ucarr \\
		a_{i+1,1}&	\cdots&		a_{i+1,n}&    	\\
		\vdots&		\ddots&		\vdots& 		\\
		a_{n,1}&	\cdots&		a_{n,n}& 		\\	
		\end{block}
		\end{blockarray}
%
		\qquad
%
		B =
		\begin{pmatrix}
		\bl a_{i,1}&\bl\cdots&	\bl a_{i,n}\\
		a_{1,1}&	\cdots&		a_{1,n}\\
		a_{2,1}&	\cdots&		a_{2,n}\\
		\vdots&		\ddots&		\vdots\\
		a_{i-1,1}&	\cdots&		a_{i-1,n}\\
		a_{i+1,1}&	\cdots&		a_{i+1,n}\\
		\vdots&		\ddots&		\vdots\\
		a_{n,1}&	\cdots&		a_{n,n}\\
		\end{pmatrix}\,.
		\]
		Llavors, com a conseqüència directa del lema \ref{lema:intercanvi}, 
		\begin{equation}\label{eq:detrel}
			\det A = (-1)^{i-1}\det B\,.
		\end{equation}
		
		Desenvolupem $\det B$ segons la seva definició.
		\begin{equation}\label{eq:det-iesima}
			\det B = \sum_{j=1}^{n} b_{1,j}\cdot (-1)^{1+j}\det B_{1,j}\,.
		\end{equation}
		Per construcció, $b_{1,j} = a_{i,j}$ per tot $j$. D'altra banda, la matriu $B_{1,j}$ és equivalent a la matriu $A_{i,j}$---ja que, novament per construcció, la primera fila de $B$ és la $i$-èsima fila d'$A$; vegi's el diagrama anterior per visualitzar-ho. Per tant, \eqref{eq:det-iesima} es pot reformular com
		\begin{equation}\label{eq:det-iesima2}
			\det B = \sum_{j=1}^{n} a_{i,j}\cdot (-1)^{1+j}\det A_{i,j}\,.
		\end{equation}
		
		Unint les equacions \eqref{eq:detrel} i \eqref{eq:det-iesima2}, obtenim
		\begin{multline}
			\det A = (-1)^{i-1}\sum_{j=1}^{n} a_{i,j}\cdot (-1)^{1+j}\det A_{i,j} =\\
			=\sum_{j=1}^{n} (-1)^{i-1}\cdot a_{i,j}\cdot (-1)^{1+j}\det A_{i,j}=\\
			=\sum_{j=1}^{n} a_{i,j}\cdot (-1)^{i+j}\det A_{i,j}\,.
		\end{multline}
		Aquesta darrera expressió és l'expansió del determinant d'$A$ per la $i$-èsima fila.
	\end{proof}
	\end{adjustwidth}
\end{lema}

Un cop demostrats els lemes \ref{lema:columna}, \ref{lema:intercanvi}, i \ref{lema:iesimafila} la demostració del teorema complet és gairebé immediata. Ja s'ha demostrat que l'expansió per la $i$-èsima fila és equivalent a l'expansió per primera fila (lema \ref{lema:iesimafila}), i que aquesta és equivalent a l'expansió per primera columna (lema \ref{lema:columna}). Podem fer exactament el mateix raonament que hem fet per demostrar el lema \ref{lema:iesimafila}, però per columnes en lloc de files, per demostrar que l'expansió per primera columna és equivalent a l'expansió per la $j$-èsima columna. Per tant queda demostrat que expandir per la $i$-èsima fila és equivalent a expandir per la $j$-èsima columna.

\end{proof}	


\end{teo}