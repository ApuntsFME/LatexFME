\section{Integración multiple}


\subsection{Integral de Riemann sobre rectangulos compactos}

\begin{defi}
	Un rectángulo de $\real^n$ es un producto $A := I_1 \times \dots \times I_n$
	donde $I_j \in \real$ son intervalos que suponemos acotados y no degenerados,
	es decir, ni vacios, ni reducidos a un punto.
	
	Si los $I_j$ son compactos o abiertos, también lo es $A$.
\end{defi}

\begin{defi}
	La medida o volúmen $n$-dimensional (o área si $n=2$) de un rectángulo
	acotado $A = I_1 \times \cdots \times I_n$ es el producto de las longitudes
	de sus costados, es decir
	\[
		\vol(A) = \text{long} (I_1) \times \dots \times \text{long} (I_n)
	\]
\end{defi}

\begin{obs}
	Recoredemos que denominamos partición de un intervalo compacto $[a,b]$ a un
	subconjunto finito de puntos $\mathcal{P}=\setb{x_0,x_1,\dots,x_n}$ tales que
	$a = x_0 < x_1 < \cdots < x_{N-1} < x_N = b$. La partición expresa el intervalo
	como la unión de $N$ subintervalos
	\[
		[a,b] = [x_0,x_1] \cup \cdots \cup [x_{N-1},x_N]
	\]
\end{obs}
\begin{obs*}
	Una partición $\mathcal{P}^\prime$ se dice que es más fina que otra
	$\mathcal{P}$ cuando $\mathcal{P} \subset \mathcal{P}^\prime$ (es decir, cuando
	tiene más puntos).
\end{obs*}

\begin{defi}
	Dado un rectángulo compacto $A = I_1 \times \cdots \times I_n$, denominamos
	una partición $\Pa$ de $A$ al resultado de hacer una partición $\Pa_j$ a cada
	intervalo $I_j$.
	La partición de $A$ viene representada por $\Pa = \Pa_1 \times \cdots \times
	\Pa_n$ y expresa el rectángulo $A$ como unión de $N = (\abs{\Pa_1}-1) \times
	\cdots \times (\abs{\Pa_n}-1)$ subrectángulos más pequeños.
\end{defi}

\begin{obs*}
	Sean $A^\prime,A^{\prime\prime}\subset A$ rectángulos de la partición, entonces
	$\mathring{A^\prime} \cap \mathring{A^{\prime\prime}} = \emptyset$
\end{obs*}

\begin{lema}
	Si $A$ es un rectángulo, y $\Pa$ una partición de $A$, se tiene que
	\[
		\vol(A) = \sum_{R \in \Pa} \vol(R)
	\]
\end{lema}

\begin{defi}
	Dadas dos particiones $\Pa = \prod\limits^n_{j=1} \Pa_j$ y
	$P^\prime = \prod\limits^n_{j=1} \Pa^\prime_j$ de un rectángulo $A$. Diremos
	que la partición $P^\prime$ es más fina que $\Pa$ si cada $\Pa^\prime_j$ es más
	fina que $\Pa_j$ (es decir, $P_j \subset P^\prime_j$ $\forall j \iff \Pa
	\subset \Pa$).

	Entonces, cada subrectángulo de $\Pa$ es unión de subrectángulos de
	$\Pa^\prime$
\end{defi}

\begin{defi}
	Sea $A \subset \real^n$ un rectángulo compacto y $f \colon A \to \real$ una
	función acotada. Sea $\Pa$ una partición de $A$. Para cada subrectángulo $R$
	de $\Pa$ escribimos
	\[
		m_R = \inf_{x \in R} f(x) \qquad M_R = \sup_{x \in R} g(x)
	\]
	Denominamos suma inferior y suma superior de $f$ respecto a $\Pa$ a los números
	\[
		s(f;\Pa) = \sum_R m_R\vol(R) \qquad S(f;\Pa) = \sum_R M_R\vol(R)
	\]
\end{defi}

\begin{obs}
	Sea $\Pa$ una partición de $A$
	\[
		m_A\vol(A) \leq s(f;\Pa) \leq S(f;\Pa) \leq M_A\vol(A)
	\]
\end{obs}
\begin{obs}
	Si $\Pa$ y $\Pa^\prime$ son dos particiones y $\Pa^\prime$ es más fina que
	$\Pa$, entonces
	\[
		s(f;\Pa) \leq s(f;\Pa^\prime) \leq S(f;\Pa^\prime) \leq S(f;\Pa)
	\]
\end{obs}

\begin{lema}
	Si $\Pa$ y $\Pa^\prime$ son dos particiones de un rectángulo $A$, existe una
	partición $\Pa^{\prime\prime}$ de $A$ que es más fina que $\Pa$ y que
	$\Pa^\prime$.
\end{lema}

\begin{col*}
	Si $\Pa, \Pa^\prime$ son dos particiones de $A$, entonces,
	$s(f;\Pa) \leq S(f;\Pa^\prime)$. Por lo tanto,
	$\setb{s(f;\Pa) \vert \Pa \text{ partición de } A}$ está acotado superiormente
	y $\setb{S(f;\Pa) \vert\Pa\text{ partición de } A}$ está acotado inferiormente.
\end{col*}

\begin{defi}
	Sea $A$ un rectángulo compacto y sea $f \colon A \to \real$ una función
	acotada. Denominamos integral inferior e integral superior de $f$ en $A$ a los
	números
	\[
		\underline{\int}_A f = \sup_{\Pa} s(f;\Pa) \qquad \text{y} \qquad
		\overline{\int}_A f = \inf_{\Pa} S(f;\Pa)
	\]
	donde el supremo y el ínfimo se toman sobre el conjunto de todas las posibles
	particiones $\Pa$ de $A$. Obviamente, $\underline{\int}_A f \leq
	\overline{\int}_A f$.
\end{defi}

\begin{defi}
	Diremos que una función $f$ acotada, es integrable en $A$ cuando sus integrales
	inferior y superior coinciden. En este caso, su valor común se denomina
	integral de Riemmand de $f$ en $A$ y se denota por
	\[
		\int_A f, \quad \int_A f(x) \dif^nx, \quad \int_A f(x_1,\dots,x_n)
		\dif x_1 \cdots \dif x_n \quad \text{o} \quad \int_A f \dif V
	\]
	En el caso de $n=2$ o $n=3$ se habla de integral doble o integral triple
	respectivamente, ya que es habitual poner dos o tres signos de integral para
	representarlas.
\end{defi}

\begin{prop}[Criterio de Riemman]
	Sea $A \subset \real^n$ un rectángulo compacto y $f \colon A \to \real$ una
	función acotada. $f$ es integrable Riemman sii $\forall \varepsilon > 0$,
	$\exists \Pa$ partición de $A$ tal que $S(f;\Pa)-s(f;\Pa) < \varepsilon$.
\end{prop}
\begin{proof}
	Cálculo I
\end{proof}

\begin{example*}
	\begin{itemize}
		\item[]
		\item Si $f \colon A \to \real$ constante, entonces  $f(x) = c$ y
			$m_R = M_R = c$ $\forall R$
			\[
				s(f;\Pa) = \sum_R m_R \vol(R) = c \sum_R \vol(R)
				= c\vol(A) = S(f;\Pa)
			\]
			Por lo tanto, $f$ es integrable Riemman, y además
			\[
				\int_A f = c\vol(a) \implies \int_A 1 = \vol(A)
			\]
		\item Consideramos la función $f \colon [0,1] \times [0,1] \to \real$
			definida por $f(x,y) = \begin{cases}
				0 \quad \text{si } x,y \in \q \\
				1 \quad \text{En otro caso}
			\end{cases}$, entonces, $m_R = 0$ y $M_R = \vol(R)$ para todo
			$R$, y entonces
			\[
				\underline{\int}_A f = 0 \qquad
				\overline{\int}_A f = \vol(A) = 1\times1 = 1
			\]
			Y por lo tanto, $f$ no es integrable Riemman.
	\end{itemize}
\end{example*}

\begin{prop}[Linealidad]
	Sea $A$ un rectángulo compacto y $f,g \colon A \to \real$ integrables Riemman.
	Entonces
	\begin{enumerate}[i)]
		\item $f+g$ es integrable Riemman y $\int_A (f+g) = \int_A f+\int_A g$.
		\item Si $\lambda \in \real$, entonces $\lambda f$ es integrable
			Riemman y $\int_A (\lambda f) = \lambda \int_A f$.
	\end{enumerate}

	Es decir, $\text{Rie}(A) = \setb{f \colon A \to \real \vert f \text{ integrable
	Riemman}}$ es un $\real$-espacio vectorial y 
	\[
		\begin{aligned}
			\text{Rie}(A) &\to \real \\ f &\mapsto \int_A f
		\end{aligned}
	\]
	es una forma lineal.
\end{prop}
\begin{proof}
	
\end{proof}
