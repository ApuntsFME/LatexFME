\documentclass[12pt]{article}

\usepackage[margin=1in]{geometry}
\usepackage[pdftex]{hyperref}
\usepackage{amsmath,amsthm,amssymb,graphicx,mathtools,tikz,hyperref,enumerate}
\usepackage{mdframed,cleveref,cancel,stackengine,pgfplots,pgf,mathrsfs}
\usepackage{xfrac,stmaryrd,commath}
\usepackage[spanish]{babel}

\newmdenv[leftline=false,topline=false]{topright}
\let\proof\relax
\usepackage[utf8]{inputenc}
\usetikzlibrary{positioning,arrows, calc}
\usetikzlibrary{external}
\tikzexternalize[prefix=figures/]
\pgfplotsset{compat=1.11}
\newcommand{\n}{\mathbb{N}}
\newcommand{\z}{\mathbb{Z}}
\newcommand{\q}{\mathbb{Q}}
\newcommand{\cx}{\mathbb{C}}
\newcommand{\real}{\mathbb{R}}
\newcommand{\E}{\mathbb{E}}
\newcommand{\F}{\mathbb{F}}
\newcommand{\R}{\mathcal{R}}
\newcommand{\bb}[1]{\mathbb{#1}}
\let\k\relax
\newcommand{\k}{\mathbf{k}}
\newcommand{\ita}[1]{\textit{#1}}
\newcommand\inv[1]{#1^{-1}}
\newcommand\setb[1]{\left\{#1\right\}}
\newcommand{\vbrack}[1]{\langle #1\rangle}
\newcommand{\determinant}[1]{\begin{vmatrix}#1\end{vmatrix}}
\newcommand{\Po}{\mathbb{P}}
\DeclareMathOperator{\Id}{Id}
\DeclareMathOperator{\rg}{rg}
\DeclareMathOperator{\car}{car}
\DeclareMathOperator{\im}{Im}
\let\emptyset\varnothing

\def\parse#1{
	\def\param{#1}
	\ifx\param\empty
	\else
	(#1)
	\fi
}

\hypersetup{
	colorlinks,
	linkcolor=blue
}
 
 \renewcommand*\contentsname{Contenidos}

\newtheoremstyle{break}% name
{}%         Space above, empty = `usual value'
{}%         Space below
{}% Body font
{}%         Indent amount (empty = no indent, \parindent = para indent)
{\bfseries}% Thm head font
{}%        Punctuation after thm head
{\newline}% Space after thm head: \newline = linebreak
{#1 #2 \normalfont \parse{#3}}%         Thm head spec

\newtheoremstyle{breakthm}% name
{}%         Space above, empty = `usual value'
{}%         Space below
{}% Body font
{}%         Indent amount (empty = no indent, \parindent = para indent)
{\bfseries}% Thm head font
{}%        Punctuation after thm head
{\newline}% Space after thm head: \newline = linebreak
{#1 \normalfont #3 \addcontentsline{toc}{subsubsection}{#1 #3}}%         Thm head spec
\newtheoremstyle{normal}% name
{}%         Space above, empty = `usual value'
{}%         Space below
{}% Body font
{}%         Indent amount (empty = no indent, \parindent = para indent)
{\bfseries}% Thm head font
{}%        Punctuation after thm head
{5pt plus 1pt minus 1pt}% Space after thm head: \newline = linebreak
{#1 #2 \normalfont #3}%         Thm head spec

\theoremstyle{normal}
\newtheorem{lema}{Lema}[subsection]
\newtheorem*{lema*}{Lema}
\newtheorem{obs}[lema]{Observación}
\newtheorem*{obs*}{Observación}

\theoremstyle{break}
\newtheorem{prop}[lema]{Proposición}
\newtheorem*{prop*}{Proposición}
\newtheorem*{proof}{Demostración}
\newtheorem{defi}[lema]{Definición}
\newtheorem*{defi*}{Definición}
\newtheorem{col}[lema]{Corolario}
\newtheorem*{col*}{Corolario}
\newtheorem{ej}[lema]{Ejercicio}
\newtheorem*{ej*}{Ejercicio}
\newtheorem{example}[lema]{Ejemplo}
\newtheorem*{example*}{Ejemplo}

\theoremstyle{breakthm}
\newtheorem{thm}{Teorema}[subsection]



 
\begin{document}
\date{}

\title{Cálculo integral} % TITULO 

\maketitle

\tableofcontents

\section{Series numéricas e integrales impropias}

\subsection{Series absolutamente convergentes y condicionalmente convergentes}

\begin{proof}[(1.3.10)]
    \[
    s_{2n+2}=s_{2n}+(-a_{2n+1}+a_{2n+2}) \leq s_{2n} \rightarrow
    (s_{2n}) \text{ decreixent} 
    \]
    \[
    s_{2n+3}=s_{2n+1}+(a_{2n+2}-a_{2n+3}) \geq s_{2n+1}
    \rightarrow (s_{2n+1}) \text{ creixent}
    \]
    
    Por tanto, tenemos que: $s_{2n+1} \leq s_{2n} \leq s_0 = a_0
    \implies (s_{2n+1})$ es creciente i acotada $\implies
    (s_{2n+1})$ tiene limite $s$.
    \[
        lim(s_{2n}-s_{2n-1})=lim(a_n)=0 \implies lim(s_{2n}) = s
        \implies lim(s_n)=s
    \]
    Finalmente, como $s$ esta dentro del intervalo de extremos
    $s_n$, $s_{n+1}$ y su longitud es $a_{n+1}$, tenemos que $\mid
    s-s_n \mid \leq a_{n+1}$.
\end{proof}

\subsection{Series de potencias}

\begin{proof}[(1.4.2)]
    Si $0 \leq s \leq r$ y $\sum |a_n| r^n$ converge $\implies \sum
    |a_n|s_n$ también, porqué $|a_n|s^n \leq |a_n|r^n$.
\end{proof}

\begin{proof}[(1.4.4)]
    Caso $0 < R < + \infty$: Sea $0<|x|<R$, $\exists c$ tal que
    $|x|<cR \iff \frac{1}{R} < \frac{c}{|x|}$, si $n$
    suficientemente grande, $|a_n|^{1/n} \leq \frac{c}{|x|} \iff
    \mid a_nx^n \mid \leq c^n$ (serie geométrica de razón $c<1$). \\
    Sea $|x|>R \iff \frac{1}{R} > \frac{1}{|x|}$, hay infinitos $n$
    tales que $|a_n|^{1/n} > \frac{1}{|x|} \implies \mid a_nx^n \mid >1
    \rightarrow a_nx^n$ no tiende a $0 \implies \sum a_nx^n$ no converge.
\end{proof}
 %INPUTS


\end{document}
