\documentclass[12pt]{article}

\usepackage[margin=1in]{geometry}
\usepackage[pdftex]{hyperref}
\usepackage{amsmath,amsthm,amssymb,graphicx,mathtools,tikz,hyperref,enumerate}
\usepackage{mdframed,cleveref,cancel,stackengine,pgfplots,pgf,mathrsfs}
\usepackage{xfrac,stmaryrd,commath}
%\usepackage[spanish]{babel}

\newmdenv[leftline=false,topline=false]{topright}
\let\proof\relax
\usepackage[utf8]{inputenc}
\usetikzlibrary{positioning,arrows, calc}
\usetikzlibrary{external}
\tikzexternalize[prefix=figures/]
\pgfplotsset{compat=1.11}

\newcommand{\n}{\mathbb{N}}
\newcommand{\z}{\mathbb{Z}}
\newcommand{\q}{\mathbb{Q}}
\newcommand{\cx}{\mathbb{C}}
\newcommand{\real}{\mathbb{R}}
\newcommand{\E}{\mathbb{E}}
\newcommand{\F}{\mathbb{F}}
\newcommand{\R}{\mathcal{R}}
\newcommand{\bb}[1]{\mathbb{#1}}
\let\k\relax
\newcommand{\k}{\mathbf{k}}
\newcommand{\ita}[1]{\textit{#1}}
\newcommand\inv[1]{#1^{-1}}
\newcommand\setb[1]{\left\{#1\right\}}
\newcommand{\vbrack}[1]{\langle #1\rangle}
\newcommand{\determinant}[1]{\begin{vmatrix}#1\end{vmatrix}}
\newcommand{\Po}{\mathbb{P}}
\DeclareMathOperator{\Id}{Id}
\DeclareMathOperator{\rg}{rg}
\DeclareMathOperator{\car}{car}
\DeclareMathOperator{\im}{Im}
\let\emptyset\varnothing

\def\parse#1{
	\def\param{#1}
	\ifx\param\empty
	\else
	(#1)
	\fi
}

\hypersetup{
	colorlinks,
	linkcolor=blue
}

 \renewcommand*\contentsname{Contenidos}

\newtheoremstyle{break}% name
{}%         Space above, empty = `usual value'
{}%         Space below
{}% Body font
{}%         Indent amount (empty = no indent, \parindent = para indent)
{\bfseries}% Thm head font
{}%        Punctuation after thm head
{\newline}% Space after thm head: \newline = linebreak
{#1 #2 \normalfont \parse{#3}}%         Thm head spec

\newtheoremstyle{breakthm}% name
{}%         Space above, empty = `usual value'
{}%         Space below
{}% Body font
{}%         Indent amount (empty = no indent, \parindent = para indent)
{\bfseries}% Thm head font
{}%        Punctuation after thm head
{\newline}% Space after thm head: \newline = linebreak
{#1 \normalfont #3 (#2)\addcontentsline{toc}{subsubsection}{#1 #3}}%         Thm head spec
\newtheoremstyle{normal}% name
{}%         Space above, empty = `usual value'
{}%         Space below
{}% Body font
{}%         Indent amount (empty = no indent, \parindent = para indent)
{\bfseries}% Thm head font
{}%        Punctuation after thm head
{5pt plus 1pt minus 1pt}% Space after thm head: \newline = linebreak
{#1 #2 \normalfont #3}%         Thm head spec

\theoremstyle{normal}
\newtheorem{lema}{Lema}[subsection]
\newtheorem*{lema*}{Lema}
\newtheorem{obs}[lema]{Observación}
\newtheorem*{obs*}{Observación}

\theoremstyle{break}
\newtheorem{prop}[lema]{Proposición}
\newtheorem*{prop*}{Proposición}
\newtheorem*{proof}{Demostración}
\newtheorem{defi}[lema]{Definición}
\newtheorem*{defi*}{Definición}
\newtheorem{col}[lema]{Corolario}
\newtheorem*{col*}{Corolario}
\newtheorem{ej}[lema]{Ejercicio}
\newtheorem*{ej*}{Ejercicio}
\newtheorem{example}[lema]{Ejemplo}
\newtheorem*{example*}{Ejemplo}

\theoremstyle{breakthm}
\newtheorem{thm}[lema]{Teorema}




\begin{document}
\date{}

\title{Cálculo integral} % TITULO

\maketitle

\tableofcontents

\chapter{Espai de probabilitat}

\section{Definició axiomàtica de probabilitat}

\begin{defi}[espai!de probabilitat]
    Un espai de probabilitat és un espai de mesura $(\Omega,\Asuc, p)$ tal que $p(\Omega)=1$.
\end{defi}

\begin{defi}[espai!mostral]
    Diem que $\Omega$ és l'espai mostral.
\end{defi}

\begin{defi}[conjunt!d'esdeveniments]
    Diem que $\Asuc$ és el conjunt d'esdeveniments o de successos.
\end{defi}

\begin{defi}[funció!de probabilitat]
    Diem que $p$ és la funció de probabilitat.
\end{defi}

\begin{obs}
    Recordem que $\lp\Omega,\Asuc\rp$ és un espai mesurable si $\Asuc\subseteq \Pa\lp\Omega\rp$ és una $\sigma$-àlgebra d'$\Omega$, és a dir,
    \begin{enumerate}[i)]
        \item $\emptyset \in \Asuc$,
        \item $A \in \Asuc \iff A^C \in \Asuc$,
        \item Si $\lc A_i\rc _{i\in\n}\subseteq \Asuc$, aleshores $\bigcup_{i\in\n}{A_i} \in \Asuc$.
    \end{enumerate}
    I que $\lp\Omega,\Asuc,\mu\rp$ és un espai de mesura si $\mu$ és una mesura sobre l'espai mesurable $\lp\Omega,\Asuc\rp$, és a dir,
    \begin{enumerate}[i)]
        \item $\mu(\emptyset) = 0$,
        \item $\forall A \in \Asuc,\quad \mu(A) \ge0$,
        \item ($\sigma$-additivitat) Si $\lc A_i\rc_{i\in\n}\subseteq\Asuc$ és tal que $\forall i \neq j, \, A_i \cap A_j = \emptyset$,
        aleshores 
        \[
            \mu\lp\bigcup_{i\in\n}{A_i}\rp = \sum_{i\in\n}{\mu(A_i)}.
        \]
    \end{enumerate}
\end{obs}

\begin{prop}
    Sigui $(\Omega,\Asuc, p)$ un espai de probabilitat. Aleshores,
    \begin{enumerate}[i)]
        \item Si $A_1, \dots, A_r \in \Asuc$ són tals que $\forall i\neq j,\, A_i \cup A_j = \emptyset,$ aleshores $p\lp \bigcap\limits_{i=1}^{r} A_i \rp= \sum\limits_{i=1}^{r} p\lp A_i \rp.$.
        \item \label{item:esp_prob_2}$A \in \Asuc \implies p(\overline{A})=1-p(A)$.
        \item \label{item:esp_prob_3}$A,B \in \Asuc, A \subseteq B \implies p\lp B\setminus A \rp = p\lp B\rp - p\lp A\rp$.
        \item $A,B \in \Asuc, A \subseteq B \implies p(A) \le p(B)$.
        \item \label{item:esp_prob_5}Successions monòtones:
        \begin{enumerate}[a)]
         \item Si $\left\{A_i\right\}_{i\in\n} \subseteq \Asuc$ són tals que $A_i\subseteq A_{i+1}$, aleshores $p\lp \bigcup\limits_{i\in\n} A_i\rp = \lim\limits_{i\to\infty} p\lp A_i\rp$.
         \item Si $\left\{A_i\right\}_{i\in\n} \subseteq \Asuc$ són tals que $A_i\supseteq A_{i+1}$, aleshores $p\lp \bigcap\limits_{i\in\n} A_i\rp = \lim\limits_{i\to\infty} p\lp A_i\rp$.
        \end{enumerate}
    \end{enumerate}
\end{prop}
\begin{proof}
    \begin{enumerate}
        \item[]
        \item Conseqüència directa de la $\sigma$-additivitat.
        \item Conseqüència diecta de \ref{item:esp_prob_2} usant que $\Asuc = A\cup A^C$.
        \item Com que $A\subseteq B,\, B=\lp B\setminus A\rp \cup A$ i, per tant, $p\lp B\setminus A \rp = p\lp B\rp - p\lp A\rp$.
        \item Conseqüència directa de \ref{item:esp_prob_3} ja que $p\lp B\setminus A\rp\geq 0$.
        \item 
        \begin{enumerate}[a)]
            \item[]
            \item Sigui $B_0=A_0$ i per $i>0$ sigui $B_i = A_i\setminus A_{i-1}$. Aleshores, es compleix que $\forall i\neq j, \, B_i \cap B_j =\emptyset$ i que $\bigcup\limits_{i\in \n} B_i = \bigcup\limits_{i\in\n} A_i$, de manera que
            \begin{gather*}
                p\lp\bigcup\limits_{i\in\n} A_i\rp = p\lp\bigcup\limits_{i\in\n} B_i\rp = \sum\limits_{i\in\n} p\lp B_i\rp =\\
                = \lim\limits_{N\to\infty} \sum_{i=0}^N p\lp B_i\rp = \lim\limits_{N\to\infty} p\lp \bigcup_{i=0}^N B_i\rp = \lim\limits_{N\to\infty} p\lp A_N\rp.
            \end{gather*}
            \item Anàleg al cas anterior.
        \end{enumerate}
    \end{enumerate}
\end{proof}

\ref{item:esp_prob_5} només es pot aplicar en casos molt particulars. En general, si tenim $A_i,\dots,A_r$ succcessos,
hi ha estimacions per a $p(\bigcup_{i=1}^{r}{A_i}$:

\begin{prop}[Desigualtats de Bonferroni]
    Siguin $A_1,\dots,A_r\in\Asuc$, i per $I\subseteq\{1,\dots,r\}$ sigui $A_I = bigcap_{i \in I}{A_i}$. Definim
    \[
        S_k = \sum_{I \in \{1,\dots,n\},\#I=k}{p(A_I)}
    \],
    això és, $S_1 = \sum{p(A_i)}$, $S_2 = \sum_{i \neq j}{p(A_i \cap A_j}$... Aleshores:
    \begin{enumerate}[i)]
         \item Si $t$ és parell,
            \[p\lp\bigcup_{i=1}^{r}{A_i}\rp \geq \sum_{i=1}^{r}{(-1)^{i+1}S_i}\]
         \item Si $t$ és senar,
            \[p\lp\bigcup_{i=1}^{r}{A_i}\rp \leq \sum_{i=1}^{r}{(-1)^{i+1}S_i}\]
    \end{enumerate}
\end{prop}

\begin{obs}
    Amb els casos $t=1$ (desigualtat de Boole) i $t=2$ es poden donar fites inferiors i superiors.
\end{obs}


\begin{example}[Espais de probabilitat]
    %TODO
\end{example}


\section{Probabilitat condicionada}
\begin{defi}[probabilitat!condicionada]
    Sigui $\lp \Omega, \Asuc, p\rp$ un espai de probabilitat i siguin $A, B \in \Asuc$. Definim la probabilitat d'$A$ condicionada a $B$ com
    \[
        p\lp A\mid B\rp = \frac{p\lp A\cap B\rp}{p\lp B\rp}.
    \]
\end{defi}
\begin{obs}
    Sigui $\lp \Omega, \Asuc, p\rp$ un espai de probabilitat i sigui $B \in \Asuc$ tal que $p\lp B\rp > 0$. Aleshores, l'aplicació
    \begin{align*}
        p_B\colon \Asuc &\to \real \\
        A &\mapsto p_B\lp A\rp := p\lp A\mid B\rp
    \end{align*}
    defineix un espai de probabilitat $\lp \Omega, \Asuc, p_B\rp$.
\end{obs}

\begin{prop}
    Sigui $I$ un conjunt numerable o finit i siguin $\lc A_i \rc_{i\in I} \subseteq \Asuc$ tals que 
    \begin{enumerate}[a)]
        \item $p\lp A_i\rp>0$,
        \item $i\neq j \implies A_i \cap A_j = \varnothing$,
        \item $\bigcup\limits_{i\in I} A_i = \Omega$.
    \end{enumerate}
    Aleshores,
    \begin{enumerate}[1)]
        \item Probabilitat total:
            \[
                p\lp B\rp=\sum_{i\in I} p\lp B\mid A_i\rp p\lp A_i\rp, \quad \forall B\in \Asuc.
            \]
        \item Fórmula de Bayes:
            \[
                p\lp A_i\mid B\rp=\frac{P\lp B\mid A_i\rp p\lp A_i\rp}{\sum_{j\in I} p\lp B\mid A_j\rp p\lp A_j\rp}, \quad \forall B\in \Asuc \text{ amb } p\lp B\rp>0.
            \]
    \end{enumerate}
\end{prop}

\begin{proof}
    \begin{enumerate}[1)]
        \item[]
        \item Com que els $A_i$ són disjunts i $\bigcup_{i \in I}{A_i} = \Omega$, $\forall B \in \Asuc$,
        $B = \bigcup_{i\in I}{B \cap A_i}$, i la unió és disjunta. Es té
        \[
            p(B) = p\lp\bigcup_{i\in I}{B \cap A_i}\rp \stackrel{\sigma-add.}{=} \sum_{i\in I}{p(B \cap A_i)} =
            \sum_{i\in I}{p(B|A_i)p(A_i)}.
        \]
        \item
        \begin{gather*}
            p(A_i|B) \sum_{j \in I}{p(B|A_j)p(A_j)} \stackrel{i)}{=} p(A_i|B)p(B) =\\
            \frac{p(B\cap A_i)}{p(B)}p(B) = p(B \cap A_i) = P\lp B\mid A_i\rp p\lp A_i\rp.
        \end{gather*}
    \end{enumerate}
\end{proof}

\begin{problema}[Ruïna del jugador]
    Partim d'un capital de $k$ unitats i, en cada jugada (sense memòria) augmenta o disminueix el capital en una unitat,
    amb probabilitats 1/2 i 1/2. El joc acaba si ens quedem sense capital o si assolim un objectiu $N$ ($N>k$).
    Quina és la probabilitat de perdre tot el capital?
\end{problema}
\begin{sol}
    
    Sigui $A_k$ el succés ``el jugador, començant amb capital $k$, perd''.
    
    Condicionem $A_k$ a la primera tirada de la moneda, definim $B$: ``la primera tirada ix cara''.
    
    \[p(A_k) = p(A_k|B)p(B) + p(A_k|\overline{B})p(\overline{B}) = p(A_k|B)\frac{1}{2} + p(A_k|\overline{B})\frac{1}{2} \implies\]
    \[\implies 2p(A_k)=p(A_{k-1}) + p(A_{k+1}) \implies p(A_k) - p(A_{k-1}) = p(A_{k+1}) - p(A_k) = C\]
    
    és constant. Per tant $p(A_k) = p(A_0)+kC$. Sabent que $p(A_0)=1$ i $p(A_N)=0$:
    \[0 = 1 + CN \implies C = -\frac{1}{n} \implies p(A_k) = 1 - \frac{k}{N}\]
\end{sol}

\section{Independència}
\begin{defi}
    Sigui $\lp \Omega, \Asuc, p\rp$ un espai de probabilitat, sigui $I$ un conjunt finit o numerable i sigui $\lc A_i\rc_{i\in I} \subseteq \Asuc$. Diem que els esdeveniments $A_i$ són independents si per tot $J\subseteq I$ amb $\abs{J}\in\n$ es té que
    \[
        p\lp\bigcap_{j\in J} A_j\rp = \prod_{j\in J} p\lp A_j\rp.
    \]
\end{defi}

\begin{example}
    \begin{enumerate}[1.]
        \item[]
        \item $\varnothing, \Omega$ són independents entre si.
        \item $A$ és independent amb si mateix si i només si $p\lp A\rp=1$ o $p\lp A\rp =0$.
    \end{enumerate}
\end{example}




 %INPUTS


\end{document}
