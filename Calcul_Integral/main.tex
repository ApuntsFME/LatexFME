\documentclass[12pt]{book}

\usepackage[utf8]{inputenc}
\usepackage[margin=1in]{geometry}
\usepackage[pdftex]{hyperref}
\usepackage{amsmath,amsthm,amssymb,graphicx,mathtools,tikz,hyperref,enumerate}
\usepackage{mdframed,cleveref,cancel,stackengine,pgfplots,pgf,mathrsfs,thmtools}
\usepackage{xfrac,stmaryrd,commath}
%\usepackage[spanish]{babel}

\newmdenv[leftline=false,topline=false]{topright}
\let\proof\relax
\usetikzlibrary{positioning,arrows, calc, babel}
\usetikzlibrary{external}
\tikzexternalize[prefix=figures/]
\pgfplotsset{compat=1.11}

\newcommand{\n}{\mathbb{N}}
\newcommand{\z}{\mathbb{Z}}
\newcommand{\q}{\mathbb{Q}}
\newcommand{\cx}{\mathbb{C}}
\newcommand{\real}{\mathbb{R}}
\newcommand{\E}{\mathbb{E}}
\newcommand{\F}{\mathbb{F}}
\newcommand{\R}{\mathcal{R}}
\newcommand{\C}{\mathcal{C}}
\newcommand{\Pa}{\mathcal{P}}
\newcommand{\bb}[1]{\mathbb{#1}}
\let\k\relax
\newcommand{\k}{\mathbf{k}}
\newcommand{\ita}[1]{\textit{#1}}
\newcommand\inv[1]{#1^{-1}}
\newcommand\setb[1]{\left\{#1\right\}}
\newcommand{\vbrack}[1]{\langle #1\rangle}
\newcommand{\determinant}[1]{\begin{vmatrix}#1\end{vmatrix}}
\newcommand{\Po}{\mathbb{P}}
\newcommand{\lp}{\left(}
\newcommand{\rp}{\right)}
\DeclareMathOperator{\Id}{Id}
\DeclareMathOperator{\rg}{rg}
\DeclareMathOperator{\car}{car}
\DeclareMathOperator{\im}{Im}
\DeclareMathOperator{\vol}{vol}
\DeclareMathOperator{\grad}{grad}
\DeclareMathOperator{\sinc}{sinc}
\DeclareMathOperator{\graf}{graf}
\DeclareMathOperator{\tq}{\text{t.q.}}
\let\emptyset\varnothing

\hypersetup{
	colorlinks,
	linkcolor=blue
}

\renewcommand*\contentsname{Contenidos}

\newtheoremstyle{break}% name
{}%         Space above, empty = `usual value'
{}%         Space below
{}% Body font
{}%         Indent amount (empty = no indent, \parindent = para indent)
{\bfseries}% Thm head font
{}%        Punctuation after thm head
{\newline}% Space after thm head: \newline = linebreak
{#1 #2 \normalfont \parse{#3}}%         Thm head spec

\newtheoremstyle{breakthm}% name
{}%         Space above, empty = `usual value'
{}%         Space below
{}% Body font
{}%         Indent amount (empty = no indent, \parindent = para indent)
{\bfseries}% Thm head font
{}%        Punctuation after thm head
{\newline}% Space after thm head: \newline = linebreak
{#1 \normalfont #3 (#2)\addcontentsline{toc}{subsubsection}{#1 #3}}%         Thm head spec
\newtheoremstyle{normal}% name
{}%         Space above, empty = `usual value'
{}%         Space below
{}% Body font
{}%         Indent amount (empty = no indent, \parindent = para indent)
{\bfseries}% Thm head font
{}%        Punctuation after thm head
{5pt plus 1pt minus 1pt}% Space after thm head: \newline = linebreak
{#1 #2 \normalfont #3}%         Thm head spec

\theoremstyle{normal}
\declaretheorem[style=breakthm,name=Lema,numberwithin=section]{lema}
\declaretheorem[style=breakthm,name=Lema,numbered=no]{lema*}
\declaretheorem[style=breakthm,name=Observación,sibling=lema]{obs}
\declaretheorem[style=breakthm,name=Observación,numbered=no]{obs*}

\theoremstyle{break}
\declaretheorem[style=break,name=Proposición,sibling=lema]{prop}
\declaretheorem[style=break,name=Proposición,numbered=no]{prop*}
\declaretheorem[style=break,name=Demostración,qed=$\square$,numbered=no]{proof}
\declaretheorem[style=break,name=Definición,sibling=lema]{defi}
\declaretheorem[style=break,name=Definición,numbered=no]{defi*}
\declaretheorem[style=break,name=Corolario,sibling=lema]{col}
\declaretheorem[style=break,name=Corolario,numbered=no]{col*}
\declaretheorem[style=break,name=Ejercicio,sibling=lema]{ej}
\declaretheorem[style=break,name=Ejercicio,numbered=no]{ej*}
\declaretheorem[style=break,name=Ejemplo,sibling=lema]{example}
\declaretheorem[style=break,name=Ejemplo,numbered=no]{example*}

\let\thm\relax
\theoremstyle{breakthm}
\declaretheorem[style=breakthm,name=Teorema,sibling=lema]{teo}

\title{Cálculo Integral}
\author{O. Benedito \and J. Castellví \and E. Lanchares \and M. Ortega}
\date{\ifcase \month \or Enero\or Febrero\or Marzo\or %
Abril\or Mayo \or Junio\or Julio\or 
Agosto\or Setiembre\or Octubre\or Noviembre\or %
Diciembre\fi \:  \number \year}


\begin{document}
 
\maketitle

\tableofcontents

\chapter{Espai de probabilitat}

\section{Definició axiomàtica de probabilitat}

\begin{defi}[espai!de probabilitat]
    Un espai de probabilitat és un espai de mesura $(\Omega,\Asuc, p)$ tal que $p(\Omega)=1$.
\end{defi}

\begin{defi}[espai!mostral]
    Diem que $\Omega$ és l'espai mostral.
\end{defi}

\begin{defi}[conjunt!d'esdeveniments]
    Diem que $\Asuc$ és el conjunt d'esdeveniments o de successos.
\end{defi}

\begin{defi}[funció!de probabilitat]
    Diem que $p$ és la funció de probabilitat.
\end{defi}

\begin{obs}
    Recordem que $\lp\Omega,\Asuc\rp$ és un espai mesurable si $\Asuc\subseteq \Pa\lp\Omega\rp$ és una $\sigma$-àlgebra d'$\Omega$, és a dir,
    \begin{enumerate}[i)]
        \item $\emptyset \in \Asuc$,
        \item $A \in \Asuc \iff A^C \in \Asuc$,
        \item Si $\lc A_i\rc _{i\in\n}\subseteq \Asuc$, aleshores $\bigcup_{i\in\n}{A_i} \in \Asuc$.
    \end{enumerate}
    I que $\lp\Omega,\Asuc,\mu\rp$ és un espai de mesura si $\mu$ és una mesura sobre l'espai mesurable $\lp\Omega,\Asuc\rp$, és a dir,
    \begin{enumerate}[i)]
        \item $\mu(\emptyset) = 0$,
        \item $\forall A \in \Asuc,\quad \mu(A) \ge0$,
        \item ($\sigma$-additivitat) Si $\lc A_i\rc_{i\in\n}\subseteq\Asuc$ és tal que $\forall i \neq j, \, A_i \cap A_j = \emptyset$,
        aleshores 
        \[
            \mu\lp\bigcup_{i\in\n}{A_i}\rp = \sum_{i\in\n}{\mu(A_i)}.
        \]
    \end{enumerate}
\end{obs}

\begin{prop}
    Sigui $(\Omega,\Asuc, p)$ un espai de probabilitat. Aleshores,
    \begin{enumerate}[i)]
        \item Si $A_1, \dots, A_r \in \Asuc$ són tals que $\forall i\neq j,\, A_i \cup A_j = \emptyset,$ aleshores $p\lp \bigcap\limits_{i=1}^{r} A_i \rp= \sum\limits_{i=1}^{r} p\lp A_i \rp.$.
        \item \label{item:esp_prob_2}$A \in \Asuc \implies p(\overline{A})=1-p(A)$.
        \item \label{item:esp_prob_3}$A,B \in \Asuc, A \subseteq B \implies p\lp B\setminus A \rp = p\lp B\rp - p\lp A\rp$.
        \item $A,B \in \Asuc, A \subseteq B \implies p(A) \le p(B)$.
        \item \label{item:esp_prob_5}Successions monòtones:
        \begin{enumerate}[a)]
         \item Si $\left\{A_i\right\}_{i\in\n} \subseteq \Asuc$ són tals que $A_i\subseteq A_{i+1}$, aleshores $p\lp \bigcup\limits_{i\in\n} A_i\rp = \lim\limits_{i\to\infty} p\lp A_i\rp$.
         \item Si $\left\{A_i\right\}_{i\in\n} \subseteq \Asuc$ són tals que $A_i\supseteq A_{i+1}$, aleshores $p\lp \bigcap\limits_{i\in\n} A_i\rp = \lim\limits_{i\to\infty} p\lp A_i\rp$.
        \end{enumerate}
    \end{enumerate}
\end{prop}
\begin{proof}
    \begin{enumerate}
        \item[]
        \item Conseqüència directa de la $\sigma$-additivitat.
        \item Conseqüència diecta de \ref{item:esp_prob_2} usant que $\Asuc = A\cup A^C$.
        \item Com que $A\subseteq B,\, B=\lp B\setminus A\rp \cup A$ i, per tant, $p\lp B\setminus A \rp = p\lp B\rp - p\lp A\rp$.
        \item Conseqüència directa de \ref{item:esp_prob_3} ja que $p\lp B\setminus A\rp\geq 0$.
        \item 
        \begin{enumerate}[a)]
            \item[]
            \item Sigui $B_0=A_0$ i per $i>0$ sigui $B_i = A_i\setminus A_{i-1}$. Aleshores, es compleix que $\forall i\neq j, \, B_i \cap B_j =\emptyset$ i que $\bigcup\limits_{i\in \n} B_i = \bigcup\limits_{i\in\n} A_i$, de manera que
            \begin{gather*}
                p\lp\bigcup\limits_{i\in\n} A_i\rp = p\lp\bigcup\limits_{i\in\n} B_i\rp = \sum\limits_{i\in\n} p\lp B_i\rp =\\
                = \lim\limits_{N\to\infty} \sum_{i=0}^N p\lp B_i\rp = \lim\limits_{N\to\infty} p\lp \bigcup_{i=0}^N B_i\rp = \lim\limits_{N\to\infty} p\lp A_N\rp.
            \end{gather*}
            \item Anàleg al cas anterior.
        \end{enumerate}
    \end{enumerate}
\end{proof}

\ref{item:esp_prob_5} només es pot aplicar en casos molt particulars. En general, si tenim $A_i,\dots,A_r$ succcessos,
hi ha estimacions per a $p(\bigcup_{i=1}^{r}{A_i}$:

\begin{prop}[Desigualtats de Bonferroni]
    Siguin $A_1,\dots,A_r\in\Asuc$, i per $I\subseteq\{1,\dots,r\}$ sigui $A_I = bigcap_{i \in I}{A_i}$. Definim
    \[
        S_k = \sum_{I \in \{1,\dots,n\},\#I=k}{p(A_I)}
    \],
    això és, $S_1 = \sum{p(A_i)}$, $S_2 = \sum_{i \neq j}{p(A_i \cap A_j}$... Aleshores:
    \begin{enumerate}[i)]
         \item Si $t$ és parell,
            \[p\lp\bigcup_{i=1}^{r}{A_i}\rp \geq \sum_{i=1}^{r}{(-1)^{i+1}S_i}\]
         \item Si $t$ és senar,
            \[p\lp\bigcup_{i=1}^{r}{A_i}\rp \leq \sum_{i=1}^{r}{(-1)^{i+1}S_i}\]
    \end{enumerate}
\end{prop}

\begin{obs}
    Amb els casos $t=1$ (desigualtat de Boole) i $t=2$ es poden donar fites inferiors i superiors.
\end{obs}


\begin{example}[Espais de probabilitat]
    %TODO
\end{example}


\section{Probabilitat condicionada}
\begin{defi}[probabilitat!condicionada]
    Sigui $\lp \Omega, \Asuc, p\rp$ un espai de probabilitat i siguin $A, B \in \Asuc$. Definim la probabilitat d'$A$ condicionada a $B$ com
    \[
        p\lp A\mid B\rp = \frac{p\lp A\cap B\rp}{p\lp B\rp}.
    \]
\end{defi}
\begin{obs}
    Sigui $\lp \Omega, \Asuc, p\rp$ un espai de probabilitat i sigui $B \in \Asuc$ tal que $p\lp B\rp > 0$. Aleshores, l'aplicació
    \begin{align*}
        p_B\colon \Asuc &\to \real \\
        A &\mapsto p_B\lp A\rp := p\lp A\mid B\rp
    \end{align*}
    defineix un espai de probabilitat $\lp \Omega, \Asuc, p_B\rp$.
\end{obs}

\begin{prop}
    Sigui $I$ un conjunt numerable o finit i siguin $\lc A_i \rc_{i\in I} \subseteq \Asuc$ tals que 
    \begin{enumerate}[a)]
        \item $p\lp A_i\rp>0$,
        \item $i\neq j \implies A_i \cap A_j = \varnothing$,
        \item $\bigcup\limits_{i\in I} A_i = \Omega$.
    \end{enumerate}
    Aleshores,
    \begin{enumerate}[1)]
        \item Probabilitat total:
            \[
                p\lp B\rp=\sum_{i\in I} p\lp B\mid A_i\rp p\lp A_i\rp, \quad \forall B\in \Asuc.
            \]
        \item Fórmula de Bayes:
            \[
                p\lp A_i\mid B\rp=\frac{P\lp B\mid A_i\rp p\lp A_i\rp}{\sum_{j\in I} p\lp B\mid A_j\rp p\lp A_j\rp}, \quad \forall B\in \Asuc \text{ amb } p\lp B\rp>0.
            \]
    \end{enumerate}
\end{prop}

\begin{proof}
    \begin{enumerate}[1)]
        \item[]
        \item Com que els $A_i$ són disjunts i $\bigcup_{i \in I}{A_i} = \Omega$, $\forall B \in \Asuc$,
        $B = \bigcup_{i\in I}{B \cap A_i}$, i la unió és disjunta. Es té
        \[
            p(B) = p\lp\bigcup_{i\in I}{B \cap A_i}\rp \stackrel{\sigma-add.}{=} \sum_{i\in I}{p(B \cap A_i)} =
            \sum_{i\in I}{p(B|A_i)p(A_i)}.
        \]
        \item
        \begin{gather*}
            p(A_i|B) \sum_{j \in I}{p(B|A_j)p(A_j)} \stackrel{i)}{=} p(A_i|B)p(B) =\\
            \frac{p(B\cap A_i)}{p(B)}p(B) = p(B \cap A_i) = P\lp B\mid A_i\rp p\lp A_i\rp.
        \end{gather*}
    \end{enumerate}
\end{proof}

\begin{problema}[Ruïna del jugador]
    Partim d'un capital de $k$ unitats i, en cada jugada (sense memòria) augmenta o disminueix el capital en una unitat,
    amb probabilitats 1/2 i 1/2. El joc acaba si ens quedem sense capital o si assolim un objectiu $N$ ($N>k$).
    Quina és la probabilitat de perdre tot el capital?
\end{problema}
\begin{sol}
    
    Sigui $A_k$ el succés ``el jugador, començant amb capital $k$, perd''.
    
    Condicionem $A_k$ a la primera tirada de la moneda, definim $B$: ``la primera tirada ix cara''.
    
    \[p(A_k) = p(A_k|B)p(B) + p(A_k|\overline{B})p(\overline{B}) = p(A_k|B)\frac{1}{2} + p(A_k|\overline{B})\frac{1}{2} \implies\]
    \[\implies 2p(A_k)=p(A_{k-1}) + p(A_{k+1}) \implies p(A_k) - p(A_{k-1}) = p(A_{k+1}) - p(A_k) = C\]
    
    és constant. Per tant $p(A_k) = p(A_0)+kC$. Sabent que $p(A_0)=1$ i $p(A_N)=0$:
    \[0 = 1 + CN \implies C = -\frac{1}{n} \implies p(A_k) = 1 - \frac{k}{N}\]
\end{sol}

\section{Independència}
\begin{defi}
    Sigui $\lp \Omega, \Asuc, p\rp$ un espai de probabilitat, sigui $I$ un conjunt finit o numerable i sigui $\lc A_i\rc_{i\in I} \subseteq \Asuc$. Diem que els esdeveniments $A_i$ són independents si per tot $J\subseteq I$ amb $\abs{J}\in\n$ es té que
    \[
        p\lp\bigcap_{j\in J} A_j\rp = \prod_{j\in J} p\lp A_j\rp.
    \]
\end{defi}

\begin{example}
    \begin{enumerate}[1.]
        \item[]
        \item $\varnothing, \Omega$ són independents entre si.
        \item $A$ és independent amb si mateix si i només si $p\lp A\rp=1$ o $p\lp A\rp =0$.
    \end{enumerate}
\end{example}




 %INPUTS
\chapter{Espacios topológicos y aplicaciones continuas}

\begin{eje}
    \begin{enumerate}[(a)]
        \item[]
        \item Comprobamos que
            \begin{enumerate}[i)]
                \item El intervalo $\lp a, a \rp = \emptyset \in \T$ y $\lp -\infty, +\infty \rp = X \in \T$.
                \item Sea $U = \bigcup\limits_{i \in I} U_i$ la unión de un numero arbitrario de abiertos, entonces, $\forall x \in U, \exists i \in I \tq x \in U_i \subseteq U$ y por tanto $x$ es un punto interior y $U$ es un abierto.
                \item Sea $U = \bigcap\limits_{i = 1}^n U_i$ la intersección de un numero finito de abiertos, entonces, $\forall x \in U, \forall i \in \left\{ 1, \dots, n \right\}, \exists a_i, b_i \tq x \in \left(a_i, b_i\right) \subseteq U_i$, porque $x \in U_i$ y $U_i$ abierto. Sean
                    \begin{gather*}
                        a = \max_{i \in \left\{ 1, \dots, n \right\}} \left\{a_i\right\}, \\
                        b = \min_{i \in \left\{ 1, \dots, n \right\}} \left\{b_i\right\},
                    \end{gather*}
                entonces $x \in \lp a, b \rp \subseteq \bigcap\limits_{i = 1}^n U_i$ y por tanto $x$ es un punto interior y $U$ es un abierto.
            \end{enumerate}
        \item Sea $\B = \left\{ \lp a, b \rp \colon a, b \in X \cup \left\{ \pm \infty \right\} \right\}$, comprobamos que
            \begin{enumerate}[i)]
                \item $\forall x \in X, x \in \lp -\infty, +\infty \rp \in \B$.
                \item $\forall a, b, c, d \in X \cup \left\{ \pm \infty \right\}$, si $\lp a, b \rp \cap \lp c, d \rp \neq \emptyset$, entonces,
                    \begin{gather*}
                        \alpha = \max \left\{ a, c \right\}, \\
                        \beta = \min \left\{ b, d \right\}, \\
                        \lp a, b \rp \cap \lp c, d \rp = \lp \alpha, \beta \rp.
                    \end{gather*}
                    y por tanto $\forall x \in \lp a, b \rp \cap \lp c, d \rp, x \in \lp \alpha, \beta \rp \subseteq \lp a, b \rp \cap \lp c, d \rp$, y $\lp \alpha, \beta \rp \in \B$.
            \end{enumerate}
        \item Si vemos que $\forall x \in X, \left\{ x \right\}$ es un abierto, ya abremos acabado, ya que todo conjunto de $\Pa \lp X \rp$ contiene únicamente puntos de $X$ y por lo tanto sera la unión de abiertos. En $\z$, $\forall x \in \z$, tenemos que $\lc x \rc = \lp n-1, n+1 \rp$ y por tanto ya estamos. En $\n$, $\forall x \in \n \setminus \lc 1 \rc$, tenemos que $\lc x \rc = \lp n-1, n+1 \rp$ y $\lc 1 \rc = \lp -\infty, 2 \rp$ y por tanto ya estamos, suponiendo que $0 \notin \n$.
        \item Sea $X$ un espacio topológico con la topología del orden, entonces $\forall x, y \in X, x < y$,
            \begin{itemize}
                \item Si $\exists z$ tal que $x < z < y$, entonces $x \in \lp -\infty, z \rp, y \in \lp z, +\infty \rp, \lp -\infty, z \rp \cap \lp z, +\infty \rp = \emptyset$.
                \item Si $\nexists z$ tal que $x < z < y$, entonces $x \in \lp -\infty, y \rp, y \in \lp x, +\infty \rp, \lp -\infty, y \rp \cap \lp x, +\infty \rp = \emptyset$.
            \end{itemize}
        \item Aquest dibuix està en contrucció. % TODO dibuix!!
        \item Tenemos que ver que $\T_{\text{ord}} \subset \T_\leq$, es decir, $\T_{\text{ord}} \subseteq \T_\leq$ y $\T_\leq \neq \T_{\text{ord}}$.
            \begin{itemize}
                \item Veamos que $\T_{\text{ord}} \subseteq \T_\leq$. Sea $U \subseteq \T_{\text{ord}},\, \forall x \equiv \lp x_1, x_2 \rp \in U, \, \exists r \in \real^+ \tq B_r \lp x \rp \subseteq U$, y por tanto, $A = \lp \lp x_1 - r, x_2 \rp, \lp x_1 + r, x_2 \rp \rp \subseteq B_r \lp x \rp, A \in \T_\leq$. Así pues, todos los puntos de $U$ son interiores en la topología del orden y por tanto $U \in \T_\leq$.
                \item Veamos que $\T_\leq \neq \T_{\text{ord}}$. Sean $x = \lp 0,0 \rp, y = \lp 0, 1 \rp$, entonces $\lp x, y \rp \in \T_\leq, \lp x, y \rp \notin \T_{\text{ord}}$. Por tanto $\T_\leq \neq \T_{\text{ord}}$.
            \end{itemize}
        \item Son las topologías discretas.
    \end{enumerate}
\end{eje}
\begin{eje}
    \begin{enumerate}[(a)]
        \item[]
        \item Comprobamos que
            \begin{enumerate}[i)]
                \item $\emptyset = \left[ x, x \rp \in \T_\ell$ y $X = \bigcup\limits_{x \in X} \left[ x, \infty \rp \in \T_\ell$.
                \item Sea $U = \bigcup\limits_{i \in I} U_i$ la unión de un numero arbitrario de abiertos, entonces, $\forall x \in U, \exists i \in I \tq x \in U_i \subseteq U$ y por tanto $x$ es un punto interior y $U$ es un abierto.
                \item Sea $U = \bigcap\limits_{i = 1}^n U_i$ la intersección de un numero finito de abiertos, entonces, $\forall x \in U, \forall i \in \left\{ 1, \dots, n \right\}, \exists a_i, b_i \tq x \in \left[a_i, b_i\right) \subseteq U_i$, porque $x \in U_i$ y $U_i$ abierto. Sean
                    \begin{gather*}
                        a = \max_{i \in \left\{ 1, \dots, n \right\}} \left\{a_i\right\}, \\
                        b = \min_{i \in \left\{ 1, \dots, n \right\}} \left\{b_i\right\},
                    \end{gather*}
                entonces $x \in \left[ a, b \rp \subseteq \bigcap\limits_{i = 1}^n U_i$ y por tanto $x$ es un punto interior y $U$ es un abierto.
            \end{enumerate}
        \item Sea $\B = \left\{ \left[ a, b \rp \colon a \in X, b \in X \cup \left\{ \infty \right\} \right\}$, comprobamos que
            \begin{enumerate}[i)]
                \item $\forall x \in X, x \in \left[ x, \infty \rp \in \B$.
                \item $\forall a, b, c, d \in X \cup \left\{ \infty \right\}$, si $\left[ a, b \rp \cap \left[ c, d \rp \neq \emptyset$, entonces,
                    \begin{gather*}
                        \alpha = \max \left\{ a, c \right\}, \\
                        \beta = \min \left\{ b, d \right\}, \\
                        \left[ a, b \rp \cap \left[ c, d \rp = \left[ \alpha, \beta \rp.
                    \end{gather*}
                    y por tanto $\forall x \in \left[ a, b \rp \cap \left[ c, d \rp, x \in \left[ \alpha, \beta \rp \subseteq \left[ a, b \rp \cap \left[ c, d \rp$, y $\left[ \alpha, \beta \rp \in \B$.
            \end{enumerate}
        \item \item[] % OJO amb aqueta cutrada
            \begin{center}
                \begin{tabular}{|l||c|c|c|c|c|c|} \hline
                    & $\lp a, b \rp$ & $\left[ a, b \rp$ & $\lp a, b \right]$ & $\left[ a,b \right]$ & $\lc 0 \rc \cup \lc \sfrac{1}{n} \rc_{n\geq 1}$ & $\lc 0 \rc \cup \lc \sfrac{-1}{n} \rc_{n\geq 1}$ \\ \hline \hline
                    Adherencia & $\left[a,b\rp$ & $\left[ a, b \rp$ & $\left[ a, b \right]$ & $\left[ a, b \right]$ & $\lc 0 \rc \cup \lc \sfrac{1}{n} \rc_{n\geq 1}$ & $\lc 0 \rc \cup \lc \sfrac{-1}{n} \rc_{n\geq 1}$ \\ \hline
                    Interior & $\lp a, b \rp$ & $\left[ a, b \rp$ & $\lp a, b \rp$ & $\left[ a, b \rp$ & $\emptyset$ & $\emptyset$ \\ \hline
                    Frontera & $\lc a \rc$ & $\emptyset$ & $\lc a, b \rc$ & $\lc b \rc$ & $\lc 0 \rc \cup \lc \sfrac{1}{n} \rc_{n\geq 1}$ & $\lc 0 \rc \cup \lc \sfrac{-1}{n} \rc_{n\geq 1}$ \\ \hline
                    Acumulación & $\left[a,b\rp$ & $\left[ a, b \rp$ & $\left[ a, b \rp$ & $\left[ a, b \right)$ & $\lc 0 \rc$ & $\emptyset$ \\ \hline
                    Puntos aislados & $\emptyset$ & $\emptyset$ & $\lc b \rc$ & $\lc b \rc$ & $\lc \sfrac{1}{n} \rc_{n\geq 1}$ & $\lc 0 \rc \cup \lc \sfrac{-1}{n} \rc_{n\geq 1}$ \\ \hline
                \end{tabular}
            \end{center}
    \end{enumerate}
\end{eje}
\begin{eje}
    \begin{enumerate}[(a)]
        \item[]
        \item Comprobamos que
            \begin{enumerate}[i)]
                \item $\emptyset = \left[ x, x \rp \in \T_\ell$
                \item Sea $U = \bigcup\limits_{i \in I} U_i$ la unión de un numero arbitrario de abiertos, entonces, $\forall x \in U, \exists i \in I \tq x \in U_i \subseteq U$ y por tanto $x$ es un punto interior y $U$ es un abierto.
                \item Sea $U = \bigcap\limits_{i = 1}^n U_i$ la intersección de un numero finito de abiertos, entonces, $\forall x \in U, \forall i \in \left\{ 1, \dots, n \right\}, \exists a_i, b_i \tq x \in \left(a_i, b_i\right) \subseteq U_i$. Sean
                    \begin{gather*}
                        a = \max_{i \in \left\{ 1, \dots, n \right\}} \left\{a_i\right\}, \\
                        b = \min_{i \in \left\{ 1, \dots, n \right\}} \left\{b_i\right\},
                    \end{gather*}
                entonces $x \in \lp a, b \rp \subseteq \bigcap\limits_{i = 1}^n U_i$ y por tanto $x$ es un punto interior y $U$ es un abierto.
            \end{enumerate}
    \end{enumerate}
\end{eje}

\begin{eje}
 Este ejercicio aún no está resuelto.
\end{eje}

\begin{eje}
 Este ejercicio aún no está resuelto
\end{eje}

\begin{eje}
 Este ejercicio aún no está resuelto
\end{eje}

\begin{eje}
 Sea $\T$ una topología y $X$ un espacio topológico. Por definición de $\psi$ se tiene que $\forall A\subseteq X,\, \psi\lp A\rp =\overline{A}$ es un cerratdo de $\T$. Por lo tanto, podemos definir un conjunto de abiertos de $\T$ como
 \[
  \B = \lc B\subseteq X\, | \,\exists A \subseteq X \tq B=X\setminus \psi\lp A\rp \rc
 \]
 que son abiertos por ser el complementario de un cerrado ($\overline{A}$ es el cerrado más pequeño que contiene a $A$).
 
 Ahora demostraremos que $\B$ es una base de la topología $\T$. 

 Primero veamos que
 \begin{gather*}
  X\setminus \psi\lp \varnothing\rp = X\setminus\varnothing = X \in \B\\
  X\setminus \psi\lp X\rp =X\setminus X = \varnothing \in \B.
 \end{gather*}
 
Entonces tenemos lo siguiente:
\begin{enumerate}[i)]
 \item $\forall x\in X,\, \exists A\subseteq \tq \overline{A}\cup\lc x\rc = \varnothing \implies x\in B = X\setminus \psi\lp A\rp$.
 En concreto podemos coger $A=\varnothing$.
 \item Sean $B_1,B_2 \in \B$ y sea $x\in X \tq x\in B_1\cap B_2$. Entonces
 \begin{gather*}
  x\in B_1\cap B_2 = \lp X\setminus \psi\lp A_1\rp \rp \cap \lp X\setminus\psi\lp A_2\rp \rp = X\setminus \lp\psi\lp A_1\rp\cup\psi\lp A_2\rp\rp = \\
  = X\setminus\psi\lp A_1 \cup A_2\rp = X\setminus\psi\lp\psi\lp A_1 \cup A_2\rp \rp,
 \end{gather*}
 y como $A=\psi\lp A_1\cup A_2\rp \subseteq X$, $B=X\setminus\psi\lp A\rp \in \B$, tenemos que $\exists B \in \B \tq x\in B \subseteq B_1 \cap B_2$.
\end{enumerate}
Por lo tanto, $\B$ es la base de una topología (de $\T$), lo que nos dice que como $\B$ son los mínimos abiertos que debe contener una topología que cumpla que $\psi\lp A\rp = \overline{A}$ y una base define una única topología, existe una única topología $\T$ que lo cumple. 
\end{eje}


%\chapter{Cónicas y cuádricas}

\section{Definiciones y propiedades básicas}

\begin{defi}
    \begin{enumerate}
        \item[]
        \item Una cuádrica $Q$ de $\Po^n_\k = \Po\lp \E\rp$ es la clase de equivalencia de una forma cuadrática  $q \colon \E \to \k, q\sim q' \iff \exists \lambda\neq0\tq q' = \lambda q$. Notación: $Q = [q]$.
        \item Si $n = 2$, las llamamos cónicas.
        \item $p = [u] \in \Po^n, p\in Q \iff q\lp u\rp=0$ ($p$ es un punto de $Q$).
        \item También notaremos por $Q$, el conjunto de todos los puntos de $Q$:
            \[ Q = \left\{ p=[u]\;|\;q\lp u\rp=0\right\} \]
    \end{enumerate}
\end{defi}
\begin{example}
    \begin{enumerate}
        \item[]
        \item A $\Po^2_\real$, sigui $Q \colon x_0^2 + x_1^2 - x_2^2$ una cònica.
        \begin{itemize}
            \item $p = \left[\lp1,0,1\rp^t\right] \in Q:$ verifica $x_0^2 + x_1^2 -x_2^2=0$.
            \item $p' = \left[\lp0,1,1\rp^t\right] \in Q$.
            \item $p'' = \left[\lp1,1,\sqrt{2}\rp^t\right] \in Q$.
        \end{itemize}
        \item A $\Po^2_\real$, sigui $Q \colon x_0^2 + x_1^2 - x_2^2 \implies$ puntos = $\emptyset$.
        \begin{itemize}
            \item $Q_1 \colon x_0^2 + 2x_1^2 = 0 \implies p = \lp0:0:1\rp$ punto único.
            \item $Q_1 \colon 3x_0^2 + 5x_1^2 = 0 \implies p = \lp0:0:1\rp$ punto único.
        \end{itemize}
    \end{enumerate}
\end{example}
\begin{obs}
    Sean $Q,\, q\colon \E \to \k$ (car$\k \neq 2$) asociada a $\varphi \colon \E \times \E \to \k, Q = [q]$, entonces $q \leftrightarrow \varphi$ de la siguiente forma:
    \begin{align*}
        q\lp u\rp &\rightarrow \varphi\lp u,v\rp = \frac{1}{2}\left[q\lp u+v\rp - q\lp u\rp -q\lp v\rp\right] \\
        q\lp u\rp = \varphi\lp u, u\rp &\leftarrow \varphi\lp u,v\rp
    \end{align*}
\end{obs}
\begin{defi}
    $\Po^n = \Po\lp\E\rp$, sea $q \colon \E \to \k$ asociada a $\varphi \colon \E \times \E \to \k, Q = [q]$. Sean $\R$ una referencia de $\Po^n$, $B$ una base de $\E$ adaptada a $\R$.
    \[ M_\R \lp Q\rp := M_B\lp q\rp \stackrel{\text{por def.}}{=} M_B\lp\varphi\rp \]
    $M_\R\lp Q\rp$ está definida salvo multiplicar por una constante.
\end{defi}
\begin{example}
    En $\Po^2_\real$, sean $\R$ (de $\Po^2$), $B$ (de $\real^3$). Sea $Q\colon x_0^2+ x_1^2+2x_2^2-2x_0x_1+4x_0x_2 + 6x_1x_2 = 0$.
    \begin{gather*}
        M_\R\lp Q\rp = M_B\lp q\rp = A =
        \begin{pmatrix}
            1 & -1 & 2 \\
            -1 & 1 & 3 \\
            2 & 3 & 2 \\
        \end{pmatrix}, \\
        \quad p\in Q \iff
        \begin{pmatrix}
            x_0 & x_1 & x_2 \\
        \end{pmatrix}
        \begin{pmatrix}
            1 & -1 & 2 \\
            -1 & 1 & 3 \\
            2 & 3 & 2 \\
        \end{pmatrix}
        \begin{pmatrix}
            x_0 \\
            x_1 \\
            x_2 \\
        \end{pmatrix}
        = 0.
    \end{gather*}
\end{example}
\begin{obs}
    Si $A = M_\R\lp Q\rp, p_\R = \lp x_0:\dots:x_n\rp = x$, entonces tenemos que $p\in Q \iff x^tAx = 0$.
\end{obs}
\begin{example}
    Sean $Q = [q], \R, A = M_\R\lp Q\rp$. Sea $\R'$ otro sistema de referencia en $\Po^n$ y sea $S_{\R, \R'}$ la matriz de cambio de sistema de referencia de $\R$ a $\R'$. Entonces $A' = M_{\R'}\lp Q\rp = S^tAS$.
\end{example}

% A PARTIR D'AQUI JORDI

\begin{defi}[(intersección de cónicas y variedades lineales)]
    Sea $Q= \left[ q \right] \subseteq \Po ^n$ una cuádrica y sea $V = \left[ F \right] \subseteq \Po ^n$ una variedad lineal. Definimos su intersección como
    \[
        Q_{|V} := \left[q_{|F} \right].
    \]
    Si $q_{|F} \neq 0$, entonces $Q_{|V}$ es una cuádrica en $V$.
\end{defi}

\begin{example}
    En $\Po _{\real} ^3$, sea $q=x_0^2+x_2^2+x_2^2-x_3^2+2x_0x_1-2x_1x_2=0$ y sea $V=\left[ F \right] \colon x_0+x_1=0$ (un plano). Tenemos que $F=\left[ \lp 1, -1, 0, 0 \rp ^t, \lp 0, 0, 1, 0 \rp ^t, \lp 0, 0, 0, 1 \rp ^t \right]$.
    \begin{gather*}
        A=M_B\lp q \rp = \begin{pmatrix}
            1 & 1 & 0 & 0 \\
            1 & 1 & -1 & 0 \\
            0 & -1 & 1 & 0 \\
            0 & 0 & 0 & -1
        \end{pmatrix} \implies \\
        \implies M_B \lp q_{|F} \rp = \begin{pmatrix}
            1 & -1 & 0 & 0 \\
            0 & 0 & 1 & 0 \\
            0 & 0 & 0 & 1
        \end{pmatrix} \begin{pmatrix}
            1 & 1 & 0 & 0 \\
            1 & 1 & -1 & 0 \\
            0 & -1 & 1 & 0 \\
            0 & 0 & 0 & -1
        \end{pmatrix} \begin{pmatrix}
            1 & 0 & 0 \\
            -1 & 0 & 0 \\
            0 & 1 & 0 \\
            0 & 0 & 1
        \end{pmatrix} = \\
        = \begin{pmatrix}
            0 & 1 & 0 \\
            1 & 1 & 0 \\
            0 & 0 & -1
        \end{pmatrix}.
    \end{gather*}
    La matriz $M_B \lp q_{|F} \rp$ es, como se puede observar, la matriz de una cuádrica en $V$.
\end{example}
\begin{prop}
    Sea $Q= \left[ q \right] \subseteq \Po ^n$ una cuádrica y sea $V = \left[ F \right] \subseteq \Po ^n$ una variedad lineal. Se satisface que
    \begin{enumerate}[(1)]
        \item $q_{|F} = 0 \iff V \subseteq Q$.
        \item $q_{|F} \neq 0 \implies Q_{|V} = Q \cap V $.
    \end{enumerate}
\end{prop}

\begin{proof}
    \begin{enumerate}[(1)] \item[]
        \item $q_{|F} = 0 \iff \forall v \in F, \, q \lp v \rp = 0 \iff \forall p = \left[ v \right] \in V, \, p \in Q$.
        \item Sea $p=\left[ v \right]$. Entonces, 
        \[
            p \in Q_{|V} \iff \left\{ \begin{array}{c} q \lp v \rp = 0 \\ v \in F \end{array} \right\} \iff \left\{ \begin{array}{c} p \in Q \\ p \in V \end{array} \right\} \iff p \in Q\cap V.
        \]
    \end{enumerate}    
\end{proof}

\begin{prop}
     Sea $Q=\left[q \right]$ y sea $f \colon \Po^n \to \Po^n$ una homografía.
     \begin{enumerate}[(1)]
        \item $f\lp Q \rp$ es una cuádrica.
        \item Sea $\R$ una referencia en $\Po^n$ y sean $A=M_{\R} \lp Q \rp, \, P=M_{\R} \lp f \rp$. Entonces,
            \[
                M_{\R} \lp f \lp Q \rp \rp = \lp P^{-1} \rp ^t A \lp P^{-1} \rp.
            \]
     \end{enumerate}
\end{prop}

\begin{proof}
    \begin{enumerate}[(1)] \item[]
        \item Sea $p \in f \lp Q \rp$ y sea $y=p_{\R} = \lp y_0 , \dots , y_n \rp ^t$. Por ser $f$ una homografía, es biyectiva y $\lp f ^{-1} \lp p \rp \rp _{\R} = P^{-1} y$.
            \begin{gather*}
                p \in f \lp Q \rp \iff f^{-1} \lp p \rp \in Q \iff \lp P^{-1} y \rp ^t A \lp P^{-1} y \rp = 0 \iff \\
                \iff y^t \lp \lp P ^{-1} \rp ^t A P^{-1} \rp y = 0 ,
            \end{gather*}
            y como que $\lp \lp P ^{-1} \rp ^t A P^{-1} \rp$ es una matriz simétrica, $f \lp Q \rp$ es una cuádrica.
        \item $y^t \lp \lp P ^{-1} \rp ^t A P^{-1} \rp y = 0 \iff p \in f \lp Q \rp \iff y^t M_{\R} \lp f \lp Q \rp \rp y = 0$, de lo que concluimos que $M_{\R} \lp f \lp Q \rp \rp = \lp P^{-1} \rp ^t A \lp P^{-1} \rp$.
    \end{enumerate}
\end{proof}

\begin{defi}[(puntos singulares y lisos)]
    Sea $Q=\left[ q \right] \subseteq \Po ^n$, sea $p = \left[ v \right] \in Q$ y sea $\varphi$ la forma bilineal simétrica asociada a $q$. Entonces, 
    \begin{enumerate}[(1)] 
        \item Decimos que $p$ es un \textit{punto singular} de $Q \iff \varphi \lp v, \cdot \rp = 0$.
        \item Decimos que $p$ es un \textit{punto liso} de $Q$ si no es un \textit{punto singular}.
        \item Decimos que $Q$ es \textit{no degenerada} $\iff Q$ no tiene \textit{puntos singulares}. 
    \end{enumerate}
\end{defi}

\begin{obs}
    Sea $A=M_{\R} \lp Q \rp$, sea $p \in Q$ y sea $p_{\R} = \lp a_0, \dots , a_n \rp ^t$.
    \begin{enumerate}[(1)]
        \item $p\in Q$ es un punto singular $\iff \lp x_0, \dots , x_n \rp A \lp a_0, \dots , a_n \rp ^t = 0, \; \forall \, \lp x_0, \dots , x_n \rp \iff \lp a_0, \dots , a_n \rp \in \nuc A$.
        \item $Q$ es no degenerada $\iff \rg \lp A \rp = n+1 \iff \det A \neq 0$.
    \end{enumerate}
\end{obs}



\section{Tangencia y polaridad}

\begin{example} Cuádricas de $\Po^1$
   \begin{enumerate}
      \item $\Po^1_\real$\\
      Sea $q:$ $ax^{2}_{0} + cx^{2}_{1}+2bx_0x_1=0$ una cuádrica. Tenemos:
      \[
	Q \iff q \stackrel{\R}{\iff}A=
      \begin{pmatrix}
	a & b\\
	c & d\\
      \end{pmatrix}
      \]
      Y por lo tanto podemos considerar dos casos:
      \begin{enumerate}
	\item $a=0$ Tenemos entonces que:
	\[
	  0=cx^{2}_{1}+2bx_0x_1 = x_1 \lp cx_1 +2bx_0 \rp 
	\]
	Lo que nos da dos casos:
	\begin{gather*}
	  x_1=0 \implies p_1=\lp 1,0\rp \text{ y $p_2$ cualquiera}\\
	  \text{o bien}\\
	  p_2=\lp -c,2b\rp \text{ y $p_1$ cualquiera}\\
	\end{gather*}
	Notemos que si $a=0$ y $b=0$ tenemos que $p_1=p_2$.
	
	\item $a\neq 0 \implies \lp 1,0\rp$ no es solución 
	$\implies a\lp\frac{x_0}{x_1}\rp ^2 +2b\lp\frac{x_0}{x_1}\rp +c=0$,
	y si definimos $x=\frac{x_0}{x_1}$ tenemos $ax^2 +2bx+c=0 \implies
	x=\frac{-2b \pm \sqrt{-4\det A}}{2a}$\\
      \end{enumerate}
      Y por lo tanto, si $\det A = 0$, $q$ es un punto doble en $\Po^1_\real$,
      si $\det A>$ no hay ningun punto que pertenezca a la cuádrica y si 
      $\det A <0$ $q$ son dos puntos distintos.
      
      \item $\Po^1_\mathbb{C}$. Usando el mismo analisis que antes llegamos a que si
      $\det A=0$, la cuádrica es un punto doble y si $\det A \neq 0$ la cuádrica 
      son dos puntos distintos.
   \end{enumerate}
\end{example}
\begin{obs}
  Sea $Q\subseteq \Po^n$ una cuádrica y $L\subseteq \Po^n$ una recta y $F$ un subespacio 
  vectorial de dimensión $2$ $\tq [F]=L$. Podemos estudiar las posiciones relativas
  de la cuádrica y la recta:
  \begin{itemize}
   \item $Q_{|L}=\left[ q_{|F} \equiv 0\right] \iff L\subseteq Q$ (ver ejemplo \ref{exampleLinQ})
   \item $L\not\subseteq Q \implies Q_{|L}$ es una cuádrica de $L \implies L\cap 
   Q = \emptyset$, $L\cap Q\equiv$ $1$ punto doble, $L\cap Q\equiv$ 2 puntos distintos.
  \end{itemize}
\end{obs}
\begin{example}
 Sea $Q\subseteq \Po^3 \tq Q:x^2_0 +x^2_1 -x²_2-x^2_3=0$, sean $p_1=\lp 0:1:1:0\rp$ y $p_2=\lp 
 1:0:0:1\rp$ puntos del proyectivo y $L=p_1 \vee p_2$ una recta. Entonces tenemos que:
 \[
  Q_{|L}=
  \begin{pmatrix}
   0 & 1 & 1 & 0\\
   1 & 0 & 0 & 1\\
  \end{pmatrix}
  \begin{pmatrix}
   1 & & &\\
   & 1 & &\\
   & & -1 &\\
   & & & -1\\
  \end{pmatrix}
  \begin{pmatrix}
   1 & 0\\
   0 & 1\\
   0 & 1\\
   1 & 0\\
  \end{pmatrix}
  =\emptyset
 \]
 \label{exampleLinQ}
\end{example}

\begin{defi}
 Sean $L$ una recta y $Q$ una cuádrica:
 \begin{enumerate}[(1)]
  \item Si $L\subseteq Q$ diremos que $L$ es una generatriz de $Q$.
  \item Si $L\cap Q \equiv$ $2$ puntos diremos que $L$ es secante a $Q$.
  \item Si $L\cap Q \equiv$ $1$ punto (doble) diremos que $L$ es tangente a $Q$.
 \end{enumerate}
\end{defi}

\begin{prop}
 Sean $\phi$ una forma bilineal y $q$ y $Q$ una cuádrica $\tq [\phi]=[q]=Q\subseteq \Po^2$.
 Entonces tenemos los siguientes resultados para cuádricas:
 \begin{enumerate}[(1)]
  \item Sea $L=p_1\vee p_2 \tq$ $p_1=[v_1]$ y $p_2=[v_2]$. Entonces:
  \begin{itemize}
   \item $L$ es generatriz de $Q \iff \phi \lp v_1,v_1\rp = \phi
   \lp v_2,v_2 \rp = \phi \lp v_1,v_2 \rp =0$.
   \item $L$ es tangente a $Q \iff \phi \lp v_1,v_1 \rp \phi \lp v_2,v_2 \rp
   - \phi \lp v_1,v_2 \rp ^2 =0$.
  \end{itemize}
  \item Si $p\in Q$ y $p$ singular, entonces toda recta $L \tq p\in L$ es tangente a $Q$.
  \item si $p\in Q$ y $p$ liso, entonces $\{ p'\subseteq \Po^n | L=p\vee p'$ es tangente a $Q \}$
  es un hiperplano. A demás, $T_p\lp Q \rp$ (hiperplano tangente a $Q$ en $p$) tiene de ecuación
  \[
   \lp a_0,\dots,a_n \rp A 
   \begin{pmatrix}
    x_0\\
    \vdots\\
    x_n\\
   \end{pmatrix} = 0
  \]
  Donde $p_\R = \lp a_0:\dots :a_n\rp$, $p'_\R = \lp x_0:\dots :x_n\rp$ y $A = M_\R \lp Q \rp$.
 \end{enumerate}
\end{prop}
\begin{proof}
 \begin{enumerate}[(1)]
  \item[]
  \item Sea $F$ un subespacio vectorial $\tq L=[F]=[v_1,v_2]$. Entonces
  \[
   M_\R \lp Q_{|L} \rp = M_\R \lp q_{|F}\rp =
   \begin{pmatrix}
    \phi\lp v_1,v_1\rp & \phi\lp v_1,v_2\rp\\
    \phi\lp v_2,v_1\rp & \phi\lp v_2,v_2\rp\\
   \end{pmatrix} =B
  \]
  Y por lo tanto:
  \begin{itemize}
   \item $L$ generatriz $\stackrel{\text{por definición}}{\iff} q_{|F}\equiv 0
   \iff \phi\lp v_1,v_1\rp = \phi\lp v_1,v_2\rp = \phi\lp v_2,v_2\rp=0$.
   \item $L$ es tangente a $Q \iff \left\{
   \begin{array}{c}
    L \text{ generatriz }  \iff B=0 \\
    L\cap Q \equiv 1 \text{ punto doble }  \iff \det B=0\\
   \end{array}\right\} \iff \phi\lp v_1,v_1\rp \phi\lp v_2,v_2\rp - \phi\lp v_1,v_2\rp ^2 =0$
  \end{itemize}
  \item Recordemos que $p=[v]$ singular $\iff \phi\lp v,·\rp=0$. Ahora sea $L=p\vee p'$.
  Entonces tenemos:
  \[
   B=M_\R\lp Q_{|F}\rp = 
   \begin{pmatrix}
    \cancelto{0}{\phi\lp v_1,v_1\rp} & \cancelto{0}{\phi\lp v_1,v_2\rp}\\
    \cancelto{0}{\phi\lp v_2,v_1\rp} & \phi\lp v_2,v_2\rp
   \end{pmatrix} \implies \det B=0 \stackrel{(1)}{\implies} L \text{ es tangente a } Q
  \]
  \item 
 \end{enumerate}
\end{proof}




%%%%%%%%%%%%%%%%%%%%%%%%%%%%%%%%%%%%%%%%%%%%%%%%%%%%%%%%%%%

%%%%%%%%%%%%%%%%A PARTIR D'AQUI JORDI%%%%%%%%%%%%%%%%%%%%%%

%%%%%%%%%%%%%%%%%%%%%%%%%%%%%%%%%%%%%%%%%%%%%%%%%%%%%%%%%%%





%%%%%%%%%%%%%%%%%%%%%%%%%%%%%%%%%%%%%%%%%%%%%%%%%%%%%%%%%%%

%%%%%%%%%%%%%%%%A PARTIR D'AQUI OSCAR%%%%%%%%%%%%%%%%%%%%%%

%%%%%%%%%%%%%%%%%%%%%%%%%%%%%%%%%%%%%%%%%%%%%%%%%%%%%%%%%%%

\end{document}
