\u{C. Cauchy (sèries)}: $\sum a_n$ conv. $\iff \forall \varepsilon > 0, \exists n_0 \tq m>n\geq n_0$ llavors, $\abs{s_m-s_n} = \abs{a_{n+1} + \dots + a_m} < \varepsilon$. \\
\ci La convergència és lineal i associativa. \\
\u{S\`eries geom\`etriques}: $\sum_{n \geq 1} \alpha^n$ conv. sii $\alpha \in (-1, 1)$, div. sii $\alpha > 1$, oscil·lant altrament.

\section{Sèries n. positius}
Dins aquest apartat les successions són totes de termes positius.\\
\u{C. comp. dir.}: $a_n \leq b_n \forall n \geq n_0 \implies \sum_{n=n_0}^\infty a_n \leq \sum_{n=n_0}^\infty b_n 
\implies (\sum b_n$ conv. $\implies \sum a_n$ conv.$)$ i $(\sum a_n$ div. $\implies \sum b_n$ div.$)$. \\
\u{C. comp. al límit}: $\lim\frac{a_n}{b_n}=l\in[0,+\infty]$. Si $l < \infty, \, \sum b_n$ conv. $\implies \sum a_n$ conv.. Si $l > 0, \, \sum a_n$ conv. $\implies
\sum b_n$ conv.\\
\u{C. arrel Cauchy}: $(a_n)$ positiva i $\exists \lim a_n^{\nicefrac{1}{n}}=\alpha \implies (\alpha >1$ div.$)$ i $(\alpha < 1$ conv.$)$.\\
\u{C. quo. Alambert}: $(a_n)$ estr. pos. i $\exists\lim\frac{a_{n+1}}{a_n}=\alpha \implies (\alpha <1$ div.$)$ i $(\alpha > 1$ conv.$)$.\\
\u{C. Raabe}: $(a_n)$ estr. pos. i $\exists\lim n(1-\frac{a_{n+1}}{a_n}) = L \implies (L > 1$ conv.$)$ i $(L<1$ div.$)$.\\
\u{S\`erie Rie.}: $\sum_{n \geq 1} \frac{1}{n^p}$ \'es conv. sii $p > 1$, div. altrament.
\section{Altres sèries}
\ci Sèrie cond. conv. $\implies$ podem reordenar per tal que $\sum = s\in[-\infty, +\infty]$.\\
\ci Sèrie alternada $\implies$ un pos., un neg., ... \\
\u{C. Leibniz sèr. alt.}: $(a_n)$ decr. i $\lim a_n = 0$, llavors $\sum (-1)^na_n$ és conv.. A més, $\abs{s-s_N} < a_{n+1}$. \\
\u{C. Dirichlet}: Si $s_n$ d'$(a_n)$ fitades i $(b_n)$ decr., $\lim b_n = 0$, llavors $\sum a_nb_n$ convergeix. \\

\section{Sèries de potències}

\u{Radi de convergència}: Màxim $r \tq \sum a_nr^n$ és conv. \\
\u{Domini de conv.}: $\lp -R, R\rp$, on $R$ és radi de conv.. És possible qe convergeixi als exterms. \\
\u{T. Cauchy-Hadamard} Sigui $\sum a_nx^n, R$ ve donada per $\frac{1}{R} = \lim \sup \abs{a_n}^{\nicefrac{1}{n}}$. La sèrie de potències és abs. conv. si $\abs{x} < R$ i div. si $\abs{x} > R$. Si $\abs{x} = 1$ no sabem res. \\
\u{Càlcul radi de conv.}: $\frac{1}{R} = \lim \abs{a_n}^{\nicefrac{1}{n}}$ o $\frac{1}{R} = \lim\frac{\abs{a_{n+1}}}{\abs{a_n}}$.

\section{Integrals impròpies}

\ci La convergència d'integrals és lineal. \\
\u{C. Cauchy per a int. impròpies}: $f\colon [a, b) \to \real. \int_a^b f$ és conv. $\iff \forall \varepsilon > 0, \exists c_0 \in [a, b) \tq$ si $c_1, c_2 > c_0$, llavors $\abs{\int_{c_1}^{c_2} f} < \varepsilon$. \\
\u{C. comp. dir} $f, g:[a,b) \to \real$, $f, g > 0$, $f \leq g$ localment integrables.
Aleshores $\int_a^b f \leq \int_a^b g$. Si la segona conv., la primera també. Si
la primera div., la segona també.\\
\u{C. comp. al límit} $f, g:[a,b) \to \real$, $f, g > 0$ localment integrables.
Suposem $\exists \lim\limits_{x \to b} \frac{f(x)}{g(x)} = l$. Si $l < \infty, \, \int_a^b g$ conv.
$\implies$ $\int_a^b f$ conv.. Si $l > 0, \, \int_a^b f$ conv. $\implies \int_a^b g$ conv..\\
\u{C. Dirichlet} $f, g:[a,b) \to \real$ localment integrables. Suposem 
$\exists M>0 \tq$ si $a<c<b,\, \abs{\int_a^c f(x) \dif x} \leq M$ i $g$ decreixent
amb $\lim\limits_{x\to b} g=0$. Aleshores $\int_a^b fg$ és conv..\\

\section{Integrals a rectangles}

\u{Suma inf. del rect.}: $m_R = \inf_{x\in\real} f(x), \, s(f; \Pa) = \sum_R m_R \vol(R)$. \\
\u{Suma sup. del rect.}: $M_R = \sup_{x\in\real} f(x), \, S(f; \Pa) = \sum_R M_R \vol(R)$. \\
\u{Si $\Pa'$ més fina que $\Pa$}: $s(f; \Pa) \leq s(f; \Pa') \leq S(f; \Pa') \leq S(f; \Pa)$. \\
\ci $\lowint_A f = \sup_\Pa s(f; \Pa), \upint_A f = \inf_\Pa S(f; \Pa) \to$ si són iguals, $f$ és integrable Riemann.\\
\u{C. Riemann}: $f$ int. Rie. $\iff \forall \varepsilon>0, \exists\Pa \tq S(f;\Pa) - s(f;\Pa) < \varepsilon$.\\
\ci Integrabilitat Riemann és lineal. \\
\u{Suma de Rie.}: Siguin $\xi_k \in R_k,$ la suma és $R\lp f;\Pa; \xi\rp = \sum_k f\lp \xi_k\rp \vol\lp R_k\rp$.

\section{Mesura nul·la}

\u{Mesura nul·la}: Recobert per numerables rectangles de mesura $< \varepsilon, \, \forall \varepsilon > 0$. \\
\u{Contingut nul}: Mesura nul·la amb un nombre finit de rectangles. \\
\ci Si un conjunt té un punt interior, no és de mesura nul·la. \\
\u{Quadrat}: volum: $c^n$, diàmetre: $c\sqrt{n}$ (a $real^n$, on $c$ costat). \\
\ci Sigui $z \subset \real^n$ mesura nul·la. $\forall \varepsilon, \exists$ família numerable de quadrats compactes $Q_k \tq z\in\bigcup_k Q_k, \sum\vol(Q_k) < \varepsilon$. \\
\ci Sigui $f\colon U\to \real$ classe $\mathcal{C}^1$ o lipschitziana, $z \subset U$ mesura nul·la, llavors $f(z) \subset \real^n$ té mesura nul·la.

\section{Teorema de Lebesgue}

\ci Sigui $X$ espai mètric, $f\colon X\to \real$, l'oscil·lació de $f$ sobre $E\subset X$ és el diàmetre de $f(E)$: $\omega(f,E) = \sup_{x,y\in E} \dif ( f( x ), f( y ) ) \in [ 0, +\infty ]$. Finita $\iff f_{|E}$ fitada, $0\iff f_{|E}$ constant. \\
\u{Oscil·lació en $a\in X$}: $\omega( f,a ) = \lim\limits_{r\to 0}\omega( f, B( a;r ) ) = \inf_{r>0} \omega( f, B( a; r ) )$. \\
\ci $f$ cont. en $a \iff \omega( f, a ) = 0$. \\
\u{T. Lebesgue}: Sigui $A\subset \real^n$ rectangle compacte, $f\colon A\to \real$ fitada. Llavors $f$ és integrable Riemann $\iff \disc( f )$ és de mesura nul·la $\iff$ contínua gairebé pertot.

\section{Integral de Rie.}

\ci $C\subset\real^n$ és admissible o mesurable Jordan si és fitat i $\fr( C )$ té mesura nul·la.\\
\ci $\fr( A \cup A' ), \fr( A\cap A' ), \fr( A\setminus A' ) \subseteq \fr( A ) \cup \fr( A' )$. \\
\ci $\fr( A\times B ) = ( \fr( A )\times \overline{B} ) \cup ( \overline{A} \times \fr( B ) )$.\\
\ci $A,A'\subset\real^n$ admissibles $\implies A\cup A', A\cap A', A\setminus A'$ admissibles. \\
\ci $A\subset\real^n, B\subset\real^m$ admissibles $\implies A\times B \subset \real^{n+m}$ admissible. \\
\ci Els rectangles fitats i les boles euclidianes són admissibles. \\
\u{Funció característica de $C\subset X$}: (o indicatriu) $\chi_C\colon X \to \real, \chi_C(x) = 1$ si $x \in C$, val $0$ altrament.\\
\ci $\chi$ no és contínua a $\fr( C ) \implies$ ($C$ adm. $\iff C$ fitat i $\forall R, \exists \int_R \chi_C \tq C\subset R$). \\
\ci $g\colon E\to\real, \tilde{g}\colon X\to\real ( \tilde{g}(x) = 0 \forall x \notin E )$. Aleshores $\disc( g ) \subseteq \disc( \tilde{g} )\subseteq \disc( g )\cup \fr( E )$. \\
\ci $f\chi_C$ integrable Rie. en R $\iff \disc( f )$ de mesura nul·la. \\
\u{Pel T. Lebesgue}: $C\subset \real^n$ admissible. $f\colon C \to \real$ és integrable Rie. $\iff \disc( f )$ mesura nul·la. \\
\ci Si $C$ adm., $\vol( C ) = \int_C 1$ és la mesura (o contingut) de Jordan o volum ($n$-dimensional) de $C$. \\
\ci $C\subset\real^n$ té contingut nul $\iff C$ adm. i $\vol( C ) = 0$.

\section{Propietats de la int. de Rie.}

\ci Sigui $E\subset\real^n$ mesurable Jordan, $\rie( E ) = \{f | f$ int. Rie. en $E \}$ és un $\real$-e.v. i $Rie\colon E \to \real, Rie(f) = \int_E f$ és una forma lineal positiva i monòtona. \\
\u{T. valor mitjà per a integrals}: Sigui $E$ m.J., $f\colon E\to\real$ int. Rie.. $m\leq f\leq M \implies m\vol( E )\leq \int_E f\leq M\vol( E )$. \\
\ci $E$ m.J. connex, $f\colon E\to\real$ fitada i cont., $\exists x_0 \in E \tq \int_E f = f( x_0 )\vol( E )$. \\
\ci $E$ m.J., $f\colon E\to \real$ int. Rie., $h \colon f( E ) \to \real$ cont., $h\circ f$ és int. Rie.. \\
\ci $f, h$ int. Rie. no implica $h\circ f$ int. Rie.. \\
\ci $f$ int. Rie. $\implies \abs{f}$ int. Rie. i $\abs{\int_E f} \leq \int_E \abs{f}$. \\
\ci $f, g$ int. Rie. $\implies f\times g$ int. Rie.. \\
\ci Siguin $A,B\subset\real^n$ m.J. $f\colon A\cup B\to \real$ fitada. $f$ és int. Rie. si ho és a $A$, $B$ i $A\cap B$  i es compleix: $\int_{A\cup B} f= \int_A f + \int_B f - \int_{A\cap B} f$. \\
\ci $E$ m.J., $f\colon E \to \real$ positiva i int. Rie., aleshores $\int_E f=0 \iff f$ nul·la gairebé pertot. \\
\ci Dues funcions int. i iguals gairebé pertot tenen la mateixa intgral (tot i que canviar els valors en un conjunt de mesura nul·la pot destruir l'integrabilitat).

\section{Teorema de Fubini}

\u{T. Fubini}: $A\subset\real^n, B \subset \real^m$ rect. comp., $f\colon A\times B \to \real$ int. Rie.. Sigui $\Phi\colon A\to\real \tq \lowint_B f( x,· ) \leq \Phi( x ) \leq \upint_B f( x, · )$. Aleshores $\Phi$ int. Rie. i $\int_{A\times B} f = \int_A \Phi,\; ( A \leftrightarrow B \text{ també})$. \\
\ci $x \in A \tq f( x, · )$ no int. Rie. té mesura nul·la. \\
\ci $D\subset X, f\colon D\to \real$ cont., $E = \left\{ ( x, y  )\in X\times \real | x\in D, y \geq f( x ) \right\}$, llavors ($D\subset X$ tancat $\implies E\subset X\times\real$ tancat) i ($\fr( E ) \subset \graf( f ) \cup ( \fr( D )\times \real  )$). \\
\ci $D\subset \real^{n-1}$ comp., m.J., $\varphi, \psi\colon D\to \real$ cont. t.q. $\varphi \leq \psi$, $E = \{ ( x,y )\in\real^{n-1}\times \real\;|\; x\in D, \varphi( x ) \leq y \leq \psi( x ) \} \subset \real^n$ és compacte i m.J.. (Si $f\colon E\to \real, \int_E f = \int_D \dif x\int_{\varphi( x )}^{\psi( x )}\dif y\, f( x, y )$). \\
\ci $E$ és una regió elemental de $\real^n$ (per $n=1$ és un interval compacte).

\section{Canvi de variables}

\ci Sigui $V\subset\real^n$ obert, $\varphi\colon V\to\real^n$ injectiva, classe $\mathcal{C}^1$ amb $\det \dif \varphi( y ) \neq 0, \forall y \in V$. Sigui $U = \varphi( V ) (\varphi\colon V\to U$ difeo. classe $\mathcal{C}^1)$. Si $f\colon U\to \real$ int., $\int_U f = \int_V( f \circ \varphi )\abs{\det \dif \varphi}$.

\subsection{Alguns canvis de variables}
\u{Polars a $\real^2$}: $\int_U f( x,y ) \dif x\dif y = \int_V f( r\cos\varphi, r\sin\varphi )r \dif r\dif \varphi$. \\
\u{Cilíndriques a $\real^3$}: $\int_U f( x,y,z ) \dif x\dif y\dif z = \int_V f( \rho\cos\varphi, \rho\sin\varphi, z )\rho \dif \rho\dif \varphi\dif z$. \\
% TODO aqui a baix sobra un $ $
\u{Esfèriques a $\real^3$}: $\int_U f( x,y,z ) \dif x\dif y\dif z = \int_V f ( r\cos\varphi\sin\theta, r\sin\varphi\sin\theta, r\cos\theta )$ $r^2\sin\theta \dif r\dif \varphi\dif \theta$.

\section{Integrals impròpies}

\u{Integral de Riemann impròpia}: Sigui $( E_i )$ exhaustió, \'es $\int_E f := \lim_i \int_{E_i} f$ (suposant que no depengui de l'exhaustió). \\
\ci Si $E$ m.J., $( E_i )$ exh., $\lim_i \vol( E_i ) = \vol ( E )$, i si $f\colon E\to\real$ int. Rie., $f$ \'es int. Rie. a cada $E_i$ i int. Rie. coincideix amb la int. impròpia. \\
\ci Siguin $E\subset\real^n, f\colon E\to \real, f \geq 0$, llavors $\lim_i\int_{E_i} f$ no depèn de l'exh. considerada. \\
\u{Lema ceba}: Sigui $U\subset\real^n$ obert no buit, $\exists (V_i)$ de conj. oberts m.J., $\overline{V}_i \subset U$ t.q. $\overline{V}_i$ \'es compacte, $\overline{V}_i \subset V_{i+1}, \, \bigcup_i V_i = U$.

\section{Camins i long. de corba}

\u{Longitud d'una poligonal}: $L( \gamma, \Pa ) = \sum_{i=0}^N ||\gamma( t_i ) - \gamma( t_{i-1} )||$. \\
\u{Longitud d'una corba}: $L( \gamma ) = \sup_\Pa L( \gamma, \Pa ) \in [0, +\infty]$. \\
\u{Camí rectificable}: Si longitud \'es finita. \\
\u{Additivitat camí}: $a<c<b \implies L( \gamma ) = L( \gamma_{|[ a,c ]} ) + L( \gamma_{|[ c, b ]} )$. \\
\ci Sigui $\gamma\colon [ a,b ]\to\real, l(a) = 0$ i $l(t) = L( \gamma_{|[ a,t ]} ), a < t \leq b,\, l :[ a,b ]\to\real$ \'es creixent i continua. \\
\ci Si $\vec{f}\colon E\to\real^n$ int., $\|\vec{f}\|$ també (norma euclidiana). \\
\ci $\gamma \colon I\to \real^n$ classe $\mathcal{C}^1 \implies$ rectificable i $L(\gamma) = \int_I \|\gamma'\|$.

\section{Integrals de l\'inia}

\u{Int. de l\'inia f. escalars}: $\int_C f \dif l = \int_\sigma f \dif l := \int_I f(\sigma(s)) \|\sigma'(s)\| \dif s$. \\
\u{Int. de l\'inia c. vectorials}: (o circulaci\'o)$\int_\sigma \vec{f} \dif\vec{l} = \int_I \vec{f} (\sigma(s)) \cdot \sigma'(s) \dif s$. \\
\u{$\tau$ i $\sigma$ equivalents}: $\int_\tau \vec{f} \dif \vec{l} = \pm \int_\sigma \vec{f} \dif \vec{l}$, signe \'es el de $\varphi'$. \\
\u{Vector tangent a parametritzaci\'o}: $\vec{t}(\sigma(s)) = \frac{\sigma'(s)}{\|\sigma'(s)\|}$. \\
\u{Component tangencial}: $f_t = \vec{f}\cdot\vec{t}$. \\
\ci $\int_C \vec{f} \dif \vec{l} = \int_C f_t \dif l$.

\section{Integrals de superf\'icie}

\u{Vectors tangents}: $\sigma\colon U\subset\real^2 \to \real^n,\, \vec{J}_\sigma = (\vec{T}_1 \,\,\, \vec{T}_2)$. \\
\u{Int. de superf\'icie f. escalars}: $\sigma\colon U\subset\real^2\to\real^3,\, \int_\sigma f \dif S= \int_U f(\sigma(\omega)) \|\vec{T}_1 \times \vec{T}_2\| \dif\omega_1 \dif\omega_2$. \\
\u{Int. de superf\'icie c. vectorials}: (o flux a trav\'es de $\sigma$) $\sigma\colon U\subset\real^2\to\real^3$, $\int_\sigma \vec{f}\dif \vec{S} := \int_U \vec{f} (\sigma(u))\cdot(\vec{T}_1 \times \vec{T}_2) \dif u_1 \dif u_2$. \\
\u{$\sigma$ difeo.}: $\int_{\tilde{\sigma}} \vec{f} \dif \vec{S} = \pm \int_\sigma \vec{f} \dif \vec{S}$, signe de $\det J_\sigma$. \\
\u{Vector normal a sup.}: $\vec{n}(\sigma(u)) = \frac{\vec{T}_1\times\vec{T}_2}{\|\vec{T}_1\times \vec{T}_2\|}$. \\
\u{Component normal}: $f_n = \vec{f}\cdot\vec{n}$. \\
\ci $\int_M \vec{f} \dif\vec{S} = \int_M f_n \dif S$.

\section{Operadors dif. a $\real^3$}

\u{Gradient}: $\grad f := \pdv{f}{x} \hat{i} + \pdv{f}{y} \hat{j} + \pdv{f}{z} \hat{k}$. \\
\u{Rotacional}: $\rot \vec{F} := ( \pdv{F_3}{y} - \pdv{F_2}{z} )\hat{i} + ( \pdv{F_1}{z} - \pdv{F_3}{x} ) \hat{j} + ( \pdv{F_2}{x} - \pdv{F_1}{y} ) \hat{k}$. \\
\u{Diverg\`encia}: $\diver \vec{F} := \pdv{F_1}{x} + \pdv{F_2}{y} + \pdv{F_3}{z}$. \\
\u{Regles de Leibniz}: \\
\quad \ci $\grad(fg) = f \grad g + g \grad f$, \\
\quad \ci $\rot ( f \vec{G} ) = f \rot \vec{G} + \vec{\grad} f \times \vec{G}$, \\
\quad \ci $\diver ( f \vec{G} ) = f \diver \vec{G} + \vec{\grad} f \cdot \vec{G}$, \\
\quad \ci $\diver ( \vec{F} \times \vec{G} ) = \vec{G} \cdot \rot \vec{F} - \vec{F} \cdot \rot \vec{G}$. \\
\u{T. Schwarz}: $f, \vec{F} \in \mathcal{C}^2 \implies \rot( \grad f ) = 0, \diver ( \rot \vec{F} ) = 0$. \\
\u{Camp conservatiu}: $\vec{F} = \grad f$. \\
\u{Camp irrotacional}: $\rot \vec{F} = 0$. \\
\u{Camp solenoidal}: $\vec{G} = \rot \vec{F}$. \\
\u{Camp sense div.}: $\diver \vec{G} = 0$. \\
\u{Laplaci\`a}: $\Delta f := \diver ( \grad f ), \Delta = \nabla^2 = \pdv[2]{}{x} + \pdv[2]{}{y} + \pdv[2]{}{z}$.

\section{F\'ormules int. camps}

\u{T. fon. c\`alcul}: (o gradient) $W \subseteq \real^n$ obert, $f \colon W \to \real$ classe $\C^1$. $C$ corba regular orientada clase $\C^1, \tq \bar{C} \subset W$ compacte $\implies \int_C \vec{\grad} f \dif \vec{l} = \int_{\partial C} f = f( x_1 ) - f( x_0 ), \partial C = \setb{x_0, x_1}$ (orientada $x_0\to x_1$). $\partial C = \emptyset\implies \int_{\partial C} f = 0$. \\
\u{T. Kelvin-Stokes}: (o rotacional) $W \subseteq \real^3$ obert, $\vec{F} \colon W \to \real^3$ classe $\C^1$, $M \subset W$ superficie orientada classe $\C^2 \tq \bar{M}$ compacto, $\bar{M} \subset W$, $\partial M$ $^\ast$ $\implies \int_M \rot \vec{F} \dif \vec{S} = \int_{\partial M} \vec{F} \dif \vec{l}$. \\
\u{T. Gauss-Ostrogradski}: (o diverg\`encia) $W \subset \real^3$ obert, $\vec{F} \colon W \to \real^3$ classe $\C^1$, $B \subset W$ obert t.q. $\bar{B}$ compacte, $\bar{B} \subset W$, $\partial B$ $^\ast$ $\implies \int_B \diver \vec{F} \dif V = \int_{\partial B} \vec{F} \dif \vec{S}$.

\section{Potencials}

\u{Potencial escalar}: $f$ \'es el pot. esc. de $\vec{F} \iff \grad f = \vec{f}$. \\
\u{Propietats}: $U \subseteq \real^3$ obert connex, $\vec{F} \colon U \to \real^3$ classe $\C^1$, s\'on equivalents: \\
\quad \ci $\vec{F}$ conservatiu, \\
\quad \ci $p_0, p_1 \in U$, $C \subset U$ corba orientada t.q. $\partial U = \setb{p_0, p_1}$, circulación $\int_C \vec{F} \dif \vec{l} = f(p_0, p_1)$, \\
\quad \ci $\forall C \subset U$ corba tancada, $\oint_C \vec{F} \dif \vec{l} = 0$. \\
\ci $U \subset \real^3$ obert simplement connex (def. apunts profe) i $\vec{F} \colon U \to \real^3$ classe $\C^1$ irrotacional $\implies \vec{F}$ conservatiu. \\

\u{T. Green}: $U \subseteq \real^2$ obert, $\vec{F} \colon U \to \real^2$ classe $\C^1$, $M \subset U$ obert t.q. $\bar{M}\subset U$ compacte, $\partial M$ $^\ast$ $\implies \int_M ( \pdv{F_2}{x} - \pdv{F_1}{y} ) \dif x \dif y = \int_{\partial M} \vec{F} \dif \vec{l}$.

%%%%%%%%%%%%%%
%   TEMA 5   %
%%%%%%%%%%%%%%

\section{Tema 5}

\ci $(f \dif x^I) \wedge (g \dif x^J) = fg \dif x^I \wedge \dif x^J$. \\
\ci $\alpha \wedge \beta = (-1)^{\abs{\alpha}\abs{\beta}} \beta \wedge \alpha$. \\
\ci $f \in \C^\infty ( \real^n ) = \Omega^0 ( \real^n )$, podem construir $\dif f := \sum^n_{i=1} \frac{\partial f}{\partial x^i} \dif x^i \in \Omega^1 ( \real^n )$. \\
\u{Diferencial exterior}: Aplicaci\'o lineal, $\dif\,(f \dif x^I) = \dif f \wedge \dif x^I$. \\
\ci $\dif \left( \alpha \wedge \beta \right) = \left( \dif \alpha \right) \wedge \beta + (-1)^{\abs{\alpha}} \alpha \wedge \dif \beta$. \\
\ci $\dif\,\circ \dif = 0$. \\
\u{Tancada}: $\alpha$ tancada $\iff \dif \alpha = 0$. \\
\u{Exacta}: $\beta$ exacta $\iff \exists \alpha \tq \beta = \dif \alpha$. \\
\ci Exacta $\implies$ tancada. \\
\u{Lema Poincar\'e}: En $\real^n$, $\alpha$ tancada i $|\alpha| \geq 1 \implies \alpha$ exacta. \\

\u{Pullback}: $F \colon \real^m \to \real^n$ classe $\C^\infty$, $g \in \C^\infty( \real^n )$, el pullback \'es $F^\ast(g) := g \circ F \in \C^\infty \colon \real^m \to \real$, $F^\ast$ \'es $\real$-lineal, $F^\ast(g\dif y^{i_1}\wedge\dots\wedge\dif y^{i_k}) := F^\ast (g) \dif F^\ast (y^{i_1}) \wedge \dots \wedge \dif F^\ast (y^{i_k})$. \\
\ci $F^\ast (\alpha \wedge \beta) = F^\ast (\alpha) \wedge F^\ast (\beta)$. \\
\ci $F^\ast (\dif x) = \dif F^\ast(x)$. \\
\u{Integral de $\omega$ al llarg de $\sigma$}: $\omega \in \Omega^k (\real^n), \sigma \colon \real^k \to \real^n, \implies \int_\sigma \omega := \int_{\real^k} \sigma^\ast (\omega)$. \\
\u{T. Stokes}: $M$ varietat amb vora, orientada i $\dim M = m$, $\omega \in \Omega^{m-1} (M)$ de suport compacte i $\partial M$ t\'e orientació induïda $\implies \int_M \dif \omega = \int_{\partial M} \omega$.


% TODO vigilar amb aquesta cutrada
\setcounter{section}{20}

\subsection{Altres. Sèries}
\ci $\sum r^n$ conv. $\iff \abs{r} < 1$ (\verb|else| div.). \\
\ci $\sum \frac{1}{n^p}$ conv. $\iff p > 1$ (\verb|else| div.).

\subsection{Altres. Integrals}
\ci $\int_1^{+\infty} \frac{1}{x^\alpha}$ d$x$ conv. $\iff \alpha < 1$ i és $\frac{1}{\alpha-1}$. \\
\ci $\int_0^1 \frac{1}{x^\alpha}$ d$x$ conv. $\iff \alpha < 1$ i és $\frac{1}{1-\alpha}$. \\
\ci $\int_0^{+\infty} e^{-\alpha t}$ d$t$ conv. $\iff \alpha > 0$ i és $\frac{1}{\alpha}$.

\subsection{Altres. Taylor}
\ci $e^x = \sum_{n\geq 0} \frac{x^n}{n!}$. \\
\ci $\cos x = \sum_{n\geq 0} (-1)^n \frac{x^{2n}}{(2n)!}$. \\
\ci $\sin x = \sum_{n\geq 0} (-1)^n \frac{x^{2n+1}}{(2n+1)!}$. \\
\ci $\log (1+x) = \sum_{n\geq 1} (-1)^n+1 \frac{x^n}{n}$. \\
\ci $(1+x)^p = \sum_{n\geq 0} \binom{p}{n} x^n$. \\
\ci $(1+x)^{-1} = \sum_{n\geq 0} (-1)^n x^n$. \\

\subsection{Altres. Trigonometria}
\ci $\sin(a \pm b) = \sin(a)\cos(b) \pm \cos(a)\sin(b)$. \\
\ci $\cos(a \pm b) = \cos(a)\cos(b) \mp \sin(a)\sin(b)$. \\
\ci $\sin(a) + \sin(b) = 2\sin(\frac{a+b}{2})\cos(\frac{a-b}{2})$. \\
\ci $\cos(a) + \cos(b) = 2\cos(\frac{a+b}{2})\cos(\frac{a-b}{2})$.

\noindent\makebox[\linewidth]{\rule{\linewidth}{1pt}}
$^\ast$ amb orientació induïda (m\`a dreta) i punts frontera de $M$ regulars o conjunt de punts frontera singulars \'es finit. \\
\vspace{3pt}
\raggedleft
{\large Nom: \underline{\hspace{4.5cm}}}
