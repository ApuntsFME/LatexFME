\u{C. Cauchy (sèries)}: $\sum a_n$ conv. $\iff \forall \varepsilon > 0, \exists n_0 tq m>n\geq n_0$ llavors, $\abs{s_m-s_n} = \abs{a_{n+1} + \dots + a_m} < \varepsilon$. \\
\ci La convergència és lineal i associativa.

\section{Sèries n. positius}

\u{C. comp. dir.}: $\sum_{n=n_0}^\infty a_n \leq \sum_{n=n_0}^\infty b_n \implies (b_n$ conv. $\implies a_n$ conv$)$ i $(a_n$ div. $\implies b_n$ div.$)$. \\
\u{C. comp. al límit}: $\lim\frac{a_n}{b_n}=l\in[0,+\infty]$ (cons. apunts d'ell).\\
\u{C. arrel Cauchy}: $\sum a_n$ positiva i $\exists \lim a_n^{\nicefrac{1}{n}}=\alpha \implies (\alpha >1$ div.$)$ i $(\alpha < 1$ conv.$)$.\\
\u{C. quo. Alambert}: $\sum a_n$ estr. pos. i $\exists\lim\frac{a_{n+1}}{a_n}=\alpha \implies (\alpha <1$ div.$)$ i $(\alpha > 1$ conv.$)$.\\
\u{C. Raabe}: $\sum a_n$ estr. pos. i $\exists\lim n\left(1-\frac{a_{n+1}}{a_n}\right) = L \implies (L > 1$ conv.$)$ i $(L<1$ div.$)$.\\
\ci Sèrie cond. conv. $\implies$ podem reordenar per tal que $\sum = s\in\left[-\infty, +\infty\right]$.\\
\ci Sèrie alternada $\implies$ un pos., un neg., ... \\
\u{C. Leibniz sèr. alt.}: $(a_n)$ decr. i $\lim a_n = 0$, llavors $\sum (-1)^na_n$ és conv.. A més, $\abs{s-s_N} < a_{n+1}$. \\
\u{C. Dirichlet}: Si $s_N$ fitades i $\sum b_n$ decr., $\lim b_n = 0$, llavors $\sum a_nb_n$ convergeix.

\section{Sèries de potències}

\u{Radi de convergència}: Màxim $r \tq \sum a_nr^n$ és conv. \\
\u{Domini de conv.}: $\lp -R, R\rp$, on $R$ és radi de conv.. És possible qe convergeixi als exterms. \\
\u{T. Cauchy-Hadamard} Sigui $\sum a_nx^n, R$ ve donada per $\frac{1}{R} = \lim \sup \abs{a_n}^{\nicefrac{1}{n}}$. La sèrie de potències és abs. conv. si $\abs{x} < R$ i div. si $\abs{x} > R$. Si $\abs{x} = 1$ no sabem res. \\
\u{Càlcul radi de conv.}: $\frac{1}{R} = \lim \abs{a_n}^{\nicefrac{1}{n}}$ o $\frac{1}{R} = \lim\frac{\abs{a_{n+1}}}{\abs{a_n}}$.

\section{Integrals impròpies}

\ci La convergència d'integrals és lineal. \\
\u{C. Cauchy per a int. impròpies}: $f\colon \left[a, b\right) \to \real. \int_a^b f$ és conv. $\iff \forall \varepsilon > 0, \exists c_0 \in \left[a, b\right) \tq$ si $c_1, c_2 > c_0$, llavors $\abs{\int_{c_1}^{c_2} f} < \varepsilon$. \\
\ci Podem aplicar els criteris: comp. directa, comp en el límit, Dirichlet ($\exists M>0 \tq$ si $a<c<b,\, \abs{\int_a^c f(x) \text{d}x} \leq M$.

\section{Integrals a rectangles}

\u{Suma inf. del rect.}: $m_R = \inf_{x\in\real} f(x), \, s(f; \Pa) = \sum_R m_R \vol(R)$. \\
\u{Suma sup. del rect.}: $M_R = \sup_{x\in\real} f(x), \, S(f; \Pa) = \sum_R M_R \vol(R)$. \\
\u{Si $\Pa'$ més fina que $\Pa$}: $s(f; \Pa) \leq s(f; \Pa') \leq S(f; \Pa') \leq S(f; \Pa)$. \\
\ci $\lowint_A f = \sup_\Pa s(f; \Pa), \upint_A f = \inf_\Pa S(f; \Pa) \to$ si són iguals, $f$ és integrable Riemann.\\
\u{C. Riemann}: $f$ int. Rie. $\iff \forall \varepsilon>0, \exists\Pa \tq S(f;\Pa) - s(f;\Pa) < \varepsilon$.\\
\ci Integrabilitat Riemann és lineal. \\
\u{Suma de Rie.}: Siguin $\xi_k \in R_k,$ la suma és $R\lp f;\Pa; \xi\rp = \sum_k f\lp \xi_k\rp \vol\lp R_k\rp$.

\section{Mesura nul·la}

\u{Mesura nul·la}: Recobert per numerables rectangles de mesura $< \varepsilon, \, \forall \varepsilon > 0$. \\
\u{Contingut nul}: Mesura nul·la amb un nombre finit de rectangles. \\
\ci Si un conjunt té un punt interior, no és de mesura nul·la. \\
\u{Quadrat}: volum: $c^n$, diàmetre: $c\sqrt{n}$ (a $real^n$, on $c$ costat). \\
\ci Sigui $z \subset \real^n$ mesura nul·la. $\forall \varepsilon, \exists$ família numerable de quadrats compactes $Q_k \tq z\in\bigcup_k Q_k, \sum\vol(Q_k) < \varepsilon$. \\
\ci Sigui $f\colon U\to \real$ classe $\mathcal{C}^1$ o lipschitziana, $z \subset U$ mesura nul·la, llavors $f(z) \subset \real^n$ té mesura nul·la.

\section{Teorema de Lebesgue}

\ci Sigui $X$ espai mètric, $f\colon X\to \real$, l'oscil·lació de $f$ sobre $E\subset X$ és el diàmetre de $f(E)$: $\omega(f,E) = \sup_{x,y\in E} \text{d}\left( f\left( x \right), f\left( y \right) \right) \in \left[ 0, +\infty \right]$. Finita $\iff f_{|E}$ fitada, $0\iff f_{|E}$ constant. \\
\u{Oscil·lació en $a\in X$}: $\omega\left( f,a \right) = \lim_{r\to 0}\omega\left( f, B\left( a;r \right) \right) = \inf_{r>0} \omega\left( f, B\left( a; r \right) \right)$. \\
\ci $f$ cont. en $a \iff \omega\left( f, a \right) = 0$. \\
\u{T. Lebesgue}: Sigui $A\subset \real^n$ rectangle compacte, $f\colon A\to \real$ fitada. Llavors $f$ és integrable Riemann $\iff \disc\left( f \right)$ és de mesura nul·la $\iff$ contínua gairebé pertot. \\

\section{Integral de Rie. sobre conjunts ms generals}

\ci $C\subset\real^n$ és admissible o mesurable Jordan si és fitat i $\fr\left( C \right)$ té mesura nul·la.\\
\ci $\fr\left( A \cup A' \right), \fr\left( A\cap A' \right), \fr\left( A\setminus A' \right) \subseteq \fr\left( A \right) \cup \fr\left( A' \right)$. \\
\ci $\fr\left( A\times B \right) = \left( \fr\left( A \right)\times \overline{B} \right) \cup \left( \overline{A} \times \fr\left( B \right) \right)$.\\
\ci $A,A'\subset\real^n$ admissibles $\implies A\cup A', A\cap A', A\setminus A'$ admissibles. \\
\ci $A\subset\real^n, B\subset\real^m$ admissibles $\implies A\times B \subset \real^{n+m}$ admissible. \\
\ci Els rectangles fitats i les boles euclidianes són admissibles. \\
\u{Funció característica de $C\subset X$}: (o indicatriu) $\chi_C\colon X \to \real, \chi_C(x) = 1$ si $x \in C$, val $0$ altrament.\\
\ci $\chi$ no AUXs contínua a $\fr\left( C \right) \implies$ ($C$ adm. $\iff C$ fitat i $\forall R, \exists \int_R \chi_C \tq C\subset R$). \\
\ci $g\colon E\to\real, \tilde{g}\colon X\to\real \left( \tilde{g}(x) = 0 \forall x \notin E \right)$. Aleshores $\disc\left( g \right) \subseteq \disc\left( \tilde{g} \right)\subseteq \disc\left( g \right)\cup \fr\left( E \right)$. \\
\ci $f\chi_C$ integrable Rie. en R $\iff \disc\left( f \right)$ de mesura nul·la. \\
\u{Pel T. Lebesgue}: $C\subset \real^n$ admissible. $f\colon C \to \real$ AUXs integrable Rie. $\iff \disc\left( f \right)$ mesura nul·la. \\
\ci Si $C$ adm., $\vol\left( C \right) = \int_C 1$ AUXs la mesura (o contingut) de Jordan o volum ($n$-dimensional) de $C$. \\
\ci $C\subset\real^n$ tAUX contingut nul $\iff C$ adm. i $\vol\left( C \right) = 0$.

\section{Propietats de la integral de Riemann}

\ci Sigui $E\subset\real^n$ mesurable Jordan, $\rie\left( E \right) = \{f | f$ int. Rie. en $E \}$ AUXs un $\real$-e.v. i $Rie\colon E \to \real, Rie(f) = \int_E f$ AUXs una forma lineal positiva i monòtona. \\
\u{T. valor mitjà per a integrals}: Sigui $E$ m.J., $f\colon E\to\real$ int. Rie.. $m\leq f\leq M \implies m\vol\left( E \right)\leq \int_E f\leq M\vol\left( E \right)$. \\
\ci $E$ m.J. connex, $f\colon E\to\real$ fitada i cont., $\exists x_0 \in E \tq \int_E f = f\left( x_0 \right)\vol\left( E \right)$. \\
\ci $E$ m.J., $f\colon E\to \real$ int. Rie., $h \colon f\left( E \right) \to \real$ cont., $h\circ f$ AUXs int. Rie.. \\
\ci $f, h$ int. Rie. no implica $h\circ f$ int. Rie.. \\
\ci $f$ int. Rie. $\implies \abs{f}$ int. Rie. i $\abs{\int_E f} \leq \int_E \abs{f}$. \\
\ci $f, g$ int. Rie. $\implies f\times g$ int. Rie.. \\
\ci Siguin $A,B\subset\real^n$ m.J. $f\colon A\cup B\to \real$ fitada. $f$ AUXs int. Rie. si ho AUXs a $A$, $B$ i $A\cap B$  i es compleix: $\int_{A\cup B} f= \int_A f + \int_B f - \int_{A\cap B} f$. \\
\ci $E$ m.J., $f\colon E \to \real$ positiva i int. Rie., aleshores $\int_E f=0 \iff f$ nul·la gairebAUX pertot. \\
\ci Dues funcions int. i iguals gairebAUX pertot tenen la mateixa intgral (tot i que canviar els valors en un conjunt de mesura nul·la pot destruir l'integrabilitat).

\section{Teorema de Fubini}

\u{T. Fubini}: $A\subset\real^n, B \subset \real^m$ rect. comp., $f\colon A\times B \to \real$ int. Rie.. Sigui $\:w
Phi\colon A\to\real \tq \lowint_B f\left( x,· \right) \leq \Phi\left( x \right) \leq \upint_B f\left( x, · \right)$. Aleshores $\Phi$ int. Rie. i $\int_{A\times B} f = \int_A \Phi,\; \left( A \leftrightarrow B \text{ tambAUX}\right)$. \\
\ci $x \in A \tq f\left( x, · \right)$ no int. Rie. tAUX mesura nul·la. \\
\ci $D\subset X, f\colon D\to \real$ cont., $E = \left\{ \left( x, y  \right)\in X\times \real | x\in D, y \geq f\left( x \right) \right\}$, llavors ($D\subset X$ tancat $\implies E\subset X\times\real$ tancat) i ($\fr\left( E \right) \subset \graf\left( f \right) \cup \left( \fr\left( D \right)\times \real  \right)$). \\
\ci $D\subset \real^{n-1}$ comp., m.J., $\varphi, \psi\colon D\to \real$ cont. t.q. $\varphi \leq \psi$, $E = \left\{ \left( x,y \right)\in\real^{n-1}\times \real\;|\; x\in D, \varphi\left( x \right) \leq y \leq \psi\left( x \right) \right\} \subset \real^n$ AUXs compacte i m.J.. (Si $f\colon E\to \real, \int_E f = \int_D \text{d}x\int_{\varphi\left( x \right)}^{\psi\left( x \right)}\text{d}y\, f\left( x, y \right)$). \\
\ci $E$ AUXs una regió elemental de $\real^n$ (per $n=1$ AUXs un interval compacte).

\section{Canvi de variables}

% \ci Sigui $V\subset\real^n$ obert, $\varphi\colon V\to\real^n$ injectiva, classe 
