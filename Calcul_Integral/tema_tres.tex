\chapter{INTEGRALES DE LÍNEA Y DE SUPERFICIE}

\section{Longitud de una curva}

\begin{defi}
    Denominamos camino o curva parametrizada a una aplicación continua $\gamma \colon I \to \real^n$, donde
    $I \subset \real$ es un intervalo no degenerado.

    La imagen $C := \gamma(I) \equiv \im(\gamma) \subset \real^n$ la llamamos (vulgarmente) curva, aunque no
    siempre lo es.
\end{defi}

\begin{obs*}
    De aquí e adelante, a no ser que se especifique lo contrario, supondremos que $I$ es un intervalo compato
\end{obs*}

\begin{defi}
    Sea $\gamma \colon [a, b] \to \real$ y  sea $\Pa = \setb{t_0, \dots, t_n}$ una partición de $[a, b]$.
    Entonces, los $\gamma(t_i)$ son los v\'ertices de una poligonal, que denominamos poligonal de longitud y cuya logitud es
    \[
        L(\gamma, \Pa) = \sum^N_{i=1} \norm{\gamma(t_i) - \gamma(t_{i-1})}
    \]
\end{defi}
\begin{obs*}
    Refinar la partición aumenta la logitud. (Demostración por la desigualdad triangular)
\end{obs*}

\begin{defi}
    Sea $\gamma \colon I \to \real^n$ una curva. La longitud de $\gamma$ es
    \[
        L(\gamma) = \sup_{\Pa \text{ partición de } I} L(\gamma, \Pa) \in [0, \infty]
    \]
\end{defi}

\begin{obs*}
    Si $L(\gamma) = 0$ entonces $\gamma$ es constante.
\end{obs*}

\begin{defi}
    Diremos que un camino es rectificable, cuando su logitud es finita, es decir, cuando
    \[
        L(\gamma) < +\infty
    \]
\end{defi}

\begin{prop}
    Sea $\gamma \colon I \to \real^n$ una curva lipschiciana, entonces $\gamma$ es rectificable.
\end{prop}

\begin{proof}
    Como $\gamma$ es lipschiciana, entonces $\exists c \geq 0$ tal que
    \[
        \norm{\gamma(t) - \gamma(s)} \leq c \abs{t - s} \qquad \forall t,s \in I
    \]
    Tomamos ahora, una partición $\Pa$ de $I$, entonces, se cumple que
    \begin{gather*}
        L(\gamma, \Pa) = \sum_i \norm{\gamma(t_i) - \gamma(t_{i-1})} \leq \sum_i c \abs{t_i - t_{i-1}} = \\
        = c \sum_i \left( t_i - t_{i-1} \right) \implies L(\gamma, \Pa) \leq c (b - a) \implies
        L(\gamma) \leq c(b-a) < +\infty
    \end{gather*}
    (Suponiendo que $I = [a, b]$).
\end{proof}

\begin{obs*}
    Hay caminos (contiuos) definidos en un intervalo compacto con longitud infinita.
\end{obs*}

\begin{prop}
    Sea $\gamma \colon [a, b] \to \real^n$ un camino y sea $a < c < b$, entonces
    \[
        L(\gamma) = L\left( \gamma\vert_{[a,c]} \right) + L\left( \gamma\vert_{[c,b]} \right)
    \]
\end{prop}

\begin{proof}
    Consideramos $\Pa$, una partición cualquiera de $[a,b]$, ahora definimos $\Pa^\prime = \Pa \cup \{c\}$,
    $\Pa_1 = \Pa^\prime \cap [a,c]$ partición de $[a,c]$ y $\Pa_2 = \Pa^\prime \cap [c, b]$ partición de $[c, b]$.
    Ahora, se tiene que
    \[
        L(\gamma, \Pa) \leq L(\gamma, \Pa^\prime) = \L\left( \gamma\vert_{[a,c]}, \Pa_1 \right) +
        L\left( \gamma\vert_{[c,b]}, \Pa_2 \right) \leq L\left( \gamma\vert_{[a,c]} \right) +
        L\left( \gamma\vert_{[c,b]} \right)
    \]
    Por lo tanto, $L(\gamma) \leq L\left( \gamma\vert_{[a,c]} \right) + L \left( \gamma\vert_{[c,b]} \right)$.

    Sea $\Pa_1$ una partició de $[a, c]$ y $\Pa_2$ una partición de $[c, b]$. Definimos $\Pa = \Pa_1 \cup \Pa_2$,
    entonces
    \[
        L\left( \gamma\vert_{[a,c]}, \Pa_1 \right) + L\left( \gamma\vert_{[c,b]}, \Pa_2 \right) = L(\gamma, \Pa) \leq L(\gamma)
    \]
    De dode concluimmos que
    \[
        L(\gamma) = L\left( \gamma\vert_{[a,c]} \right) + L\left( \gamma\vert_{[c,b]} \right)
    \]
\end{proof}

\begin{prop}[continuidad de la longitud]
    Sea $\gamma \colon [a,b] \to \real^n$ un camino rectificable, para cada $a \leq t \leq b$ definimos
    \[
        l(t) = L\left( \gamma\vert_{[a,c]} \right) \qquad (l(a) := 0)
    \]
    Entonces, $l$ es creciente y continua. Si $\gamma$ no es constante en ningún subintervalo de $[a,b]$,
    entonces, $l$ es estrictamente creciente
\end{prop}

\begin{defi}
    Sean $\sigma \colon [a, b] \to \real^n$ un camino y $\varphi \colon [c, d] \to [a, b]$ un homeomorfismo. Entonces,
    la composición $\tau = \sigma \circ \varphi \colon [c,d] \to \real^n$ es un camino, y diremos que $\sigma$ y $\tau$
    son caminos equivalentes.

    Tambi\'en diremos que $\tau$ es el camino parametrizado de $\sigma$ para la reparametrización $\varphi$.
\end{defi}

\begin{obs*}
    $\sigma$ y $\tau$ recorren la misma curva, $\im(\sigma) = \im(\tau) = \im(\sigma \circ \varphi)$
\end{obs*}

\begin{obs}
    Si $\varphi$ es creciente, $\sigma$ y $\tau$ tienen la misma orientación, si $\varphi$ es decreciente,
    $\sigma$ y $\tau$ tienen orientacions opuestas.
\end{obs}

\begin{prop}
    Si $\sigma$ y $\tau$ son dos caminos equivalentes, tienen la misma longitud
    \[
        L(\sigma) = L(\tau)
    \]
\end{prop}

\begin{proof}
    Consideramos $\varphi \colon [c, d] \to [a, b]$ homeomorfismo tal que $\tau = \sigma \circ \varphi$
    (existe por la definición de caminos equivalentes). Tomamos ahora $Q = \setb{t_0, \dots, t_n}$, una partición de $[c, d]$,
    entonces, $\varphi(Q)$ es una partición de $[a,b]$, ahora, se tiene que
    \begin{gather*}
        L(\tau, Q) \stackrel{\text{def}}{=} \sum_i \norm{(\sigma \circ \varphi) \left( t_i \right) - (\sigma \circ \varphi)\left( t_{i-1} \right)}
        = \sum_i \norm{\sigma\left( \varphi(t_i) \right) - \sigma\left( \varphi\left( t_{i-1} \right) \right)} = \\ =
        L(\sigma, \varphi(Q)) \leq L(\sigma) \stackrel{\text{supremo}}{\implies} L(\sigma \circ \varphi) \leq L(\sigma)
    \end{gather*}
    Cambiando $\varphi$ por $\inv{\varphi}$, obtenemos $L(\sigma) \leq L(\sigma \circ \varphi)$
\end{proof}

\begin{prop}
    Si $\sigma \colon [a, b] \to \real^n$ y $\tau \colon [c,d] \to \real^n$ son caminos inyectivos con la misma imagen $C$, entonces, son equivalentes.
\end{prop}

\begin{proof}
    Consideramos $\sigma \colon [a,b] \to C$ continua (pq $\sigma$ continua en $\real^n$). Ahora, $[a, b]$ compacto $\implies$ C compacto
    (la imagen de un intervalo cerrado es cerrado) $\implies \sigma$ es un homeomorfismo y $\varphi(t) := \inv{\sigma}\left( \tau(t) \right)$
\end{proof}

\begin{obs}
    Los resultados anteriores sugieren que la longitud de una curva es una propiedad de $C$ más que una propiedad de $\sigma$, es decir,
    es una propiedad más analítica que geom\'etrica. Si $C \subset \real^n$ es una ``curva'' parametrizable por un camino inyectivo,
    podemos definir la longitud de $C$ como $L(C) = L(\sigma)$ donde $\sigma$ es cualquier parametrización inyectiva de $C$.
\end{obs}

\subsection*{Integración de funciones vectoriales}

\begin{defi*}
    Sea $E \subset \real^m$ un conjunto medible Jordan y $\vec{f} \colon E \to \real^n$, tenemos que
    \[
        \vec{f} = \left( f_1,\dots, f_n \right)
    \]
    Si todas las $f_i$ son integrables, definimos
    \[
        \int_E \vec{f} = \left( \int_E f_1 , \dots, \int_E f_n \right) \in \real^n
    \]
\end{defi*}

\begin{prop*}[Regla de Barrow]\label{prop:regla_barrow}
    Sea $E = [a, b]$ un intervalo y $\vec{f} \colon E \to \real^n$ una función. Si $\vec{f}$ tiene una primitiva $\vec{F}$
    $\left( \vec{F}^\prime = \vec{f}\right)$, entonces
    \[
        \int^b_a \vec{f} = \vec{F}(b) - \vec{F}(a)
    \]
\end{prop*}

\begin{prop*}
    Sea $\vec{f} \colon E \to \real^n$ una función cualquiera, si $\vec{f}$ es integrable, $\norm{\vec{f}}$ tambi\'en y
    \[
        \norm{\int_E \vec{f}} \leq \int_E \norm{\vec{f}}
    \]
    (considerando la norma euclideana)
\end{prop*}

\begin{proof}
    Tenemos que $\norm{\vec{f}} = \left( f^2_1 + \cdots + f^2_n \right)^{\sfrac{1}{2}}$, que es integrable por ser composición de
    integrables con continuidad.

    Sea ahora $\vec{a} = \int_E \vec{f}$, por la desigualdad de Cauchy-Bunyakovsky-Schwarz, tenemos que
    \begin{gather*}
        \vec{a} \cdot \vec{f}(\cdot) \leq \norm{\vec{a}} \norm{\vec{f}(\cdot)} \implies
        \norm{\vec{a}} \int_E \norm{\vec{f}(t)} \dif t \geq \int_E \left( \vec{a} \cdot \vec{f}(t) \right) \dif t = \\ =
        \int_E \sum^n_{i=1} \left(a_if_i(t) \right) \dif t = \sum_i a_i \overbrace{\int_E f_i(t) \dif t}^{a_i} = \norm{\vec{a}}^2
    \end{gather*}
    Si $\norm{\vec{a}} \neq 0$, teemos que
    \[
        \norm{\vec{a}} \leq \int_E \norm{\vec{f}(t)} \dif t \implies
        \norm{\int_E \vec{f}} \leq \int_E \norm{\vec{f}}
    \]
\end{proof}


\begin{prop}
    Si $\gamma \colon I \to \real^n$ es de clase $\C^1$, entonces es rectificable y su longitud es
    \[
        L(\gamma) = \int_I \norm{\gamma^\prime}
    \]
	(Considerando la norma euclideana)
\end{prop}

\begin{proof}
	Consideramos $\Pa$ una partición de $I$, por la regla de Barrow
	\[
        \gamma(t_i) - \gamma\left( t_{i-1} \right) = \int^{t_i}_{t_{i-1}} \gamma^\prime(t) \dif t \implies
        \norm{\gamma\left( t_i \right) - \gamma\left( t_{i-1} \right)} = \norm{\int^{t_i}_{t_{i-1}}} \leq \int^{t_i}_{t_{i-1}} \norm{\gamma^\prime}
	\]
    Si sumamos
    \[
        L(\gamma, \Pa) \leq \int_I \norm{\gamma^\prime} \stackrel{\text{supremo}}{\implies} L(\gamma) \leq \int_I \norm{\gamma^\prime}
    \]

    Por otro lado, $\gamma^\prime$ continua en un compacto $\implies \gamma$ uniformemente continue. Sea $\varepsilon > 0$, $\exists \delta > 0$
    tal que $\abs{s - t} < \delta \implies \norm{\gamma^\prime(s) - \gamma^\prime(t)} < \varepsilon$ para todo $s,t \in I$. Si $\Pa = \setb{t_0, \dots, t_n}$
    es una partición de $I$ con subintervalos de longitud $< s$, entnoces
    \[
        \int^{t_i}_{t_{i-1}} \norm{\gamma^\prime} \leq \overbrace{\cdots}^{\text{dem. no en clase}} \leq \norm{\gamma\left( t_i \right) - \gamma\left( t_{i-1} \right)}
        + 2 \varepsilon\left( t_i - t_{i-1} \right)
    \]
    Si $I = [a, b]$ y sumamos
    \[
        \int_I \norm{\gamma^\prime} \leq L(\gamma, \Pa) + 2\varepsilon(b-a) \leq L(\gamma) + \overbrace{2\varepsilon(b-a)}^{\text{tan pequeño como queramos}}
    \]
    Por tanto $\int_I \norm{\gamma^\prime} \leq L(\gamma)$, pero ya hemos visto qeu $\int_I \norm{\gamma^\prime} \geq L(\gamma)$ es decir, que
    $\int_I \norm{\gamma^\prime} = L(\gamma)$.
\end{proof}

\begin{obs}
    La fórmula anterior es igualmente válida si $\gamma$ es $\C^1$ a trozos.
\end{obs}

\begin{defi}
    Sea $\gamma \colon I \to \real^n$ un camino definido en un intervalo no compacto. Podemos definir la longitud de $\gamma$, $L(\gamma)$,
    como el supremo de las longitudes $L(\gamma\vert_{k})$, donde $k \subset I$ es un subintervalo compacto de $I$.

    Cuando $\gamma$ es de clase $\C^1$ definida a trozos, la logitud viene dada por $\int_I \norm{\gamma^\prime}$ (tomada como integral impropia
    si es necesario)
\end{defi}


\begin{defi} \label{defi:observaciones_curvas}
    Sea $C \subset \real^n$ una curva regular de clase $\C^1$. Si $C$ es la imagen de una parametrización inyectiva $\gamma \colon II \to \real^n$,
    definiremos la longitud de $C$ como la longitud de $\gamma$.

    Si la imagen de $\gamma$ es todo $C$ salvo en un número finito de puntos, tambi\'en diremos que la longitud de $C$ es la longitud de $\gamma$.

    Si $C$ es la unión disjunta de un número finito de curvas como las precedentes, definiremos la longitud de $C$ como la suma de las longitudes
    de estas.
\end{defi}

\begin{example}
    Consideramos $C \colon \left( x - x_0 \right)^2 + \left( y - y_0 \right)^2 = R^2$ y
    \[
        \begin{aligned}
            \gamma \colon [0,2\pi] &\to C \subset \real^2 \\
            t &\mapsto \gamma(t) = \left( x_0 + R\cos(t), y_0 + R\sin(t) \right)
        \end{aligned}
    \]
    entonces, tenemos que
    \[
        \gamma^\prime(t) = \begin{pmatrix} -R\sin(t) \\ R \cos(t) \end{pmatrix} \qquad
        \norm{\gamma^\prime(t)} = R
    \]
    Y, por último
    \[
        L(C) = L(\gamma) = \int^{2\pi}_0 R \dif t = 2\pi R
    \]
\end{example}

\section{Integrales de línea de funciones escalares}

\begin{defi}
    Sea $\sigma \colon I \to \real^n$ un camino de clase $\C^1$ a trozos, $C = \sigma(I)$ y $f \colon C \to \real$. Definimos
    la integral de linea de $f$ a lo largo de $\sigma$ como
    \[
        \int_\sigma f \dif l := \int_I f\left( \sigma(s) \right) \norm{\sigma^\prime(s)} \dif s
    \]

    La integral existe, por ejemplo, si $I$ es compacto y $f$ continua
\end{defi}

\begin{obs*}
    Cuando $f$ es 1, recuperamos la lonfitud $\int_\sigma \dif l = L(\sigma)$
\end{obs*}

\begin{prop}[Invariancia bajo reparametrizaciones]
    Si tenemos $\sigma \colon I \to C$ y $\tau \colon J \to C$ dos curvas equivalentes, y consideramos una función $f$
    \[
        \int_\tau f \dif l = \int_\sigma f \dif l
    \]
\end{prop}

\begin{proof}
    Considermaos $\varphi \colon I \to J$ un difeomorfismo de clase $\C^1$ tal que $\tau = \sigma \circ \varphi$
    \begin{gather*}
        \int_\tau f \dif l = \int_I f\left( \tau(t) \right) \norm{\tau^\prime(t)} \dif t \stackrel{TCV}{=}
        \int_I f( \underbrace{\tau\left( \varphi(s) \right)}_{\sigma(s)} ) \overbrace{\norm{\tau^\prime\left( \varphi(s) \right)}
        \abs{\varphi^\prime(s)}}^{\norm{(\tau \circ \varphi)^\prime (s)} = \norm{\sigma^\prime (s)}} = \\
        = \int_I f\left( \sigma\left( s \right) \right) \norm{\sigma^\prime(s)} \dif s = \int_\sigma f \dif l
    \end{gather*}
\end{proof}

\begin{defi}
    Sea $C \subset \real^n$ una curva regular. Si $C$ es la imagen de una parametrización inyectiva $\sigma \colon I \to \real^n$,
    llamamos integral de linea de $f$ a lo largo de $C$ a
    \[
        \int_C f \dif l := \int_\sigma f \dif l
    \]
\end{defi}

\begin{obs*}
    Está bien definido y no depende de la parametrización tomada.
\end{obs*}
\begin{obs*}
    Se pueden hacer las mismas consideraciones que en \ref{defi:observaciones_curvas} 
\end{obs*}

\section{Integrales de línea de campos vectoriales}

\begin{defi}
    Sean $I \subset \real$ un intervalo, $\sigma \colon I \to \real^n$ un camino de clase $\C^1$ a trozos con $C = \sigma(I)$, por último,
    sea $\vec{f} \colon C \to \real^n$ un campo vectorial definido en un conjunto abierto $W \supset C$. Definimos la integral de línea o de circulación
    de $\vec{f}$ a lo largo de $\sigma$ como
    \[
        \int_\sigma \vec{f} \dif \vec{l} = \int_I \vec{f} \left( \sigma(s) \right) \cdot \sigma^\prime(s) \dif s
    \]
    (donde $\cdot$ marca el producto escalar euclideano en $\real^n$)
\end{defi}

\begin{prop}[invariancia bajo reparametrizaciones]
    Sea $\sigma  \colon I \to \real^n$ un camino, sea $\varphi \colon I \to J$ un difeomorfismo de clase $\C^1$. Consideramos
    el camino reparametrizado $\tau = \sigma \circ \phi^{-1}$. Entonces, si $\vec{f}$ es una función cualquiera
    \[
        \int_\tau \vec{f} \dif \vec{l} = \pm \int_\sigma \vec{f} \dif \vec{l}
    \]
    donde el signo depende del signo de $\varphi^\prime$
\end{prop}

\begin{proof}
    Ejercicio
\end{proof}

\begin{defi}
    Sea $C \subset \real^n$ una curva regular parametrizada por una parametrización regular inyectiva. Una orientación de $C$ es una clase
    de equivalencia de parametrizaciones de $C$, donde se considera que dos parametrizaciones son equivalentes cuando la reparametrización
    tiene una derivada positiva. Así mismo, diremos que una curva está orientada cuando se ha tomado una orientación.
\end{defi}

\begin{obs}
    Suponemos $C$ orientada por una parametrización regular inyectiva $\sigma \colon I \to \real^n$ ($C = \sigma(I)$). Los vectores tangentes
    $\sigma^\prime(s) \in T_{\sigma(s)}C$ definen orientaciones de los respectivos espacios tangentes. Esta orientación no cambia si se usa
    un camino reparametrizado $\tau = \sigma \circ \inv{\varphi}$ con $\varphi^\prime > 0$.
\end{obs}

\begin{obs}
    Una curva regular compleja puede requerir 2 parametrizaciones regulares para recubrirla completamente. En tal caso, para orientar $C$, se
    requiere que, en la imagen común de las 2 parametrizaciones, las orientaciones coincidan.
\end{obs}
\begin{example*}
    Una curva conexa tiene exactamente dos orientaciones.
\end{example*}
\begin{example*}
    Una curva no conexa, se orienta orientando cada componente conexa.
\end{example*}

\begin{defi}
    Sea $C \subset \real^n$ una curva orientada por una parametrización regular inyectiva $\sigma \colon I \to \real^n$. Sea $\vec{f} \colon C \to \real^n$.
    Definimos la integral de línea o circulación de $\vec{f}$ a lo largo de $C$
    \[
        \int_C \vec{f} \dif \vec{l} = \int_\sigma \vec{f} \dif \vec{l}
    \]
\end{defi}
\begin{obs*}
    Si $C$ es una curva regular orientada y $\sigma$ la parametriza toda excepto un número finito de puntos, tambi\'en escribiremos
    \[
        \int_C \vec{f} \dif \vec{l} = \int_\sigma \vec{f} \dif \vec{l}
    \]
\end{obs*}
\begin{obs*}
    Si $C$ es unión disjunta de un número finito de curvas como la anterior, igualmente calculamos la circulación al largo de cad una y sumamos.
\end{obs*}

\begin{obs}
    Si $C^\circ$ es la curva $C$ con orientaciñon opuesta, entonces
    \[
        \int_{C^\circ} \vec{f} \dif \vec{l} = - \int_C \vec{f} \dif \vec{l}
    \]
\end{obs}

\begin{obs}
    Es habitual usar la notación
    \[
        \int_C \vec{f} \dif \vec{l} =  \int_C \left( f_1 \dif x_1 + \cdots + f_N \dif X_N \right)
    \]
    Si $C$ es cerrada (definición más adelante), usaremos
    \[
        \oint \vec{f} \dif \vec{l} = \int_C \vec{f} \dif \vec{l}
    \]
\end{obs}

\begin{prop}
    Sea $C \subset \real^n$ una curva regular orientada, si $p \in C$. Respecto al producto escalar estándar de $\real^n$, el
    espacio tangente $T_pC$ tiene un único vector tangente \emph{unitario} $\vec{t}(p)$ perteneciente a la orientación de $C$. Es
    el vector tangente unitario a $C$ en $p$.

    Si $\sigma \colon I \to \real^n$ es una parametrización regular inyectiva correspondiente a la orientación de $C$, entonces
    \[
        \vec{t}\left( \sigma(s) \right) = \frac{\sigma^\prime(s)}{\norm{\sigma^\prime(s)}} \in T_{\sigma(s)}C
    \]
\end{prop}
\begin{obs*}
    Orientar $C$ equivale a hacer una elección continua de un vector tangente unitario sobre $C$.
\end{obs*}

\begin{defi}
    Sea $C \subset \real^n$ una curva y $\vec{f} \colon C \to \real^n$ un campo vectorial. Llamamos componente tangencial de $\vec{f}$ a
    la función
    \[
        f_t = \vec{f} \cdot \vec{t}
    \]
    ($f_t \colon C \to \real$).
\end{defi}

\begin{obs*}
    Si $E$ es un e.v. euclideano y $F \subseteq E$, $E = F \oplus F^\perp$. Si $F = <u_1, \dots, u_r>$, $\Pi_F(x) = <x,u_1>u_1 + \cdots + <x,u_r>u_r$, lo
    que quiere decir que
    \[
        T_p\real^n = T_pC \oplus T_pC^\perp \quad \text{y} \quad \underbrace{\left< \vec{f}(p), \vec{t}(p) \right>}_{f_t(p)}\vec{t}(p)
    \]
\end{obs*}

\begin{prop}
    Bajo las condiciones anteriores,
    \[
        \int_C \vec{f} \dif \vec{l} = \int_C f_T \dif l
    \]
\end{prop}

\begin{proof}
    \begin{gather*}
        \int_C \vec{f} \dif \vec{l} = \int_\sigma \vec{f} \dif \vec{l} = \int_I \vec{f}\left( \sigma(s) \right)\sigma^\prime(s) \dif s =
        \int_I \underbrace{\vec{f}\left( \sigma(s) \right) \overbrace{\frac{\sigma^\prime(s)}{\norm{\sigma^\prime(s)}}}^{\vec{t}\left( \sigma(s) \right)}}_
        {f_t\left( \sigma(s) \right)} \norm{\sigma^\prime(s)} \dif s = \\ = \int_I f_t\left( \sigma(s) \right) \norm{\sigma^\prime(s)} \dif s =
        \int_\sigma f_t \dif s = \int_C f_t \dif l
    \end{gather*}
\end{proof}

\begin{example*}
    En $\real^2$ consideramos $C \colon y = x^3$ desde (1,1) a (2,8) y $\vec{F}(x, y) = \left( 6x^3y, 10xy^2 \right)$. Consideramos
    la parametrización $\gamma(x) = x,x^3$ para $x \in [1,2]$, entonces
    \[
        \int_C \vec{F} \dif \vec{l} = \cdots = 1132
    \]
\end{example*}

\section{Integral de superficie de funciones escalares}

\begin{defi}
    Sea $U \subset  \real^2$ abierto y $\sigma \colon U \to \real^n$ una superficie parametrizada con $M = \sigma(U)$. Denotamos por
    $\vec{T_1} = \Dif_1\sigma$ y $\vec{T_2} = \Dif_2 \sigma$, a los vectores tangentes a la parametrización, es decir
    \[
        \vec{J}_\sigma =
        \begin{pmatrix}
            \vec{T_1} & \vec{T_2}
        \end{pmatrix}
    \]
    Diremos que $\sigma$ es regular cuando $\rg(J_\sigma) = 2$ en todos los puntos. Esto, es equivalente a que $\vec{T_1}$ y $\vec{T_2}$ sean
    l.i. que ocurre sii la matriz de Gram
    \[
        G = 
        \begin{pmatrix}
            <\vec{T_1}, \vec{T_1}> & <\vec{T_1}, \vec{T_2}> \\
            <\vec{T_2}, \vec{T_1}> & <\vec{T_2}, \vec{T_2}>
        \end{pmatrix}
    \]
    sea invertible en todo punto, representaremos por $g := \det G$.
\end{defi}

\begin{defi}
    Sea $M$ una superficie parametrizada por $\sigma$ y sea $f \colon M \subset \real^n \to \real$, definimos la integral de superficie de $F$
    sobre/a lo largo de $\sigma$ como
    \[
        \int_\sigma f \dif S := \int_U f\left( \sigma(w) \right) \sqrt{\det G(w)} \dif w_1 \dif w_2
    \]
\end{defi}

\begin{obs*}
    Siempre supondremos que $\sigma \in \C^1$.
\end{obs*}

\begin{obs*}
    Si suponemos que $U$ es un abierto medible Jordan, $f$ es continua, $\sqrt{\det G(w)}$ acotado, podemos asegurar que la integral existe.
\end{obs*}

\begin{lema}
    Sean $\varphi \colon U \to \widetilde{U}$ difeomorfismo de clase $\C^1$, $\sigma \colon U \to \real^n$ una superficie parametrizada.
    Consideramos $\widetilde{\sigma} := \sigma \circ \inv{\varphi} \colon \widetilde{U} \to \real^n$, entonces
    \[
        \int_{\widetilde{\sigma}} f \dif S = \int_\sigma f \dif S
    \]
\end{lema}

\begin{obs*}
    Eso quiere decir que la integral de superficie no está referida a la parametrización, sino al objeto parametrizado
\end{obs*}

\begin{defi}
    Sea $M \subseteq \real^n$ una superficie regular. Si $M$ es la imagen de una parametrización regular inyectiva $\sigma \colon U \to \real^n$,
    denominamos integral de superficie de una función $f$ sobre $M$ a
    \[
        \int_M f \dif S := \int_\sigma f \dif S
    \]
\end{defi}

\begin{obs*}
    Ya hemos visto que no depende del $\sigma$ elegido.
\end{obs*}

\begin{obs}\label{obs:anterior_a}
    Si $M$ es, a parte de un conjunto ``negligible'' (unión finita de puntos y curvas regulares conexas), la imagen de una parametrización regular inyectiva,
    definimos igualmente
    \[
        \int_M f \dif S = \int_\sigma f \dif S
    \]
\end{obs}

\begin{obs*}
    Si $M$ es unión finita disjunta de superficies como las de la observación \ref{obs:anterior_a}, calculamos
    \[
        \int_M f \dif S = \sum_i \int_{M_i} f \dif S
    \]
\end{obs*}

\begin{defi}
    Definimos el area de $M$ (una superficie regular) como
    \[
        \text{area}(M) := \int_M \dif S
    \]
\end{defi}

\begin{lema}
    Sea $\sigma \colon U \to \real^3$ una superficie parametrizada. Entonces
    \[
        \int_\sigma f \dif S = \int_U f\left( \sigma(w) \right) \norm{\vec{T_1} \times \vec{T_2}} \dif w_1 \dif w_2
    \]
\end{lema}

\begin{proof}
    \[
        \determinant{\vec{T_1}\cdot \vec{T_1} & \vec{T_1} \cdot \vec{T_2} \\ \vec{T_2}\cdot\vec{T_1} & \vec{T_2}\cdot\vec{T_2}} =
        \norm{\vec{T_1}}^2 \norm{\vec{T_2}}^2 - \norm{\vec{T_1}}^2\norm{\vec{T_2}}^2 \cos^2 \theta = \norm{\vec{T_1}}^2\norm{\vec{T_2}}^2 \sin^2 \theta
        = \norm{\vec{T_1} \times \vec{T_2}}^2
    \]
\end{proof}

\begin{example*}
    Consideramos $M: x^2 + y^2 + z^2 = R^2$ una esfera en $\real^3$. Vamos a calcular su área. Consideramos la parametrización
    \[
        \sigma(\theta, \phi) = \left( R \cos \phi \sin \theta, R \sin\phi \sin \theta, R \cos \theta \right)
    \]
    definida sobre $U = (0,\pi) \times (0,2\pi)$, entoces
    \[
        J_\sigma\left( T_\theta, T_\phi \right) =
        \begin{pmatrix}
            R \cos \phi \cos \theta & - R \sin\phi\sin\theta \\
            R \sin \phi \cos \theta & R \cos\phi\sin\theta \\
            -R\sin\theta & 0
        \end{pmatrix}
    \]
    \[
        G =
        \begin{pmatrix}
            R^2 & 0 \\ 0 & R^2 \sin^2\theta
        \end{pmatrix}
        \qquad
        g = \sqrt{\det G} = R^2 \abs{\sin \theta} \stackrel{\theta \in (0,\pi)}{=} R^2 \sin \theta
    \]
    \begin{gather*}
        \text{Área}(M) = \int_M \dif S = \int_\sigma \dif S = \int_U R^2 \sin \theta \dif \theta \dif \phi = R^2 \int^{2\pi}_0 \dif\phi \int^\pi_0 \sin \theta \dif \theta = \\
        = R^2 2\pi \left( \left[ -\cos\theta \right]^\pi_0 \right) = 4\pi R^2
    \end{gather*}
\end{example*}

\section{Integrales de superfice en campos vectoriales (en $\real^3$)}

\begin{defi}
    Sea $U \subseteq \real^2$ abierto, $\sigma \colon U \to \real^3$ una superficie parametrizada regular con
    $M = \sigma(U)$. Sea $J_\sigma = \left( \vec{T_1}, \vec{T_2} \right)$, con $\vec{T_i} = \Dif_i \sigma$.
    Sea $\vec{f} \colon M \to \real^3$ un campo vectorial a lo largo de $M$. Denominamos integral de superficie de $\vec{f}$
    sobre $\sigma$ (o el flujo de $\vec{f}$ a trav\'es de $\sigma$) a
    \[
        \int_\sigma \vec{f} \dif \vec{S} := \int_U \vec{f}\left( \sigma(u) \right) \left( \vec{T_1} \times \vec{T_2} \right) \dif u_1 \dif u_2
    \]
\end{defi}

\begin{obs*}
    Suponiendo $\vec{f}$ continua, $U$ abierto medible Jordan, $\sigma$ con derivadas parciales acotadas (p.e. $\sigma \in \C^1$),
    podemos asegurar que la integral anterior existe.
\end{obs*}

\begin{lema}
    Sea $\varphi \colon U \to \widetilde{U}$ un difeomorfismo entre abiertos conexos de $\real^2$, $\sigma \colon U \to \real^3$ una superficie
    parametrizada regular y sea $\widetilde{\sigma} = \sigma \circ \varphi^{-1}$ su reparametrización. Entonces
    \[
        \int_{\widetilde{\sigma}} \vec{f} \dif \vec{S} = \pm \int_\sigma \vec{f} \dif \vec{S}
    \]
    Donde el signo es el mismo que el de $\det J_\varphi$.
\end{lema}

\begin{defi}
    Sea $M \subset \real^3$ una superficie regular parametrizada por una parametrización regular inyectiva. Una orientación de $M$ es una
    clase de equivalencia de parametrizaciones donde se dice que dos parametrizaciones son equivalentes si el difeomorfismo que nos permite
    obtener una reparametrización de la otra tiene jacobiana positiva.
\end{defi}

\begin{defi}
    Diremos que una superficie $M \subset \real^3$ está orientada, cuando hemos elegido una orientación. A partir de la orientación de $M$ dada
    por $\sigma$, cada $T_pM$ está orientado por la base $\begin{pmatrix} \vec{T_1} & \vec{T_2} \end{pmatrix}$. Esta orientación no cambia si
    usamos otra parametrización de $M$ correspondiente a la misma orientación que $\sigma$.
\end{defi}

\begin{obs}
    En general, $M$ requiere dos parametrizaciones inyectivas regulares para recubrirla totalmente. Entonces, para orientar $M$, se requiere que
    estas parametrizaciones definan las mismas orientacions de los espacios tangentes de los puntos que recubren comunmente.
\end{obs}

\begin{defi}
    Una superficie se dice que es orientable cuando tiene una orientación, así mismo, una superficie orientable, diremos que está
    orientada cuando se ha elegido una orientación.
\end{defi}

\begin{obs}
    \begin{itemize}
        \item[]
        \item Una superficie conexa orientable tiene 2 orientaciones.
        \item Una superficie \emph{NO} conexa se orienta orientando cada componente conexa.
    \end{itemize}
\end{obs}
\begin{obs}
    Hay superficies no orientables, como por ejemplo, la banda de Möbius.
\end{obs}

\begin{defi}
    Sea $M$ una superficie orientada. Suponemos que es la imagen de una parametrización inyectiva regular $\sigma \colon U \to \real^3$
    correspondiente a la orientación de $M$. Sea $\vec{f} \colon M \to \real^3$ un campo vectorial. La integral de superficie, o flujo de $\vec{f}$
    a trav\'es de $M$ es
    \[
        \int_M \vec{f} \dif \vec{S} := \int_\sigma \vec{f} \dif \vec{S}
    \]
\end{defi}

\begin{obs}
    Si $M$ es, salvo un conjunto ``negligible'' de puntos, la imagen de una parametrización de estas, tambi\'en definimos el flujo como
    \[
        \int_M \vec{f} \dif \vec{S} := \int_\sigma \vec{f} \dif \vec{S}
    \]

    Si $M$ es unión disjunta de un número finito de superficies como la anterior, el flujo se calcula para cada una y se suma.
\end{obs}

\begin{obs}
    Si $M^\circ$ es $M$ con la orientación opuesta,
    \[
        \int_{M^\circ} \vec{f} \dif \vec{S} = - \int_M \vec{f} \dif \vec{S}
    \]
\end{obs}

\begin{obs}
    Si $M$ es cerrada, a veces, se escribe
    \[
        \oint_M \vec{f} \dif \vec{S} = \int_M \vec{f} \dif \vec{S}
    \]
\end{obs}

\begin{defi}
    Sea $M \subset \real^3$ una superficie regular orientada, $p \in M$, el subespacio $T_pM \subset T_p\real^3$ está orientado y tiene
    dimensión 2, y por tanto, su ortogonal $\left( T_pM \right)^\perp$ tambin está orientado y tiene dimensión 1. Respecto al producto
    escalar estándar de $\real^3$, el espacio $\left( T_pM \right)^\perp$, tiene un único vector normal \emph{unitario} $\vec{n}(p)$, 
    correspondiente a la orientación del espacio. Es el vector normal unitario a $M$ en $p$.
\end{defi}

\begin{obs}
    Si $M$ está parametrizada por $\sigma \colon U \to \real^3$, entonces
    \[
        \vec{n} \left( \sigma(u) \right) = \frac{\vec{T_1} \times \vec{T_2}}{\norm{\vec{T_1} \times \vec{T_2}}}
    \]
    Orientar una superficie en $\real^3$ equivale a hacer una elección continua de un vector normal unitario sobre $M$.

    Por este motivo, se dice que una superficie tiene ``dos caras''.
\end{obs}

\begin{defi}
    Sea $\vec{f} \colon M \to \real^3$ un campo vectorial. La componente normal de $\vec{f}$ sobre $M$ es 
    \[
        f_n = \vec{f} \cdot \vec{n}
    \]
    donde $\vec{n}$ es el vector normal unitario de $M$.
\end{defi}

\begin{prop}
    Bajo las condiciones anteriores,
    \[
        \int_M \vec{f} \dif \vec{S} = \int_M f_n \dif S
    \]
\end{prop}

\begin{proof}
    \[
        \int_M \vec{f} \dif \vec{S} = \int_U \underbrace{\vec{f}\left( \sigma(u) \right) \overbrace{\frac{\vec{T_1} \times \vec{T_2}}{\norm{\vec{T_1} \times \vec{T_2}}}}^
        {\vec{n}}}_{f_n\left( \sigma(u) \right)} \norm{\vec{T_1} \times \vec{T_2}} \dif u_1 \dif u_2 = \int_\sigma f_n\left( \sigma(u) \right) \dif S =
        \int_M f_n \dif S
    \]
\end{proof}

\begin{example*}
    Sea $M : x^2 + y^2 + z^2 = R^2$ orientada ``hacia el exterior'', consideramos la función $\vec{f} = r^\alpha \vec{r}$, donde
    $\vec{r} = (x,y,z)$, y $r = \norm{\vec{r}}$, es decir, $r = \sqrt{x^2 + y^2 + z^2}$.

    Tomamos la parametrización de $M$ en coordenadas esfricas
    \[
        \begin{aligned}
            \sigma \colon U &\to \real^3 \\
            (\theta, \phi) &\mapsto \left( R \cos \phi \sin \theta, R\sin \phi \sin \theta, R\cos \theta \right)
        \end{aligned}
    \]
    Entonces, $J_\sigma = \begin{pmatrix} \vec{T_\theta} & \vec{T_\phi} \end{pmatrix}$ y
    \[
        \vec{T_\theta} \times \vec{T_\phi} = R^2 \sin\theta
        \begin{pmatrix}
            \cos\phi\sin\theta \\ \sin\phi\sin\theta \\ \cos\theta
        \end{pmatrix} \qquad
        \vec{T_\theta} \times \vec{T_\phi} = R \sin\theta \sigma(\theta, \phi)
    \]
    Como apunta ``hacia afuera'', 
    \begin{gather*}
        \int_M \vec{f} \dif \vec{S} = \textbf{+} \int_\sigma \vec{f} \dif \vec{S} = 
        \int_U \vec{f} \left( \sigma(\theta, \phi)\right) \left( \vec{T_\theta} \times \vec{T_\phi} \right) \dif \theta \dif \phi =
        \int_U R^{\alpha+3} \sin \theta \dif \theta \dif \phi =\\= R^{\alpha+3} \underbrace{\int^{2\pi}_0 \dif \phi}_{2\pi}
        \underbrace{ \int^\pi_0 \dif \theta \sin \theta }_{2} = 4\pi R^{\alpha+3}
    \end{gather*}

    Por otro lado, $\vec{n} = \left. \frac{\vec{r}}{r} \right\vert$
    \[
        f_n = \vec{f} \cdot \vec{n} = r^{\alpha} \vec{r} \cdot \left.\frac{\vec{r}}{r}\right\vert = \left. r^{alpha+1}\right\vert = R^{\alpha+1}
    \]
    Ahora, simplemente
    \[
        \int_M \vec{f} \dif \vec{S} = \int_M f_n \dif S = R^{\alpha+1} \underbrace{\int_M \dif M}_{\text{Área}(M) = 4\pi R^2} = 4\pi R^{\alpha+3}
    \]
\end{example*}

\section{Integración de funciones en una subvariedad $m$-dimensional de $\real^n$}

Este tema no se ha dado en clase, si se está interesado, consultar los puntes del profesor.
% QUE LO HAGAN LOS NOVATOS, AHI SE JODAN
