\chapter{Integración multiple}

\section{Integral de Riemann sobre rectángulos compactos}

\begin{defi}
    Un rectángulo de $\real^n$ es un producto $A := I_1 \times \dots \times I_n$
    donde $I_j \in \real$ son intervalos que suponemos acotados y no degenerados,
    es decir, ni vacios, ni reducidos a un punto.
    
    Si los $I_j$ son compactos o abiertos, también lo es $A$.
\end{defi}

\begin{defi}
    La medida o volumen $n$-dimensional (o área si $n=2$) de un rectángulo
    acotado $A = I_1 \times \cdots \times I_n$ es el producto de las longitudes
    de sus costados, es decir
    \[
        \vol(A) = \text{long} (I_1) \times \dots \times \text{long} (I_n)
    \]
\end{defi}

\begin{obs}
    Recoredemos que denominamos partición de un intervalo compacto $[a,b]$ a un
    subconjunto finito de puntos $\Pa=\setb{x_0,x_1,\dots,x_n}$ tales que
    $a = x_0 < x_1 < \cdots < x_{N-1} < x_N = b$. La partición expresa el intervalo
    como la unión de $N$ subintervalos
    \[
        [a,b] = [x_0,x_1] \cup \cdots \cup [x_{N-1},x_N]
    \]
\end{obs}
\begin{obs*}
    Una partición $\Pa^\prime$ se dice que es más fina que otra
    $\Pa$ cuando $\Pa \subset \Pa^\prime$ (es decir, cuando
    tiene más puntos).
\end{obs*}

\begin{defi}
    Dado un rectángulo compacto $A = I_1 \times \cdots \times I_n$, denominamos
    una partición $\Pa$ de $A$ al resultado de hacer una partición $\Pa_j$ a cada
    intervalo $I_j$.
    La partición de $A$ viene representada por $\Pa = \Pa_1 \times \cdots \times
    \Pa_n$ y expresa el rectángulo $A$ como unión de $N = (\abs{\Pa_1}-1) \times
    \cdots \times (\abs{\Pa_n}-1)$ subrectángulos más pequeños.
\end{defi}

\begin{obs*}
    Sean $A^\prime,A^{\prime\prime}\subset A$ rectángulos de la partición, entonces
    $\mathring{A^\prime} \cap \mathring{A^{\prime\prime}} = \emptyset$
\end{obs*}

\begin{lema}
    Si $A$ es un rectángulo, y $\Pa$ una partición de $A$, se tiene que
    \[
        \vol(A) = \sum_{R \in \Pa} \vol(R)
    \]
\end{lema}

\begin{defi}
    Dadas dos particiones $\Pa = \prod\limits^n_{j=1} \Pa_j$ y
    $P^\prime = \prod\limits^n_{j=1} \Pa^\prime_j$ de un rectángulo $A$. Diremos
    que la partición $P^\prime$ es más fina que $\Pa$ si cada $\Pa^\prime_j$ es más
    fina que $\Pa_j$ (es decir, $P_j \subset P^\prime_j$ $\forall j \iff \Pa
    \subset \Pa$).

    Entonces, cada subrectángulo de $\Pa$ es unión de subrectángulos de
    $\Pa^\prime$
\end{defi}

\begin{defi}
    Sea $A \subset \real^n$ un rectángulo compacto y $f \colon A \to \real$ una
    función acotada. Sea $\Pa$ una partición de $A$. Para cada subrectángulo $R$
    de $\Pa$ escribimos
    \[
        m_R = \inf_{x \in R} f(x) \qquad M_R = \sup_{x \in R} g(x)
    \]
    Denominamos suma inferior y suma superior de $f$ respecto a $\Pa$ a los números
    \[
        s(f;\Pa) = \sum_R m_R\vol(R) \qquad S(f;\Pa) = \sum_R M_R\vol(R)
    \]
\end{defi}

\begin{obs}
    Sea $\Pa$ una partición de $A$
    \[
        m_A\vol(A) \leq s(f;\Pa) \leq S(f;\Pa) \leq M_A\vol(A)
    \]
\end{obs}
\begin{obs}
    Si $\Pa$ y $\Pa^\prime$ son dos particiones y $\Pa^\prime$ es más fina que
    $\Pa$, entonces
    \[
        s(f;\Pa) \leq s(f;\Pa^\prime) \leq S(f;\Pa^\prime) \leq S(f;\Pa)
    \]
\end{obs}

\begin{lema}
    Si $\Pa$ y $\Pa^\prime$ son dos particiones de un rectángulo $A$, existe una
    partición $\Pa^{\prime\prime}$ de $A$ que es más fina que $\Pa$ y que
    $\Pa^\prime$.
\end{lema}

\begin{col}
    Si $\Pa, \Pa^\prime$ son dos particiones de $A$, entonces,
    $s(f;\Pa) \leq S(f;\Pa^\prime)$. Por lo tanto,
    $\setb{s(f;\Pa) \vert \Pa \text{ partición de } A}$ está acotado superiormente
    y $\setb{S(f;\Pa) \vert\Pa\text{ partición de } A}$ está acotado inferiormente.
\end{col}

\begin{defi}
    Sea $A$ un rectángulo compacto y sea $f \colon A \to \real$ una función
    acotada. Denominamos integral inferior e integral superior de $f$ en $A$ a los
    números
    \[
        \underline{\int}_A f = \sup_{\Pa} s(f;\Pa) \qquad \text{y} \qquad
        \overline{\int}_A f = \inf_{\Pa} S(f;\Pa)
    \]
    donde el supremo y el ínfimo se toman sobre el conjunto de todas las posibles
    particiones $\Pa$ de $A$. Obviamente, $\underline{\int}_A f \leq
    \overline{\int}_A f$.
\end{defi}

\begin{defi}
    Diremos que una función $f$ acotada, es integrable en $A$ cuando sus integrales
    inferior y superior coinciden. En este caso, su valor común se denomina
    integral de Riemmand de $f$ en $A$ y se denota por
    \[
        \int_A f, \quad \int_A f(x) \dif^nx, \quad \int_A f(x_1,\dots,x_n)
        \dif x_1 \cdots \dif x_n \quad \text{o} \quad \int_A f \dif V
    \]
    En el caso de $n=2$ o $n=3$ se habla de integral doble o integral triple
    respectivamente, ya que es habitual poner dos o tres signos de integral para
    representarlas.
\end{defi}

\begin{prop}[Criterio de Riemman]
    Sea $A \subset \real^n$ un rectángulo compacto y $f \colon A \to \real$ una
    función acotada. $f$ es integrable Riemman sii $\forall \varepsilon > 0$,
    $\exists \Pa$ partición de $A$ tal que $S(f;\Pa)-s(f;\Pa) < \varepsilon$.
\end{prop}
\begin{proof}
    Cálculo I
\end{proof}

\begin{example*}
    \begin{itemize}
        \item[]
        \item Si $f \colon A \to \real$ constante, entonces  $f(x) = c$ y
            $m_R = M_R = c$ $\forall R$
            \[
                s(f;\Pa) = \sum_R m_R \vol(R) = c \sum_R \vol(R)
                = c\vol(A) = S(f;\Pa)
            \]
            Por lo tanto, $f$ es integrable Riemman, y además
            \[
                \int_A f = c\vol(a) \implies \int_A 1 = \vol(A)
            \]
        \item Consideramos la función $f \colon [0,1] \times [0,1] \to \real$
            definida por $f(x,y) = \begin{cases}
                0 \quad \text{si } x,y \in \q \\
                1 \quad \text{En otro caso}
            \end{cases}$, entonces, $m_R = 0$ y $M_R = \vol(R)$ para todo
            $R$, y entonces
            \[
                \underline{\int}_A f = 0 \qquad
                \overline{\int}_A f = \vol(A) = 1\times1 = 1
            \]
            Y por lo tanto, $f$ no es integrable Riemman.
    \end{itemize}
\end{example*}

\begin{prop}[Linealidad]\label{prop:lin_int}
    Sea $A$ un rectángulo compacto y $f,g \colon A \to \real$ integrables Riemman.
    Entonces
    \begin{enumerate}[i)]
        \item $f+g$ es integrable Riemman y $\int_A (f+g) = \int_A f+\int_A g$.
        \item Si $\lambda \in \real$, entonces $\lambda f$ es integrable
            Riemman y $\int_A (\lambda f) = \lambda \int_A f$.
    \end{enumerate}

    Es decir, $\text{Rie}(A) = \setb{f \colon A \to \real \vert f \text{ integrable
    Riemman}}$ es un $\real$-espacio vectorial y 
    \[
        \begin{aligned}
            \text{Rie}(A) &\to \real \\ f &\mapsto \int_A f
        \end{aligned}
    \]
    es una forma lineal.
\end{prop}
\begin{proof}
    \begin{enumerate}[i)]
        \item Observamos que $\inf(f) + \inf(g) \leq \inf(f+g)$ y que
            $\sup(f)+\sup(g) \geq \sup(f+g)$, y por lo tanto,
            \[
                m_R(f) + m_R(g) \leq m_R(f+g) \leq M_R(f+g) \leq
                M_R(f) + M_R(g)
            \]
            Por lo tanto, para toda $\Pa$ partición de $A$
            \[
                s(f;\Pa) + s(g;\Pa) \leq s(f+g;\Pa) \leq S(f+g;\Pa)
                \leq S(f;\Pa) + S(g;\Pa)
            \]
            Y por último
            \[
                \underline{\int}_A f + \underline{\int}_A g \leq
                \underline{\int}_A f+g \leq \overline{\int}_A f+g \leq
                \overline{\int}_A f + \overline{\int}_A g
            \]
            Como $f,g$ son integrables Riemman, $(f+g)$ tambi\item n lo es
            y vale $\int_A f + \int_A g$
        \item Suponemos que $\lambda > 0$, entonces,
            \[
                \inf (\lambda f) = \lambda \inf (f) \implies
                \underline{\int}_A (\lambda f) = \lambda
                \underline{\int}_A f
            \]
            Análogamente,
            \[
                \sup(\lambda f) = \lambda \sup(f) \implies
                \overline{\int}_A (\lambda f) = \lambda
                \overline{\int}_A f
            \]
            Y por lo tanto, $\int_A (\lambda f) = \lambda \int_A f$.
            Y para demostrar el caso de $\lambda < 0$, usaremos que
            $\int_A (-f) = - \int_A f$.
    \end{enumerate}
\end{proof}

\begin{prop}[Positividad]\label{prop:pos_int}
    Sea $f \colon A \to \real$ integrable Riemman. Si $f \geq 0$, entonces
    $\int_A f \geq 0$.
\end{prop}
\begin{proof}
    \[
        f \geq 0 \implies s(f;\Pa) \geq 0 \implies \underline{\int}_A f \geq 0
        \implies \int_A f \geq 0
    \]
\end{proof}

\begin{col}
    Sean $f,g \colon A \to \real$ integrables Riemman. Si $f \leq g$, entonces
    \[
        \int_A f \leq \int_A g
    \]
    Sean $m,M \in \real$ tales que $m < f(x) < M$ $\forall x \in A$. Entonces
    $m\vol(A) \leq \int_A f \leq M\vol(A)$.
\end{col}

\begin{proof}
    \[
        f \leq g \implies g - f \geq 0 \stackrel{\ref{prop:pos_int}}{\implies}
        \int_A (g-f) \geq 0 \stackrel{\ref{prop:lin_int}}{\implies} \int_A g -
        \int_A f \geq 0 \implies \int_A g \geq \int_A f
    \]
\end{proof}

\begin{prop}
    Sea $A$ un rectángulo compacto, sea $f \colon A \to \real$ una función acotada,
    sea $\Pa$ una partición de $A$ y $\xi = \left(\xi_k\right)$ una familia de
    puntos $\xi_k \in R_k$, donde $R_k$ son los rectángulos de la partición. Se
    define la suma de Riemman
    \[
        R(f,\Pa,\xi) = \sum_k f(\xi_k)\vol(R_k) 
    \]
    Si $f$ es integrable Riemman entonces, en un sentido que habría que precisar,
    $R(f,\Pa,\xi)$ se aproxima al valor de $\int_A f$ a medida que la malla de
    $\Pa$ (el máximo de las longitudes de los costados de los rectángulos $R_k$ de
    $\Pa$) tiende a 0.
\end{prop}

\section{Conjuntos de medida nula}

\begin{defi}
    Diremos que un subconjunto $T \subset \real^n$ tiene medida ($n$-dimensional)
    cero o medida nula, si, $\forall \varepsilon > 0$ Se puede recubrir $T$ con
    una familia numerable de rectángulos compactos tales que la suma de sus
    medidas $n$-dimensionales sea $< \varepsilon$. En otras palabras, hay una
    sucesión (finita o infinita) de rectángulos compactos $R_k$ tales que
    \[
        T \subset \bigcup_k R_k \quad \text{y} \quad \sum_k \vol(R_k) <
        \varepsilon
    \]
\end{defi}
\begin{obs*}
    En la definición es irrelevante usar rectángulos compactos o retángulos
    abiertos.
\end{obs*}
\begin{obs}
    Todo conjunto finito tiene medida nula.\\
    Si $T$ tiene medida nula, entonces todo $S \subset T$ tiene medida nula
\end{obs}
\begin{prop} \label{prop_2_2_3}
    La reunión de una familia numerable de conjuntos de medida nula tiene
    medida nula.
\end{prop}
\begin{proof}
    Si $T = \bigcup\limits_{i \in \n} T_i$ donde $T_i$ tiene medida nula para todo
    $i$. Recubrimos $T_i$ con rectángulos compactos $R_{ik}$ ($k \in \n$) tal que
    $\sum\limits_k \vol(R_{ik}) < \frac{\varepsilon}{2^{i+1}}$. Entonces,
    $T = \bigcup\limits_{i,k} R_{ik}$ y
    \[
        \sum_{i,k} \vol(R_{i,k}) = \sum_i \left( \sum_k \vol(R_{i,k}) \right) <
        \sum_i \frac{\varepsilon}{2^{i+1}} = \varepsilon
    \]
\end{proof}

\begin{col*}
    Todo conjunto numerable tiene medida nula
\end{col*}

\begin{example*}
    \begin{itemize}
        \item[]
        \item $\begin{rcases} \z \subset \real \\ \q \subset \real\end{rcases}$
            Son conjuntos de medida nula, porque son numerables
        \item El conjunto de Cantor $K \subset \real$ tiene medida nula, pero
            no es numerable.
        \item En $\real^3$, $T = [0,1]\times[0,1]\times\{0\}$, tiene medida
            nula, ya que
            \[
                T \subset [0,1] \times [0,1] \times
                \left[\frac{\varepsilon}{2},\frac{\varepsilon}{2}
                \right] (\vol(R)=\varepsilon) \implies
                \real^2\times\{0\}
            \]
            Ya que
            \[
                \real^2 \times \{0\} = \bigcup_{(k,l) \in \z^2}
                [k,k+1] \times [l,l+1] \times \{0\}
            \]
    
            (unión numerable de $R$ con medida nula).
        \item Si $a < b$, entonces $[a,b] \subset \real$ no tiene medida nula.
    \end{itemize}
\end{example*}
\begin{col}
    Dentro de $\real^n$, el subespacio, $\real^m \times \{0\} \subset \real^n$
    tiene medida nula.
\end{col}

\begin{defi}
    Diremos que un subconjunto $A \subset \real^n$ tiene contenido
    ($n$-dimensional) cero o contenido nulo, si $\forall \varepsilon > 0$, hay un
    recubrimiento finito de $A$ por rectángulos compactos $R_i$ tal que
    \[
        \sum \vol(R_i) < \varepsilon
    \]
\end{defi}
\begin{obs*}
    En la definición, se pueden substituir los rectángulos compactos por
    rectángulos abiertos.
\end{obs*}
\begin{obs*}
    Un conjunto de contenido cero, tiene medida cero.
\end{obs*}
\begin{obs*}
    Que un conjunto tenga medida cero, no implica que tenga contenido cero.
\end{obs*}

\begin{prop}
    Si $A \subset \real^n$ es compacto y tiene medida cero, entoces tiene contenido
    cero.
\end{prop}
\begin{proof}
    Sea $\varepsilon > 0$, existe una sucesión $(R_i)$ de rectángulos abiertos tal
    que
    \[
        A \subset \bigcup_i R_i \qquad \qquad \sum_i \vol(R_i) < \varepsilon
    \]
    Como $A$ es compacto, un número finito de los $R_i$ recubren $A$ (condición
    de Heine-Borel de compacto) y la suma de sus volúmenes es $< \varepsilon$.
\end{proof}

\begin{lema}
    Si $a < b$, $[a,b] \subset \real$ no tiene contenido cero. Más precisamente,
    sea $\{R_1,\dots,R_n\}$ un recubrimiento de $[a,b]$ por intervalos compactos.
    Entonces, $\sum\limits^{n}_{i=1} \text{long}(R_i) \geq b-a$
\end{lema}
\begin{proof}
    Hacemos inducción sobre $n$. Si $n = 1$, el resultado es trivial. Ahora,
    suponemos cierto el resultado anterior para $n$ y consideramos
    $\{R_1,\dots,R_{n+1}\}$ un recubrimiento de $[a,b]$ por intervalos compactos.
    Podemos suponer, sin p\'erdida de generalidad, que $a \in R_1$, por tanto,
    $R_1 = [\alpha,\beta]$ donde $\alpha \leq a \leq \beta$. Ahora, hay dos
    opciones. Si $\beta \geq b$, entonces $\text{long}(R_1) \geq b - a$ y ya está.
    Si $\beta< b$, entonces, $[\beta,b]$ está recubierto por$\{R_2,\dots,R_{n+1}\}$
    y por hipótesis de inducción $\sum\limits^{n+1}_{i=2} \text{long}(R_i) \geq
    b - \beta$. Por tanto, $\sum\limits^{n+1}_{i=1} \text{long}(R_i) \geq
    (\beta - a) + (b - \beta) = b - a$.
\end{proof}
\begin{col*}
    Si $a < b$, $[a,b]\subset \real$, no tiene medida nula.
\end{col*}
\begin{obs*}
    De la misma manera, un rectángulo compacto no degenerado de $\real^n$ no
    tiene medida nula.
\end{obs*}
\begin{col*}
    Si un conjunto $A \subset \real^n$ tiene un punto interior, entonces, no tiene
    medida nula.
\end{col*}

\begin{obs}
    Un conjunto de contenido nulo es, necesariamente, acotado. Pero hay conjuntos
    acotados de medida nula, que no son de contenido nulo.
\end{obs}
\begin{prop}
    Sea $D\subset \real ^{n-1}$ acotado, $f\colon D \to \real$ uniformemente contínua. 
    El grafo de $f$, $\graf \left( f \right) = \left\{ \left( x, f \left( 
    x \right) \right) | x \in D \right\}$, tiene medida nula.
\end{prop}
\begin{proof}
    Sea $\varepsilon > 0$. Sea $R \supset D $ un rectángulo compacto de volumen
    $\left| R \right|$.
    \[
        \exists \delta > 0 \; \tq \; d\left( x, y \right) < \delta \implies d\left( 
        f\left( x \right), f\left( y \right) \right) < \frac{\varepsilon}{\left| 
        R \right|}.
    \]
    Partimos ahora $R$ en subrectángulos compactos $I_k$ suficientemente pequeños 
    como para que cada uno de ellos esté contenido en una bola de radio 
    $\frac{\delta}{2}$. Entonces, $f\left( I_k \right) \subset J_k$, siendo
    $J_k$ un intervalo compacto de longitud menor o igual a 
    $\frac{\varepsilon}{\left| R \right|}$. Además, $\graf \left( f \right) 
    \subset \bigcup_k I_k \times J_k $. Finalmente,
    \[
        \sum_k \vol \left( I_k \times J_k \right) = \sum_k \vol \left( I_k \right)
        \cdot \vol \left( J_K \right) \leq \sum_k \vol \left( I_k \right) \cdot
        \frac{\varepsilon}{\left| R \right|} = \varepsilon.
    \]    
\end{proof}
\begin{col*}
    Si $D \subset \real ^{n-1}$ es compacto y $f\colon D \to \real$ contínua,
    $\graf \left( f \right) \subset \real^n$ tiene medida nula.
\end{col*}
\begin{col*}
    Si $D \subset \real ^{n-1}$ es abierto y $f\colon D \to \real$ contínua,
    $\graf \left( f \right) \subset \real^n$ tiene medida nula.
\end{col*}
\begin{proof}
    Cualquier punto de $D$ está contenido en una bola cerrada (que es un conjunto
    compacto) de centro en coordenadas racionales y radio racional. Además, como el
    conjunto de bolas de centro en coordenadas racionales y radio racional es
    el producto cartesiano de conjuntos numerables, es a su vez numerable. Así pues,
    $D$ es la unión numerable de conjuntos compactos, sean estos $K_i$. Entonces,
    $\graf \left( f \right) \subset \bigcup_{i\in \n} \graf \left( f_{|K_i} \right)$.
    Finalmente, por el corolario anterior y la proposición \ref{prop_2_2_3}, 
    $\graf \left( f \right) $ tiene medida nula.
    
\end{proof}
\begin{defi}
    Diremos que un \textit{cuadrado} es un rectángulo cuyos lados tienen la misma
    longitud.
\end{defi}
\begin{prop*}
    Sea $Q$ un cuadrado en $\real ^n$. Si la longitud de su lado es $l$, su volumen
    es $l^n$ i su diámetro (en la norma euclideana de $\real^n$) es $l\sqrt{n}$.
\end{prop*}
\begin{prop}
    Sea $Z \subset \real^n$ de medida nula. Para toda $\varepsilon >0$ existe una
    familia numerable de cuadrados compactos $Q_k$ tales que $Z\subset
    \bigcup_{k \in \n} Q_k$ y $\sum_{k \in \n} \vol \left( Q_k \right) <
    \varepsilon$.
\end{prop}
\begin{lema}
    Sean $A\subset \real^n$ un rectángulo compacto y $f\colon A \to \real^n$ una
    función lipschitziana. Si $Z \subset A$ tiene medida nula, $f\left(Z \right)$
    tiene medida nula.
\end{lema}
\begin{proof}
    Más adelante.
\end{proof}
\begin{prop}
    Sea $f \colon U \to \real^n$ de clase $\C ^1$ en un conjunto abierto $U \subset
    \real ^n$. Si $Z \subset U$ tiene medida nula, $f\left( Z \right) $ tiene
    medida nula.\\
    El resultado es falso si $f$ es solamente $\C ^0$. Un ejemplo de esto es la
    curva de Peano.
\end{prop}
\begin{proof}
    Más adelante.
\end{proof}
\begin{col}
    Toda subvariedad $M \subset \real^n$ de clase $C^1$ de dimensión $m<n$ tiene
    medida nula.
\end{col}

\section{El Teorema de Lebesgue}
\begin{defi}
    Sean $X$ un espacio métrico y $f\colon X \to \real$. Llamamos 
    \textit{oscilación} de $f$ sobre $E \subset X$ a
    \[
        \omega\left( f, E \right) := \sup_{x, y \in E} \left| f\left( x\right) - 
        f\left( y\right)\right|.
    \]
    También se llama a este valor diámetro de $f\left( E\right) \in \left[ 
    0, +\infty \right]$. $\omega\left( f, E \right)$ es zero si y solo si $f_{|E}$ es 
    constante y $+\infty$ si y solo si $f_{|E}$ no es acotada.
\end{defi}
\begin{obs}
    $E \subset E^{\prime} \implies \omega\left( f, E\right) \leq \omega\left( f, E^{
    \prime}\right)$.
\end{obs}
\begin{lema}
    Si $f\colon E \to \real$ es acotada, $\omega \left( f, E \right)= \sup_{x\in E}
    f\left( x\right) - \inf_{x\in E} f\left( x\right)$.
\end{lema}
\begin{defi}
    Sean $f\colon X \to \real$, $a\in X$. Llamamos oscilación de $f$ en $a$ a
    \[
        \omega\left( f, a\right) := \lim_{r\to 0} \omega\left( f, B\left( a, r 
        \right) \right).
    \]
\end{defi}
\begin{obs*}
    \[
        \omega\left( f, a\right) = \inf_{r>0} \omega\left( f, B\left( 
        a, r\right) \right).
    \]
\end{obs*}
\begin{lema}
    \[
        f\colon X \to \real \text{  es contínua en $a$} \iff \omega\left( f, a\right)
        = 0.
    \]
\end{lema}
\begin{proof}
    $\implies$ \\
    Supongamos que $f$ es contínua en $a$. Entonces,
    \[
        \forall \varepsilon > 0, \exists \delta > 0 \; \tq \; d\left(x, a \right)
        <\delta \implies \left| f\left(x \right) - f\left( a \right) \right| 
        < \varepsilon.
    \]
    Sean $x, y \in B\left( x, \delta \right)$. Tenemos que
    \[
        \begin{gathered}
            \left| f\left(x \right) - f\left( y \right) \right| \leq
            \left| f\left(x \right) - f\left( a \right) \right| +
            \left| f\left(a \right) - f\left( y \right) \right| \leq 2\varepsilon
            \implies \\
            \implies \omega\left( f, B\left( a, \delta \right) \right) \leq 
            2\varepsilon \implies \omega \left( f, a \right) =0.
        \end{gathered}
    \] \\
    $\impliedby$ \\
    Supongamos $\omega \left( f, a \right) = 0$. Sea $\varepsilon > 0, \exists \delta > 0 \tq \omega \left( f, \text{B}\left(a, \delta \right) \right) < \varepsilon$. Per tant, $|f(x) - f(a)| < \varepsilon$ si $\text{d}\left( x, a \right) < \delta \implies f$ continua en a.
\end{proof}
\begin{example*}
    Sea $f: \real \to \real$
    \begin{enumerate}
        \item $f(x) = \begin{cases}
            \frac{1}{x} & \text{si } x \neq 0 \\
            0 & \text{si } x = 0
            \end{cases}$ \\
            $\omega \left( f, 0 \right) = \lim_{r \to 0} \omega \left( f, \text{B} \left( 0, r \right) \right) = \infty \implies f$ no es continua en $x = 0$.
        \item $f(x) = \begin{cases}
            \sin \frac{1}{x} & \text{si } x \neq 0 \\
            0 & \text{si } x = 0
            \end{cases}$ \\
            $\omega \left( f, 0 \right) = \lim_{r \to 0} \omega \left( f, \text{B} \left( 0, r \right) \right) = 2 \implies f$ no es continua en $x = 0$.
    \end{enumerate}
\end{example*}
\begin{lema} \label{lema:2.3.6}
    Sea $a \tq \omega \left( f, a \right) < c. \exists \delta \tq \omega \left( f, \text{B}(a, \delta \right) < c.$
\end{lema}
\begin{proof}
    Por contrarecíproco. $\forall \delta \omega \left( f, \text{B} \left( a, \delta \right) \right) \geq c \implies \omega \left( f, a \right) \geq c$. Contradicción.
\end{proof}
\begin{prop}
    Sea $X$ un espacio métrico, $f : X \to \real$. El conjunto $a = \{ x \in X | \omega \left( f, x \right) < c \}$ es abierto.
\end{prop}
\begin{proof}
    Sea $a \in A \tq \omega \left( f, a \right) < c, c > 0$. Por el lema \ref{lema:2.3.6} $\exists \delta \tq \omega \left( f, \text{B} \left( a, \delta \right) \right) < c$. \\
    Sea $b \in \text{B} \lp a, \delta \rp, \exists \delta' \tq \text{B} \lp b, \delta' \rp \subset \text{B} \lp a, \delta \rp$.
    \[
    \omega \lp f, \text{B} \lp b, \delta' \rp \rp \leq \omega \lp f, \text{B} \lp a, \delta \rp \rp < c \stackrel{\ref{lema:2.3.6}}{\implies} \omega \lp f, b \rp < c
    \]
    Por tanto, $b \in A$ i B$\lp a, \delta \rp \subset A \implies a$ es un punto interior de $A \implies A$ es abierto.
\end{proof}
\begin{col*}
    Si $X$ es compacto, $\{x \in X | \omega \lp f, x \rp \geq c\}$ es compacte.
\end{col*}
\begin{proof}
    Todo conjunto cerrado de un espacio compacto es compacto.
\end{proof}
\begin{lema}\label{lema:teo_lebesgue}
    Sea $A \subset \real^n$ rectángulo compacto. Sean $T_1, \dots, T_n \subset A$ rectángulos compactos $\tq A = \bigcup_{i=1}^n T_i$. Existe una partición $\Pa$ de $A$ t.q. para cada subrectángulo $R$ de $\Pa$, existe un $i$ t.q. $R \subset T_i$.
\end{lema}
\begin{proof}
    $A = I_1 \times \dots \times I_k \times \dots \times I_n \implies \Pa = \Pa_1 \times \dots \times \Pa_n$ cada
    \[ \Pa_k = \{x_0 = a, x_{1^-}, x_{1^+}, \dots, x_{i^-}, x_{i^+}, \dots, x_{n^-}, x_{n^+}, b\}, \]
    donde $x_{i^-}, x_{i^+}$ son los extremos inferiores y superiores (respectivamente) de $T_i$. Por construcción, $R \subseteq T_i$.
\end{proof}
\begin{lema*}\label{lema:uno_lebesgue}
    Sea $A \subseteq \real^n$ rectángulo compacto, $f : A \to \real, f$ fitada. Suposem $\forall x  \in A, \omega \lp f, x \rp < \varepsilon$. Entonces existe una partición $\Pa$ de $A \tq S \lp f, \Pa \rp - s \lp f, \Pa \rp < \varepsilon \vol \lp A \rp$.
\end{lema*}
\begin{proof}
    Sea $x \in A$, existe un rectángulo compacto $R_x \subset A \tq x \in \mathring{R}_x$ (interior relativo a $A$) i $\omega \lp f, R_x \rp = M_{R_x} - m_{R_x} < \varepsilon$. \\
    Sean $R_{x_i}$ un conjunto finito de los $R_x$ que recubren $A$. Sea $\Pa$ una partición de $A \tq$ cada subrectángulo de $\Pa$ esté dentro de uno de los $R_{x_i}$, para cada cada subrectángulo $R$ de $\Pa$, $M_R - m_R < varepsilon$. Enonces,
    \[ S \lp f,\Pa \rp = s \lp f,\Pa \rp = \sum_{R \in \Pa} (M_R - m_r) \vol \lp R \rp < \varepsilon \sum_{R \in \Pa} vol \lp R \rp = \varepsilon \vol \lp A \rp. \]
\end{proof}
\begin{col*}
    $f$ continua $\implies f$ integrable Riemann (ya que $f$ continua $\implies \forall x \omega \lp f, x \rp = 0$).
\end{col*}
\begin{lema*}[2]\label{lema:dos_lebesgue}
    Sean $A \subset \real^n$ rectángulo compacto, $f : A \to \real$ fitada, $c > 0$. Si $f$ es integrable Riemann, el conjunto $B_c = \{x \in A | \omega \lp f, x \rp \geq c\}$ tiene mesura nula.
\end{lema*}
\begin{proof}
    Sea $\varepsilon > 0$, aplicando el criterio de Riemann, existe una partición $\Pa$ de $A \tq S \lp f, \Pa \rp - s \lp f, \Pa \rp < c\varepsilon$. \\
    Observamos que 
    \[
        A = A_1 \cup A_2 \qquad A_1 = \bigcup_{R \in \Pa} \mathring{R}
    \]
    $A_2$ tiene medida nula por ser unión finita de subconjuntos de subespacios de
    $\dim < n$. Ahora, separamos los subrectángulos de $\Pa$ en dos clases
    \begin{itemize}
        \item $\Pa^\prime$ subrectángulos $R$ tal que $B_c$ corta a $R \implies$ en
            todos ellos, $M_R - m_r \geq c$.
        \item $\Pa^{\prime\prime}$ el resto.
    \end{itemize}
    Tenemos
    \begin{gather*}
        c \sum_{R \in \Pa^\prime}\vol\left( R \right) \leq 
        \sum_{R \in \Pa^\prime}\left( M_R - m_R \right) \vol\left( R \right) \leq
        \sum_{R \in \Pa} \left( M_R -m_R \right)\vol \left( R \right) =\\=
        S(f, \Pa) - s(f, \Pa) < c\varepsilon \implies \sum_{R \in \Pa^\prime}
        \vol\left( R \right) < \varepsilon
    \end{gather*}
    Así pues, hemos recubierto $B_c \cap A_1$ con un número finito de rectángulos de
    volumenes $< \varepsilon$. Tambi\'en podemos hacerlo con $B_c \cap A_2$, ya que
    tiene medida nula $\implies$ podemos recorrer $B_c$ con rectángulos de volumenes
    $< 2\varepsilon \implies B_c$ tiene medida nula.
\end{proof}

\begin{teo}[de Lebesgue]
    Sean $A \subseteq \real^n$ un rectángulo compacto, $f \colon A \to \real$ una
    función acotada. Sea $\disc\left( f \right) = \setb{x \in A \vert x \text{ no
    es continua en }A}$. Entonces, $f$ es integrable Riemman en $A$ si y solo si
    $\disc\left( f \right)$ tiene medida nula.
\end{teo}

\begin{proof}
    Por comodidad, llamaremos $B = \disc\left( f \right)$ y $B_c = \setb{x \in A
    \vert \omega\left( f, x \right) \geq c}$.

    \bimplies

    Sea $f$ integrable Riemman en $A$. Observamos que
    \[
        B = \bigcup_{k \geq 1} B_{\sfrac{1}{k}}
    \]
    por el lema \hyperref[lema:dos_lebesgue]{2}, $B_{\sfrac{1}{k}}$ tienen medida
    nula. Como es una unión numerable, $B$ tiene medida nula.

    \bimpliedby

    Suponemos que $B$ tiene medida nula. Sea $\varepsilon > 0$ y $B_\varepsilon
    \subset B$, que tiene medida nula y es compacto, por el lema
    \hyperref[lema:dos_lebesgue]{2}, por lo tanto, tiene contenido nulo y, por lo
    tanto, existen rectángulos compactos $T_1, \dots, T_N \subset A$ tales que
    \[
        B_\varepsilon = \bigcup_{i = 1}{N} T_i \qquad
        \sum_{i=1}{N}\vol\left( T_i \right) < \varepsilon
    \]
    Sea $\Pa$ una partición de $A$. Separamos los subrectángulos $R$ en los
    siguientes tipos
    \begin{itemize}
        \item $R_1$: $R$ está dentro de algún $T_i$
        \item $R_2$: el resto.
    \end{itemize}

    Tenemos ahora que
    \[
        \sum_{R \in R_1}\left( M_R - m_R \right)\vol\left( R \right) \leq
        \left( M_A -m_A \right)\sum^{N}_{i=1} \vol\left( T_i \right) <
        \left( M_A - m_A \right) \varepsilon
    \]
    Ahora, si $R \in R_2$, entonces $R$ no conta a $B_\varepsilon$, por lo tanto,
    $w(f,x) < \varepsilon$ $\forall x \in R$, por el lema
    \hyperref[lema:uno_lebesgue]{1}, existe una partición $\Pa_R$ de $R$ tal que
    $S\left( f, \Pa_R \right) - s\left( f, \Pa_R \right) < \varepsilon
    \vol \left( R \right)$.
    
    Podemos refinar $\Pa$ a una partición $\Pa^\prime$ tal que todos sus
    subrectángulos, est\'en dentro de los subrectángulos de $\Pa_R$ (por el lema
    \hyperref[lema:teo_lebesgue]{teórico})
    \begin{gather*}
        S\left( f, \Pa^\prime \right) - s\left( f, \Pa^\prime \right) =
        \sum_{R^\prime \in R_1} \left( M_{R^\prime} - m_{R^\prime} \right)
        \vol \left( R^\prime \right) + \sum_{R^\prime \in R_2}
        \left( M_{R^\prime} - m_{R^\prime}\right) \vol\left( R^\prime \right)
        < \\ < \left( M_A - m_A \right)\varepsilon + \sum_{R^\prime \in R_2}
        \varepsilon \vol\left( R^\prime \right) = \varepsilon \left( 
        \left( M_A - m_A \right) + \vol \left( A \right)\right) 
    \end{gather*}
    Teniendo en cuenta que
    \[
        \sum_{R \in R_2} \sum_{R^\prime \in R} \left( M_{R^\prime} - 
        m_{R^\prime}\right) \vol\left( R^\prime \right) < \sum_{R \in R_2}
        \varepsilon \vol\left( R \right)
    \]
    Ahora, tan solo queda obervar que $\varepsilon \left( \left( M_A - m_A \right)
    + \vol \left( A \right)\right)$ lo podemos hacer tan pequeño como queramos.
\end{proof}

\begin{defi}
    Una propiedad que se satisface para todos los puntos de un conjunto $A$,
    excepto en un conjunto $N \subset A$ de medida nula, se dice que se satisface
    casi para todo.
\end{defi}

\begin{prop}
    Toda función $f \colon A \to \real$ continua sobre un rectángulo compacto es
    integrable.
\end{prop}

\section{Integral de Riemman sobre conjuntos más generales}

 \begin{defi}
    Un subconjunto $C \subset \real^n$ se dice que es admisible si es acotado y $\fr(C)$ es de medida nula.
 \end{defi}
 \begin{obs}
    Sea $X$ un espacio métrico, $C \subset X \colon \fr(C) = \overline{C} \setminus \mathring{C} = \overline{C} \cap \overline{\lp X \setminus C \rp} = X \setminus \lp \inte(C) \cup \ext(C) \rp$ y $X = \inte(C) \cup \fr(C) \cup \ext(C)$.
 \end{obs}
 \begin{lema} 
    \begin{enumerate}[(1)]
        \item[]
        \item $\fr \lp A\cup A' \rp, \fr \lp A\cap A' \rp, \fr \lp A\setminus A' \rp \subset \fr(A)\cup\fr(A')$.
        \item $A \subset X$, $B \subset Y \implies \fr \lp A \times B\rp = \lp \fr(A)\times \overline{B} \rp \cup \lp \overline{A} \times \fr(B) \rp$.
    \end{enumerate}
\end{lema}
\begin{col}
    \begin{enumerate}[(1)]
        \item[]
        \item $A$, $A' \subset \real^n$ admisibles $\Longrightarrow A\cup A'$, $A\cap A'$ i $A\setminus A'$ son admisibles.
        \item $A \subset \real^m, B \subset \real^n$ son admisibles $\implies A\times B \subset \real^{m+n}$ es admisible.
    \end{enumerate}
\end{col}

% A PARTIR D'AQUI RAUL

% A PARTIR D'AQUI OSCAR
