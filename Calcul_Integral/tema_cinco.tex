\chapter{Integración de formas diferenciales y teorema de Stokes}

En este tema, consideraremos, salvo mención contraria, que todas las funciones son de clase $\C^\infty$

\section{Formas diferenciales en $\real^n$}

\begin{defi}
    Definiremos $\Omega^1$ como el espacio vectorial generado por $\dif x^1, \dots, \dif x^n$, que es un espacio vectorial real de dimensión $n$.
    Denotaremos como $\Omega^\bullet$ el álgebra exterior de $\Omega^1$. Su poducto está regido por
    \[
        \dif x^i \wedge \dif x^i = 0
        \qquad
        \dif x^i \wedge \dif x^j = -\dif x^j \wedge x^i \quad (i \neq j)
    \]
    Así pues, tenemos $\Omega^\bullet = \bigoplus\limits^{n}_{r = 0} \Omega^r$, donde estos subespacios tienen dimensión y bases
    \begin{itemize}
        \item $\dim \Omega^0 = 1$, base 1
        \item $\dim \Omega^1 = n$, base $\left( \dif x^i \right)_{1 \leq i \leq n}$
        \item $\dim \Omega^2 = \binom{n}{2}$, base $\left( \dif x^i \wedge \dif x^j \right)_{i < j}$
        \item $\vdots$
        \item $\dim \Omega^{n-1} = n$, base $\left( \dif x^1 \wedge \cdots \wedge \dif x^{i-1} \wedge \dif x^{i+1} \wedge \cdots \wedge \dif x^n \right)_i$
        \item $\dim \Omega^n = 1$, base $\dif x^1 \wedge \cdots \wedge \dif x^n$.
    \end{itemize}

    En general, se tiene que $\Omega^k$ tiene domensión $\binom{n}{k}$, y una base es $\left( \dif x^I \right)$ donde
    \[
        \dif x^I \equiv \dif x^{i_1} \wedge \cdots \wedge \dif x^{i_k}
    \]
    e $I = \left( i_1, \dots, i_k \right)$ es un multiíndice de longitud $k$ estrictamente creciente. Así pues, $\Omega^\bullet = 2^n$.
\end{defi}

\begin{defi}
    Denominamos formas diferenciales en $\real^n$ a los elementos del conjunto $\Omega^\bullet\left( \real^n \right) = \C^\infty\left( \real^n \right) \otimes
    \Omega^\bullet$, es decir, expresiones de la forma
    \[
        \omega = \sum_I f_I \dif x^I
    \]
    Donde $f_I \in \C^{\infty}\left( \real^n \right)$ e $I$ es un multiíndice. Tambi\'en tenemos la gradución de $\Omega^\bullet \left( \real^n \right)$
    \[
        \Omega^\bullet \left( \real^n \right) = \bigoplus^n_{k=0} \Omega^k \left( \real^n \right) \qquad \text{donde}
        \Omega^k \left( \real^n \right) = \C^\infty \left( \real^n \right) \otimes \Omega^k
    \]
    son las formas diferenciales de grado $k$ ($k$-forma diferencial). Así, podemos escribir
    \[
        \omega = \omega_0 + \omega_1 + \cdots + \omega_n
    \]

    Los elementos de $\Omega^n\left( \real^n \right)$ se denominan por abuso del lenguaje, ``formas diferenciales de grado máximo''. Si
    $\omega \in \Omega^k \left( \real^n \right)$, escribiremos $\abs{\omega} := k$.
\end{defi}

\begin{defi}
    Las operaciones algebraicas de $\Omega^\bullet$ se trasladan a operaciones algebraicas de $\Omega^\bullet \left( \real^n \right)$. En particular,
    el producto exterior de dos formas diferenciales se calcula según la distributividad respecto de la suma, y con la regla
    \[
        \left( f \dif x^{i_1} \wedge \cdots \wedge \dif x^{i_k} \right) \wedge \left( g \dif x^{j_1} \wedge \cdots \wedge \dif x^{j_l} \right) =
        fg \dif x^{i_1} \wedge \cdots \wedge \dif x^{i_k} \wedge \dif x^{j_1} \wedge \cdots \wedge x^{j_l}
    \]
\end{defi}

\begin{prop}
    El producto exterior en $\Omega^\bullet \left( \real^n \right)$ tiene elemento neutro (la función constante igual a 1), es asociativo
    y anticonmutativo: si $\alpha$ y $\beta$ son formas diferenciales de grados $\abs{\alpha}$ y $\abs{\beta}$, entonces
    \[
        \beta \wedge \alpha = (-1)^{\abs{\alpha}\abs{\beta}} \alpha \wedge \beta
    \]
\end{prop}

\section{La diferencial exterior en $\real^n$}

\begin{defi}
   Dada una función $f \in \C^\infty \left( \real^n \right) = \Omega^0 \left( \real^n \right)$, podemos construir una 1-forma diferencial 
   $\dif f \in \Omega^1 \left( \real^n \right)$, de acuerdo con la fórmula
   \[
       \dif f := \sum^n_{i=1} \frac{\partial f}{\partial x^i} \dif x^i
   \]
   donde $\frac{\partial f}{\partial x^i}$ representa la derivada parcial $\Dif_i f$ de $f$ respecto a la variable $i$-\'esima.
\end{defi}

\begin{obs}
    Observamos que si $f \left( a^1, \dots, a^n \right) = a^i$, entonces $\dif f = \dif x^i$. Está función se conoce como la coordenada
    cartesiana $i$-\'esima y es habitual denotarla por $x^i$
\end{obs}

\begin{defi}
    La diferencial actuando sobre funciones se puede extender a una aplicación $\real$-lineal, denominada \emph{diferencial exterior} y definida
    de la manera siguiente
    \[
        \begin{aligned}
            \dif \colon \Omega^\bullet \left( \real^n \right) &\to \Omega^\bullet \left( \real^n \right)
            f \dif x^I &\mapsto \dif\left( f \dif x^I \right) := \dif f \wedge \dif x^I
        \end{aligned}
    \]
\end{defi}

\begin{obs*}
    Si $\omega \in \Omega^k \left( \real^n \right)$, entonces, $\dif \omega \in \Omega^{k+1} \left( \real^n \right)$
\end{obs*}

\begin{example*}
    En $\real^2$, se tiene que
    \begin{gather*}
        \dif f = \pdv{f}{x} \dif x + \pdv{f}{y} \dif y \\
        \dif \left( f \dif x \wedge \dif y \right) = 0 \\
        \dif\left( f \dif x + g \dif y \right) = \dif f \wedge \dif x + \dif g \wedge \dif y =\\=
        \left( \cancel{\pdv{f}{x} \dif x} + \pdv{f}{y} \dif y \right) \wedge \dif x + \left( \pdv{g}{x} \dif x + \cancel{\pdv{g}{y} \dif y} \right)\wedge \dif y =
        \left( -\pdv{f}{y} + \pdv{g}{x} \right) \dif x \wedge \dif y
    \end{gather*}
    En $\real^3$, se tiene que
    \begin{gather*}
        \dif f = \pdv{f}{x} \dif x + \pdv{f}{y} \dif y + \pdv{f}{z} \dif z \\
        \dif \left( f_1 \dif x + f_2 \dif y + f_3 \dif z \right) = \cdots = \\ = \left( \pdv{f_3}{y} - \pdv{f_2}{x} \right)\dif y \wedge \dif z +
        \left( \pdv{f_1}{z} - \pdv{f_3}{x} \right) \dif z \wedge \dif x + \left( \pdv{f_2}{x} - \pdv{f_1}{y} \right)\dif x \wedge \dif y \\
        \dif \left( g_1 \dif y \wedge \dif z +  g_2 \dif z \wedge \dif x + g_3 \dif x \wedge \dif y \right) = \cdots =
        \left( \pdv{g1}{x} + \pdv{g_2}{y} + \pdv{g_3}{z} \right) \dif x \wedge \dif y \wedge \dif z
    \end{gather*}
\end{example*}

\begin{obs}
    La diferencial exterior $\dif \colon \Omega^\bullet \left( \real^n \right) \to \Omega^\bullet \left( \real^n \right)$ es una antiderivación de
    grado $+1$: es $\real$-lineal, aumenta el grado en 1 y sigue la regla de Leibniz ``graduada'':
    \[
        \dif \left( \alpha \wedge \beta \right) = \left( \dif \alpha \right) \wedge \beta + (-1)^{\abs{\alpha}} \alpha \wedge \dif \beta
    \]
\end{obs}

\begin{prop}\label{prop:d-cuadrado}
    $\dif \circ \dif = 0$
\end{prop}
\begin{proof}
    Basta demostrarlo para 
    \[
        \dif \left( \dif f \right) = \sum_{i,j} \frac{\partial}{\partial x^j} \left( \frac{\partial f}{\partial x^i} \right) \dif x^j \wedge \dif x^i = 0
    \]
    Aplicando el lema
    \[
        \begin{cases}
            \delta_{ij} = \delta_{ji} \\ A_{ij} = - A_{ji}
        \end{cases} \implies
        \sum_{i,j} \delta_{ij} A_{ij} = 0
    \]
\end{proof}

\begin{defi}
    Diremos que una forma diferencial $\alpha$ es cerrada si $\dif \alpha =0$. Diremos que una forma diferencial $\beta$ es exacta, si existe otra
    forma diferencial $\alpha$ tal que $\beta = \dif \alpha$
\end{defi}

\begin{obs*}
    La propiedad \ref{prop:d-cuadrado} significa que exacta $\implies$ cerrada.
\end{obs*}

\begin{prop}
    Sean $y^i$ $(1 \leq j \leq n$ funciones en $\real^n$. Tenemos
    \[
        \dif y^j = \sum^n_{i=1} \frac{\partial y^j}{\partial x^i} \dif x^i
    \]
    La aplicación $\psi = \left( y^1, \dots, y^n \right) \colon \real^n \to \real^n$ es un difeomorfismo alrededor de cada punto
    $\stackrel{\text{T.Función Inversa}}{\iff} J_\psi = \left( \frac{\partial y^j}{\partial x^i} \right)$ es invertible en cada punto.
    Y $\psi$ es un difeomorfismo, si es biyectiva y $J_\psi$ es invertible en todos los puntos.

    En estas condiciones, las diferenciales $\dif y^j$ tambi\'en son base para las 1-formas diferenciales. Observamos además que
    \[
        \dif y^1 \wedge \cdots \wedge \dif y^n = \det \left( \frac{\partial y^j}{\partial x^i} \right) \dif x^1 \wedge \cdots \wedge \dif x^n
    \]
\end{prop}

\begin{obs}
    Sea $g \colon \real^n \to \real$. Interpretamos $g$ como una función de las $y^j$, enotnces
    \[
        \sum_j \frac{\partial g}{\partial y^j} \dif y^j = \sum_{i, j} \frac{\partial g}{\partial y^j} \frac{\partial y^j}{\partial x^i} \dif x^i
        \substack{\text{regla} \\ = \text{cadena}} \sum_i \frac{\partial g}{\partial x^i} \dif x^i = \dif g
    \]

    Esta relación, demuestra que la diferencial exterior no depende del sistema de coordenadas utilizado.
\end{obs}

\begin{defi}
    Para cada $k \geq 1$ definimos una función $\real$-lineal
    \[
        \begin{aligned}
            K \colon \Omega^k\left( \real^n \right) &\to \Omega^{k-1}\left( \real^n \right) \\
            f \dif x^{i_!} \wedge \cdots \wedge \dif x^{i_k} &\mapsto \left. \sum^k_{r=1} (-1)^r \left( \int^1_0 t^{k-1} f(tp) \dif t\right) x^{i_r}
            \dif x^{i_1} \wedge \cdots \wedge \widehat{\dif x^{i_r}} \wedge \cdots \wedge \dif x^{i_n} \right\vert_p
        \end{aligned}
    \]
    donde $\widehat{\dif x^{i_r}}$ significa que ese t\'ermino no está presente.
\end{defi}

\begin{lema}
    Para toda forma diferencial $\omega$ de grado $\geq 1$ en $\real^n$
    \[
        \omega = K \dif \omega + \dif K \omega
    \]
\end{lema}

\begin{teo}[lema de Poincar\'e]
    Toda forma diferencial en $\real^n$ cerrada de grado $\geq 1$ es exacta.
\end{teo}
\begin{proof}
    \[
        \omega = \cancel{K\dif \omega} + \dif K \omega
    \]
\end{proof}

\begin{example*}
    En $\real^2$, podemos considerar el cambio de coordenadas de ($x, y$) a ($r, \phi$), entonces
    \begin{gather*}
        x = r \cos \phi \qquad y r \sin \phi \\
        \dif x = \cos \phi \dif r - r \sin \phi \dif \phi \\
        \dif y = \sin \phi \dif r + r \cos \phi \dif \phi \\
        \pdv{(r, \phi)}{(x,y)} =
        \begin{pmatrix}
            \cos \phi & - r \sin \phi \\ \sin \phi & r \cos \phi
        \end{pmatrix}
    \end{gather*}
    Consideramos ahora la función
    \[
        \begin{aligned}
            f \colon \real^2 &\to \real \\
            (x,y) &\mapsto f(x,y) = x^2 + y^2 = r^2
        \end{aligned}
    \]
    Y se tiene que
    \begin{gather*}
        \dif f = 2x \dif x + 2y \dif y \qquad \dif f = 2 r \dif r \\
        \dif x \wedge \dif y = \det
        \begin{pmatrix}
            \cos \phi & - r \sin \phi \\
            \sin \phi &   r \cos \phi
        \end{pmatrix} \dif r \wedge \dif \phi = r \dif r \wedge \dif \phi
    \end{gather*}
\end{example*}

\section{Pullback de formas diferenciales en $\real^n$}<++>

\section{Variedades con borde y teorema de Stokes}\label{section:5-8}
