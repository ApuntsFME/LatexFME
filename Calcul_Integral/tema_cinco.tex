\chapter{Integración de formas diferenciales y teorema de Stokes}

En este tema, consideraremos, salvo mención contraria, que todas las funciones son de clase $\C^\infty$

\section{Formas diferenciales en $\real^n$}

\begin{defi}
    Definiremos $\Omega^1$ como el espacio vectorial generado por $\dif x^1, \dots, \dif x^n$, que es un espacio vectorial real de dimensión $n$.
    Denotaremos como $\Omega^\bullet$ el álgebra exterior de $\Omega^1$. Su poducto está regido por
    \[
        \dif x^i \wedge \dif x^i = 0
        \qquad
        \dif x^i \wedge \dif x^j = -\dif x^j \wedge x^i \quad (i \neq j)
    \]
    Así pues, tenemos $\Omega^\bullet = \bigoplus\limits^{n}_{r = 0} \Omega^r$, donde estos subespacios tienen dimensión y bases
    \begin{itemize}
        \item $\dim \Omega^0 = 1$, base 1
        \item $\dim \Omega^1 = n$, base $\left( \dif x^i \right)_{1 \leq i \leq n}$
        \item $\dim \Omega^2 = \binom{n}{2}$, base $\left( \dif x^i \wedge \dif x^j \right)_{i < j}$
        \item $\vdots$
        \item $\dim \Omega^{n-1} = n$, base $\left( \dif x^1 \wedge \cdots \wedge \dif x^{i-1} \wedge \dif x^{i+1} \wedge \cdots \wedge \dif x^n \right)_i$
        \item $\dim \Omega^n = 1$, base $\dif x^1 \wedge \cdots \wedge \dif x^n$.
    \end{itemize}

    En general, se tiene que $\Omega^k$ tiene dimensión $\binom{n}{k}$, y una base es $\left( \dif x^I \right)$ donde
    \[
        \dif x^I \equiv \dif x^{i_1} \wedge \cdots \wedge \dif x^{i_k}
    \]
    e $I = \left( i_1, \dots, i_k \right)$ es un multiíndice de longitud $k$ estrictamente creciente. Así pues, $\Omega^\bullet = 2^n$.
\end{defi}

\begin{defi}
    Denominamos formas diferenciales en $\real^n$ a los elementos del conjunto $\Omega^\bullet\left( \real^n \right) = \C^\infty\left( \real^n \right) \otimes
    \Omega^\bullet$, es decir, expresiones de la forma
    \[
        \omega = \sum_I f_I \dif x^I
    \]
    Donde $f_I \in \C^{\infty}\left( \real^n \right)$ e $I$ es un multiíndice. Tambi\'en tenemos la gradución de $\Omega^\bullet \left( \real^n \right)$
    \[
        \Omega^\bullet \left( \real^n \right) = \bigoplus^n_{k=0} \Omega^k \left( \real^n \right) \qquad \text{donde}
        \Omega^k \left( \real^n \right) = \C^\infty \left( \real^n \right) \otimes \Omega^k
    \]
    son las formas diferenciales de grado $k$ ($k$-forma diferencial). Así, podemos escribir
    \[
        \omega = \omega_0 + \omega_1 + \cdots + \omega_n
    \]

    Los elementos de $\Omega^n\left( \real^n \right)$ se denominan por abuso del lenguaje, ``formas diferenciales de grado máximo''. Si
    $\omega \in \Omega^k \left( \real^n \right)$, escribiremos $\abs{\omega} := k$.
\end{defi}

\begin{defi}
    Las operaciones algebraicas de $\Omega^\bullet$ se trasladan a operaciones algebraicas de $\Omega^\bullet \left( \real^n \right)$. En particular,
    el producto exterior de dos formas diferenciales se calcula según la distributividad respecto de la suma, y con la regla
    \[
        \left( f \dif x^{i_1} \wedge \cdots \wedge \dif x^{i_k} \right) \wedge \left( g \dif x^{j_1} \wedge \cdots \wedge \dif x^{j_l} \right) =
        fg \dif x^{i_1} \wedge \cdots \wedge \dif x^{i_k} \wedge \dif x^{j_1} \wedge \cdots \wedge x^{j_l}
    \]
\end{defi}

\begin{prop}
    El producto exterior en $\Omega^\bullet \left( \real^n \right)$ tiene elemento neutro (la función constante igual a 1), es asociativo
    y anticonmutativo: si $\alpha$ y $\beta$ son formas diferenciales de grados $\abs{\alpha}$ y $\abs{\beta}$, entonces
    \[
        \beta \wedge \alpha = (-1)^{\abs{\alpha}\abs{\beta}} \alpha \wedge \beta
    \]
\end{prop}

\section{La diferencial exterior en $\real^n$}

\begin{defi}
   Dada una función $f \in \C^\infty \left( \real^n \right) = \Omega^0 \left( \real^n \right)$, podemos construir una 1-forma diferencial
   $\dif f \in \Omega^1 \left( \real^n \right)$, de acuerdo con la fórmula
   \[
       \dif f := \sum^n_{i=1} \frac{\partial f}{\partial x^i} \dif x^i
   \]
   donde $\frac{\partial f}{\partial x^i}$ representa la derivada parcial $\Dif_i f$ de $f$ respecto a la variable $i$-\'esima.
\end{defi}

\begin{obs}
    Observamos que si $f \left( a^1, \dots, a^n \right) = a^i$, entonces $\dif f = \dif x^i$. Está función se conoce como la coordenada
    cartesiana $i$-\'esima y es habitual denotarla por $x^i$
\end{obs}

\begin{defi}
    La diferencial actuando sobre funciones se puede extender a una aplicación $\real$-lineal, denominada \emph{diferencial exterior} y definida
    de la manera siguiente
    \[
        \begin{aligned}
            \dif \colon \Omega^\bullet \left( \real^n \right) &\to \Omega^\bullet \left( \real^n \right)
            f \dif x^I &\mapsto \dif\left( f \dif x^I \right) := \dif f \wedge \dif x^I
        \end{aligned}
    \]
\end{defi}

\begin{obs*}
    Si $\omega \in \Omega^k \left( \real^n \right)$, entonces, $\dif \omega \in \Omega^{k+1} \left( \real^n \right)$
\end{obs*}

\begin{example*}
    En $\real^2$, se tiene que
    \begin{gather*}
        \dif f = \pdv{f}{x} \dif x + \pdv{f}{y} \dif y \\
        \dif \left( f \dif x \wedge \dif y \right) = 0 \\
        \dif\left( f \dif x + g \dif y \right) = \dif f \wedge \dif x + \dif g \wedge \dif y =\\=
        \left( \cancel{\pdv{f}{x} \dif x} + \pdv{f}{y} \dif y \right) \wedge \dif x + \left( \pdv{g}{x} \dif x + \cancel{\pdv{g}{y} \dif y} \right)\wedge \dif y =
        \left( -\pdv{f}{y} + \pdv{g}{x} \right) \dif x \wedge \dif y
    \end{gather*}
    En $\real^3$, se tiene que
    \begin{gather*}
        \dif f = \pdv{f}{x} \dif x + \pdv{f}{y} \dif y + \pdv{f}{z} \dif z \\
        \dif \left( f_1 \dif x + f_2 \dif y + f_3 \dif z \right) = \cdots = \\ = \left( \pdv{f_3}{y} - \pdv{f_2}{x} \right)\dif y \wedge \dif z +
        \left( \pdv{f_1}{z} - \pdv{f_3}{x} \right) \dif z \wedge \dif x + \left( \pdv{f_2}{x} - \pdv{f_1}{y} \right)\dif x \wedge \dif y \\
        \dif \left( g_1 \dif y \wedge \dif z +  g_2 \dif z \wedge \dif x + g_3 \dif x \wedge \dif y \right) = \cdots =
        \left( \pdv{g1}{x} + \pdv{g_2}{y} + \pdv{g_3}{z} \right) \dif x \wedge \dif y \wedge \dif z
    \end{gather*}
\end{example*}

\begin{obs}
    La diferencial exterior $\dif \colon \Omega^\bullet \left( \real^n \right) \to \Omega^\bullet \left( \real^n \right)$ es una antiderivación de
    grado $+1$: es $\real$-lineal, aumenta el grado en 1 y sigue la regla de Leibniz ``graduada'':
    \[
        \dif \left( \alpha \wedge \beta \right) = \left( \dif \alpha \right) \wedge \beta + (-1)^{\abs{\alpha}} \alpha \wedge \dif \beta
    \]
\end{obs}

\begin{prop}\label{prop:d-cuadrado}
    $\dif \circ \dif = 0$
\end{prop}
\begin{proof}
    Basta demostrarlo para
    \[
        \dif \left( \dif f \right) = \sum_{i,j} \frac{\partial}{\partial x^j} \left( \frac{\partial f}{\partial x^i} \right) \dif x^j \wedge \dif x^i = 0
    \]
    Aplicando el lema
    \[
        \begin{cases}
            \delta_{ij} = \delta_{ji} \\ A_{ij} = - A_{ji}
        \end{cases} \implies
        \sum_{i,j} \delta_{ij} A_{ij} = 0
    \]
\end{proof}

\begin{defi}
    Diremos que una forma diferencial $\alpha$ es cerrada si $\dif \alpha =0$. Diremos que una forma diferencial $\beta$ es exacta, si existe otra
    forma diferencial $\alpha$ tal que $\beta = \dif \alpha$
\end{defi}

\begin{obs*}
    La propiedad \ref{prop:d-cuadrado} significa que exacta $\implies$ cerrada.
\end{obs*}

\begin{prop}
    Sean $y^i$ $(1 \leq j \leq n$ funciones en $\real^n$. Tenemos
    \[
        \dif y^j = \sum^n_{i=1} \frac{\partial y^j}{\partial x^i} \dif x^i
    \]
    La aplicación $\psi = \left( y^1, \dots, y^n \right) \colon \real^n \to \real^n$ es un difeomorfismo alrededor de cada punto
    $\stackrel{\text{T.Función Inversa}}{\iff} J_\psi = \left( \frac{\partial y^j}{\partial x^i} \right)$ es invertible en cada punto.
    Y $\psi$ es un difeomorfismo, si es biyectiva y $J_\psi$ es invertible en todos los puntos.

    En estas condiciones, las diferenciales $\dif y^j$ tambi\'en son base para las 1-formas diferenciales. Observamos además que
    \[
        \dif y^1 \wedge \cdots \wedge \dif y^n = \det \left( \frac{\partial y^j}{\partial x^i} \right) \dif x^1 \wedge \cdots \wedge \dif x^n
    \]
\end{prop}

\begin{obs}
    Sea $g \colon \real^n \to \real$. Interpretamos $g$ como una función de las $y^j$, enotnces
    \[
        \sum_j \frac{\partial g}{\partial y^j} \dif y^j = \sum_{i, j} \frac{\partial g}{\partial y^j} \frac{\partial y^j}{\partial x^i} \dif x^i
        \substack{\text{regla} \\ = \text{cadena}} \sum_i \frac{\partial g}{\partial x^i} \dif x^i = \dif g
    \]

    Esta relación, demuestra que la diferencial exterior no depende del sistema de coordenadas utilizado.
\end{obs}

\begin{defi}
    Para cada $k \geq 1$ definimos una función $\real$-lineal
    \[
        \begin{aligned}
            K \colon \Omega^k\left( \real^n \right) &\to \Omega^{k-1}\left( \real^n \right) \\
            f \dif x^{i_!} \wedge \cdots \wedge \dif x^{i_k} &\mapsto \left. \sum^k_{r=1} (-1)^r \left( \int^1_0 t^{k-1} f(tp) \dif t\right) x^{i_r}
            \dif x^{i_1} \wedge \cdots \wedge \widehat{\dif x^{i_r}} \wedge \cdots \wedge \dif x^{i_n} \right\vert_p
        \end{aligned}
    \]
    donde $\widehat{\dif x^{i_r}}$ significa que ese t\'ermino no está presente.
\end{defi}

\begin{lema}
    Para toda forma diferencial $\omega$ de grado $\geq 1$ en $\real^n$
    \[
        \omega = K \dif \omega + \dif K \omega
    \]
\end{lema}

\begin{teo}[lema de Poincar\'e]
    Toda forma diferencial en $\real^n$ cerrada de grado $\geq 1$ es exacta.
\end{teo}
\begin{proof}
    \[
        \omega = \cancel{K\dif \omega} + \dif K \omega
    \]
\end{proof}

\begin{example*}
    En $\real^2$, podemos considerar el cambio de coordenadas de ($x, y$) a ($r, \phi$), entonces
    \begin{gather*}
        x = r \cos \phi \qquad y r \sin \phi \\
        \dif x = \cos \phi \dif r - r \sin \phi \dif \phi \\
        \dif y = \sin \phi \dif r + r \cos \phi \dif \phi \\
        \pdv{(r, \phi)}{(x,y)} =
        \begin{pmatrix}
            \cos \phi & - r \sin \phi \\ \sin \phi & r \cos \phi
        \end{pmatrix}
    \end{gather*}
    Consideramos ahora la función
    \[
        \begin{aligned}
            f \colon \real^2 &\to \real \\
            (x,y) &\mapsto f(x,y) = x^2 + y^2 = r^2
        \end{aligned}
    \]
    Y se tiene que
    \begin{gather*}
        \dif f = 2x \dif x + 2y \dif y \qquad \dif f = 2 r \dif r \\
        \dif x \wedge \dif y = \det
        \begin{pmatrix}
            \cos \phi & - r \sin \phi \\
            \sin \phi &   r \cos \phi
        \end{pmatrix} \dif r \wedge \dif \phi = r \dif r \wedge \dif \phi
    \end{gather*}
\end{example*}

\section{Pullback de formas diferenciales en $\real^n$}

\begin{defi}
    Sea $F \colon \real^m \to \real^n$ una aplicación de clase $\C^\infty$. Dada una función $g \in \C^\infty\left( \real^n \right)$, la función
    \[
        F^\ast(g) := g \circ F \in \C^\infty \colon \real^m \to \real
    \]
    se denomina \emph{pullback} de $g$ por $F$. Así, tenemos una aplicación $\real$-lineal $F^\ast \colon \C^\infty \left( \real^n \right)
    \to \C^\infty \left( \real^n \right)$
\end{defi}
\begin{obs*}
    Es similar a a aplicación transponer de álgebra lineal
\end{obs*}

\begin{defi}
    Podemos extender el pullback a formas diferenciales, definiendo una aplicación $F^\ast \colon \Omega^\bullet \left( \real^n \right) \to
    \Omega^\bullet\left( \real^m \right)$ denominada pullback y definida por las propiedaddes siguientes
    \begin{itemize}
        \item La aplicación es $\real$-lineal.
        \item Sobre $\Omega^0\left( \real^n \right)$ es el pullback de funciones.
        \item La aplicación conmuta con el producto exterior
            \[
                F^\ast(\alpha \wedge \beta) = F^\ast(\alpha) \wedge F^\ast(\beta)
            \]
        \item Si $g \in \Omega^0 \left( \real^n \right)$, entonces $F^\ast\left( \dif g \right) = \dif F^\ast(g)$
    \end{itemize}
    Por lo tanto, $F^\ast$ aplica de $k$-formas en $k$-formas y $F^\ast$ conmuta con la difencial exterior
    \[
        \dif \circ F^\ast = F^\ast \circ \dif
    \]
    $F^\ast$ se define poniendo
    \[
        F^\ast\left( g \dif y^{j_1} \wedge \cdots \wedge \dif y^{j_l} \right) = F^\ast(g)\dif F^\ast\left( y^{j_1} \right) \wedge
        \cdots \wedge \dif F^\ast\left( y^{j_l} \right)
    \]
\end{defi}

\begin{obs}
    En particular, tenemos
    \[
        F^\ast \left( \dif y^j \right) = \dif F^\ast\left( y^j \right) =
        \sum_i = \frac{\partial F^\ast\left( y^j \right)}{\partial x^i} \dif x^i
    \]
    Donde podemos observar la apración de la jacobiana de $F$
    \[
        J_F = \left( \frac{\partial F^\ast \left( y^j \right)}{\partial x^i} \right)
    \]
\end{obs}

\begin{obs}
    En el caso particular de que $m=n$, tenemos que
    \[
        F^\ast \left( \dif y^1 \wedge \cdots \wedge \dif y^n \right) = \det J_F \dif x^1 \wedge \cdots \wedge \dif x^n
    \]
\end{obs}

\begin{prop}
    Cuando $F \colon \real^m \to \real^n$ es un difeomorfismo (necesita $m=n$), entonces las diferenciales $\dif F^\ast \left( y^j \right)$
    tambi\'en una base para las formas diferenciales de grado 1 en $\real^m$, y la matriz de cambio de base respecto a la base de las
    $\dif x^i$ es la matriz jacobiana $J_F$.
\end{prop}

\begin{example*}
    Consideramos la función
    \[
        \begin{aligned}
            F \colon \real^2 &\to \real^2 \\
            (u,v) &\mapsto F(u,v) = (u^2 - v^2, 2uv)
        \end{aligned}
    \]
    Entonces,
    \begin{gather*}
        F^\ast(x) = u^2 - v^2 \qquad F^\ast(y) = 2uv \\
        F^\ast \left( x^2 + y^2 \right) = \left(u^2 - v^2\right)^2 + (2uv)^2 = \left(u^2 +v^2\right)^2 \\
        F^\ast\left( \dif x \right) = \dif F^\ast(x) = 2u\dif u - 2v\dif v
        \qquad
        F^\ast\left( \dif y \right) = \dif F^\ast (y) = 2v \dif u + 2u \dif v \\
        F^\ast\left( \dif x \wedge \dif y \right) = F^\ast\left( \dif x \right) \wedge F^\ast\left( \dif y \right) = \cdots =
        4 \left( u^2 + v^2 \right) \dif u \wedge \dif v \\
        F^\ast\left( -y\dif x + x \dif y\right) = \cdots = 2 \left( u^2 + v^2 \right) \left( -v \dif u + u \dif v \right)
    \end{gather*}
\end{example*}

\section{Integral de formas diferenciales}

\begin{obs}
    REcordemos que si $f \colon \real^n \to \real$ es una función continua de clase $\C^\infty$, está definida la integral
    \[
        \int_{\real^n} f \equiv \int_{\real^n} f(x) \dif x^1 \cdots \dif x^n
    \]
\end{obs}

\begin{defi}
    Sea $\omega \in \Omega^n\left( \real^n \right)$ una $n$-forma diferencial escrita como $\omega = f \dif x^1 \wedge \cdots \wedge \dif x^n$,
    con $f$ de clase $\C^\infty$. Si $f$ tiene soporte compacto, se define su integral como
    \[
        \int_{\real^n} \omega := \int_{\real^n} f\dif x^1 \wedge \cdots \wedge \dif x^n = \int_{\real^n} f \dif x^1 \cdots \dif x^n
    \]
\end{defi}

\begin{prop}
    Sea $F \colon \real^n \to \real^n$ un difeomorfismo y $\omega \in \Omega^n\left( \real^n \right)$ con soporte compacto, entonces
    \[
        \int_{\real^n} F^\ast(\omega) = \pm \int_{\real^n} \omega
    \]
    donde el signo es el de $J_F$
\end{prop}

\begin{proof}
    Por un lado, tenemos que
    \[
        \int_{\real^n} \omega = \int_{\real^n} f(x) \dif x^1 \cdots \dif x^n \stackrel{F \text{ C.V.}}{=} \int_{\real^n}
        f\left( F(y) \right) \abs{\det J_F(y)} \dif y^1 \cdots \dif y^n
    \]
    Por otro lado,
    \[
        \int_{\real^n} F^\ast(\omega) = \int_{\real^n} F^\ast (f) \det J_F \dif y^1 \wedge \cdots \dif y^n =
        \int_{\real^n} f\left( F(y) \right) \det J_F(y) \dif y^1 \cdots \dif y^n
    \]
\end{proof}

\begin{obs}
    Si interpretamos $F$ como un cambio de variables, esta fórmula, demuestra que la definición dada de integral de una forma diferencial
    de grado máximo \textbf{NO} depende del sistema de coordenadas utilizado, aunque cambia de signo dependiendo de la orientación.
\end{obs}

\begin{defi}
    Sea $\omega \in \Omega^k\left( \real^n \right)$ una forma diferencial de grado $k < n$. Si $\sigma \colon \real^k \to \real^n$ es
    una aplicación, se define la integral de $\omega$ a lo largo de $\sigma$ como la integral
    \[
        \int_{\sigma} \omega := \int_{\real^k} \sigma^\ast(\omega)
    \]
    suponiendo que $\sigma^\ast(\omega)$ tenga soporte compacto.
\end{defi}

\begin{obs}
    Todo lo que hemos hecho, se puede hacer en abiertos de $\real^n$ en lugar de en todo $\real^n$. Si $V \subseteq \real^n$ abierto,
    $\Omega^\bullet(V)$, $\Omega^k(V)$, $\omega = \sum\limits_I g_I \dif x^I$ $\left( g_I \in \C^\infty(V) \right)$.
    Pullback de $F \colon U \subseteq \real^m \to V \subseteq \real^n$, $F^\ast \colon \Omega^\bullet (V) \to \Omega^\bullet (U)$.

    Para el lema de Poincar\'e es necesario que $V \subset \real^n$ sea estrellado respecto al origen.
\end{obs}

\begin{defi}
    Decimos que una forma diferencial es de clase $\C^k$ cuando sus coeficientes son de clase $\C^k$, en los apartados anteriores hemos
    considerado diferenciabilidad de clase $\C^\infty$, pero algunas operaciones con formas diferenciales no requieren tanta diferenciabilidad.
\end{defi}

\section{Subvariedades de $\real^n$}

Este tema es un repaso de cálculo diferencial, así que referenciamos a los apuntees del profesor
ante cualquier duda, este apartado se trata de un simple repaso de las propiedadaes basicas de las
subvariedades

\section{Formas diferenciales en una variedad}

\begin{obs*}
    No trataremos el tema de los campos vectoriales en subvariedades, pero sí que aparecen en el resumen de apuntes. Como siempre, los curiosos
    a los apuntes.
\end{obs*}

\setcounter{lema}{4}

\begin{defi}
    Sea $M$ una subvariedad, llamaremos 1-forma diferencial en $M$ a una aplicación
    \[
        \begin{aligned}
            \theta \colon M &\to T^\ast M \equiv \bigcup_{p \in M} T^\ast_pM \\
            p &\mapsto \theta_p \equiv \theta(p) \in T^\ast_p M
        \end{aligned}
    \]
\end{defi}

\begin{obs}
    Si $f \colon M \to \real$, $\dif f \colon M \to T^\ast M$ que envia cada punto $p \in M$ a la difencial $\dif f(p)$.

    En particular, en el dominio de una carta, tenemos las diferenciales de las funciones de las funciones coordenadas, $\dif x^i$.
\end{obs}

\begin{defi}
    En el dominio $V$ de una carta, una 1-forma diferencial $\theta$ se expresa de manera única como
    \[
        \theta \vert_V = \sum^m_{i=!} a_i \dif x^i
    \]
    donde $a_i \colon V \to \real$ son los componenetes de $\theta$ en la carta.

    Diremos que $\theta$ es de clase $\C^\infty$ cunado sus componentes son de clase $\C^\infty$.
\end{defi}

\begin{prop}
    En el dominio de una carta, la diferencial de una función se expresa como
    \[
        \dif f \vert_V = \sum^m_{i= 1} \frac{\partial f}{\partial x^i} \dif x^i
    \]
\end{prop}

\begin{defi}
    Definiremos una $k$-forma diferencial como una expresión del tipo
    \[
        \omega\vert_V = \sum_{i_1 < \cdots < i_k} f_{i_1 \cdots i_k} \dif x^{i_1} \wedge \cdots \wedge \dif x^{i_k}
    \]
\end{defi}

El resto de este tema tampoco se ha realizado en clase, ante cualquier duda o curiosidad, consultar los apuntes del profesor.

\section{Integración de formas diferenciales en variedades orientadas}

\begin{defi}
    Una parametrización $g \colon W \to M$ de una variedad define en cada espacio tangente $T_pM$ una base de vectores tangentes
    coordenados, y, por tanto, una orientación de $T_pM$.

    Otra parametrización $f \colon V \to M$ de $M$ define la misma orientación de los espacios tangentes sii el cambio de coordenadas
    $\inv{f} \circ g \colon W \to V$ tiene jacobiana positiva.
\end{defi}

\begin{defi}
    Un atlas orientado de $M$ es un conjunto de cartas que recubren $M$ y tal que sus cambios de coordenadas (en sus dominios comunes)
    tienen jacobianas positivas. Un atlas orienta los diferentes espacios tangentes de $M$ de manera coherente. En caso de existir dicho atlas,
    se dice que $M$ es orientable.
\end{defi}

\begin{defi}
    Una orientación de una variedad orientable viene dada por la elleción de un atlas orientado. Una variedad orientada es una variedad
    orientable en la cual se ha elegido una orientación.
\end{defi}

\begin{defi}
    Si $M$ es una variedad orientada, diremos que una carta de $M$ pertenece a la orientación de $M$ o que es una carta positiva cuando su
    cambio de coordenadas respecto a las cartas de la orientación de $M$ tiene jacobiana positiva.
\end{defi}

\begin{defi}
    Una forma diferencial $\Omega$ en $M$ de grado máximo que no se anula en ningún punto, se denomina forma de volumen
\end{defi}

\begin{prop*}
    Si una variedad tiene una forma de volumen, entonces es orientable.
\end{prop*}

\begin{proof}
    En cada punto $p \in M$ la forma diferencial $\Omega_p$ define una orientación del espcaio tangente $T_pM$:
    una base $\left\{ \vec{v_1}, \dots, \vec{v_n} \right\}$ de $T_pM$ es positiva cuando
    $\Omega_p\left( \vec{v_1}, \dots, \vec{v_n} \right) > 0$. De manera análoga, una carta $\varphi = \left( x^1, \dots, x^n \right)$
    es positiva cuando
    \[
        \Omega\left( \frac{\partial}{\partial x^1}, \dots, \frac{\partial}{\partial x^m} \right) > 0
    \]
    Las cartas positivas constituyen un atlas orientado en $M$, y por tanto, definen na orientación.
\end{proof}

\begin{obs*}
    Reciprocamente, se puede demostrar que si una variedad tiene un atlas orientado, entonces tiene una forma de volumen.
\end{obs*}

\begin{obs}
    Una variedad orientada conexa tiene exactamente 2 orientaciones.
\end{obs}

\begin{defi}
    Sea $M \subset \real^n$ una subvariedad $m$-dimensional orientada, $\omega \in \Omega^m\left( M \right)$ una forma diferencial
    de grado máximo con soporte compacto. Si $M$ admite una parametrización global $g \colon W \subset \real^n \to M$ correspondiente
    a la orientación de $M$, se define la integral de $\omega$ como
    \[
        \int_M \omega := \int_W g^\ast(\omega)
    \]
\end{defi}

\begin{lema*}
    La integral anterior no depende de la parametrización escogida.
\end{lema*}

\begin{proof}
    Sea $g^\prime \colon W^\prime \to M$ otra parametrización y sea $X \colon W^\prime \to W$ un difeomorfismo, entonces
    \[
        \int_{W^\prime} {g^\prime}^\ast(\omega) = \int_{W^\prime} X^\ast\left( g^\ast(\omega) \right) = \pm \int_W g^\ast(\omega)
    \]
    donde el signo es el signo de la jacobiana de $X$.
\end{proof}

\begin{obs}
    Si $M$ requiere diversas parametrizaciones, se requiere un instrumento t\'ecnico para definir la integral de una forma diferencial,
    si alguno está interesado puede consultar en los apuntes del profesor (en el mismo índice).
\end{obs}

\begin{defi}
    Sea $\phi_\alpha \colon V_\alpha \to M$ una familia de cartas positivas que recubren todo $M$ y $\left( \chi_\alpha \right)$ es una
    partición de la unidad subordinada, tenemos pues que $\omega = \sum\limits_\alpha \chi_\alpha \omega$, lo cual nos permite definir
    \[
        \int_M \omega = \sum_\alpha \int_M \chi_\alpha \omega
    \]
    Donde $\int_M \chi_\alpha \omega$ se debe entender como la integral sobre los abiertos $V_\alpha$ ya que $\chi_\alpha$ fuera de ellos vale
    0.
\end{defi}

\begin{obs}
    Como hemos supuesto que $\omega$ tiene soporte compacto, la suma es una suma finita, y, en general, $M$ se puede expresar como
    \[
        M = \underbrace{\left( \bigcup_\alpha V_\alpha \right)}_{\text{finito}} \cup \underbrace{Z}_{\text{medida nula}}
    \]
    y entonces,
    \[
        \int_M \omega = \sum_\alpha \int_{V_\alpha} \omega
    \]
\end{obs}

\section{Variedades con borde y teorema de Stokes}\label{section:5-8}

\begin{obs}
    El modelo local de variedad con borde es el semiespacio cerrado $H^m$,
    \[
        H^m = \setb{\left( x^1, \dots, x^m \right) \in \real^m \vert x^1 \leq 0} = (-\infty, 0] \times \real^{m-1}
    \]
    Formado por $(-\infty, 0) \times \real^{m-1}$, el interior de $H^m$ y su frontera $\{0\} \times \real^{m-1}$.
\end{obs}

\begin{defi}
    Una subvariedad con borde de dimensión $m$ en $\real^n$ es un subconjunto $M \subset \real^n$ que satisface
    la propiedad:

    para cada $p \in M$ existe un conjunto abierto $V \subset \real^n$, con $p \in V$, y un difeomorfismo
    $\Phi \colon V \to U$ entre abiertos de $\real^n$, tal que
    \[
        \Phi\left( V \cap M \right) = U \cap \left( H^m \times \{0\}\right)
    \]
\end{defi}

\begin{defi}
    Para un punto $p \in M$ una subvariedad, y un difeomorfismo $\Phi$ como el de antes, hay dos posiblidades,
    \begin{itemize}
        \item $\Phi(p) \in \text{Int}\left( H^m \right) \times \{0\}$
        \item $\Phi(p) \in \fr \left( H^m \right) \times \{0\}$
    \end{itemize}
    Que ocurra una o la otra, no depende del difeomorfismo utilizado. A los puntos del primer tipo los llamaremos
    ``puntos interiores'' de $M$. Los puntos del segundo tipo, forman el borde de $M$, que se representa por $\partial M$.
\end{defi}

\begin{prop}
    Si $M$ es una subvariedad con borde, $M = \left( M \smallsetminus \partial M \right) \cup \partial M$ donde,
    $\left( M \smallsetminus \partial M \right)$ es una subvariedad de dimensión $m$ y $\partial M$ es una subvariedad
    de dimensión $m-1$.
\end{prop}

Es tanto interesante como recomendable, consultar los apuntes del profesor de la parte que falta.

\setcounter{lema}{7}

\begin{prop}
    Si una subvariedad $M$ $m$-dimensional está orientada, entonces $\partial M$ tiene una orientación canónica (``hacia el
    exterior''), definida de la marea siguiente:

    Si $\left( x^1, \dots, x^n \right)$ es una carta positiva de $M$, definida alrededor de un punto $p$ del borde según la definición
    del principio (es decir, $x^1 \leq 0$), entonces $\left( x^2, \dots, x^m \right)$ es una carta positiva del borde.
\end{prop}

\begin{teo}[de Stokes]
    Sean $M \subset \real^n$ una variedad con borde, orientada y de dimensión $m$, y $\omega \in \Omega^{m-1}(M)$ una forma
    diferencial de grado $m-1$ en $M$ de soporte compacto. Entonces, se satisface la fórmula de \emph{Stokes}
    \[
        \int_M \dif \omega = \int_{\partial M} \omega
    \]
    donde el borde tiene la orientación inducida por $M$.
\end{teo}

Es interesante, aunque no se ha hecho en clase, consultar los apuntes del profesor para ver cómo se aplican las diversas fórmulas
y teoremas vistas durante los temas anteriores a subvariedades con bordes.
