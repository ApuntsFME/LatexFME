\chapter{Teoremas integrales del análisis vectorial}

\section{Operadores diferenciales canónicos en $\real^3$}

\begin{obs}
    Sea $U \subset \real^3$ un abierto no vacío. Denotaremos por
    \begin{gather*}
        \Es^{(k)} (U) := \setb{\text{campos escalares } \C^k \text{ sobre } U}, \\ 
        \V^{(k)}(U) := \setb{\text{campos vectoriales } \C^k \text{ sobre } U}.
    \end{gather*}
    Si no se especifica el grado de diferenciabilidad, se entederá que es el que sea necesario para la realización
    de cálculos.

    Así pues, un campo vectorial $\vec{F} \in \V^{\left( k \right)}(U)$, se identifica con una función vectorial
    $\vec{F} \colon U \to \real^3$, aunque sería más preciso escribirlo por componentes. Si representamos
    la base canónica de $\real^3$ por $\setb{e_1, e_2, e_3}$, entonces $\vec{F} = f_1e_1 + f_2e_2 + f_3e_3$.
    Aunque tambi\'en es habitual representarla por $\setb{\hat{i}, \hat{j}, \hat{k}}$.
\end{obs}

\begin{defi}
    En $\real^3$ hay tres operadores diferenciales lineales de primer orden canónicos.
    \begin{itemize}
 
        \item \emph{Gradiente} $\grad \colon \Es^{(1)}(U) \to \V^{0}(U)$, que en coordenadas cartesianas
            se expresa como
            \[
                \grad f := \pdv{f}{x} \hat{i} + \pdv{f}{y} \hat{j} + \pdv{f}{z} \hat{k}.
            \]
        \item \emph{Rotacional} $\rot \colon \V^{(1)} (U) \to \V^{(0)}(U)$, que en coordenadas cartesianas
            se expresa como
            \[
                \rot \vec{F} := \left( \pdv{F_3}{y} - \pdv{F_2}{z} \right)\hat{i} +
                \left( \pdv{F_1}{z} - \pdv{F_3}{x} \right) \hat{j} + \left( \pdv{F_2}{x} - \pdv{F_1}{y} \right) \hat{k}.
            \]
        \item \emph{Divergencia} $\diver \colon \V^{(1)} (U) \to \Es^{(0)} (U)$, que en coordenadas cartesianas
            se expresa como
            \[
                \diver \vec{F} := \pdv{F_1}{x} + \pdv{F_2}{y} + \pdv{F_3}{z}.
            \]
    \end{itemize}
\end{defi}

\begin{obs*}
    Si los campos son de clase $\C^k$ el resultado es de clase $\C^{k-1}$.
\end{obs*}

\begin{obs*}
    Las definiciones de gradiente y de divergnecia se pueden aplicar sin casi ningún cambio a $\real^n$.
\end{obs*}

\begin{obs}
    Recordemos que el gradiente tiene una relación directa con la diferencial a trav\'es del producto escalar
    de $\real^n$: si $\Dif_f(p) \colon \real^n \to \real$ es la diferencial de $f$ en $p$, $\grad f(p)$ es el
    vector tal que, $\forall \vec{u} \in \real^n$, $\left( \grad f(p) \vert u \right) = \Dif_f(p) \cdot u$.
\end{obs}

\begin{defi}
    Definimos es operador \emph{nabla} como
    \[
        \vec{\nabla} := \hat{i} \pdv{}{x} + \hat{j} \pdv{}{y} + \hat{k} \pdv{}{z}.
    \]

    Entonces
    \begin{gather*}
        \grad f = \vec{\nabla} f := \left( \hat{i} \pdv{}{x} + \hat{j} \pdv{}{y} + \hat{k} \pdv{}{z} \right)f,
        \\
        \rot \vec{F} = \vec{\nabla} \times \vec{F} := \determinant{
            \hat{i} & \pdv{}{x} & f_1 \\
            \hat{j} & \pdv{}{y} & f_2 \\
            \hat{k} & \pdv{}{z} & f_3
        },
        \\
        \diver \vec{F} = \vec{\nabla} \cdot \vec{F} := \left( \hat{i} \pdv{}{x} + \hat{j} \pdv{}{y} + \hat{k} \pdv{}{z} \right) \cdot \vec{F}.
    \end{gather*}
\end{defi}

\begin{prop}
    $\grad$, $\rot$ y $\diver$ son lineales, es decir
    \begin{gather*}
        \grad(f+g) = \grad f + \grad g, \\
        \grad(cf) = c\grad f.
    \end{gather*}
    (Análogamente con $\rot$ y $\diver$).
\end{prop}

\begin{prop}[reglas de Leibnitz]
    Si $f,g$ son campos escalares y $\vec{F},\vec{G}$ campos vectoriales, todos de clase $\C^1$, se tiene que
    \begin{itemize}
        \item $\grad(fg) = f \grad g + g \grad f$,
        \item $\rot \left( f \vec{G} \right) = f \rot \vec{G} + \vec{\grad} f \times \vec{G}$,
        \item $\diver \left( f \vec{G} \right) = f \diver \vec{G} + \vec{\grad} f \cdot \vec{G}$,
        \item $\diver \left( \vec{F} \times \vec{G} \right) = \vec{G} \cdot \rot \vec{F} - \vec{F}
            \cdot \rot \vec{G}$.
    \end{itemize}
\end{prop}

\begin{col}[T.Schwarz]\label{col:luego_lo_ref}
    Sea $f$ un campo escalar de clase $\C^2$, $\vec{F}$ un campo vectorial de clase $\C^2$. Entonces, se tiene que
    \begin{gather*}
        \rot\left( \grad f \right) = 0, \\
        \diver \left( \rot \vec{F} \right) = 0.
    \end{gather*}
\end{col}

\begin{defi}
    Sea $\vec{F}$ un campo vectorial. Diremos que $\vec{F}$ es un campo conservador si $\exists f$ tal que
    \[
        \vec{F} = \grad f.
    \]
    Tambi\'en diremos que $\vec{F}$ es un campo irrotacional si
    \[
        \rot \vec{F} = 0.
    \]
\end{defi}
\begin{defi}
    Sea $\vec{G}$ un campo vectorial. Diremos que $\vec{G}$ es solenoidal si $\exists \vec{F}$ tal que
    \[
        \vec{G} = \rot \vec{F}.
    \]
    Y diremos que $\vec{G}$ es sin divergencia si
    \[
        \diver \vec{G} = 0.
    \]
\end{defi}

\begin{obs}
    Los resultados de \ref{col:luego_lo_ref}, se pueden resumir en que todo campo conservador es irrotacional y que todo
    campo solenoidal es sin divergencia.

    Posteriormente estudiaremos los recíprocos de estas afirmaciones.
\end{obs}

\begin{defi}
    Sea $f \in \Es^{(2)}(U)$. definimos el Lapaciano como
    \[
        \Delta f := \diver \left( \grad f \right),
    \]
    que es un operador diferencial de segundo orden, por tanto actua sobre campos escalares de clase $\C^2$. En
    coordenadas cartesianas
    \[
        \Delta = \nabla^2 \equiv \vec{\nabla} \cdot \vec{\nabla} = \pdv[2]{}{x} + \pdv[2]{}{y} + \pdv[2]{}{z}.
    \]
\end{defi}

\begin{obs*}
    Podemos extender la definición del laplaciano a $\real^n$, y en coordenadas cartesianas sería
    \[
        \Delta = \pdv[2]{}{x_1} + \cdots + \pdv[2]{}{x_n}.
    \]
\end{obs*}

\begin{obs}
    La expresión en coordenadas cartesiandas del laplaciano, permite aplicarlo tambin a campos vectoriales en
    $\real^3$, aplicandolo a cada componente, y, de hecho, se satisface que
    \[
        \rot \left( \rot \vec{F} \right) = \grad \left( \diver \vec{F} \right) - \nabla^2\vec{F}.
    \]
\end{obs}

\begin{obs}
    La equación de Laplace, es la ecuación en derivadas parciales
    \[
        \nabla^2 f = 0.
    \]
    las soluciones de esta ecuación se llaman \emph{funciones harmónicas}.
\end{obs}
\begin{obs}
    El laplaciano interviene en otras ecuaciones de gran importancia, como la ecuación del calor
    \[
        \pdv{f}{t} = \alpha \nabla^2 f,
    \]
    con $\alpha > 0$ o en la ecuación de las ondas
    \[
        \pdv[2]{f}{t} = c^2 \nabla^2 f,
    \]
    donde $c > 0$, está ecuación tambi\'en se puede expresar como $\square f = 0$, donde $\square = \frac{1}{c^2} \pdv[2]{}{t} - \nabla^2$
    es el operador \emph{d'alembertiano}.
\end{obs}

\section{Fórmulas integrales de la teoría de campos}

\begin{prop}
    Sea $W \subseteq \real^n$ abierto, $f \colon W \to \real$ un campo escalar de clase $\C^1$ y $\gamma \colon [t_0,t_1] \to W$
    un camino de clase $\C^1$. Con $\gamma\left( t_0 \right) = x_0$ y $\gamma\left( t_1 \right) = x_1$, entonces
    \[
        \int_\gamma \vec{\grad} f \dif \vec{l} = f(x_1) - f(x_0).
    \]
\end{prop}
\begin{proof}
    \begin{gather*}
        \int_\gamma \vec{\grad} f \dif \vec{l} = \int^{t_1}_{t_0} \vec{\grad} f\left( \gamma(t) \right) \vec{\gamma}^\prime(t) \dif t
        \stackrel{\text{def. grad.}}{=} \int^{t_1}_{t_0} \Dif f\left( \gamma(t) \right) \vec{\gamma}^\prime (t) \dif t = \\
        \stackrel{\text{regla cadena}}{=} \int^{t_1}_{t_0} \Dif \left( f \circ \gamma \right)(t) \dif t \stackrel{\text{Regla Barrow}}{=}
        \left( f \circ \gamma \right)\left( t_1 \right) - \left( f \circ \gamma \right)\left( t_0 \right) = f\left( x_1 \right) - 
        f\left( x_0 \right).
    \end{gather*}
\end{proof}

\begin{teo}[fundamental del cálculo]
    Sea $W \subseteq \real^n$ abierto, $f \colon W \to \real$ una función de clase $\C^1$. $C$ una curva regular orientada de clase
    $\C^1$, tal que $\bar{C} \subset W$ es compacto. Entonces
    \[
        \int_C \vec{\grad} f \dif \vec{l} = \int_{\partial C} f,
    \]
    (con $\partial C = \fr(C)$) donde interpretamos
    \[
        \int_{\partial C} f = f\left( x_1 \right) - f\left( x_0 \right).
    \]
    Si $\partial C = \setb{x_0, x_1}$ con $C$ ``orientada'' de $x_0$ a $x_1$. Y $\int_{\partial C} f = 0$ si $\partial C = \emptyset$.
\end{teo}
\needspace{5\baselineskip}
\begin{teo}[de Kelvin-Stokes, o del rotacional]\label{teo:kelvin-stokes}
    Sea $W \subseteq \real^3$ abierto, $\vec{F} \colon W \to \real^3$ un campo vectorial de clase $\C^1$ y $M \subset W$ una superficie
    orientada de clase $\C^2$ tal que $\bar{M}$ es compacto y $\bar{M} \subset W$. Sea $\partial M$ el borde de $M$ con la orientación
    inducida (``regla del tornillo'' o ``regla de la mano derecha'').

    Entonces, se satisface la fórmula de \emph{Kelvin-Stokes}
    \[
        \int_M \rot \vec{F} \dif \vec{S} = \int_{\partial M} \vec{F} \dif \vec{l}.
    \]
    Si todos los puntos frontera de $M$ son regulares; o, más generalmente, si el conjunto de puntos de la frontera singulares es finito.
\end{teo}

\begin{obs*}
    Los puntos frontera de $M$ se clasifican en:
    \begin{itemize}
        \item Puntos frontera regulares, si pertenecen a una curva regular incluída en la frontera de $M$,
        \item Puntos singulares.
    \end{itemize}
\end{obs*}

\begin{example*}
    Consideramos
    \[
        H =
        \begin{cases}
            x^2 + y^2 + z^2 = R^2 \\ z > 0
        \end{cases},
        \qquad
        \partial H  =
        \begin{cases}
            x^2 + y^2 = R^2 \\ z = 0
        \end{cases}
    \]
    orientadas ``hacia arriba'' y en ``sentido positivo'' al plano $xy$. Consideramos la función
    \[
        \vec{F}(x, y, z) = -y \hat{i} + x \hat{j} + z \hat{k}.
    \]
    Comprobaremos ahora el T.K.S. (\ref{teo:kelvin-stokes}), ahora parametrizamos $H$ como
    \begin{gather*}
        \sigma\left( \theta, \phi \right) = \left( R\cos\phi\sin\theta, R\sin\phi\sin\theta, R\cos\theta \right)
        \qquad
        \vec{T_\theta} \times \vec{T_\phi} = R^2 \sin\theta
        \begin{pmatrix}
            \cos\phi\sin\theta \\ \sin\phi\sin\theta \\ \cos\theta
        \end{pmatrix}.
    \end{gather*}
    Por último, 
    \[
        \rot \vec{F} = \determinant{\hat{i} & \partial x & - y \\ \hat{j} & \partial y & x \\ \hat{k} & \partial z & z} = 2 \hat{k}.
    \]
    Entonces, tenemos
    \begin{gather*}
        \int_H \rot \vec{F} \dif \vec{S} \stackrel{\text{orientación}}{=} \int^{2\pi}_0 \dif \phi \int^{\sfrac{\pi}{2}}_0 \dif \theta R^2 \sin \theta \underbrace{\cos\theta}_
        {z} 2 = 4 \pi \R^2 \frac{1}{2} \int^{\sfrac{\pi}{2}}_0 2 \sin\theta\cos\theta \dif \theta =\\= 2\pi R^2 \left[ \sin^2 \theta \right]^{\sfrac{\pi}{2}}_0 = 2\pi R^2.
    \end{gather*}
    Por otro lado, podemos parametrizar $\partial H$ como
    \[
        \gamma(t) = \left( R\cos t, R \sin t, 0 \right), \qquad 0 < t < 2\pi, \qquad
        \gamma^\prime (t) =
        \begin{pmatrix}
            -R \sin t \\ R \cos t \\ 0
        \end{pmatrix}.
    \]
    Ahora, podemos calcular
    \[
        \int_{\partial H} \vec{F} \dif \vec{l} = \int^{2\pi}_{0} \dif t R^2 = 2 \pi R^2.
    \]
    Y como vemos, se satisface el teorema.
\end{example*}

\begin{teo}[de Gauss-Ostrogradski, o de la divergencia]
    Sea $W \subset \real^3$ abierto, $\vec{F} \colon W \to \real^3$ de clase $\C^1$, $B \subset W$ un abierto tal que $\bar{B}$ es compacto y $\bar{B} \subset W$.
    Sea $\partial B$ el borde de $B$ con la orientación inducida (``normal hacia afuera''). Entonces, se satisface la fórmula de \emph{Gauss-Ostrogradskii}
    \[
        \int_B \diver \vec{F} \dif V = \int_{\partial B} \vec{F} \dif \vec{S}.
    \]
    Si todos los puntos frontera de $B$ son regulares; o, más generalmente, si el conjunto de puntos frontera singulares de $B$ es ``negligible'' (como que est\'e 
    contenido en una unión finita de curvas regulares conexas).
\end{teo}

\begin{obs}
    La definición precisa de puntos frontera regulares y singulares, de borde y su orientación, se explicarán con más detalle en la sección \ref{section:5-8}.
\end{obs}

\begin{example*}
    Consideramos
    \[
        \vec{r} =
        \begin{pmatrix}
            x \\ y \\ z
        \end{pmatrix}
    \]
    un campo vectorial radial con $\diver \vec{r} = 3$. Sea
    \begin{gather*}
        B : x^2  + y^2 + z^2 < R^2, \\
        \partial B : x^2 + y^2 + z^2 = R^2.
    \end{gather*}
    Ahora, tenemos
    \[
        \hat{r} = \frac{\vec{r}}{r} \qquad
        \vec{r} \hat{r} = \frac{r^2}{r} = r \stackrel{B}{=} R.
    \]
    Por último,
    \[
        \vol(B) = \int_B \dif V = \frac{1}{3} \int_B \diver \vec{r} \dif V = \frac{1}{3} \int_{\partial B} \vec{r} \dif \vec{S}
        = \frac{1}{3} \int_{\partial B} \left( \vec{r} \hat{r} \right) \dif S.
    \]
    Por lo tanto,
    \[
        \vol(B) = \frac{1}{3} R \vol\left( \partial B \right).
    \]
\end{example*}

\section{Potenciales}

\begin{defi}
    Sea $M \subseteq \real^n$ un abierto conexo, $\vec{F} \colon U \to \real^n$ un campo vectorial. Diremos que $\vec{F}$ es conservador
    cuando existe $f \colon U \to \real$ de clase $\C^1$ tal que $\vec{F} = \grad f$, diremos que $f$ es el potencial escalar para $\vec{F}$.
\end{defi}

\begin{obs}
    Si $c \in \real$ y $f$ es potencial escalar para $\vec{F}$, $f+c$ tambi\'en lo es $\left( \grad(f+c)=\grad f \right)$
\end{obs}

\begin{prop*}
    Sea $U \subset \real^n$ conexo, $\vec{F} \colon U \to \real^n$ un campo vectorial, si $f,g$ son dos potenciales espalares para $\vec{F}$, entonces
    $f - g$ es una constante.
\end{prop*}
\begin{proof}
    $f,g$ son potenciales escalares para $\vec{F} \implies \grad f = \grad g \implies \grad (f-g) = 0 \stackrel{U \text{ conexo}}{\implies} f-g =$ constante.
\end{proof}

\begin{obs}
    Sea $\gamma \colon \left[ t_0, t_1 \right] \to \real^n$ un camino de clase $\C^1$ a trozos, con $\gamma\left( t_0 \right) = x_0$, $\gamma\left( t_1 \right)=x_1$.
    Suponemos $\vec{F}$ conservador, es decir $\vec{F} = \grad f$, entonces,
    \[
        \int_\gamma \vec{F} \dif \vec{l} = f\left( x_1 \right) - f\left( x_0 \right)
    \]
    En otras palabras, la circulación de $\vec{F}$ solo depende de los puntos inicial y final.
    
    En este caso, escribiremos
    \[
        \int^{x_1}_{x_0} \vec{F} \dif \vec{l}
    \]
    Tabi\'en, tenemos que $\int^{x_0}_{x_0} \vec{F} \dif \vec{l} = 0$, $\oint \vec{F} \dif \vec{l} = 0$y las fórmulas
    \[
        \int^{x_1}_{x_0} \vec{F} \dif \vec{l} + \int^{x_2}_{x_1} \vec{F} \dif \vec{l} = \int^{x_2}_{x_0} \vec{F} \dif \vec{l}
        \qquad
        \int^{x_1}_{x_0} \vec{F} \dif \vec{l} = - \int^{x_0}_{x_!} \vec{F} \dif \vec{l}
    \]
\end{obs}

\begin{prop}
    Sea $U \subseteq$ un abierto convexo y $\vec{F} \colon U \to \real^n$ un campo vectorial de clase $\C^0$. Suponemos que la circulación de $\vec{F}$
    a lo largo de cualquier camino de clase $\C^1$ a trozos solo depende de los puntos iniciales y final. Entonces, $\vec{F}$ es conservador.
\end{prop}

\begin{obs*}
    Se puede construir $f$ fijando $x_0 \in U$ un punto que llamaremos el origen del potencial. Entonces, la función
    \[
        \begin{aligned}
            f \colon U &\to \real \\ x &\mapsto f(x) := \int^{x}_{x_0} \vec{F} \dif \vec{l}
        \end{aligned}
    \]
    que está bien definida y $\grad f = \vec{F}$.
\end{obs*}

\begin{proof}
    Sea $x \in U$, $\vec{u} \in \real^n$. Calculamos
    \[
        \Dif_{x, \vec{u}} f = \left. \dv{}{t} \right\vert_{t = 0} f\left( x + t \vec{u} \right)
    \]
    Sea $t > 0$ suficientemente pequeño, de manera que $\left[ x - t \vec{u}, x + t\vec{u}\right] \subset U$
    \begin{gather*}
        f \left( x + t \vec{u} \right) = \int^{x+ t \vec{u}}_{x_0} \vec{Fj \dif \vec{l}} = \underbrace{\int^x_{x_0} \vec{F} \dif \vec{l}}_{\text{constante}}
        + \int^{x + t \vec{u}}_x \vec{F} \dif \vec{l}
    \end{gather*}
    Aplicando un cambio de variable
    \[
        \int^{x + t \vec{u}}_{x} \vec{F} \dif \vec{l} = \int^t_0 \vec{F} \left( x + s \vec{u} \right) \vec{u} \dif S
    \]
    Por lo tanto,
    \[
        \left. \dv{}{t} \right\vert_{t=0} \int^{x + t \vec{u}}_x \vec{F} \dif \vec{l} \stackrel{\text{T.F.C.}}{=}
        \left. \vec{F}\left( x + t \vec{u} \right) \cdot \vec{u} \right\vert_{t=0} = \vec{F}(x) \cdot \vec{u} \implies
        \Dif_{x, \vec{u} }f = \vec{F}(x) \cdot \vec{u}
    \]
    Tomando ahora $\vec{u} = e_i$, obtenemos
    \[
        D_{e_i} f = F_i \quad \forall i \implies \grad f = \vec{F}
    \]
\end{proof}

\begin{teo*}
    Sea $U \subseteq \real^3$ abierto conexo. $\vec{F} \colon U \to \real^3$ de clase $\C^1$. Consideramos las propiedades
    \begin{enumerate}[(i)]
        \item $\vec{F}$ es conservador
        \item Dados $p_0, p_1 \in U$ y una curva orientada $C \subset U$ tal que $\partial U = \setb{p_0, p_!}$; la circulación
            $\int_C \vec{F} \dif \vec{l}$ solo depende de $p_0, p_1$.
        \item Para toda curva cerrada $C \subset U$ $\oint_C \vec{F} \dif \vec{l} = 0$
        \item $\rot \vec{F} = 0$, es decir, $\vec{F}$ es irrotacional
    \end{enumerate}

    Las tres primeras afirmaciones son equivalentes e implican la cuarta. Sin embargo, el recíproco es falso en general,
    depende de la topología de $U$ y es cierto en algunal circunstancias. 
\end{teo*}

\begin{defi}
    Un subconjunto $A \subset \real^n$ se dice que es estrellado respecto a un punto $p \in A$  si $\forall q \in A$,
    $[p, q] \subset A$. Diremos que $A$ es estrellado si lo es respecto a un punto.
\end{defi}

\begin{obs*}
    Un conjunto es convexo $\iff$ es estrellado respecto a todos sus puntos.
\end{obs*}

\begin{example*}
    Los siguientes conjuntos son estrellados
    \begin{itemize}
        \item Las semirectas en $\real^2$
        \item Un semiplano en $\real^3$
    \end{itemize}
    Lo siguientes, sin embargo, no lo son
    \begin{itemize}
        \item Un punto en $\real^2$
        \item Un punto en $\real^3$
    \end{itemize}
\end{example*}

\begin{lema}[de Poincar\'e]
    Sea $U \subset \real^3$ un abierto estrellado y $\vec{F} \colon U \to \real^3$ es un campo vectorial
    de clase $\C^1$. Si $\vec{F}$ es irrotacional $\left( \rot \vec{F} = 0 \right)$, entonces es conservador
    (tiene potencial escalar)
\end{lema}

\begin{defi}
    Sea $A$ un espacio m\'etrico. Sean $\gamma_0, \gamma_1 \colon [0,1] \to A$ dos caminos con los mismos puntos
    de origen $p$ y t\'ermino $q$. Diremos que $\gamma_0$ y $\gamma_1$ son caminos homótopos (con extremos fijos)
    si existe una homotopía de caminos entre $\gamma_0$ y $\gamma_1$, es decir:
    
    una familia $\left( \Gamma_s \right)_{0 \leq s \leq 1}$ de caminos en $A$, de origen en $p$ y t\'ermino en $q$,
    tales que
    \begin{itemize}
        \item $\Gamma_0 = \gamma_0$
        \item $\Gamma_1 = \gamma_1$
        \item La aplicación $\Gamma \colon [0,1] \times [0,1] \to A$ definida por $\Gamma(s,t) = \Gamma_s(t)$ es continua.
    \end{itemize}
\end{defi}
\begin{defi*}
    Un camino cerrado o lazo de punto base $p$, es un camino con origen y t\'ermino $p$.
\end{defi*}

\begin{defi}
    Diremos que un conjunto $A$ es simplemente conexo si
    \begin{enumerate}[i)]
        \item Es arco-conexo.
        \item\label{item:2-simp_conexo} Dados dos caminos $\gamma_0$, $\gamma_1$ con el mismmo origen y final,
            $\gamma_0$ es homotopo a $\gamma_1$.
    \end{enumerate}
\end{defi}

\begin{obs*}
    \ref{item:2-simp_conexo} es equipalente a requerir que cualquier camino $\gamma_0$ cerrado es homótopo al camino
    constante.
\end{obs*}

\begin{obs}
    Ejemplos de conjuntos simplemente conexos son: $\real^n$, las bolas, los rectángulos, los semiespacios de $\real^n$,
    los conjuntos estrellados, o $\real^2 \setminus$ semirecta.

    Ejemplos de conjuntos \emph{NO} simplemente conexos son: $\real^2 \setminus$ punto, $\real^3 \setminus$ recta o
    $\real^3 \setminus$ circunferencia.
\end{obs}

\begin{prop}
    Si $U \subset \real^3$ es un conjunto abierto simplemente conexo y $\vec{F} \colon U \to \real^3$, es un campo vectorial
    de clase $\C^1$ irrotacional $\left( \rot \vec{F} = 0 \right)$, entonces, $\vec{F}$ es conservador (tiene potencial
    escalar).
\end{prop}

\begin{obs}
    En caso de que $U = \real^3$ y $\vec{F}$ sea irrotacional, el cálculo del potencial escalar se puede hacer integrando
    la ecuación en derivadas parciales $\grad f = \vec{F}$. Son tres ecuaciones que se integran sucesivamente de manera
    elemental, en términos del cálculo de primitivas.
\end{obs}

\begin{example*}
    Consideramos $U = \real^3$ y 
    \[
        \vec{F} =
        \begin{pmatrix}
            y \\ z \cos (yz) + x \\ y \cos(yz)
        \end{pmatrix}
    \]
    con $\rot \vec{F} = 0$. Ahora, intentaremos resolver $\grad f = \vec{F}$
    \begin{gather*}
        \pdv{f}{x} = F_1 = y \stackrel{\text{integrar}}{\rightarrow} f = xy + \underbrace{g(y,z)}_{\text{constante respecto a } x} \\
        \pdv{f}{y} = F_2 = z \cos(yz) + x \rightarrow \cancel{x} + \pdv{g}{y} = \cancel{x} + z \cos(yz) \stackrel{\text{int}}{\rightarrow}
        g = \sin(yz) + h(z) \implies \\ \implies f = xy + \sin(yz) + h(z) \\
        \pdv{f}{z} = F_3 = y \cos (yz) \rightarrow \cancel{y \cos(yz)} + \dv{h}{z} = \cancel{y \cos(yz)} \implies h(z) = C
    \end{gather*}
    Por lo tanto, concluimos que
    \[
        f(x, y, z) = xy + \sin(yz) + C
    \]
\end{example*}

\begin{example*}
    Sea $U = \real^3 \setminus$eje $z$. Tomamos
    \[
        \vec{F} = 
        \begin{pmatrix}
            -\frac{y}{x^2 + y^2} \\ \frac{x}{x^2+y^2} \\ 0
        \end{pmatrix}
        \qquad
        \rot \vec{F} = 0
    \]
    Tomamos
    \[
        C =
        \begin{cases}
            x^2 + y^2 \\ z = 0
        \end{cases}
    \]
    orientada ``positivamente''. Se puede comprobar que $\oint_C \vec{F} \dif \vec{l} = 2\pi \neq 0$, es decir,
    que $\vec{F}$ no es conservativa
\end{example*}

\subsection*{Potenciales vectoriales}

Esta tema tampoco se ha estudiado (debido a la falta de tiempo) en clase, así que hacemos un pequeño resumen y referenciamos, como
siempre, a los curiosos a los apuntes del profesor.

Recordemos que decíamos que un campo $\vec{G}$ es solenoidal si existe $\vec{F}$ tal que $\vec{G} = \rot \vec{F}$, esto, implica
que $\diver \vec{G} = 0$, es decir, todo campo solenoidal es sin divergencia. Sin embargo, el recíproco es, en general falso. A pesar de
ello, el lema de Poincar\'e, nos asegura que si $U$ es estrellado, todo campo sin divergencia es solenoidal.

\section{Campos en el plano: Teorema de Green y potenciales}

\begin{defi}
    Sea $U \subseteq \real^2$ un abierto. Se puede definir el gradiente de un campo escalar y la divergencia de un campo
    vectorial en $U$, de hecho, en coordenadas cartesianas
    \[
        \grad f = \pdv{f}{x} \hat{e_1} + \pdv{f}{y} \hat{e_2} \qquad
        \diver \vec{F} = \pdv{F_1}{x} + \pdv{F_2}{y}
    \]
    Tambi\'en, podemos considerar $\vec{F} \colon U \to \real^2$ como un campo vectorial en $\real^3$, estonces,
    \[
        \vec{F} = 
        \begin{pmatrix}
            F_1(x,y) \\ F_2(x,y) \\ 0
        \end{pmatrix} \implies
        \rot \vec{F} =
        \begin{pmatrix}
            0 \\ 0 \\ \pdv{F_2}{x} - \pdv{F_1}{y}
        \end{pmatrix}
    \]
\end{defi}

\begin{teo}[de Green]
    Sean $U \subseteq \real^2$ un abierto, $\vec{F} \colon U \to \real^2$ de clase $\C^1$, $M \subset U$ abierto tal que
    $\bar{M}$ compacto con $\bar{M} \subset U$, sea $\partial M$ es el borde de $M$ con la orientación inducida (``la parte de dentro,
    a la izquierda''). Entonces, se satisface la fórmula de \emph{Green}
    \[
        \int_M \left( \pdv{F_2}{x} - \pdv{F_1}{y} \right) \dif x \dif y = \int_{\partial M} \vec{F} \dif \vec{l}
    \]
    suponiendo que todos los puntos frontera de $M$ son regulares, o, más generalmente, si el conjunto de puntos frontera singulares
    de $M$ es finito.
\end{teo}

\begin{obs}
    El teorema de Green se puede considerar como caso particular del teorema de Kelvin-Stokes (\ref{teo:kelvin-stokes}), consideranco $\vec{F}$
    como campo vectorial en $\real^3$ y $M$ como superficie dentro de $\real^3$.
\end{obs}

\begin{example}[Aplicación del teorema de Green para el cálculo de áreas]
    Sea $\vec{F} \colon \real^2 \to \real^2$ tal que $\pdv{F_2}{x} - \pdv{F_1}{y} = 1$, por ejemplo
    \[
        \vec{F} = \frac{1}{2} \left( -y \hat{i} + x \hat{j} \right)
    \]
    Aplicando la fórmula de Green, tenemos que
    \[
        \text{área}(M) = \int_M \dif x \dif y = \int_{\partial M} \vec{F} \dif \vec{l} \implies
        \text{área}(M) = \frac{1}{2} \int_{\partial M} \left( -y \dif x + x \dif y \right)
    \]
\end{example}

\begin{prop}
    Sea $\vec{F} \colon U \to \real^2$ un campo vectorial de clase $\C^1$ definido en un conjunto abierto $U \subset \real^2$.
    Recordamos que $\vec{F}$ es conservador cuando existe $f \colon U \to \real$ tal que $\vec{F} = \grad f$. En tal caso, la
    circulación de $\vec{F}$ a lo largo de una curva cerrada es 0. Además, si $\vec{F}$ es conservador, necesariamente se cumple
    \[
        \pdv{F_2}{x} = \pdv{F_1}{y}
    \]
\end{prop}

\begin{prop}
    Igualmente, si consideramos $\vec{F} \colon U \to \real^2$ un campo vectorial tal que
    \[
        \pdv{F_2}{x} = \pdv{F_1}{y}
    \]
    $\vec{F}$ es conservador, si $U$ es simplemente conexo.
\end{prop}


De nuevo, por falta de tiempo, se ``prescindió'' de este tema en clase, considerandolo como un caso particular del $\real^3$, aún
así, puede ser interesante consultar los apuntes del profesor, tanto de lo que queda de esta sección como de la siguiente.

\section{Los operadores diferenciales en otros sistemas de coordenadas}

Como ya hemos anunciado, se prescindió de este tema en clase debido a la falta de tiempo, refernciamos a los curiosos a los apuntes del
profesor.
