\usepackage[spanish,es-lcroman]{babel}

% Normal
\declaretheorem[style=normal,name=Lema,numberwithin=section]{lema}
\declaretheorem[style=normal,name=Lema,numbered=no]{lema*}
\declaretheorem[style=normal,name=Observación,sibling=lema]{obs}
\declaretheorem[style=normal,name=Observación,numbered=no]{obs*}
\declaretheorem[style=normal,name=Proposición,sibling=lema]{prop}
\declaretheorem[style=normal,name=Proposición,numbered=no]{prop*}
\declaretheorem[style=normal,name=Definición,sibling=lema]{defi*}
\declaretheorem[style=normal,name=Corolario,sibling=lema]{col}
\declaretheorem[style=normal,name=Corolario,numbered=no]{col*}
\declaretheorem[style=normal,name=Ejercicio,sibling=lema]{ej}
\declaretheorem[style=normal,name=Ejercicio,numbered=no]{ej*}
\declaretheorem[style=normal,name=Ejemplo,sibling=lema]{example}
\declaretheorem[style=normal,name=Ejemplo,numbered=no]{example*}
\declaretheorem[style=normal,name=Problema,sibling=lema]{problema}
\declaretheorem[style=normal,name=Problema,numbered=no]{problema*}

% Autodefi
\declaretheorem[style=autodefi,name=Definición,sibling=lema]{defi}

% Demo
\declaretheorem[style=demo,name=Demostración,qed=$\square$,numbered=no]{proof}
\declaretheorem[style=demo,name=Solución,numbered=no]{sol}

% Break
\declaretheorem[style=break,name=Teorema,sibling=lema]{teo*}

% Breakthm
\declaretheorem[style=breakthm,name=Teorema,sibling=lema]{teo}
\declaretheorem[style=breakthm,name=Lema,numberwithin=section]{teolema}
