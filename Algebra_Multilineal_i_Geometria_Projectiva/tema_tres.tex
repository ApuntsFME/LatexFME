\chapter{Proyectividades}
\section{Definiciones y propiedades básicas}

\begin{defi}
	\begin{enumerate}
		\item[]
		\item Sean $\Po = \Po(\E), \bar{\Po} = \Po(\bar{\E})$ espacios proyectivos de dimensión $n$. Sea $\varphi : \E \to \bar{\E}$ una aplicación lineal biyectiva. Una proyectividad, definida por $\varphi$, es una aplicación
		\begin{align*}
			f := [\varphi] : \Po &\to \bar{\Po} \\
			p = [v] &\mapsto f(p) = [\varphi(v)]
		\end{align*}
		Diremos que $\varphi$ es un representante de $f$.
		\item Si $\Po = \bar{\Po}$, llamaremos homografías a las proyectividades.
	\end{enumerate}
\end{defi}
\begin{obs}
	$\varphi$ no inyectiva $\implies [\varphi]$ no es una aplicación de $\Po(\E)$ en $\Po(\bar{\E})$.
\end{obs}
\begin{obs}
	$[\varphi] = [\psi] \iff \exists \lambda \neq 0$ tal que $\psi = \lambda \varphi$.
\end{obs}
\begin{proof}
	$\impliedby$ \\
	$\exists \lambda \neq 0$ tal que $\psi = \lambda\varphi$, sea $p = [v]$, $[\psi](p) = [\psi(v)] = [\lambda\varphi(v)] = [\varphi(v)] = [\varphi](p), \forall p \implies [\psi] = [\varphi].$ \\ \\
	$\implies$ \\
	Sean $u, v \in \E$ linealmente independientes, y sean $p = [u], q = [v]$. Entonces, como $[\varphi(u)] = [\psi(u)]$ y $[\varphi(v)] = [\psi(v)]$,
	\begin{gather*}
		\exists \lambda_1 \neq 0 \text{ tal que } \psi(u) = \lambda_1 \varphi(u) \\
		\exists \lambda_2 \neq 0 \text{ tal que } \psi(v) = \lambda_2 \varphi(v).
	\end{gather*}
	Por otro lado,
	\begin{gather*}
		[\psi(u+v)] = [\varphi(u+v)] \implies \exists \lambda_3 \neq 0 \text{ tal que } \psi(u) + \psi(v) = \psi(u+v) = \lambda_3 \varphi(u+v) = \\ \lambda_3 \varphi(u) + \lambda_3 \varphi(v) \implies (\lambda_1 - \lambda_3) \varphi(u) + (\lambda_2-\lambda_3) \varphi(v) = 0 \implies \lambda_1 = \lambda_2 = \lambda_3 =: \lambda \neq 0. \\
		\psi(u) = \lambda \varphi(u), \forall u \in \E \implies \exists \lambda \neq 0 \text{ tal que } \psi = \lambda \varphi.
	\end{gather*}
\end{proof}
\begin{prop}
	\begin{enumerate}
		\item[]
		\item Operaciones
		\begin{itemize}
			\item $\Id_\Po = [\Id_\E]$ es una proyectividad.
			\item $f, g$ proyectividades $\implies g \circ f$ proyectividad y $[\psi] \circ [\varphi] = [\psi \circ \varphi]$.
			\item $f = [\varphi]$ proyectividad $\implies f^{-1}$ proyectividad y $f^{-1} = [\varphi^{-1}]$.
		\end{itemize}
		\item Relación con variedades lineales \\
		Sea $f = [\varphi] : \Po \to \bar{\Po}$,
		\begin{itemize}
			\item $V = \pi(F)$ variedad lineal de $\Po$ de dimensión $d \implies f(V) = \pi(\varphi(F))$ es una variedad lineal de $\bar{\Po}$ de dimensión $d$.
			\item $V_1 \subseteq V_2 \iff f(V_1) \subseteq f(V_2).$
			\item $f(V_1 \vee V_2) = f(V_1) \vee f(V_2)$.
			\item $f(V_1 \cap V_2) = f(V_1) \cap f(V_2)$.
			\item En particular, $p_0, \dots, p_d$ linealmente independientes $\iff f(p_0), \dots, f(p_d)$ linealmente independientes.
		\end{itemize}
	\end{enumerate}
\end{prop}
\begin{proof}
	Inmediata a partir de las propiedades de $\varphi$ (biyectiva).
\end{proof}
\begin{col}
	Toda proyectividad es una colineación (mantiene la alineación de puntos).
\end{col}
\begin{col}
	Toda proyectividad mantiene las razones dobles.
\end{col}
\begin{proof}
	$\rho = (p_1, p_2, p_3, p_4) \implies \exists u,v \in \E$ tales que $p_1 = [u], p_2 = [v], p_3 = [u+v], p_4 = [\rho u + v] \implies f(p_1) = [\varphi(u)], f(p_2) = [\varphi(v)], f(p_3) = [\varphi(u) + \varphi(v)], f(p_4) = [\rho \varphi(u) + \varphi(v)] \implies \rho = (f(p_1), f(p_2), f(p_3), f(p_4))$.
\end{proof}
\begin{defi}
	La matriz de una proyectividad $f$ es $M_{\R, \bar{\R}} (f) := M_{B, \bar{B}}(\varphi)$.
\end{defi}
\begin{obs}
	$M_{\R, \bar{\R}} (f)$ está definida salvo multiplicar por $\lambda \neq 0$ (por la indeterminación de $B$ y $\bar{B}$).
\end{obs}
\begin{obs}
	Si $p_\R = (x_0 : \dots : x_n), f(p)_{\bar{\R}} = (y_0 : \dots : y_n)$, entonces,
	\[
	\begin{pmatrix} y_0 \\ \vdots \\ y_n \end{pmatrix} = M_{\R, \bar{\R}} \begin{pmatrix} x_0 \\ \vdots \\ x_n \end{pmatrix}
	\]
\end{obs}
\begin{prop}[Propiedades]
	\begin{enumerate}
		\item[]
		\item $M_{\R, \R}(\Id) = \Id$.
		\item Si
			\begin{alignat*}{3}
				&\Po \text{ } \stackrel{f}{\to} \text{  } &&\bar{\Po} \text{ } \stackrel{g}{\to} \text{  } &&\bar{\bar{\Po}} \\
				&\R_1 &&\R_2 &&\R_3
			\end{alignat*}
			Entonces, $M_{\R_1, \R_3}(g \circ f) = M_{\R_2, \R_3}(g) \times M_{\R_1, \R_2}(f)$
		\item $\Po_\R \stackrel{f}{\to} \bar{\Po}_{\bar{\R}}, M_{\bar{\R}, \R}(f^{-1}) = (M_{\R, \bar{\R}} (f))^{-1}.$
	\end{enumerate}
\end{prop}
\begin{proof}
	Trivial a partir de las propiedades de las matrices de aplicaciones lineales biyectivas.
\end{proof}
\begin{prop}
	$f : \Po \to \bar{\Po}$ proyectividad.
	\begin{itemize}
		\item Si $\R = \{p_0, \dots, p_n; \bar{p}\}$ es un sistema de referencia en $\Po \implies f(\R) := \{f(p_0), \dots, f(p_n); f(\bar{p})\}$ es un sistema de referencia en $\bar{\Po}$ y $M_{\R, f(\R)}(f) = \Id$.
		\item $g : \Po \to \bar{\Po}$ proyectividad, $\R$ un sistema de referencia en $\Po$,
			\[
				f = g \iff f(\R) = g(\R)
			\]
		\item $\Po, \bar{\Po}$ espacios proyectivos de dimensión $n$, $\R, \bar{\R}$ sistemas de referencia. Entonces $\exists! f : \Po \to \bar{\Po}$ proyectividad tal que $\bar{\R} = f(\R)$.
	\end{itemize}
\end{prop}
\begin{proof}
	Inmediata si $\Po \leftrightarrow \E, f \leftrightarrow \varphi, \R \leftrightarrow B$.
\end{proof}

\section{Caracterización geométrica de las proyectividades}

\begin{defi}
	$h : \Po \to \bar{\Po}$ (no necesariamente dim $\Po$ = dim $\bar{\Po}$)
	\begin{enumerate}
		\item $h$ es colineación $\iff \forall p_1, p_2, p_3 \in \Po$ diferentes dos a dos y alineados, $h(p_1), h(p_2), h(p_3) \in \bar{\Po}$ son diferentes dos a dos y están alineados.
		\item $h$ conserva las razones dobles $\iff h$ es colineación y $\forall p_1, p_2, p_3, p_4 \in \Po$ alineados, al menos tres diferentes, $(h(p_1), h(p_2), h(p_3), h(p_4)) = (p_1, p_2, p_3, p_4)$.
	\end{enumerate}
\end{defi}
\begin{obs}
	$h$ colineación $\implies h$ inyectiva.
\end{obs}
\begin{prop}
	$f$ proyectividad $\implies \begin{cases} f \text{ colineación} & (1) \label{coli} \\ f \text{ conserva razones dobles} & (2) \label{cons_raz_dob} \end{cases}$
\end{prop}
\begin{proof}
	$f : \Po \to \bar{\Po}, f = [\varphi]$,
	\begin{enumerate}[(1)]
		\item $f$ biyectiva $\implies f(p) \neq f(q), \forall p \neq q$. \\
		$f$ proyectividad $\implies f(V)$ es una variedad lineal de misma dimensión. \\
		$\begin{rcases}
		L = p_1 \vee p_2 \vee p_3 \\
		p_1, p_2, p_3 \text{ alineados}
		\end{rcases}
		\implies f(L) = f(p_1) \vee f(p_2) \vee f(p_3)$ es una recta.
		\item Supongamos $p_1, p_2, p_3, p_4$ diferentes dos a dos, $\rho = (p_1, p_2, p_3, p_4), \R = \{p_1, p_2; p_3\}$ sistema de referencia en $L = p_1 \vee P_2 \vee p_3 \vee p_4 \implies$ la coordenada absoluta de $p_4$ en $\R$ es $\rho \implies \exists u_1, u_2 \in \E$ tales que $p_1 = [u_1], p_2 = [u_2], p_3 = [u_1 + u_2], p_4 = [\rho u_1 + u_2]$. \\
		Entonces, $f(p_1) = [\varphi(u_1)], f(p_2) = [\varphi(u_2)], f(p_3) = [\varphi(u_1 + u_2)], f(p_4) = [\varphi(\rho u_1 + u_2)] = [\rho \varphi(u_1) + \varphi(u_2)]$ están alineados, al menos 3 son diferentes y $\rho = (f(p_1), f(p_2), f(p_3), f(p_4))$.
	\end{enumerate}
\end{proof}

\begin{teo} \label{teo:proy_dim_1}
  Sea $f : \Po \to \overline{\Po}$ una aplicación entre dos espacios proyectivos
  de dimensión $1$.
  Supongamos que:
  \begin{enumerate}
    \item $f$ es inyectiva
    \item $f$ mantinene las razones dobles
  \end{enumerate}
  Entonces $f$ es una proyectividad.
\end{teo}

\begin{proof}
  Sea $\R = \{p_1, p_2, \overline{p}\}$  una referencia en $\Po$. Por ser $f$ inyectiva,
  $f(p_1), f(p_2), f(p_3)$ son distintos entre si, con lo cual forman una referencia
  $\overline{\R} = f(\R)$ en $\overline{\Po}$.

  Por las propiedades de las proyectividades existe una única $g : \Po \to \overline{\Po}$
  proyectividad tal que $g(\R) = \overline{\R}$. Veremos que $f=g$. Sea $q \in \Po$,
  sea $\rho = (p_1, p_2, \overline{p}, q)$. Entonces, por mantener f razones dobles y
  por ser g proyectividad,
  \begin{gather*}
  (f(p_1), f(p_2), f(\overline{p}), f(q)) = \rho = (g(p_1), g(p_2), g(\overline{p}), g(q))
  \implies \\
  \implies f(q) \text{ y } g(q) \text{ tienen la misma coordenada absoluta en la
  referencia } \overline{R} \\
  \implies f(q) = g(q) \ \forall q \in \Po \implies f = g
  \end{gather*}
\end{proof}

\begin{obs}
  \begin{enumerate}
    \item[]
    \item $f$ proyectividad $\implies$ f biyectiva
    \item Fijada una referencia $\R$, mediante su coordenada absoluta existe una
    biyección $\Po_\k^1 \leftrightarrow \k \cup \{\infty\}$. En particular, $(p_1, p_2, p_3,
    q) = (p_1, p_2, p_3, \overline{q}) \implies q = \overline{q}$.
  \end{enumerate}
\end{obs}

\begin{col}
  Las funciones \textit{proyección} y \textit{sección} son proyectividades.
\end{col}
\begin{proof}
  Sea $r \in \Po^n$ una recta y $V \in \Po^n$ una variedad suplementaria a ésta
  (en particular, de dimensión $n-2$).
  \[
    \begin{rcases}
      \begin{aligned}
      \Phi \colon \Po^1 \cong r &\to V^\ast \cong \overline{\Po^1} \\
      p &\mapsto V \vee p = H_p
      \end{aligned}
      \quad
      \begin{aligned}
      \Psi \colon \overline{\Po^1} \cong V^* &\to r \cong \Po^1 \\ H &\mapsto H \cap r
      \end{aligned} \\
      \Phi, \Psi \text{ son inyectivas (de hecho biyectivas, } \Phi^{-1} = \Psi
      \text{)}\\
      \Phi, \Psi \text{ mantienen razones dobles (visto en \ref{teo:inv_rz_do})}
  \end{rcases}
  \implies
  \Phi, \Psi \text{ proyectividades.}
 \]
\end{proof}
\begin{obs}
  Las composiciones de funciones \textit{proyección} y \textit{sección} también son proyectividades.
\end{obs}
\begin{example}
    AQUÍ FALTAN LOS EJEMPLOS DE LAS PÁGINAS 5-6.
\end{example}
\begin{prop} \label{prop:demo_teo2}
    Sea $f : \Po^n \to \bar{\Po}^m \tq$:
    \begin{itemize}
        \item $f$ es colineación
        \item $f$ mantiene las razones dobles
    \end{itemize}
    Entonces $V = f \lp \Po^n \rp$ es una variedad lineal de $\bar{\Po}^m$ de dim $n$ y $f : \Po^n \to V = f \lp \Po^n \rp$ es una proyectividad.
\end{prop}
\begin{proof}
    $f : \Po^n \to \bar{\Po}^m$. Sigui $\R = \{ p_0, \dots, p_n; \bar{p} \}$ sistema de referencia, $W := f \lp p_0 \rp \vee \dots \vee f \lp p_n \rp$ variedad lineal de $\bar{\Po}^m$ de $\dim \leq n$. \\
    Inducción sobre $n$ (fijamos $m$):
    \begin{itemize}
        \item $n = 1, f : \Po^1 \to \bar{\Po}^m$ \\
        $\begin{rcases}
            f \text{ colineación } \implies 
            \begin{rcases}
                V = f \lp \Po^1 \rp \subseteq L \text{ recta} \\
                f \text{ inyectiva}
            \end{rcases}
            \implies f : \Po^1 \to L \cong \bar{\Po}^1 \\
            f \text{ mantiene las razones dobles}
        \end{rcases}
        \stackrel{\ref{teo:proy_dim_1}}{\implies} f$ es una proyectividad (en particular $V = f \lp \Po^1 \rp = L$).
        \item $n > 1$, supongamos cierto por $n-1$ (hipótesis de inducción). Probemos por $n$. \\
        Definimos $H_i := p_0 \vee \dots \vee \hat{p} \vee \dots \vee p_n$, $i = 0, \dots, n$ (hiperplano sin $p_i$). Demostremos que $p \in H_i \implies f \lp p \rp \notin f \lp H_i \rp$: \\
        $\begin{rcases}
            f : \Po^n \to \bar{\Po}^m \\
            f_{|H_i} : \Po^{n-1} \cong H_i \subseteq \Po^n \to \bar{\Po}^m
        \end{rcases}
        \implies g_i = f_{|H_i} : H_i \to \bar{\Po}^m$ es una proyectividad $\implies g_i : H_i \to f \lp H_i \rp = V_i$ es biyectiva. Supongamos $f \lp p \rp \in f \lp H_i \rp = V_i$,
        \[
            \exists q_i \in H_i \tq f \lp q_i \rp = f \lp p \rp \substack{f \text{ colineación} \\ \implies \\ f \text{ inyectiva}} p = q_i \in H_i. \text{ Contradicción.}
        \]
        Por tanto queda demostrado. Una consecuencia es que la imagen de $\R$ por $f$ son puntos en posición general en $\bar{\Po}^m$. Sea $T := f \lp p_0 \rp \vee \dots \vee f \lp p_n \rp \vee f \lp \bar{p} \rp$. Entonces $\bar{\R} = f \lp \R \rp$ es un sistema de referencia en $T$.
        
        AQUESTA DEMO ESTÁ A MITJES hehe
    \end{itemize}
\end{proof}
\begin{teo}
    Sea $f : \Po^n \to \bar{\Po}^n \tq$:
    \begin{itemize}
        \item $f$ es colineación
        \item $f$ mantiene las razones dobles
    \end{itemize}
    Entonces $f$ es una proyectividad.
\end{teo}
\begin{proof}
    Es un caso particular $\lp n = m \rp$ la proposición \ref{prop:demo_teo2}.
\end{proof}








































