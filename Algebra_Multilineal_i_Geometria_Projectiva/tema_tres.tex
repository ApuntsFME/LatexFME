\chapter{Proyectividades}
\section{Definiciones y propiedades básicas}

\begin{defi}
	\begin{enumerate}
		\item[]
		\item Sean $\Po = \Po(\E), \overline{\Po} = \Po(\overline{\E})$ espacios proyectivos de dimensión $n$. Sea $\varphi : \E \to \overline{\E}$ una aplicación lineal biyectiva. Una proyectividad, definida por $\varphi$, es una aplicación
		\begin{align*}
			f := [\varphi] : \Po &\to \overline{\Po} \\
			p = [v] &\mapsto f(p) = [\varphi(v)]
		\end{align*}
		Diremos que $\varphi$ es un representante de $f$.
		\item Si $\Po = \overline{\Po}$, llamaremos homografías a las proyectividades.
	\end{enumerate}
\end{defi}
\begin{obs}
	$\varphi$ no inyectiva $\implies [\varphi]$ no es una aplicación de $\Po(\E)$ en $\Po(\overline{\E})$.
\end{obs}
\begin{obs}
	$[\varphi] = [\psi] \iff \exists \lambda \neq 0$ tal que $\psi = \lambda \varphi$.
\end{obs}
\begin{proof}
	$\impliedby$ \\
	$\exists \lambda \neq 0$ tal que $\psi = \lambda\varphi$, sea $p = [v]$, $[\psi](p) = [\psi(v)] = [\lambda\varphi(v)] = [\varphi(v)] = [\varphi](p), \forall p \implies [\psi] = [\varphi].$ \\ \\
	$\implies$ \\
	Sean $u, v \in \E$ linealmente independientes, y sean $p = [u], q = [v]$. Entonces, como $[\varphi(u)] = [\psi(u)]$ y $[\varphi(v)] = [\psi(v)]$,
	\begin{gather*}
		\exists \lambda_1 \neq 0 \text{ tal que } \psi(u) = \lambda_1 \varphi(u) \\
		\exists \lambda_2 \neq 0 \text{ tal que } \psi(v) = \lambda_2 \varphi(v).
	\end{gather*}
	Por otro lado,
	\begin{gather*}
		[\psi(u+v)] = [\varphi(u+v)] \implies \exists \lambda_3 \neq 0 \text{ tal que } \psi(u) + \psi(v) = \psi(u+v) = \lambda_3 \varphi(u+v) = \\ \lambda_3 \varphi(u) + \lambda_3 \varphi(v) \implies (\lambda_1 - \lambda_3) \varphi(u) + (\lambda_2-\lambda_3) \varphi(v) = 0 \implies \lambda_1 = \lambda_2 = \lambda_3 =: \lambda \neq 0. \\
		\psi(u) = \lambda \varphi(u), \forall u \in \E \implies \exists \lambda \neq 0 \text{ tal que } \psi = \lambda \varphi.
	\end{gather*}
\end{proof}
\begin{prop}
	\begin{enumerate}
		\item[]
		\item Operaciones
		\begin{itemize}
			\item $\Id_\Po = [\Id_\E]$ es una proyectividad.
			\item $f, g$ proyectividades $\implies g \circ f$ proyectividad y $[\psi] \circ [\varphi] = [\psi \circ \varphi]$.
			\item $f = [\varphi]$ proyectividad $\implies f^{-1}$ proyectividad y $f^{-1} = [\varphi^{-1}]$.
		\end{itemize}
		\item Relación con variedades lineales \\
		Sea $f = [\varphi] : \Po \to \overline{\Po}$,
		\begin{itemize}
			\item $V = \pi(F)$ variedad lineal de $\Po$ de dimensión $d \implies f(V) = \pi(\varphi(F))$ es una variedad lineal de $\overline{\Po}$ de dimensión $d$.
			\item $V_1 \subseteq V_2 \iff f(V_1) \subseteq f(V_2).$
			\item $f(V_1 \vee V_2) = f(V_1) \vee f(V_2)$.
			\item $f(V_1 \cap V_2) = f(V_1) \cap f(V_2)$.
			\item En particular, $p_0, \dots, p_d$ linealmente independientes $\iff f(p_0), \dots, f(p_d)$ linealmente independientes.
		\end{itemize}
	\end{enumerate}
\end{prop}
\begin{proof}
	Inmediata a partir de las propiedades de $\varphi$ (biyectiva).
\end{proof}
\begin{col}
	Toda proyectividad es una colineación (mantiene la alineación de puntos).
\end{col}
\begin{col}
	Toda proyectividad mantiene las razones dobles.
\end{col}
\begin{proof}
	$\rho = (p_1, p_2, p_3, p_4) \implies \exists u,v \in \E$ tales que $p_1 = [u], p_2 = [v], p_3 = [u+v], p_4 = [\rho u + v] \implies f(p_1) = [\varphi(u)], f(p_2) = [\varphi(v)], f(p_3) = [\varphi(u) + \varphi(v)], f(p_4) = [\rho \varphi(u) + \varphi(v)] \implies \rho = (f(p_1), f(p_2), f(p_3), f(p_4))$.
\end{proof}
\begin{defi}
	La matriz de una proyectividad $f$ es $M_{\R, \overline{\R}} (f) := M_{B, \overline{B}}(\varphi)$.
\end{defi}
\begin{obs}
	$M_{\R, \overline{\R}} (f)$ está definida salvo multiplicar por $\lambda \neq 0$ (por la indeterminación de $B$ y $\overline{B}$).
\end{obs}
\begin{obs}
	Si $p_\R = (x_0 : \dots : x_n), f(p)_{\overline{\R}} = (y_0 : \dots : y_n)$, entonces,
	\[
	\begin{pmatrix} y_0 \\ \vdots \\ y_n \end{pmatrix} = M_{\R, \overline{\R}} \begin{pmatrix} x_0 \\ \vdots \\ x_n \end{pmatrix}
	\]
\end{obs}
\begin{prop}[Propiedades]
	\begin{enumerate}
		\item[]
		\item $M_{\R, \R}(\Id) = \Id$.
		\item Si
			\begin{alignat*}{3}
				&\Po \text{ } \stackrel{f}{\to} \text{  } &&\overline{\Po} \text{ } \stackrel{g}{\to} \text{  } &&\overline{\overline{\Po}} \\
				&\R_1 &&\R_2 &&\R_3
			\end{alignat*}
			Entonces, $M_{\R_1, \R_3}(g \circ f) = M_{\R_2, \R_3}(g) \times M_{\R_1, \R_2}(f)$
		\item $\Po_\R \stackrel{f}{\to} \overline{\Po}_{\overline{\R}}, M_{\overline{\R}, \R}(f^{-1}) = (M_{\R, \overline{\R}} (f))^{-1}.$
	\end{enumerate}
\end{prop}
\begin{proof}
	Trivial a partir de las propiedades de las matrices de aplicaciones lineales biyectivas.
\end{proof}
\begin{prop}
	$f : \Po \to \overline{\Po}$ proyectividad.
	\begin{itemize}
		\item Si $\R = \{p_0, \dots, p_n; \overline{p}\}$ es un sistema de referencia en $\Po \implies f(\R) := \{f(p_0), \dots, f(p_n); f(\overline{p})\}$ es un sistema de referencia en $\overline{\Po}$ y $M_{\R, f(\R)}(f) = \Id$.
		\item $g : \Po \to \overline{\Po}$ proyectividad, $\R$ un sistema de referencia en $\Po$,
			\[
				f = g \iff f(\R) = g(\R)
			\]
		\item $\Po, \overline{\Po}$ espacios proyectivos de dimensión $n$, $\R, \overline{\R}$ sistemas de referencia. Entonces $\exists! f : \Po \to \overline{\Po}$ proyectividad tal que $\overline{\R} = f(\R)$.
	\end{itemize}
\end{prop}
\begin{proof}
	Inmediata si $\Po \leftrightarrow \E, f \leftrightarrow \varphi, \R \leftrightarrow B$.
\end{proof}

\section{Caracterización geométrica de las proyectividades}

\begin{defi}
	$h : \Po \to \overline{\Po}$ (no necesariamente dim $\Po$ = dim $\overline{\Po}$)
	\begin{enumerate}
		\item $h$ es colineación $\iff \forall p_1, p_2, p_3 \in \Po$ diferentes dos a dos y alineados, $h(p_1), h(p_2), h(p_3) \in \overline{\Po}$ son diferentes dos a dos y están alineados.
		\item $h$ conserva las razones dobles $\iff h$ es colineación y $\forall p_1, p_2, p_3, p_4 \in \Po$ alineados, al menos tres diferentes, $(h(p_1), h(p_2), h(p_3), h(p_4)) = (p_1, p_2, p_3, p_4)$.
	\end{enumerate}
\end{defi}
\begin{obs}
	$h$ colineación $\implies h$ inyectiva.
\end{obs}
\begin{prop}
	$f$ proyectividad $\implies \begin{cases} f \text{ colineación} & (1) \label{coli} \\ f \text{ conserva razones dobles} & (2) \label{cons_raz_dob} \end{cases}$
\end{prop}
\begin{proof}
	$f : \Po \to \overline{\Po}, f = [\varphi]$,
	\begin{enumerate}[(1)]
		\item $f$ biyectiva $\implies f(p) \neq f(q), \forall p \neq q$. \\
		$f$ proyectividad $\implies f(V)$ es una variedad lineal de misma dimensión. \\
		$\begin{rcases}
		L = p_1 \vee p_2 \vee p_3 \\
		p_1, p_2, p_3 \text{ alineados}
		\end{rcases}
		\implies f(L) = f(p_1) \vee f(p_2) \vee f(p_3)$ es una recta.
		\item Supongamos $p_1, p_2, p_3, p_4$ diferentes dos a dos, $\rho = (p_1, p_2, p_3, p_4), \R = \{p_1, p_2; p_3\}$ sistema de referencia en $L = p_1 \vee P_2 \vee p_3 \vee p_4 \implies$ la coordenada absoluta de $p_4$ en $\R$ es $\rho \implies \exists u_1, u_2 \in \E$ tales que $p_1 = [u_1], p_2 = [u_2], p_3 = [u_1 + u_2], p_4 = [\rho u_1 + u_2]$. \\
		Entonces, $f(p_1) = [\varphi(u_1)], f(p_2) = [\varphi(u_2)], f(p_3) = [\varphi(u_1 + u_2)], f(p_4) = [\varphi(\rho u_1 + u_2)] = [\rho \varphi(u_1) + \varphi(u_2)]$ están alineados, al menos 3 son diferentes y $\rho = (f(p_1), f(p_2), f(p_3), f(p_4))$.
	\end{enumerate}
\end{proof}

\begin{teo} \label{teo:proy_dim_1}
	Sea $f : \Po \to \overline{\Po}$ una aplicación entre dos espacios proyectivos
	de dimensión $1$.
	Supongamos que:
	\begin{enumerate}
		\item $f$ es inyectiva
		\item $f$ mantinene las razones dobles
	\end{enumerate}
	Entonces $f$ es una proyectividad.
\end{teo}

\begin{proof}
	Sea $\R = \{p_1, p_2, \overline{p}\}$  una referencia en $\Po$. Por ser $f$ inyectiva,
	$f(p_1), f(p_2), f(p_3)$ son distintos entre si, con lo cual forman una referencia
	$\overline{\R} = f(\R)$ en $\overline{\Po}$.

	Por las propiedades de las proyectividades existe una única $g : \Po \to \overline{\Po}$
	proyectividad tal que $g(\R) = \overline{\R}$. Veremos que $f=g$. Sea $q \in \Po$,
	sea $\rho = (p_1, p_2, \overline{p}, q)$. Entonces, por mantener f razones dobles y
	por ser g proyectividad,
	\begin{gather*}
	(f(p_1), f(p_2), f(\overline{p}), f(q)) = \rho = (g(p_1), g(p_2), g(\overline{p}), g(q))
	\implies \\
	\implies f(q) \text{ y } g(q) \text{ tienen la misma coordenada absoluta en la
	referencia } \overline{R} \\
	\implies f(q) = g(q) \ \forall q \in \Po \implies f = g
	\end{gather*}
\end{proof}

\begin{obs}
	\begin{enumerate}
	  \item[]
	  \item $f$ proyectividad $\implies$ f biyectiva
	  \item Fijada una referencia $\R$, mediante su coordenada absoluta existe una
	  biyección $\Po_\k^1 \leftrightarrow \k \cup \{\infty\}$. En particular, $(p_1, p_2, p_3,
	  q) = (p_1, p_2, p_3, \overline{q}) \implies q = \overline{q}$.
	\end{enumerate}
\end{obs}

\begin{col}
	Las funciones \textit{proyección} y \textit{sección} son proyectividades.
\end{col}
\begin{proof}
	Sea $r \in \Po^n$ una recta y $V \in \Po^n$ una variedad suplementaria a ésta
	(en particular, de dimensión $n-2$).
	\[
		\begin{rcases}
			\begin{aligned}
			\Phi \colon \Po^1 \cong r &\to V^\ast \cong \overline{\Po^1} \\
			p &\mapsto V \vee p = H_p
			\end{aligned}
			\quad
			\begin{aligned}
			\Psi \colon \overline{\Po^1} \cong V^* &\to r \cong \Po^1 \\ H &\mapsto H \cap r
			\end{aligned} \\
			\Phi, \Psi \text{ son inyectivas (de hecho biyectivas, } \Phi^{-1} = \Psi
			\text{)}\\
			\Phi, \Psi \text{ mantienen razones dobles (visto en \ref{teo:inv_rz_do})}
	\end{rcases}
	\implies
	\Phi, \Psi \text{ proyectividades.}
 \]
\end{proof}
\begin{obs}
	Las composiciones de funciones \textit{proyección} y \textit{sección} también son proyectividades.
\end{obs}
\begin{prop} \label{prop:demo_teo2}
	Sea $f : \Po^n \to \overline{\Po}^m \tq$:
	\begin{itemize}
		\item $f$ es colineación
		\item $f$ mantiene las razones dobles
	\end{itemize}
	Entonces $V = f \lp \Po^n \rp$ es una variedad lineal de $\overline{\Po}^m$ de dim $n$ y $f : \Po^n \to V = f \lp \Po^n \rp$ es una proyectividad.
\end{prop}
\begin{proof}
	$f : \Po^n \to \overline{\Po}^m$. Sigui $\R = \{ p_0, \dots, p_n; \overline{p} \}$ sistema de referencia, $W := f \lp p_0 \rp \vee \dots \vee f \lp p_n \rp$ variedad lineal de $\overline{\Po}^m$ de $\dim \leq n$. \\
	Inducción sobre $n$ (fijamos $m$):
	\begin{itemize}
		\item $n = 1, f : \Po^1 \to \overline{\Po}^m$ \\
		$\begin{rcases}
			f \text{ colineación } \implies 
			\begin{rcases}
				V = f \lp \Po^1 \rp \subseteq L \text{ recta} \\
				f \text{ inyectiva}
			\end{rcases}
			\implies f : \Po^1 \to L \cong \overline{\Po}^1 \\
			f \text{ mantiene las razones dobles}
		\end{rcases}
		\stackrel{\ref{teo:proy_dim_1}}{\implies} f$ es una proyectividad (en particular $V = f \lp \Po^1 \rp = L$).
		\item $n > 1$, supongamos cierto por $n-1$ (hipótesis de inducción). Probemos por $n$. \\
		Definimos $H_i := p_0 \vee \dots \vee \hat{p} \vee \dots \vee p_n$, $i = 0, \dots, n$ (hiperplano sin $p_i$). Demostremos que $p \in H_i \implies f \lp p \rp \notin f \lp H_i \rp$: \\
		$\begin{rcases}
			f : \Po^n \to \overline{\Po}^m \\
			f_{|H_i} : \Po^{n-1} \cong H_i \subseteq \Po^n \to \overline{\Po}^m
		\end{rcases}
		\implies g_i = f_{|H_i} : H_i \to \overline{\Po}^m$ es una proyectividad $\implies g_i : H_i \to f \lp H_i \rp = V_i$ es biyectiva. Supongamos $f \lp p \rp \in f \lp H_i \rp = V_i$,
		\[
			\exists q_i \in H_i \tq f \lp q_i \rp = f \lp p \rp \substack{f \text{ colineación} \\ \implies \\ f \text{ inyectiva}} p = q_i \in H_i. \text{ Contradicción.}
		\]
		Por tanto queda demostrado. Una consecuencia es que la imagen de $\R$ por $f$ son puntos en posición general en $\overline{\Po}^m$. Sea $T := f \lp p_0 \rp \vee \dots \vee f \lp p_n \rp \vee f \lp \overline{p} \rp$. Entonces $\overline{\R} = f \lp \R \rp$ es un sistema de referencia en $T$.
		
		Entonces $\exists! g : \Po^n \to T, g \lp \R \rp = \overline{\R}, g$ proyectividad. Veremos que $f = g$.
		De momento sabemos que:
		\begin{enumerate}[(1)]
			\item\label{item:1_dela_prop} $\begin{cases} f \lp p_i \rp = g \lp p_i \rp \\ f \lp \overline{p} \rp = g \lp p \rp \end{cases}$
			\item\label{item:2_dela_prop} $F_{|H_i} = g_i = g_{|H_i}$ (porque mandan un sistema de referencia al mismo sitio y son proyectividades)
		\end{enumerate}
		Hemos de ver que $\forall p \in \Po^n, f \lp p \rp = g \lp p \rp$:
		\begin{itemize}
			\item Si $p = \overline{p}$ o $p \in \bigcup_{i=0}^n H_i$ ya lo sabemos por \eqref{item:1_dela_prop} i \eqref{item:2_dela_prop}.
			\item Si $p \neq \overline{p}$ y $p \notin \bigcup_{i=0}^n H_i$
		\end{itemize}
		Sean $L := p \vee \overline{p}$ y sean $q_n := L \cap H_n, q_0 := L \cap H_0$. Como $f$ es una colineación, $f \lp L \rp$ es una recta, $f \lp L \rp = f \lp p \rp \vee f \lp \overline{p} \rp$.
		Entonces:
		\begin{gather*}
			\lp f \lp q_0 \rp, f \lp q_1 \rp, f \lp \overline{p} \rp, f \lp p \rp \rp \substack{f \text{ mantiene} \\ = \\ \text{razón doble}} \lp q_0, q_1, \overline{p}, p \rp \stackrel{g \text{ proyec.}}{=} \lp g \lp q_0 \rp, g \lp q_1 \rp, g \lp \overline{p} \rp, g \lp p \rp \rp = \\
			\stackrel{\eqref{item:1_dela_prop} + \eqref{item:2_dela_prop}}{=} \lp f \lp p_0 \rp, f \lp p_1 \rp, f \lp \overline{p} \rp, g \lp p \rp \rp \implies f \lp p \rp = g \lp p \rp.
		\end{gather*}
	\end{itemize}
\end{proof}
\begin{teo} \label{teo:proyectividades}
	Sea $f : \Po^n \to \overline{\Po}^n \tq$:
	\begin{itemize}
		\item $f$ es colineación
		\item $f$ mantiene las razones dobles
	\end{itemize}
	Entonces $f$ es una proyectividad.
\end{teo}
\begin{proof}
	Es un caso particular $\lp n = m \rp$ la proposición \ref{prop:demo_teo2}.
\end{proof}
\begin{defi}
	Sean $V_1, V_2 \subseteq \Po^n$ variedades proyectivas de dimesión $d$. Sea $W$ una variedad suplementaria de $V_1$ y $V_2$. La perspectividad de centro $W$ de $V_1$ a $V_2$ es:
	\begin{align*}
		f : V_1 &\to V_2 \\
		p &\mapsto f \lp p \rp = \lp W \vee p \rp \cap V_2
	\end{align*}
\end{defi}
\begin{obs}
	Esta bien definido:
	\begin{align*}
		\dim \lp \lp W \vee p \rp \cap V_2 \rp &= \overbrace{\dim \lp W \vee p \rp}^{ n -d -1 + 1} + \overbrace{\dim V_2}^{d} - \overbrace{\dim \lp \lp W \vee p \rp \vee V_2 \rp}^n \\
		&= n -d -1 + 1 + d - n = 0
	\end{align*}
	Por lo tanto $W \vee p \cap V_2$ es un punto $\lp f \lp p \rp \rp$
\end{obs}
\begin{obs}
	En $d = 1$, $f$ perspectividad $=$ proyección y sección.
\end{obs}
\begin{col}[(del teorema \ref{teo:proyectividades})]
	Toda perspectividad es una proyectividad.
\end{col}
\begin{proof}
	\begin{enumerate}[(1)]
		\item[]
		\item \label{item:col_teo2_it1} En primer lugar, veamos que $L \subseteq V_1$ recta $\implies f \lp L \rp \subseteq V_2$ recta. $f \lp L \rp = \lp W \vee L \rp \cap V_2$,
		\begin{align*}
			\dim f \lp L \rp &= \dim \lp \lp W \vee L \rp \cap V_2 \rp = \overbrace{\dim \lp W \vee L \rp}^{n-d+1} + \overbrace{\dim V_2}^d - \overbrace{\dim \lp W \vee L \vee V_2 \rp}^{n \; \lp W \vee V_2 = \Po^n \rp} \\
			&= n - d + 1 + d - n = 1.
		\end{align*}
		Por lo tanto, $f \lp L \rp$ es una recta.
		\item \label{item:col_teo2_it2} Veamos ahora que $f$ es inyectiva. Sean $p_1, p_2 \in V_1, p_1 \neq p_2, \tq f \lp p_1 \rp = f \lp p_2 \rp$ y sean
		\begin{itemize}
			\item $L_1 := f \lp p_1 \rp \vee p_1$
			\item $L_2 := f \lp p_2 \rp \vee p_2$
			\item $q_1 = L_1 \cap W$
			\item $q_2 = L_2 \cap W$
		\end{itemize}
		Si $p_1 \neq p_2 \implies q_1 \neq q_2$. Sean $r = p_1 \vee p_2, s = q_1 \vee q_2, \pi = L_1 \vee L_2$ (es un plano porqu $L_1 \cap L_2 = f \lp p_1 \rp = f \lp p_2 \rp$).
		\begin{gather*}
			\begin{rcases}
				\{ p_1, p_2 \} \subseteq r \cap \pi \implies r \subseteq \pi \\
				\{ q_1, q_2 \} \subseteq s \cap \pi \implies s \subseteq \pi
			\end{rcases}
			\implies \emptyset \neq r \cap s \stackrel{(*)}{\subseteq} V_1 \cap " \stackrel{\text{supl.}}{=} \emptyset. \text{ Contradicción.}
		\end{gather*}
		\item \label{item:col_teo2_it3} Ahora veamos que $f$ mantiene las razones dobles. Sea $\widetilde{W} = W \vee L $. Observamos que $f \lp L \rp \subseteq \widetilde{W}$ y que $\dim \widetilde{W} = n-d-1+1+1= n-d+1$. \\
		En $\widetilde{W} \; \lp \widetilde{W}\cong\Po^{n-d+1} \rp$ tenemos que $f_{|L} = \lp \text{sección con }f \lp L \rp \rp \circ \lp \text{proyección desde } W \rp$. Como la razón doble se conserva por proyección y sección,
		\[ \lp p_1, p_2, p_3, p_4 \rp = \lp f \lp p_1 \rp, f \lp p_2 \rp, f \lp p_3 \rp, f \lp p_4 \rp \rp \implies f \text{ conserva razones dobles.}
		\]
	\end{enumerate}
	\noindent Los puntos \eqref{item:col_teo2_it1} y \eqref{item:col_teo2_it2} implican que $f$ es una colineación. Los puntos \eqref{item:col_teo2_it1}, \eqref{item:col_teo2_it2} y \eqref{item:col_teo2_it3} implican (por el teorema \ref{teo:proyectividades}) que $f$ es una proyectividad
\end{proof}
\begin{teo}[fundamenteal de la geometría proyectiva]
	\[ f : \Po^n \to \Po^n \text{ colineación} \implies f = [\varphi], \varphi : \E \to \E, \varphi \text{ aplicación lineal}. \]
	En particular, si $\k = \real, f$ colineación $\implies f$ proyectividad.
\end{teo}
\begin{example}
	\begin{align*}
		\varphi : \cx^2 &\to \cx^2 \\
		v = \lp z_1, z_2 \rp &\mapsto \varphi \lp v \rp = \lp \overline{z_1}, \overline{z_2} \rp
	\end{align*}
	$\varphi \lp \lambda v \rp = \overline{\lambda} \varphi \lp v \rp$ no es lineal. Sea 
	\[ f = [\varphi] : \Po^1_\cx \to \Po^1_\cx, \]
	$f$ no es una proyectividad, pero $f$ es una colineación.
\end{example}
\begin{defi}
	Sea $\k$ un cuerpo. Un automorfismo de $\k$ es:
	\[ \sigma : \k \to \k \text{ biyectiva} \]
	tal que:
	\begin{itemize}
		\item $\sigma \lp 0 \rp = 0$
		\item $\sigma \lp 1 \rp = 1$
		\item $\sigma \lp a_1 + a_2 \rp = \sigma \lp a_1 \rp + \sigma \lp a_2 \rp$
		\item $\sigma \lp a_1 \times a_2 \rp = \sigma \lp a_1 \rp \times \sigma \lp a_2 \rp$
	\end{itemize}
\end{defi}
\begin{example}
	Sea
	\begin{align*}
		\sigma : \cx &\to \cx \\
		z &\mapsto \sigma \lp z \rp = \overline{z},
	\end{align*}
	$\sigma$ es un automorfismo.
\end{example}
\begin{ej}
	$\sigma : \real \to \real$ automorfismo $\implies \sigma = \Id$.
\end{ej}
\begin{defi}
	Sea $\E$ un $\k$ espacio vectorial, $\varphi : \E \to \E$ es semilineal si y solo si
	\begin{enumerate}[(1)]
		\item $\forall u_1, u_2, \; \varphi \lp v_1 + u_2 \rp = \varphi \lp u_1 \rp + \varphi \lp u_2 \rp$
		\item $\exists \sigma: \k \to \k$ automorfismo t.q. $\varphi \lp \lambda u \rp = \sigma \lp \lambda \rp \varphi \lp u \rp, \forall \lambda \in \k$.
	\end{enumerate}
\end{defi}
\begin{obs}
	En $\real, \varphi$ semilineal $\implies \varphi$ lineal.
\end{obs}

\section{El teorema de Poncelet}

\begin{teo}[de Poncelet]
	Sean $V_1, V_2$ variedades proyectivas de dim $d$ en $\Po^n$. Toda proyectividad $f \colon V_1 \to V_2$
	es composición de perspectividades.
\end{teo}

Demostraremos el teorema tan solo para $d = 1$. Empezamos
demostrando el siguiente lema.

\begin{lema} 
    \label{lema_poncelet}
    Sean $r_1, r_2 \in \Po^2$, $f \colon r_1 \to r_2$ proyectividad.
    
    \[f \text{ perspectividad} \iff f(O) = O,\]
    donde $O = r_1 \cap r_2$.
\end{lema}

\begin{proof}
    Veamos primero la implicación directa. Sea $W = {q}$
    el centro de $f$. Entonces

    \begin{align*}
	f : r_1 &\to r_2 \\
	p &\mapsto pq \cap r_2
    \end{align*}
    
    Entonces $f(O) = Oq \cap r_2 = O$.
    
    Veamos ahora la implicación recíproca. Sean $A_1, A_2 \in r_1
    \setminus \{O\}$, tales que $\lp A_1 \neq A_2\rp$. 
    Sean $B_1 = f(A_1), B_2 = f(A_2)$. Entonces $\R = \left \{ A_1, 
    A_2, O \right \}, \R' = \left \{ B_1, B_2, O \right \}$ son
    referencias en $r_1$ y $r_2$ respectivamente. La segunda 
    es una referencia ya que $B_1, B_2 \neq O$ porque $f(O) = O$ y
    $f$ es inyectiva por ser proyectividad, y por el mismo motivo
    $B_1 \neq B_2$.
    
    Sea $q = A_1B_1 \cap A_2B_2$, que no es vacío pues en el
    plano dos rectas se cortan. Sea $g \colon r_1 \to r_2$ 
    perspectividad de centro $W = \{ q\}$. Por construcción:
    
    \[
    \begin{rcases}
        g(O) = O = f(O) \\
        g(A_1) = B_1 = f(A_1) \\
        g(A_2) = B_2 = f(A_2)
    \end{rcases}
    \implies
    g(R) = f(R) \implies g = f,
    \]
    por ser $g$ y $f$ proyectividades.
\end{proof}
\begin{teo}
    \label{primer_teo_poncelet}
    Sean $r_1, r_2 \in \Po^2$ rectas diferentes. Sea $f \colon 
    r_1 \to r_2$ proyectividad. Entonces $f$ es composición
    de 1 o 2 perspectividades.
\end{teo}
\begin{proof} % COSAS_QUE_FALTAN FALTA DIBUJO ESTA 
              % PARTE DEMO (pag 11 DIVO)
    Sea $O = r_1 \cap r_2$. Consideraremos dos casos.
    \begin{enumerate}
        \item Suponemos $f(O) = O$. Por \ref{lema_poncelet}, 
        $f$ es perspectividad.
        \item Suponemos $f(O) \neq O$. Sean $A_1, A_2, A_3 \in r_1$
        (diferentes dos a dos, $A_i \neq O$). En consecuencia,
        $\R = \{A_1, A_2; A_3\}$ forma una referencia en $r_1$.
        Sea $B_i = f(A_i)$. Podemos asegurar que $B_i \neq 0$
        si tomamos $A_i$ adecuadas. Sea $s \ni B_1$ ($ s 
        \not \ni B_j, A_i$), y $q \in A_1B_1$, $q \not
        \in r_1, r_2$. Postulamos que podemos encontrar
        dos perspectividades $g_1, g_2$ que cumplan
        \begin{alignat*}{3}
            r_1 &\stackrel{g_1}\longrightarrow &&s&&
            \stackrel{g_2}\longrightarrow r_2\\
            A_1 &\longmapsto B_1&&=C_1 &&\longmapsto B_1\\
            A_2 &\longmapsto &&C_2&&\longmapsto B_2\\
            A_3 &\longmapsto &&C_3&&\longmapsto B_3.
        \end{alignat*}
        $g_1$ es una perspectividad desde $q$, concretamente
        $g_1 = \psi(s) \circ \varphi(q)$. Para ver que existe
        una perspectividad $g_2$ adecuada, consideramos la
        proyectividad que envía la referencia de $s$
        $\{C_1, C_2; C_3\}$ a la referencia de $r_2$ $\{B_1,
        B_2; B_3\}$. Dado que $f(C_1) = B_1 = C_1$, y $C_1 = 
        s \cap r_2$, podemos aplicar el primer para ver que 
        esta proyectividad es también una perspectividad.
        
        Finalmente, $(g_2 \circ g_1)(A_i) = B_i = f(A_i) \implies
        f = g_2 \circ g_1$, por ser ambas proyectividades.
        
    \end{enumerate}
\end{proof}

Antes de proceder con el siguiente teorema, demostraremos
el siguiente lema que resultará de utilidad.

\begin{lema} %COSAS_QUE_FALTAN DEMO DAIXO
    \label{lema_rectas_poncelet}
    Sean $l_1, l_2 \subseteq \Po^3$ rectas disjuntas. Sea $p \not \in 
    l_1, l_2$. Entonces $\exists! \: s$ recta 
    tal que $p \in s$, $s \cap l_1 \neq \varnothing$, 
    $s \cap l_2 \neq \varnothing$.
\end{lema}
\begin{proof}
    
\end{proof}
\begin{teo} %COSAS_QUE_FALTAN Igual un dibujo aqui tambien
    Sean $r_1, r_2 \subseteq \Po^n$ ($n \geq 3$) rectas disjuntas.
    Sea $f \colon r_1 \to r_2$ proyectividad, entonces $f$
    es una perspectividad. 
\end{teo}

\begin{proof} %COSAS_QUE_FALTAN ACABAR DEMO
    Consideraremos dos casos.
    \begin{enumerate}
        \item Supongamos $n = 3$. Consideramos $A_1, A_2, A_3 \in
        r_1$ diferentes dos a dos, de manera que forman una
        referencia. Consideramos $B_1 = f(A_1), B_2 = f(A_2),
        B_3 = f(A_3)$, que forman una referencia en $r_2$.
        Sean $l_i = A_i \vee B_i$.
        
        Veamos que $l_1 \cap l_2 = \varnothing$, y análogamente
        para las demás parejas de índices. Para ello, operamos
        por reducción al absurdo:
        \[
        l_1 \cap l_2 \neq \varnothing \implies l_1 \vee l_2 = \pi
        \implies
        \begin{cases}
        A_1, A_2 \in \pi \implies r_1 = A_1 \vee A_2 \subseteq \pi\\
        B_1, B_2 \in \pi \implies r_2 = B_2 \vee B_3 \subseteq \pi
        \end{cases}
        \]
        De lo que $r_1 \cap r_2 \neq \varnothing$,
        una contradicción.
        
        Cojamos $p \in A_1, B_1$, $p \not \in r_1, r_2$. Consideramos
        la recta $s$ dada por \ref{lema_rectas_poncelet} que cumple
        $p \in s, s \cap l_2 \neq \varnothing, s \cap l_3 \neq 
        \varnothing$.
        Sea g
        \item
    \end{enumerate}
\end{proof}

\begin{obs}
    Si $r_1 \neq r_2$ pero $r_1 \cap r_2 \neq \varnothing$, entonces
    $r_1 \vee r_2 = V \subseteq \Po^n$, donde $V$ tiene dimensión
    2 por Grassman y por lo tanto se puede 
    aplicar \ref{primer_teo_poncelet}.
\end{obs}

\section{Clasificación y estudio de homografías}

\subsection{Clasificación de homografías}

\begin{defi}
	Sean $f, g : \Po \to \Po$ homografías $\lp f = [\varphi], g = [\psi], \varphi, \psi : \E \to \E \text{ endomorfismos}\rp.$ Decimos que $f \sim g$ si y solo si (son equivalentes):
	\begin{enumerate}[(i)]
		\item $\exists h$ homografía $\tq g = \inv{h}fh$
		\item $\exists \eta : \E \to \E$ endomorfismo, $\exists \lambda \in \k, \lambda \neq 0 \tq \psi = \lambda \inv{\eta} \varphi \eta$
		\item $\exists S \in \mathcal{M}_{m+1, n+1}\lp\k\rp, \exists \lambda\in\k, \lambda \neq 0 \tq B = \lambda \inv{S} A S$, donde
		\begin{itemize}
			\item $A = M\lp f; \R\rp = M\lp f; \R,\R\rp$
			\item $B = M\lp g; \R\rp = M\lp g; \R,\R\rp$
		\end{itemize}
		\item $\exists \R_1,\R_2\tq M\lp f;\R_1\rp = M \lp g; \R_2\rp$
	\end{enumerate}
\end{defi}
\begin{obs}
	$B = \lambda \inv{S} A S = \inv{S} \lp \lambda A \rp S \iff B$ y $\lambda A$ son equivalentes como matrices de endomorfismos $\iff$
	\[
		\substack{Q_A\lp t \rp \text{ descompone} \\ \iff \\ \text{totalmente}}
		\begin{cases}
			\{ \text{vaps } B \} = \lambda \overbrace{\{ \text{vaps } A \}}^{\mu_i} \\
			\forall i, \dim \lp \text{Nuc}\lp B - \lambda \mu_i I\rp^k \rp = \dim \lp\text{Nuc}\lp A - \mu_i I \rp ^k \rp, \forall k
		\end{cases}
	\]
	Es la misma estructura que en las cajas de Jordan.
\end{obs}
\begin{example}
	\begin{itemize}
		\item[]
		\item $f \to A= M_\R \lp f \rp \to Q_A\lp t \rp = \lp t-2 \rp^2\lp t-4 \rp\lp t-6 \rp^3$
		\item $g \to B= M_\R \lp g \rp \to Q_B\lp t \rp = \lp t-1 \rp^2\lp t-2 \rp\lp t-3 \rp^3$
	\end{itemize}
	\[
		J_A = 
		\begin{pmatrix}
			2 & 0 &   &   &   &   \\
			1 & 2 &   &   &   &   \\
			  &   & 4 &   &   &   \\
			  &   &   & 6 & 0 &   \\
			  &   &   & 1 & 6 &   \\
			  &   &   &   &   & 6 \\
		\end{pmatrix}; \qquad
		J_B =
		\begin{pmatrix}
			1 & 0 &   &   &   &   \\
			1 & 1 &   &   &   &   \\
			  &   & 2 &   &   &   \\
			  &   &   & 3 & 0 &   \\
			  &   &   & 1 & 3 &   \\
			  &   &   &   &   & 3 \\
		\end{pmatrix}
	\]
\end{example}

\subsection{Homografía dual y variedades $f$-invariantes}

\begin{defi}
	En $\Po = \Po \lp\E\rp$, sea $f = [\varphi] : \Po \to \Po$ una homografía, la homografía dual de $f$ es
	\[
		f^* := [\varphi^*] : \Po^* \to \Po^*.
	\]
\end{defi}
\begin{obs}
	$\Po, \R$ referencia $\to \lp\Po^*, \R^*\rp$
	\begin{enumerate}
		\item $f : \Po \to \Po, A = M \lp f; \R\rp$ mueve puntos de $\Po$. $f^* : \Po^* \to \Po^*, A^t = M \lp f^*; \R^* \rp$ mueve puntos de $\Po^*$ (hiperplanos de $\Po$).
		\item $J_A = J_{A^t}$, ya que
			\begin{itemize}
			\item $Q_A \lp t \rp = \det\lp A- t\Id\rp = \det\lp\lp A-t\Id\rp^t\rp = \det\lp A^t-t\Id\rp = Q_{A^t} \lp t\rp \implies$ vaps de $A =$ vaps de $A^t$ 
			\item $\dim\lp\text{Nuc}\lp A-\lambda_i\Id\rp^k\rp=\dim\lp\text{Nuc}\lp A^t-\lambda_i\Id\rp^k\rp$
			\end{itemize}
	\end{enumerate}
\end{obs}
\begin{defi}
	Sea $f = [\varphi]: \Po \to \Po$ homografía,
	\begin{enumerate}
		\item $p \in \Po$ es un punto fijo de $f \iff f\lp p \rp = p$.
		\item Sea $V \subseteq \Po$ una variedad lineal, $V$ es una variedad lineal de puntos fijos $\iff \forall p \in V, f\lp p\rp = p$.
		\item Sea $V \subseteq \Po$ una variedad lineal, $V$ es una variedad fija de $f$ (o es $f$-invariante) $\iff f\lp V\rp \subseteq V \stackrel{f \text{ biyectiva}}{\iff} f \lp V \rp = V$.
	\end{enumerate}
\end {defi}
\begin{obs}
	\begin{enumerate}
		\item[]
		\item $p = [v]$ es un punto fijo de $f \iff f\lp p\rp = [\varphi \lp v\rp] = [v] = p \iff \exists\lambda\neq 0 \tq \varphi\lp v\rp=\lambda v \iff v$ es vep de $\varphi$.
		\item $V=[F]$ variedad de puntos fijos $\iff \forall v \in F, v$ es vep de $\varphi \iff \exists\lambda\neq 0 \tq \forall v \in F, \varphi\lp v\rp = \lambda v \iff \exists \lambda \neq 0 \tq F \subseteq \text{Nuc}\lp f - \lambda\Id\rp$.
		\item $V = [F]$ es $f$-invariante $\iff F$ es $\varphi$-invariante.
	\end{enumerate}
\end{obs}
\begin{example}
	Queremos encontrar las variedades $f$-invariantes (o los subespacios $\varphi$-invariantes). \\
	Sea $f:\Po^4\to\Po^4$ y sea $\R = \{p_0, p_1,p_2,p_3,p_4;\overline{p}\}$ y sea
	\[
		A = M\lp f;\R\rp =
		\begin{pmatrix}
			2 & 0 &   &   &   \\
			1 & 2 &   &   &   \\
			  &   & 3 &   &   \\
			  &   &   & 3 & 0 \\
			  &   &   & 1 & 3 \\
		\end{pmatrix}
	\]
	Tenemos que:
	\begin{itemize}
		\item Puntos fijos: $p_1, p_2 \vee p_4$ (recta de puntos fijos).
		\item Rectas invariantes: $p_2 \vee p_4, p_0 \vee p_1, p_3 \vee p_4$.
		\item Planos invariantes: $p_0 \vee p_1 \vee p_2, p_2 \vee p_3 \vee p_4, p_1 \vee p_3 \vee p_4, p_0 \vee p_1 \vee p_4$.
	\end{itemize}
	Veamos ahora los hiperplanos fijos: $f\lp H\rp = H \iff f^*\lp H^*\rp = H^*$
	\begin{gather*}
		H:a_0x_0+\dots+a_nx_n=0 \iff H^* = \lp a_0:\dots:a_n\rp^t \\
		A^t= M_{\R^*}\lp f^*\rp = \begin{pmatrix}
			2 & 1 &   &   &   \\
			0 & 2 &   &   &   \\
			  &   & 3 &   &   \\
			  &   &   & 3 & 1 \\
			  &   &   & 0 & 3 \\
		\end{pmatrix}; \qquad
		\begin{array}{c}
			\R^* = \{p_0^*,p_1^*,p_2^*,p_3^*,p_4^*;\overline{p}\} \\
			B^* = \{\omega_0,\dots,\omega_4\}, \; \omega_i = v_i^*
		\end{array}
	\end{gather*}
	Encontramos los veps de $A^t$: \\
	\indent $\omega_0 \lambda_1 = 2 \rightarrow \omega_0 = \lp 2:0:\dots:0\rp^t$, \\
	\indent $\begin{rcases} \omega_2 \\ \omega_3 \end{rcases} \lambda_2 = 3 \implies [\omega_2, \omega_3]$ veps de vap $3, a\lp0:1:0:0:0\rp^t + b\lp0:0:1:0:0\rp^t = \lp 0:a:b:0:0\rp^t$. \\
	Por lo tanto,
	\begin{itemize}
		\item Hiperplanos invariantes:
		\begin{itemize}
			\item $H_0 : x_0 = 0$
			\item $H_{a,b} : ax_1 + bx_2 = 0, \forall a, b \in \real$
		\end {itemize}
	\end{itemize}
\end{example}
\begin{prop}
	Sea $f : \overline{\Po} \to \overline{\Po}$ homografía,
	\begin{enumerate}[(1)]
		\item \label{item:1_homografias} $p = [v]$ es un punto fijo de $f \iff v$ vep de $f$, es decir, de $A$,
		\item \label{item:2_homografias} $H$ es un hiperplano $f$-invariante $\iff H^*$ vep de $f^*$, es decir, de $A^t$,
		\item Hay una biyección entre puntos fijos e hiperplanos $f$-invariantes,
		\item $V_1, V_2\; f$-invariantes $\implies V_1 \cap V_2, V_1 \vee V_2$ son $f$-invariantes,
		\item \label{item:5_homografias} $V$ es $f$-invariante $\iff V^*$ es $f^*$-invariante,
		\item Si $Q_\varphi \lp t\rp$ descompone totalmente, $V$ $f$-invariante $\implies \exists p\in \overline{\Po}$ punto fijo, $\exists H\subseteq\overline{\Po}$ hiperplano $f$-invariante $\tq p \in V \subseteq H$.
	\end{enumerate}
\end{prop}
\begin{proof}
	\begin{enumerate}[(1)]
		\item[]
		\item Visto.
		\item Por dualidad de \eqref{item:1_homografias} y \eqref{item:5_homografias}.
		\item $\begin{rcases}
			Q_A\lp t\rp = Q_A\in\lp t\rp\\
			J_A = J_{A^t}
		\end{rcases}
		\implies \dim\lp\text{Nuc}\lp A-\lambda\Id\rp\rp=\dim\lp\text{Nuc}\lp A^t - \lambda\Id\rp\rp$.
		\item $F_1, F_2$ son $\varphi$-invariantes $\implies F_1 \cap F_2, F_1 + F_2$ son $\varphi$-invariantes.
		\item Veamos cada implicación por separado \\
		$\bimplies$ \\
		$V^* = \{ H\subseteq\Po | V \subseteq H\}$. Sea $H_0\in V^* \implies V \subseteq H_0 \implies f\lp V\rp\subseteq f\lp H_0\rp\implies f^*\lp H_0^*\rp = f\lp H_0\rp \in V^*\implies V^*$ es $f^*$-invariante. \\ \\
		$\bimpliedby$ \\
		Dual de la otra impliecación: $\lp f^*\rp^* = f, \lp V^*\rp^*= V$.
		\item $\exists p\in\Po$ punto fijo $\tq p \in V, V = [F] \subseteq\Po = \Po\lp\E\rp, f=[\varphi]$. $V \; f$-invariante $\iff F\; \varphi$-invariante. \\
		$F \subseteq \E\; \varphi$-invariante $\implies Q_{\varphi_{|F}}\lp t\rp$ divide a $Q_\varphi\lp t \rp$. En este caso,
		\begin{align*}
			Q_\varphi\lp t\rp &= \lp-1\rp^n\lp t-\lambda_1\rp^{\delta_1} \dots \lp t-\lambda_r\rp^{\delta_r} \implies \\
			Q_{\varphi_{|F}}\lp t\rp &= \lp-1\rp^d\lp t-\lambda_1\rp^{\mu_1}\dots\lp t-\lambda_r\rp^{\mu_r}, 0\leq\mu_i\leq\delta_i.
		\end{align*}
		Por tanto $\exists \lambda_i$ vap de $\varphi_{|F}$ y por tanto $\exists \omega_i\in F, \varphi\lp\omega_i\rp=\lambda_i\omega_i \implies f\lp p\rp=p\;\lp p = [\omega_i]\in V\rp.$ \\
		Veamos que $\exists H\subseteq \Po$ hiperplano $f$-invariante t.q. $V\subseteq H$,
		\[
			V \subseteq H \iff \overbrace{V^*}^{f\text{-inv. por }\eqref{item:5_homografias}} \supseteq \overbrace{H^*}^{f\text{-inv. por }\eqref{item:2_homografias}}
		\]
	\end{enumerate}
\end{proof}
\begin{example}
	Sea $f : \Po^3 \to \Po^3$, $\R$ una referencia de $\Po^3$.
	\[
		A = M_\R\lp f\rp = \begin{pmatrix}
			0 & -1 & & \\
			1 & 2 & & \\
			& & -1 & \\
			& & & -1
		\end{pmatrix}
	\]
\end{example}

\subsection{Algunas familias de homografías}

\subsubsection{Involuciones}

\begin{defi}
	Sea $f \colon \Po \to \Po$ una homografía,
	\[ f\text{ es una involución} \iff f \neq \Id, f^2 = \Id \]
\end{defi}
\begin{obs}
	\[
		f = [\varphi]\text{ es involución} \iff
		\begin{cases}
			\varphi \neq \Id  \\
			\exists\lambda\tq\varphi^2=\lambda\Id & \lp \lambda \neq 0 \rp
		\end{cases}
	\]
\end{obs}
\begin{obs}
	Sea $\k = \real, m_\varphi\lp t\rp | \lp t^2-\lambda\rp$
	\begin{itemize}
		\item Caso 1: $\lambda<0 \rightarrow$ nada 
		\item Caso 2: $\lambda = a^2>0 \rightarrow m_\varphi\lp t\rp = \text{ó}
		\begin{cases}
			\lp t-a\rp \rightarrow \varphi -a\Id=0\implies\varphi=a\Id \rightarrow \text{ No} \\
			\lp t+a\rp \rightarrow \varphi +a\Id=0\implies\varphi=-a\Id \rightarrow \text{ No} \\
			\lp t^2-a^2\rp = \lp t-a\rp\lp t+a\rp
		\end{cases}$
	\end{itemize}
	Conclusión: $\varphi$ diagonaliza.
	Por tanto, $\exists B$ t.q.:
	\begin{gather*}
		M_B\lp\varphi\rp =
		\begin{pmatrix}
			a & & & & & \\
			& \ddots & & & & \\
			& & a & & & \\
			& & & -a & & \\
			& & & & \ddots & \\
			& & & & & -a
		\end{pmatrix}, \quad
		M\lp f;\R\rp \substack{\text{Matriz de } \varphi \\=\\ \text{salvo const.}} \begin{pmatrix}
			1 & & & & & \\
			& \ddots & & & & \\
			& & 1 & & & \\
			& & & -1 & & \\
			& & & & \ddots & \\
			& & & & & -1\\
		\end{pmatrix}
	\end{gather*}
	$\R = \{q_0 = [v_0], \dots, q_r=[v_r], q_{r+1} = [v_{r+1}], \dots, q_n = [v_n]; \overline{q}\}$.\\
	Sean $V = q_0 \vee \dots q_r, W = q_{r+1} \vee \dots \vee q_n$ de puntos fijos, $V$ y $W$ suplementarias.
	Sea $q \in \Po$.
	\begin{itemize}
		\item $q \in V$ o $q \in W \implies f\lp q\rp=q$.
		\item $q \notin V, W$. Como $V,W$ son suplementarias, $\exists! L$ recta t.q. $q\in L, L\cap V=\{q_0\}, L\cap W = \{q_1\}$. Como $q_0\in V, q_1 \in W \implies q_0, q_1$ puntos fijos $\implies l = q_o \vee q_1$ es $f$-invariante. \\
		$q \in L \implies f\lp q\rp \in L$,
	\end{itemize}
	\[\lp q_0, q_1, q, f\lp q\rp\rp = \lp f\lp q_0\rp, f\lp q_1\rp, f\lp q\rp, q\rp = \lp q_0, q_1, f\lp q\rp, q\rp \implies \lp q_0, q_1, q, f\lp q\rp\rp = -1.\] % aixo es un apanyo per a que no es surti de la pagina, pero hauria de ser dins del itemize
\end{obs}
\begin{example}
	Sea $f \colon \Po^3 \to \Po^3$ \\
	\[
		A = M\lp f; \R\rp = \begin{pmatrix}
		0 & -1 & & \\
		1 & 2 & & \\
		& & -1 & \\
		& & & -1
		\end{pmatrix}, \quad \begin{array}{l}
		\R = \{p_0, p_1, p_2, p_3; \overline{p}\} \\
		B = \{v_0, v_1, v_2, v_3\}
		\end{array}
	\]
	$Q_A = \lp t+1\rp^2\lp t-1\rp^2$, $\lambda_1 = -1, \delta_1 = 2 $ (multiplicidad algebraica), $\lambda_2 = 1, \delta_2 = 2$. Veps: $\left[ \lp0:0:1:0\rp^t, \lp0:0:0:1\rp^t \right], \left[ \lp1:-1:0:0\rp^t \right]$.
	\begin{itemize}
		\item Puntos fijos: $p_2 \vee p_3$ (recta), $q =\lp 1:-1:0:0\rp$.
		\item Planos fijos: \\
		$\R^\alpha = \left\{ [\omega_0], [\omega_1], [\omega_2], [\omega_3]; [\overline{\omega}]\right\}, \omega_i = v_i^\alpha$.
		\[
			A^t = \begin{pmatrix}
				0 & 1 & & \\
				-1 & 2 & & \\
				& & -1 & \\
				& & & -1
			\end{pmatrix}, \quad
			\begin{array}{ll}
				\lambda_1 = -1 & \left[ \omega_2, \omega_3 \right] = \left\{ \omega \in E^\alpha | \omega = \lp 0:0:A:B\rp^t \right\} \\
				\lambda_2 = 1 & \left[\lp1:1:0:0\rp^t\right]
			\end{array}
		\]
		$H_1 \colon x_0 + x_1 = 0$ \\
		$H_{A,B} \colon Ax_2 + Bx_3 = 0$\\
		Por tanto, vemos que hay una biyección entre puntos e hiperplanos fijos.
		\item Rectas fijas: \\
		$L = p_2 \vee p_3$ (de puntos fijos).\\
		Usaremos que $v_1,v_2$ fijas $\implies v_1 \vee v_2$ fijas: $L_{\alpha,\beta} = \left[\lp1:-1:0:0\rp^t\right]\vee\left[\lp0:0:\alpha:\beta\rp^t\right]$.
		Usaremos que $v_1, v_2$ fijas $\implies v_1 \cap v_2$ fijas:
		\begin{gather*}
			H_{A=1,B=0} \cap H_{A=0,B=1} = \begin{Bmatrix}
				x_2 = 0 \\
				x_3 = 0
			\end{Bmatrix} = p_0 \vee p_1 \\
			H \cap H_{A,B} = \begin{Bmatrix}
				x_0 + x_1 = 0 \\
				Ax_2 + Bx_3 = 0
			\end{Bmatrix} \implies \left[\lp1:-1:0:0\rp^t\right] \vee \left[\lp0:0:B:-A\rp^t\right] = L_{B,-A}.
		\end{gather*}
	\end{itemize}
\end{example}
\begin{obs}[Método para encontrar todas las rectas $f$-invariantes]
	Tambo variedades de dim $n-2$)
	\begin{itemize}
		\item Opción 1 (sistemática): $L$ $f$-invariante $\implies \exists p,\pi$, un punto y un hiperplano, respectivamente, $f$-invariantes $\tq p\in L\subseteq\pi$. Nos dice que todas las variedades de dim $n-2$ viven dentro de hiperplanos fijos. Por tanto, $\forall \pi$ $f-$invariante, consideramos $f_{|\pi}$ y, como $L$ son hiperplanos dentro de $\pi$, encontramos $L$ $f$-invariantes como los veps de $\lp f_{|\pi}\rp^*$.
		\item Opción 2: En todo hiperplano $f$-invariante $\pi$, tiene que haber el mismo número de puntos fijos que de variedades de dim $n-2$ $f$-invariantes (hay una biyección entre puntos e hiperplanos fijos de $\pi$).
	\end{itemize}
\end{obs}
\begin{example}
	\begin{gather*}
		\pi = H \colon x_0 + x_1 = 0 \equiv \left[\lp1:-1:0:0\rp^t\right] \vee \left[\lp0:0:1:0\rp^t\right] \vee \left[\lp0:0:0:1\rp^t\right] \\
		C = M\lp f_{|H}\rp = M\lp\varphi_{|F}\rp =
		\begin{pmatrix}
			1 & 0 & 0 \\
			0 & -1 & 0 \\
			0 & 0 & -1
		\end{pmatrix} \\
		L \subseteq \pi\;f\text{-invariante }\iff\text{ su ecuación en }\pi\text{ es un vep de }C^t\text{ (hiperplano en }\pi\text{).}
	\end{gather*}
	Inductivamente podríamos encontrar variedades de dim $< n-2$ considerándolas dentro de variedades fijas de dim $1$. Además, siendo así hiperplanos que serán fijos $\iff$ son veps de la aplicación dual de la restricción.
\end{example}
\begin{example}[(Involuciones)]
	\begin{enumerate}
		\item[]
		\item $\Po^1$ \\
			\begin{gather*}
			A = M\lp f; \R\rp =
			\begin{pmatrix}
				3 & 1 \\
				7 & -3
			\end{pmatrix}, \begin{array}{cc}
				\lambda_1 = 4 & v_1 = (1:1) \\
				\lambda_2 = -4 & v_2 = (1:-7)
			\end{array} \implies \\
			\exists\R \overline{A} = M\lp f; \R\rp =
			\begin{pmatrix}
				4 & 0 \\
				0 & -4
			\end{pmatrix} \sim
			\begin{pmatrix}
				1 & 0 \\
				0 & -1
			\end{pmatrix} \\
			A^2 = 16\Id \implies f^2 = Id.
		\end{gather*}
		\item $A = M\lp f;\R\rp =
		\begin{pmatrix}
			0 & -1 \\
			1 & 0
		\end{pmatrix}$,
		\[
			A^2 = \begin{pmatrix}
				-1 & 0 \\
				0 & -1
			\end{pmatrix}
			= -\Id \implies f^2 = \Id \implies f\text{ involución}
		\]
		A no diagonaliza, $Q_A \lp t\rp = t^2+1$.
	\end{enumerate}
\end{example}
\begin{obs}
	Si $\k = \cx, f^2 = \Id \implies f$ siempre diagonaliza:
	\[
		\begin{pmatrix}
			a & & & & & \\
			& \ddots & & & & \\
			& & a & & & \\
			& & & -a & & \\
			& & & & \ddots & \\
			& & & & & -a \\
		\end{pmatrix}.
	\]
	Es un caso diagonalizable de $\k = \real$.
\end{obs}

% A PARTIR D'AQUI JORDI

\subsubsection{Homologías}

\begin{defi}
    Sea $f \colon \Po \to \Po$ una homografía $\lp f \neq \Id \rp$
    \begin{itemize}
        \item $f$ es una homología $\iff f$ tiene un hiperplano de puntos fijos.
        \item $f$ es una homología general $\iff f$ es una homología y tiene, además, otro punto fijo.
        \item $f$ es una homología especial $\iff f$ es un a homología y no es una homología especial.
    \end{itemize}
\end{defi}

\begin{prop}
    Sea $f \colon \Po \to \Po$ una homología. Entonces, $\exists \R$ referencia tal que $A=M\lp f ; \R \rp$.
    \begin{itemize}
        \item $f$ es una homología general $ \iff A=
            \begin{pmatrix}
                1 & & & \\
                & \ddots & & \\
                & & 1 & \\
                & & & a \\
            \end{pmatrix}, a \neq 1.$
        \item $f$ es una homología especial $ \iff A=
            \begin{pmatrix}
                1 & 0 & & \\
                1 & 1 & & \\
                & & \ddots &  \\
                & & & 1 \\
            \end{pmatrix}.$
    \end{itemize}
\end{prop}

\begin{proof}
    $f=\left[ \varphi \right]$. $\Po = \Po \lp \E \rp$; $\dim \Po = n, \; \dim \E = n+1$.
    \[
        Q_{\varphi} \lp t \rp = \lp -1 \rp ^{n+1} \underbrace{\lp t-\lambda \rp ^{n}}_{\substack{\text{hay un hiperplano}\\ \text{de puntos fijos}}} \lp t-\mu \rp, \; \lambda \neq 0, \; \gr \lp Q_{\varphi} \lp t \rp \rp = n+1.
    \]
    \begin{itemize}
        \item Si $\mu \neq \lambda$:
            \begin{gather*}
                \left\{
                \begin{array}{cccccc}
                    \lambda & \rightarrow & m_a=n & \overbrace{m_g \geq n}^{\dim \nuc \lp \varphi - \lambda \Id \rp \geq n} & \implies & m_g=n \\
                    \mu & \rightarrow & m_a=1 & m_g = 1 & &  
                \end{array} \right\} \substack{\text{Teorema de} \\ \implies \\ \text{diagonalización}} \\
                \implies \varphi \; \text{diagonaliza} \implies \; \exists B \tq A=M \lp \varphi ; B \rp = 
                \begin{pmatrix}
                    \lambda & & & \\
                    & \ddots & & \\
                    & & \lambda & \\
                    & & & \mu
                \end{pmatrix} = \\
                = \begin{pmatrix}
                    1 & & & \\
                    & \ddots & & \\
                    & & 1 & \\
                    & & & a
                \end{pmatrix}, \; a\neq 1 \text{ (otro punto fijo) } \implies f \text{ es una homología general}.
            \end{gather*}
        \item Si $\mu = \lambda, \; Q_{\varphi} \lp t \rp = \lp -1 \rp ^{n+1} \lp t-\lambda \rp^{n+1}$. Entonces, 
        $n+1\geq \dim \lp \varphi - \lambda \Id \rp \geq n $.
        \begin{itemize}
            \item{$\dim \lp \varphi - \lambda \Id \rp = n+1$}
                \[
                    \varphi \text{ diagonaliza } \implies \exists B \tq A=\lambda \Id \implies f= \Id \to \text{ contradicción}.
                \]
            \item{$\dim \lp \varphi - \lambda \Id \rp = n$}
                \begin{gather*}
                    \exists B \tq M \lp \varphi, B \rp =
                    \begin{pmatrix}
                        \lambda & 0 &  &  &  \\
                        1 & \lambda &  &  &  \\
                         &  & \lambda &  &  \\
                         &  &  & \ddots &  \\
                         &  &  &  & \lambda
                    \end{pmatrix} \implies \\
                    \implies \exists \R \tq M \lp \varphi, \R \rp =
                    \begin{pmatrix}
                        1 & 0 &  &  &  \\
                        1 & 1 &  &  &  \\
                         &  & 1 &  &  \\
                         &  &  & \ddots &  \\
                         &  &  &  & 1
                    \end{pmatrix} \implies \\
                    \text{no hay más puntos fijos } \implies f \text{ es una homología especial}
                \end{gather*}
        \end{itemize}
    \end{itemize}
\end{proof}

\subsection{Homografías de $\Po_{\real}^1$}
Posibles formas de Jordan:
\begin{enumerate}[(1)]
    \item $f=\Id, A= \begin{pmatrix} 1 & 0 \\ 0 & 1 \end{pmatrix}$ \\
        $A$ diagonaliza y tiene un solo valor propio. Todos los puntos son fijos.
    \item $A= \begin{pmatrix} 1 & 0 \\ 0 & a \end{pmatrix}, a \neq 1$ \\
        $A$ diagonaliza y tiene dos valores propios distintos. Hay dos puntos fijos, $f$ es una homología general. Para $a=-1$, $f$ es una involución.
    \item $A= \begin{pmatrix} 1 & 0 \\ 1 & 1 \end{pmatrix}$ \\
        $A$ no diagonaliza y tiene un solo valor propio. Hay un punto fijo, $f$ es una homología especial.
    \item $A= \begin{pmatrix} a & b \\ c & d \end{pmatrix}$ \\
        $Q_A \lp t \rp = t^2 - \lp \tr A \rp t + \det A \implies \Delta = \lp \tr A \rp ^2 - 4 \det A < 0$. A no tiene forma de Jordan. No hay ningún punto fijo.
\end{enumerate}
\begin{obs}
    Las homografías en $\Po_{\real}^2$ pueden clasificarse en siete categorías.
\end{obs}

% A PARTIR D'AQUI ERIC

\section{Relación entre afinidades y proyectividades}
    \begin{minipage}[c]{0,49\textwidth}
        \begin{align*}
                f : \A^n &\to \bar{\A}^n=\Po^n \\
                \R &\mapsto \overline{\R} \\
                \lp x_1, \ldots, x_n\rp &\mapsto \lp1:x_1:\ldots:x_n\rp\\
                \lp\frac{\tilde{x}_1}{x_0},\ldots,\frac{\tilde{x}_n}{x_0}\rp &\mapsfrom \lp\tilde{x}_0:\ldots:\tilde{x}_n\rp \\
        \end{align*}
    \end{minipage}
    \vline
    \begin{minipage}[c]{0,49\textwidth}
        \begin{gather*}
            f : \A^n \to \bar{\A}^n  \text{  afinidad (biyectiva) }\\
            f\lp x_1,\ldots,x_n\rp = 
            \begin{pmatrix}
                a_1 \\
                \vdots \\
                a_n \\
            \end{pmatrix}
            +A
            \begin{pmatrix}
                x_1 \\
                \vdots \\
                x_n \\
            \end{pmatrix} = \\ =
            \underbrace{\lp
            \begin{array}{c|ccc}
                1 & 0 & \ldots & 0 \\ \hline
                a_1 & & & \\
                \vdots & & A & \\
                a_n & & & 
            \end{array} \rp}_{\overline{A}}
            \begin{pmatrix}
                1\\
                x_1\\
                \vdots\\
                x_n\\
            \end{pmatrix}
        \end{gather*}
    \end{minipage} \\ \\
    \subsection{Las afinidades definen proyectividades}
    \begin{gather*}
    f\text{ biyectiva} \leftrightarrow \det A \ne 0 \leftrightarrow \det \overline{A} \ne 0 \leftrightarrow \overline{A} \text{ invertible}
    \end{gather*}
    \begin{defi}
        Llamaremos proyectividad asociada a $f$ a una $\overline{f} \tq$:
        \begin{gather*}
            \overline{f} \colon \bar{\A}^n = \Po^n \mapsto \bar{\A}^n = \Po^n \\
            M_{\overline{\R}}\lp \overline{f}\rp = \overline{A} = 
            \lp
            \begin{array}{c|ccc}
                1 &  & v & \\ \hline
                a_1 & & & \\
                \vdots & & A & \\
                a_n & & & 
            \end{array} \rp \\ 
        \end{gather*}
        donde $v$ es un vector cualquiera.
    \end{defi}
    \begin{prop}
        \begin{enumerate}
            \item[]
            \item $\A_\infty$ es $\overline{f}$-invariante
            \item $\A^n = \Po^n \setminus \A_\infty^n$ es $\overline{f}$-invariante i $\overline{f}_{|\A^n}=f$
            \item $M_{\overline{\R}_\infty}\lp \overline{f}_{|\A_\infty} \rp = A$
        \end{enumerate}
    \end{prop}
    \begin{proof}
        \begin{enumerate}
            \item[]
            \item $\A_\infty^n$: $x_0=0 \rightarrow \overline{f}$-invariante porque $\lp1,0,\ldots,0\rp$ es vector propio de $\overline{A}$
            \item Sea $p\in\A^n\subseteq\Po^n$, $p=\lp1:b_1:\ldots:b_n\rp$. Entonces:
            \begin{gather*}
                \overline{f}\lp p \rp = \overline{A} 
                \begin{pmatrix}
                    1 \\
                    b_1\\
                    \vdots \\
                    b_n\\
                \end{pmatrix}
                = 
                \lp
            \begin{array}{c|ccc}
                1 &  & v &  \\ \hline
                a_1 & & & \\
                \vdots & & A & \\
                a_n & & & 
            \end{array} \rp
            \begin{pmatrix}
                1\\
                b_1\\
                \vdots\\
                b_n\\
            \end{pmatrix}
            = f\lp b_1,\ldots,b_n \rp\in \A^n
            \end{gather*}
            \item Como que en $\A_\infty^n$, $x_0 =0$, y mientras que $\overline{\R}=\{p_0, p_1, \dots, p_n; \overline{p}\}$, tenemos que para $\overline{\R}_\infty = \{p_1, \dots, p_n, \overline{p}_\infty\}$ hay una coordenada menos debido a que se elimina $x_0$, y por lo tanto:
            \begin{gather*}
                \overline{\A} = M_{\overline{\R}}\lp \overline{f}\rp = \lp
                \begin{array}{c|ccc}
                    1 &  & v & \\ \hline
                    a_1 & & & \\
                    \vdots & & A & \\
                    a_n & & & 
                \end{array} \rp 
                \implies 
                M_{\overline{\R}_\infty}\lp \overline{f}_{|\A_\infty} \rp = A
            \end{gather*}
        \end{enumerate}
    \end{proof}
\subsection{Las proyectividades definen afinidades}
Sea $g: \Po^n \to \Po^n$ una proyectividad, sea $\overline{A}=M_{\overline{\R}}\lp g\rp$ y supongamos $H$ un hiperplano $g$-invariante. \\
    \begin{minipage}[c]{0,44\textwidth}
        \begin{align*}
            \A^n = \Po^n\setminus H &\to  \Po^n = \bar{\A}^n \\
            \R &\mapsto  \underbrace{\overline{\R}= \{\overbrace{p_0}^{\notin H}, \overbrace{p_1,\dots, p_n}^{\in H};\overline{p}\}}_{H\text{: }x_0=0}
        \end{align*}
    \end{minipage}
    \vline
    \begin{minipage}[c]{0,56\textwidth}
        \begin{gather*}
            \overline{\A}=\lp
                \begin{array}{c|ccc}
                    \lambda & 0 & \dots & 0 \\ \hline
                    \tilde{a}_1 & & & \\
                    \vdots & & \tilde{A} & \\
                    \tilde{a}_n & & & 
                \end{array} \rp 
                \substack{\sfrac{1}{\lambda} \\ \sim \\ \lambda\neq0}
            \lp
                \begin{array}{c|ccc}
                    1 & 0 & \dots & 0 \\ \hline
                    a_1 & & & \\
                    \vdots & & A & \\
                    a_n & & & 
                \end{array} \rp
        \end{gather*}
    \end{minipage}\\
    \begin{defi}
        Llamaremos $g_\text{af} : \A^n \to \A^n$ a la afinidad inducida por $g$ en $\A^n = \Po^n\setminus H$ y tenemos que:
        \begin{gather*}
            M_\R \lp g_\text{af} \circ f\rp =
            \lp
                \begin{array}{c|ccc}
                    1 & 0 & \dots & 0 \\ \hline
                    a_1 & & & \\
                    \vdots & & A & \\
                    a_n & & & 
                \end{array} \rp
        \end{gather*}
    \end{defi}
    \begin{example}
        \begin{enumerate}
            \item[]
            \item Consideramos la proyectividad $g:\Po^2_\real \to \Po^2_\real$ y una referencia $\overline{\R} \tq$: \\
            \[M_{\overline{\R}}\lp g\rp=
            \begin{pmatrix}
                1 & & \\
                1 & 1 & \\
                 & & 1 \\
            \end{pmatrix}\]
            que tiene como valores y vectores propios:
            \begin{gather*}
                \lambda_1 = 1 \implies v_1 = \lp 1,0,0 \rp \implies H_1\text{: }x_0=0\\
                \lambda_2 = 1 \implies v_2 = \lp 0,0,1 \rp \implies H_2\text{: }x_2=0
            \end{gather*}
            \item
        \end{enumerate}
    \end{example}
    
% A PARTIR D'AQUI MIQUEL

% A PARTIR D'AQUI ERNESTO
