\section{Proyectividades}
\subsection{Definiciones y propiedades básicas}
\begin{defi}
	\begin{enumerate}
		\item[]
		\item Sean $\Po = \Po(\E), \bar{\Po} = \Po(\bar{\E})$ espacios proyectivos de dimensión $n$. Sea $\varphi : \E \rightarrow \bar{\E}$ una aplicación lineal biyectiva. Una proyectividad, definida por $\varphi$, es una aplicación
		\begin{align*}
			f := [\varphi] : \Po &\rightarrow \bar{\Po} \\
			p = [v] &\mapsto f(p) = [\varphi(v)]
		\end{align*}
		Diremos que $\varphi$ es un representante de $f$.
		\item Si $\Po = \bar{\Po}$, llamaremos homografías a las proyectividades.
	\end{enumerate}
\end{defi}
\begin{obs}
	$\varphi$ no inyectiva $\implies [\varphi]$ no es una aplicación de $\Po(\E)$ en $\Po(\bar{\E})$.
\end{obs}
\begin{obs}
	$[\varphi] = [\psi] \iff \exists \lambda \neq 0$ tal que $\psi = \lambda \varphi$.
\end{obs}
\begin{proof}
	$\impliedby$ \\
	$\exists \lambda \neq 0$ tal que $\psi = \lambda\varphi$, sea $p = [v]$, $[\psi](p) = [\psi(v)] = [\lambda\varphi(v)] = [\varphi(v)] = [\varphi](p), \forall p \implies [\psi] = [\varphi].$ \\ \\
	$\implies$ \\
	Sean $u, v \in \E$ linealmente independientes, y sean $p = [u], q = [v]$. Entonces, como $[\varphi(u)] = [\psi(u)]$ y $[\varphi(v)] = [\psi(v)]$,
	\begin{gather*}
		\exists \lambda_1 \neq 0 \text{ tal que } \psi(u) = \lambda_1 \varphi(u) \\
		\exists \lambda_2 \neq 0 \text{ tal que } \psi(v) = \lambda_2 \varphi(v).
	\end{gather*}
	Por otro lado,
	\begin{gather*}
		[\psi(u+v)] = [\varphi(u+v)] \implies \exists \lambda_3 \neq 0 \text{ tal que } \psi(u) + \psi(v) = \psi(u+v) = \lambda_3 \varphi(u+v) = \\ \lambda_3 \varphi(u) + \lambda_3 \varphi(v) \implies (\lambda_1 - \lambda_3) \varphi(u) + (\lambda_2-\lambda_3) \varphi(v) = 0 \implies \lambda_1 = \lambda_2 = \lambda_3 =: \lambda \neq 0. \\
		\psi(u) = \lambda \varphi(u), \forall u \in \E \implies \exists \lambda \neq 0 \text{ tal que } \psi = \lambda \varphi.
	\end{gather*}
\end{proof}
\begin{prop}
	\begin{enumerate}
		\item[]
		\item Operaciones
		\begin{itemize}
			\item $\Id_\Po = [\Id_\E]$ es una proyectividad.
			\item $f, g$ proyectividades $\implies g \circ f$ proyectividad y $[\psi] \circ [\varphi] = [\psi \circ \varphi]$.
			\item $f = [\varphi]$ proyectividad $\implies f^{-1}$ proyectividad y $f^{-1} = [\varphi^{-1}]$.
		\end{itemize}
		\item Relación con variedades lineales \\
		Sea $f = [\varphi] : \Po \rightarrow \bar{\Po}$,
		\begin{itemize}
			\item $V = \pi(F)$ variedad lineal de $\Po$ de dimensión $d \implies f(V) = \pi(\varphi(F))$ es una variedad lineal de $\bar{\Po}$ de dimensión $d$.
			\item $V_1 \subseteq V_2 \iff f(V_1) \subseteq f(V_2).$
			\item $f(V_1 \vee V_2) = f(V_1) \vee f(V_2)$.
			\item $f(V_1 \cap V_2) = f(V_1) \cap f(V_2)$.
			\item En particular, $p_0, \dots, p_d$ linealmente independientes $\iff f(p_0), \dots, f(p_d)$ linealmente independientes.
		\end{itemize}
	\end{enumerate}
\end{prop}
\begin{proof}
	Inmediata a partir de las propiedades de $\varphi$ (biyectiva).
\end{proof}
\begin{col}
	Toda proyectividad es una colineación (mantiene la alineación de puntos).
\end{col}
\begin{col}
	Toda proyectividad mantiene las razones dobles.
\end{col}
\begin{proof}
	$\rho = (p_1, p_2, p_3, p_4) \implies \exists u,v \in \E$ tales que $p_1 = [u], p_2 = [v], p_3 = [u+v], p_4 = [\rho u + v] \implies f(p_1) = [\varphi(u)], f(p_2) = [\varphi(v)], f(p_3) = [\varphi(u) + \varphi(v)], f(p_4) = [\rho \varphi(u) + \varphi(v)] \implies \rho = (f(p_1), f(p_2), f(p_3), f(p_4))$.
\end{proof}
\begin{defi}
	La matriz de una proyectividad $f$ es $M_{\R, \bar{\R}} (f) := M_{B, \bar{B}}(\varphi)$.
\end{defi}
