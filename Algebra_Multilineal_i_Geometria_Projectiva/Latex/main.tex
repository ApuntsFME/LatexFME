\documentclass[12pt]{article}
 
\usepackage[margin=1in]{geometry}
\usepackage[pdftex]{hyperref}
\usepackage{amsmath,amsthm,amssymb,graphicx,mathtools,tikz,hyperref,enumerate}
\usepackage{mdframed,cleveref,cancel, stackengine}

\newmdenv[leftline=false,topline=false]{topright}
\let\proof\relax
\usepackage[utf8]{inputenc}
\usetikzlibrary{positioning}
\newcommand{\n}{\mathbb{N}}
\newcommand{\z}{\mathbb{Z}}
\newcommand{\q}{\mathbb{Q}}
\newcommand{\cx}{\mathbb{C}}
\newcommand{\real}{\mathbb{R}}
\newcommand{\E}{\mathbb{E}}
\newcommand{\Po}{\mathbb{P}}
\newcommand{\bb}[1]{\mathbb{#1}}
\let\k\relax
\newcommand{\k}{\mathbf{k}}
\newcommand{\ita}[1]{\textit{#1}}
\newcommand\inv[1]{#1^{-1}}
\newcommand\setb[1]{\left\{#1\right\}}
\newcommand{\vbrack}[1]{\langle #1\rangle}
\newcommand{\determinant}[1]{\begin{vmatrix}#1\end{vmatrix}}
\newcommand{\abs}[1]{\left\vert #1 \right\vert}
\DeclareMathOperator{\Id}{Id}
\DeclareMathOperator{\rg}{rg}


\hypersetup{
	colorlinks,
	linkcolor=blue
}
 
 \renewcommand*\contentsname{Contenidos}

\newtheoremstyle{break}% name
{}%         Space above, empty = `usual value'
{}%         Space below
{}% Body font
{}%         Indent amount (empty = no indent, \parindent = para indent)
{\bfseries}% Thm head font
{}%        Punctuation after thm head
{\newline}% Space after thm head: \newline = linebreak
{#1 #2 \normalfont #3}%         Thm head spec

\newtheoremstyle{breakthm}% name
{}%         Space above, empty = `usual value'
{}%         Space below
{}% Body font
{}%         Indent amount (empty = no indent, \parindent = para indent)
{\bfseries}% Thm head font
{}%        Punctuation after thm head
{\newline}% Space after thm head: \newline = linebreak
{#1 \normalfont #3 (#2)\addcontentsline{toc}{subsubsection}{#1 #3}}%         Thm head spec
\newtheoremstyle{normal}% name
{}%         Space above, empty = `usual value'
{}%         Space below
{}% Body font
{}%         Indent amount (empty = no indent, \parindent = para indent)
{\bfseries}% Thm head font
{}%        Punctuation after thm head
{5pt plus 1pt minus 1pt}% Space after thm head: \newline = linebreak
{#1 #2 \normalfont #3}%         Thm head spec

\theoremstyle{normal}
\newtheorem{lema}{Lema}[subsection]
\newtheorem{obs}[lema]{Observación}

\theoremstyle{break}
\newtheorem{prop}[lema]{Proposición}
\newtheorem*{proof}{Demostración}
\newtheorem{defi}[lema]{Definición}
\newtheorem{col}[lema]{Corolario}
\newtheorem{ej}[lema]{Ejercicio}
\newtheorem{example}[lema]{Ejemplo}

\theoremstyle{breakthm}
\newtheorem{thm}[lema]{Teorema}



 
\begin{document}
\date{}

\title{Álgebra Multilineal y Geometría Proyectiva}
 
\maketitle

\tableofcontents

\setcounter{section}{-1}

\chapter{Permutaciones}

\section{Repaso de permutaciones}

El grupo simétrico $\lp S_n, \circ \rp$ es el grupo de las permutaciones de los elementos
$\setb{1, 2, \dots, n}$ y el cardinal de $S_n$ es $\# S_n = \abs{S_n} = n!$

Si $\sigma \in S_n$, podemos escribir $\sigma$ como
\[
    \begin{pmatrix}
        1 & 2 & 3 & \cdots & n \\
        \sigma(1) & \sigma(2) & \sigma(3) & \cdots & \sigma(n)
    \end{pmatrix}.
\]

Cualquier permutación se descompone en ciclos, por ejemplo
$\sigma = (1, 4, 5, 2) \in S_5$ es lo mismo que
\[
    \sigma = 
    \begin{pmatrix}
        1 & 2 & 3 & 4 & 5 \\
        4 & 1 & 3 & 5 & 2
    \end{pmatrix}.
\]

Entonces $\sigma = \sigma_1 \cdots \sigma_r$, siendo $\sigma_i$ ciclos disjuntos.

\begin{obs}
    La multiplicación no es conmutativa. Pero las permutaciones con elementos disjuntos sí que conmutan.
\end{obs}

Todo ciclo se puede descomponer como producto de transposiciones $z = (i, j)$.
Por lo tanto, podemos descomponer toda permutación como producto de transposiciones, pero esta
descomposición no es única.
Lo que sí que se mantiene es la paridad del número de trasposiciones. Es decir,
\[
    \begin{rcases}
        \sigma = z_1 \cdots z_r \\
        \sigma = \bar{z}_1 \cdots \bar{z}_s
    \end{rcases} \implies \lp r \text{ par} \iff s \text{ par}\rp.
\]

Esto nos permite definir unequívocamente el signo de la permutación:
\[
    \sgn(\sigma) = (-1)^{r},
\]
donde $r$ es el número de trasposiciones de cualquiera de sus descomposiciones en transposiciones.

\begin{defi}[orden!de una permutación]
    Definimos el orden de una permutación $\sigma$ como el mínimo $k$ tal que $\sigma^k = \Id$.
\end{defi}

\begin{example}
    $\sigma = (1, 4, 5, 2)$. Calcular el orden de $\sigma$.
    \[
        \sigma^2 = \sigma \cdot \sigma =
        \begin{pmatrix}
            1 & 2 & 3 & 4 & 5 \\
            4 & 1 & 3 & 5 &2
        \end{pmatrix}
        \begin{pmatrix}
            1 & 2 & 3 & 4 & 5 \\
            4 & 1 & 3 & 5 & 2
        \end{pmatrix} = 
        \begin{pmatrix}
            1 & 2 & 3 & 4 & 5 \\
            5 & 4 & 3 & 2 & 1
        \end{pmatrix},
    \]
    y así sucesivamente, llegaremos a que $\sigma^4 = \Id$.
\end{example}

\begin{prop}
Más en general, se tiene que, si $\sigma = \sigma_1 \sigma_2 \cdots \sigma_r$ es una descomposición en permutaciones disjuntas, entonces
\[
    \ord(\sigma) = \mcm \left( \ord (\sigma_1), \ord(\sigma_2), \dots, \ord(\sigma_r) \right).
\]
\end{prop}

\section{Ejercicios}

\begin{ej} %TODO poner ejercicios 3 y 4
    En general toda permutación de $S_n$ descompone en producto de trasposiciones
    $(1, 2), (1, 3), \dots (1, n)$.
\end{ej}

\begin{proof}
    En general tenemos que una trasposición cualquiera
    \[
        (i, j) = (1, i)(1, j)(1,i).
    \]
\end{proof}

\begin{ej}
    \[
        \sigma =
        \begin{pmatrix}
            1 & 2 & 3 & 4 & 5 & 6 & 7 & 8 & 9 \\
            3 & 7 & 8 & 9 & 4 & 5 & 2 & 1 & 6
        \end{pmatrix}
        = (1, 3, 8) (2, 7) (4, 9, 6, 5)
    \]
    Por lo tanto, el orden de $\sigma$ es
    \[
        \ord(\sigma) = \mcm\left( 3, 2, 4 \right) = 12.
    \]
    Ahora, descomponemos en trasposiciones.
    \[
        \begin{aligned}
            (1, 3, 8) &= (1, 8) (1, 3), \\
            (2, 7) &= (2, 7), \\
            (4, 9, 6, 5) &= (4, 5) (4, 6) (4, 9),
        \end{aligned}
    \]
    con lo cual $\sgn(\sigma) = (-1)^6 = 1$.
\end{ej}

\begin{ej} %TODO de verdad nos queremos aburrir con los casos? preguno eh.
    Encontrar todos los valores $x, y, z, t$ tales que 
    \[
        \sigma =
        \begin{pmatrix}
            1 & 2 & 3 & 4 & 5 & 6 & 7 & 8 \\
            3 & 5 & 6 & x & y & 1 & z & t
        \end{pmatrix}
    \]
    tenga orden tres.

    Primero descomponemos $\sigma$.
    \[
        \sigma = (1, 3, 6)
        \begin{pmatrix}
            2 & 4 & 5 & 7 & 8 \\
            5 & x & y & z & t
        \end{pmatrix}.
    \]
    Queremos que el segundo miembro tenga orden 3.

    \begin{itemize}
        \item Si $y = 2$, tenemos el ciclo $(2, 5)$ que tiene orden 2 y por lo tanto $\ord(\sigma)$
            es m\'ultiplo de 2.
        \item Los otros casos quedan como ejercicio.% Si $y = 4$??%TODO
    \end{itemize}
\end{ej}

\chapter{Espai de probabilitat}

\section{Definició axiomàtica de probabilitat}

\begin{defi}[espai!de probabilitat]
    Un espai de probabilitat és un espai de mesura $(\Omega,\Asuc, p)$ tal que $p(\Omega)=1$.
\end{defi}

\begin{defi}[espai!mostral]
    Diem que $\Omega$ és l'espai mostral.
\end{defi}

\begin{defi}[conjunt!d'esdeveniments]
    Diem que $\Asuc$ és el conjunt d'esdeveniments o de successos.
\end{defi}

\begin{defi}[funció!de probabilitat]
    Diem que $p$ és la funció de probabilitat.
\end{defi}

\begin{obs}
    Recordem que $\lp\Omega,\Asuc\rp$ és un espai mesurable si $\Asuc\subseteq \Pa\lp\Omega\rp$ és una $\sigma$-àlgebra d'$\Omega$, és a dir,
    \begin{enumerate}[i)]
        \item $\emptyset \in \Asuc$,
        \item $A \in \Asuc \iff A^C \in \Asuc$,
        \item Si $\lc A_i\rc _{i\in\n}\subseteq \Asuc$, aleshores $\bigcup_{i\in\n}{A_i} \in \Asuc$.
    \end{enumerate}
    I que $\lp\Omega,\Asuc,\mu\rp$ és un espai de mesura si $\mu$ és una mesura sobre l'espai mesurable $\lp\Omega,\Asuc\rp$, és a dir,
    \begin{enumerate}[i)]
        \item $\mu(\emptyset) = 0$,
        \item $\forall A \in \Asuc,\quad \mu(A) \ge0$,
        \item ($\sigma$-additivitat) Si $\lc A_i\rc_{i\in\n}\subseteq\Asuc$ és tal que $\forall i \neq j, \, A_i \cap A_j = \emptyset$,
        aleshores 
        \[
            \mu\lp\bigcup_{i\in\n}{A_i}\rp = \sum_{i\in\n}{\mu(A_i)}.
        \]
    \end{enumerate}
\end{obs}

\begin{prop}
    Sigui $(\Omega,\Asuc, p)$ un espai de probabilitat. Aleshores,
    \begin{enumerate}[i)]
        \item Si $A_1, \dots, A_r \in \Asuc$ són tals que $\forall i\neq j,\, A_i \cup A_j = \emptyset,$ aleshores $p\lp \bigcap\limits_{i=1}^{r} A_i \rp= \sum\limits_{i=1}^{r} p\lp A_i \rp.$.
        \item \label{item:esp_prob_2}$A \in \Asuc \implies p(\overline{A})=1-p(A)$.
        \item \label{item:esp_prob_3}$A,B \in \Asuc, A \subseteq B \implies p\lp B\setminus A \rp = p\lp B\rp - p\lp A\rp$.
        \item $A,B \in \Asuc, A \subseteq B \implies p(A) \le p(B)$.
        \item \label{item:esp_prob_5}Successions monòtones:
        \begin{enumerate}[a)]
         \item Si $\left\{A_i\right\}_{i\in\n} \subseteq \Asuc$ són tals que $A_i\subseteq A_{i+1}$, aleshores $p\lp \bigcup\limits_{i\in\n} A_i\rp = \lim\limits_{i\to\infty} p\lp A_i\rp$.
         \item Si $\left\{A_i\right\}_{i\in\n} \subseteq \Asuc$ són tals que $A_i\supseteq A_{i+1}$, aleshores $p\lp \bigcap\limits_{i\in\n} A_i\rp = \lim\limits_{i\to\infty} p\lp A_i\rp$.
        \end{enumerate}
    \end{enumerate}
\end{prop}
\begin{proof}
    \begin{enumerate}
        \item[]
        \item Conseqüència directa de la $\sigma$-additivitat.
        \item Conseqüència diecta de \ref{item:esp_prob_2} usant que $\Asuc = A\cup A^C$.
        \item Com que $A\subseteq B,\, B=\lp B\setminus A\rp \cup A$ i, per tant, $p\lp B\setminus A \rp = p\lp B\rp - p\lp A\rp$.
        \item Conseqüència directa de \ref{item:esp_prob_3} ja que $p\lp B\setminus A\rp\geq 0$.
        \item 
        \begin{enumerate}[a)]
            \item[]
            \item Sigui $B_0=A_0$ i per $i>0$ sigui $B_i = A_i\setminus A_{i-1}$. Aleshores, es compleix que $\forall i\neq j, \, B_i \cap B_j =\emptyset$ i que $\bigcup\limits_{i\in \n} B_i = \bigcup\limits_{i\in\n} A_i$, de manera que
            \begin{gather*}
                p\lp\bigcup\limits_{i\in\n} A_i\rp = p\lp\bigcup\limits_{i\in\n} B_i\rp = \sum\limits_{i\in\n} p\lp B_i\rp =\\
                = \lim\limits_{N\to\infty} \sum_{i=0}^N p\lp B_i\rp = \lim\limits_{N\to\infty} p\lp \bigcup_{i=0}^N B_i\rp = \lim\limits_{N\to\infty} p\lp A_N\rp.
            \end{gather*}
            \item Anàleg al cas anterior.
        \end{enumerate}
    \end{enumerate}
\end{proof}

\ref{item:esp_prob_5} només es pot aplicar en casos molt particulars. En general, si tenim $A_i,\dots,A_r$ succcessos,
hi ha estimacions per a $p(\bigcup_{i=1}^{r}{A_i}$:

\begin{prop}[Desigualtats de Bonferroni]
    Siguin $A_1,\dots,A_r\in\Asuc$, i per $I\subseteq\{1,\dots,r\}$ sigui $A_I = bigcap_{i \in I}{A_i}$. Definim
    \[
        S_k = \sum_{I \in \{1,\dots,n\},\#I=k}{p(A_I)}
    \],
    això és, $S_1 = \sum{p(A_i)}$, $S_2 = \sum_{i \neq j}{p(A_i \cap A_j}$... Aleshores:
    \begin{enumerate}[i)]
         \item Si $t$ és parell,
            \[p\lp\bigcup_{i=1}^{r}{A_i}\rp \geq \sum_{i=1}^{r}{(-1)^{i+1}S_i}\]
         \item Si $t$ és senar,
            \[p\lp\bigcup_{i=1}^{r}{A_i}\rp \leq \sum_{i=1}^{r}{(-1)^{i+1}S_i}\]
    \end{enumerate}
\end{prop}

\begin{obs}
    Amb els casos $t=1$ (desigualtat de Boole) i $t=2$ es poden donar fites inferiors i superiors.
\end{obs}


\begin{example}[Espais de probabilitat]
    %TODO
\end{example}


\section{Probabilitat condicionada}
\begin{defi}[probabilitat!condicionada]
    Sigui $\lp \Omega, \Asuc, p\rp$ un espai de probabilitat i siguin $A, B \in \Asuc$. Definim la probabilitat d'$A$ condicionada a $B$ com
    \[
        p\lp A\mid B\rp = \frac{p\lp A\cap B\rp}{p\lp B\rp}.
    \]
\end{defi}
\begin{obs}
    Sigui $\lp \Omega, \Asuc, p\rp$ un espai de probabilitat i sigui $B \in \Asuc$ tal que $p\lp B\rp > 0$. Aleshores, l'aplicació
    \begin{align*}
        p_B\colon \Asuc &\to \real \\
        A &\mapsto p_B\lp A\rp := p\lp A\mid B\rp
    \end{align*}
    defineix un espai de probabilitat $\lp \Omega, \Asuc, p_B\rp$.
\end{obs}

\begin{prop}
    Sigui $I$ un conjunt numerable o finit i siguin $\lc A_i \rc_{i\in I} \subseteq \Asuc$ tals que 
    \begin{enumerate}[a)]
        \item $p\lp A_i\rp>0$,
        \item $i\neq j \implies A_i \cap A_j = \varnothing$,
        \item $\bigcup\limits_{i\in I} A_i = \Omega$.
    \end{enumerate}
    Aleshores,
    \begin{enumerate}[1)]
        \item Probabilitat total:
            \[
                p\lp B\rp=\sum_{i\in I} p\lp B\mid A_i\rp p\lp A_i\rp, \quad \forall B\in \Asuc.
            \]
        \item Fórmula de Bayes:
            \[
                p\lp A_i\mid B\rp=\frac{P\lp B\mid A_i\rp p\lp A_i\rp}{\sum_{j\in I} p\lp B\mid A_j\rp p\lp A_j\rp}, \quad \forall B\in \Asuc \text{ amb } p\lp B\rp>0.
            \]
    \end{enumerate}
\end{prop}

\begin{proof}
    \begin{enumerate}[1)]
        \item[]
        \item Com que els $A_i$ són disjunts i $\bigcup_{i \in I}{A_i} = \Omega$, $\forall B \in \Asuc$,
        $B = \bigcup_{i\in I}{B \cap A_i}$, i la unió és disjunta. Es té
        \[
            p(B) = p\lp\bigcup_{i\in I}{B \cap A_i}\rp \stackrel{\sigma-add.}{=} \sum_{i\in I}{p(B \cap A_i)} =
            \sum_{i\in I}{p(B|A_i)p(A_i)}.
        \]
        \item
        \begin{gather*}
            p(A_i|B) \sum_{j \in I}{p(B|A_j)p(A_j)} \stackrel{i)}{=} p(A_i|B)p(B) =\\
            \frac{p(B\cap A_i)}{p(B)}p(B) = p(B \cap A_i) = P\lp B\mid A_i\rp p\lp A_i\rp.
        \end{gather*}
    \end{enumerate}
\end{proof}

\begin{problema}[Ruïna del jugador]
    Partim d'un capital de $k$ unitats i, en cada jugada (sense memòria) augmenta o disminueix el capital en una unitat,
    amb probabilitats 1/2 i 1/2. El joc acaba si ens quedem sense capital o si assolim un objectiu $N$ ($N>k$).
    Quina és la probabilitat de perdre tot el capital?
\end{problema}
\begin{sol}
    
    Sigui $A_k$ el succés ``el jugador, començant amb capital $k$, perd''.
    
    Condicionem $A_k$ a la primera tirada de la moneda, definim $B$: ``la primera tirada ix cara''.
    
    \[p(A_k) = p(A_k|B)p(B) + p(A_k|\overline{B})p(\overline{B}) = p(A_k|B)\frac{1}{2} + p(A_k|\overline{B})\frac{1}{2} \implies\]
    \[\implies 2p(A_k)=p(A_{k-1}) + p(A_{k+1}) \implies p(A_k) - p(A_{k-1}) = p(A_{k+1}) - p(A_k) = C\]
    
    és constant. Per tant $p(A_k) = p(A_0)+kC$. Sabent que $p(A_0)=1$ i $p(A_N)=0$:
    \[0 = 1 + CN \implies C = -\frac{1}{n} \implies p(A_k) = 1 - \frac{k}{N}\]
\end{sol}

\section{Independència}
\begin{defi}
    Sigui $\lp \Omega, \Asuc, p\rp$ un espai de probabilitat, sigui $I$ un conjunt finit o numerable i sigui $\lc A_i\rc_{i\in I} \subseteq \Asuc$. Diem que els esdeveniments $A_i$ són independents si per tot $J\subseteq I$ amb $\abs{J}\in\n$ es té que
    \[
        p\lp\bigcap_{j\in J} A_j\rp = \prod_{j\in J} p\lp A_j\rp.
    \]
\end{defi}

\begin{example}
    \begin{enumerate}[1.]
        \item[]
        \item $\varnothing, \Omega$ són independents entre si.
        \item $A$ és independent amb si mateix si i només si $p\lp A\rp=1$ o $p\lp A\rp =0$.
    \end{enumerate}
\end{example}





\chapter{Espacios topológicos y aplicaciones continuas}

\begin{eje}
    \begin{enumerate}[(a)]
        \item[]
        \item Comprobamos que
            \begin{enumerate}[i)]
                \item El intervalo $\lp a, a \rp = \emptyset \in \T$ y $\lp -\infty, +\infty \rp = X \in \T$.
                \item Sea $U = \bigcup\limits_{i \in I} U_i$ la unión de un numero arbitrario de abiertos, entonces, $\forall x \in U, \exists i \in I \tq x \in U_i \subseteq U$ y por tanto $x$ es un punto interior y $U$ es un abierto.
                \item Sea $U = \bigcap\limits_{i = 1}^n U_i$ la intersección de un numero finito de abiertos, entonces, $\forall x \in U, \forall i \in \left\{ 1, \dots, n \right\}, \exists a_i, b_i \tq x \in \left(a_i, b_i\right) \subseteq U_i$, porque $x \in U_i$ y $U_i$ abierto. Sean
                    \begin{gather*}
                        a = \max_{i \in \left\{ 1, \dots, n \right\}} \left\{a_i\right\}, \\
                        b = \min_{i \in \left\{ 1, \dots, n \right\}} \left\{b_i\right\},
                    \end{gather*}
                entonces $x \in \lp a, b \rp \subseteq \bigcap\limits_{i = 1}^n U_i$ y por tanto $x$ es un punto interior y $U$ es un abierto.
            \end{enumerate}
        \item Sea $\B = \left\{ \lp a, b \rp \colon a, b \in X \cup \left\{ \pm \infty \right\} \right\}$, comprobamos que
            \begin{enumerate}[i)]
                \item $\forall x \in X, x \in \lp -\infty, +\infty \rp \in \B$.
                \item $\forall a, b, c, d \in X \cup \left\{ \pm \infty \right\}$, si $\lp a, b \rp \cap \lp c, d \rp \neq \emptyset$, entonces,
                    \begin{gather*}
                        \alpha = \max \left\{ a, c \right\}, \\
                        \beta = \min \left\{ b, d \right\}, \\
                        \lp a, b \rp \cap \lp c, d \rp = \lp \alpha, \beta \rp.
                    \end{gather*}
                    y por tanto $\forall x \in \lp a, b \rp \cap \lp c, d \rp, x \in \lp \alpha, \beta \rp \subseteq \lp a, b \rp \cap \lp c, d \rp$, y $\lp \alpha, \beta \rp \in \B$.
            \end{enumerate}
        \item Si vemos que $\forall x \in X, \left\{ x \right\}$ es un abierto, ya abremos acabado, ya que todo conjunto de $\Pa \lp X \rp$ contiene únicamente puntos de $X$ y por lo tanto sera la unión de abiertos. En $\z$, $\forall x \in \z$, tenemos que $\lc x \rc = \lp n-1, n+1 \rp$ y por tanto ya estamos. En $\n$, $\forall x \in \n \setminus \lc 1 \rc$, tenemos que $\lc x \rc = \lp n-1, n+1 \rp$ y $\lc 1 \rc = \lp -\infty, 2 \rp$ y por tanto ya estamos, suponiendo que $0 \notin \n$.
        \item Sea $X$ un espacio topológico con la topología del orden, entonces $\forall x, y \in X, x < y$,
            \begin{itemize}
                \item Si $\exists z$ tal que $x < z < y$, entonces $x \in \lp -\infty, z \rp, y \in \lp z, +\infty \rp, \lp -\infty, z \rp \cap \lp z, +\infty \rp = \emptyset$.
                \item Si $\nexists z$ tal que $x < z < y$, entonces $x \in \lp -\infty, y \rp, y \in \lp x, +\infty \rp, \lp -\infty, y \rp \cap \lp x, +\infty \rp = \emptyset$.
            \end{itemize}
        \item Aquest dibuix està en contrucció. % TODO dibuix!!
        \item Tenemos que ver que $\T_{\text{ord}} \subset \T_\leq$, es decir, $\T_{\text{ord}} \subseteq \T_\leq$ y $\T_\leq \neq \T_{\text{ord}}$.
            \begin{itemize}
                \item Veamos que $\T_{\text{ord}} \subseteq \T_\leq$. Sea $U \subseteq \T_{\text{ord}},\, \forall x \equiv \lp x_1, x_2 \rp \in U, \, \exists r \in \real^+ \tq B_r \lp x \rp \subseteq U$, y por tanto, $A = \lp \lp x_1 - r, x_2 \rp, \lp x_1 + r, x_2 \rp \rp \subseteq B_r \lp x \rp, A \in \T_\leq$. Así pues, todos los puntos de $U$ son interiores en la topología del orden y por tanto $U \in \T_\leq$.
                \item Veamos que $\T_\leq \neq \T_{\text{ord}}$. Sean $x = \lp 0,0 \rp, y = \lp 0, 1 \rp$, entonces $\lp x, y \rp \in \T_\leq, \lp x, y \rp \notin \T_{\text{ord}}$. Por tanto $\T_\leq \neq \T_{\text{ord}}$.
            \end{itemize}
        \item Son las topologías discretas.
    \end{enumerate}
\end{eje}
\begin{eje}
    \begin{enumerate}[(a)]
        \item[]
        \item Comprobamos que
            \begin{enumerate}[i)]
                \item $\emptyset = \left[ x, x \rp \in \T_\ell$ y $X = \bigcup\limits_{x \in X} \left[ x, \infty \rp \in \T_\ell$.
                \item Sea $U = \bigcup\limits_{i \in I} U_i$ la unión de un numero arbitrario de abiertos, entonces, $\forall x \in U, \exists i \in I \tq x \in U_i \subseteq U$ y por tanto $x$ es un punto interior y $U$ es un abierto.
                \item Sea $U = \bigcap\limits_{i = 1}^n U_i$ la intersección de un numero finito de abiertos, entonces, $\forall x \in U, \forall i \in \left\{ 1, \dots, n \right\}, \exists a_i, b_i \tq x \in \left[a_i, b_i\right) \subseteq U_i$, porque $x \in U_i$ y $U_i$ abierto. Sean
                    \begin{gather*}
                        a = \max_{i \in \left\{ 1, \dots, n \right\}} \left\{a_i\right\}, \\
                        b = \min_{i \in \left\{ 1, \dots, n \right\}} \left\{b_i\right\},
                    \end{gather*}
                entonces $x \in \left[ a, b \rp \subseteq \bigcap\limits_{i = 1}^n U_i$ y por tanto $x$ es un punto interior y $U$ es un abierto.
            \end{enumerate}
        \item Sea $\B = \left\{ \left[ a, b \rp \colon a \in X, b \in X \cup \left\{ \infty \right\} \right\}$, comprobamos que
            \begin{enumerate}[i)]
                \item $\forall x \in X, x \in \left[ x, \infty \rp \in \B$.
                \item $\forall a, b, c, d \in X \cup \left\{ \infty \right\}$, si $\left[ a, b \rp \cap \left[ c, d \rp \neq \emptyset$, entonces,
                    \begin{gather*}
                        \alpha = \max \left\{ a, c \right\}, \\
                        \beta = \min \left\{ b, d \right\}, \\
                        \left[ a, b \rp \cap \left[ c, d \rp = \left[ \alpha, \beta \rp.
                    \end{gather*}
                    y por tanto $\forall x \in \left[ a, b \rp \cap \left[ c, d \rp, x \in \left[ \alpha, \beta \rp \subseteq \left[ a, b \rp \cap \left[ c, d \rp$, y $\left[ \alpha, \beta \rp \in \B$.
            \end{enumerate}
        \item \item[] % OJO amb aqueta cutrada
            \begin{center}
                \begin{tabular}{|l||c|c|c|c|c|c|} \hline
                    & $\lp a, b \rp$ & $\left[ a, b \rp$ & $\lp a, b \right]$ & $\left[ a,b \right]$ & $\lc 0 \rc \cup \lc \sfrac{1}{n} \rc_{n\geq 1}$ & $\lc 0 \rc \cup \lc \sfrac{-1}{n} \rc_{n\geq 1}$ \\ \hline \hline
                    Adherencia & $\left[a,b\rp$ & $\left[ a, b \rp$ & $\left[ a, b \right]$ & $\left[ a, b \right]$ & $\lc 0 \rc \cup \lc \sfrac{1}{n} \rc_{n\geq 1}$ & $\lc 0 \rc \cup \lc \sfrac{-1}{n} \rc_{n\geq 1}$ \\ \hline
                    Interior & $\lp a, b \rp$ & $\left[ a, b \rp$ & $\lp a, b \rp$ & $\left[ a, b \rp$ & $\emptyset$ & $\emptyset$ \\ \hline
                    Frontera & $\lc a \rc$ & $\emptyset$ & $\lc a, b \rc$ & $\lc b \rc$ & $\lc 0 \rc \cup \lc \sfrac{1}{n} \rc_{n\geq 1}$ & $\lc 0 \rc \cup \lc \sfrac{-1}{n} \rc_{n\geq 1}$ \\ \hline
                    Acumulación & $\left[a,b\rp$ & $\left[ a, b \rp$ & $\left[ a, b \rp$ & $\left[ a, b \right)$ & $\lc 0 \rc$ & $\emptyset$ \\ \hline
                    Puntos aislados & $\emptyset$ & $\emptyset$ & $\lc b \rc$ & $\lc b \rc$ & $\lc \sfrac{1}{n} \rc_{n\geq 1}$ & $\lc 0 \rc \cup \lc \sfrac{-1}{n} \rc_{n\geq 1}$ \\ \hline
                \end{tabular}
            \end{center}
    \end{enumerate}
\end{eje}
\begin{eje}
    \begin{enumerate}[(a)]
        \item[]
        \item Comprobamos que
            \begin{enumerate}[i)]
                \item $\emptyset = \left[ x, x \rp \in \T_\ell$
                \item Sea $U = \bigcup\limits_{i \in I} U_i$ la unión de un numero arbitrario de abiertos, entonces, $\forall x \in U, \exists i \in I \tq x \in U_i \subseteq U$ y por tanto $x$ es un punto interior y $U$ es un abierto.
                \item Sea $U = \bigcap\limits_{i = 1}^n U_i$ la intersección de un numero finito de abiertos, entonces, $\forall x \in U, \forall i \in \left\{ 1, \dots, n \right\}, \exists a_i, b_i \tq x \in \left(a_i, b_i\right) \subseteq U_i$. Sean
                    \begin{gather*}
                        a = \max_{i \in \left\{ 1, \dots, n \right\}} \left\{a_i\right\}, \\
                        b = \min_{i \in \left\{ 1, \dots, n \right\}} \left\{b_i\right\},
                    \end{gather*}
                entonces $x \in \lp a, b \rp \subseteq \bigcap\limits_{i = 1}^n U_i$ y por tanto $x$ es un punto interior y $U$ es un abierto.
            \end{enumerate}
    \end{enumerate}
\end{eje}

\begin{eje}
 Este ejercicio aún no está resuelto.
\end{eje}

\begin{eje}
 Este ejercicio aún no está resuelto
\end{eje}

\begin{eje}
 Este ejercicio aún no está resuelto
\end{eje}

\begin{eje}
 Sea $\T$ una topología y $X$ un espacio topológico. Por definición de $\psi$ se tiene que $\forall A\subseteq X,\, \psi\lp A\rp =\overline{A}$ es un cerratdo de $\T$. Por lo tanto, podemos definir un conjunto de abiertos de $\T$ como
 \[
  \B = \lc B\subseteq X\, | \,\exists A \subseteq X \tq B=X\setminus \psi\lp A\rp \rc
 \]
 que son abiertos por ser el complementario de un cerrado ($\overline{A}$ es el cerrado más pequeño que contiene a $A$).
 
 Ahora demostraremos que $\B$ es una base de la topología $\T$. 

 Primero veamos que
 \begin{gather*}
  X\setminus \psi\lp \varnothing\rp = X\setminus\varnothing = X \in \B\\
  X\setminus \psi\lp X\rp =X\setminus X = \varnothing \in \B.
 \end{gather*}
 
Entonces tenemos lo siguiente:
\begin{enumerate}[i)]
 \item $\forall x\in X,\, \exists A\subseteq \tq \overline{A}\cup\lc x\rc = \varnothing \implies x\in B = X\setminus \psi\lp A\rp$.
 En concreto podemos coger $A=\varnothing$.
 \item Sean $B_1,B_2 \in \B$ y sea $x\in X \tq x\in B_1\cap B_2$. Entonces
 \begin{gather*}
  x\in B_1\cap B_2 = \lp X\setminus \psi\lp A_1\rp \rp \cap \lp X\setminus\psi\lp A_2\rp \rp = X\setminus \lp\psi\lp A_1\rp\cup\psi\lp A_2\rp\rp = \\
  = X\setminus\psi\lp A_1 \cup A_2\rp = X\setminus\psi\lp\psi\lp A_1 \cup A_2\rp \rp,
 \end{gather*}
 y como $A=\psi\lp A_1\cup A_2\rp \subseteq X$, $B=X\setminus\psi\lp A\rp \in \B$, tenemos que $\exists B \in \B \tq x\in B \subseteq B_1 \cap B_2$.
\end{enumerate}
Por lo tanto, $\B$ es la base de una topología (de $\T$), lo que nos dice que como $\B$ son los mínimos abiertos que debe contener una topología que cumpla que $\psi\lp A\rp = \overline{A}$ y una base define una única topología, existe una única topología $\T$ que lo cumple. 
\end{eje}



\end{document}
