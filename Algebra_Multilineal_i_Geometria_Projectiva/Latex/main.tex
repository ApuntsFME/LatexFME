\documentclass[12pt]{article}
 
\usepackage[margin=1in]{geometry}
\usepackage[pdftex]{hyperref}
\usepackage{amsmath,amsthm,amssymb,graphicx,mathtools,tikz,hyperref,enumerate}
\usepackage{mdframed,cleveref,cancel,stackengine,pgfplots,pgf,mathrsfs}
\usepackage{xfrac,stmaryrd,commath}
\usepackage[spanish]{babel}

\newmdenv[leftline=false,topline=false]{topright}
\let\proof\relax
\usepackage[utf8]{inputenc}
\usetikzlibrary{positioning,arrows, calc}
\usetikzlibrary{external}
\tikzexternalize[prefix=figures/]
\pgfplotsset{compat=1.11}
\newcommand{\n}{\mathbb{N}}
\newcommand{\z}{\mathbb{Z}}
\newcommand{\q}{\mathbb{Q}}
\newcommand{\cx}{\mathbb{C}}
\newcommand{\real}{\mathbb{R}}
\newcommand{\E}{\mathbb{E}}
\newcommand{\F}{\mathbb{F}}
\newcommand{\R}{\mathcal{R}}
\newcommand{\bb}[1]{\mathbb{#1}}
\let\k\relax
\newcommand{\k}{\mathbf{k}}
\newcommand{\ita}[1]{\textit{#1}}
\newcommand\inv[1]{#1^{-1}}
\newcommand\setb[1]{\left\{#1\right\}}
\newcommand{\vbrack}[1]{\langle #1\rangle}
\newcommand{\determinant}[1]{\begin{vmatrix}#1\end{vmatrix}}
\newcommand{\Po}{\mathbb{P}}
\DeclareMathOperator{\Id}{Id}
\DeclareMathOperator{\rg}{rg}
\DeclareMathOperator{\car}{car}
\DeclareMathOperator{\im}{Im}
\let\emptyset\varnothing


\hypersetup{
	colorlinks,
	linkcolor=blue
}
 
 \renewcommand*\contentsname{Contenidos}

\newtheoremstyle{break}% name
{}%         Space above, empty = `usual value'
{}%         Space below
{}% Body font
{}%         Indent amount (empty = no indent, \parindent = para indent)
{\bfseries}% Thm head font
{}%        Punctuation after thm head
{\newline}% Space after thm head: \newline = linebreak
{#1 #2 \normalfont #3}%         Thm head spec

\newtheoremstyle{breakthm}% name
{}%         Space above, empty = `usual value'
{}%         Space below
{}% Body font
{}%         Indent amount (empty = no indent, \parindent = para indent)
{\bfseries}% Thm head font
{}%        Punctuation after thm head
{\newline}% Space after thm head: \newline = linebreak
{#1 \normalfont #3 (#2)\addcontentsline{toc}{subsubsection}{#1 #3}}%         Thm head spec
\newtheoremstyle{normal}% name
{}%         Space above, empty = `usual value'
{}%         Space below
{}% Body font
{}%         Indent amount (empty = no indent, \parindent = para indent)
{\bfseries}% Thm head font
{}%        Punctuation after thm head
{5pt plus 1pt minus 1pt}% Space after thm head: \newline = linebreak
{#1 #2 \normalfont #3}%         Thm head spec

\theoremstyle{normal}
\newtheorem{lema}{Lema}[subsection]
\newtheorem{obs}[lema]{Observación}
\newtheorem{rec}[lema]{Recordatorio}

\theoremstyle{break}
\newtheorem{prop}[lema]{Proposición}
\newtheorem*{proof}{Demostración}
\newtheorem{defi}[lema]{Definición}
\newtheorem{col}[lema]{Corolario}
\newtheorem{ej}[lema]{Ejercicio}
\newtheorem{example}[lema]{Ejemplo}

\theoremstyle{breakthm}
\newtheorem{thm}[lema]{Teorema}



 
\begin{document}
\date{}

\title{Álgebra Multilineal y Geometría Proyectiva}
 
\maketitle

\tableofcontents

\setcounter{section}{-1}

\chapter{Permutaciones}

\section{Repaso de permutaciones}

El grupo simétrico $\lp S_n, \circ \rp$ es el grupo de las permutaciones de los elementos
$\setb{1, 2, \dots, n}$ y el cardinal de $S_n$ es $\# S_n = \abs{S_n} = n!$

Si $\sigma \in S_n$, podemos escribir $\sigma$ como
\[
    \begin{pmatrix}
        1 & 2 & 3 & \cdots & n \\
        \sigma(1) & \sigma(2) & \sigma(3) & \cdots & \sigma(n)
    \end{pmatrix}.
\]

Cualquier permutación se descompone en ciclos, por ejemplo
$\sigma = (1, 4, 5, 2) \in S_5$ es lo mismo que
\[
    \sigma = 
    \begin{pmatrix}
        1 & 2 & 3 & 4 & 5 \\
        4 & 1 & 3 & 5 & 2
    \end{pmatrix}.
\]

Entonces $\sigma = \sigma_1 \cdots \sigma_r$, siendo $\sigma_i$ ciclos disjuntos.

\begin{obs}
    La multiplicación no es conmutativa. Pero las permutaciones con elementos disjuntos sí que conmutan.
\end{obs}

Todo ciclo se puede descomponer como producto de transposiciones $z = (i, j)$.
Por lo tanto, podemos descomponer toda permutación como producto de transposiciones, pero esta
descomposición no es única.
Lo que sí que se mantiene es la paridad del número de trasposiciones. Es decir,
\[
    \begin{rcases}
        \sigma = z_1 \cdots z_r \\
        \sigma = \bar{z}_1 \cdots \bar{z}_s
    \end{rcases} \implies \lp r \text{ par} \iff s \text{ par}\rp.
\]

Esto nos permite definir unequívocamente el signo de la permutación:
\[
    \sgn(\sigma) = (-1)^{r},
\]
donde $r$ es el número de trasposiciones de cualquiera de sus descomposiciones en transposiciones.

\begin{defi}[orden!de una permutación]
    Definimos el orden de una permutación $\sigma$ como el mínimo $k$ tal que $\sigma^k = \Id$.
\end{defi}

\begin{example}
    $\sigma = (1, 4, 5, 2)$. Calcular el orden de $\sigma$.
    \[
        \sigma^2 = \sigma \cdot \sigma =
        \begin{pmatrix}
            1 & 2 & 3 & 4 & 5 \\
            4 & 1 & 3 & 5 &2
        \end{pmatrix}
        \begin{pmatrix}
            1 & 2 & 3 & 4 & 5 \\
            4 & 1 & 3 & 5 & 2
        \end{pmatrix} = 
        \begin{pmatrix}
            1 & 2 & 3 & 4 & 5 \\
            5 & 4 & 3 & 2 & 1
        \end{pmatrix},
    \]
    y así sucesivamente, llegaremos a que $\sigma^4 = \Id$.
\end{example}

\begin{prop}
Más en general, se tiene que, si $\sigma = \sigma_1 \sigma_2 \cdots \sigma_r$ es una descomposición en permutaciones disjuntas, entonces
\[
    \ord(\sigma) = \mcm \left( \ord (\sigma_1), \ord(\sigma_2), \dots, \ord(\sigma_r) \right).
\]
\end{prop}

\section{Ejercicios}

\begin{ej} %TODO poner ejercicios 3 y 4
    En general toda permutación de $S_n$ descompone en producto de trasposiciones
    $(1, 2), (1, 3), \dots (1, n)$.
\end{ej}

\begin{proof}
    En general tenemos que una trasposición cualquiera
    \[
        (i, j) = (1, i)(1, j)(1,i).
    \]
\end{proof}

\begin{ej}
    \[
        \sigma =
        \begin{pmatrix}
            1 & 2 & 3 & 4 & 5 & 6 & 7 & 8 & 9 \\
            3 & 7 & 8 & 9 & 4 & 5 & 2 & 1 & 6
        \end{pmatrix}
        = (1, 3, 8) (2, 7) (4, 9, 6, 5)
    \]
    Por lo tanto, el orden de $\sigma$ es
    \[
        \ord(\sigma) = \mcm\left( 3, 2, 4 \right) = 12.
    \]
    Ahora, descomponemos en trasposiciones.
    \[
        \begin{aligned}
            (1, 3, 8) &= (1, 8) (1, 3), \\
            (2, 7) &= (2, 7), \\
            (4, 9, 6, 5) &= (4, 5) (4, 6) (4, 9),
        \end{aligned}
    \]
    con lo cual $\sgn(\sigma) = (-1)^6 = 1$.
\end{ej}

\begin{ej} %TODO de verdad nos queremos aburrir con los casos? preguno eh.
    Encontrar todos los valores $x, y, z, t$ tales que 
    \[
        \sigma =
        \begin{pmatrix}
            1 & 2 & 3 & 4 & 5 & 6 & 7 & 8 \\
            3 & 5 & 6 & x & y & 1 & z & t
        \end{pmatrix}
    \]
    tenga orden tres.

    Primero descomponemos $\sigma$.
    \[
        \sigma = (1, 3, 6)
        \begin{pmatrix}
            2 & 4 & 5 & 7 & 8 \\
            5 & x & y & z & t
        \end{pmatrix}.
    \]
    Queremos que el segundo miembro tenga orden 3.

    \begin{itemize}
        \item Si $y = 2$, tenemos el ciclo $(2, 5)$ que tiene orden 2 y por lo tanto $\ord(\sigma)$
            es m\'ultiplo de 2.
        \item Los otros casos quedan como ejercicio.% Si $y = 4$??%TODO
    \end{itemize}
\end{ej}

\chapter{Espai de probabilitat}

\section{Definició axiomàtica de probabilitat}

\begin{defi}[espai!de probabilitat]
    Un espai de probabilitat és un espai de mesura $(\Omega,\Asuc, p)$ tal que $p(\Omega)=1$.
\end{defi}

\begin{defi}[espai!mostral]
    Diem que $\Omega$ és l'espai mostral.
\end{defi}

\begin{defi}[conjunt!d'esdeveniments]
    Diem que $\Asuc$ és el conjunt d'esdeveniments o de successos.
\end{defi}

\begin{defi}[funció!de probabilitat]
    Diem que $p$ és la funció de probabilitat.
\end{defi}

\begin{obs}
    Recordem que $\lp\Omega,\Asuc\rp$ és un espai mesurable si $\Asuc\subseteq \Pa\lp\Omega\rp$ és una $\sigma$-àlgebra d'$\Omega$, és a dir,
    \begin{enumerate}[i)]
        \item $\emptyset \in \Asuc$,
        \item $A \in \Asuc \iff A^C \in \Asuc$,
        \item Si $\lc A_i\rc _{i\in\n}\subseteq \Asuc$, aleshores $\bigcup_{i\in\n}{A_i} \in \Asuc$.
    \end{enumerate}
    I que $\lp\Omega,\Asuc,\mu\rp$ és un espai de mesura si $\mu$ és una mesura sobre l'espai mesurable $\lp\Omega,\Asuc\rp$, és a dir,
    \begin{enumerate}[i)]
        \item $\mu(\emptyset) = 0$,
        \item $\forall A \in \Asuc,\quad \mu(A) \ge0$,
        \item ($\sigma$-additivitat) Si $\lc A_i\rc_{i\in\n}\subseteq\Asuc$ és tal que $\forall i \neq j, \, A_i \cap A_j = \emptyset$,
        aleshores 
        \[
            \mu\lp\bigcup_{i\in\n}{A_i}\rp = \sum_{i\in\n}{\mu(A_i)}.
        \]
    \end{enumerate}
\end{obs}

\begin{prop}
    Sigui $(\Omega,\Asuc, p)$ un espai de probabilitat. Aleshores,
    \begin{enumerate}[i)]
        \item Si $A_1, \dots, A_r \in \Asuc$ són tals que $\forall i\neq j,\, A_i \cup A_j = \emptyset,$ aleshores $p\lp \bigcap\limits_{i=1}^{r} A_i \rp= \sum\limits_{i=1}^{r} p\lp A_i \rp.$.
        \item \label{item:esp_prob_2}$A \in \Asuc \implies p(\overline{A})=1-p(A)$.
        \item \label{item:esp_prob_3}$A,B \in \Asuc, A \subseteq B \implies p\lp B\setminus A \rp = p\lp B\rp - p\lp A\rp$.
        \item $A,B \in \Asuc, A \subseteq B \implies p(A) \le p(B)$.
        \item \label{item:esp_prob_5}Successions monòtones:
        \begin{enumerate}[a)]
         \item Si $\left\{A_i\right\}_{i\in\n} \subseteq \Asuc$ són tals que $A_i\subseteq A_{i+1}$, aleshores $p\lp \bigcup\limits_{i\in\n} A_i\rp = \lim\limits_{i\to\infty} p\lp A_i\rp$.
         \item Si $\left\{A_i\right\}_{i\in\n} \subseteq \Asuc$ són tals que $A_i\supseteq A_{i+1}$, aleshores $p\lp \bigcap\limits_{i\in\n} A_i\rp = \lim\limits_{i\to\infty} p\lp A_i\rp$.
        \end{enumerate}
    \end{enumerate}
\end{prop}
\begin{proof}
    \begin{enumerate}
        \item[]
        \item Conseqüència directa de la $\sigma$-additivitat.
        \item Conseqüència diecta de \ref{item:esp_prob_2} usant que $\Asuc = A\cup A^C$.
        \item Com que $A\subseteq B,\, B=\lp B\setminus A\rp \cup A$ i, per tant, $p\lp B\setminus A \rp = p\lp B\rp - p\lp A\rp$.
        \item Conseqüència directa de \ref{item:esp_prob_3} ja que $p\lp B\setminus A\rp\geq 0$.
        \item 
        \begin{enumerate}[a)]
            \item[]
            \item Sigui $B_0=A_0$ i per $i>0$ sigui $B_i = A_i\setminus A_{i-1}$. Aleshores, es compleix que $\forall i\neq j, \, B_i \cap B_j =\emptyset$ i que $\bigcup\limits_{i\in \n} B_i = \bigcup\limits_{i\in\n} A_i$, de manera que
            \begin{gather*}
                p\lp\bigcup\limits_{i\in\n} A_i\rp = p\lp\bigcup\limits_{i\in\n} B_i\rp = \sum\limits_{i\in\n} p\lp B_i\rp =\\
                = \lim\limits_{N\to\infty} \sum_{i=0}^N p\lp B_i\rp = \lim\limits_{N\to\infty} p\lp \bigcup_{i=0}^N B_i\rp = \lim\limits_{N\to\infty} p\lp A_N\rp.
            \end{gather*}
            \item Anàleg al cas anterior.
        \end{enumerate}
    \end{enumerate}
\end{proof}

\ref{item:esp_prob_5} només es pot aplicar en casos molt particulars. En general, si tenim $A_i,\dots,A_r$ succcessos,
hi ha estimacions per a $p(\bigcup_{i=1}^{r}{A_i}$:

\begin{prop}[Desigualtats de Bonferroni]
    Siguin $A_1,\dots,A_r\in\Asuc$, i per $I\subseteq\{1,\dots,r\}$ sigui $A_I = bigcap_{i \in I}{A_i}$. Definim
    \[
        S_k = \sum_{I \in \{1,\dots,n\},\#I=k}{p(A_I)}
    \],
    això és, $S_1 = \sum{p(A_i)}$, $S_2 = \sum_{i \neq j}{p(A_i \cap A_j}$... Aleshores:
    \begin{enumerate}[i)]
         \item Si $t$ és parell,
            \[p\lp\bigcup_{i=1}^{r}{A_i}\rp \geq \sum_{i=1}^{r}{(-1)^{i+1}S_i}\]
         \item Si $t$ és senar,
            \[p\lp\bigcup_{i=1}^{r}{A_i}\rp \leq \sum_{i=1}^{r}{(-1)^{i+1}S_i}\]
    \end{enumerate}
\end{prop}

\begin{obs}
    Amb els casos $t=1$ (desigualtat de Boole) i $t=2$ es poden donar fites inferiors i superiors.
\end{obs}


\begin{example}[Espais de probabilitat]
    %TODO
\end{example}


\section{Probabilitat condicionada}
\begin{defi}[probabilitat!condicionada]
    Sigui $\lp \Omega, \Asuc, p\rp$ un espai de probabilitat i siguin $A, B \in \Asuc$. Definim la probabilitat d'$A$ condicionada a $B$ com
    \[
        p\lp A\mid B\rp = \frac{p\lp A\cap B\rp}{p\lp B\rp}.
    \]
\end{defi}
\begin{obs}
    Sigui $\lp \Omega, \Asuc, p\rp$ un espai de probabilitat i sigui $B \in \Asuc$ tal que $p\lp B\rp > 0$. Aleshores, l'aplicació
    \begin{align*}
        p_B\colon \Asuc &\to \real \\
        A &\mapsto p_B\lp A\rp := p\lp A\mid B\rp
    \end{align*}
    defineix un espai de probabilitat $\lp \Omega, \Asuc, p_B\rp$.
\end{obs}

\begin{prop}
    Sigui $I$ un conjunt numerable o finit i siguin $\lc A_i \rc_{i\in I} \subseteq \Asuc$ tals que 
    \begin{enumerate}[a)]
        \item $p\lp A_i\rp>0$,
        \item $i\neq j \implies A_i \cap A_j = \varnothing$,
        \item $\bigcup\limits_{i\in I} A_i = \Omega$.
    \end{enumerate}
    Aleshores,
    \begin{enumerate}[1)]
        \item Probabilitat total:
            \[
                p\lp B\rp=\sum_{i\in I} p\lp B\mid A_i\rp p\lp A_i\rp, \quad \forall B\in \Asuc.
            \]
        \item Fórmula de Bayes:
            \[
                p\lp A_i\mid B\rp=\frac{P\lp B\mid A_i\rp p\lp A_i\rp}{\sum_{j\in I} p\lp B\mid A_j\rp p\lp A_j\rp}, \quad \forall B\in \Asuc \text{ amb } p\lp B\rp>0.
            \]
    \end{enumerate}
\end{prop}

\begin{proof}
    \begin{enumerate}[1)]
        \item[]
        \item Com que els $A_i$ són disjunts i $\bigcup_{i \in I}{A_i} = \Omega$, $\forall B \in \Asuc$,
        $B = \bigcup_{i\in I}{B \cap A_i}$, i la unió és disjunta. Es té
        \[
            p(B) = p\lp\bigcup_{i\in I}{B \cap A_i}\rp \stackrel{\sigma-add.}{=} \sum_{i\in I}{p(B \cap A_i)} =
            \sum_{i\in I}{p(B|A_i)p(A_i)}.
        \]
        \item
        \begin{gather*}
            p(A_i|B) \sum_{j \in I}{p(B|A_j)p(A_j)} \stackrel{i)}{=} p(A_i|B)p(B) =\\
            \frac{p(B\cap A_i)}{p(B)}p(B) = p(B \cap A_i) = P\lp B\mid A_i\rp p\lp A_i\rp.
        \end{gather*}
    \end{enumerate}
\end{proof}

\begin{problema}[Ruïna del jugador]
    Partim d'un capital de $k$ unitats i, en cada jugada (sense memòria) augmenta o disminueix el capital en una unitat,
    amb probabilitats 1/2 i 1/2. El joc acaba si ens quedem sense capital o si assolim un objectiu $N$ ($N>k$).
    Quina és la probabilitat de perdre tot el capital?
\end{problema}
\begin{sol}
    
    Sigui $A_k$ el succés ``el jugador, començant amb capital $k$, perd''.
    
    Condicionem $A_k$ a la primera tirada de la moneda, definim $B$: ``la primera tirada ix cara''.
    
    \[p(A_k) = p(A_k|B)p(B) + p(A_k|\overline{B})p(\overline{B}) = p(A_k|B)\frac{1}{2} + p(A_k|\overline{B})\frac{1}{2} \implies\]
    \[\implies 2p(A_k)=p(A_{k-1}) + p(A_{k+1}) \implies p(A_k) - p(A_{k-1}) = p(A_{k+1}) - p(A_k) = C\]
    
    és constant. Per tant $p(A_k) = p(A_0)+kC$. Sabent que $p(A_0)=1$ i $p(A_N)=0$:
    \[0 = 1 + CN \implies C = -\frac{1}{n} \implies p(A_k) = 1 - \frac{k}{N}\]
\end{sol}

\section{Independència}
\begin{defi}
    Sigui $\lp \Omega, \Asuc, p\rp$ un espai de probabilitat, sigui $I$ un conjunt finit o numerable i sigui $\lc A_i\rc_{i\in I} \subseteq \Asuc$. Diem que els esdeveniments $A_i$ són independents si per tot $J\subseteq I$ amb $\abs{J}\in\n$ es té que
    \[
        p\lp\bigcap_{j\in J} A_j\rp = \prod_{j\in J} p\lp A_j\rp.
    \]
\end{defi}

\begin{example}
    \begin{enumerate}[1.]
        \item[]
        \item $\varnothing, \Omega$ són independents entre si.
        \item $A$ és independent amb si mateix si i només si $p\lp A\rp=1$ o $p\lp A\rp =0$.
    \end{enumerate}
\end{example}





\chapter{Espacios topológicos y aplicaciones continuas}

\begin{eje}
    \begin{enumerate}[(a)]
        \item[]
        \item Comprobamos que
            \begin{enumerate}[i)]
                \item El intervalo $\lp a, a \rp = \emptyset \in \T$ y $\lp -\infty, +\infty \rp = X \in \T$.
                \item Sea $U = \bigcup\limits_{i \in I} U_i$ la unión de un numero arbitrario de abiertos, entonces, $\forall x \in U, \exists i \in I \tq x \in U_i \subseteq U$ y por tanto $x$ es un punto interior y $U$ es un abierto.
                \item Sea $U = \bigcap\limits_{i = 1}^n U_i$ la intersección de un numero finito de abiertos, entonces, $\forall x \in U, \forall i \in \left\{ 1, \dots, n \right\}, \exists a_i, b_i \tq x \in \left(a_i, b_i\right) \subseteq U_i$, porque $x \in U_i$ y $U_i$ abierto. Sean
                    \begin{gather*}
                        a = \max_{i \in \left\{ 1, \dots, n \right\}} \left\{a_i\right\}, \\
                        b = \min_{i \in \left\{ 1, \dots, n \right\}} \left\{b_i\right\},
                    \end{gather*}
                entonces $x \in \lp a, b \rp \subseteq \bigcap\limits_{i = 1}^n U_i$ y por tanto $x$ es un punto interior y $U$ es un abierto.
            \end{enumerate}
        \item Sea $\B = \left\{ \lp a, b \rp \colon a, b \in X \cup \left\{ \pm \infty \right\} \right\}$, comprobamos que
            \begin{enumerate}[i)]
                \item $\forall x \in X, x \in \lp -\infty, +\infty \rp \in \B$.
                \item $\forall a, b, c, d \in X \cup \left\{ \pm \infty \right\}$, si $\lp a, b \rp \cap \lp c, d \rp \neq \emptyset$, entonces,
                    \begin{gather*}
                        \alpha = \max \left\{ a, c \right\}, \\
                        \beta = \min \left\{ b, d \right\}, \\
                        \lp a, b \rp \cap \lp c, d \rp = \lp \alpha, \beta \rp.
                    \end{gather*}
                    y por tanto $\forall x \in \lp a, b \rp \cap \lp c, d \rp, x \in \lp \alpha, \beta \rp \subseteq \lp a, b \rp \cap \lp c, d \rp$, y $\lp \alpha, \beta \rp \in \B$.
            \end{enumerate}
        \item Si vemos que $\forall x \in X, \left\{ x \right\}$ es un abierto, ya abremos acabado, ya que todo conjunto de $\Pa \lp X \rp$ contiene únicamente puntos de $X$ y por lo tanto sera la unión de abiertos. En $\z$, $\forall x \in \z$, tenemos que $\lc x \rc = \lp n-1, n+1 \rp$ y por tanto ya estamos. En $\n$, $\forall x \in \n \setminus \lc 1 \rc$, tenemos que $\lc x \rc = \lp n-1, n+1 \rp$ y $\lc 1 \rc = \lp -\infty, 2 \rp$ y por tanto ya estamos, suponiendo que $0 \notin \n$.
        \item Sea $X$ un espacio topológico con la topología del orden, entonces $\forall x, y \in X, x < y$,
            \begin{itemize}
                \item Si $\exists z$ tal que $x < z < y$, entonces $x \in \lp -\infty, z \rp, y \in \lp z, +\infty \rp, \lp -\infty, z \rp \cap \lp z, +\infty \rp = \emptyset$.
                \item Si $\nexists z$ tal que $x < z < y$, entonces $x \in \lp -\infty, y \rp, y \in \lp x, +\infty \rp, \lp -\infty, y \rp \cap \lp x, +\infty \rp = \emptyset$.
            \end{itemize}
        \item Aquest dibuix està en contrucció. % TODO dibuix!!
        \item Tenemos que ver que $\T_{\text{ord}} \subset \T_\leq$, es decir, $\T_{\text{ord}} \subseteq \T_\leq$ y $\T_\leq \neq \T_{\text{ord}}$.
            \begin{itemize}
                \item Veamos que $\T_{\text{ord}} \subseteq \T_\leq$. Sea $U \subseteq \T_{\text{ord}},\, \forall x \equiv \lp x_1, x_2 \rp \in U, \, \exists r \in \real^+ \tq B_r \lp x \rp \subseteq U$, y por tanto, $A = \lp \lp x_1 - r, x_2 \rp, \lp x_1 + r, x_2 \rp \rp \subseteq B_r \lp x \rp, A \in \T_\leq$. Así pues, todos los puntos de $U$ son interiores en la topología del orden y por tanto $U \in \T_\leq$.
                \item Veamos que $\T_\leq \neq \T_{\text{ord}}$. Sean $x = \lp 0,0 \rp, y = \lp 0, 1 \rp$, entonces $\lp x, y \rp \in \T_\leq, \lp x, y \rp \notin \T_{\text{ord}}$. Por tanto $\T_\leq \neq \T_{\text{ord}}$.
            \end{itemize}
        \item Son las topologías discretas.
    \end{enumerate}
\end{eje}
\begin{eje}
    \begin{enumerate}[(a)]
        \item[]
        \item Comprobamos que
            \begin{enumerate}[i)]
                \item $\emptyset = \left[ x, x \rp \in \T_\ell$ y $X = \bigcup\limits_{x \in X} \left[ x, \infty \rp \in \T_\ell$.
                \item Sea $U = \bigcup\limits_{i \in I} U_i$ la unión de un numero arbitrario de abiertos, entonces, $\forall x \in U, \exists i \in I \tq x \in U_i \subseteq U$ y por tanto $x$ es un punto interior y $U$ es un abierto.
                \item Sea $U = \bigcap\limits_{i = 1}^n U_i$ la intersección de un numero finito de abiertos, entonces, $\forall x \in U, \forall i \in \left\{ 1, \dots, n \right\}, \exists a_i, b_i \tq x \in \left[a_i, b_i\right) \subseteq U_i$, porque $x \in U_i$ y $U_i$ abierto. Sean
                    \begin{gather*}
                        a = \max_{i \in \left\{ 1, \dots, n \right\}} \left\{a_i\right\}, \\
                        b = \min_{i \in \left\{ 1, \dots, n \right\}} \left\{b_i\right\},
                    \end{gather*}
                entonces $x \in \left[ a, b \rp \subseteq \bigcap\limits_{i = 1}^n U_i$ y por tanto $x$ es un punto interior y $U$ es un abierto.
            \end{enumerate}
        \item Sea $\B = \left\{ \left[ a, b \rp \colon a \in X, b \in X \cup \left\{ \infty \right\} \right\}$, comprobamos que
            \begin{enumerate}[i)]
                \item $\forall x \in X, x \in \left[ x, \infty \rp \in \B$.
                \item $\forall a, b, c, d \in X \cup \left\{ \infty \right\}$, si $\left[ a, b \rp \cap \left[ c, d \rp \neq \emptyset$, entonces,
                    \begin{gather*}
                        \alpha = \max \left\{ a, c \right\}, \\
                        \beta = \min \left\{ b, d \right\}, \\
                        \left[ a, b \rp \cap \left[ c, d \rp = \left[ \alpha, \beta \rp.
                    \end{gather*}
                    y por tanto $\forall x \in \left[ a, b \rp \cap \left[ c, d \rp, x \in \left[ \alpha, \beta \rp \subseteq \left[ a, b \rp \cap \left[ c, d \rp$, y $\left[ \alpha, \beta \rp \in \B$.
            \end{enumerate}
        \item \item[] % OJO amb aqueta cutrada
            \begin{center}
                \begin{tabular}{|l||c|c|c|c|c|c|} \hline
                    & $\lp a, b \rp$ & $\left[ a, b \rp$ & $\lp a, b \right]$ & $\left[ a,b \right]$ & $\lc 0 \rc \cup \lc \sfrac{1}{n} \rc_{n\geq 1}$ & $\lc 0 \rc \cup \lc \sfrac{-1}{n} \rc_{n\geq 1}$ \\ \hline \hline
                    Adherencia & $\left[a,b\rp$ & $\left[ a, b \rp$ & $\left[ a, b \right]$ & $\left[ a, b \right]$ & $\lc 0 \rc \cup \lc \sfrac{1}{n} \rc_{n\geq 1}$ & $\lc 0 \rc \cup \lc \sfrac{-1}{n} \rc_{n\geq 1}$ \\ \hline
                    Interior & $\lp a, b \rp$ & $\left[ a, b \rp$ & $\lp a, b \rp$ & $\left[ a, b \rp$ & $\emptyset$ & $\emptyset$ \\ \hline
                    Frontera & $\lc a \rc$ & $\emptyset$ & $\lc a, b \rc$ & $\lc b \rc$ & $\lc 0 \rc \cup \lc \sfrac{1}{n} \rc_{n\geq 1}$ & $\lc 0 \rc \cup \lc \sfrac{-1}{n} \rc_{n\geq 1}$ \\ \hline
                    Acumulación & $\left[a,b\rp$ & $\left[ a, b \rp$ & $\left[ a, b \rp$ & $\left[ a, b \right)$ & $\lc 0 \rc$ & $\emptyset$ \\ \hline
                    Puntos aislados & $\emptyset$ & $\emptyset$ & $\lc b \rc$ & $\lc b \rc$ & $\lc \sfrac{1}{n} \rc_{n\geq 1}$ & $\lc 0 \rc \cup \lc \sfrac{-1}{n} \rc_{n\geq 1}$ \\ \hline
                \end{tabular}
            \end{center}
    \end{enumerate}
\end{eje}
\begin{eje}
    \begin{enumerate}[(a)]
        \item[]
        \item Comprobamos que
            \begin{enumerate}[i)]
                \item $\emptyset = \left[ x, x \rp \in \T_\ell$
                \item Sea $U = \bigcup\limits_{i \in I} U_i$ la unión de un numero arbitrario de abiertos, entonces, $\forall x \in U, \exists i \in I \tq x \in U_i \subseteq U$ y por tanto $x$ es un punto interior y $U$ es un abierto.
                \item Sea $U = \bigcap\limits_{i = 1}^n U_i$ la intersección de un numero finito de abiertos, entonces, $\forall x \in U, \forall i \in \left\{ 1, \dots, n \right\}, \exists a_i, b_i \tq x \in \left(a_i, b_i\right) \subseteq U_i$. Sean
                    \begin{gather*}
                        a = \max_{i \in \left\{ 1, \dots, n \right\}} \left\{a_i\right\}, \\
                        b = \min_{i \in \left\{ 1, \dots, n \right\}} \left\{b_i\right\},
                    \end{gather*}
                entonces $x \in \lp a, b \rp \subseteq \bigcap\limits_{i = 1}^n U_i$ y por tanto $x$ es un punto interior y $U$ es un abierto.
            \end{enumerate}
    \end{enumerate}
\end{eje}

\begin{eje}
 Este ejercicio aún no está resuelto.
\end{eje}

\begin{eje}
 Este ejercicio aún no está resuelto
\end{eje}

\begin{eje}
 Este ejercicio aún no está resuelto
\end{eje}

\begin{eje}
 Sea $\T$ una topología y $X$ un espacio topológico. Por definición de $\psi$ se tiene que $\forall A\subseteq X,\, \psi\lp A\rp =\overline{A}$ es un cerratdo de $\T$. Por lo tanto, podemos definir un conjunto de abiertos de $\T$ como
 \[
  \B = \lc B\subseteq X\, | \,\exists A \subseteq X \tq B=X\setminus \psi\lp A\rp \rc
 \]
 que son abiertos por ser el complementario de un cerrado ($\overline{A}$ es el cerrado más pequeño que contiene a $A$).
 
 Ahora demostraremos que $\B$ es una base de la topología $\T$. 

 Primero veamos que
 \begin{gather*}
  X\setminus \psi\lp \varnothing\rp = X\setminus\varnothing = X \in \B\\
  X\setminus \psi\lp X\rp =X\setminus X = \varnothing \in \B.
 \end{gather*}
 
Entonces tenemos lo siguiente:
\begin{enumerate}[i)]
 \item $\forall x\in X,\, \exists A\subseteq \tq \overline{A}\cup\lc x\rc = \varnothing \implies x\in B = X\setminus \psi\lp A\rp$.
 En concreto podemos coger $A=\varnothing$.
 \item Sean $B_1,B_2 \in \B$ y sea $x\in X \tq x\in B_1\cap B_2$. Entonces
 \begin{gather*}
  x\in B_1\cap B_2 = \lp X\setminus \psi\lp A_1\rp \rp \cap \lp X\setminus\psi\lp A_2\rp \rp = X\setminus \lp\psi\lp A_1\rp\cup\psi\lp A_2\rp\rp = \\
  = X\setminus\psi\lp A_1 \cup A_2\rp = X\setminus\psi\lp\psi\lp A_1 \cup A_2\rp \rp,
 \end{gather*}
 y como $A=\psi\lp A_1\cup A_2\rp \subseteq X$, $B=X\setminus\psi\lp A\rp \in \B$, tenemos que $\exists B \in \B \tq x\in B \subseteq B_1 \cap B_2$.
\end{enumerate}
Por lo tanto, $\B$ es la base de una topología (de $\T$), lo que nos dice que como $\B$ son los mínimos abiertos que debe contener una topología que cumpla que $\psi\lp A\rp = \overline{A}$ y una base define una única topología, existe una única topología $\T$ que lo cumple. 
\end{eje}


\chapter{Variables aleatòries discretes}

\section[Definició i objectes relacionats]
    {Definició i objectes relacionats
    \sectionmark{Definició i objectes relacionats}}
    \sectionmark{Definició i objectes relacionats}

\begin{defi}[variable aleatòria!discreta]
    Sigui $\lp \Omega, \Asuc, p\rp$ un e. prob. i $X\colon \Omega\to\real$ una variable aleatòria. Diem que $X$ és discreta si
    $\im(X)$ és numerable.
\end{defi}

\begin{obs}
    En la pràctica $\im(X) = \lc x_1 < x_2 < \dots \rc$ és un conjunt numerable ordenat, en els casos que veurem $\im(X)\subseteq\z$.
    Escriurem $p_i = p(X=x_i)$.
\end{obs}

\begin{obs}
    Sigui $A\subset\mathcal{B}$, aleshores $p_X(A) = p(x\in A) = \sum_{i\in A}{p_i}$.
\end{obs}

\begin{obs}
    Donat $\lc x_i \rc_{i \ge 1} \subset\real$ creixent i valors $\lc x_i \rc_{i \ge 1} \subset [0,1] \tq \sum{p_i}=1$, es pot definir
    una variable discreta $X$ que pren valors a $\lc x_i \rc$ tal que $p(X=x_i) = p_i$ $\forall\,i$.
\end{obs}

\begin{obs}
    La funció de distribució, amb $|\im(X)|<+\infty$, és
    \[F_X(x) = \begin{cases}
        0 &\text{si } x<x_1\\
        p_1+\cdots+p_j &\text{si } x_j \le x < x_{j+1}\\
        1 &\text{si } x \ge x_n
    \end{cases}.\]
\end{obs}

\begin{prop}[Operador esperança] Sigui $X$ v.a. discreta, aleshores
    \begin{enumerate}[i)]
        \item $\esp[X] = \sum\limits_{i \ge 1}{x_i p_i}$,
        \item $g\colon \real\to\real$ mesurable, $\esp[g(x)] = \sum\limits_{i \ge 1}{g(x_i) p_i}$.
    \end{enumerate}
\end{prop}

\begin{proof}
    \begin{enumerate}[i)]
        \item[]
        \item Fent servir la definició d'esperança tenim que
	  \[ \esp[X] = \int_{\real}{x\dif P_x} = \sum{x_i P_X(X=x_i) + \int_{\real\setminus\cup\{x_i\}}{x\mathbb{I}(x) \dif P_X}} = \sum_{i \ge 1}{x_i p_i} + 0\]
            perquè $P_X(\real\setminus\cup\{x_i\}) = 0$.
        \item Directe a partir del cas anterior i del fet que $g$ és mesurable.
    \end{enumerate}
\end{proof}

\begin{obs}
    Sigui $X$ una variable aleatòria, aleshores,
    \[\var[X] = \esp[X^2] - \esp[X]^2 = \sum_{i \ge 1}{x_i^2 p_i} - \left(\sum_{i \ge 1}{x_i p_i}\right)^2.\]
\end{obs}

\begin{prop}
    $X, Y$ v.a. discretes, $\im(X) = \lc x_i \rc_{i \ge 1}$, $\im(Y) = \lc y_i \rc_{i \ge 1}$, $g\colon \real^2\to\real$. Aleshores
    \[\esp[g(X,Y)] = \sum_{i,j \ge 1}{g(x_i,y_i) p(X=x_i, Y=y_i)}.\]
\end{prop}
\begin{proof}
    Directe a partir de la definició i del fet que $g$ és mesurable.
\end{proof}


\begin{prop}
    Siguin $X,Y$ v.a. discretes. Són independents si i només si, $\forall x \in\im(X)$ i $\forall y \in\im(Y)$,
    \[P(X=x, Y=y) = P(X=x)P(Y=y)\]
\end{prop}
\begin{proof}
    Exercici.
\end{proof}


\begin{defi}[vector de variables aleatòries!discret]
    Sigui $(X,Y)\colon \Omega\to\real^2$ un vector de variables aleatories. Direm que és discret si $\im((X,Y))$ és numerable.
\end{defi}

\begin{obs}
    Sigui $(X,Y)$ vector de variables aleatories. Aleshores és discret $\iff$ $X$ i $Y$ són discretes.
\end{obs}

\begin{defi}
    Sigui $(X,Y)$ vector de variables aleatories discret, $\forall(X,Y) \in \im((X,Y))$ definim
    \[P_{(X,Y)}(x,y) = P(X=x)P(Y=y)\]
    que si $X,Y$ són independents és $P(X=x,Y=y)$.
\end{defi}

\begin{lema}
    Si $X,Y$ són v.a. discretes independents, amb $\esp[|X|], \esp[|Y|] < +\infty$,
    \[\esp[XY] = \esp[X]\esp[Y].\]
\end{lema}
%TODO Acabar la demo porfa
\begin{proof}
    \[\begin{aligned}
    \esp[xy] =& \int_\real x\dif P_{XY} \\
    &= \sum_{u\in\im(X,Y)}{uP(XY=u)} \\
    &= \sum_{x\in\im(X)\setminus\{0\}} \lp\sum_{\frac{u}{x}\in\im(Y)} uP\lp X=x,Y=\frac{u}{x} \rp\rp
    \end{aligned}\]
\end{proof}


\section{Funció generadora de probabilitat}
D'aquí en endavant, prendrem $X$ variable aleatòria discreta amb $\im(X) \subseteq \n_{\geq 0}$.
\begin{defi}[funció!generadora de probabilitat]
    La funció generadora de probabilitat de $X$ \'es la sèrie formal de potències
    \[G_X(z) = \sum_{n \geq 0} P(X = n) z^n.\]
    Podem pensar-la també com $\esp[z^X]$.
\end{defi}

\begin{prop}
    $G_X(z)$ satisfà les següents propietats:
    \begin{enumerate}[i)]
    \item $G_X(z)$ \'es una funció holomorfa al voltant de $z = 0$ amb radi de convergència
        major o igual a $1$.
    \item $G_X(0) = P(X = 0)$ i $G_X(1)$ = 1.
    \item $\eval{\frac{\text{d}^kG_X(z)}{\text{d}z}}_{z = 1} = 
        \esp\lb X(X-1)\dots(X-k+1)\rb = \esp \lb (X)_k\rb $.
    \end{enumerate}
\end{prop}

\begin{proof}
    \begin{enumerate}[i)]
        \item[]
        \item Si $\rho \in \cx$, $|\rho| < 1$, aleshores:
            \[
                0 \leq |G_X(\rho)| = \left|\sum_{n \geq 0} P(X = n) \rho^n \right|
                \leq \sum P(X=n) |\rho|^n
            \]
            que, quan $|\rho| \leq 1$, \'es menor o igual a
            \[
                \sum_{n \geq 0} P(X = n) = 1.
            \]
            Per tant $G_X(\rho)$ \'es analítica (es pot expressar com una sèrie de
            potències convergent) a $B_1(0)$, i per tant, com s'ha vist a variable
            complexa, $G_x(\rho)$ \'es holomorfa a $B_1(0)$ (per tant infinitament
            derivable en sentit complex).
        \item  Directe a partir de la definició.
        \item Si derivem terme a terme obtenim
            \[
                \frac{\text{d}^kG_X(z)}{\text{d}z} = \frac{\text{d}^k}{\text{d}z} \lp \sum_{n \geq 0} P(X = n) z^n\rp=
                \sum_{n \geq 0} n(n-1)\dots(n-k+1)P(X = n) z^{n-k},
            \]
            que avaluat en $z = 1$ \'es
            \[ 
                \sum_{n \geq 0} n(n-1)\dots(n-k+1)P(X=n) = \esp\lb(X)_n\rb.
            \]
    \end{enumerate}
\end{proof}

\begin{example}
   Definim $X$ com una variable aleatòria discreta tal que $P(X = 0) = 0$ i
   $P(X = n) = \frac{6}{\pi^2}\cdot \frac{1}{n^2}$. Aleshores
    \[
        G_X(z) =  \frac{6}{\pi^2} \sum_{n \geq 1} \frac{1}{n^2}z^n,
    \]
    que t\'e radi de convergència igual a $1$, i
    \[
        G_X(1) = \frac{6}{\pi^2} \sum_{n \geq 1} \frac{1}{n^2} =
        \frac{6}{\pi^2}\cdot \frac{\pi^2}{6} = 1.
    \]
    Finalment, en calculem la seva esperança,
    \[
        \esp[X] = \eval{\dv{G_X(z)}{z}}_{z = 1} = 
        \frac{6}{\pi^2}\sum_{n \geq 1} \frac{1}{n} = \infty.
    \]
\end{example}

\begin{obs}
    $G_X(z)$ codifica totes les probabilitats $P(X = n)$ i per tant coneixent
    $G_x(z)$ coneixem $X$.
\end{obs}

L'aplicació m\'es útil de les funcions generadores de probabilitat \'es que
ens permet trobar convolucions discretes de variables aleatòries.

\begin{obs}
    Siguin $X, Y$ variables aleatòries discretes, amb $\im(X) = \im(Y) =
    \n_{\geq 0}$. Aleshores
    \[
        P(X+Y = n) = \sum_{k = 0}^{n} P(X+Y = n, X = k) = 
        \sum_{k = 0}^{n} P(Y = n-k, X = k). 
    \]
\end{obs}

\begin{prop}
    Si $X, Y$ són variables aleatòries discretes independents amb $\im(X) = \im(Y) =
    \n_{\geq 0}$ aleshores:
    \[
        G_{X+Y}(z) = G_X(z)G_Y(z)
    \]
\end{prop}

\begin{proof}
    \[
        G_X(z)G_Y(z) = \sum_{i \geq 0} P(X = i) z^i  \sum_{j \geq 0} P(X = j) z^j =
         \sum_{i, j \geq 0} P(X = i)P(Y = j) z^{(i+j)}
    \]
    I, com $X$ i $Y$ són independents, podem unir el producte i obtenim
    \begin{align*} 
        \sum_{i, j \geq 0} P(X = i, Y = j) z^{(i+j)} & = \sum_{n \geq 0} \sum_{i = 0}^n
        P(X = i, Y = n-i)z^n =\\
        &= \sum_{n \geq 0} \sum_{i = 0}^n P(X = i, X+Y = n)z^n =\\
        & = \sum_{n \geq 0} \sum_{i = 0}^n P(X = i | X+Y = n) P(X+Y = n)z^n = \\
        & = \sum_{n \geq 0} \ P(X+Y = n)z^n \sum_{i = 0}^n P(X = i | X+Y = n).
    \end{align*}
    Si observem que la suma interior val $1$ perquè està sumant la probabilitat de tots
    els esdeveniments possibles, ens queda
    \[
        \sum_{n \geq 0} P(X+Y = n)z^n = G_{X+Y}(z).
    \]
\end{proof}

\begin{obs}
    Això \'es equivalent a que si $X$ i $Y$ són variables aleatòries discretes independents
    amb $\im(X) = \im(Y) = \n_{\geq 0}$ aleshores
    \[
        \esp[z^X]\esp[z^Y] = \esp[z^{X+Y}].
    \]
\end{obs}

\begin{obs}
    En general, si $X_1, \dots, X_n$ són variables aleatòries discretes independents amb
    $\im X_i = \n_{\geq 0}$:
    \[
        G_{X_1,\dots,X_n}(z) = \prod_{i = 1}^n G_{X_i}(z).
    \]
\end{obs}

\section{Models de variables aleatòries discretes}

En aquesta secció introduirem les variables aleatòries discretes més comunes que torbarem

\begin{obs}
    En general escriurem $X \sim Y$ si $X$ i $Y$ tenen la mateixa distribució de probabilitat.
\end{obs}

\subsection*{Distribució de Bernoulli}

Modela l'èxit o fracàs d'un experiment amb probabilitat $p$ d'èxit

\begin{defi}[distribució!de Bernoulli]
    Sigui $X$ una variable aleatoria. Direm que $X$ segueix una distribució de Bernoulli 
    \[X \sim \operatorname{B}(p) \iff \begin{cases}
                       p(X=1) = p \\
                       p(X=0) = 1-p
                       \end{cases}.
    \]
    També es pot escriure $\operatorname{Be}(p)$.
\end{defi}

\begin{prop}
    Sigui $X$ una variable aleatòria que segueix una distribució de Bernoulli. Aleshores,
    \begin{enumerate}[i)]
        \item $G_X(z) = (1-p)z^0 + pz^1 = (1-p) + p(z)$,
        \item $\esp[X] = p$,
        \item $\var[X] = p(1-p)$.
    \end{enumerate}
\end{prop}

\begin{proof} %TODO falten apartats per demostrar
    \begin{enumerate}[i)]
        \item[]
        \item[iii)] $\esp[x^2-x]= \esp[x(x-1)] = 0 \implies \esp[x^2] = p \implies \var[x] = \esp[x^2] + \esp[x]^2
        = p(1-p)$
    \end{enumerate}
\end{proof}

\subsection*{Distribució binomial}

Modela el nombre d'èxits en fer $N$ experiments independents, on cadascun és $\operatorname{Be}(p)$.

\begin{defi}[distribució!binomial]
  Sigui $X$ una variable aleatoria. Direm que $X$ segueix una distribució binomial
    \[X \sim \operatorname{Bin}(N,p) \iff X = X_1 + \cdots + X_N,\]
    on $\lc X_i \rc_{i=1}^{N}$ són independents i $X_i \sim \operatorname{B}(p) \enspace\forall i$.
\end{defi}

\begin{prop}
    Sigui $X$ una variable aleatòria que segueix una distribució binomial. Aleshores,
    \begin{enumerate}[i)]
        \item $p(X=i) = \binom{N}{i}p^i(1-p)^{N-i}$,
        \item $G_X(z) = \sum_{i=0}^{n} \binom{N}{i}p^i(1-p)^{N-i}z^i$,
        \item $\esp[X] = N\esp[X_1] = Np$,
        \item $\var[X] = N\var[X_1] = Np(1-p)$.
    \end{enumerate}
\end{prop}

\begin{proof} %TODO falten apartats per demostrar
    \begin{enumerate}[i)]
        \item[]
        \item[ii)] Per ser $X_i$ independents,
        \begin{gather*}
        G_X(z) = G_{X_1 + \cdots + X_N}(z) = \prod_{i=1}^{n}G_{X_i}(z) = (pz+(1-p))^N = \\
        = \sum_{i=0}^n \binom{N}{i}(pz)^i(1-p)^{N-i} = \sum_{i=0}^{n} \binom{N}{i}p^i(1-p)^{N-i}z^i.
        \end{gather*}
    \end{enumerate}
\end{proof}

\begin{obs} La suma de dos variables amb aquesta distribució també té distribució binomial:
    \begin{gather*}
    \begin{rcases*} X \sim \operatorname{Bin}(N_1,p) \\ Y \sim \operatorname{Bin}(N_2,p) \end{rcases*} \implies
    \begin{rcases*} G_X(z) = (pz+(1-p))^{N_1} \\ G_Y(z) = (pz+(1-p))^{N_2} \end{rcases*}
    \implies \\
    \implies G_{X+Y}(z) = G_X(z)G_Y(z) = (pz+(1-p))^{N_1+N_2} \implies \\
    \implies X+Y \sim \operatorname{Bin}(N_1+N_2,p).
    \end{gather*}
\end{obs}

\subsection*{Distribució uniforme}

\begin{defi}[distribució!uniforme]
    Sigui $X$ una variable aleatoria. Direm que $X$ segueix una distribució uniforme
    \[X \sim \operatorname{U}[1,n] \iff p(X=i)=\frac{1}{N} \text{ per } i = 1,\dots,N.\]
\end{defi}

\begin{prop}
    Sigui $X$ una variable aleatòria que segueix una distribució binomial. Aleshores,
    \[ G_X(z) =  \frac{1}{N}\frac{z(z^N-1)}{z-1}.\]
\end{prop}

\begin{proof}
  Directament de la definició de distribució uniforme tenim que
  \[G_X(z) = \sum_{n=1}^N \frac{1}{N}z^n = \frac{1}{N}(z+z^2+\cdots+z^N) = \frac{1}{N}\frac{z(z^N-1)}{z-1}.\]
\end{proof}


\subsection*{Distribució de Poisson}

S'usa per modelar successos ``estranys'' (persones en una cua, emissió de partícules, etc).

\begin{defi}[distribució!de Poisson]
    Sigui $X$ una variable aleatoria. Direm que $X$ segueix una distribució de Poisson
    \[X \sim \operatorname{P}(\lambda) \iff P(X=i) = \frac{\lambda^i }{i!}e^{-\lambda}.\]
    També es pot escriure $\operatorname{Po}(\lambda)$.
\end{defi}

\begin{obs}
    La distribució està ben definida:
    \[\sum_{i=0}^\infty p(X=i) = \sum_{i=0}^\infty \frac{1}{i!}\lambda^i e^{-\lambda} = e^\lambda e^{-\lambda} = 1.\]
\end{obs}

\begin{prop}
    Sigui $X$ una variable aleatòria que segueix una distribució de Poisson. Aleshores,
    \begin{enumerate}[i)]
        \item $G_X(z) = e^{\lambda(z-1)}$,
        \item $\esp[X] = \lambda$,
        \item $\var[X] = \lambda$.
    \end{enumerate}
\end{prop}

\begin{proof}
    \begin{enumerate}[i)]
        \item[]
        \item $G_X(z) = \sum\limits_{i=0}^\infty \frac{1}{i!}\lambda^i e^{-\lambda} z^i =
            e^{-\lambda} \sum\limits_{i=0}^\infty \frac{1}{i!}(\lambda z)^i = e^{-\lambda}e^{\lambda z} = e^{\lambda(z-1)}$.
        \item $\esp[X] = \lambda e^{\lambda(z-1)}|_{z=1} = \lambda$.
        \item $\esp[X(X-1)] = \lambda^2 e^{\lambda(z-1)}|_{z=1} = \lambda^2 \implies \esp[x^2] = \lambda^2+\lambda$,  i per tant, 
         $\var[X] = \esp[X^2] - \esp[X]^2 = \lambda^2 + \lambda - \lambda^2 = \lambda$.
    \end{enumerate}
\end{proof}

\begin{obs} La suma de dos variables amb distribució de Poisson també té distribució de Poisson:
    \begin{gather*}
    \begin{rcases*} X \sim \operatorname{Po}(\lambda_1) \\ Y \sim \operatorname{Po}(\lambda_2) \end{rcases*} \implies
    \begin{rcases*} G_X(z) = e^{\lambda_1(z-1)} \\ G_Y(z) = e^{\lambda_2(z-1)} \end{rcases*}
    \implies \\
    \implies G_{X+Y}(z) = G_X(z)G_Y(z) = e^{(\lambda_1+\lambda_2)(z-1)} \implies \\
    \implies X+Y \sim \operatorname{Po}(\lambda_1+\lambda_2).
    \end{gather*}
\end{obs}

\subsection*{Distribució geomètrica}


\subsection*{Distribució binomial negativa}


\end{document}
