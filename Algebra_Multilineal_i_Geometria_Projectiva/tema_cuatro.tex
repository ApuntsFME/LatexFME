\chapter{Cónicas y cuádricas}

\section{Definiciones y propiedades básicas}

\begin{defi}
	\begin{enumerate}
		\item[]
		\item Una cuádrica $Q$ de $\Po^n_\k = \Po\lp \E\rp$ es la clase de equivalencia de una forma cuadrática  $q \colon \E \to \k, q\sim q' \iff \exists \lambda\neq0\tq q' = \lambda q$. Notación: $Q = [q]$.
		\item Si $n = 2$, las llamamos cónicas.
		\item $p = [u] \in \Po^n, p\in Q \iff q\lp u\rp=0$ ($p$ es un punto de $Q$).
		\item También notaremos por $Q$, el conjunto de todos los puntos de $Q$:
			\[ Q = \left\{ p=[u]\;|\;p\lp u\rp=0\right\} \]
	\end{enumerate}
\end{defi}
\begin{example}
	\begin{enumerate}
		\item[]
		\item A $\Po^2_\real$, sigui $Q \colon x_0^2 + x_1^2 - x_2^2$ una cònica.
		\begin{itemize}
			\item $p = \left[\lp1:0:1\rp^t\right] \in Q:$ verifica $x_0^2 + x_1^2 -x_2^2=0$.
			\item $p' = \left[\lp0:1:1\rp^t\right] \in Q$.
			\item $p'' = \left[\lp1:1:\sqrt{2}\rp^t\right] \in Q$.
		\end{itemize}
	\end{enumerate}
\end{example}

% A PARTIR D'AQUI JORDI
