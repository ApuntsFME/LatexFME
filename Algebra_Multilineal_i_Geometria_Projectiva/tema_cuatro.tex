\chapter{Cónicas y cuádricas}

\section{Definiciones y propiedades básicas}

\begin{defi}
	\begin{enumerate}
		\item[]
		\item Una cuádrica $Q$ de $\Po^n_\k = \Po\lp \E\rp$ es la clase de equivalencia de una forma cuadrática  $q \colon \E \to \k, q\sim q' \iff \exists \lambda\neq0\tq q' = \lambda q$. Notación: $Q = [q]$.
		\item Si $n = 2$, las llamamos cónicas.
		\item $p = [u] \in \Po^n, p\in Q \iff q\lp u\rp=0$ ($p$ es un punto de $Q$).
		\item También notaremos por $Q$, el conjunto de todos los puntos de $Q$:
			\[ Q = \left\{ p=[u]\;|\;q\lp u\rp=0\right\} \]
	\end{enumerate}
\end{defi}
\begin{example}
	\begin{enumerate}
		\item[]
		\item A $\Po^2_\real$, sigui $Q \colon x_0^2 + x_1^2 - x_2^2$ una cònica.
		\begin{itemize}
			\item $p = \left[\lp1,0,1\rp^t\right] \in Q:$ verifica $x_0^2 + x_1^2 -x_2^2=0$.
			\item $p' = \left[\lp0,1,1\rp^t\right] \in Q$.
			\item $p'' = \left[\lp1,1,\sqrt{2}\rp^t\right] \in Q$.
		\end{itemize}
		\item A $\Po^2_\real$, sigui $Q \colon x_0^2 + x_1^2 - x_2^2 \implies$ puntos = $\emptyset$.
		\begin{itemize}
			\item $Q_1 \colon x_0^2 + 2x_1^2 = 0 \implies p = \lp0:0:1\rp$ punto único.
			\item $Q_1 \colon 3x_0^2 + 5x_1^2 = 0 \implies p = \lp0:0:1\rp$ punto único.
		\end{itemize}
	\end{enumerate}
\end{example}
%\begin{obs}
%	$Q \longleftrightarrow q \colon E \to \k$ forma cuadrática $\stackrel{\car\k\neq2}{\longleftrightarrow} \varphi \colon E\times E \to \k$ forma bilineal. simétrica.
%\end{obs}

% COSAS_QUE_FALTAN LA OBS ANTERIOR ESTA A MITJES (PER AIXO ESTA COMENTADA)

% A PARTIR D'AQUI JORDI

\begin{defi}[(intersección de cónicas y variedades lineales)]
    Sea $Q= \left[ q \right] \subseteq \Po ^n$ una cuádrica y sea $V = \left[ F \right] \subseteq \Po ^n$ una variedad linea. Definimos su intersección como
    \[
        Q_{|V} := \left[q_{|F} \right].
    \]
    Si $q_{|F} \neq 0$, entonces $Q_{|V}$ es una cuádrica en $V$.
\end{defi}

\begin{example}
    En $\Po _{\real} ^3$, sea $q=x_0^2+x_2^2+x_2^2-x_3^2+2x_0x_1-2x_1x_2=0$ y sea $V=\left[ F \right] \colon x_0+x_1=0$ (un plano). Tenemos que $F=\left[ \lp 1, -1, 0, 0 \rp ^t, \lp 0, 0, 1, 0 \rp ^t, \lp 0, 0, 0, 1 \rp ^t \right]$.
\end{example}
