\chapter{Cónicas y cuádricas}

\section{Definiciones y propiedades básicas}

\begin{defi}
    \begin{enumerate}
        \item[]
        \item Una cuádrica $Q$ de $\Po^n_\k = \Po\lp \E\rp$ es la clase de equivalencia de una forma cuadrática  $q \colon \E \to \k, q\sim q' \iff \exists \lambda\neq0\tq q' = \lambda q$. Notación: $Q = [q]$.
        \item Si $n = 2$, las llamamos cónicas.
        \item $p = [u] \in \Po^n, p\in Q \iff q\lp u\rp=0$ ($p$ es un punto de $Q$).
        \item También notaremos por $Q$, el conjunto de todos los puntos de $Q$:
            \[ Q = \left\{ p=[u]\;|\;q\lp u\rp=0\right\} \]
    \end{enumerate}
\end{defi}
\begin{example}
    \begin{enumerate}
        \item[]
        \item A $\Po^2_\real$, sigui $Q \colon x_0^2 + x_1^2 - x_2^2$ una cònica.
        \begin{itemize}
            \item $p = \left[\lp1,0,1\rp^t\right] \in Q:$ verifica $x_0^2 + x_1^2 -x_2^2=0$.
            \item $p' = \left[\lp0,1,1\rp^t\right] \in Q$.
            \item $p'' = \left[\lp1,1,\sqrt{2}\rp^t\right] \in Q$.
        \end{itemize}
        \item A $\Po^2_\real$, sigui $Q \colon x_0^2 + x_1^2 - x_2^2 \implies$ puntos = $\emptyset$.
        \begin{itemize}
            \item $Q_1 \colon x_0^2 + 2x_1^2 = 0 \implies p = \lp0:0:1\rp$ punto único.
            \item $Q_1 \colon 3x_0^2 + 5x_1^2 = 0 \implies p = \lp0:0:1\rp$ punto único.
        \end{itemize}
    \end{enumerate}
\end{example}
%\begin{obs}
%    $Q \longleftrightarrow q \colon E \to \k$ forma cuadrática $\stackrel{\car\k\neq2}{\longleftrightarrow} \varphi \colon E\times E \to \k$ forma bilineal. simétrica.
%\end{obs}
% COSAS_QUE_FALTAN LA OBS ANTERIOR ESTA A MITJES (PER AIXO ESTA COMENTADA)
\begin{defi}
    $\Po^n = \Po\lp\E\rp$, sea $q \colon \E \to \k$ asociada a $\varphi \colon \E \times \E \to \k, Q = [q]$. Sean $\R$ una referencia de $\Po^n$, $B$ una base de $\E$ adaptada a $\R$.
    \[ M_\R \lp Q\rp := M_B\lp q\rp \stackrel{\text{por def.}}{=} M_B\lp\varphi\rp \]
    $M_\R\lp Q\rp$ está definida salvo multiplicar por una constante.
\end{defi}
\begin{example}
    En $\Po^2_\real$, sean $\R$ (de $\Po^2$), $B$ (de $\real^3$). Sea $Q\colon x_0^2+ x_1^2+2x_2^2-2x_0x_1+4x_0x_2 + 6x_1x_2 = 0$.
    \begin{gather*}
        M_\R\lp Q\rp = M_B\lp q\rp = A =
        \begin{pmatrix}
            1 & -1 & 2 \\
            -1 & 1 & 3 \\
            2 & 3 & 2 \\
        \end{pmatrix}, \\
        \quad p\in Q \iff
        \begin{pmatrix}
            x_0 & x_1 & x_2 \\
        \end{pmatrix}
        \begin{pmatrix}
            1 & -1 & 2 \\
            -1 & 1 & 3 \\
            2 & 3 & 2 \\
        \end{pmatrix}
        \begin{pmatrix}
            x_0 \\
            x_1 \\
            x_2 \\
        \end{pmatrix}
        = 0.
    \end{gather*}
\end{example}
\begin{obs}
    Si $A = M_\R\lp Q\rp, p_\R = \lp x_0:\dots:x_n\rp = x$, entonces tenemos que $p\in Q \iff x^tAx = 0$.
\end{obs}
\begin{example}
    Sean $Q = [q], \R, A = M_\R\lp Q\rp$. Sea $\R'$ otro sistema de referencia en $\Po^n$ y sea $S_{\R, \R'}$ la matriz de cambio de sistema de referencia de $\R$ a $\R'$. Entonces $A' = M_{\R'}\lp Q\rp = S^tAS$.
\end{example}

% A PARTIR D'AQUI JORDI

\begin{defi}[(intersección de cónicas y variedades lineales)]
    Sea $Q= \left[ q \right] \subseteq \Po ^n$ una cuádrica y sea $V = \left[ F \right] \subseteq \Po ^n$ una variedad lineal. Definimos su intersección como
    \[
        Q_{|V} := \left[q_{|F} \right].
    \]
    Si $q_{|F} \neq 0$, entonces $Q_{|V}$ es una cuádrica en $V$.
\end{defi}

\begin{example}
    En $\Po _{\real} ^3$, sea $q=x_0^2+x_2^2+x_2^2-x_3^2+2x_0x_1-2x_1x_2=0$ y sea $V=\left[ F \right] \colon x_0+x_1=0$ (un plano). Tenemos que $F=\left[ \lp 1, -1, 0, 0 \rp ^t, \lp 0, 0, 1, 0 \rp ^t, \lp 0, 0, 0, 1 \rp ^t \right]$.
    \begin{gather*}
        A=M_B\lp q \rp = \begin{pmatrix}
            1 & 1 & 0 & 0 \\
            1 & 1 & -1 & 0 \\
            0 & -1 & 1 & 0 \\
            0 & 0 & 0 & -1
        \end{pmatrix} \implies \\
        \implies M_B \lp q_{|F} \rp = \begin{pmatrix}
            1 & -1 & 0 & 0 \\
            0 & 0 & 1 & 0 \\
            0 & 0 & 0 & 1
        \end{pmatrix} \begin{pmatrix}
            1 & 1 & 0 & 0 \\
            1 & 1 & -1 & 0 \\
            0 & -1 & 1 & 0 \\
            0 & 0 & 0 & -1
        \end{pmatrix} \begin{pmatrix}
            1 & 0 & 0 \\
            -1 & 0 & 0 \\
            0 & 1 & 0 \\
            0 & 0 & 1
        \end{pmatrix} = \\
        = \begin{pmatrix}
            0 & 1 & 0 \\
            1 & 1 & 0 \\
            0 & 0 & -1
        \end{pmatrix}.
    \end{gather*}
    La matriz $M_B \lp q_{|F} \rp$ es, como se puede observar, la matriz de una cuádrica en $V$.
\end{example}
\begin{prop}
    Sea $Q= \left[ q \right] \subseteq \Po ^n$ una cuádrica y sea $V = \left[ F \right] \subseteq \Po ^n$ una variedad lineal. Se satisface que
    \begin{enumerate}[(1)]
        \item $q_{|F} = 0 \iff V \subseteq Q$.
        \item $q_{|F} \neq 0 \implies Q_{|V} = Q \cap V $.
    \end{enumerate}
\end{prop}

\begin{proof}
    \begin{enumerate}[(1)] \item[]
        \item $q_{|F} = 0 \iff \forall v \in F, \, q \lp v \rp = 0 \iff \forall p = \left[ v \right] \in V, \, p \in Q$.
        \item Sea $p=\left[ v \right]$. Entonces, 
        \[
            p \in Q_{|V} \iff \left\{ \begin{array}{c} q \lp v \rp = 0 \\ v \in F \end{array} \right\} \iff \left\{ \begin{array}{c} p \in Q \\ p \in V \end{array} \right\} \iff p \in Q\cap V.
        \]
    \end{enumerate}    
\end{proof}

\begin{prop}
     Sea $Q=\left[q \right]$ y sea $f \colon \Po^n \to \Po^n$ una homografía.
     \begin{enumerate}[(1)]
        \item $f\lp Q \rp$ es una cuádrica.
        \item Sea $\R$ una referencia en $\Po^n$ y sean $A=M_{\R} \lp Q \rp, \, P=M_{\R} \lp f \rp$. Entonces,
            \[
                M_{\R} \lp f \lp Q \rp \rp = \lp P^{-1} \rp ^t A \lp P^{-1} \rp.
            \]
     \end{enumerate}
\end{prop}

\begin{proof}
    \begin{enumerate}[(1)] \item[]
        \item Sea $p \in f \lp Q \rp$ y sea $y=p_{\R} = \lp y_0 , \dots , y_n \rp ^t$. Por ser $f$ una homografía, es biyectiva y $\lp f ^{-1} \lp p \rp \rp _{\R} = P^{-1} y$.
            \begin{gather*}
                p \in f \lp Q \rp \iff f^{-1} \lp p \rp \in Q \iff \lp P^{-1} y \rp ^t A \lp P^{-1} y \rp = 0 \iff \\
                \iff y^t \lp \lp P ^{-1} \rp ^t A P^{-1} \rp y = 0 ,
            \end{gather*}
            y como que $\lp \lp P ^{-1} \rp ^t A P^{-1} \rp$ es una matriz simétrica, $f \lp Q \rp$ es una cuádrica.
        \item $y^t \lp \lp P ^{-1} \rp ^t A P^{-1} \rp y = 0 \iff p \in f \lp Q \rp \iff y^t M_{\R} \lp f \lp Q \rp \rp y = 0$, de lo que concluimos que $M_{\R} \lp f \lp Q \rp \rp = \lp P^{-1} \rp ^t A \lp P^{-1} \rp$.
    \end{enumerate}
\end{proof}

\begin{defi}[(puntos singulares y lisos)]
    Sea $Q=\left[ q \right] \subseteq \Po ^n$, sea $p = \left[ v \right] \in Q$ y sea $\varphi$ la forma bilineal simétrica asociada a $q$. Entonces, 
    \begin{enumerate}[(1)] 
        \item Decimos que $p$ es un \textit{punto singular} de $Q \iff \varphi \lp v, \cdot \rp = 0$.
        \item Decimos que $p$ es un \textit{punto liso} de $Q$ si no es un \textit{punto singular}.
        \item Decimos que $Q$ es \textit{no degenerada} $\iff Q$ no tiene \textit{puntos singulares}. 
    \end{enumerate}
\end{defi}

\begin{obs}
    Sea $A=M_{\R} \lp Q \rp$, sea $p \in Q$ y sea $p_{\R} = \lp a_0, \dots , a_n \rp ^t$.
    \begin{enumerate}[(1)]
        \item $p\in Q$ es un punto singular $\iff \lp x_0, \dots , x_n \rp A \lp a_0, \dots , a_n \rp ^t = 0, \; \forall \, \lp x_0, \dots , x_n \rp \iff \lp a_0, \dots , a_n \rp \in \nuc A$.
        \item $Q$ es no degenerada $\iff \rg \lp A \rp = n+1 \iff \det A \neq 0$.
    \end{enumerate}
\end{obs}





