\chapter{Cónicas y cuádricas}

\section{Definiciones y propiedades básicas}

\begin{defi}
  \begin{enumerate}
    \item[]
    \item Una cuádrica $Q$ de $\Po^n_\k = \Po\lp \E\rp$ es la clase de equivalencia de una forma cuadrática  $q \colon \E \to \k, q\sim q' \iff \exists \lambda\neq0\tq q' = \lambda q$. Notación: $Q = [q]$.
    \item Si $n = 2$, las llamamos cónicas.
    \item $p = [u] \in \Po^n, p\in Q \iff q\lp u\rp=0$ ($p$ es un punto de $Q$).
    \item También notaremos por $Q$, el conjunto de todos los puntos de $Q$:
      \[ Q = \left\{ p=[u]\;|\;q\lp u\rp=0\right\} \]
  \end{enumerate}
\end{defi}
\begin{example}
  \begin{enumerate}
    \item[]
    \item A $\Po^2_\real$, sigui $Q \colon x_0^2 + x_1^2 - x_2^2$ una cònica.
      \begin{itemize}
        \item $p = \left[\lp1,0,1\rp^t\right] \in Q:$ verifica $x_0^2 + x_1^2 -x_2^2=0$.
        \item $p' = \left[\lp0,1,1\rp^t\right] \in Q$.
        \item $p'' = \left[\lp1,1,\sqrt{2}\rp^t\right] \in Q$.
      \end{itemize}
    \item A $\Po^2_\real$, sigui $Q \colon x_0^2 + x_1^2 - x_2^2 \implies$ puntos = $\emptyset$.
      \begin{itemize}
        \item $Q_1 \colon x_0^2 + 2x_1^2 = 0 \implies p = \lp0:0:1\rp$ punto único.
        \item $Q_1 \colon 3x_0^2 + 5x_1^2 = 0 \implies p = \lp0:0:1\rp$ punto único.
      \end{itemize}
  \end{enumerate}
\end{example}
\begin{obs}
  Sean $Q,\, q\colon \E \to \k$ (car$\k \neq 2$) asociada a $\varphi \colon \E \times \E \to \k, Q = [q]$, entonces $q \leftrightarrow \varphi$ de la siguiente forma:
  \begin{align*}
    q\lp u\rp &\rightarrow \varphi\lp u,v\rp = \frac{1}{2}\left[q\lp u+v\rp - q\lp u\rp -q\lp v\rp\right] \\
    q\lp u\rp = \varphi\lp u, u\rp &\leftarrow \varphi\lp u,v\rp
  \end{align*}
\end{obs}
\begin{defi}
  $\Po^n = \Po\lp\E\rp$, sea $q \colon \E \to \k$ asociada a $\varphi \colon \E \times \E \to \k, Q = [q]$. Sean $\R$ una referencia de $\Po^n$, $B$ una base de $\E$ adaptada a $\R$.
  \[ M_\R \lp Q\rp := M_B\lp q\rp \stackrel{\text{por def.}}{=} M_B\lp\varphi\rp \]
  $M_\R\lp Q\rp$ está definida salvo multiplicar por una constante.
\end{defi}
\begin{example}
  En $\Po^2_\real$, sean $\R$ (de $\Po^2$), $B$ (de $\real^3$). Sea $Q\colon x_0^2+ x_1^2+2x_2^2-2x_0x_1+4x_0x_2 + 6x_1x_2 = 0$.
  \begin{gather*}
    M_\R\lp Q\rp = M_B\lp q\rp = A =
    \begin{pmatrix}
      1 & -1 & 2 \\
      -1 & 1 & 3 \\
      2 & 3 & 2 \\
    \end{pmatrix}, \\
    \quad p\in Q \iff
    \begin{pmatrix}
      x_0 & x_1 & x_2 \\
    \end{pmatrix}
    \begin{pmatrix}
      1 & -1 & 2 \\
      -1 & 1 & 3 \\
      2 & 3 & 2 \\
    \end{pmatrix}
    \begin{pmatrix}
      x_0 \\
      x_1 \\
      x_2 \\
    \end{pmatrix}
    = 0.
  \end{gather*}
\end{example}
\begin{obs}
  Si $A = M_\R\lp Q\rp, p_\R = \lp x_0:\dots:x_n\rp = x$, entonces tenemos que $p\in Q \iff x^tAx = 0$.
\end{obs}
\begin{example}
  Sean $Q = [q], \R, A = M_\R\lp Q\rp$. Sea $\R'$ otro sistema de referencia en $\Po^n$ y sea $S_{\R, \R'}$ la matriz de cambio de sistema de referencia de $\R$ a $\R'$. Entonces $A' = M_{\R'}\lp Q\rp = S^tAS$.
\end{example}

\begin{defi}[(intersección de cónicas y variedades lineales)]
  Sea $Q= \left[ q \right] \subseteq \Po ^n$ una cuádrica y sea $V = \left[ F \right] \subseteq \Po ^n$ una variedad lineal. Definimos su intersección como
  \[
    Q_{|V} := \left[q_{|F} \right].
  \]
  Si $q_{|F} \neq 0$, entonces $Q_{|V}$ es una cuádrica en $V$.
\end{defi}

\begin{example}
  En $\Po _{\real} ^3$, sea $q=x_0^2+x_2^2+x_2^2-x_3^2+2x_0x_1-2x_1x_2=0$ y sea $V=\left[ F \right] \colon x_0+x_1=0$ (un plano). Tenemos que $F=\left[ \lp 1, -1, 0, 0 \rp ^t, \lp 0, 0, 1, 0 \rp ^t, \lp 0, 0, 0, 1 \rp ^t \right]$.
  \begin{gather*}
    A=M_B\lp q \rp = \begin{pmatrix}
      1 & 1 & 0 & 0 \\
      1 & 1 & -1 & 0 \\
      0 & -1 & 1 & 0 \\
      0 & 0 & 0 & -1
    \end{pmatrix} \implies \\
    \implies M_B \lp q_{|F} \rp = \begin{pmatrix}
      1 & -1 & 0 & 0 \\
      0 & 0 & 1 & 0 \\
      0 & 0 & 0 & 1
    \end{pmatrix} \begin{pmatrix}
      1 & 1 & 0 & 0 \\
      1 & 1 & -1 & 0 \\
      0 & -1 & 1 & 0 \\
      0 & 0 & 0 & -1
    \end{pmatrix} \begin{pmatrix}
      1 & 0 & 0 \\
      -1 & 0 & 0 \\
      0 & 1 & 0 \\
      0 & 0 & 1
    \end{pmatrix} = \\
    = \begin{pmatrix}
      0 & 1 & 0 \\
      1 & 1 & 0 \\
      0 & 0 & -1
    \end{pmatrix}.
  \end{gather*}
  La matriz $M_B \lp q_{|F} \rp$ es, como se puede observar, la matriz de una cuádrica en $V$.
\end{example}
\begin{prop}
  Sea $Q= \left[ q \right] \subseteq \Po ^n$ una cuádrica y sea $V = \left[ F \right] \subseteq \Po ^n$ una variedad lineal. Se satisface que
  \begin{enumerate}[(1)]
    \item $q_{|F} = 0 \iff V \subseteq Q$.
    \item $q_{|F} \neq 0 \implies Q_{|V} = Q \cap V $.
  \end{enumerate}
\end{prop}

\begin{proof}
\begin{enumerate}[(1)] \item[]
    \item $q_{|F} = 0 \iff \forall v \in F, \, q \lp v \rp = 0 \iff \forall p = \left[ v \right] \in V, \, p \in Q$.
    \item Sea $p=\left[ v \right]$. Entonces, 
      \[
        p \in Q_{|V} \iff \left\{ \begin{array}{c} q \lp v \rp = 0 \\ v \in F \end{array} \right\} \iff \left\{ \begin{array}{c} p \in Q \\ p \in V \end{array} \right\} \iff p \in Q\cap V.
      \]
  \end{enumerate}    
\end{proof}

\begin{prop}
  Sea $Q=\left[q \right]$ y sea $f \colon \Po^n \to \Po^n$ una homografía.
  \begin{enumerate}[(1)]
    \item $f\lp Q \rp$ es una cuádrica.
    \item Sea $\R$ una referencia en $\Po^n$ y sean $A=M_{\R} \lp Q \rp, \, P=M_{\R} \lp f \rp$. Entonces,
      \[
        M_{\R} \lp f \lp Q \rp \rp = \lp P^{-1} \rp ^t A \lp P^{-1} \rp.
      \]
  \end{enumerate}
\end{prop}

\begin{proof}
\begin{enumerate}[(1)] \item[]
    \item Sea $p \in f \lp Q \rp$ y sea $y=p_{\R} = \lp y_0 , \dots , y_n \rp ^t$. Por ser $f$ una homografía, es biyectiva y $\lp f ^{-1} \lp p \rp \rp _{\R} = P^{-1} y$.
      \begin{gather*}
        p \in f \lp Q \rp \iff f^{-1} \lp p \rp \in Q \iff \lp P^{-1} y \rp ^t A \lp P^{-1} y \rp = 0 \iff \\
        \iff y^t \lp \lp P ^{-1} \rp ^t A P^{-1} \rp y = 0 ,
      \end{gather*}
      y como que $\lp \lp P ^{-1} \rp ^t A P^{-1} \rp$ es una matriz simétrica, $f \lp Q \rp$ es una cuádrica.
    \item $y^t \lp \lp P ^{-1} \rp ^t A P^{-1} \rp y = 0 \iff p \in f \lp Q \rp \iff y^t M_{\R} \lp f \lp Q \rp \rp y = 0$, de lo que concluimos que $M_{\R} \lp f \lp Q \rp \rp = \lp P^{-1} \rp ^t A \lp P^{-1} \rp$.
  \end{enumerate}
\end{proof}

\begin{defi}[(puntos singulares y lisos)]
  Sea $Q=\left[ q \right] \subseteq \Po ^n$, sea $p = \left[ v \right] \in Q$ y sea $\varphi$ la forma bilineal simétrica asociada a $q$. Entonces, 
  \begin{enumerate}[(1)] 
    \item Decimos que $p$ es un \textit{punto singular} de $Q \iff \varphi \lp v, \cdot \rp = 0$.
    \item Decimos que $p$ es un \textit{punto liso} de $Q$ si no es un \textit{punto singular}.
    \item Decimos que $Q$ es \textit{no degenerada} $\iff Q$ no tiene \textit{puntos singulares}. 
  \end{enumerate}
\end{defi}

\begin{obs}
  Sea $A=M_{\R} \lp Q \rp$, sea $p \in Q$ y sea $p_{\R} = \lp a_0, \dots , a_n \rp ^t$.
  \begin{enumerate}[(1)]
    \item $p\in Q$ es un punto singular $\iff \lp x_0, \dots , x_n \rp A \lp a_0, \dots , a_n \rp ^t = 0, \; \forall \, \lp x_0, \dots , x_n \rp \iff \lp a_0, \dots , a_n \rp \in \nuc A$.
    \item $Q$ es no degenerada $\iff \rg \lp A \rp = n+1 \iff \det A \neq 0$.
  \end{enumerate}
\end{obs}



\section{Tangencia y polaridad}

\begin{example} 
  \begin{enumerate}
   \item []
    \item Cuádricas de $\Po^1$
      \begin{enumerate}
        \item $\Po^1_\real$\\
          Sea $q:$ $ax^{2}_{0} + cx^{2}_{1}+2bx_0x_1=0$ una cuádrica. Tenemos:
          \[
            Q \iff q \stackrel{\R}{\iff}A=
            \begin{pmatrix}
              a & b\\
              c & d\\
            \end{pmatrix}
          \]
          Y por lo tanto podemos considerar dos casos:
          \begin{enumerate}
            \item $a=0$ Tenemos entonces que:
              \[
                0=cx^{2}_{1}+2bx_0x_1 = x_1 \lp cx_1 +2bx_0 \rp 
              \]
              Lo que nos da dos casos:
              \begin{gather*}
                x_1=0 \implies p_1=\lp 1,0\rp \text{ y $p_2$ cualquiera}\\
                \text{o bien}\\
                p_2=\lp -c,2b\rp \text{ y $p_1$ cualquiera}\\
              \end{gather*}
              Notemos que si $a=0$ y $b=0$ tenemos que $p_1=p_2$.

            \item $a\neq 0 \implies \lp 1,0\rp$ no es solución 
              $\implies a\lp\frac{x_0}{x_1}\rp ^2 +2b\lp\frac{x_0}{x_1}\rp +c=0$,
              y si definimos $x=\frac{x_0}{x_1}$ tenemos $ax^2 +2bx+c=0 \implies
              x=\frac{-2b \pm \sqrt{-4\det A}}{2a}$\\
          \end{enumerate}
          Y por lo tanto, si $\det A = 0$, $q$ es un punto doble en $\Po^1_\real$,
          si $\det A>$ no hay ningun punto que pertenezca a la cuádrica y si 
          $\det A <0$ $q$ son dos puntos distintos.

        \item $\Po^1_\mathbb{C}$. Usando el mismo analisis que antes llegamos a que si
          $\det A=0$, la cuádrica es un punto doble y si $\det A \neq 0$ la cuádrica 
          son dos puntos distintos.
      \end{enumerate}
    \item $\Po^2_\real$
      \begin{center}
        \begin{tikzpicture}[line cap=round,line join=round,>=triangle 45,x=0.5cm,y=0.5cm]
          \clip(-8,-1) rectangle (3,7);
          \draw [rotate around={30.718904264029163:(-2.37,3.64)},color=blue] (-2.37,3.64) ellipse (3.3529125013086922 and 2.3332857179162856);
          \draw [color=red,domain=-12:14.92] plot(\x,{(-7.0616-2.04*\x)/10.8});
          \draw [color=red,domain=-12:14.92] plot(\x,{(--46.0782--0.6*\x)/10.54});
          \draw [color=red,domain=-12:14.92] plot(\x,{(--2512.038474626351--53.88340705491726*\x)/386.0081646973978});
          \begin{scriptsize}
            \draw[color=blue] (-2.42,1.84) node {$Q$};
            \draw[color=red] (-7.1,5.92) node {$L$};
            \draw[color=red] (-7.06,4.38) node {$L'$};
            \draw[color=red] (-6.96,1.3) node {$L''$};
          \end{scriptsize}
        \end{tikzpicture}
      \end{center}
  \end{enumerate}
\end{example}
\begin{obs}
  Sea $Q\subseteq \Po^n$ una cuádrica y $L\subseteq \Po^n$ una recta y $F$ un subespacio 
  vectorial de dimensión $2$ $\tq [F]=L$. Podemos estudiar las posiciones relativas
  de la cuádrica y la recta:
  \begin{itemize}
    \item $Q_{|L}=\left[ q_{|F} \equiv 0\right] \iff L\subseteq Q$ (ver ejemplo \ref{exampleLinQ})
    \item $L\not\subseteq Q \implies Q_{|L}$ es una cuádrica de $L \implies L\cap 
      Q = \emptyset$, $L\cap Q\equiv$ $1$ punto doble, $L\cap Q\equiv$ 2 puntos distintos.
  \end{itemize}
\end{obs}
\begin{example}
  Sea $Q\subseteq \Po^3 \tq Q:x^2_0 +x^2_1 -x^2_2-x^2_3=0$, sean $p_1=\lp 0:1:1:0\rp$ y $p_2=\lp 
  1:0:0:1\rp$ puntos del proyectivo y $L=p_1 \vee p_2$ una recta. Entonces tenemos que:
  \[
    Q_{|L}=
    \begin{pmatrix}
      0 & 1 & 1 & 0\\
      1 & 0 & 0 & 1\\
    \end{pmatrix}
    \begin{pmatrix}
      1 & & &\\
      & 1 & &\\
      & & -1 &\\
      & & & -1\\
    \end{pmatrix}
    \begin{pmatrix}
      1 & 0\\
      0 & 1\\
      0 & 1\\
      1 & 0\\
    \end{pmatrix}
    =\emptyset
  \]
  \label{exampleLinQ}
\end{example}

\begin{defi}
  Sean $L$ una recta y $Q$ una cuádrica:
  \begin{enumerate}[(1)]
    \item Si $L\subseteq Q$ diremos que $L$ es una generatriz de $Q$.
    \item Si $L\cap Q \equiv$ $2$ puntos diremos que $L$ es secante a $Q$.
    \item Si $L\cap Q \equiv$ $1$ punto (doble) o $L \subseteq Q$
    diremos que $L$ es tangente a $Q$.
  \end{enumerate}
\end{defi}

\begin{prop}
  Sean $\phi$ una forma bilineal y $q$ y $Q$ una cuádrica $\tq [\phi]=[q]=Q\subseteq \Po^2$.
  Entonces tenemos los siguientes resultados para cuádricas:
  \begin{enumerate}[(1)]
    \item Sea $L=p_1\vee p_2 \tq$ $p_1=[v_1]$ y $p_2=[v_2]$. Entonces:
      \begin{itemize}
        \item $L$ es generatriz de $Q \iff \phi \lp v_1,v_1\rp = \phi
          \lp v_2,v_2 \rp = \phi \lp v_1,v_2 \rp =0$.
        \item $L$ es tangente a $Q \iff \phi \lp v_1,v_1 \rp \phi \lp v_2,v_2 \rp
          - \phi \lp v_1,v_2 \rp ^2 =0$.
      \end{itemize}
    \item Si $p\in Q$ y $p$ singular, entonces toda recta $L \tq p\in L$ es tangente a $Q$.
    \item si $p\in Q$ y $p$ liso, entonces $\{ p'\subseteq \Po^n | L=p\vee p'$ es tangente a $Q \}$
      es el hiperplano $T_p\lp Q \rp$ (el hiperplano tangente a $Q$ en $p$) y tiene de ecuación
      \[
        \lp a_0,\dots,a_n \rp A 
        \begin{pmatrix}
          x_0\\
          \vdots\\
          x_n\\
        \end{pmatrix} = 0
      \]
      Donde $p_\R = \lp a_0:\dots :a_n\rp$, $p'_\R = \lp x_0:\dots :x_n\rp$ y $A = M_\R \lp Q \rp$.
  \end{enumerate}
\end{prop}
\begin{proof}
  \begin{enumerate}[(1)]
    \item[]
    \item Sea $F$ un subespacio vectorial $\tq L=[F]=[v_1,v_2]$. Entonces
      \[
        M_\R \lp Q_{|L} \rp = M_\R \lp q_{|F}\rp =
        \begin{pmatrix}
          \phi\lp v_1,v_1\rp & \phi\lp v_1,v_2\rp\\
          \phi\lp v_2,v_1\rp & \phi\lp v_2,v_2\rp\\
        \end{pmatrix} =B
      \]
      Y por lo tanto:
      \begin{itemize}
        \item $L$ generatriz $\stackrel{\text{por definición}}{\iff} q_{|F}\equiv 0
          \iff \phi\lp v_1,v_1\rp = \phi\lp v_1,v_2\rp = \phi\lp v_2,v_2\rp=0$.
        \item $L$ es tangente a $Q \iff \left\{
            \begin{array}{c}
              L \text{ generatriz }  \iff B=0 \\
              L\cap Q \equiv 1 \text{ punto doble }  \iff \det B=0\\
            \end{array}\right\} \iff \phi\lp v_1,v_1\rp \phi\lp v_2,v_2\rp - \phi\lp v_1,v_2\rp ^2 =0$
      \end{itemize}
    \item Recordemos que $p=[v]$ singular $\iff \phi\lp v,·\rp=0$. Ahora sea $L=p\vee p'$.
      Entonces tenemos:
      \[
        B=M_\R\lp Q_{|F}\rp = 
        \begin{pmatrix}
          \cancelto{0}{\phi\lp v_1,v_1\rp} & \cancelto{0}{\phi\lp v_1,v_2\rp}\\
          \cancelto{0}{\phi\lp v_2,v_1\rp} & \phi\lp v_2,v_2\rp
        \end{pmatrix} \implies \det B=0 \stackrel{(1)}{\implies} L \text{ es tangente a } Q
      \]
    \item Recordemos que $p=[v]$ es punto liso de $q \iff \phi\lp v,v\rp = q\lp v\rp =0$.
      Sea $L=p\vee p' = [v]\vee [v']$ y sea :
      \[
        B=M_\R \lp q_{|F}\rp = 
        \begin{pmatrix}
          \phi\lp v,v\rp & \phi\lp v,v'\rp\\
          \phi\lp v',v\rp & \phi\lp v',v'\rp\\
        \end{pmatrix}
      \]
      Entonces tenemos que $L$ es tangente a $Q \iff \det B =0 \iff \phi\lp v,v\rp\phi\lp v',v'\rp
      - \phi\lp v,v'\rp^2 \iff \phi \lp v,v'\rp =0$. Pero además, como
      \[
        \phi\lp v,v'\rp =0\implies \lp a_0, \dots ,a_n \rp A 
        \begin{pmatrix}
          x_0\\ \vdots\\ x_n\\
        \end{pmatrix}=0
      \]
      Y por lo tanto, como $\lp a_0,\dots,a_n\rp A \neq 0$ porque $A$ es simétrica y $A\begin{pmatrix}
        a_0\\ \vdots\\ a_n\\ \end{pmatrix} \neq 0$ (ya que p no es singular), el conjunto de puntos $p' \tq$
      $L$ es tangente a $Q$ cumplen la ecuación anterior y son un hiperplano.
  \end{enumerate}
\end{proof}

\begin{example}
  \begin{enumerate}
    \item []
    \item Sea $Q\subseteq\Po^3$ una cuádrica $\tq Q:x^2_0 +x^2_1 -x^2_2 -x^2_3=0$, $A$ la matriz asociada y $p=\lp 1:0:0:1\rp$
      un punto de $Q$. Como $p\not\in \nuc \lp A\rp$, $p$ es un punto liso y por lo tanto, aplicando el 
      apartado $3$ de la proposición anterior temeos que $T_p\lp Q \rp$ tiene como ecuación:
      \[
        \lp 1,0,0,1 \rp A \begin{pmatrix}x_0\\x_1\\x_2\\x_3\\ \end{pmatrix} = 0 \implies 
        \lp 1,0,0,-1 \rp \begin{pmatrix}x_0\\x_1\\x_2\\x_3\\ \end{pmatrix} = 0 \implies
        x_0-x_3=0
      \]
    \item Sea $Q:-x^2_0 +x^2_1 +x^2_2 +x^2_3=0$ y $p=\lp 1:0:0:1\rp \in Q$, igual que en el
      ejemplo anterior tenemos que $p$ es liso, y siguiendo el prodecimiento del ejemplo anterior llegamos 
      a que $T_p\lp Q \rp$ tiene como ecuación $-x_0 +x_3=0$.
  \end{enumerate}
\end{example}

\subsection{Polaridad}
\begin{defi}[de una cuádrica no degenerada]
  Sea $Q=[q]=[\phi]$ una cuádrica y sea $A=M_\R\lp Q\rp$ su matriz. Entonces diremos que $Q$ es no
  degenerada $\iff \det A \neq 0$. Es decir, las cuádricas no degeneradas son aquellas que no tienen
  puntos singulares.
\end{defi}

\begin{defi}[de dos puntos polares respecto a una cuádrica]
  Sea $Q$ una cuádrica no degenerada $\tq Q=[q]=[\phi]$ y $A=M_\R\lp Q\rp$. Sean $p_1=[v_1]$ y 
  $p_2=[v_2]$ dos puntos $\in \Po^n$. Entonces diremos que $p_1$ y $p_2$ son polares respecto a $Q\iff
  \phi\lp v_1,v_2\rp=0$ y lo notaremos con $p_1 \stackrel{Q}{\sim} p_2$. 
\end{defi}

\begin{prop}
  \label{propietats_polaritat}
  Sea $Q \subseteq \Po^n$ una cuádrica y $p_1, p_2\in\Po^n$ dos puntos. Entonces:
  \begin{enumerate}[(1)]
    \item $p_1 \stackrel{Q}{\sim} p_1 \iff p\in Q$.
    \item Si $p_1,p_2\in Q$, entonces $p_1 \stackrel{Q}{\sim} p_2 \iff L=p_1\vee p_2 \subseteq Q$.
    \item Si $p_1\in Q$, $p_2 \not\in Q$, entonces $p_1 \stackrel{Q}{\sim} p_2 \iff L=p_1\vee p_2 
      \subseteq T_{p_1}\lp Q \rp$.
    \item Si $p_1,p_2\not\in Q$, $ L=p_1\vee p_2$ y $ L\cap Q \neq \varnothing$, entonces
      $p_1 \stackrel{Q}{\sim} p_2 \iff L$ es secante a $Q$ y $\lp p_1,p_2,p_3,p_4\rp =-1$, donde
      $p_3, p_4$ son los puntos de $L \cap Q$. 
    \item Fijado $p=[v]$, entonces $\left\{p'\in\Po^n | p' \stackrel{Q}{\sim} p\right\} = H_p\lp Q\rp$
      es un hiperplano (hiperplano polar a $p$) de ecuación $vA\begin{pmatrix}x_0\\ \vdots\\ x_n\\ \end{pmatrix}=0$.
  \end{enumerate}
\end{prop}
\begin{proof}
  Sea $B$ una matriz $\tq B=M_\R \lp Q\cap L\rp = \begin{pmatrix}
    \phi\lp v_1,v_1\rp & \phi\lp v_1,v_2\rp\\
    \phi\lp v_2,v_1\rp & \phi\lp v_2,v_2\rp\\
  \end{pmatrix}$. Entonces:
  \begin{enumerate}[(1)]
    \item Por definición, $p_1 \stackrel{Q}{\sim} p_1 \iff \phi\lp v,v\rp=0 \iff q\lp v\rp=0 \iff p\in Q$.
    \item Por definición, $p_1 \stackrel{Q}{\sim} p_2 \iff \phi\lp v_1,v_2\rp=\phi\lp v_2,v_1\rp=0$, y como
      $p_1,p_2\in Q \implies \phi\lp v_1,v_1\rp=\phi\lp v_2,v_2\rp=0$, entonces $\phi\lp v_1,v_2\rp=
      \phi\lp v_2,v_1\rp=0 \iff B=\varnothing \iff L\subseteq Q$.
    \item Por definición, $p_1 \stackrel{Q}{\sim} p_2 \iff \phi\lp v_1,v_2\rp=\phi\lp v_2,v_1\rp=0$, y como 
      $p_1\in Q \implies \phi\lp v_1,v_1\rp=0$, entonces $\phi\lp v_1,v_2\rp=
      \phi\lp v_2,v_1\rp=0 \iff B= 
      \begin{pmatrix}
        0 & 0\\
        0 & c\\
      \end{pmatrix}$ con $c\neq 0$ porque $p_2 \not\in Q$ y por lo tanto $\det B \neq 0$ lo que nos dice que 
      $B= 
      \begin{pmatrix}
        0 & 0\\
        0 & c\\
      \end{pmatrix} \iff L\subseteq T_{p_1}\lp Q\rp$.
    \item Por definición, equivalentemente a el prodecimiento de los apartados anteriores llegamos a que $B=
      \begin{pmatrix}
        a & 0\\
        0 & b\\
      \end{pmatrix}$ con $a,b\neq 0$. Cojamos una referencia $\R \tq p_1=\lp 1:0\rp$ y $p_2(0:1)$ y consideremos la ecuación dada por $L\cap Q$:
      \[
        \lp \alpha,\beta\rp
        \begin{pmatrix}
          a & 0\\
          0 & b\\
        \end{pmatrix}
        \begin{pmatrix}
          \alpha\\
          \beta\\
        \end{pmatrix}=0 \implies \alpha^2 a+\beta^2 b=0 
      \]
      que como $p_1\not\in Q \implies \beta\neq 0$ y por lo tanto si $z=\frac{\alpha}{\beta}$ tenemos:
      \[
        a\lp\frac{\alpha}{\beta}\rp^2 +b =0 \implies az^2 +b=0 \implies z=\pm \sqrt{\frac{-b}{a}}
      \]
      que tiene dos soluciones porque tenemos como hipótesis que $L\cap Q\neq\varnothing$ y por lo tanto
      $L$ es secante a $Q$. Además tenemos que $p_3=\lp\sqrt{\frac{-b}{a}}:1\rp$ y $p_4=\lp-\sqrt{\frac{-b}{a}}:1\rp$.
      Finalmente es fácil comprobar que $(p_1,p_2,p_3,p_4)=-1$.
    \item Por definición, $p \stackrel{Q}{\sim} p' \iff \phi\lp v,v'\rp=0 \iff vA\begin{pmatrix}x_0\\ \vdots\\ x_n\\ \end{pmatrix}=0$
      y como $vA\neq 0$, $H_p\lp Q\rp$ es un hiperplano.
  \end{enumerate}
\end{proof}

\begin{obs}
  \begin{enumerate}[(1)]
    \item []
    \item Si $p\in Q$ entonces $H_p\lp Q\rp = T_p\lp Q\rp$.
    \item Como $p_1 \stackrel{Q}{\sim} p_2 \iff p_2 \stackrel{Q}{\sim} p_1$, tenemos que $p_1\in H_{p_2}\lp Q\rp \iff
      p_2\in H_{p_1}\lp Q\rp$.
  \end{enumerate}
\end{obs}

\begin{example}
  En $\Po^2$:\\
  \begin{center}
    \begin{tikzpicture}[line cap=round,line join=round,>=triangle 45,x=0.8cm,y=0.8cm]
      \clip(-8,2) rectangle (2.4,6);
      \draw [rotate around={-172.80274371739344:(-2.5948300768279142,4.470680337271209)},color=blue] (-2.5948300768279142,4.470680337271209) ellipse (3.973436165338986 and 1.330253566678655);
      \draw [color=red,domain=-10.595414382769706:2.500720072572623] plot(\x,{(--2998.1976338746463--154.82772036198662*\x)/837.4336989206226});
      \begin{scriptsize}
        \draw [fill=teal] (-1.7658998659356222,3.2537350559886766) circle (2.5pt);
        \draw[color=teal] (-1.697846349314976,3.4339125358867024) node {$p$};
        \draw[color=red] (-7.219597340524728,2.594199251509183) node {$H_p = T_p$};
      \end{scriptsize}
    \end{tikzpicture}
    \begin{tikzpicture}[line cap=round,line join=round,>=triangle 45,x=0.8cm,y=0.8cm]
      \clip(1,-1) rectangle (12,4.2);
      \draw [rotate around={-172.52238947185185:(5.58430056014799,2.319535352849599)},color=blue] (5.58430056014799,2.319535352849599) ellipse (4.063981256849501 and 1.4858815602390578);
      \draw [color=teal,domain=-3.7981066867017312:19.40249436513903] plot(\x,{(--2.4696406066332823--1.0398473193995683*\x)/6.128734815311686});
      \draw [color=green,domain=-3.7981066867017312:19.40249436513903] plot(\x,{(--2497.140860927716-353.76546246339603*\x)/1639.0668888171813});
      \draw [color=red,domain=-3.7981066867017312:19.40249436513903] plot(\x,{(-5251.952072960202--824.3765555968888*\x)/1134.6581928847716});
      \begin{scriptsize}
        \draw [fill=green] (2.906744285068728,0.8961410511547316) circle (2.5pt);
        \draw[color=green] (2.8585424492862592,1.3578812922614554) node {$p_1$};
        \draw [fill=red] (9.035479100380414,1.9359883705543) circle (2.5pt);
        \draw[color=red] (9.259744552967712,1.7) node {$p_2$};
        \draw [fill=teal] (6.528374407097482,0.1144721249308355) circle (2.5pt);
        \draw[color=teal] (6.600090157776123,-0.19) node {$p$};
        \draw[color=red] (10,3.8) node {$H_{p_2}\left(Q\right) = T_{p_2}\left(Q\right)$};
        \draw[color=green] (10.37,-0.1) node {$H_{p_1}\left(Q\right) = T_{p_1}\left(Q\right)$};
      \end{scriptsize}
    \end{tikzpicture}
    \begin{tikzpicture}[line cap=round,line join=round,>=triangle 45,x=0.8cm,y=0.8cm]
      \clip(-8,-1.5) rectangle (3.2,4.5);
      \draw [rotate around={-166.64953823051044:(-3.6958343717826576,2.4001698384783072)},color=blue] (-3.6958343717826576,2.4001698384783072) ellipse (3.2278434115319956 and 1.7272612187217231);
      \draw [color=green,domain=-11.88617290832004:5.544782051740591] plot(\x,{(--5.825787876122232--0.4045348527412902*\x)/3.2024818479860375});
      \draw [color=green,domain=-11.88617290832004:5.544782051740591] plot(\x,{(-3111.699543848952-628.0428125467149*\x)/878.5469677946085});
      \draw [color=green,domain=-11.88617290832004:5.544782051740591] plot(\x,{(--2900.6520266166162--874.2194962583005*\x)/1070.2996041172978});
      \draw [color=orange,domain=-11.88617290832004:5.544782051740591] plot(\x,{(--5.744458157149122--0.7631084376391317*\x)/2.343724533660228});
      \draw [color=orange,domain=-11.88617290832004:5.544782051740591] plot(\x,{(-779.0328966364686-301.45934984121845*\x)/1347.5422207319493});
      \draw [color=orange,domain=-11.88617290832004:5.544782051740591] plot(\x,{(--2159.394431179412--950.1005851563904*\x)/644.6207560726057});
      \draw [color=red,domain=-11.88617290832004:5.544782051740591] plot(\x,{(--1.1967420881381787--0.5634245726491658*\x)/1.767992977268901});
      \begin{scriptsize}
        \draw [fill=blue] (-3.1706954282897257,1.4186282026262316) circle (2.5pt);
        \draw[color=blue] (-3.0803795994811205,1.6613519925493554) node {$p$};
        \draw [fill=green] (-6.373177276275763,1.0140933498849414) circle (2.5pt);
        \draw [fill=green] (-1.2903827454426602,1.6561477422737316) circle (2pt);
        \draw [fill=orange] (-5.514419961949954,0.6555197649870999) circle (2.5pt);
        \draw [fill=orange] (-0.7827893373810072,2.1961219993978656) circle (2pt);
        \draw [fill=red] (-4.081835419330292,-0.6239076188674535) circle (2pt);
        \draw[color=red] (-4.028695801971466,-0.9) node {$A_1$};
        \draw [fill=red] (-2.3138424420613912,-0.06048304621828766) circle (2pt);
        \draw[color=red] (-2.165931832794002,-0.3) node {$A_2$};
        \draw[color=green] (1.2,1.65) node {$l = H_{A_1}\left(Q\right)$};
        \draw[color=orange] (1.9,2.55) node {$l' = H_{A_2}\left(Q\right)$};
        \draw[color=red] (1.2,0.6) node {$H_p\left(Q\right)$};
      \end{scriptsize}
    \end{tikzpicture}
  \end{center}
\end{example}

\subsection{Cuádrica dual}
\begin{defi}[de la cuádrica dual]
  Sea $Q\subseteq \Po^n$ una cuádrica, $A=M_\R\lp Q\rp$ su matriz asociada y sea
  \begin{align*}
    F_Q (=F):\Po^n & \to \lp\Po^n \rp ^*\\
    \begin{pmatrix}
      a_0\\
      \vdots\\
      a_n\\
    \end{pmatrix} &\mapsto A
    \begin{pmatrix}
      a_0\\
      \vdots\\
      a_n\\
    \end{pmatrix}
  \end{align*}
  una proyectividad $\tq$ si $p\in\Po^n$, $F(p)=\lp H_p\lp Q\rp\rp^*$. Entonces definimos 
  $Q^* := F\lp Q\rp$ la cuádrica dual de $Q$.
\end{defi}

\begin{prop}
  Sea $Q\in\Po^n$ una cuádrica, $A=M_\R\lp Q\rp$ su matriz asociada y $Q^*=F(Q)$ su cuádrica dual.
  Entonces $M_{\R^*}\lp Q^* \rp = A^{-1}$
\end{prop}
\begin{proof}
  Sea $\tilde{p}_{\R^*}= \lp b_0:\dots:b_n\rp \in \lp \Po^n \rp ^* \tq \tilde{p}\in Q^*=F(Q)
  \stackrel{F \text{ biyectiva}}{\iff} A^{-1}
  \begin{pmatrix}
    b_0\\
    \vdots\\
    b_n\\
  \end{pmatrix}=F^{-1}\lp\tilde{p}\rp \in Q$.
  Entonces por definición tenemos que 
  \[
    \lp A^{-1}
    \begin{pmatrix}
      b_0\\
      \vdots\\
      b_n\\
    \end{pmatrix}\rp ^T A \lp A^{-1}
    \begin{pmatrix}
      b_0\\
      \vdots\\
      b_n\\
    \end{pmatrix}\rp=0 \iff (b_0,\dots,b_n)\lp A^{-1}\rp ^T 
    \begin{pmatrix}
      b_0\\
      \vdots\\
      b_n\\
    \end{pmatrix}=0 \iff 
  \]
  \[
    \iff (b_0,\dots,b_n) A^{-1}
    \begin{pmatrix}
      b_0\\
      \vdots\\
      b_n\\
    \end{pmatrix}=0
  \]



\end{proof}

\section{Cuádricas proyectivas y afines}

\begin{align*}
  \A^n &\leftrightarrow \overline{\A^n} = \Po^n \\
  \left( x_0, \dots, x_n \right) &\mapsto \left( 1:x_1:\dots:x_n \right) \\
  \left( \frac{\overline{x}_1}{\overline{x}_0}, \dots, \frac{\overline{x}_n}{\overline{x}_0} \right) &\mapsfrom \left( \overline{x}_0:\dots:\overline{x}_n \right), \overline{x}_0 \neq 0.
\end{align*}
\begin{obs}
  $Q$ proyectiva $\implies Q_0$ afín: $M_{\overline{\R}}\left( Q \right) = A$.
  \[ Q_0 = Q\cap \A^n = \left\{ p \in \A^n \;|\; p\in Q \right\} \to \text{ Ecuaciones: } \left( 1,x_1,\dots,x_n \right)A
    \begin{pmatrix}
      1 \\
      x_1 \\
      \vdots \\
      x_n
    \end{pmatrix} = 0.
  \]
\end{obs}
\begin{example}
  $\A^2 \to \Po^2$ \\
  \begin{gather*}
    \begin{array}{c}
      Q_0 = Q\cap \A^2 \\
      x_0 = 1
    \end{array}, \quad
    \begin{cases}
      Q \colon x_0^2 + x_1^2 + x_2^2 + 4x_0x_1 + 6x_0x_2 + 8x_1x_2 = 0 \\
      A = 
      \begin{pmatrix}
        1 & 2 & 3 \\
        2 & 1 & 4 \\
        3 & 4 & 1
      \end{pmatrix}
    \end{cases} \\
    1 + x_0^2 + x_1^2 + x_2^2 + 4x_0x_1 + 6x_0x_2 + 8x_1x_2 = 0 \implies A = 
    \begin{pmatrix}
      1 & 2 & 3 \\
      2 & 1 & 4 \\
      3 & 4 & 1
    \end{pmatrix}
  \end{gather*}
\end{example}
\begin{defi}
  Dada una cuádrica afín $Q$ que cumple
  \begin{gather*}
    A = M_\R\left( Q \right) =
    \left( \begin{array}{c|ccc}
      c & b_1 & \dots & b_n \\ \hline
      b_1 & & & \\
      \vdots & & A_\infty & \\
      b_n & & & 
    \end{array} \right)
  \end{gather*}
  Llamamos completación proyectiva de $Q$ a la cuádrica proyectiva $\overline{Q}$
  que cumple $A = M_{\overline{\R}}(\overline{Q})$.
  
\end{defi}
\begin{obs}
  Dada una cuádrica afín $Q$, su completación proyectiva $\overline{Q}$
  es tal que $\overline{Q} \cap \A^n = Q$.
\end{obs}

\begin{example}
  $Q \subseteq \A^2$ \\
  \begin{gather*}
    Q \colon 1+2x+4y+x^2+-5y^2+4xy = 0 \\ \lp
    \begin{array}{c|cc}
      1 & 1 & 2 \\ \hline
      1 & 1 & 2 \\
      2 & 2 & -5
    \end{array}\rp \\
    \overline{Q} \colon M_{\overline{\R}}\left( Q \right) = A, \, x_0^2+x_1^2-5x_2^2 + 2x_0x_1 + 4x_0x_2 + 4x_1x_2 = 0.
  \end{gather*}
\end{example}
\begin{obs}
  $Q \subseteq \A^n$,
  \[
    A = M_\R\left( Q \right) =
    \left( \begin{array}{c|ccc}
      c & b_1 & \dots & b_n \\
      b_1 & & & \\
      \vdots & & A_\infty & \\
      b_n & & &
    \end{array} \right)
    \stackrel{\overline{x}_0 = 0}{\implies} \overline{Q} \cap \overline{\A}^n_\infty \colon A_\infty
  \]
\end{obs}
\begin{example}
  $\Po^2, Q \colon x_0^2 + x_1^2 - x_2^2 = 0$.
  % TODO Dibuix en 2D facil pero que no compila amb geogebra
  \begin{enumerate}
    \item Sea $L_1 \colon x_0 = 0$, $Q\cap \left( \A^2 = \Po^2 \setminus L_1 \right) = C_1$, tenemos que
      \begin{gather*}
        x_0 = 1 \implies 1+x_1^2 - x_2^2 = 0 \implies C_1 \text{ hipérbola}, \\
        Q \cap L_1 \colon x_1^2 - x_2^2 = 0 \\
        \left( x_1 -x_2 \right)\left( x_1 + x_2 \right) = 0, p_1 = \left( 0,1,1 \right)^t, p_2 = \left( 0,-1,1 \right)^t.
      \end{gather*}
    \item Sea $L_2 \colon x_2 = 0$, $Q\cap \left( \A^2 = \Po^2 \setminus L_2 \right) = C_2$, tenemos que
      \begin{gather*}
        x_0 = 1 \implies x_0^2+x_1^2 - 1 = 0 \implies C_2 \text{ elipse}, \\
        \emptyset = Q \cap L_1 \colon x_0^2 - x_1^2 = 0 \\
        \left( x_1 -x_2 \right)\left( x_1 + x_2 \right) = 0, p_1 = \left( 0,1,1 \right)^t, p_2 = \left( 0,-1,1 \right)^t.
      \end{gather*}
    \item $p_1 = \left( 0,1,1 \right), H_{p_1}Q = T_{p_1}Q \colon
      \begin{pmatrix}
        0 & 1 & 1
      \end{pmatrix}
      \begin{pmatrix}
        1 & 0 & 0 \\
        0 & 1 & 0 \\
        0 & 0 & -1
      \end{pmatrix}
      \begin{pmatrix}
        x_0 \\ x_1 \\ x_2
      \end{pmatrix}
      = 0 \to x_1 - x_2 = 0$. \\
      Sea $L_3 = T_{p_1}Q \colon x_1 - x_2 = 0$, $Q\cap \left( \A^2 = \Po^2 \setminus L_3 \right) = C_3$,
      \begin{gather*}
        \begin{rcases}
          \overline{x}_0 = x_0 \\
          \overline{x}_1 = x_1 - x_2 \\
          \overline{x}_2 = x_1 + x_2
        \end{rcases}
        \implies
        \begin{pmatrix}
          \overline{x}_0 \\
          \overline{x}_1 \\
          \overline{x}_2
        \end{pmatrix} =
        \overbrace{ \begin{pmatrix}
          1 & 0 & 0 \\
          0 & 1 & -1 \\
          0 & 1 & 1
        \end{pmatrix}}^{\inv{S}}
        \begin{pmatrix}
          x_0 \\ x_1 \\ x_2
        \end{pmatrix}
      \end{gather*}
      $Q$ en $\overline{\R} \colon 0 = x_0^2 + x_1^2 - x_2^2 = x_0^2 + \left( x_1 - x_2 \right) \left( x_1 + x_2 \right) \implies 0 = \overline{x}_0^2 + \overline{x}_1\overline{x}_2 \stackrel{\overline{x}_1 = 1}{\implies} 0 = \overline{x}_0^2 + \overline{x}_2$ \\
      $L_3$ en $\overline{\R} \colon \overline{x}_1 = 0$
  \end{enumerate}
\end{example}
\begin{defi}
  Sea $Q$ una cuádrica afín y $\overline{Q}$ su completación proyectiva 
  $\left( \overline{\A^n} = \Po^n \right)$. Sea $p=[v]\in\A^n$, $p$ es un 
  centro de $Q$ $\iff \varphi\left( v, \omega \right) = 0, \, 
  \forall [\omega] \in\overline{\A^n}_\infty$. Si $Q$ no es degenerada,
  \[ p \text{ es un centro de }Q \iff H_p\left( \overline{Q} \right) = 
  \overline{\A^n}_\infty \]
\end{defi}
% TODO dibuixos de la pag 10, son en 2D facilets
\begin{obs}[Cálculo del centro para $Q$ no degenerada $\left( \det A \neq 0 \right)$]
  \begin{align*}
    \A^n &\to \Po^n = \overline{A}^n \\
    Q &\mapsto \overline{Q}
  \end{align*}
  $A = M_\R \left( Q \right) = M_{\overline{\R}} \left( \overline{Q} \right) =
  \left(\begin{array}{c|ccc}
    c & b_1 & \dots & b_n \\
    \hline
    b_1 & & & \\
    \vdots & & A_\infty & \\
    b_n & & &
  \end{array}\right)$ \\
  Supongamos que queremos encontrar un centro de $Q$, $p = 
  \left( 1, a_1, \dots, a_n \right)\in \A^n$. Sabemos que $\overline{\A^n_\infty} = 
  H_p(\overline{Q}).$ Por lo tanto, tiene que cumplir
  \[\begin{pmatrix}
      1 & a_1 & \dots & a_n 
    \end{pmatrix}
    A
    \begin{pmatrix}
      0 \\ x_1 \\ \vdots \\ x_n 
    \end{pmatrix}
    = 0 \qquad \forall
    \begin{pmatrix} 0 \\ x_1 \\ \vdots \\ x_n \end{pmatrix} \in \overline{\A^n_\infty}.
  \]
  Esto se satisface si y solo si, para $\alpha \neq 0$,
  \begin{gather*}
    \begin{pmatrix}
      1 & a_1 & \dots & a_n 
    \end{pmatrix}
    A = 
    \begin{pmatrix} 
      \alpha & 0 & \dots & 0 
    \end{pmatrix} 
    \iff \\
    \iff
    \begin{pmatrix}
      1 & a_1 & \dots & a_n 
    \end{pmatrix}
    \left(\begin{array}{c|ccc}
      c & b_1 & \dots & b_n \\
      \hline
      b_1 & & & \\
      \vdots & & A_\infty & \\
      b_n & & &
    \end{array}\right)
    = 
    \begin{pmatrix} 
      \alpha & 0 & \dots & 0 
    \end{pmatrix}
  \end{gather*}
  Que, finalmente, podemos desarrollar para obtener la siguiente condición para
  calcular los centros de una cuádrica:
  \[
    \begin{pmatrix}
      b_1 \\ \vdots \\ b_n 
    \end{pmatrix}
    +
    A_\infty
    \begin{pmatrix}
      a_1 \\ \vdots \\ a_n 
    \end{pmatrix}
    =
    \begin{pmatrix}
      0 \\ \vdots \\ 0 
    \end{pmatrix}
  \]
\end{obs}
\begin{obs}
  $Q$ no tiene centro $\iff \overline{\A^n_\infty}$ es tangente a $\overline{Q}$
\end{obs}
\begin{obs}
  $p \in \A^n$ es un centro de $Q \iff p$ es un centro de simetría (es decir, $\forall L\ni p$ (recta), si $L\cap Q\neq \emptyset$, entonces, $L\cap Q = \left\{ p_1, p_2 \right\}$ y $p = \frac{1}{2}\left( p_1 + p_2 \right)$).
\end{obs}
\begin{proof}
  Sea $p'$ el punto de corte entre la recta $L$ completada y el hiperplano
  del infinito. 
  \begin{itemize}
    \item $p \stackrel{\overline{Q}}{\sim} p'$
    \item $p \notin \overline{Q}$, de lo contrario $\overline{\A^n_\infty}
    = H_p\left( \overline{Q} 
    \right) = T_p \left( \overline{Q} \right)$, cosa que contradice el hecho
    de que $p$ es del afín.
    \item $p' \notin \overline{Q}$, de lo contrario, como $p \not \in \overline{Q}$
    pero $p \stackrel{\overline{Q}}{\sim} p'$, \ref{propietats_polaritat} nos 
    dice que $L$ tiene un único punto de corte con $\overline{Q}$, es decir, 
    $L \cap Q = \varnothing$.
  \end{itemize}
  Por tanto, como hemos visto en \ref{propietats_polaritat}
  \begin{gather*}
    \begin{cases}
      L\cap\overline{Q} = \left\{ p_1, p_2 \right\} \\
      \left( p_1, p_2, p, p' \right) = -1 = \left( p_1, p_2, p \right)
    \end{cases}
    \iff p = \frac{1}{2}\left( p_1 + p_2 \right).
  \end{gather*}
\end{proof}
\begin{defi}
  Sea $Q$ una cuádrica con centro $p\in\A^n$. Una asíntota de $Q$ es una recta $L \ni p$ t.q. $\overline{L}$ es tangente a $\overline{Q}$ en un punto de $\overline{\A^n_\infty}$.
\end{defi}
\begin{example}
  $\A^2 \to Q \colon 2xy + 4x +4y +2 = 0$,
  $A = \left( \begin{array}{c|cc}
    2 & 2 & 2 \\ \hline
    2 & 0 & 1 \\
    2 & 1 & 0
  \end{array} \right), \det A \neq 0$ (no degenerada). \\
  $\Po^2 \to Q \colon 2x_1x_2 + 4x_1x_0 + 4x_2x_0 + 2x_0^2 = 0,
  A = \begin{pmatrix}
    2 & 2 & 2 \\
    2 & 0 & 1 \\
    2 & 1 & 0
  \end{pmatrix}$,
  \begin{gather*}
    \begin{rcases}
      \overline{Q} \cap r_\infty \\
      r_\infty \colon x_0 = 0
    \end{rcases}
    \implies x_1x_2 = 0 \implies
    \begin{cases}
      p_1 = \left( 0,0,1 \right) \\
      p_2 = \left( 0,1,0 \right)
    \end{cases}
  \end{gather*}
  Asíntotas:
  \begin{itemize}
    \item $\overline{r}_1 = T_{p_1}\left( \overline{Q} \right)$
      \[
        \begin{pmatrix}
          0 & 0 & 1
        \end{pmatrix}
        \begin{pmatrix}
          2 & 2 & 2 \\
          2 & 0 & 1 \\
          2 & 1 & 0
        \end{pmatrix}
        \begin{pmatrix}
          x_0 \\ x_1 \\ x_2
        \end{pmatrix}
        = 0 \implies 2x_0 + x_1 = 0 \;\substack{\text{afín} \\ \implies \\ x_0 = 1}\; r_1 \colon 2 + x = 0
      \]
    \item $\overline{r}_2 = T_{p_2}\left( \overline{Q} \right)$
      \[
        \begin{pmatrix}
          0 & 1 & 0
        \end{pmatrix}
        \begin{pmatrix}
          2 & 2 & 2 \\
          2 & 0 & 1 \\
          2 & 1 & 0
        \end{pmatrix}
        \begin{pmatrix}
          x_0 \\ x_1 \\ x_2
        \end{pmatrix}
        = 0 \implies 2x_0 + x_2 = 0 \; \substack{\text{afín} \\ \implies \\ x_0 = 1}\; r_2 \colon 2 + y = 0
      \]
  \end{itemize}
\end{example}

\section{Clasificación proyectiva de cuádricas}

\begin{defi}
  $\Po^n = \Po\left( \E \right)$. Sean $Q_1 = [q_1], Q_2 = [q_2], q_1,q_2 \colon \E \to \k$ (formas cuadráticas).
  \[
    Q_1 \sim Q_2 \iff \exists \lambda \neq 0 \tq q_1\sim \lambda q_2 \text{ (como formas cuadráticas)}
  \]
\end{defi}
\begin{obs}
  Son equivalentes:
  \begin{itemize}
    \item $Q_1 \sim Q_2$
    \item $\exists \lambda \neq 0, \exists S \left( \det S\neq 0\right) \tq A_1 = \lambda S^tA_2S$
    \item $\exists f\colon \Po^n \to \Po^n$ homografía t.q. $f\left( Q_1 \right) = Q_2$
    \item $\exists \tilde{\R} \tq M_{\tilde{\R}} \left( Q_2 \right) = M_\R\left( Q_1 \right)$
  \end{itemize}
\end{obs}
\begin{obs}
  Con el método convergencia-pivote: 
  \[
    \left( 
    \begin{array}{c|c}
      A & \Id
    \end{array} \right)
    \sim \dots \sim
    \left( 
    \begin{array}{ccccccc|c}
      \lambda_1 & & & & & & & \\
      & \ddots & & & & & & \\
      & & \ddots & & & & & \\
      & & & \lambda_r & & & &\quad S^t\quad \\
      & & & & 0 & & & \\
      & & & & & \ddots & & \\
      & & & & & & 0 & \\
    \end{array}
    \right)
  \]
  \begin{itemize}
    \item Si $\k = \cx$: $\lambda_1 = \dots = \lambda_r = 1 \to \begin{vmatrix}\left\{ \lambda_i\neq 0 \right\}\end{vmatrix} = r = \rg\left( A \right)$.
    \item Si $\k = \real$: $\lambda_i =
      \begin{cases}
        1 \\
        -1
      \end{cases}
      \to \begin{vmatrix}\left\{ \lambda_i\neq 0 \right\}\end{vmatrix} = i_+ + i_- = \rg\left( A \right)$
  \end{itemize}
\end{obs}
\begin{prop}
  $\Po_\cx^n, Q_1, Q_2$,
  \[
    Q_1 \sim Q_2 \iff \rg\left( Q_1 \right) = \rg\left( Q_2 \right),\text{ donde }\rg\left( Q \right) = \rg\left( M_\R | Q \right).
  \]
\end{prop}
\begin{proof}
  \[ q_1 \sim q_2 \iff \rg \left( q_1 \right) = \rg \left( q_2 \right) \]
\end{proof}

\begin{obs}
  Cuando trabajamos sobre $\real$, podemos tener dos formas bilineales que no son 
  equivalentes y cuyas cuádricas sí lo son. Por ejemplo, las formas
  \[q_1 = \begin{pmatrix} 1 & & \\ & 1 & \\ & & -1 \end{pmatrix}, \, q_2 = 
  \begin{pmatrix} -1 & & \\ & -1 & \\ & & 1 \end{pmatrix}\]
  no cumplen $q_1 \sim q_2$ (en particular, porque ${i_{+}}_1 \neq {i_{+}}_2$) pero sin 
  embargo $Q_1 = [q_1], Q_2 = [q_2]$ sí cumplen $Q_1 \sim Q_2$, pues $\exists \lambda = -1
  \text{ t.q. } q_1 \sim q_2$.

  Lo importante para clasificar $Q$ es la pareja $\{i_+, i_-\}$.
\end{obs}
\begin{defi}
  Definimos, para una cuádrica $Q$, los valores $r = i_+ + i_-$ y $m = \min \{i_+, i_-\}$. 
\end{defi}
\begin{prop}
  Sean $Q_1, Q_2 \subseteq \Po^2_\real$ cuádricas.
  \[Q_1 \sim Q_2 \iff r_1 = r_2, \, m_1 = m_2\]
  Hacemos notar que $r = \rg(Q)$.
\end{prop}

\begin{example}
  \begin{enumerate}[i)]
    \item[]
    \item Clasificación en $\Po^2_\cx$.
      \begin{center}
        \begin{tabular}{ c c c }
          \underline{$r$} & \underline{Ecuación reducida} & 
          \underline{Aspecto o nombre} \\[1ex] 
           $3$ & $x_0^2 + x_1^2 + x_2^2 = 0$ & Cónica no degenerada \\  
           $2$ & $x_0^2 + x_1^2 = (x_0+ix_1)(x_0-ix_1) = 0$ & Par de rectas \\
           $1$ & $x_0^2 = 0$ & Recta doble
        \end{tabular}
      \end{center}
    \item Clasificación en $\Po^2_\real$.
      \begin{center}
        \begin{tabular}{ c c c c }
          \underline{$r$} & \underline{$m$} & \underline{Ecuación reducida} & 
          \underline{Aspecto o nombre} \\[1ex] 
           $3$ & $0$ & $x_0^2 + x_1^2 + x_2^2 = 0$ & Cónica no degenerada (imaginaria)\\  
           $3$ & $1$ & $x_0^2 + x_1^2 - x_2^2 = 0$ & Cónica no degenerada (real) \\  
           $2$ & $0$ & $x_0^2 + x_1^2 = 0$ & Par de rectas que se cortan en un punto real\\
           $2$ & $1$ & $x_0^2 - x_1^2 = 0$ & Par de rectas reales\\
           $1$ & $0$ & $x_0^2 = 0$ & Recta doble
        \end{tabular}
      \end{center}
    \item Clasificación en $\Po^3_\real$.
      \begin{center}
        \begin{tabular}{ c c c c }
          \underline{$r$} & \underline{$m$} & \underline{Ecuación reducida} & 
          \underline{Aspecto o nombre} \\[1ex] 
           $4$ & $0$ & $x_0^2 + x_1^2 + x_2^2 + x_3^2 = 0$ & Cuádrica no degenerada imaginaria\\  
           $4$ & $1$ & $x_0^2 + x_1^2 + x_2^2 - x_3^2 = 0$ & Cuádrica no degenerada real, no reglada \\   
           $4$ & $2$ & $x_0^2 + x_1^2 - x_2^2 - x_3^2 = 0$ & Cuádrica no degenerada real, reglada \\  
           $3$ & $0$ & $x_0^2 + x_1^2 + x_2^2 = 0$ & Cono de base cónica no degenerada imaginaria\\  
           $3$ & $1$ & $x_0^2 + x_1^2 - x_2^2 = 0$ &  Cono de base cónica no degenerada real \\
           $2$ & $0$ & $x_0^2 + x_1^2 = 0$ & Dos planos imaginarios con intersección real\\
           $2$ & $1$ & $x_0^2 - x_1^2 = 0$ & Dos planos reales\\
           $1$ & $0$ & $x_0^2 = 0$ & Plano doble
        \end{tabular}
      \end{center}
  \end{enumerate}
\end{example}

