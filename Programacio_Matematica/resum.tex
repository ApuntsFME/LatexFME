\section{Introducció a la programació lineal no entera i símplex primal}

\begin{defi}[Direcci\'o b\`asica factible]
    Una DBF sobre la SBF $x \in P_e$ associada a $q \in \N$ és $d =
    \begin{bmatrix}
        d_B \\
        d_N
    \end{bmatrix}
    \in \real^n$ tal que:
    \begin{itemize}
        \item $d_{N\left(i\right)} \stackrel{\text{def.}}{=}
            \begin{cases}
                1 & N\left(i\right) = q \\
                0 & N\left(i\right) \neq q
            \end{cases}
            ,\, \forall i \in \left\{1, \dots, n-m \right\}$,
        \item $A \left(x + \theta d\right) = b$ per algun $\theta \in \real^+ \implies d_B \stackrel{\text{def.}}{=} -B^{-1} A_q$.
    \end{itemize}
\end{defi}
\pagebreak                  %TODO VIGILAR AQUESTA CUTRADA
\begin{prop}[Càlcul de $\theta^*$]
    Calculem $\theta^* \stackrel{\text{def.}}{=} \max \left\{ \theta > 0 \, |\, y = x + \theta d \in P_e\right\}$:
    \begin{enumerate}
        \item $A\left(x + \theta d\right) = b,\, \forall \theta$ és cert.
        \item $y = x + \theta d \geq 0$:
            \begin{gather*}
                x_{B\left(i\right)} + \theta d_{B\left(i\right)} \geq 0 \iff \theta \geq -\frac{x_{B\left(i\right)}}{d_{B\left(i\right)}}, \\
                \theta^* = \min_{i \in \left\{1, \dots, m \right\} \, |\, d_{B\left(i\right)} < 0} \left\{ - \frac{x_{B\left(i\right)}}{d_{B\left(i\right)}} \right\}.
            \end{gather*}
    \end{enumerate}
\end{prop}
\begin{prop}
    Sigui $d$ una DBF sobre $x$, una SBF de $P_e$,
    \begin{enumerate}
        \item Si $P_e$ és no degenerat, $d$ és factible:
            \begin{enumerate}[a)]
                \item $d_B \ngeq 0 \implies \theta^* > 0$.
                \item $d_B \geq 0 \implies \theta^*$ no definida, $\forall \theta > 0, x + \theta d \in P_e,\, d$ és un raig extrem.
            \end{enumerate}
        \item Si $P_e$ degenerat ($\exists i \in \B$ tal que $x_{B\left(i\right)} = 0$) $d$ pot no ser factible:
            \begin{gather*}
                \min \left\{ -\frac{x_{B\left(i\right)}}{d_{B\left(i\right)}} \right\} = 0 \implies \nexists \theta > 0 \tq y = x+\theta d \implies d \text{ infactible}.
            \end{gather*}
    \end{enumerate}
\end{prop}
\begin{prop}
    Siguin $q$ i $B\left(p\right)$ les variables que entren i surten de la base, respectivament,
    \begin{gather*}
        \bar{\B} := \left\{ \bar{B}\left(1\right), \dots, \bar{B}\left(m\right) \right\}, \text{ on } \bar{B}\left(i\right) =
        \begin{cases}
            B\left(i\right) & i \neq p \\
            q & i = p
        \end{cases},
    \end{gather*}
    i la nova base és
    \[ \bar{B} = \left[A_{B\left(1\right)}, \dots, A_{B\left(p-1\right)}, A_q, A_{B\left(p+1\right)}, \dots, A_{B\left(m\right)} \right]. \]
\end{prop}
\begin{defi}[DBF de descens]
    \begin{itemize}
        \item[]
        \item $d$ és una DBF de descens si $\forall \theta > 0, c'\left(x + \theta d\right) < c'x \iff c'd < 0$.
        \item Si $d$ és DBF sobre $x$ (SBF), $c'x + \theta^*c'd = c'x + \theta^*r_q$ i
            \begin{itemize}
                \item $r_q = c'd$.
                \item Si $P_e$ no degenerat, llavors la DBF $d$ associada a $q \in \N$ és de descens $\iff r_q < 0$.
            \end{itemize}
    \end{itemize}
\end{defi}
\begin{teo*}[Condicions d'optimalitat de SBF]
    \begin{enumerate}[a)]
        \item[]
        \item $r \geq \left[0\right] \implies x$ és SBF òptima.
        \item $x$ SBF i no degenerada $\implies r \geq \left[0\right]$.
    \end{enumerate}
\end{teo*}
\begin{alg}[del símplex primal]
    \begin{enumerate}
        \item[]
        \item {\bf Inicialització}: Trobem una SBF ($\B, \N, x_B, z$).
        \item \label{simp_pri_pas2} {\bf Identificació de la SBF òptima i selecció VNB entrant}:
            \begin{itemize}
                \item Calculem els costos reduïts: $r' = c_N' - c_B'B^{-1}A_n$.
                \item Si $r' \geq \left[0\right]$, llavors és la SBF òptima. \textcolor{red}{\bf STOP!}
            \end{itemize}
            Altrament seleccionem una $q$ tal que $r_q < 0$ (VNB entrant).
        \item {\bf Càlcul de DBF de descens}:
            \begin{itemize}
                \item $d_B = -B^{-1}A_q$
                \item Si $d_B \geq \left[0\right]$, DBF de descens il·limitat $\implies$ $\left(PL\right)$ il·limitat. \textcolor{red}{\bf STOP!}
            \end{itemize}
        \item {\bf Càlcul de $\theta^*$ i $B\left(p\right)$}:
            \begin{itemize}
                \item Càlcul de $\theta^*$: 
                    \[\theta^* = \min_{i \in \left\{ 1, \dots, m \right\} \,|\, d_{B\left(i\right)} < 0} \left\{-\frac{x_{B\left(i\right)}}{d_{B\left(i\right)}} \right\}.\]
                \item Variable bàsica de sortida: $B\left(p\right)$ tal que $\theta^* = -\frac{x_{B\left(p\right)}}{d_{B\left(p\right)}}$.
            \end{itemize}
        \item {\bf Actualitzacions i canvi de base}:
            \begin{itemize}
                \item $x_B := x_B + \theta^*d_B$, \\
                    $x_q := \theta^*$, \\
                    $z := z + \theta^* r_q$.
                \item $\B := \B \setminus \left\{B\left(p\right)\right\} \cup \left\{q\right\}$, \\
                    $\N := \N \setminus \left\{q\right\} \cup \left\{B\left(p\right)\right\}$.
            \end{itemize}
        \item {\bf Anar} a \ref{simp_pri_pas2}.
    \end{enumerate}
\end{alg}
\begin{obs}[Fase 1 del símplex]
    A la fase 1 del símplex resolem el problema:
    \begin{equation*}
        \lp P_I \rp \begin{dcases*}
            \min\, \sum_{i = 1}^m y_i \\
            \text{s.a.: } \\
            \quad \left( 1 \right) \quad Ax + Iy = b \\
            \quad \left( 2 \right) \quad x, y \geq 0 \\
        \end{dcases*}
    \end{equation*}
    El resultat pot ésser:
    \begin{itemize}
        \item $z_I^* > 0 \implies$ $\left(P\right)$ infactible.
        \item $z_I^* = 0 \implies$ $\left(P\right)$ factible. Dos casos:
            \begin{itemize}
                \item $\B_I^*$ no conté variables $y \implies \B_I^*$ és SBF de $\left(P\right)$.
                \item $\B_I^*$ conté alguna variable $y$. Tenim que $y_B^* = \left[0\right] \implies \B_I^*$ és SBF degenerada de $\left(P_I\right)$ i per tant podem obtenir una SBF de $\left(P\right)$ a partir de $\B_I^*$.
            \end{itemize}
    \end{itemize}
\end{obs}
\begin{rgl}[de Bland] \label{rgl:bland}
    Usem la regla de Bland per a no entrar en bucle al utilitzar el símplex per a resoldre un problema degenerat.
    \begin{enumerate}
        \item Seleccionem com VNB d'entrada la VNB d'índex menor que compleix $r_q < 0$.
        \item Si al seleccionar la variable de sortida hi ha empat, seleccionem la VB amb índex menor.
    \end{enumerate}
\end{rgl}

\section{Teoria de dualitat}

\begin{teo}[feble de dualitat]
    Sigui $x$ SBF de $\left(P\right)$ i $\lambda$ SBF del $\left(D\right)$ associat, llavors
    \[ \lambda' b\leq c'x. \]
\end{teo}
\begin{col}
    \begin{itemize}
        \item[]
        \item $\left(P\right)$ il·limitat $\implies \left(D\right)$ infactible.
        \item $\left(D\right)$ il·limitat $\implies \left(P\right)$ infactible.
    \end{itemize}
\end{col}
\begin{teo}[fort de dualitat]
    Siguin $x^*, \lambda^*$ solucions òptimes de $\left(P\right)$ i el seu dual $\left(D\right)$, respectivament, llavors
    \[ \left(\lambda^*\right)^\prime b = c^\prime x^*.\]
\end{teo}
\begin{col}
    Si $\left(P\right)_e$ de rang complet amb solució, llavors $\left(D\right)$ té solució i òptim a $\lambda' = c_B'B^{-1}$.
\end{col}
\begin{teo}[de folga complementària]
    Siguin $x, \lambda$ solucions factibles de $\left(P\right)$ i $\left(D\right)$, respectivament. $x, \lambda$ són solucions òptimes si i només si
    \begin{center}
        \begin{tabular}{cl}
            $\lambda_j \left(a_j'x - b_j\right) = 0$, & $\forall j \in \left\{1, \dots, m\right\}$, \\
            $\left(c_i - \lambda'A_i\right) x_i = 0$, & $\forall i \in \left\{1, \dots, n\right\}$.
        \end{tabular}
    \end{center}
\end{teo}
\begin{defi}[Solució bàsica factible dual]
    Sigui $\left(P\right)_e$, una SBFD és tota SB de $\left(P\right)_e$ tal que $r \geq \left[0\right]$.
\end{defi}
\begin{alg}[del símplex dual]
    \begin{enumerate}
        \item[]
        \item {\bf Inicialització}: Trobem una SBFD ($\B, \N, x_B, z$).
        \item \label{simp_dual_pas2} {\bf Identificació de la SBF òptima i selecció VB sortint}:
            \begin{itemize}
                \item Si $x_B \geq \left[0\right]$, llavors és la SBF òptima. \textcolor{red}{\bf STOP!}
            \end{itemize}
            Altrament seleccionem VB $p$ amb $x_{B\left(p\right)} < 0$ (VB sortint).
        \item {\bf Càlcul de DBF de $\left(D\right)_e$}:
            \begin{itemize}
                \item $d_{r_N} = \left(\beta_p A_N\right)^\prime$ ($\beta_p$ és la fila $p$-èssima de $B^{-1}$).
                \item Si $d_{r_N} \geq \left[0\right]$, $\left(D\right)_e$ il·limitat. \textcolor{red}{\bf STOP!}
            \end{itemize}
        \item {\bf Càlcul de $\theta^*$ i selecció de la VNB entrant}:
            \begin{itemize}
                \item Càlcul de $\theta_D^*$: 
                    \[\theta_D^* = \min_{j \in \N \,|\, d_{r_{N_j}} < 0} \left\{-\frac{r_j}{d_{r_{N_j}}} \right\}.\]
                \item Variable no bàsica entrant: $q$ tal que $\theta^* = -\frac{x_q}{d_{r_{N_q}}}$.
            \end{itemize}
        \item {\bf Actualitzacions i canvi de base}:
            \begin{itemize}
                \item $r_N := r_N + \theta_D^* d_{r_N}$, \\
                    $\lambda := \lambda - \theta_D^* \beta_p^\prime$, \\
                    $r_{B\left(p\right)} := \theta_D^*$, \\
                    $z := z - \theta^* x_{B\left(p\right)}$.
                \item $\B := \B \setminus \left\{B\left(p\right)\right\} \cup \left\{q\right\}$, \\
                    $\N := \N \setminus \left\{q\right\} \cup \left\{B\left(p\right)\right\}$.
            \end{itemize}
        \item {\bf Anar} a \ref{simp_dual_pas2}.
    \end{enumerate}
\end{alg}
\begin{prop}
    Una SBFD òptima és degenerada ($\exists j \in \N$ tal que $r_j = 0$) si i només si $\left(P\right)_e$ té òptims alternatius.
\end{prop}
\begin{prop}
    Si $\left(P\right)_e$ no té cap SBFD degenerada, el símplex dual convergeix amb un nombre finit d'iteracions. Altrament, podem usar la regla de Bland (\ref{rgl:bland}) per a que convergeixi amb un nombre finit d'iteracions.
\end{prop}

\section{Programació lineal entera}

\begin{obs}
    La següent taula explica les relacions possibles entre un $\left(PLE\right)$ i la seva relaxació lineal $\left(RL\right)$.
    \begin{center}
        \begin{tabular}{|l|c|c|c|} \hline
            $\left(PLE\right) \setminus \left(RL\right)$ & Solució òptima & Infactible & Il·limitat \\ \hline
            Solució òptima & Sí & No & \thead{Sí ($A \in \mathcal{M}_{n\times m}(\real)$) \\ No ($A \in \mathcal{M}_{n\times m}(\q)$)} \\ \hline
            Infactible & Sí & Sí & Sí \\ \hline
            Il·limitat & No & No & Sí \\ \hline
        \end{tabular}
    \end{center}
\end{obs}

En aquest resum falta una part important de programació lineal entera, per a més informació, consulteu els apunts de classe!

\section{Programaci\'o no lineal sense restriccions}

\begin{teo*}[Condicions necess\`aries d'optimalitat]
    Sigui $f \colon \real^n \to \real$ un problema d'optimitzaci\'o no lineal $\min_{x\in\real^n} f\left( x \right)$. Si $x^*$ \'es un m\'inim local de $f$ i $f \in \C^2$ en un entorn de $x^*$, llavors:
    \begin{enumerate}[i)]
        \item $\nabla f\left( x^* \right) = 0$. (Condici\'o de 1r ordre)
        \item $\nabla^2 f\left( x^* \right) \geq 0$. (Condici\'o de 2n ordre)
    \end{enumerate}
\end{teo*}
\begin{teo*}[Condicions suficients d'optimalitat]
    Si $f \in \C^2$ en un entorn obert de $x^*$, $\nabla f\left( x^* \right) = 0$ i $\nabla^2 f\left( x^* \right)$ \'es definida positiva, llavors $x^*$ \'es m\'inim local estricte de $f$.
\end{teo*}
\begin{teo*}[Condicions d'optimalitat en problemes convexos]
    Si $f$ \'es convexa i diferenciable, llavors $\nabla f\left( x^* \right) = 0 \iff x^*$ \'es m\'inim global de $f$.
\end{teo*}
\begin{met}[del Gradient]
    $x^{k+1} = x^k + \alpha^kd^k$, \quad $d^k = -\nabla f\left( x^k \right)$. \\
    $\alpha^k$ ha de complir les condicions d'Armijo-Wolfe.
\end{met}
\begin{met}[de Newton]
    $x^{k+1} = x^k + \alpha^kd^k$, \quad $d^k = -\left( \nabla^2 f\left( x^k \right) \right)^{-1} \nabla f\left( x^k \right)$. \\
    $\alpha^k$ ha de complir les condicions d'Armijo-Wolfe, normalment $\alpha^k = 1$.
\end{met}
\begin{prop}[Condicions d'Armijo-Wolfe]
    \begin{enumerate}[i)]
        \item[]
        \item Condici\'o de descens suficient (AW-1): \\
            $g\left( \alpha \right) \leq g\left( 0 \right) + \alpha c_1g'\left( 0 \right),\, c_1 \in (0,1).$ \\
            $f\left( x^k + \alpha d^k \right) \leq f\left( x^k \right) + \alpha c_1 \nabla f\left( x^k \right)^t d^k,\, c_1 \in (0,1).$
        \item Condici\'o de corbatura (AW-2): \\
            $g'\left( \alpha \right) \geq c_2 g'\left( 0 \right),\, 0 < c_1 < c_2 < 1.$ \\
            $\nabla f\left( x^k + \alpha d^k \right)^td^k \geq c_2 \nabla f\left( x^k \right)d^k,\, 0 < c_1 < c_2 < 1$.
    \end{enumerate}
\end{prop}

\section{Programaci\'o no lineal amb restriccions}

Donat el problema:
\begin{equation*}
    \lp P \rp \begin{dcases*}
        \min f \left( x \right) \\
        \text{s.a.: } \\
        \quad \left( 1 \right) \quad h \left( x \right) = 0 \\
        \quad \left( 2 \right) \quad g \left( x \right) \leq 0 \\
    \end{dcases*}
\end{equation*}
amb $\mathcal{L} \left( x, \lambda, \mu \right) = f\left( x \right) + \lambda^t h\left( x \right) + \mu^t g\left( x \right)$.

\begin{prop}[Condicions necess\`aries]
    Sigui $x^*$ \`optim local. Si \'es punt regular, llavors $\exists \lambda^* \in \real^m$ i $\mu^* \in \real^p$ tals que:
    \begin{enumerate}[i)]
        \item \label{nec:1} $h\left( x^* \right) = 0,\, g\left( x^* \right) \leq 0$.
        \item \label{nec:2} $\nabla_x \mathcal{L}\left( x^*, \lambda^*, \mu^* \right) = \nabla f\left( x^* \right) + \nabla h\left( x^* \right) \lambda^* + \nabla g\left( x^* \right)\mu^* = 0$.
        \item \label{nec:3} $\mu^* \geq 0$ i $\left( \mu^* \right)^t g\left( x^* \right) = 0$ (si $g_i\left( x^* \right)$ \'es inactiva, llavors $\mu_j^* = 0$).
        \item \label{nec:4} $d^t \nabla_{x_ix_j}^2 \mathcal{L} \left( x^*, \lambda^*, \mu^* \right) d \geq 0, \forall d\in M =
            \begin{cases}
                \left( \nabla h_i\left( x^* \right) \right)^t d = 0 & i \in \left\{1,\dots,m\right\} \\
                \left( \nabla g_j\left( x^* \right) \right)^t d = 0 & j \in \mathcal{A}\left( x^* \right)
            \end{cases}$ \\
    \end{enumerate}
    Els punts \ref{nec:1}, \ref{nec:2} i \ref{nec:3} s\'on les condicions de 1r ordre del KKT i el punt \ref{nec:4} \'es la condici\'o de 2n ordre.
    \begin{adjustwidth}{1cm}{}
    {\it Nota}: $\mathcal{A}\left( x^* \right) = \left\{ j \in \left\{ 1, \dots, p \right\} \, |\, g_j \left( x \right) = 0\right\}$ \'es el conjunt d'\'index de desigualtats actives a $x$.
    \end{adjustwidth}
\end{prop}
\begin{prop}[Condicions suficients]
    Sigui $x^*$, \'es \`optim local si satisf\`a:
    \begin{enumerate}[i)]
        \item \label{suf:1} $h\left( x^* \right) = 0,\, g\left( x^* \right) \leq 0$.
        \item \label{suf:2} $\nabla_x \mathcal{L} \left( x^*, \lambda^*, \mu^* \right) = \nabla f\left( x^* \right) + \nabla h\left( x^* \right)\lambda^* + \nabla g\left( x^* \right)\mu^* = 0$.
        \item \label{suf:3} $\mu^* \geq 0$ i $\left( \mu^* \right)^t g\left( x^* \right) = 0$ ($g_i\left( x^* \right) < 0 \implies \mu_j^* = 0$).
        \item \label{suf:4} $d^t \nabla_{x_ix_j}^2 \mathcal{L}\left( x^*, \lambda^*, \mu^* \right) d > 0, d \in M' =
            \begin{cases}
                \left( \nabla h_i\left( x^* \right) \right)^t d = 0 & i \in \left\{1,\dots,m\right\} \\
                \left( \nabla g_j\left( x^* \right) \right)^t d = 0 & j \in \mathcal{A}\left( x^* \right) \cap \left\{ j \,|\, \mu^* > 0 \right\}
            \end{cases}$ \\
    \end{enumerate}
    Els punts \ref{suf:1}, \ref{suf:2} i \ref{suf:3} s\'on les condicions de 1r ordre del KKT i el punt \ref{suf:4} \'es la condici\'o de 2n ordre.
    \begin{adjustwidth}{1cm}{}
    {\it Nota}: $\mathcal{A}\left( x^* \right) = \left\{ j \in \left\{ 1, \dots, p \right\} \, |\, g_j \left( x \right) = 0\right\}$ \'es el conjunt d'\'index de desigualtats actives a $x$.
    \end{adjustwidth}
\end{prop}
