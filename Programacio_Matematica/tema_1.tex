\chapter[Introducció a la prog. lineal \& símplex primal]%
{Introducció a la programació lineal no entera i símplex primal}

\section{Poliedres}
\begin{defi}[Hiperplà]\label{defi:hiperpla}
	Un hiperplà és un subconjunt de $\real^n$ definit com \[H = \{\vb{x}\in\real^n \mid \vb{a}'\vb{x}=b \}\,, \] on $\vb{a}\in\real^n$ i s'anomena el vector normal de l'hiperplà, i $b\in\real$.
\end{defi}

\begin{defi}[Semiespai]\label{defi:semiespai}
	Donat un hiperplà $H = \{\vb{x}\in\real^n \mid \vb{a}'\vb{x}=b \}$, es defineixen dos semiespais: 
	\begin{align*}
		S_{\ge} &= \{\vb{x}\in\real^n \mid \vb{a}'\vb{x} \ge b \}\\
		S_{\le} &= \{\vb{x}\in\real^n \mid \vb{a}'\vb{x} \le b \}\,.
	\end{align*}
\end{defi}

En referència a la definició \ref{defi:semiespai}, es diu informalment que el semiespai $S_\ge$ és el conjunt de punts que estan ``per sobre'' de l'hiperplà, i $S_\le$, dels que estan ``per sota''.

\begin{defi}[Poliedre]\label{defi:poliedre}
	Un poliedre $P$ és un subconjunt de $\real^n$ definit per una intersecció de semiespais: \[P = \{\vb{x}\in\real^n \mid \vb{a}_i'\vb{x} \le b_i \quad \forall i\in \{1,\ldots,m\} \}\,,\] on $n,m\in\n\,$, $\vb{a}_1,\ldots,\vb{a}_m \in \real^n$ i $b_1,\ldots,b_m\in\real$.
\end{defi}

\section{Solucions bàsiques factibles}

El fet que hi hagi solucions òptimes que siguin punts extrems no és de gran ajuda computacional per sí sol. Per tenir una caracterització computacionalment assequible cal el concepte de solució bàsica factible, que es defineix a continuació.

\begin{defi}[Solució bàsica]\label{defi:SB}
	Donat un poliedre en forma estàndard $P_e$ associat a un problema de programació lineal, una solució bàsica és un vector $\vb{x}\in P_e$ del qual es pot fer la partició $\vb{x} = \left[\vb{x}_\B\mid\vb{x}_\N\right]$ tal que
	\begin{enumerate}[i)]
		\item $\B,\N\subseteq\{1,\ldots,n \}$ són conjunts complementaris d'índexos i $|\B| = m$, on $m$ és el nombre de restriccions (el nombre de files de la matriu $A$). $\B$ s'anomena el conjunt d'índexos de variables bàsiques o \textbf{base}, i $\N$ s'anomena el conjunt d'índexos de variables no bàsiques.
		%
		\item $\vb{x}_\B \defeq 
		\begin{bmatrix}x_{\B(1)} & x_{\B(2)} & \cdots & x_{\B(m)}\end{bmatrix}\,$
		i
		$\,\vb{x}_\N \defeq 
		\begin{bmatrix}x_{\N(1)} & x_{\N(2)} & \cdots & x_{\N(n-m)}\end{bmatrix} = \vb{0}\,$, on $\B(i)$ i $\N(i)$ denoten, respectivament, l'$i$-èsim índex de $\B$ i $\N$.
		%
		\item La matriu definida per
		\[
			B \defeq
			\begin{bmatrix}
			\arrowvert& 	\arrowvert& 	  & 	\arrowvert\\
			A_{\B(1)}&		A_{\B(1)}&	\cdots&		A_{\B(m)}\\
			\arrowvert& 	\arrowvert& 	  & 	\arrowvert\\
			\end{bmatrix}
		\]
		és no singular. Aquesta matriu s'anomena matriu bàsica.
	\end{enumerate}
\end{defi}

Aquesta definició no és gaire intuïtiva; a continuació es presenta una versió alternativa basada en el poliedre general.

\begin{defi}[Solució bàsica (definició alternativa)]\label{defi:SB-alt}
	Donat un poliedre (general) $P\subseteq\real^n$ associat a un problema de programació lineal, direm que $\vb{x}\in P$ és una solució bàsica si és la intersecció de $n$ hiperplans (associats cadascun a una restricció) linealment independents.
\end{defi}

\section{Direccions bàsiques factibles}
\begin{defi}[Direcci\'o b\`asica factible (DBF)]
    Una DBF sobre la SBF $\vb{x} \in P_e$ associada a $q \in \N$ és $\vb{d} =
    \begin{bmatrix}
        \vb{d}_{\B} \\
        \vb{d}_{\N}
    \end{bmatrix}
    \in \real^n$ tal que:
    \begin{itemize}
        \item $d_{N\left(i\right)} \defeq
            \begin{cases}
                1 & N\left(i\right) = q \\
                0 & N\left(i\right) \neq q
            \end{cases}
            ,\, \forall i \in \left\{1, \dots, n-m \right\}$,
        \item $A \left(\vb{x} + \theta \vb{d}\right) = b$ per algun $\theta \in \real^+ \implies \vb{d}_{\B} \defeq -B^{-1} A_q$.
    \end{itemize}
\end{defi}

\begin{prop}[Càlcul de $\theta^*$]
    Calculem $\theta^* \defeq \max \left\{ \theta > 0 \, |\, \vb{y} = \vb{x} + \theta \vb{d} \in P_e\right\}$:
    \begin{enumerate}
        \item $A\left(\vb{x} + \theta \vb{d}\right) = b,\, \forall \theta$ és cert.
        \item $\vb{y} = \vb{x} + \theta \vb{d} \geq 0$:
            \begin{gather*}
                \vb{x}_{B\left(i\right)} + \theta d_{B\left(i\right)} \geq 0 \iff \theta \geq -\frac{\vb{x}_{B\left(i\right)}}{d_{B\left(i\right)}}, \\
                \theta^* = \min_{\{i \in \left\{1, \dots, m \right\} \mid d_{B\left(i\right)} < 0\}} \left\{ - \frac{\vb{x}_{B\left(i\right)}}{d_{B\left(i\right)}} \right\}.
            \end{gather*}
    \end{enumerate}
\end{prop}
\begin{prop}
    Sigui $\vb{d}$ una DBF sobre $\vb{x}$, una SBF de $P_e$,
    \begin{enumerate}
        \item Si $P_e$ és no degenerat, $\vb{d}$ és factible:
            \begin{enumerate}[a)]
                \item $\vb{d}_{\B} \ngeq 0 \implies \theta^* > 0$.
                \item $\vb{d}_{\B} \geq 0 \implies \theta^*$ no definida, $\forall \theta > 0, \vb{x} + \theta \vb{d} \in P_e,\, \vb{d}$ és un raig extrem.
            \end{enumerate}
        \item Si $P_e$ degenerat ($\exists i \in \B$ tal que $\vb{x}_{B\left(i\right)} = 0$) $\vb{d}$ pot no ser factible:
            \begin{gather*}
                \min \left\{ -\frac{\vb{x}_{B\left(i\right)}}{d_{B\left(i\right)}} \right\} = 0 \implies \nexists \theta > 0 \tq \vb{y} = \vb{x}+\theta \vb{d} \implies \vb{d} \text{ infactible}.
            \end{gather*}
    \end{enumerate}
\end{prop}
\begin{prop}
    Siguin $q$ i $B\left(p\right)$ les variables que entren i surten de la base, respectivament,
    \begin{gather*}
        \bar{\B} := \left\{ \bar{B}\left(1\right), \dots, \bar{B}\left(m\right) \right\}, \text{ on } \bar{B}\left(i\right) =
        \begin{cases}
            B\left(i\right) & i \neq p \\
            q & i = p
        \end{cases},
    \end{gather*}
    i la nova matriu bàsica és
    \[ \bar{B} = \left[A_{B\left(1\right)}, \dots, A_{B\left(p-1\right)}, A_q, A_{B\left(p+1\right)}, \dots, A_{B\left(m\right)} \right]. \]
\end{prop}
\begin{defi}[DBF de descens]
    \begin{itemize}
        \item[]
        \item $\vb{d}$ és una DBF de descens si $\forall \theta > 0, \vb{c}'\left(\vb{x} + \theta \vb{d}\right) < \vb{c}'\vb{x} \iff \vb{c}'\vb{d} < 0$.
        \item Si $\vb{d}$ és DBF sobre $\vb{x}$ (SBF), $\vb{c}'\vb{x} + \theta^*\vb{c}'\vb{d} = \vb{c}'\vb{x} + \theta^*r_q$ i
            \begin{itemize}
                \item $r_q = \vb{c}'\vb{d}$.
                \item Si $P_e$ no degenerat, llavors la DBF $\vb{d}$ associada a $q \in \N$ és de descens $\iff r_q < 0$.
            \end{itemize}
    \end{itemize}
\end{defi}
\begin{teo*}[Condicions d'optimalitat de SBF]
    \begin{enumerate}[a)]
        \item[]
        \item $r \geq \vb{0} \implies \vb{x}$ és SBF òptima.
        \item $\vb{x}$ SBF i no degenerada $\implies \vb{r} \geq \vb{0}$.
    \end{enumerate}
\end{teo*}

\section{Algorisme del Símplex Primal}
\begin{alg}[del símplex primal]\label{alg:asp}
    \begin{enumerate}
        \item[]
        \item {\bf Inicialització}: Trobem una SBF ($\B, \N, \vb{x}_{\B}, z$).
        \item \label{simp_pri_pas2} {\bf Identificació de la SBF òptima i selecció VNB entrant}:
            \begin{itemize}
                \item Calculem els costos reduïts: $\vb{r}' = c_N' - c_B'B^{-1}A_n$.
                \item Si $\vb{r}' \geq \vb{0}$, llavors és la SBF òptima. \textcolor{red}{\bf STOP!}
            \end{itemize}
            Altrament seleccionem una $q$ tal que $r_q < 0$ (VNB entrant).
        \item {\bf Càlcul de DBF de descens}:
            \begin{itemize}
                \item $\vb{d}_{\B} = -B^{-1}A_q$
                \item Si $\vb{d}_{\B} \geq \vb{0}$, DBF de descens il·limitat $\implies$ $\left(PL\right)$ il·limitat. \textcolor{red}{\bf STOP!}
            \end{itemize}
        \item {\bf Càlcul de $\theta^*$ i $B\left(p\right)$}:
            \begin{itemize}
                \item Càlcul de $\theta^*$: 
                    \[\theta^* = \min_{i \in \left\{ 1, \dots, m \right\} \,|\, d_{B\left(i\right)} < 0} \left\{-\frac{x_{B\left(i\right)}}{d_{B\left(i\right)}} \right\}.\]
                \item Variable bàsica de sortida: $B\left(p\right)$ tal que $\theta^* = -\frac{\vb{x}_{B\left(p\right)}}{d_{B\left(p\right)}}$.
            \end{itemize}
        \item {\bf Actualitzacions i canvi de base}:
            \begin{itemize}
                \item $\vb{x}_{\B} := \vb{x}_{\B} + \theta^*\vb{d}_{\B}$, \\
                    $x_q := \theta^*$, \\
                    $z := z + \theta^* r_q$.
                \item $\B := \B \setminus \left\{B\left(p\right)\right\} \cup \left\{q\right\}$, \\
                    $\N := \N \setminus \left\{q\right\} \cup \left\{B\left(p\right)\right\}$.
            \end{itemize}
        \item {\bf Anar} a \ref{simp_pri_pas2}.
    \end{enumerate}
\end{alg}
\begin{obs}[Fase 1 del símplex]
    A la fase 1 del símplex resolem el problema:
    \begin{equation*}
        \lp P_I \rp \begin{dcases*}
            \min\, \sum_{i = 1}^m \vb{y}_i \\
            \text{s.a.: } \\
            \quad \left( 1 \right) \quad A\vb{x} + I\vb{y} = \vb{b} \\
            \quad \left( 2 \right) \quad \vb{x}, \vb{y} \geq 0 \\
        \end{dcases*}
    \end{equation*}
    El resultat pot ésser:
    \begin{itemize}
        \item $z_I^* > 0 \implies$ $\left(P\right)$ infactible.
        \item $z_I^* = 0 \implies$ $\left(P\right)$ factible. Dos casos:
            \begin{itemize}
                \item $\B_I^*$ no conté variables $\vb{y} \implies \B_I^*$ és SBF de $\left(P\right)$.
                \item $\B_I^*$ conté alguna variable $\vb{y}$. Tenim que $\vb{y}_B^* = \vb{0} \implies \B_I^*$ és SBF degenerada de $\left(P_I\right)$ i per tant podem obtenir una SBF de $\left(P\right)$ a partir de $\B_I^*$.
            \end{itemize}
    \end{itemize}
\end{obs}
\begin{rgl}[de Bland] \label{rgl:bland}
    Usem la regla de Bland per a no entrar en bucle al utilitzar el símplex per a resoldre un problema degenerat.
    \begin{enumerate}
        \item Seleccionem com VNB d'entrada la VNB d'índex menor que compleix $r_q < 0$.
        \item Si al seleccionar la variable de sortida hi ha empat, seleccionem la VB amb índex menor.
    \end{enumerate}
\end{rgl}