\chapter{Dualitat}

\begin{teo}[feble de dualitat]\label{teo:dualitat-feble}
    Per tota solució factible $x$ d'un problema primal $\left(P\right)$ i tota solució factible $\lambda$ del dual $\left(D\right)$ associat
    \[ \lambda'b\leq c'x\,. \]
    \begin{proof}
    	Sigui $m\times n$ l'ordre de la matriu de restriccions $A$ al problema primal. Definim, per cada $i\in\{1,\ldots,m\}$ (per cada restricció de $(P)$) i per cada $j\in\{1,\ldots,n\}$ (per cada variable de $(P)$), respectivament, 
    	\begin{align*}
    		u_i &\defeq \lambda_i(a_i'x-b_i)\\
    		v_j &\defeq (c_j - \lambda'A_j)x_j\,.
    	\end{align*}
    	
    	Primerament, veiem que $u_i$ i $v_j$ sempre són positius. Si $a_i'x_i-b_i \ge 0$, llavors la $i$-èsima restricció de $(P)$ serà de la forma \[a_i'x \ge b_i\,,\] i per tant la corresponent variable dual $\lambda_i$ complirà $\lambda_i \ge 0$. D'altra banda, si $a_i'x_i-b_i \le 0$ la variable dual $\lambda_i$ serà negativa. Finalment, si $a_i'x_i-b_i = 0$, es té que $u_i = 0$ independentment de $\lambda_i$. Per tant $u_i \ge 0$ per tot $i$. Es pot fer un raonament similar per $v_j$.
    	
    	Ara, observem que
    	\begin{alignat}{2}{}
    		\sum_{i = 1}^{m} u_i &= \sum_{i = 1}^{m} \lambda_ia_i'x-\lambda_ib_i &= \lambda'Ax - \lambda'b\,, \label{eq:ui}\\
    		\sum_{j = 1}^{n} v_j &= \sum_{j = 1}^{n} c_jx_j - \lambda'A_jx_j &= c'x - \lambda'Ax\,. \label{eq:vj}
    	\end{alignat}
    	Si sumem \eqref{eq:ui} i \eqref{eq:vj} i tenim en compte que $u_i \ge 0$ i $v_j \ge 0$ per tot $i$ i $j$, obtenim
    	\begin{equation}\label{eq:sum-ui-vj}
    		\sum_{i = 1}^{m} u_i + \sum_{j = 1}^{n} v_i= c'x-\lambda'b \ge 0\,
    	\end{equation}
    	i per tant \[c'x \ge \lambda'b\,.\]
    \end{proof}
\end{teo}
\begin{col}[Corol·laris del teorema feble de dualitat (\ref{teo:dualitat-feble})]\label{col:feb}
	Per qualsevol problema primal $(P)$ i el seu dual $(D)$: 
    \begin{enumerate}[i)]
        \item $\left(P\right)$ il·limitat $\implies \left(D\right)$ infactible.
        \item $\left(D\right)$ il·limitat $\implies \left(P\right)$ infactible.
        \item Siguin $x^\ast \in P$ i $\lambda^\ast \in D$ solucions factibles del primal i del dual respectivament. Si ${\lambda^\ast}'b = c'x^\ast$ llavors $\lambda^\ast$ i $x^\ast$ són solucions òptimes pel problema dual i primal respectivament.
    \end{enumerate}
	\begin{proof}
		\begin{enumerate}[i)]
			\item[]
			\item Sigui $x$ solució factible de $(P)$ i $\lambda$ solució factible de $(D)$. Per hipòtesi $(P)$ és il·limitat i per tant \[\forall k \in \real\ \exists x\in P\qcolon c'x < k\,. \] Suposem que existís una solució dual $\lambda$ factible. Llavors, pel \hyperref[teo:dualitat-feble]{teorema feble de dualitat}, $\forall x\in P\colon \lambda'b \le c'x$. D'altra banda, per la proposició anterior, tenim que $\exists x\in P\colon \lambda'b > c'x$---prenent el valor $\lambda'b$ per $k\in\real$; contradicció. Per tant $(D)$ és infactible.
			%
			\item Sigui $\lambda$ solució factible de $(D)$ i $x$ solució factible de $(P)$. Per hipòtesi $(D)$ és il·limitat i per tant \[\forall k \in \real\ \exists x\in P\qcolon \lambda'b > k\,. \] Suposem que existís una solució primal $x$ factible. Llavors, pel {teorema feble de dualitat}, $\forall \lambda\in D\colon \lambda'b \le c'x$. D'altra banda, per la proposició anterior, tenim que $\exists \lambda\in D\colon \lambda'b > c'x\,$; contradicció. Per tant $(P)$ és infactible.
			%
			\item Pel teorema feble de dualitat, tenim que $\forall \lambda \in D: \lambda'b \le c'{x^\ast}$, i per tant, aplicant la hipòtesi, \[\forall \lambda \in D: \lambda 'b \le {\lambda^\ast}'b\,;\] ergo, $\lambda^\ast$ és l'òptim del dual (ja que és un problema de maximització). D'altra banda, novament pel teorema anterior, $\forall x \in P: {\lambda^\ast}'b \le c'x\,$, cosa que implica per hipòtesi que \[\forall x \in D: c'{x^\ast} \le c'x\] i per tant $x^\ast$ és un òptim.
		\end{enumerate}
	\end{proof}
\end{col}

\begin{teo}[fort de dualitat]\label{teo:dualitat-fort}
    Sigui $\left(P\right)$ un problema primal i $\left(D\right)$ el seu dual. Llavors $(D)$ té òptim si i només si $(P)$ té òptim. A més, si en tenen, el valor de la funció objectiu en l'òptim coincideix.
    \begin{proof}
    	Considerem la forma estàndard del problema primal $(P)$,
    	
    	\[
    		(P_e)
    		\begin{cases}
    		\begin{aligned}
			   	\min 			&&& c'x\\
	    		\text{s.a.}		&&& Ax = b\\
	    						&&& x\ge0
    		\end{aligned}
    		\end{cases}\ ,
    	\]
    	on la matriu $A$ és d'ordre $m\times n$, i el dual corresponent (a $P_e$):
    	\[
	    	(D_{P_e})
	    	\begin{cases}
	    	\begin{aligned}
	    	\max 			&&& \lambda'b\\
	    	\text{s.a.}		&&& \lambda'A \le c'
	    	\end{aligned}
	    	\end{cases}\ ,
    	\]
    	
    	Primer demostrarem que si $(P_e)$ té òptim, $(D_{P_e})$ també. Si $(P_e)$ té òptim, l'algorisme del símplex primal aplicat a $(P_e)$ terminarà amb una SBF òptima $x^\ast$ de $(P_e)$ tal que el vector de costos reduïts sigui no negatiu: \[r' = c_\N '- c_\B' B^{-1} A_\N \ge 0\,.\] Sigui $\lambda^\ast$ el vector definit per ${\lambda^\ast}'\defeq c_\B'B^{-1}$. Demostrarem que $\lambda^\ast$ és solució òptima de $(D_{P_e})$.
    	
    	Primer hem de comprovar si és factible. Efectivament, 
    	\[
    		{\lambda^\ast}'A = 
    		c_\B' B^{-1}
    		\begin{bmatrix}
		   		B 	\vert 	A_\N
    		\end{bmatrix} =
    		\begin{bmatrix}
    			c_\B' \Id & c_\B B^{-1} A_\N    			
    		\end{bmatrix}
    		\le
    		\begin{bmatrix}
	    		c_\B' 	&c_\N'
    		\end{bmatrix}
    		= c'\,.
    	\]
    	D'altra banda, \[{\lambda^\ast}'b = c_\B' B^{-1} b = c_\B' {x^\ast}_\B = c'{x^\ast}\,. \] Aplicant el corol·lari \ref{col:feb}, deduïm que $\lambda^\ast$ és solució òptima de $(D_{P_e})$. Queda demostrat que si $(P_e)$ té solució òptima existeix una solució òptima de $(D_{P_e})$ (i pel corol·lari mencionat produeixen el mateix valor de la f.o.).
    	
    	Finalment, podem afirmar que aquest resultat és vàlid per els problemes que no estan en forma estàndard, ja que el problema $(P)$ i el seu estàndard són equivalents---sempre es pot passar d'una solució de l'un a una solució de l'altre, i el valor de la f.o. en l'òptim és el mateix---, i el dual de l'estàndard $(D_{P_e})$ és equivalent al dual de $(P)$.
    \end{proof}
\end{teo}

\begin{col}[Corol·lari del teorema fort de dualitat (\ref{teo:dualitat-fort})]
    Si $\left(P\right)_e$ és un problema factible i amb matriu de restriccions $A$ de rang complet, llavors $\left(D\right)$ té solució factible òptima a $\lambda' = c_\B'B^{-1}$.
\end{col}

\begin{teo}[de folga complementària]
    Siguin $x, \lambda$ solucions factibles de $\left(P\right)$ i $\left(D\right)$, respectivament. Llavors $x$ i $\lambda$ són solucions òptimes si i només si
    \begin{alignat*}{2}{}
        u_i &\defeq \lambda_i \left(a_i'x - b_i\right) &= 0 &\quad \forall i \in \left\{1, \dots, m\right\} \\
        v_j &\defeq \left(c_j - \lambda'A_j\right) x_j &= 0 &\quad \forall j \in \left\{1, \dots, n\right\}\,.
    \end{alignat*}
    \begin{proof}
    	Demostrem primer que si $u_i = 0 \land v_j = 0$ per tot $i$ i tot $j$, llavors $x$ i $\lambda$ són òptimes. Com hem vist a la demostració del teorema feble de dualitat---l'equació \eqref{eq:sum-ui-vj}---, 
    	\begin{equation*}
	    	\sum_{i = 1}^{m} u_i + \sum_{j = 1}^{n} v_j = c'x - \lambda'b\,.
    	\end{equation*}
    	Sabent que $u_i = 0$ per tot $i$ i $v_j = 0$ per tot $j$, tenim que \[\sum_{i = 1}^{m} u_i + \sum_{j = 1}^{n} v_j = 0 = c'x - \lambda'b \Leftrightarrow c'x = \lambda'b\,,  \] i pel corol·lari \ref{col:feb} del teorema feble, $x$ i $\lambda$ són solucions òptimes per al problema respectiu.
    	
    	Per demostrar la implicació recíproca, només cal notar que, com hem demostrat a la demostració del corol·lari \ref{col:feb}, $u_i$ i $v_j$ sempre són no negatius, i per tant l'única manera de que es compleixi \[\sum_{i=1}^{m} u_i + \sum_{j=1}^{n} v_j = 0 \] és que $u_i = 0$ per tot $i$ i $v_j = 0$ per tot $j$. Per tant només cal recórrer en sentit recíproc totes les implicacions de la demostració que acabem de fer.
    \end{proof}
\end{teo}

\begin{defi}[Solució bàsica factible dual]
    Sigui $\left(P\right)_e$, una SBFD és tota SB de $\left(P\right)_e$ tal que $r \geq \left[0\right]$.
\end{defi}

\begin{alg}[del símplex dual]
    \begin{enumerate}
        \item[]
        \item {\bf Inicialització}: Trobem una SBFD ($\B, \N, x_B, z$).
        \item \label{simp_dual_pas2} {\bf Identificació de la SBF òptima i selecció VB sortint}:
            \begin{itemize}
                \item Si $x_B \geq \left[0\right]$, llavors és la SBF òptima. \textcolor{red}{\bf STOP!}
            \end{itemize}
            Altrament seleccionem VB $p$ amb $x_{B\left(p\right)} < 0$ (VB sortint).
        \item {\bf Càlcul de DBF de $\left(D\right)_e$}:
            \begin{itemize}
                \item $d_{r_N} = \left(\beta_p A_N\right)^\prime$ ($\beta_p$ és la fila $p$-èssima de $B^{-1}$).
                \item Si $d_{r_N} \geq \left[0\right]$, $\left(D\right)_e$ il·limitat. \textcolor{red}{\bf STOP!}
            \end{itemize}
        \item {\bf Càlcul de $\theta^*$ i selecció de la VNB entrant}:
            \begin{itemize}
                \item Càlcul de $\theta_D^*$: 
                    \[\theta_D^* = \min_{i \in \N \,|\, d_{r_{N_j}} < 0} \left\{-\frac{r_j}{d_{r_{N_j}}} \right\}.\]
                \item Variable no bàsica entrant: $q$ tal que $\theta^* = -\frac{x_q}{d_{r_{N_q}}}$.
            \end{itemize}
        \item {\bf Actualitzacions i canvi de base}:
            \begin{itemize}
                \item $r_N := r_N + \theta_D^* d_{r_N}$, \\
                    $\lambda := \lambda - \theta_D^* \beta_p^\prime$, \\
                    $r_{B\left(p\right)} := \theta_D^*$, \\
                    $z := z - \theta^* x_{B\left(p\right)}$.
                \item $\B := \B \setminus \left\{B\left(p\right)\right\} \cup \left\{q\right\}$, \\
                    $\N := \N \setminus \left\{q\right\} \cup \left\{B\left(p\right)\right\}$.
            \end{itemize}
        \item {\bf Anar} a \ref{simp_dual_pas2}.
    \end{enumerate}
\end{alg}

\begin{prop}
    Una SBFD òptima és degenerada ($\exists j \in \N$ tal que $r_j = 0$) si i només si $\left(P\right)_e$ té òptims alternatius.
\end{prop}

\begin{prop}
    Si $\left(P\right)_e$ no té cap SBFD degenerada, el símplex dual convergeix amb un nombre finit d'iteracions. Altrament, podem usar la regla de Bland (\ref{rgl:bland}) per a que convergeixi amb un nombre finit d'iteracions.
\end{prop}