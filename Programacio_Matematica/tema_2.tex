\chapter{Dualitat}

\section{Definició del problema dual}

La idea intuïtiva\footnote{L'explicació d'aquesta secció és una paràfrasi de
	\textit{Introduction to Linear Optimization}, Bertsimas \& Tsitsiklis, 
	capítol
	4, secció 1.} darrere de l'anomenat ``problema dual'' és la següent. 
	Suposem que
$(P)$ és el següent problema lineal de minimització:
\[
(P)
\begin{cases}
\begin{aligned}
\min 			&&& \vb{c}'\vb{x}\\
\text{s.a.}		&&& A\vb{x} = \vb{b}\\
&&& \vb{x}\ge0
\end{aligned}
\end{cases}\ .
\]
En comptes de considerar el problema com està, trobant una solució òptima
$x^\ast$ que satisfaci estrictament les constriccions, podem pensar en una
versió ``relaxada'' del problema, on minimitzem sense constriccions la mateixa
funció objectiu, però sumant-li un cert terme que indica la ``quantitat'' amb la
qual s'estan violant les constriccions. 

Si, per cada [$j$-èsima] restricció, introduïm un paràmetre $\lambda_j\in\real$
que indiqui el ``preu que implica violar la restricció'', podem formular el
mencionat problema ``relaxat'',
\[
(\tilde{P})
\begin{cases}
\begin{aligned}
\min 			&&& \vb{c}'\vb{x} + \vb{\lambda}'(\vb{b}-A\vb{x})\\
\text{s.a.}		&&& \vb{x} \ge 0\\
\end{aligned}
\end{cases}\ ,
\]
on $\vb{\lambda}$ és el vector dels preus $\lambda_j$ (per tant és del mateix
ordre que $\vb{b}$), i $\vb{b}-A\vb{x}$ es pot pensar com un vector que conté la
quantitat amb què s'està violant cada constricció. Anomenem $z$ la funció
objectiu de $(P)$ i $\tilde{z}(\vb{\lambda})$---en funció de $\vb{\lambda}$---la
funció objectiu de $(\tilde{P})$. Noteu que si la constricció es satisfà
estrictament, és a dir, $A\vb{x}=\vb{b}$, llavors $b-A\vb{x} = 0$, i per tant $z
= \tilde{z}(\vb{\lambda})$.

Podem intuir que el valor de $\tilde{z}(\vb{\lambda})$ sempre serà menor o igual
que $z$, ja que el problema $(\tilde{P})$ és una versió més ``flexible'' de
$(P)$---el conjunt de solucions factibles és molt més ampli---i per tant es
podrà aconseguir un mínim més petit (o almenys igual) que a $(P)$; i,
efectivament, per qualsevol $x^\ast$ solució factible de $(P)$,
\[
\tilde{z}^\ast(\vb{\lambda}) \defeq \min\limits_{\vb{x\ge 0}}\{\vb{c}'\vb{x} +
\vb{\lambda}'(\vb{b}-A\vb{x})\} \ \le\  \vb{c}'\vb{x^\ast} +
\vb{\lambda}'(\vb{b}-A\vb{x^\ast}) = \vb{c}'\vb{x^\ast}\,.
\]
ja que $\quad\min A \le a\quad$ per qualsevol conjunt $A$ i element $a\in A$.

Per tant, si trobem $\vb{\lambda}$ tal que $\tilde{z}^\ast(\vb{\lambda})$ sigui
màxim trobarem la fita inferior del valor òptim de la f.o. de $(P)$ més ajustada
(la més ``propera''). Podem expressar això com el problema de PL
\begin{equation}\label{eq:dual}
(D)
\begin{cases}
\begin{aligned}
\max_{\vb{\lambda}}		&&& \tilde{z}^\ast(\vb{\lambda})\\
\text{s.a.}				&&& \text{\small\textit{(cap constricció)}}\\
\end{aligned}
\end{cases}\
\end{equation}
és una versió ``primitiva'' del que anomenem \textit{problema dual} de $(P)$. De
fet, l'interès principal del dual és, com veurem més endavant al
\hyperref[teo:dualitat-fort]{teorema fort de dualitat}, que el valor en l'òptim
de $(D)$ coincideix amb el valor en l'òptim de $(P)$---és a dir, que la ``fita
inferior més ajustada'' de l'òptim de $(P)$ coincideix exactament amb aquest.

Segons la definició de $\tilde{z}^\ast(\vb{\lambda})$,
%
\begin{equation}\label{eq:dual-of}
\begin{gathered}
\tilde{z}^\ast(\vb{\lambda})
= \min_{\vb{x}\ge 0} \{\vb{c}'\vb{x}+\vb{\lambda}'(\vb{b}-A\vb{x})\}
= \vb{\lambda}'\vb{b}+
\min_{\vb{x}\ge0}\{\vb{c}'\vb{x}-\vb{\lambda}'A\vb{x}\}=\\
=  \vb{\lambda}'\vb{b}+ \min_{\vb{x}\ge0}\{(\vb{c}'-\vb{\lambda}'A)\vb{x}\}\,.
\end{gathered}
\end{equation}
%
Noteu que si \[\vb{c}'-\vb{\lambda}'A \le \vb{0}\,,\] tenint en compte que
$\vb{x}\ge\vb{0}$, llavors podem fer $(\vb{c}'-\vb{\lambda}'A)\vb{x}$ tant
``negatiu'' com vulguem, i per tant no és un cas que ens interessi pel problema
$(D)$---que és un problema de maximització de $\tilde{z}^\ast(\vb{\lambda})$. En
canvi, si \[\vb{c}'-\vb{\lambda}'A \ge \vb{0}\,,\] llavors
$\min_{\vb{x}\ge0}\{(\vb{c}'-\vb{\lambda}'A)\vb{x}\}$ és 0. Per tant només hem
de tenir en compte els casos en què $\vb{\lambda}'A \le \vb{c}'$.  Podem fer un
raonament similar pels problemes on $\vb{x}\le 0$ o on $\vb{x}$ és
lliure---vegeu la taula \ref{tab:dual}. En síntesi, continuant a partir de
\eqref{eq:dual-of}, el problema $(D)$ que hem enunciat a \eqref{eq:dual} es pot
reduir al següent:
\[
(D)
\begin{cases}
\begin{aligned}
\max_{\vb{\lambda}}		&&& \vb{\lambda}'\vb{b}\\
\text{s.a.}				&&& \vb{\lambda}'A \le \vb{c}'\\
\end{aligned}
\end{cases}\ .
\]

Ara continuem amb la definició formal de problema dual.

\begin{defi}[Problema Dual]
	Donat un problema primal $(P)$, el problema dual corresponent (denotat per
	$(D)$) es defineix segons la taula \ref{tab:dual}.
\end{defi}


\begin{table}[ht]
	\centering
	\setlength{\tabcolsep}{6ex}
	\renewcommand{\arraystretch}{1.5}
	\[
	\begin{array}{ll|ll}
	\multicolumn{2}{c|}{(P)}				
	&\multicolumn{2}{c}{(D)} \\
	%
	\hline
	\min 	&\vb{c}'\vb{x}
	&\max 	&\vb{\lambda}'\vb{b}\\
	%
	\hline
	\text{s.a.}\quad	&\vb{a_i}'\vb{x} \ge \vb{b}
	&\text{s.a.}\quad	&{\lambda_i} \ge 0\\
	%
	&\vb{a_i}'\vb{x} \le \vb{b}
	&					&{\lambda_i} \le 0\\
	%
	&\vb{a_i}'\vb{x} = \vb{b}
	&					&\lambda_i \text{ lliure}\\
	%
	\hline
	\text{s.a.}\quad	&x_j \ge 0
	&\text{s.a.}\quad	&\vb{\lambda}'A_j \le \vb{c}\\
	%
	&x_j \le 0
	&					&\vb{\lambda}'A_j \ge \vb{c}\\
	%
	&x_j \text{ lliure}
	&					&\vb{\lambda}'A_j = \vb{c}\\
	\end{array}
	\]
	\caption{Taula de transformació d'un problema primal al seu
		dual.}\label{tab:dual}
\end{table}


\begin{prop}[Equivalència de duals]
	Donat un problema de PL qualsevol $(P)$, són equivalents:
	\begin{enumerate}[(i)]
		\item El dual de $(P)$---que denotarem $(D)$
		\item El dual de $(P_e)$---que denotarem $(D^e)$
		\item La forma estàndard de $(D)$---que denotarem $(D_e)$
		\item La forma estàndard de $(D^e)$---que denotarem $(D^e_e)$
	\end{enumerate}
\end{prop}

\section{Propietats del dual}
\begin{teo}[Teorema feble de dualitat]\label{teo:dualitat-feble}
	Per tota solució factible $\vb{x}$ d'un problema primal $\left(P\right)$ i
	tota solució factible $\vb{\lambda}$ del dual $\left(D\right)$ associat
	\[ \vb{\lambda}'\vb{b}\leq \vb{c}'\vb{x}\,. \]
	\begin{proof}
		Sigui $m\times n$ l'ordre de la matriu de restriccions $A$ al problema
		primal. Definim, per cada $i\in\{1,\ldots,m\}$ (per cada restricció de 
		$(P)$) i
		per cada $j\in\{1,\ldots,n\}$ (per cada variable de $(P)$), 
		respectivament, 
		\begin{align*}
		u_i &\defeq {{\lambda}}_i(\vb{\vb{a_i}}'\vb{\vb{x}}-b_i)\\
		v_j &\defeq (c_j - \vb{\vb{\lambda}}'A_j)x_j\,.
		\end{align*}
		
		Primerament, veiem que $u_i$ i $v_j$ sempre són positius. Si
		$\vb{\vb{a_i}}'\vb{\vb{x}}-b_i \ge 0$, llavors la $i$-èsima restricció 
		de $(P)$
		serà de la forma \[\vb{\vb{a_i}}'\vb{\vb{x}} \ge b_i\,,\] i per tant la
		corresponent variable dual $\lambda_i$ complirà $\lambda_i \ge 0$. 
		D'altra
		banda, si $\vb{\vb{a_i}}'\vb{\vb{x}}_i-b_i \le 0$ la variable dual 
		$\lambda_i$
		serà negativa. Finalment, si $\vb{\vb{a_i}}'\vb{\vb{x}}-b_i = 0$, es té 
		que $u_i
		= 0$ independentment de $\lambda_i$. Per tant $u_i \ge 0$ per tot $i$. 
		Es pot
		fer un raonament similar per $v_j$.
		
		Ara, observem que
		\begin{alignat}{2}{}
		\sum_{i = 1}^{m} u_i &= \sum_{i = 1}^{m}
		\lambda_i\vb{\vb{a_i}}'\vb{\vb{x}}-\lambda_ib_i &=
		\vb{\vb{\lambda}}'A\vb{\vb{x}} - \vb{\vb{\lambda}}'\vb{b}\,, 
		\label{eq:ui}\\
		\sum_{j = 1}^{n} v_j &= \sum_{j = 1}^{n} c_jx_j - 
		\vb{\vb{\lambda}}'A_jx_j
		&= \vb{\vb{c}}'\vb{\vb{x}} - \vb{\vb{\lambda}}'A\vb{\vb{x}}\,. 
		\label{eq:vj}
		\end{alignat}
		Si sumem \eqref{eq:ui} i \eqref{eq:vj} i tenim en compte que $u_i \ge 
		0$ i
		$v_j \ge 0$ per tot $i$ i $j$, obtenim
		\begin{equation}\label{eq:sum-ui-vj}
		\sum_{i = 1}^{m} u_i + \sum_{j = 1}^{n} v_i=
		\vb{\vb{c}}'\vb{\vb{x}}-\vb{\vb{\lambda}}'\vb{b} \ge 0\,
		\end{equation}
		i per tant \[\vb{\vb{c}}'\vb{\vb{x}} \ge \vb{\vb{\lambda}}'\vb{b}\,.\]
	\end{proof}
\end{teo}
\begin{col}[Corol\lgem laris del teorema feble de dualitat
	(\ref{teo:dualitat-feble})]\label{col:feb}
	Per qualsevol problema primal $(P)$ i el seu dual $(D)$: 
	\begin{enumerate}[i)]
		\item $\left(P\right)$ il\lgem limitat $\implies \left(D\right)$ 
		infactible.
		\item $\left(D\right)$ il\lgem limitat $\implies \left(P\right)$ 
		infactible.
		\item Siguin $\vb{\vb{x}}^\ast \in P$ i $\vb{\vb{\lambda}}^\ast \in D$
		solucions factibles del primal i del dual respectivament. Si
		${\vb{\vb{\lambda}}^\ast}'\vb{b} = \vb{\vb{c}}'\vb{\vb{x}}^\ast$ llavors
		$\vb{\vb{\lambda}}^\ast$ i $\vb{\vb{x}}^\ast$ són solucions òptimes pel 
		problema
		dual i primal respectivament.
	\end{enumerate}
	\begin{proof}
		\begin{enumerate}[i)]
			\item[]
			\item Sigui $\vb{\vb{x}}$ solució factible de $(P)$ i 
			$\vb{\vb{\lambda}}$
			solució factible de $(D)$. Per hipòtesi $(P)$ és il\lgem limitat i 
			per 
			tant \[\forall
			k \in \real\ \exists \vb{x}\in P\qcolon \vb{c}'\vb{x} < k\,. \] 
			Suposem que
			existís una solució dual $\vb{\lambda}$ factible. Llavors, pel
			\hyperref[teo:dualitat-feble]{teorema feble de dualitat}, $\forall
			\vb{\vb{x}}\in P\colon \vb{\vb{\lambda}}'\vb{b} \le 
			\vb{\vb{c}}'\vb{\vb{x}}$.
			D'altra banda, per la proposició anterior, tenim que $\exists 
			\vb{\vb{x}}\in
			P\colon \vb{\vb{\lambda}}'\vb{b} > 
			\vb{\vb{c}}'\vb{\vb{x}}$---prenent el valor
			$\vb{\vb{\lambda}}'\vb{b}$ per $k\in\real$; contradicció. Per tant 
			$(D)$ és
			infactible.
			%
			\item Sigui $\vb{\vb{\lambda}}$ solució factible de $(D)$ i 
			$\vb{\vb{x}}$
			solució factible de $(P)$. Per hipòtesi $(D)$ és il\lgem limitat i 
			per 
			tant \[\forall
			k \in \real\ \exists \vb{\vb{x}}\in P\qcolon 
			\vb{\vb{\lambda}}'\vb{b} > k\,. \]
			Suposem que existís una solució primal $\vb{\vb{x}}$ factible. 
			Llavors, pel
			{teorema feble de dualitat}, $\forall \vb{\vb{\lambda}}\in D\colon
			\vb{\vb{\lambda}}'\vb{b} \le \vb{\vb{c}}'\vb{\vb{x}}$. D'altra 
			banda, per la
			proposició anterior, tenim que $\exists \vb{\lambda}\in D\colon
			\vb{\lambda}'\vb{b} > \vb{c}'\vb{x}\,$; contradicció. Per tant 
			$(P)$ és
			infactible.
			%
			\item Pel teorema feble de dualitat, tenim que $\forall 
			\vb{\lambda} \in D:
			\vb{\lambda}'\vb{b} \le \vb{c}'{\vb{x}^\ast}$, i per tant, aplicant 
			la hipòtesi,
			\[\forall \vb{\lambda} \in D: \vb{\lambda} '\vb{b} \le
			{\vb{\lambda}^\ast}'\vb{b}\,;\] ergo, $\vb{\lambda}^\ast$ és 
			l'òptim del dual
			(ja que és un problema de maximització). D'altra banda, novament 
			pel teorema
			anterior, $\forall \vb{x} \in P: {\vb{\lambda}^\ast}'\vb{b} \le
			\vb{c}'\vb{x}\,$, cosa que implica per hipòtesi que \[\forall 
			\vb{x} \in D:
			\vb{c}'{\vb{x}^\ast} \le \vb{c}'\vb{x}\] i per tant $\vb{x}^\ast$ 
			és un òptim.
		\end{enumerate}
	\end{proof}
\end{col}

\begin{teo}[Teorema fort de dualitat]\label{teo:dualitat-fort}
	Sigui $\left(P\right)$ un problema primal i $\left(D\right)$ el seu dual.
	Llavors $(D)$ té òptim si i només si $(P)$ té òptim. A més, si en tenen, el
	valor de la funció objectiu en l'òptim coincideix.
	\begin{proof}
		Considerem la forma estàndard del problema primal $(P)$,
		
		\[
		(P_e)
		\begin{cases}
		\begin{aligned}
		\min 			&&& \vb{c}'\vb{x}\\
		\text{s.a.}		&&& A\vb{x} = \vb{b}\\
		&&& \vb{x}\ge0
		\end{aligned}
		\end{cases}\ ,
		\]
		on la matriu $A$ és d'ordre $m\times n$, i el dual corresponent (a 
		$P_e$):
		\[
		(D^e)
		\begin{cases}
		\begin{aligned}
		\max 			&&& \vb{\lambda}'\vb{b}\\
		\text{s.a.}		&&& \vb{\lambda}'A \le \vb{c}'
		\end{aligned}
		\end{cases}\ ,
		\]
		
		Primer demostrarem que si $(P_e)$ té òptim, $(D^e)$ també. Si $(P_e)$ té
		òptim, l'algorisme del símplex primal aplicat a $(P_e)$ terminarà amb 
		una SBF
		òptima $\vb{x}^\ast$ de $(P_e)$, amb matriu bàsica $B$, tal que el 
		vector de
		costos reduïts sigui no negatiu: \[\mathbf{r}' = \vb{c_\N} '- 
		\vb{c_\B}' B^{-1}
		A_\N \ge 0\,.\] Sigui $\vb{\lambda}^\ast$ el vector definit per
		${\vb{\lambda}^\ast}'\defeq \vb{c_\B}'B^{-1}$. Demostrarem que
		$\vb{\lambda}^\ast$ és solució òptima de $(D^e)$.
		
		Primer hem de comprovar si és factible. Efectivament, 
		\[
		{\vb{\lambda}^\ast}'A = 
		\vb{c_\B}' B^{-1}
		\begin{bmatrix}
		B 	\vert 	A_\N
		\end{bmatrix} =
		\begin{bmatrix}
		\vb{c_\B}' \Id & \vb{c_\B} B^{-1} A_\N    			
		\end{bmatrix}
		\le
		\begin{bmatrix}
		\vb{c_\B}' 	&\vb{c_\N}'
		\end{bmatrix}
		= \vb{c}'\,.
		\]
		D'altra banda, \[{\vb{\lambda}^\ast}'\vb{b} = \vb{c_\B}' B^{-1} \vb{b} =
		\vb{c_\B}' {\vb{x}^\ast}_\B = \vb{c}'{\vb{x}^\ast}\,. \] Aplicant el 
		corol\lgem lari
		\ref{col:feb}, deduïm que $\vb{\lambda}^\ast$ és solució òptima de 
		$(D^e)$.
		Queda demostrat que si $(P_e)$ té solució òptima existeix una solució 
		òptima de
		$(D^e)$ (i pel corol\lgem lari mencionat produeixen el mateix valor de 
		la 
		f.o.).
		
		Finalment, podem afirmar que aquest resultat és vàlid per els problemes 
		que
		no estan en forma estàndard, ja que el problema $(P)$ i el seu 
		estàndard són
		equivalents---sempre es pot passar d'una solució de l'un a una solució 
		de
		l'altre, i el valor de la f.o. en l'òptim és el mateix---, i el dual de
		l'estàndard---$(D^e)$---és equivalent al dual de $(P)$.
	\end{proof}
\end{teo}

\begin{col}[Coro\lgem ari del teorema fort de dualitat 
(\ref{teo:dualitat-fort})]
	Si $\left(P\right)_e$ és un problema factible i amb matriu de restriccions
	$A$ de rang complet, i té una SBF òptima $\vb{x}^\ast$ amb matriu bàsica 
	$B$,
	llavors $\left(D\right)$ té solució factible òptima a $\vb{\lambda}' =
	\vb{c_\B}'B^{-1}$.
\end{col}

Noteu que, donada una SBF òptima primal $\vb{x^\ast} = B^{-1}\vb{b}$, la
implicació ``$\vb{\lambda^\ast}'= \vb{c_{\B}}'B^{-1} \implies \vb{\lambda^\ast}
\text{ òptima dual}$'', que hem trobat durant la demostració del teorema
\ref{teo:dualitat-fort}, és unidireccional: no tota solució òptima dual és de la
forma $\vb{\lambda}' = \vb{c_{\B}}'B^{-1}$, on $B$ és la matriu bàsica de la SBF
òptima primal. 

D'altra banda, però, donada una SB primal qualsevol $\vb{x}$ amb $\vb{r} \ge
\vb{0}$---és a dir, una \hyperref[defi:SBFD]{SBFD}---, sent $B$ la matriu bàsica
de $\vb{x}$, llavors el vector $\vb{\lambda}$ definit per $\vb{\lambda}' \defeq
\vb{c_\B}'B^{-1}$ és una solució factible del problema dual, ja que
\[
\vb{r}'\ge \vb{0} \iff \vb{c}' - \vb{c_\N}'B^{-1}A \ge \vb{0} \iff \vb{c}' -
\vb{\lambda}'A \ge \vb{0}  \iff \vb{\lambda}'A \le \vb{c}'\,.
\]
Utilitzarem aquest fet durant el desenvolupament del símplex dual per
``aprofitar'' l'estructura del símplex primal; a més, pel que hem vist al
teorema \ref{teo:dualitat-fort}, si en l'algorisme del símplex dual anem
canviant de solució dual $\vb{\lambda}'$ que sigui de la forma $\vb{c_\B}'
B^{-1}$, llavors, quan arribem a la solució òptima $\lambda^\ast$, haurem trobat
la SBF òptima primal $\vb{x^\ast} = B^{-1}b$.

\begin{teo}[Teorema de folga complementària]\label{teo:folga-complement}
	Siguin $\vb{x}, \vb{\lambda}$ solucions factibles de $\left(P\right)$ i
	$\left(D\right)$, respectivament. Llavors $\vb{x}$ i $\vb{\lambda}$ són
	solucions òptimes si i només si
	\begin{alignat*}{2}{}
	u_i &\defeq \lambda_i \left(\vb{a_i}'\vb{x} - b_i\right) &= 0 &\quad
	\forall i \in \left\{1, \dots, m\right\} \\
	v_j &\defeq \left(c_j - \vb{\lambda}'A_j\right) x_j &= 0 &\quad \forall
	j \in \left\{1, \dots, n\right\}\,.
	\end{alignat*}
	\begin{proof}
		Demostrem primer que si $u_i = 0 \land v_j = 0$ per tot $i$ i tot $j$,
		llavors $\vb{x}$ i $\vb{\lambda}$ són òptimes. Com hem vist a la 
		demostració del
		teorema feble de dualitat---l'equació \eqref{eq:sum-ui-vj}---, 
		\begin{equation*}
		\sum_{i = 1}^{m} u_i + \sum_{j = 1}^{n} v_j = \vb{c}'\vb{x} -
		\vb{\lambda}'\vb{b}\,.
		\end{equation*}
		Sabent que $u_i = 0$ per tot $i$ i $v_j = 0$ per tot $j$, tenim que
		\[\sum_{i = 1}^{m} u_i + \sum_{j = 1}^{n} v_j = 0 = \vb{c}'\vb{x} -
		\vb{\lambda}'\vb{b} \iff \vb{c}'\vb{x} = \vb{\lambda}'\vb{b}\,,  \] i 
		pel
		corol\lgem lari \ref{col:feb} del teorema feble, $\vb{x}$ i 
		$\vb{\lambda}$ 
		són
		solucions òptimes per al problema respectiu.
		
		Per demostrar la implicació recíproca, només cal notar que, com hem
		demostrat a la demostració del corol\lgem lari \ref{col:feb}, $u_i$ i 
		$v_j$ 
		sempre
		són no negatius, i per tant l'única manera de que es compleixi 
		\[\sum_{i=1}^{m}
		u_i + \sum_{j=1}^{n} v_j = 0 \] és que $u_i = 0$ per tot $i$ i $v_j = 
		0$ per tot
		$j$. Per tant només cal recórrer en sentit recíproc totes les 
		implicacions de la
		demostració que acabem de fer.
	\end{proof}
\end{teo}

Noteu que el teorema \ref{teo:folga-complement} s'anomena de ``folga
complementària'' perquè les expressions \[(\vb{a_i}'\vb{x}-b_i) \quad
\text{i}\quad (c_j-\vb{\lambda}'A_j) \] es poden interpretar com les ``folgues''
associades a les restriccions primals i duals \[A\vb{x} \lesseqgtr \vb{b} \quad
\text{i}\quad \vb{\lambda}'A \lesseqgtr \vb{c}\,,\] respectivament;
``complementària'' perquè la ``folga'' d'un problema està associada amb una
component del vector variable de l'altre problema---$(\vb{a_i}'\vb{x}-b_i)$ amb
$\lambda_i$ i $(c_j-\vb{\lambda}'A_j)$ amb $x_j$.

\begin{defi}[Solució bàsica factible dual]\label{defi:SBFD}
	Sigui $\left(P\right)$ un problema de PL i sigui $(P_e)$ la seva forma
	estàndard. Llavors, direm que tota solució bàsica $\vb{x}$ de 
	$\left(P\right)_e$
	tal que $\mathbf{r} \geq \vb{0}$ és una solució bàsica factible dual (SBFD).
\end{defi}

Noteu que la definició \ref{defi:SBFD} deriva directament dels resultats que hem
obtingut durant la demostració del teorema fort de dualitat
(\ref{teo:dualitat-fort}): extrapolant l'argument que vam utilitzar, per tota
solució bàsica $\vb{x}$---d'una forma estàndard $(P_e)$---amb matriu bàsica $B$
i un vector de costos reduïts no negatius, el vector $\vb{\lambda}' \defeq
\vb{c_\B}' B^{-1}$ és una solució factible del problema dual corresponent.

Observeu també que l'adjectiu ``factible'' en la definició de SBFD es refereix a
una solució bàsica de $(P_e)$, que \textbf{pot no ser factible} per a aquest (és
a dir, que $\vb{x}\notin P_e$), que es \textit{correspon}, però, a una
\textbf{solució factible del problema dual} $(D^e)$ corresponent: $\vb{\lambda}
\defeq (\vb{c_\B}' B^{-1})' \in D^e$. Potser ajuda pensar en ``Solució bàsica
factible dual'' en dos blocs: ``Solució bàsica'' i ``factible dual''.

Per últim, noteu que si una solució bàsica $\vb{x}$ de $(P_e)$ és factible dual
(per tant és SBFD) i és alhora factible primal ($\vb{x} \in P_e \Leftrightarrow
A\vb{x} = \vb{b}$), llavors és solució òptima de $(P_e)\,$, ja que és una
solució factible amb $\vb{r}\ge \vb{0}$.


\section{Símplex dual}

Per desenvolupar l'algorisme del símplex dual, partirem de l'estructura general 
del símplex primal. 


\subsection{Forma estàndard del problema dual}
Per poder emprar solucions bàsiques, hem de poder definir una forma estàndard 
del dual. Partirem amb la següent definició provisional.

\begin{defi*}[Forma estàndard del dual]\label{defi:std-dual}
	Sigui $(P_e)$ un problema en forma estàndard i sigui $(D^e)$ el seu dual:
	\[
	(P_e)
	\begin{cases}
	\begin{aligned}
	\min 			&&& \vb{c}'\vb{x}\\
	\text{s.a.}		&&& A\vb{x} = \vb{b}\\
	&&& \vb{x}\ge0
	\end{aligned}
	\end{cases}\quad\rightsquigarrow\quad
	%
	(D^e)
	\begin{cases}
	\begin{aligned}
	\max 			&&& \vb{\lambda}'\vb{b}\\
	\text{s.a.}		&&& A'\vb{\lambda} \le \vb{c}
	\end{aligned}
	\end{cases}\,.
	\]
	Definirem la forma estàndard del dual (diferent de la forma estàndard
	habitual), i la notarem amb $(D^e_{\tilde{e}})$ així:
	\[
	{\color[gray]{0.6}(D^e)\rightsquigarrow\;}(D^e_{\tilde{e}})
	\begin{cases}
	\begin{aligned}
	\min 			&&& -\vb{\lambda}'\vb{b}\\
	\text{s.a.}		&&& A'\vb{\lambda} +  \vb{\rho} = \vb{c}\\
	&&& \vb{\rho} \ge \vb{0}
	\end{aligned}
	\end{cases}\,,
	\]
	on $\vb{\rho} \in \real^n$ és un vector de variables de folga. Si definim 
	la matriu $A^D \in \matspace_{n\times(m+n)}(\real)$, el vector de variables 
	$\vb{y}\in \real^{m+n}$ i el vector de ``costos'' del dual 
	$\vb{g}\in\real^{m+n}$ com segueix, 
	\[
	A^D\defeq 
	\left[\begin{array}{{@{}c|c@{}}}
	A' & \Id_n
	\end{array}\right]\,, \quad
	\vb{y} \defeq
	\left[\begin{array}{@{}c@{}}
	\vb{\lambda}\\
	\hline
	\vb{\rho}
	\end{array}\right]\,, \quad
	%\def
	\vb{g} \defeq
	\left[\begin{array}{@{}c@{}}
	\vb{-b}\\
	\hline
	\vb{0}
	\end{array}\right]\,,
	\]
	podem reescriure $(D^e_{\tilde{e}})$ com segueix:
	\begin{equation}\label{eq:std-dual}
	(D^e_{\tilde{e}})
	\begin{cases}
	\begin{aligned}
	\min 			&&& \vb{\lambda}'\vb{g}\\
	\text{s.a.}		&&& A^D\vb{y} = \vb{c}\\
	&&& \vb{\rho} \ge \vb{0}
	\end{aligned}
	\end{cases}\,.
	\end{equation}
\end{defi*}

Ara considerem una \hyperref[defi:SBFD]{SBFD} del problema $(P_e)$. Aquesta
tindrà associada una base $\B$ de variables bàsiques, i com s'ha vist durant
el desenvolupament del símplex primal, podem fer una partició de paràmetres i
variables segons si estan associades a un índex de $\B$ o no---per exemple,
partim la matriu $A$ en $\left[B\vert A_\N\right]$. Podem aplicar aquesta
partició al problema dual també. Donada una SBFD de $(P_e)$ amb base $\B$, les
constriccions del problema dual en forma estàndard $(D^e_{\tilde{e}})$
contingudes a $A^D\vb{y} = \vb{c}$ es particionen així,
\begin{multline}\label{eq:partition}
A^D\vb{y} =
\left[\begin{array}{{@{}c|c@{}}}
A' & \Id_n
\end{array}\right]
\left[\begin{array}{{@{}c@{}}}
\vb{\lambda}\\
\hline
\vb{\rho}
\end{array}\right] 
=
\left[\begin{array}{{@{}c|cc@{}}}
B'		& \Id_m 		  &	\varnothing	\\
\hline
A_\N' 	& \varnothing & \Id_{n-m}
\end{array}\right]
\left[\begin{array}{{@{}c@{}}}
\vb{\lambda}\\
\hline
\vb{\rho}
\end{array}\right] 
=\\
=
\left[\begin{array}{{@{}c|c|c@{}}}
B'		& \Id_m 		  &	\varnothing	\\
\hline
A_\N' 	& \varnothing & \Id_{n-m}
\end{array}\right]
\left[\begin{array}{{@{}c@{}}}
\vb{\lambda}\\
\hline
\vb{\rho_\B}\\
\hline
\vb{\rho_\N}
\end{array}\right]
\doteq \vb{c}
=
\left[\begin{array}{{@{}c@{}}}
\vb{c_\B}\\
\hline
\vb{c_\N}
\end{array}\right]\,,
\end{multline}
on $\varnothing$ indica una submatriu nul\lgem la, $\vb{\rho_\B}$ és un vector
d'ordre $m$ que conté totes les variables de $\vb{\rho}$ associades a índexos de
la base $\B$, i $\vb{\rho_\N}$ conté la resta (variables de $\vb{\rho}$
associades a un índex de $\N$). Finalment, podem reescriure \eqref{eq:partition}
com
\begin{equation}\label{eq:partition2}
A^D\vb{y} = \left[\begin{array}{{@{}c|c@{}}}
B'	  &	\varnothing	\\
\hline
A_\N' & \Id_{n-m}
\end{array}\right]
%
\left[\begin{array}{{@{}c@{}}}
\vb{\lambda}\\
\hline
\vb{\rho_\N}
\end{array}\right]
%
+
%
\left[\begin{array}{{@{}c@{}}}
\Id_m\\
\hline
\varnothing
\end{array}\right]
%
\vb{\rho_\B}
%
=
\left[\begin{array}{{@{}c@{}}}
\vb{c_\B}\\
\hline
\vb{c_\N}
\end{array}\right]\,.
\end{equation}

Ara bé, si restringim els vectors $\vb{\lambda}$ als que tenen la forma 
$\vb{\lambda}'=\vb{c_\B}'B^{-1}$ per una certa matriu bàsica (primal) $B$, 
llavors concloem el que s'enuncia a la proposició \ref{prop:equiv-rho-r}.


\begin{prop}\label{prop:equiv-rho-r}
	Donat un problema primal en forma estàndard $(P_e)$ i el seu dual
	$(D^e_{\tilde{e}})$ (segons la definició \ref{defi:std-dual}), si considerem
	només els vectors duals de la forma $\vb{\lambda}'=\vb{c_\B}'B^{-1}$ per una
	certa SBFD  $\vb{x}$ de $(P_e)$ amb matriu bàsica $B$, llavors $\vb{\rho} =
	\vb{r}$, on $\vb{r}$ és el vector de costos reduïts associat a $\vb{x}$, i 
	per
	tant
	\[
	(D^e_{\tilde{e}})
	\begin{cases}
	\begin{aligned}
	\min 			&&& -\vb{\lambda}'\vb{b}\\
	\text{s.a.}		&&& A'\vb{\lambda} +  \vb{r} = \vb{c}\\
	&&& \vb{\rho} \ge \vb{0}
	\end{aligned}
	\end{cases}\,, 
	%
	\quad \vb{y} =
	\left[
	\begin{array}{@{}c@{}}
	\vb{\lambda}\\
	\hline
	\vb{r_\B}\\
	\hline
	\vb{r_\N}
	\end{array}
	\right]\,.
	\]
	
	\begin{proof}
		Sigui $\vb{x}$ una SBFD de $(P)$ amb matriu bàsica $B$, i sigui 
		$\vb{\lambda}'
		= \vb{c_\B}'B^{-1}$. Per la \hyperref[defi:SBFD]{definició de SBFD}, el 
		vector
		de costos reduïts $\vb{r}$ associat a $\vb{x}$ és no negatiu. A més,
		\[
		\vb{\lambda}'A = \vb{c_\B}'B^{-1}A
		\] 
		i per tant
		\[
		\vb{\lambda}'A + \vb{r}' = \vb{c_\B}'B^{-1}A + (\vb{c}' - 
		\vb{c_\B}B^{-1}A) =
		\vb{c}'\,;
		\]
		és a dir, que les folgues $\vb{\rho}$ de la restricció $\vb{\lambda}'A 
		+ 
		\vb{\rho}' = \vb{c}'$ coincideixen amb $\vb{r}$.
	\end{proof}
\end{prop}

\subsection{SBF del poliedre dual}
Ara, per poder utilitzar l'estructura del símplex primal en el desenvolupament
del símplex dual, hem de trobar una manera de treballar amb solucions bàsiques
factibles del poliedre $D^e_{\tilde{e}}$. Com veurem a la proposició
\ref{prop:SD-SDBF}, les solucions duals definides a \ref{defi:SD} són SBF de
$D^e_{\tilde{e}}$. 

\begin{defi}[Solució dual associada a una SBFD]\label{defi:SD}
	Donat un problema primal en forma estàndard $(P_e)$ i una
	\hyperref[defi:SBFD]{SBFD} $\vb{x}$ d'aquest, amb matriu bàsica $B$ i 
	vector de
	costos reduïts $\vb{r}$, anomenarem ``solució dual associada a la SBFD
	$\vb{x}$'' al vector $\vb{y}$ tal que
	\[
	\vb{y} = 
	\left[\begin{array}{@{}c@{}}
	\vb{y_{\B^D}}\\
	\hline
	\vb{y_{\N^D}}
	\end{array}\right],\quad\text{on }
	%
	\vb{y_{\B^D}}=
	\left[\begin{array}{@{}c@{}}
	\vb{\lambda}\\
	\hline
	\vb{r_\N}
	\end{array}\right],\ 
	%
	\vb{y_{\N^D}} = \vb{r_\B} \doteq \vb{0}\,.	
	\]
	La última igualtat ($\vb{r_\B}=\vb{0}$) es satisfà, per definició de 
	$\vb{r}$,
	per qualsevol SB de $(P_e)$.
\end{defi}


Podem aplicar la mateixa partició al vector $\vb{g}$,
\[
\vb{g_{\B^D}} = 
\left[\begin{array}{@{}c@{}}
\vb{-b}\\
\hline
\vb{0}\\
\end{array}\right], \quad
%
\vb{g_{\N^D}} = \vb{0}\,,
\]
on $\vb{g_{\B^D}}\in \real^{n}$, i $\vb{g_{\N^D}}\in \real^m$.

\begin{prop}\label{prop:SD-SDBF}
	Donada una SBFD $\vb{x}$ del problema primal $(P_e)$, la solució dual 
	$\vb{y}$ associada a $\vb{x}$ és una SBF del poliedre $D^e_{\tilde{e}}$.
	\begin{proof}
		Emprant la definició \ref{defi:SB-alt} de solució bàsica, una SB és 
		tota solució factible amb un nombre de constriccions actives linealment 
		independents igual al nombre de variables (la dimensió del poliedre). 
		En el cas del poliedre $D^e_{\tilde{e}}$, aquest nombre és $m+n$ (ja 
		que $\vb{y}' = \left[\vb{\lambda}'\vert \vb{r}'\right]\in \real^{m+n}$).
		
		A partir de l'equació \eqref{eq:partition2}, tenint en compte que 
		$\vb{r_\B}=\vb{0}$, tenim que
		\[
		\left[\begin{array}{{@{}c|c@{}}}
		B'	  &	\varnothing	\\
		\hline
		A_\N' & \Id_{n-m}
		\end{array}\right]
		%
		\vb{y_{\N^D}} = \vb{c}\,.
		\]
		Analitzem la següent matriu:
		\[
		B_D \defeq \left[
		\begin{array}{@{}c|c@{}}
		B'	  &	\varnothing	\\
		\hline
		A_\N' & \Id_{n-m}
		\end{array}
		\right]\,.
		\]
		Les últimes $n-m$ files de $B_D$ són l.i. entre sí, ja que les darreres 
		$n-m$ columnes d'aquestes formen la matriu $\Id_{n-m}$. D'altra banda, 
		la matriu $B$ és no singular per definició, i per tant les $m$ primeres 
		files de $B_D$ són l.i. entre sí. Finalment, les $m$ primeres files i 
		les $n-m$ darreres són l.i. entre elles, ja que la submatriu formada 
		per les primeres $m$ files i les darreres $n-m$ columnes és nul\lgem 
		la, 
		mentre que la matriu formada per les $n-m$ darreres files i les $n-m$ 
		darreres columnes és $\Id_{n-m}$.
		
		Per tant, el nombre total de files l.i. de $B_D$ és $m+(n-m) = n$. És a 
		dir, que el vector $\vb{y_{\N^D}}$ està subjecte a $n$ restriccions 
		actives l.i. (ja que $B_D \vb{y_{\N^D}} = \vb{c}$). D'altra banda, 
		$\vb{r_\B} = 0$, cosa que afegeix $m$ restriccions actives l.i. 
		addicionals (un subconjunt de les restriccions $\vb{r}\ge\vb{0}$). 
		Unint aquests dos fets, tenim que $\vb{y}$ està subjecte a $n+m$ 
		restriccions actives l.i., i per tant $\vb{y}$ és una solució bàsica (i 
		factible per construcció) del poliedre $D^e_{\tilde{e}}$.
	\end{proof}
\end{prop}

Noteu que ara, havent utilitzat la definició alternativa de SB 
(\ref{defi:SB-alt}), hem de redefinir a part els conceptes de matriu bàsica, 
variables bàsiques i no bàsiques pel problema dual. Tot i així, les definicions 
que farem a continuació (\ref{defi:mbd} i \ref{defi:vd}) coincideixen 
idènticament amb les definicions que havíem fet a \ref{defi:SB}, i impliquen 
les mateixes propietats.

\begin{defi*}[Matriu bàsica d'una SBF dual]\label{defi:mbd}
	Donada una solució dual $\vb{y}$ associada a una SBFD $\vb{x}$ del problema 
	primal $(P_e)$ amb matriu bàsica $B$, definirem la ``matriu bàsica de 
	$\vb{y}$'' com
	\[
	B_D \defeq \left[
	\begin{array}{@{}c|c@{}}
	B'	  &	\varnothing	\\
	\hline
	A_\N' & \Id_{n-m}
	\end{array}
	\right]
	\in \matspace_{n\times n}(\real)  \,.
	\]
\end{defi*}

\begin{defi*}[Matriu no bàsica d'una SBF dual]
	Donada una solució dual $\vb{y}$ associada a una SBFD $\vb{x}$ del problema 
	primal $(P_e)$, definirem la ``matriu no bàsica de  $\vb{y}$'' com
	\[
	A^D_{\N^D} \defeq 
	\left[
	\begin{array}{@{}c@{}}
	\Id_m \\
	\hline
	\varnothing
	\end{array}
	\right]
	\in \matspace_{n\times m}(\real)\,.
	\]
\end{defi*}

\begin{defi*}[Partició en variables bàsiques i no bàsiques d'una SBF 
	dual]\label{defi:vd}
	Donada una solució dual associada a una SBFD $\vb{x}$ del problema primal 
	$(P_e)$ amb base $\B$, anomenarem ``variables bàsiques de $\vb{y}$'' a les 
	variables contingudes en el vector
	\[
	\vb{y_{\B^D}} \defeq
	\left[\begin{array}{@{}c@{}}
	\vb{\lambda}\\
	\hline
	\vb{r_\N}
	\end{array}\right]\,,
	\]
	i anomenarem ``variables no bàsiques de $\vb{y}$'' a les variables 
	contingudes en el vector
	\[
	\vb{y_{\N^D}} \defeq \vb{r_\B} \doteq \vb{0}\,.
	\]
	També definirem $\B^D \subseteq \{1,\ldots, m+n\}$ (tal que $|\B^D| = n$) 
	com el conjunt d'índexos associats a una variable bàsica (de 
	$\vb{y_{\B^D}}$) i l'anomenarem \textit{base dual}, i definirem $\N^D$ com 
	el complementari de $\B^D$ en $\{1,\ldots,m+n\}$ (de manera que $|\N^D|=m$).
\end{defi*}

\begin{prop}
	La matriu bàsica d'una SBF dual és no singular.
	\begin{proof}
		Com hem vist a la demostració de la proposició \ref{prop:SD-SDBF}, la 
		matriu $B_D$, que és d'ordre $n\times n$, té $n$ files l.i.; és 
		conseqüència directa que $B_D$ és no singular.
	\end{proof}
\end{prop}

\subsection{Costos reduïts, DBF i longitud de pas màxima en el problema dual}
Ara, per poder desenvolupar l'algorisme del símplex dual a partir del símplex 
primal, encara ens falta determinar els costos reduïts duals (que denotarem 
$\vb{r^D}$), les direccions bàsiques factibles, i la longitud de passa màxima 
pel problema dual.

\begin{prop}\label{prop:r-dual-x-primal}
	El vector de costos reduïts (associats a VNB duals) per una solució dual 
	associada a una SBFD $\vb{x}$ amb base $\B$ coincideix amb $\left[\vb{0} 
	\;|\; \vb{x_\B}' \right]' \doteq \left[\vb{0}' \;\vert\; \vb{x_\N}' 
	\;\vert\; \vb{x_\B}' \right]'$.
	\begin{proof}
		Pel que sabem del símplex primal, el vector de costos reduïts d'una SBF 
		$\vb{x}$, amb base $\B$ i matriu bàsica $B$, d'un problema en forma 
		estàndard
		\[
		(P_e)
		\begin{cases}
		\begin{aligned}
		\min 			&&& \vb{c}'\vb{x}\\
		\text{s.a.}		&&& A\vb{x} = \vb{b}\\
		&&& \vb{x}\ge0
		\end{aligned}
		\end{cases}\
		\]
		està definit per
		\[
		\vb{r}' \defeq \vb{c}' - \vb{c_\B}'B^{-1}A\,.
		\]
		Apliquem aquesta noció directament al problema dual definit a 
		\eqref{eq:std-dual}:
		\begin{multline}\label{eq:rd-dev}
		\vb{r^D}' = \vb{g}' - \vb{g_{\B^D}}'B_D^{-1}A^D = 
		\left[\vb{g_{\B^D}}' \;\vert\; \vb{g_{\N^D}}' \right] - 
		\left[-\vb{b} \;|\; 0\right] B_D^{-1}
		\left[\begin{array}{{@{}c|c@{}}}
		B_D & A^D_{\N^D}
		\end{array}\right] = \\
		%
		%
		= \left[\vb{g_{\B^D}}' \;\vert\; \vb{0}' \right] - 
		\left[\vb{g_{\B^D}}'\Id_n \;|\; -\vb{g_{\B^D}}' B_D^{-1} 
		A^D_{\N^D}\right]
		%
		= \left[\vb{0}' \;\vert\; -\vb{g_{\B^D}}' B_D^{-1} 
		A^D_{\N^D} \right]\,.
		\end{multline}
		Desenvolupem la darrera expressió:
		\begin{multline}\label{eq:rd-dev2}
		-\vb{g_{\B^D}}' B_D^{-1} 
		A^D_{\N^D} 
		= -\left[-\vb{b}' \;\vert\; \vb{0} \right]
		\left[\begin{array}{@{}c|c@{}}
		B'	  &	\varnothing	\\
		\hline
		A_\N' & \Id_{n-m}
		\end{array}\right]^{-1}
		\left[\begin{array}{@{}c@{}}
		\Id_m \\
		\hline
		\varnothing
		\end{array}\right] = \\
		%
		%
		= \left[\vb{b}' \;\vert\; \vb{0} \right]
		\left[\begin{array}{@{}c|c@{}}
		(B^{-1})'	  &	\varnothing	\\
		\hline
		-A_\N'( B^{-1})' & \Id_{n-m}
		\end{array}\right]
		\left[\begin{array}{@{}c@{}}
		\Id_m \\
		\hline
		\varnothing
		\end{array}\right]
		%
		= \left[\vb{b}' \;\vert\; \vb{0} \right]
		\left[\begin{array}{@{}c@{}}
		(B^{-1})' \\
		\hline
		-A_\N'( B^{-1})'
		\end{array}\right]
		\Id_m = \\
		%
		%
		= \vb{b}'(B^{-1})' = (B^{-1}\vb{b})' = \vb{x_\B}'\,.
		\end{multline}
		
		Per tant, unint \eqref{eq:rd-dev} i \eqref{eq:rd-dev2}, obtenim que
		\[
		\vb{r^D}' = \left[\vb{0}' \;\vert\; \vb{x_\B}' \right] \doteq 
		\left[\vb{0}' \;\vert\; \vb{x_\N}' 
		\;\vert\; \vb{x_\B}' \right]\,.
		\]
		Com pel problema primal, tenim que $\vb{r^D_{\B^D}} = \vb{0}$.
	\end{proof}
\end{prop}

De la proposició \ref{prop:r-dual-x-primal} deduïm, com era d'esperar a partir 
del teorema fort de dualitat, que la condició d'optimalitat pel problema dual 
($\vb{r^D} \ge \vb{0}$) coincideix amb la factibilitat primal ($\vb{x} \ge 
\vb{0}$). D'altra banda, també veiem que l'algorisme del símplex primal 
seleccionarà a cada iteració, per tal de millorar el valor de la f.o., una 
variable no bàsica dual $y_{\N^D(p)} \doteq r_{\B(p)}$ tal que la variable 
primal $x_{\B(p)}$ sigui negativa: $r^D_{\N^D(p)}<0 \Leftrightarrow x_{\B(p)} < 
0$.

\subsubsection{DBF duals}
Ara, per acabar de fonamentar les bases del símplex dual, cal determinar les 
direccions bàsiques factibles. Com abans, definim una DBF $\vb{d^D}$ sobre una 
SBF dual $\vb{y}$, associada a una VNB dual d'índex $\N^D(p)$, de manera que
\[
{d^D_{\N^D(i)}} =
\begin{cases}
1 \quad \text{si } i = p\\
0 \quad \text{si } i \ne p
\end{cases}
\qquad \forall i \in \{1,\ldots,m\}\,.
\]
Com en el cas del problema primal, volem que la direcció $\vb{d^D}$ conservi la 
factibilitat dual. Llavors, emprant la formulació \eqref{eq:std-dual} del dual, 
i sent $\theta$ un escalar positiu,
\begin{multline*}
A^D(\vb{y}+\theta\vb{d^D}) = \vb{c} \Rightarrow A^D\vb{d^D} = \vb{0} 
\Leftrightarrow \left[B_D \;\vert\; A^D_{\N^D} \right]
\left[
\begin{array}{@{}c@{}}
\vb{d^D_{\B^D}}\\
\hline
\vb{d^D_{\N^D}}
\end{array}
\right]
= \vb{0}\\
%
\Leftrightarrow
B_D\vb{d^D_{\B^D}} + A^D_{\N^D}\vb{d^D_{\N^D}} = \vb{0}\,;
\end{multline*}
com abans, la definició que acabem de fer de $\vb{d^D_{\N^D}}$ implica que 
$A^D_{\N^D}\vb{d^D_{\N^D}} = \vb{A^D_{\N^D(p)}}$. Llavors tenim que
\begin{equation}\label{eq:DBF-dual}
B_D\vb{d^D_{\B^D}} + \vb{A^D_{\N^D(p)}} = \vb{0} \Leftrightarrow 
\vb{d^D_{\B^D}} = -B^{-1}_D \vb{A^D_{\N^D(p)}}\,. 
\end{equation}

Recordant que $A^D_{\N^D} = \left[\Id_{m} \;|\; \varnothing \right]'$, la 
$p$-èsima columna d'$A^D_{\N^D}$ és
\[
\left[
\begin{array}{@{}c@{}}
\vb{e_p}\\
\hline
\vb{0}
\end{array}
\right]\,,
\]
on $\vb{e_p}$ denota el vector unitari que té un 1 a la posició $p$ i zeros a 
la resta d'entrades. Per tant, seguint a partir de \eqref{eq:DBF-dual},
\[
\vb{d^D_{\B^D}} = -B^{-1}_D
\left[
\begin{array}{@{}c@{}}
\vb{e_p}\\
\hline
\vb{0}
\end{array}
\right]
%
= 
%
- \left[
\begin{array}{@{}c|c@{}}
(B^{-1})'	  &	\varnothing	\\
\hline
-A_\N'(B^{-1})' & \Id_{n-m}
\end{array}
\right]
\left[
\begin{array}{@{}c@{}}
\vb{e_p}\\
\hline
\vb{0}
\end{array}
\right]
%
=
%
\left[
\begin{array}{@{}c@{}}
-(B^{-1})'\\
\hline
A_\N'(B^{-1})'
\end{array}
\right]
\vb{e_p}\,.
\]
Denotant amb $\vb{\beta_p}$ el vector fila de la p-èsima fila de $B^{-1}$, 
simplifiquem l'expressió anterior:
\[
\vb{d^D_{\B^D}} =
\left[
\begin{array}{@{}c@{}}
-(B^{-1})'\vb{e_p}\\
\hline
A_\N'(B^{-1})'\vb{e_p}
\end{array}
\right]
%
=
%
\left[
\begin{array}{@{}c@{}}
-(\vb{e_p}'B^{-1})'\\
\hline
A_\N'(\vb{e_p}'B^{-1})'
\end{array}
\right]
%
=
%
\left[
\begin{array}{@{}c@{}}
-\vb{\beta_p}'\\
\hline
A_\N'\vb{\beta_p}'
\end{array}
\right]\,.
\]
Recordant la partició del vector 
\[
\vb{y_{\B^D}} =
\left[
\begin{array}{@{}c@{}}
\vb{\lambda}\\
\hline
\vb{r_\N}
\end{array}
\right]\,,
\]
definim
\[
\vb{d^D_\lambda} \defeq -\vb{\beta_p}'\,, \quad \vb{d^D_{r_\N}} \defeq 
(\vb{\beta_p} A_\N)'\,.
\]

\subsubsection{Longitud de pas dual}
Les variables $\vb{\lambda}$ són lliures. Per tant, a l'hora de calcular la 
longitud de pas màxima de manera que es conservi la factibilitat dual, només 
caldrà tenir en compte les variables $\vb{r_\N}$. Aquestes estan subjectes a la 
restricció $\vb{r_\N} \ge \vb{0}$. Per tant, la màxima longitud de pas dual 
$\theta^\ast_D$ vindrà donada per
\[
\theta_D^\ast = \min\limits_{\{j\in \N \mid d^D_j < 0 \}} 
\left\{\frac{-r_j}{d^D_j}\right\}\,.
\]

\subsection{Algorisme del Símplex Dual}

L'esquema general de l'algorisme del símplex dual (que està resumit a 
\ref{alg:ASD}) és el 
següent. Comencem amb una SBFD inicial $\vb{x}$ i la seva solució dual 
associada $\vb{y}$---si no en tenim cap, com pel símplex primal, haurem de 
resoldre un problema de ``fase I''. 

A cada iteració, primer durem a terme el test d'optimalitat per determinar si 
la SBFD actual és òptima. Si ho és, hem acabat. En cas contrari, seleccionarem 
una VNB dual entrant ${y_{\N^D(p)}}$ que tingui associat un cost reduït 
negatiu---és a dir, tal que $x_{\B(p)} < 0$. A continuació, calcularem la DBF 
dual corresponent. Si obtenim que $\vb{d^D_{r_\N}} \ge 0$, el problema dual és 
il\lgem limitat i per tant el primal és infactible (pel corol\lgem lari 
\ref{col:feb}); 
terminarem l'algorisme. En cas contrari, procedirem a calcular la longitud de 
passa màxima.

Per terminar la iteració caldrà fer totes les actualitzacions adients. Les 
variables duals es poden actualitzar directament mitjançant la DBF dual que hem 
calculat. Quant a les variables primals: tenint 
en compte que la variable $y_{\N^D(p)}$ passarà a ser una variable 
bàsica a la iteració següent, la VB $x_{\B(p)}$ passarà a ser VNB (en el 
context del problema primal), ja que el 
cost reduït de $y_{\N^D(p)}$ serà 0 i per tant---pel que hem vist a la 
proposició \ref{prop:r-dual-x-primal}---el valor de $x_{\B(p)}$ passarà a ser 
0. És a dir, que $x_{\B(p)}$ sortirà de la base primal (i per això l'anomenarem 
VB sortint). D'altra banda i de manera similar, la VNB primal entrant vindrà
determinada pel càlcul 
de $\theta^\ast_D$, ja que la VB dual sortint serà $r_{\N(q)}$ tal que 
\[\theta_D^\ast = \frac{-r_{\N(q)}}{d^D_{r_\N(q)}}\,,\] i llavors, altre cop 
per 
la proposició \ref{prop:r-dual-x-primal}, $x_{\N(q)}$ passarà a ser VB primal 
(amb valor $\theta^\ast_D$).

Per tant, per actualitzar les variables primals, caldrà calcular la DBF primal 
associada a la VNB primal $x_{\N(q)}$, i la longitud de passa màxima 
corresponent serà
\[
	\theta^\ast = \frac{-x_{\B(p)}}{d_{\B(p)}}\,.
\] 

Per últim, el valor de la funció objectiu (primal, que és equivalent a la dual 
\textit{sense forma estàndard}) variarà segons $\vb{\lambda}$. 
Sigui $\vb{\lambda^1}$ el valor de $\vb{\lambda}$ en ls iteració actual, i 
sigui $\vb{\lambda^2}$ el seu valor en la següent iteració; i siguin $z_1$ i 
$z_2$ els valors de la f.o. en la iteració actual i següent, respectivament. 
Tenim que
\begin{multline*}
	-z_2 = -\vb{b}'\vb{\lambda^2} = -\vb{b}'(\vb{\lambda^1} + 
	\theta_D^*\vb{d^D_\lambda}) = -\vb{b}'(\vb{\lambda^1} - 
	\theta_D^*\vb{\beta_p}') = -\vb{b}'\vb{\lambda^1} + 
	\theta^*_D(\vb{\beta_p}\vb{b})' = \\
	= -z_1 + x_{\B(p)} \Rightarrow z_2 = z_1 - \theta_D^* x_{\B(p)}\,.
\end{multline*}

\begin{alg}[del símplex dual]\label{alg:ASD}
	\begin{enumerate}
		\item[]
		\item {\bf Inicialització}: Trobem una SBFD i opcionalment la solució 
		dual associada \\
		($\B, \N, \vb{x_\B}, \vb{r}, z\left[, \vb{\lambda}\right]$).
		%
		\item \label{item:simp_dual_pas2} {\bf Test d'optimalitat}:
		Si $\vb{x_\B} \geq \vb{0}$, hem trobat la solució òptima.
			\textcolor{red}{\bf STOP!}
		%
		\item {\bf Selecció de VB sortint}:
		Seleccionem un índex $\B(p)$ tal que $\vb{x_{\B\left(p\right)}} < 0$ (VB
		sortint).
		%
		\item {\bf Càlcul de DBF dual}:
		\begin{itemize}
			\item $\vb{d^D_{r_\N}} = \left(\vb{\beta_p} A_\N\right)^\prime$
			($\vb{\beta_p}$ és la fila $p$-èsima de $B^{-1}$).
			\item $\left[ \vb{d^D_{\lambda}} = -\vb{\beta_p}' \right]$
		\end{itemize}
		Si $\vb{d^D_{r_\N}} \geq \vb{0}$, 
		$\left(D^e_{\tilde{e}}\right)$
		il\lgem limitat $\Rightarrow$ $(P_e)$ infactible. \textcolor{red}{\bf 
			STOP!}
		%
		\item {\bf Càlcul de $\theta^\ast_D$ (i VNB primal entrant)}:
		\begin{flalign*}
			&\theta_D^* = \min_{\{j \in \N \,|\, d^D_j <	0\}} 
			\left\{-\frac{r_j}{d^D_{r_j}} \right\}.&
		\end{flalign*}
		VNB primal entrant: $x_{\N(q)}$ tal que $\theta^\ast_D =
		-\dfrac{x_{\N(q)}}{d^D_{r_{\N(q)}}}$.	
		%
		\item {\bf Càlcul de DBF primal i $\theta^*$}:
		\begin{flalign*}
			&\vb{d_\B} = -B^{-1}\vb{A_{\N(q)}}&\\
			&\theta^* = \frac{-x_{\B(p)}}{d_{\B(p)}}&
		\end{flalign*}
		%
		\item {\bf Actualitzacions i canvi de base}:
		\begin{enumerate}
			\item {\bf Actualització variables duals}
			\begin{flalign*}
				&\vb{r_\N} \coloneqq \vb{r}_\N + \theta_D^* 
				\vb{d^D_{\vb{r}_\N}}& \\
				&r_{B\left(p\right)} \coloneqq \theta_D^*& \\
				&z \coloneqq z - \theta^* {x_{\B\left(p\right)}}&\\
				[ &\vb{\lambda} \coloneqq \vb{\lambda} + \theta_D^* 
				\vb{d^D_{\lambda}}]&
			\end{flalign*}
			%
			\item {\bf Actualització variables primals} 
			\begin{flalign*}
				&\vb{x_\B} \coloneqq \vb{x_\B} + \theta^\ast\vb{d_\B}& \\
				&x_{\N(q)} = \theta^*&\\
			\end{flalign*}
			%
			\item {\bf Canvi de base} 
			\begin{flalign*}
				&\B \coloneqq \B \setminus \left\{\B\left(p\right)\right\} 
				\cup \left\{\N(q)\right\}\,,& \\
				&\N \coloneqq \N \setminus \left\{\N(q)\right\} \cup 
				\left\{B\left(p\right)\right\}\,.&
			\end{flalign*}
		\end{enumerate}
		\item {\bf Anar} a \ref{item:simp_dual_pas2}.
	\end{enumerate}
\end{alg}

\begin{prop}
	Una SBFD òptima és degenerada ($\exists j \in \N$ tal que $r_j = 0$) si i
	només si $\left(P\right)_e$ té òptims alternatius.
\end{prop}

\begin{prop}
	Si $\left(P\right)_e$ no té cap SBFD degenerada, el símplex dual terminarà
	amb un nombre finit d'iteracions. Altrament, podem usar la regla de Bland
	(\ref{rgl:bland}) per a que convergeixi amb un nombre finit d'iteracions.
\end{prop}
